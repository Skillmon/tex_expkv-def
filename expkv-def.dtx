% \iffalse meta-comment
%
% File: expkv-def.dtx Copyright (C) 2020 Jonathan P. Spratte
%
% This work  may be  distributed and/or  modified under  the conditions  of the
% LaTeX Project Public License (LPPL),  either version 1.3c  of this license or
% (at your option) any later version.  The latest version of this license is in
% the file:
%
%   http://www.latex-project.org/lppl.txt
%
% ------------------------------------------------------------------------------
%
%<*driver>^^A>>=
\def\nameofplainTeX{plain}
\ifx\fmtname\nameofplainTeX\else
  \expandafter\begingroup
\fi
\input l3docstrip.tex
\askforoverwritefalse
\preamble

--------------------------------------------------------------
expkv-def -- a key-defining frontend for expkv
E-mail: jspratte@yahoo.de
Released under the LaTeX Project Public License v1.3c or later
See http://www.latex-project.org/lppl.txt
--------------------------------------------------------------

Copyright (C) 2020 Jonathan P. Spratte

This  work may be  distributed and/or  modified under  the conditions  of the
LaTeX Project Public License (LPPL),  either version 1.3c  of this license or
(at your option) any later version.  The latest version of this license is in
the file:

  http://www.latex-project.org/lppl.txt

This work is "maintained" (as per LPPL maintenance status) by
  Jonathan P. Spratte.

This work consists of the file  expkv-def.dtx
and the derived files           expkv-def.pdf
                                expkv-def.sty
                                expkv-def.tex

\endpreamble
% stop docstrip adding \endinput
\postamble
\endpostamble
\generate{\file{expkv-def.sty}{\from{expkv-def.dtx}{pkg}}}
\generate{\file{expkv-def.tex}{\from{expkv-def.dtx}{tex}}}
\ifx\fmtname\nameofplainTeX
  \expandafter\endbatchfile
\else
  \expandafter\endgroup
\fi
%
\IfFileExists{expkv-def.tex}{% \iffalse meta-comment
%
% File: expkv-def.dtx Copyright (C) 2020-2021 Jonathan P. Spratte
%
% This work  may be  distributed and/or  modified under  the conditions  of the
% LaTeX Project Public License (LPPL),  either version 1.3c  of this license or
% (at your option) any later version.  The latest version of this license is in
% the file:
%
%   http://www.latex-project.org/lppl.txt
%
% ------------------------------------------------------------------------------
%
%<*driver>^^A>>=
\def\nameofplainTeX{plain}
\ifx\fmtname\nameofplainTeX\else
  \expandafter\begingroup
\fi
\input l3docstrip.tex
\askforoverwritefalse
\preamble

--------------------------------------------------------------
expkv-def -- a key-defining frontend for expkv
E-mail: jspratte@yahoo.de
Released under the LaTeX Project Public License v1.3c or later
See http://www.latex-project.org/lppl.txt
--------------------------------------------------------------

Copyright (C) 2020-2021 Jonathan P. Spratte

This  work may be  distributed and/or  modified under  the conditions  of the
LaTeX Project Public License (LPPL),  either version 1.3c  of this license or
(at your option) any later version.  The latest version of this license is in
the file:

  http://www.latex-project.org/lppl.txt

This work is "maintained" (as per LPPL maintenance status) by
  Jonathan P. Spratte.

This work consists of the file  expkv-def.dtx
and the derived files           expkv-def.pdf
                                expkv-def.sty
                                expkv-def.tex
                                t-expkv-def.tex

\endpreamble
% stop docstrip adding \endinput
\postamble
\endpostamble
\generate{\file{expkv-def.sty}{\from{expkv-def.dtx}{pkg}}}
\generate{\file{expkv-def.tex}{\from{expkv-def.dtx}{tex}}}
\generate{\file{t-expkv-def.tex}{\from{expkv-def.dtx}{ctx}}}
\ifx\fmtname\nameofplainTeX
  \expandafter\endbatchfile
\else
  \expandafter\endgroup
\fi
%
\IfFileExists{expkv-def.tex}{% \iffalse meta-comment
%
% File: expkv-def.dtx Copyright (C) 2020-2021 Jonathan P. Spratte
%
% This work  may be  distributed and/or  modified under  the conditions  of the
% LaTeX Project Public License (LPPL),  either version 1.3c  of this license or
% (at your option) any later version.  The latest version of this license is in
% the file:
%
%   http://www.latex-project.org/lppl.txt
%
% ------------------------------------------------------------------------------
%
%<*driver>^^A>>=
\def\nameofplainTeX{plain}
\ifx\fmtname\nameofplainTeX\else
  \expandafter\begingroup
\fi
\input l3docstrip.tex
\askforoverwritefalse
\preamble

--------------------------------------------------------------
expkv-def -- a key-defining frontend for expkv
E-mail: jspratte@yahoo.de
Released under the LaTeX Project Public License v1.3c or later
See http://www.latex-project.org/lppl.txt
--------------------------------------------------------------

Copyright (C) 2020-2021 Jonathan P. Spratte

This  work may be  distributed and/or  modified under  the conditions  of the
LaTeX Project Public License (LPPL),  either version 1.3c  of this license or
(at your option) any later version.  The latest version of this license is in
the file:

  http://www.latex-project.org/lppl.txt

This work is "maintained" (as per LPPL maintenance status) by
  Jonathan P. Spratte.

This work consists of the file  expkv-def.dtx
and the derived files           expkv-def.pdf
                                expkv-def.sty
                                expkv-def.tex
                                t-expkv-def.tex

\endpreamble
% stop docstrip adding \endinput
\postamble
\endpostamble
\generate{\file{expkv-def.sty}{\from{expkv-def.dtx}{pkg}}}
\generate{\file{expkv-def.tex}{\from{expkv-def.dtx}{tex}}}
\generate{\file{t-expkv-def.tex}{\from{expkv-def.dtx}{ctx}}}
\ifx\fmtname\nameofplainTeX
  \expandafter\endbatchfile
\else
  \expandafter\endgroup
\fi
%
\IfFileExists{expkv-def.tex}{% \iffalse meta-comment
%
% File: expkv-def.dtx Copyright (C) 2020-2021 Jonathan P. Spratte
%
% This work  may be  distributed and/or  modified under  the conditions  of the
% LaTeX Project Public License (LPPL),  either version 1.3c  of this license or
% (at your option) any later version.  The latest version of this license is in
% the file:
%
%   http://www.latex-project.org/lppl.txt
%
% ------------------------------------------------------------------------------
%
%<*driver>^^A>>=
\def\nameofplainTeX{plain}
\ifx\fmtname\nameofplainTeX\else
  \expandafter\begingroup
\fi
\input l3docstrip.tex
\askforoverwritefalse
\preamble

--------------------------------------------------------------
expkv-def -- a key-defining frontend for expkv
E-mail: jspratte@yahoo.de
Released under the LaTeX Project Public License v1.3c or later
See http://www.latex-project.org/lppl.txt
--------------------------------------------------------------

Copyright (C) 2020-2021 Jonathan P. Spratte

This  work may be  distributed and/or  modified under  the conditions  of the
LaTeX Project Public License (LPPL),  either version 1.3c  of this license or
(at your option) any later version.  The latest version of this license is in
the file:

  http://www.latex-project.org/lppl.txt

This work is "maintained" (as per LPPL maintenance status) by
  Jonathan P. Spratte.

This work consists of the file  expkv-def.dtx
and the derived files           expkv-def.pdf
                                expkv-def.sty
                                expkv-def.tex
                                t-expkv-def.tex

\endpreamble
% stop docstrip adding \endinput
\postamble
\endpostamble
\generate{\file{expkv-def.sty}{\from{expkv-def.dtx}{pkg}}}
\generate{\file{expkv-def.tex}{\from{expkv-def.dtx}{tex}}}
\generate{\file{t-expkv-def.tex}{\from{expkv-def.dtx}{ctx}}}
\ifx\fmtname\nameofplainTeX
  \expandafter\endbatchfile
\else
  \expandafter\endgroup
\fi
%
\IfFileExists{expkv-def.tex}{\input{expkv-def.tex}}{}
\ProvidesFile{expkv-def.dtx}
  [\csname ekvdDate\endcsname\ a key-defining frontend for expkv]
\PassOptionsToPackage{full}{textcomp}
\documentclass{l3doc}
\RequirePackage[oldstylenums,nott]{kpfonts}
\input{glyphtounicode}
\pdfgentounicode=1
\RequirePackage{tcolorbox}
\newtcolorbox{exresult}[2][]
  {colback=expkvgrey!10!white,colframe=expkvgrey,fontupper=\small,width={#2},#1}
\newtcbox\exres[1][]
  {
     colback=expkvgrey!10!white
    ,colframe=expkvgrey
    ,size=small
    ,nobeforeafter
    ,tcbox raise base
    ,fontupper=\small
    ,#1
  }
\RequirePackage{listings}
\RequirePackage{booktabs}
\RequirePackage{array}
\RequirePackage{collcell}
\RequirePackage{siunitx}
\RequirePackage{xcolor}
\RequirePackage{caption}
\RequirePackage{microtype}
\RequirePackage{accsupp}
\lstdefinelanguage{expkvd}[primitive]{TeX}
  {
    ,moretexcs=[4]^^A e-TeX
      {
        expanded,
        numexpr,
        protected,
      }
    ,moretexcs=[5]^^A plain/LaTeX
      {
        approx,
        begin,
        empty,
        item,
        LaTeX,
        makeatletter,makeatother,
        newcommand,newdimen,
        RequirePackage,
        rule,
        TeX,
        textit,texttt,
        usepackage,
      }
    ,moretexcs=[6]^^A used packages
      {
        ^^A xfp
        fpeval,
        ^^A keyval
        setkeys,
        ^^A yax
        defactiveparameter,storevalue,setparameterlist
      }
    ,moretexcs=[2]^^A expkv-def macros
      {
        ekvdefinekeys
      }
    ,moretexcs=[3]^^A expkv-pkg macros
      {
        ^^A expkv
        ekvdef,ekvdefNoVal,^^A
        ekvlet,ekvletNoVal,ekvletkv,ekvletkvNoVal,^^A
        ekvdefunknown,ekvdefunknownNoVal,^^A
        ekvredirectunknown,ekvredirectunknownNoVal,^^A
        ekvset,ekvsetSneaked,^^A
        ekvsetdef,ekvsetSneakeddef,ekvsetdefSneaked,^^A
        ekvparse,^^A
        ekvVersion,ekvDate,^^A
        ekvifdefined,ekvifdefinedNoVal,ekvifdefinedset,^^A
        ekvbreak,ekvbreakPreSneak,ekvbreakPostSneak,^^A
        ekvsneak,ekvsneakPre,^^A
        ekvchangeset,^^A
        ekvoptarg,ekvoptargTF,^^A
        ekverr,
        ^^A expkv-cs
        ekvcSecondaryKeys,ekvcSplitAndForward,
      }
  }
\colorlet{codeparam}{cyan!65!black}
\def\mylstwd{.55em}
\lstset
  {
    ,language=expkvd
    ,flexiblecolumns=false
    ,basewidth=\mylstwd
    ,gobble=2
    ,basicstyle=\fontfamily{jkp}\itshape
    ,texcsstyle=*[1]{\bfseries\color{expkvgrey}} ^^A primitives
    ,texcsstyle=*[2]{\bfseries\color{expkvred}}  ^^A expkv
    ,texcsstyle=*[3]{\color{expkvred}}           ^^A expkv-pkg
    ,texcsstyle=*[4]{\bfseries\color{expkvgrey}} ^^A e-TeX
    ,texcsstyle=*[5]{\bfseries\color{expkvgrey}} ^^A plain/LaTeX macros
    ,texcsstyle=*[6]{}                           ^^A macros of other packages
    ,commentstyle=\color[gray]{0.4}
    ,literate=
              {\{} {{\CodeUpBf\{}}{1}
              {\}} {{\CodeUpBf\}}}{1}
              {$}  {{\CodeUpBf\$}}{1}
              {[}  {{\CodeUp[}}{1}
              {]}  {{\CodeUp]}}{1}
              {(}  {{\CodeUp(}}{1}
              {)}  {{\CodeUp)}}{1}
              {*}  {{$*$}}{1}
              {1}  {{\CodeUp{1}}}{1}
              {2}  {{\CodeUp{2}}}{1}
              {3}  {{\CodeUp{3}}}{1}
              {4}  {{\CodeUp{4}}}{1}
              {5}  {{\CodeUp{5}}}{1}
              {6}  {{\CodeUp{6}}}{1}
              {7}  {{\CodeUp{7}}}{1}
              {8}  {{\CodeUp{8}}}{1}
              {9}  {{\CodeUp{9}}}{1}
              {0}  {{\CodeUp{0}}}{1}
              {##} {{\CodeColored{codeparam}{1}{\#}}}{1}
              {##1}{{\CodeColored{codeparam}{2}{\#1}}}{2}
              {##2}{{\CodeColored{codeparam}{2}{\#2}}}{2}
              {##3}{{\CodeColored{codeparam}{2}{\#3}}}{2}
              {##4}{{\CodeColored{codeparam}{2}{\#4}}}{2}
              {##5}{{\CodeColored{codeparam}{2}{\#5}}}{2}
              {##6}{{\CodeColored{codeparam}{2}{\#6}}}{2}
              {##7}{{\CodeColored{codeparam}{2}{\#7}}}{2}
              {##8}{{\CodeColored{codeparam}{2}{\#8}}}{2}
              {##9}{{\CodeColored{codeparam}{2}{\#9}}}{2}
              {<key>}{{$\langle$}key{$\rangle$}}{5}
              {<set>}{{$\langle$}set{$\rangle$}}{5}
              {<set1>}{{$\langle$}set1{$\rangle$}}{6}
              {<set2>}{{$\langle$}set2{$\rangle$}}{6}
    ^^A,literate=*{<key>}{\key}{4}{<set>}{\set}{4}
  }
\newcommand*\CodeColored[3]{\textcolor{#1}{\makebox[\dimexpr\mylstwd*#2]{#3}}}
\newcommand*\CodeUpBf[1]{\makebox[\mylstwd]{\textup{\textbf{#1}}}}
\newcommand*\CodeUp[1]{\makebox[\mylstwd]{\textup{#1}}}
\RequirePackage{randtext}
\let\metaORIG\meta
\protected\def\meta #1{\texttt{\metaORIG{#1}}}
\renewcommand*\thefootnote{\fnsymbol{footnote}}
\definecolor{expkvred}{HTML}{9F393D}
\colorlet{expkvgrey}{black!75}
\makeatletter
\newcommand*\expkv
  {^^A
    \texorpdfstring
      {^^A
        \mbox
          {^^A
            \BeginAccSupp{ActualText=expkv}^^A
            \href{https://github.com/Skillmon/tex_expkv}
              {^^A
                \rmfamily
                \bfseries
                {\color{expkvgrey}e\kern-.05em x\kern-.05em}^^A
                \lower.493ex
                  \hbox{{\color{expkvgrey}P}\kern-.1em{\color{expkvred}k}}^^A
                \kern-.18em{\color{expkvred}v}^^A
              }^^A
            \EndAccSupp{}^^A
          }^^A
      }
      {expkv}^^A
  }
\newcommand*\expkvd
  {^^A
    \texorpdfstring
      {^^A
        \mbox
          {^^A
            \BeginAccSupp{ActualText=expkv-def}^^A
            \href{https://github.com/Skillmon/tex_expkv-def}
              {^^A
                \rmfamily
                \bfseries
                {\color{expkvgrey}e\kern-.05em x\kern-.05em}^^A
                \lower.493ex
                  \hbox{{\color{expkvgrey}P}\kern-.1em{\color{expkvred}k}}^^A
                \kern-.18em{\color{expkvred}v}^^A
                {\color{expkvgrey}^^A
                  \kern.05em\rule[-.1ex]{.08em}{1.2ex}\kern.05em\textsc{def}^^A
                }^^A
              }^^A
            \EndAccSupp{}^^A
          }^^A
      }
      {expkv-def}^^A
  }
\newcommand\kv{\meta{key}=\meta{value}}
\newcommand\key{\meta{key}}
\newcommand\val{\meta{value}}
\newcommand\set{\meta{set}}
\newcommand\enfprefix{\textcolor{black}}
\newcommand\allprefix{\textcolor{gray}}
\newcommand\notprefix{\textcolor{red!80!black}}
\newcommand\printprefix[2]
  {^^A
    \ifcase\numexpr#1\relax\or
      \enfprefix{#2}\or
      \allprefix{#2}\or
      \notprefix{#2}\fi
  }
\newcommand\prefixes[4]
  {^^A
    \hfill
    \printprefix{#1}{new}
    \printprefix{#2}{also}
    \printprefix{#3}{protected}
    \printprefix{#4}{long}^^A
  }
\hypersetup{linkcolor=red!80!black,urlcolor=purple!80!black}
\DoNotIndex{\def,\edef,\,,\=,\begingroup,\catcode,\chardef,\csname,\endcsname}
\DoNotIndex{\endgroup,\endinput,\errmessage,\expandafter,\input,\let,\long}
\DoNotIndex{\protected,\ProvidesFile,\ProvidesPackage,\relax,\space}
\DoNotIndex{\@,\unexpanded,\string,\expanded,\dimexpr,\global,\glueexpr,\hbox}
\DoNotIndex{\numexpr,\RequirePackage,\setbox,\the,\unless,\xdef,\@firstofone}
\DoNotIndex{\@firstoftwo,\@gobble,\@secondoftwo,\meaning,\detokenize}
\DoNotIndex{\romannumeral,\^}
\DoNotIndex{\ifcsname}
\DoNotIndex{\ifx}
\DoNotIndex{\ifdefined}
\DoNotIndex{\iffalse}
\DoNotIndex{\iftrue}
\DoNotIndex{\else}
\DoNotIndex{\fi}
\@gobble\fi ^^A ignoring \ifx, \ifcsname, etc., but only one \fi
\@gobble\fi ^^A ignoring \ifx, \ifcsname, etc., but only one \fi
\@gobble\fi ^^A ignoring \ifx, \ifcsname, etc., but only one \fi
\@gobble\fi ^^A ignoring \ifx, \ifcsname, etc., but only one \fi
\@ifdefinable\gobbledocstriptag{\def\gobbledocstriptag#1>{}}
\makeatother
\begin{document}
  \title
    {^^A
      \texorpdfstring
        {^^A
          \huge\expkvd
          \\[\medskipamount]
          \Large a key-defining frontend for \expkv
        }{expkv-def - a key-defining frontend for expkv}^^A
    }
  \date{\ekvdDate\space v\ekvdVersion}
  \author{Jonathan P. Spratte\thanks{\protect\randomize{jspratte@yahoo.de}}}
  \DocInput{expkv-def.dtx}
\end{document}
%</driver>^^A=<<
% \fi
%
% \maketitle
% \renewcommand*\thefootnote{\arabic{footnote}}
%
% \begin{abstract}
% \noindent\parfillskip=0pt
% \expkvd\ provides a small \kv\ interface to define keys for \expkv. Key-types
% are declared using prefixes, similar to static typed languages. The stylised
% name is \expkvd\ but the files use \file{expkv-def}, this is due to CTAN-rules
% which don't allow \string| in package names since that is the pipe symbol in
% *nix shells.
% \end{abstract}
%
% \tableofcontents
%
% \begin{documentation}^^A>>=
%
% \section{Documentation}
%
% Since the trend for the last couple of years goes to defining keys for a \kv\
% interface using a \kv\ interface, I thought that maybe providing such an
% interface for \expkv\ will make it more attractive for actual use, besides its
% unique selling points of being fully expandable, and fast and reliable. But at
% the same time I don't want to widen \expkv's initial scope. So here it is
% \expkvd, go define \kv\ interfaces with \kv\ interfaces.
%
% Unlike many of the other established \kv\ interfaces to define keys, \expkvd\
% works using prefixes instead of suffixes (\emph{e.g.}, |.tl_set:N| of
% \pkg{l3keys}) or directory like handlers (\emph{e.g.}, |/.store in| of
% \pkg{pgfkeys}). This was decided as a personal preference, more over in \TeX\
% parsing for the first space is way easier than parsing for the last one.
% \expkvd's prefixes are sorted into two categories: |p|-type, which are
% equivalent to \TeX's prefixes like |\long|, and |t|-type defining the type of
% the key. For a description of the available |p|-prefixes take a look at
% \autoref{sec:prefix:p}, the |t|-prefixes are described in
% \autoref{sec:prefix:t}.
%
% \expkvd\ is usable as generic code, as a \LaTeX\ package, and as a \ConTeXt\
% module. It'll automatically load \expkv\ in the same mode as well. To use it,
% just use one of
% \begin{lstlisting}
% \input expkv-def       % plainTeX
% \usepackage{expkv-def} % LaTeX
% \usemodule[expkv-def]  % ConTeXt
% \end{lstlisting}
%
% \subsection{Macros}
% Apart from version and date containers there is only a single user-facing
% macro, and that should be used to define keys.
%
% \begin{function}{\ekvdefinekeys}
%   \begin{syntax}
%     \cs{ekvdefinekeys}\marg{set}\{\kv, \ldots\}
%   \end{syntax}
%   In \set, define \key\ to have definition \val. The general syntax for \key\
%   should be
%   \begin{quote}\ttfamily\small
%     \meta{prefix} \meta{name}
%   \end{quote}
%   Where \meta{prefix} is a space separated list of optional |p|-type prefixes
%   followed by one |t|-type prefix. The syntax of \val\ is dependent on the
%   used |t|-prefix.
% \end{function}
%
% \begin{function}{\ekvdDate,\ekvdVersion}
%   These two macros store the version and date of the package.
% \end{function}
%
% \subsection{Prefixes}
% As already said there are |p|-prefixes and |t|-prefixes. Not every |p|-prefix
% is allowed for all |t|-prefixes.
%
% \subsubsection{\texttt{p}-Prefixes}\label{sec:prefix:p}
% The two |p|-type prefixes |long| and |protected| are pretty simple by nature,
% so their description is pretty simple. They affect the \key\ at use-time, so
% omitting |long| doesn't mean that a \meta{definition} can't contain a |\par|
% token, only that the \key\ will not accept a |\par| in \val.  On the other
% hand |new| and |also| might be simple on first sight as well, but their rules
% are a bit more complicated.
%
% \begin{function}{also}
%   The following key type will be \emph{added} to an existing \key's
%   definition. You can't add a type taking an argument at use time to an
%   existing key which doesn't take an argument and vice versa. Also you'll get
%   an error if you try to add an action which isn't allowed to be either |long|
%   or |protected| to a key which already is |long| or |protected| (the opposite
%   order would be suboptimal as well, but can't be really captured with the
%   current code).
%
%   A key already defined as |long| or |protected| will stay |long| or
%   |protected|, but you can as well add |long| or |protected| with the |also|
%   definition.
%
%   As a small example, suppose you want to create a boolean key, but
%   additionally to setting a boolean value you want to execute some more code
%   as well, you can use the following
% \begin{lstlisting}
% \ekvdefinekeys{also-example}
%   {
%      bool key      = \ifmybool
%     ,also code key = \domystuff{#1}
%   }
% \end{lstlisting}
%
%   If you use |also| on a |choice|, |bool|, |invbool|, or |boolpair| key it is
%   tried to determine if the key already is of one of those types. If this test
%   is true the declared choices will be added to the possible choices but the
%   key's definition will not be changed other than that. If that wouldn't have
%   been done, the callbacks of the different choices could get called multiple
%   times.
% \end{function}
%
% \begin{function}{protected,protect}
%   The following key will be defined |\protected|. Note that key-types which
%   can't be defined expandable will always use |\protected|.
% \end{function}
%
% \begin{function}{long}
%   The following key will be defined |\long|.
% \end{function}
%
% \begin{function}{new}
%   The following key must be new (so previously undefined). An error is thrown
%   if it is already defined and the new definition is ignored. |new| only
%   asserts that there are no conflicts between |NoVal| keys and other |NoVal|
%   keys or value taking keys and other value taking keys. For example you can
%   use the following without an error:
% \begin{lstlisting}
% \ekvdefinekeys{new-example}
%   {
%      code key       = \domystuffwitharg{#1}
%     ,new  noval key = \domystuffwithoutarg
%   }
% \end{lstlisting}
% \end{function}
%
% \subsubsection{\texttt{t}-Prefixes}\label{sec:prefix:t}
% Since the |p|-type prefixes apply to some of the |t|-prefixes automatically
% but sometimes one might be disallowed we need some way to highlight this
% behaviour. In the following an enforced prefix will be printed black
% (\texttt{\enfprefix{protected}}), allowed prefixes will be grey
% (\texttt{\allprefix{protected}}), and disallowed prefixes will be red
% (\texttt{\notprefix{protected}}). This will be put flush-right in the syntax
% showing line.
%
% \begin{function}{code,ecode}
%   \begin{syntax}
%     code \key\ = \marg{definition} \prefixes2222
%   \end{syntax}
%   Define \key\ to expand to \meta{definition}. The \key\ will require a \val\
%   for which you can use |#1| inside \meta{definition}. The |ecode| variant
%   will fully expand \meta{definition} inside an |\edef|.
% \end{function}
%
% \begin{function}{noval,enoval}
%   \begin{syntax}
%     noval \key\ = \marg{definition} \prefixes2223
%   \end{syntax}
%   The |noval| type defines \key\ to expand to \meta{definition}. The \key\
%   will not take a \val. |enoval| fully expands \meta{definition} inside an
%   |\edef|.
% \end{function}
%
% \begin{function}{default,qdefault,odefault,fdefault,edefault}
%   \begin{syntax}
%     default \key\ = \marg{definition} \prefixes3223
%   \end{syntax}
%   This serves to place a default \val\ for a \key\ that takes an argument, the
%   \key\ can be of any argument-grabbing kind, and when used without a \val\
%   it will be passed \meta{definition} instead. The |qdefault| variant will
%   expand the \key's code once, so will be slightly quicker, but not change if
%   you redefine \key. |odefault| is just another name for |qdefault|. The
%   |fdefault| version will expand the key code until a non-expandable token or
%   a space is found, a space would be gobbled.\footnotemark{} The |edefault| on
%   the other hand fully expands the \key-code with \meta{definition} as its
%   argument inside of an |\edef|.
% \end{function}
% \footnotetext
%   {^^A
%     For those familiar with \TeX-coding: This uses a
%     \cs[no-index]{romannumeral}-expansion.
%   }
%
% \begin{function}{initial,oinitial,finitial,einitial}
%   \begin{syntax}
%     initial \key\ = \{\val\} \prefixes3333
%     initial \key
%   \end{syntax}
%   With |initial| you can set an initial \val\ for an already defined argument
%   taking \key. It'll just call the key-macro of \key\ and pass it \val. The
%   |einitial| variant will expand \val\ using an |\edef| expansion prior to
%   passing it to the key-macro and the |oinitial| variant will expand the first
%   token in \val\ once. |finitial| will expand \val\ until a non-expandable
%   token or a space is found, a space would be
%   gobbled.\footnotemark
%
%   If you don't provide a value (and no equals sign) a |noval| \key\ of the
%   same name is called once (or, if you specified a |default| for a value
%   taking key that would be used).
% \end{function}
% \footnotetext{Again using \cs[no-index]{romannumeral}.}
%
% \begin{function}{bool,gbool,boolTF,gboolTF}
%   \begin{syntax}
%     bool \key\ = \meta{cs} \prefixes2223
%   \end{syntax}
%   The \meta{cs} should be a single control sequence, such as |\iffoo|.
%   This will define \key\ to be a boolean key, which only takes the values
%   |true| or |false| and will throw an error for other values. If the key is
%   used without a \val\ it'll have the same effect as if you use
%   \texttt{\key=true}. |bool| and |gbool| will behave like \TeX-ifs so either
%   be |\iftrue| or |\iffalse|. The |boolTF| and |gboolTF| variants will both
%   take two arguments and if true the first will be used else the second, so
%   they are always either |\@firstoftwo| or |\@secondoftwo|. The variants with
%   a leading |g| will set the control sequence globally, the others locally. If
%   \meta{cs} is not yet defined it'll be initialised as the |false| version.
%   Note that the initialisation is \emph{not} done with |\newif|, so you will
%   not be able to do |\footrue| outside of the \kv\ interface, but you could
%   use |\newif| yourself. Even if the \key\ will not be |\protected| the
%   commands which execute the |true| or |false| choice will be, so the usage
%   should be safe in an expansion context (\emph{e.g.}, you can use
%   \texttt{edefault \key\ = false} without an issue to change the default
%   behaviour to execute the |false| choice). Internally a |bool| \key\ is the
%   same as a |choice| key which is set up to handle |true| and |false| as
%   choices.
% \end{function}
%
% \begin{function}{invbool,ginvbool,invboolTF,ginvboolTF}
%   \begin{syntax}
%     bool \key\ = \meta{cs} \prefixes2223
%   \end{syntax}
%   These are inverse boolean keys, they behave like |bool| and friends but set
%   the opposite meaning to the macro \meta{cs} in each case. So if
%   \texttt{key=true} is used |invbool| will set \meta{cs} to |\iffalse| and
%   vice versa.
% \end{function}
%
% \begin{function}{boolpair,gboolpair,boolpairTF,gboolpairTF}
%   \begin{syntax}
%     boolpair \key\ = \meta{cs_1}\meta{cs_2} \prefixes2223
%   \end{syntax}
%   The |boolpair| key type behaves like both |bool| and |invbool|, the
%   \meta{cs_1} will be set to the meaning according to the rules of |bool|, and
%   \meta{cs_2} will be set to the opposite.
% \end{function}
%
% \begin{function}{store,estore,gstore,xstore}
%   \begin{syntax}
%     store \key\ = \meta{cs} \prefixes2212
%   \end{syntax}
%   The \meta{cs} should be a single control sequence, such as |\foo|.
%   This will define \key\ to store \val\ inside of the control sequence. If
%   \meta{cs} isn't yet defined it will be initialised as empty. The variants
%   behave similarly to their |\def|, |\edef|, |\gdef|, and |\xdef|
%   counterparts, but |store| and |gstore| will allow you to store macro
%   parameters inside of them by using |\unexpanded|.
% \end{function}
%
% \begin{function}{data,edata,gdata,xdata}
%   \begin{syntax}
%     data \key\ = \meta{cs} \prefixes2212
%   \end{syntax}
%   The \meta{cs} should be a single control sequence, such as |\foo|. This will
%   define \key\ to store \val\ inside of the control sequence. But unlike the
%   |store| type, the macro \meta{cs} will be a switch at the same time, it'll
%   take two arguments and if \key\ was used expands to the first argument
%   followed by \val\ in braces, if \key\ was not used \meta{cs} will expand to
%   the second argument (so behave like |\@secondoftwo|). The idea is that with
%   this type you can define a key which should be typeset formatted. The
%   |edata| and |xdata| variants will fully expand \val, the |gdata| and |xdata|
%   variants will store \val\ inside \meta{cs} globally. The |p|-prefixes will
%   only affect the key-macro, \meta{cs} will always be expandable and |\long|.
% \end{function}
%
% \begin{function}{dataT,edataT,gdataT,xdataT}
%   \begin{syntax}
%     dataT \key\ = \meta{cs} \prefixes2212
%   \end{syntax}
%   Just like |data|, but instead of \meta{cs} grabbing two arguments it'll only
%   grab one, so by default it'll behave like |\@gobble|, and if a \val\ was
%   given to \key\ the \meta{cs} will behave like |\@firstofone| appended by
%   \marg{value}.
% \end{function}
%
% \begin{function}{int,eint,gint,xint}
%   \begin{syntax}
%     int \key\ = \meta{cs} \prefixes2212
%   \end{syntax}
%   The \meta{cs} should be a single control sequence, such as |\foo|.
%   An |int| key will be a \TeX-count register. If \meta{cs} isn't defined yet,
%   |\newcount| will be used to initialise it. The |eint| and |xint| versions
%   will use |\numexpr| to allow basic computations in their \val. The |gint|
%   and |xint| variants set the register globally.
% \end{function}
%
% \begin{function}{dimen,edimen,gdimen,xdimen}
%   \begin{syntax}
%     dimen \key\ = \meta{cs} \prefixes2212
%   \end{syntax}
%   The \meta{cs} should be a single control sequence, such as |\foo|.
%   This is just like |int| but uses a dimen register, |\newdimen| and
%   |\dimexpr| instead.
% \end{function}
%
% \begin{function}{skip,eskip,gskip,xskip}
%   \begin{syntax}
%     skip \key\ = \meta{cs} \prefixes2212
%   \end{syntax}
%   The \meta{cs} should be a single control sequence, such as |\foo|.
%   This is just like |int| but uses a skip register, |\newskip| and |\glueexpr|
%   instead.
% \end{function}
%
% \begin{function}{toks,gtoks,apptoks,gapptoks,pretoks,gpretoks}
%   \begin{syntax}
%     toks \key\ = \meta{cs} \prefixes2212
%   \end{syntax}
%   The \meta{cs} should be a single control sequence, such as |\foo|.
%   Store \val\ inside of a toks-register. The |g| variants use |\global|, the
%   |app| variants append \val\ to the contents of that register, the |pre|
%   variants will prepend \val. If \meta{cs} is not yet defined it will be
%   initialised with |\newtoks|.
% \end{function}
%
% \begin{function}{box,gbox}
%   \begin{syntax}
%     box \key\ = \meta{cs} \prefixes2212
%   \end{syntax}
%   The \meta{cs} should be a single control sequence, such as |\foo|.
%   Typesets \val\ into a |\hbox| and stores the result in a box register. The
%   boxes are colour safe. \expkvd\ doesn't provide a |vbox| type.
% \end{function}
%
% \begin{function}{meta}
%   \begin{syntax}
%     meta \key\ = \{\kv, \ldots\} \prefixes2222
%   \end{syntax}
%   This key type can set other keys, you can access the \val\ which was passed
%   to \key\ inside the \kv\ list with |#1|. It works by calling a sub-|\ekvset|
%   on the \kv\ list, so a |set| key will only affect that \kv\ list and not the
%   current |\ekvset|. Since it runs in a separate |\ekvset| you can't use
%   |\ekvsneak| using keys or similar macros in the way you normally could.
% \end{function}
%
% \begin{function}{nmeta}
%   \begin{syntax}
%     nmeta \key\ = \{\kv, \ldots\} \prefixes2223
%   \end{syntax}
%   This key type can set other keys, the difference to |meta| is, that this key
%   doesn't take a value, so the \kv\ list is static.
% \end{function}
%
% \begin{function}{smeta}
%   \begin{syntax}
%     smeta \key\ = \marg{set}\{\kv, \ldots\} \prefixes2222
%   \end{syntax}
%   Yet another |meta| variant. An |smeta| key will take a \val\ which you can
%   access using |#1|, but it sets the \kv\ list inside of \set, so is equal to
%   \texttt{\cs[no-index]{ekvset}\marg{set}\{\kv, \ldots\}}.
% \end{function}
%
% \begin{function}{snmeta}
%   \begin{syntax}
%     snmeta \key\ = \marg{set}\{\kv, \ldots\} \prefixes2223
%   \end{syntax}
%   And the last |meta| variant. |snmeta| is a combination of |smeta| and
%   |nmeta|. It doesn't take an argument and sets the \kv\ list inside of \set.
% \end{function}
%
% \begin{function}{set}
%   \begin{syntax}
%     set \key\ = \marg{set} \prefixes2233
%   \end{syntax}
%   This will define \key\ to change the set of the current |\ekvset|
%   invocation to \set. You can omit \set\ (including the equals sign),
%   which is the same as using \texttt{set \key\ = \{\key\}}. The created |set|
%   key will not take a \val. Note that just like in \expkv\ it'll not be
%   checked whether \set\ is defined and you'll get a low-level \TeX\ error if
%   you use an undefined \set.
% \end{function}
%
% \begin{function}{choice}
%   \begin{syntax}
%     choice \key\ = \{\val=\meta{definition}, \ldots\} \prefixes2223
%   \end{syntax}
%   Defines \key\ to be a |choice| key, meaning it will only accept a limited
%   set of values. You should define each possible \val\ inside of the
%   \val=\meta{definition} list. If a defined \val\ is passed to \key\ the
%   \meta{definition} will be left in the input stream. You can make
%   individual values |protected| inside the \val=\meta{definition} list. By
%   default a |choice| key is expandable, an undefined \val\ will throw an error
%   in an expandable way (but see the |unknown-choice| prefix). You can add
%   additional choices after the \key\ was created by using |choice| again for
%   the same \key, redefining choices is possible the same way, but there is no
%   interface to remove certain choices.
% \end{function}
%
% \begin{function}{choice-store}
%   \begin{syntax}
%     choice-store \key\ = \meta{cs}\{\val, \ldots\} \prefixes2223
%   \end{syntax}
%   This is defines a special type of |choice| key that'll store the given
%   choice inside the macro \meta{cs} (so \meta{cs} should be a single control
%   sequence name such as |\foo|). Since storing inside a macro can't be done
%   expandably every choice-code is |\protected|, you might define the
%   |choice-store| key itself as |protected| as well if you want. Since the
%   definition of each choice is predefined with this key type the choice list
%   should just be a comma separated list of valid choices.
% \end{function}
% This means that the following |choice| and |choice-store| keys are equivalent
% at use time:
% \begin{lstlisting}
% \newcommand*\mya{}
% \ekvdefinekeys{example}
%   {
%      choice key1 = {a=\def\mya{a}, b=\def\mya{b}, c=\def\mya{c}}
%     ,choice-store key2 = \mya{a,b,c}
%   }
% \end{lstlisting}
%
% \begin{function}{choice-enum}
%   \begin{syntax}
%     choice-enum \key\ = \meta{cs}\{\val, \ldots\} \prefixes2223
%   \end{syntax}
%   This is similar to |choice-store|, the differences are: \meta{cs} should be
%   a count-register or is initialised as such if the \meta{cs} is undefined
%   (via |\newcount|); instead of the value the position of the value in the
%   given list is stored in this register (zero-based).
% \end{function}
% This means that the following |choice| and |choice-enum| keys are equivalent
% at use time:
% \begin{lstlisting}
% \newcount\myb
% \ekvdefinekeys{example}
%   {
%      choice key1 = {a={\myb=0 }, b={\myb=1 }, c={\myb=2 }}
%     ,choice-enum key2 = \myb{a,b,c}
%   }
% \end{lstlisting}
%
% \begin{function}{unknown-choice}
%   \begin{syntax}
%     unknown-choice \key\ = \marg{definition} \prefixes2323
%   \end{syntax}
%   By default an unknown \val\ passed to a |choice| or |bool| key will throw an
%   error.  However, with this prefix you can define an alternative action which
%   should be executed if \key\ received an unknown choice. In \meta{definition}
%   you can refer to the choice which was passed in with |#1|.
% \end{function}
%
% \begin{function}{unknown code}
%   \begin{syntax}
%     unknown code = \marg{definition} \prefixes2322
%   \end{syntax}
%   By default \expkv\ throws errors when it encounters unknown keys in a set.
%   With the |unknown| prefix you can define handlers that deal with undefined
%   keys, instead of a \key\ name you have to specify a subtype for this prefix,
%   here the subtype is |code|.\par
%   With |unknown code| the \meta{definition} is used for unknown keys which
%   were provided a value (so corresponds to |\ekvdefunknown|), you can access
%   the key name with |#1| and the value with |#2|.\footnotemark
% \end{function}
% \footnotetext
%   {^^A
%     There is some trickery involved to get this more intuitive argument order
%     without any performance hit if you compare this to
%     \cs[no-index]{ekvdefunknown} directly.^^A
%   }
%
% \begin{function}{unknown noval}
%   \begin{syntax}
%     unknown noval = \marg{definition} \prefixes2323
%   \end{syntax}
%   This is like |unknown code| but uses \meta{definition} for unknown keys to
%   which no value was passed (so corresponds to |\ekvdefunknownNoVal|). You can
%   access the key name with |#1|.
% \end{function}
%
% \begin{function}{unknown redirect-code}
%   \begin{syntax}
%     unknown redirect-code = \marg{set-list} \prefixes2331
%   \end{syntax}
%   This uses a predefined action for |unknown code|. Instead of throwing an
%   error, it is tried to find the \key\ in each \set\ in the comma separated
%   \meta{set-list}. The first found match will be used and the remaining
%   options from the list discarded. If the \key\ isn't found in any \set\ an
%   expandable error will be thrown eventually. Internally \expkv's
%   |\ekvredirectunknown| will be used.
% \end{function}
%
% \begin{function}{unknown redirect-noval}
%   \begin{syntax}
%     unknown redirect-noval = \marg{set-list} \prefixes2333
%   \end{syntax}
%   This behaves just like |unknown redirect-code| but will set up means to
%   forward keys for |unknown noval|. Internally \expkv's
%   |\ekvredirectunknownNoVal| will be used.
% \end{function}
%
% \begin{function}{unknown redirect}
%   \begin{syntax}
%     unknown redirect = \marg{set-list} \prefixes2333
%   \end{syntax}
%   This is a short cut to apply both, |unknown redirect-code| and
%   |unknown redirect-noval|, as a result you might get doubled error messages,
%   one from each.
% \end{function}
%
% \subsection{Bugs}
% I don't think there are any (but every developer says that), if you find some
% please let me know, either via the email address on the first page or on
% GitHub:
% \url{https://github.com/Skillmon/tex_expkv-def}
%
% \subsection{Example}
% The following is an example code defining each base key-type once. Please
% admire the very creative key-name examples.
%
% \begin{lstlisting}
% \ekvdefinekeys{example}
%   {
%      long code keyA = #1
%     ,noval     keyA = NoVal given
%     ,bool      keyB = \keyB
%     ,boolTF    keyC = \keyC
%     ,store     keyD = \keyD
%     ,data      keyE = \keyE
%     ,dataT     keyF = \keyF
%     ,int       keyG = \keyG
%     ,dimen     keyH = \keyH
%     ,skip      keyI = \keyI
%     ,toks      keyJ = \keyJ
%     ,default   keyJ = \empty test
%     ,new box   keyK = \keyK
%     ,qdefault  keyK = K
%     ,choice    keyL =
%       {
%          protected 1 = \texttt{a}
%         ,2 = b
%         ,3 = c
%         ,4 = d
%         ,5 = e
%       }
%     ,edefault  keyL = 2
%     ,meta      keyM = {keyA={#1},keyB=false}
%     ,invbool   keyN = \keyN
%     ,boolpair  keyO = \keyOa\keyOb
%   }
% \end{lstlisting}
%
% \bigskip
% Since the |data| type might be a bit strange, here is another usage example
% for it.
% \begin{lstlisting}
% \ekvdefinekeys{ex}
%   {
%      data  name  = \Pname
%     ,data  age   = \Page
%     ,dataT hobby = \Phobby
%   }
% \newcommand\Person[1]
%   {%
%     \begingroup
%     \ekvset{ex}{#1}%
%     \begin{description}
%       \item[\Pname{}{\errmessage{A person requires a name}}]
%       \item[Age] \Page{\textit}{\errmessage{A person requires an age}}
%       \Phobby{\item[Hobbies]}
%     \end{description}
%     \endgroup
%   }
% \Person{name=Jonathan P. Spratte, age=young, hobby=\TeX\ coding}
% \Person{name=Some User, age=unknown, hobby=Reading Documentation}
% \Person{name=Anybody, age=any}
% \end{lstlisting}
% In this example a person should have a name and an age, but doesn't have to
% have hobbies. The name will be displayed as the description item and the age
% in Italics. If a person has no hobbies the description item will be silently
% left out.  The result of the above code looks like this:
%
% \begin{exresult}{\linewidth}
% \ekvdefinekeys{ex}
%   {
%      data  name  = \Pname
%     ,data  age   = \Page
%     ,dataT hobby = \Phobby
%   }
% \newcommand\Person[1]
%   {%
%     \begingroup
%     \ekvset{ex}{#1}%
%     \begin{description}
%       \item[\Pname{}{\errmessage{A person requires a name}}]
%       \item[Age] \Page{\textit}{\errmessage{A person requires an age}}
%       \Phobby{\item[Hobbies]}
%     \end{description}
%     \endgroup
%   }
% \Person{name=Jonathan P. Spratte, age=young, hobby=\TeX\ coding}
% \Person{name=Some User, age=unknown, hobby=Reading Documentation}
% \Person{name=Anybody, age=any}
% \end{exresult}
%
% \subsection{License}
%
% Copyright \textcopyright\ 2020\unless\ifnum\year=2020--\the\year\fi\
% Jonathan P. Spratte
%
% \medskip\noindent
% This work may be distributed and/or modified under the conditions of the
% \LaTeX\ Project Public License (LPPL), either version 1.3c of this license or
% (at your option) any later version. The latest version of this license is in
% the file:
%
% \url{http://www.latex-project.org/lppl.txt}
%
% \noindent
% This work is ``maintained'' (as per LPPL maintenance status) by
%
% \mbox{Jonathan P. Spratte}.
%
% \end{documentation}^^A=<<
%
% \begin{implementation}^^A>>=
%
% \clearpage
%
% \section{Implementation}^^A>>=
%^^A the LaTeX package >>=
% \subsection{The \LaTeX\ Package}
% Just like for \expkv\ we provide a small \LaTeX\ package that sets up things
% such that we behave nicely on \LaTeX\ packages and files system. It'll
% |\input| the generic code which implements the functionality.
% \gobbledocstriptag
%<*pkg>
%    \begin{macrocode}
\RequirePackage{expkv}
\def\ekvd@tmp
  {%
    \ProvidesFile{expkv-def.tex}%
      [\ekvdDate\space v\ekvdVersion\space a key-defining frontend for expkv]%
  }
\input{expkv-def.tex}
\ProvidesPackage{expkv-def}%
  [\ekvdDate\space v\ekvdVersion\space a key-defining frontend for expkv]
%    \end{macrocode}
% \gobbledocstriptag
%</pkg>
%^^A=<<
%^^A the ConTeXt module >>=
% \subsection{The \ConTeXt\ module}
% \gobbledocstriptag
%<*ctx>
%    \begin{macrocode}
\writestatus{loading}{ConTeXt User Module / expkv-def}
\usemodule[expkv]
\unprotect
\input expkv-def.tex
\writestatus{loading}
  {ConTeXt User Module / expkv-def / Version \ekvdVersion\space loaded}
\protect\endinput
%    \end{macrocode}
% \gobbledocstriptag
%</ctx>
%^^A=<<
%^^A main file >>=
% \subsection{The Generic Code}
% The rest of this implementation will be the generic code.
% \gobbledocstriptag
%<*tex>
%
% Load \expkv\ if the package didn't already do so -- since \expkv\ has
% safeguards against being loaded twice this does no harm and the overhead
% isn't that big. Also we reuse some of the internals of \expkv\ to save us from
% retyping them.
%    \begin{macrocode}
\input expkv
%    \end{macrocode}
%
% We make sure that \file{expkv-def.tex} is only input once:
%    \begin{macrocode}
\expandafter\ifx\csname ekvdVersion\endcsname\relax
\else
  \expandafter\endinput
\fi
%    \end{macrocode}
%
% \begin{macro}{\ekvdVersion,\ekvdDate}
% We're on our first input, so lets store the version and date in a macro.
%    \begin{macrocode}
\def\ekvdVersion{0.8c}
\def\ekvdDate{2021-09-20}
%    \end{macrocode}
% \end{macro}
%
% If the \LaTeX\ format is loaded we want to be a good file and report back who
% we are, for this the package will have defined |\ekvd@tmp| to use
% |\ProvidesFile|, else this will expand to a |\relax| and do no harm.
%    \begin{macrocode}
\csname ekvd@tmp\endcsname
%    \end{macrocode}
%
% Store the category code of |@| to later be able to reset it and change it to
% 11 for now.
%    \begin{macrocode}
\expandafter\chardef\csname ekvd@tmp\endcsname=\catcode`\@
\catcode`\@=11
%    \end{macrocode}
% |\ekvd@tmp| will be reused later to handle expansion during the key
% defining. But we don't need it to ever store information long-term after
% \expkvd\ was initialized.
%
% \begin{macro}[internal]{\ekvd@ifprimitive}
%    \begin{macrocode}
\protected\long\def\ekvd@ifprimitive#1%
  {%
    \begingroup
      \edef\ekvd@tmpa{\string #1}%
      \edef\ekvd@tmpb{\meaning#1}%
      \expandafter
    \endgroup
    \ifx\ekvd@tmpa\ekvd@tmpb
      \ekv@fi@firstoftwo
    \fi
    \@secondoftwo
  }
%    \end{macrocode}
% \end{macro}
%
% \begin{macro}[internal]
%   {\ekvd@long,\ekvd@prot,\ekvd@clear@prefixes,\ekvd@ifalso}
% \expkvd\ will use |\ekvd@long|, |\ekvd@prot|, and |\ekvd@ifalso| to store
% whether a key should be defined as |\long| or |\protected| or adds an action
% to an existing key, and we have to clear them for every new key. By default
% |long| and |protected| will just be empty, |ifalso| will be
% |\@secondoftwo|, and |ifnew| will just use its third argument.
%    \begin{macrocode}
\protected\def\ekvd@clear@prefixes
  {%
    \let\ekvd@long\ekv@empty
    \let\ekvd@prot\ekv@empty
    \let\ekvd@ifalso\@secondoftwo
    \long\def\ekvd@ifnew##1##2##3{##3}%
  }
\ekvd@clear@prefixes
%    \end{macrocode}
% \end{macro}
%
% \begin{macro}{\ekvdefinekeys}
% This is the one front-facing macro which provides the interface to define
% keys. It's using |\ekvparse| to handle the \kv\ list, the interpretation will
% be done by |\ekvd@noarg| and |\ekvd@|. The \set\ for which the keys should be
% defined is stored in |\ekvd@set|.
%    \begin{macrocode}
\protected\def\ekvdefinekeys#1%
  {%
    \def\ekvd@set{#1}%
    \ekvparse\ekvd@noarg\ekvd@arg
  }
%    \end{macrocode}
% \end{macro}
%
% \begin{macro}[internal]{\ekvd@noarg,\ekvd@arg,\ekvd@handle}
% |\ekvd@noarg| and |\ekvd@arg| store whether there was a value in the \kv\
% pair. |\ekvd@handle| has to test whether there is a space inside the key and
% if so calls the prefix grabbing routine, else we throw an error and ignore the
% key.
%    \begin{macrocode}
\protected\long\def\ekvd@noarg#1%
  {%
    \let\ekvd@ifnoarg\@firstoftwo
    \expandafter\ekvd@handle\detokenize{#1}\ekvd@stop{}%
  }
\protected\long\def\ekvd@arg#1%
  {%
    \let\ekvd@ifnoarg\@secondoftwo
    \expandafter\ekvd@handle\detokenize{#1}\ekvd@stop
  }
\protected\long\def\ekvd@handle#1\ekvd@stop#2%
  {%
    \ekvd@clear@prefixes
    \def\ekvd@cur{#1}%
    \ekvd@ifspace{#1}%
      {\ekvd@prefix\ekv@mark#1\ekv@stop{#2}}%
      \ekvd@err@missing@type
  }
%    \end{macrocode}
% \end{macro}
%
% \begin{macro}[internal]{\ekvd@prefix,\ekvd@prefix@}
% \expkvd\ separates prefixes into two groups, the first being prefixes in the
% \TeX\ sense (|long| and |protected|) which use |@p@| in their name, the other
% being key-types (|code|, |int|, \emph{etc.}) which use |@t@| instead.
% |\ekvd@prefix| splits at the first space and checks whether its a |@p@|
% or |@t@| type prefix. If it is neither throw an error and gobble the
% definition (the value).
%    \begin{macrocode}
\protected\def\ekvd@prefix#1 {\ekv@strip{#1}\ekvd@prefix@\ekv@mark}
\protected\def\ekvd@prefix@#1#2\ekv@stop
  {%
    \ekv@ifdefined{ekvd@t@#1}%
      {\ekv@strip{#2}{\csname ekvd@t@#1\endcsname}}%
      {%
        \ekv@ifdefined{ekvd@p@#1}%
          {\csname ekvd@p@#1\endcsname\ekvd@prefix@after@p{#2}}%
          {\ekvd@err@undefined@prefix{#1}\@gobble}%
      }%
  }
%    \end{macrocode}
% \end{macro}
%
% \begin{macro}[internal]{\ekvd@prefix@after@p}
% The |@p@| type prefixes are all just modifying a following |@t@| type, so they
% will need to search for another prefix. This is true for all of them, so we
% use a macro to handle this. It'll throw an error if there is no other prefix.
%    \begin{macrocode}
\protected\def\ekvd@prefix@after@p#1%
  {%
    \ekvd@ifspace{#1}%
      {\ekvd@prefix#1\ekv@stop}%
      {\ekvd@err@missing@type\@gobble}%
  }
%    \end{macrocode}
% \end{macro}
%
% \begin{macro}[internal]
%   {\ekvd@p@long,\ekvd@p@protected,\ekvd@p@protect,\ekvd@p@also,\ekvd@p@new}
% Define the |@p@| type prefixes, they all just store some information in a
% temporary macro.
%    \begin{macrocode}
\protected\def\ekvd@p@long{\let\ekvd@long\long}
\protected\def\ekvd@p@protected{\let\ekvd@prot\protected}
\let\ekvd@p@protect\ekvd@p@protected
\protected\def\ekvd@p@also{\let\ekvd@ifalso\@firstoftwo}
\protected\def\ekvd@p@new{\let\ekvd@ifnew\ekvd@assert@new}
%    \end{macrocode}
% \end{macro}
%
% \subsubsection{Key Types}
%
% \begin{macro}[internal]{\ekvd@type@set,\ekvd@t@set}
% The |set| type is quite straight forward, just define a |NoVal| key to call
% |\ekvchangeset|.
%    \begin{macrocode}
\protected\def\ekvd@type@set#1#2%
  {%
    \ekvd@assert@not@long
    \ekvd@assert@not@protected
    \ekvd@ifnew{NoVal}{#1}%
      {%
        \ekv@ifempty{#2}%
          {\ekvd@err@missing@definition}%
          {%
            \ekvd@ifalso
              {%
                \ekv@expargtwice{\ekvd@add@noval{#1}}%
                  {\ekvchangeset{#2}}%
                  \ekvd@assert@not@protected@also
              }%
              {\ekv@expargtwice{\ekvdefNoVal\ekvd@set{#1}}{\ekvchangeset{#2}}}%
          }%
      }%
  }
\protected\def\ekvd@t@set#1#2%
  {%
    \ekvd@ifnoarg
      {\ekvd@type@set{#1}{#1}}%
      {\ekvd@type@set{#1}{#2}}%
  }
%    \end{macrocode}
% \end{macro}
%
% \begin{macro}[internal]{\ekvd@type@noval,\ekvd@t@noval,\ekvd@t@enoval}
% Another pretty simple type, |noval| just needs to assert that there is a
% definition and that |long| wasn't specified.
% There are types where the difference in the variants is so small, that we
% define a common handler for them, those common handlers are named with
% |@type@|. |noval| and |enoval| are so similar that we can use such a |@type@|
% macro, even if we could've done |noval| in a slightly faster way without it.
%    \begin{macrocode}
\protected\long\def\ekvd@type@noval#1#2#3%
  {%
    \ekvd@ifnew{NoVal}{#2}%
      {%
        \ekvd@assert@arg
          {%
            \ekvd@assert@not@long
            \ekvd@prot#1\ekvd@tmp{#3}%
            \ekvd@ifalso
              {\ekv@exparg{\ekvd@add@noval{#2}}\ekvd@tmp{}}%
              {\ekvletNoVal\ekvd@set{#2}\ekvd@tmp}%
          }%
      }%
  }
\protected\def\ekvd@t@noval{\ekvd@type@noval\def}
\protected\def\ekvd@t@enoval{\ekvd@type@noval\edef}
%    \end{macrocode}
% \end{macro}
%
% \begin{macro}[internal]{\ekvd@type@code,\ekvd@t@code,\ekvd@t@ecode}
% |code| is simple as well, |ecode| has to use |\edef| on a temporary macro,
% since \expkv\ doesn't provide an |\ekvedef|.
%    \begin{macrocode}
\protected\long\def\ekvd@type@code#1#2#3%
  {%
    \ekvd@ifnew{}{#2}%
      {%
        \ekvd@assert@arg
          {%
            \ekvd@prot\ekvd@long#1\ekvd@tmp##1{#3}%
            \ekvd@ifalso
              {\ekv@exparg{\ekvd@add@val{#2}}{\ekvd@tmp{##1}}{}}%
              {\ekvlet\ekvd@set{#2}\ekvd@tmp}%
          }%
      }%
  }
\protected\def\ekvd@t@code{\ekvd@type@code\def}
\protected\def\ekvd@t@ecode{\ekvd@type@code\edef}
%    \end{macrocode}
% \end{macro}
%
% \begin{macro}[internal]
%   {
%     \ekvd@type@default,\ekvd@t@default,\ekvd@t@qdefault,\ekvd@t@odefault,
%     \ekvd@t@fdefault
%   }
% |\ekvd@type@default| asserts there was an argument, also the key for which one
% wants to set a default has to be already defined (this is not so important for
% |default|, but |qdefault| requires is). If everything is good, |\edef| a
% temporary macro that expands |\ekvd@set| and the |\csname| for the key, and in
% the case of |qdefault| does the first expansion step of the key-macro.
%    \begin{macrocode}
\protected\long\def\ekvd@type@default#1#2#3#4%
  {%
    \ekvd@assert@arg
      {%
        \ekvifdefined\ekvd@set{#3}%
          {%
            \ekvd@assert@not@new
            \ekvd@assert@not@long
            \ekvd@prot\edef\ekvd@tmp
              {%
                \ekv@unexpanded\expandafter#1%
                  {#2\csname\ekv@name\ekvd@set{#3}\endcsname{#4}}%
              }%
            \ekvd@ifalso
              {\ekv@exparg{\ekvd@add@noval{#3}}\ekvd@tmp{}}%
              {\ekvletNoVal\ekvd@set{#3}\ekvd@tmp}%
          }%
          {\ekvd@err@undefined@key{#3}}%
      }%
  }
\protected\def\ekvd@t@default{\ekvd@type@default{}{}}
\protected\def\ekvd@t@qdefault{\ekvd@type@default{\expandafter\expandafter}{}}
\let\ekvd@t@odefault\ekvd@t@qdefault
\protected\def\ekvd@t@fdefault{\ekvd@type@default{}{\romannumeral`\^^@}}
%    \end{macrocode}
% \end{macro}
%
% \begin{macro}[internal]{\ekvd@t@edefault}
% |edefault| is too different from |default| and |qdefault| to reuse the
% |@type@| macro, as it doesn't need |\unexpanded| inside of |\edef|.
%    \begin{macrocode}
\protected\long\def\ekvd@t@edefault#1#2%
  {%
    \ekvd@assert@arg
      {%
        \ekvifdefined\ekvd@set{#1}%
          {%
            \ekvd@assert@not@new
            \ekvd@assert@not@long
            \ekvd@prot\edef\ekvd@tmp
              {\csname\ekv@name\ekvd@set{#1}\endcsname{#2}}%
            \ekvd@ifalso
              {\ekv@exparg{\ekvd@add@noval{#1}}\ekvd@tmp{}}%
              {\ekvletNoVal\ekvd@set{#1}\ekvd@tmp}%
          }%
          {\ekvd@err@undefined@key{#1}}%
      }%
  }
%    \end{macrocode}
% \end{macro}
%
% \begin{macro}[internal]
%   {\ekvd@t@initial,\ekvd@t@oinitial,\ekvd@t@finitial,\ekvd@t@einitial}
%    \begin{macrocode}
\long\def\ekvd@type@initial#1#2#3#4%
  {%
    \ekvd@assert@not@new
    \ekvd@assert@not@also
    \ekvd@assert@not@long
    \ekvd@assert@not@protected
    \ekvd@ifnoarg
      {%
        \ekvifdefinedNoVal\ekvd@set{#3}%
          {\csname\ekv@name\ekvd@set{#3}N\endcsname}%
          {\ekvd@err@undefined@noval{#3}}%
      }%
      {%
        \ekvifdefined\ekvd@set{#3}%
          {%
            #1{#2#4}%
            \csname\ekv@name\ekvd@set{#3}\expandafter\endcsname\expandafter
              {\ekvd@tmp}%
          }%
          {\ekvd@err@undefined@key{#3}}%
      }%
  }
\def\ekvd@t@initial{\ekvd@type@initial{\def\ekvd@tmp}{}}
\def\ekvd@t@oinitial{\ekvd@type@initial{\ekv@exparg{\def\ekvd@tmp}}{}}
\def\ekvd@t@einitial{\ekvd@type@initial{\edef\ekvd@tmp}{}}
\def\ekvd@t@finitial
  {\ekvd@type@initial{\ekv@exparg{\def\ekvd@tmp}}{\romannumeral`\^^@}}
%    \end{macrocode}
% \end{macro}
%
% \begin{macro}[internal]
%   {
%     \ekvd@type@bool,\ekvd@t@bool,\ekvd@t@gbool,\ekvd@t@boolTF,\ekvd@t@gboolTF,
%     \ekvd@t@invbool,\ekvd@t@ginvbool,\ekvd@t@invboolTF,\ekvd@t@ginvboolTF,
%   }
% The boolean types are a quicker version of a |choice| that accept |true| and
% |false|, and set up the |NoVal| action to be identical to \texttt{\key=true}.
% The |true| and |false| actions are always just |\let|ting the macro in |#7| to
% some other macro (\emph{e.g.}, \cs[no-index]{iftrue}).
%    \begin{macrocode}
\protected\def\ekvd@type@bool#1#2#3#4#5%
  {%
    \ekvd@ifnew{}{#4}%
      {%
        \ekvd@ifnew{NoVal}{#4}%
          {%
            \ekvd@assert@filledarg{#5}%
              {%
                \ekvd@newlet#5#3%
                \ekvd@type@choice{#4}%
                \protected\ekvdefNoVal\ekvd@set{#4}{#1\let#5#2}%
                \protected\expandafter\def
                  \csname\ekvd@choice@name\ekvd@set{#4}{true}\endcsname
                  {#1\let#5#2}%
                \protected\expandafter\def
                  \csname\ekvd@choice@name\ekvd@set{#4}{false}\endcsname
                  {#1\let#5#3}%
              }%
          }%
      }%
  }
\protected\def\ekvd@t@bool{\ekvd@type@bool{}\iftrue\iffalse}
\protected\def\ekvd@t@gbool{\ekvd@type@bool\global\iftrue\iffalse}
\protected\def\ekvd@t@boolTF{\ekvd@type@bool{}\@firstoftwo\@secondoftwo}
\protected\def\ekvd@t@gboolTF{\ekvd@type@bool\global\@firstoftwo\@secondoftwo}
\protected\def\ekvd@t@invbool{\ekvd@type@bool{}\iffalse\iftrue}
\protected\def\ekvd@t@ginvbool{\ekvd@type@bool\global\iffalse\iftrue}
\protected\def\ekvd@t@invboolTF{\ekvd@type@bool{}\@secondoftwo\@firstoftwo}
\protected\def\ekvd@t@ginvboolTF
  {\ekvd@type@bool\global\@secondoftwo\@firstoftwo}
%    \end{macrocode}
% \end{macro}
%
% \begin{macro}[internal]
%   {
%     \ekvd@type@boolpair,
%     \ekvd@t@boolpair,\ekvd@t@gboolpair,\ekvd@t@boolpairTF,\ekvd@t@gboolpairTF
%   }
% The boolean pair types are essentially the same as the boolean types, but set
% two macros instead of one.
%    \begin{macrocode}
\protected\def\ekvd@type@boolpair#1#2#3#4#5#6%
  {%
    \ekvd@ifnew{}{#4}%
      {%
        \ekvd@ifnew{NoVal}{#4}%
          {%
            \ekvd@newlet#5#3%
            \ekvd@newlet#6#2%
            \ekvd@type@choice{#4}%
            \protected\ekvdefNoVal\ekvd@set{#4}{#1\let#5#2#1\let#6#3}%
            \protected\expandafter\def
              \csname\ekvd@choice@name\ekvd@set{#4}{true}\endcsname
              {#1\let#5#2#1\let#6#3}%
            \protected\expandafter\def
              \csname\ekvd@choice@name\ekvd@set{#4}{false}\endcsname
              {#1\let#5#3#1\let#6#2}%
          }%
      }%
  }
\protected\def\ekvd@t@boolpair#1#2%
  {\ekvd@assert@twoargs{#2}{\ekvd@type@boolpair{}\iftrue\iffalse{#1}#2}}
\protected\def\ekvd@t@gboolpair#1#2%
  {\ekvd@assert@twoargs{#2}{\ekvd@type@boolpair\global\iftrue\iffalse{#1}#2}}
\protected\def\ekvd@t@boolpairTF#1#2%
  {%
    \ekvd@assert@twoargs{#2}%
      {\ekvd@type@boolpair{}\@firstoftwo\@secondoftwo{#1}#2}%
  }
\protected\def\ekvd@t@gboolpairTF#1#2%
  {%
    \ekvd@assert@twoargs{#2}%
      {\ekvd@type@boolpair\global\@firstoftwo\@secondoftwo{#1}#2}%
  }
%    \end{macrocode}
% \end{macro}
%
% \begin{macro}[internal]
%   {
%     \ekvd@type@data,\ekvd@t@data,\ekvd@t@gdata,\ekvd@t@dataT,\ekvd@t@gdataT
%   }
%    \begin{macrocode}
\protected\def\ekvd@type@data#1#2#3#4#5#6%
  {%
    \ekvd@ifnew{}{#5}%
      {%
        \ekvd@assert@filledarg{#6}%
          {%
            \ekvd@newlet#6#1%
            \ekvd@ifalso
              {%
                \let\ekvd@prot\protected
                \ekvd@add@val{#5}{\long#2#6####1#3{####1{#4}}}{}%
              }%
              {%
                \protected\ekvd@long\ekvdef\ekvd@set{#5}%
                  {\long#2#6####1#3{####1{#4}}}%
              }%
          }%
      }%
  }
\protected\def\ekvd@t@data
  {\ekvd@type@data\@secondoftwo\edef{####2}{\ekv@unexpanded{##1}}}
\protected\def\ekvd@t@edata{\ekvd@type@data\@secondoftwo\edef{####2}{##1}}
\protected\def\ekvd@t@gdata
  {\ekvd@type@data\@secondoftwo\xdef{####2}{\ekv@unexpanded{##1}}}
\protected\def\ekvd@t@xdata{\ekvd@type@data\@secondoftwo\xdef{####2}{##1}}
\protected\def\ekvd@t@dataT
  {\ekvd@type@data\@gobble\edef{}{\ekv@unexpanded{##1}}}
\protected\def\ekvd@t@edataT{\ekvd@type@data\@gobble\edef{}{##1}}
\protected\def\ekvd@t@gdataT
  {\ekvd@type@data\@gobble\xdef{}{\ekv@unexpanded{##1}}}
\protected\def\ekvd@t@xdataT{\ekvd@type@data\@gobble\xdef{}{##1}}
%    \end{macrocode}
% \end{macro}
%
% \begin{macro}[internal]{\ekvd@type@box,\ekvd@t@box,\ekvd@t@gbox}
% Set up our boxes. Though we're a generic package we want to be colour safe, so
% we put an additional grouping level inside the box contents, for the case that
% someone uses \pkg{color}. |\ekvd@newreg| is a small wrapper which tests
% whether the first argument is defined and if not does
% |\csname new#2\endcsname#1|.
%    \begin{macrocode}
\protected\def\ekvd@type@box#1#2#3%
  {%
    \ekvd@ifnew{}{#2}%
      {%
        \ekvd@assert@filledarg{#3}%
          {%
            \ekvd@newreg#3{box}%
            \ekvd@ifalso
              {%
                \let\ekvd@prot\protected
                \ekvd@add@val{#2}{#1\setbox#3=\hbox{\begingroup##1\endgroup}}{}%
              }%
              {%
                \protected\ekvd@long\ekvdef\ekvd@set{#2}%
                  {#1\setbox#3=\hbox{\begingroup##1\endgroup}}%
              }%
          }%
      }%
  }
\protected\def\ekvd@t@box{\ekvd@type@box{}}
\protected\def\ekvd@t@gbox{\ekvd@type@box\global}
%    \end{macrocode}
% \end{macro}
%
% \begin{macro}[internal]{\ekvd@type@toks,\ekvd@t@toks,\ekvd@t@gtoks}
% Similar to |box|, but set the |toks|.
%    \begin{macrocode}
\protected\def\ekvd@type@toks#1#2#3%
  {%
    \ekvd@ifnew{}{#2}%
      {%
        \ekvd@assert@filledarg{#3}%
          {%
            \ekvd@newreg#3{toks}%
            \ekvd@ifalso
              {%
                \let\ekvd@prot\protected
                \ekvd@add@val{#2}{#1#3={##1}}{}%
              }%
              {\protected\ekvd@long\ekvdef\ekvd@set{#2}{#1#3={##1}}}%
          }%
      }%
  }
\protected\def\ekvd@t@toks{\ekvd@type@toks{}}
\protected\def\ekvd@t@gtoks{\ekvd@type@toks\global}
%    \end{macrocode}
% \end{macro}
%
% \begin{macro}[internal]{\ekvd@type@preapptoks,\ekvd@t@apptoks,\ekvd@t@gapptoks}
% Just like |toks|, but expand the current contents of the |toks| register to
% append the new contents.
%    \begin{macrocode}
\ekvd@ifprimitive\toksapp
  {%
    \protected\def\ekvd@type@preapptoks#1#2#3%
      {%
        \ekvd@ifnew{}{#2}%
          {%
            \ekvd@assert@filledarg{#3}%
              {%
                \ekvd@newreg#3{toks}%
                \ekvd@ifalso
                  {%
                    \let\ekvd@prot\protected
                    \ekvd@add@val{#2}{#1#3{##1}}{}%
                  }%
                  {\protected\ekvd@long\ekvdef\ekvd@set{#2}{#1#3{##1}}}%
              }%
          }%
      }
    \protected\def\ekvd@t@apptoks{\ekvd@type@preapptoks\toksapp}
    \protected\def\ekvd@t@gapptoks{\ekvd@type@preapptoks\gtoksapp}
    \protected\def\ekvd@t@pretoks{\ekvd@type@preapptoks\tokspre}
    \protected\def\ekvd@t@gpretoks{\ekvd@type@preapptoks\gtokspre}
  }
  {%
    \protected\def\ekvd@type@apptoks#1#2#3%
      {%
        \ekvd@ifnew{}{#2}%
          {%
            \ekvd@assert@filledarg{#3}%
              {%
                \ekvd@newreg#3{toks}%
                \ekvd@ifalso
                  {%
                    \let\ekvd@prot\protected
                    \ekvd@add@val{#2}{#1#3=\expandafter{\the#3##1}}{}%
                  }%
                  {%
                    \protected\ekvd@long\ekvdef\ekvd@set{#2}%
                      {#1#3=\expandafter{\the#3##1}}%
                  }%
              }%
          }%
      }
    \protected\def\ekvd@t@apptoks{\ekvd@type@apptoks{}}
    \protected\def\ekvd@t@gapptoks{\ekvd@type@apptoks\global}
    \newtoks\ekvd@toks
    \protected\def\ekvd@type@pretoks#1#2#3%
      {%
        \ekvd@ifnew{}{#2}%
          {%
            \ekvd@assert@filledarg{#3}%
              {%
                \ekvd@newreg#3{toks}%
                \ekvd@ifalso
                  {%
                    \let\ekvd@prot\protected
                    \ekvd@add@val{#2}%
                      {%
                        \ekvd@toks={##1}%
                        #1#3=\expandafter{\the\expandafter\ekvd@toks\the#3}%
                      }%
                      {}%
                  }%
                  {%
                    \protected\ekvd@long\ekvdef\ekvd@set{#2}%
                      {%
                        \ekvd@toks={##1}%
                        #1#3=\expandafter{\the\expandafter\ekvd@toks\the#3}%
                      }%
                  }%
              }%
          }%
      }
    \protected\def\ekvd@t@pretoks{\ekvd@type@pretoks{}}
    \protected\def\ekvd@t@gpretoks{\ekvd@type@pretoks\global}
  }
%    \end{macrocode}
% \end{macro}
%
% \begin{macro}[internal]
%   {
%     \ekvd@type@reg,\ekvd@t@int,\ekvd@t@eint,\ekvd@t@gint,\ekvd@t@xint,
%     \ekvd@t@dimen,\ekvd@t@edimen,\ekvd@t@gdimen,\ekvd@t@xdimen,
%     \ekvd@t@skip,\ekvd@t@eskip,\ekvd@t@gskip,\ekvd@t@xskip
%   }
% The |\ekvd@type@reg| can handle all the types for which the assignment will
% just be \texttt{\meta{register}=\meta{value}}.
%    \begin{macrocode}
\protected\def\ekvd@type@reg#1#2#3#4#5#6%
  {%
    \ekvd@ifnew{}{#5}%
      {%
        \ekvd@assert@filledarg{#6}%
          {%
            \ekvd@newreg#6{#1}%
            \ekvd@ifalso
              {%
                \let\evkd@prot\protected
                \ekvd@add@val{#5}{#2#6=#3##1#4\relax}{}%
              }%
              {\protected\ekvd@long\ekvdef\ekvd@set{#5}{#2#6=#3##1#4\relax}}%
          }%
      }%
  }
\protected\def\ekvd@t@int{\ekvd@type@reg{count}{}{}{}}
\protected\def\ekvd@t@eint{\ekvd@type@reg{count}{}\numexpr\relax}
\protected\def\ekvd@t@gint{\ekvd@type@reg{count}\global{}{}}
\protected\def\ekvd@t@xint{\ekvd@type@reg{count}\global\numexpr\relax}
\protected\def\ekvd@t@dimen{\ekvd@type@reg{dimen}{}{}{}}
\protected\def\ekvd@t@edimen{\ekvd@type@reg{dimen}{}\dimexpr\relax}
\protected\def\ekvd@t@gdimen{\ekvd@type@reg{dimen}\global{}{}}
\protected\def\ekvd@t@xdimen{\ekvd@type@reg{dimen}\global\dimexpr\relax}
\protected\def\ekvd@t@skip{\ekvd@type@reg{skip}{}{}{}}
\protected\def\ekvd@t@eskip{\ekvd@type@reg{skip}{}\glueexpr\relax}
\protected\def\ekvd@t@gskip{\ekvd@type@reg{skip}\global{}{}}
\protected\def\ekvd@t@xskip{\ekvd@type@reg{skip}\global\glueexpr\relax}
%    \end{macrocode}
% \end{macro}
%
% \begin{macro}[internal]{\ekvd@type@store,\ekvd@t@store,\ekvd@t@gstore}
% The none-expanding |store| types use an |\edef| or |\xdef| and |\unexpanded|
% to be able to also store |#| easily.
%    \begin{macrocode}
\protected\def\ekvd@type@store#1#2#3#4%
  {%
    \ekvd@ifnew{}{#3}%
      {%
        \ekvd@assert@filledarg{#4}%
          {%
            \ekvd@newlet#4\ekv@empty
            \ekvd@ifalso
              {%
                \let\ekvd@prot\protected
                \ekvd@add@val{#3}{#1#4{#2}}{}%
              }%
              {\protected\ekvd@long\ekvdef\ekvd@set{#3}{#1#4{#2}}}%
          }%
      }%
  }
\protected\def\ekvd@t@store{\ekvd@type@store\edef{\ekv@unexpanded{##1}}}
\protected\def\ekvd@t@gstore{\ekvd@type@store\xdef{\ekv@unexpanded{##1}}}
\protected\def\ekvd@t@estore{\ekvd@type@store\edef{##1}}
\protected\def\ekvd@t@xstore{\ekvd@type@store\xdef{##1}}
%    \end{macrocode}
% \end{macro}
%
% \begin{macro}[internal]
%   {
%     \ekvd@type@meta,\ekvd@type@meta@a,\ekvd@type@meta@b,\ekvd@type@meta@c,
%     \ekvd@t@meta,\ekvd@t@nmeta
%   }
% |meta| sets up things such that another instance of |\ekvset| will be run on
% the argument, with the same \set.
%    \begin{macrocode}
\protected\long\def\ekvd@type@meta#1#2#3#4#5#6#7%
  {%
    \ekvd@ifnew{#1}{#6}%
      {%
        \ekvd@assert@filledarg{#7}%
          {%
            \edef\ekvd@tmp{\ekvd@set}%
            \expandafter\ekvd@type@meta@a\expandafter{\ekvd@tmp}{#7}{#2}%
            \ekvd@ifalso
              {\ekv@exparg{#3{#6}}{\ekvd@tmp#4}{#5}}%
              {\csname ekvlet#1\endcsname\ekvd@set{#6}\ekvd@tmp}%
          }%
      }%
  }
\protected\long\def\ekvd@type@meta@a#1#2%
  {%
    \expandafter\ekvd@type@meta@b\expandafter{\ekvset{#1}{#2}}%
  }
\protected\def\ekvd@type@meta@b
  {%
    \expandafter\ekvd@type@meta@c\expandafter
  }
\protected\long\def\ekvd@type@meta@c#1#2%
  {%
    \ekvd@prot\ekvd@long\def\ekvd@tmp#2{#1}%
  }
\protected\def\ekvd@t@meta{\ekvd@type@meta{}{##1}\ekvd@add@val{{##1}}{}}
\protected\def\ekvd@t@nmeta
  {%
    \ekvd@assert@not@long
    \ekvd@type@meta{NoVal}{}\ekvd@add@noval{}\ekvd@assert@not@long@also
  }
%    \end{macrocode}
% \end{macro}
%
% \begin{macro}[internal]
%   {\ekvd@type@smeta,\ekvd@type@smeta@,\ekvd@t@smeta,\ekvd@t@snmeta}
% |smeta| is pretty similar to |meta|, but needs two arguments inside of \val,
% such that the first is the \set\ for which the sub-|\ekvset| and the second is
% the \kv\ list.
%    \begin{macrocode}
\protected\long\def\ekvd@type@smeta#1#2#3#4#5#6#7%
  {%
    \ekvd@ifnew{#1}{#6}%
      {%
        \ekvd@assert@twoargs{#7}%
          {%
            \ekvd@type@meta@a#7{#2}%
            \ekvd@ifalso
              {\ekv@exparg{#3{#6}}{\ekvd@tmp#4}{#5}}%
              {\csname ekvlet#1\endcsname\ekvd@set{#6}\ekvd@tmp}%
          }%
      }%
  }
\protected\def\ekvd@t@smeta{\ekvd@type@smeta{}{##1}\ekvd@add@val{{##1}}{}}
\protected\def\ekvd@t@snmeta
  {%
    \ekvd@assert@not@long
    \ekvd@type@smeta{NoVal}{}\ekvd@add@noval{}\ekvd@assert@not@long@also
  }
%    \end{macrocode}
% \end{macro}
%
% \begin{macro}[internal]
%   {
%     \ekvd@type@choice,
%     \ekvd@populate@choice,\ekvd@populate@choice@,\ekvd@populate@choice@noarg,
%     \ekvd@choice@prefix,\ekvd@choice@prefix@,\ekvd@choice@prefix@done,
%     \ekvd@choice@p@protected,\ekvd@choice@p@protect,
%     \ekvd@choice@p@long,\ekvd@choice@p@long@,
%     \ekvd@t@choice
%   }
% The |choice| type is by far the most complex type, as we have to run a
% sub-parser on the choice-definition list, which should support the |@p@| type
% prefixes as well (but |long| will always throw an error, as they are not
% allowed to be long). |\ekvd@type@choice| will just define the choice-key, the
% handling of the choices definition will be done by |\ekvd@populate@choice|.
%    \begin{macrocode}
\protected\def\ekvd@type@choice#1%
  {%
    \ekvd@assert@not@long
    \ekv@exparg{\ekvd@prot\def\ekvd@tmp##1}%
      {%
        \expandafter\ekvd@h@choice\expandafter
          {\ekvd@choice@name\ekvd@set{#1}{##1}}%
      }%
    \ekvd@ifalso
      {%
        \ekvd@assert@val{#1}%
          {%
            \ekvd@if@not@already@choice{#1}%
              {%
                \ekv@exparg
                  {%
                    \expandafter\ekvd@add@aux
                      \csname\ekv@name\ekvd@set{#1}\endcsname{{##1}}{#1}%
                  }%
                  {\ekvd@tmp{##1}}%
                  {\ekvd@long\ekvdef}\ekvd@assert@not@long@also
              }%
          }%
      }%
      {\ekvlet\ekvd@set{#1}\ekvd@tmp}%
  }
%    \end{macrocode}
% |\ekvd@populate@choice| just uses |\ekvparse| and then gives control to
% |\ekvd@populate@choice@noarg|, which throws an error, and
% |\ekvd@populate@choice@|.
%    \begin{macrocode}
\protected\def\ekvd@populate@choice
  {%
    \ekvparse\ekvd@populate@choice@noarg\ekvd@populate@choice@
  }
\protected\long\def\ekvd@populate@choice@noarg#1%
  {%
    \expandafter\ekvd@err@missing@definition@msg\expandafter{\ekvd@cur : #1}%
  }
%    \end{macrocode}
% |\ekvd@populate@choice@| runs the prefix-test, if there is none we can
% directly define the choice, for that |\ekvd@set@choice| will expand to the
% current choice-key's name, which will have been defined by |\ekvd@t@choice|.
% If there is a prefix run the prefix grabbing routine, which was altered for
% |@type@choice|.
%    \begin{macrocode}
\protected\long\def\ekvd@populate@choice@#1#2%
  {%
    \ekvd@clear@prefixes
    \ekvd@ifspace{#1}%
      {\ekvd@choice@prefix{\ekv@mark#1}\ekv@mark#1\ekv@stop}%
      {%
        \expandafter\edef
          \csname\ekvd@choice@name\ekvd@set\ekvd@set@choice{#1}\endcsname
      }%
      {\unexpanded{#2}}%
  }
\protected\def\ekvd@choice@prefix#1#2
  {%
    \ekv@strip{#2}{\ekvd@choice@prefix@{#1}}\ekv@mark
  }
\protected\def\ekvd@choice@prefix@#1#2#3\ekv@stop
  {%
    \ekv@ifdefined{ekvd@choice@p@#2}%
      {%
        \csname ekvd@choice@p@#2\endcsname
        \ekvd@ifspace{#3}%
          {\ekvd@choice@prefix{#3}#3\ekv@stop}%
          {\ekvd@choice@prefix@done{#3}}%
      }%
      {\ekvd@choice@prefix@done{#1}}%
  }
\protected\def\ekvd@choice@prefix@done#1%
  {%
    \ekvd@prot\expandafter\edef
      \csname
        \ekv@strip{#1}{\ekvd@choice@name\ekvd@set\ekvd@set@choice}%
      \endcsname
  }
\protected\def\ekvd@choice@p@protected{\let\ekvd@prot\protected}
\let\ekvd@choice@p@protect\ekvd@choice@p@protected
\protected\def\ekvd@choice@invalid@p#1\ekvd@ifspace#2%
  {%
    \expandafter\ekvd@choice@invalid@p@\expandafter{\ekv@gobble@mark#2}{#1}%
    \ekvd@ifspace{#2}%
  }
\protected\def\ekvd@choice@invalid@p@#1#2%
  {%
    \expandafter\ekvd@err@no@prefix@msg\expandafter{\ekvd@cur : #2 #1}{#2}%
  }
\protected\def\ekvd@choice@p@long{\ekvd@choice@invalid@p{long}}%
\protected\def\ekvd@choice@p@also{\ekvd@choice@invalid@p{also}}%
\protected\def\ekvd@choice@p@new{\ekvd@choice@invalid@p{new}}%
%    \end{macrocode}
% Finally we're able to set up the |@t@choice| macro, which has to store the
% current choice-key's name, define the key, and parse the available choices.
%    \begin{macrocode}
\protected\long\def\ekvd@t@choice#1#2%
  {%
    \ekvd@ifnew{}{#1}%
      {%
        \ekvd@assert@arg
          {%
            \ekvd@type@choice{#1}%
            \def\ekvd@set@choice{#1}%
            \ekvd@populate@choice{#2}%
          }%
      }%
  }
%    \end{macrocode}
% \end{macro}
%
% \begin{macro}[internal]{\ekvd@t@choice-store,\ekvd@t@choice-enum}
%   These two types define a special kind of |choice| key and are quite similar,
%   the only difference is what the different choices do (hence they use a
%   shared initialisation which differs in the chosen |populate| step).
%    \begin{macrocode}
\protected\long\expandafter\def\csname ekvd@t@choice-store\endcsname
  {\ekvd@type@choicespecial\ekvd@populate@choicestore}
\protected\long\expandafter\def\csname ekvd@t@choice-enum\endcsname
  {\ekvd@type@choicespecial\ekvd@populate@choiceenum}
%    \end{macrocode}
% \begin{macro}[internal]{\ekvd@type@choicespecial}
%   Initialise similar to a |choice| key. The difference is that we require two
%   arguments (which we assert), a macro to store things in, and a |csv|-list
%   containing the allowed values. |#1| is the |populate| macro according to the
%   type used.
%    \begin{macrocode}
\protected\long\def\ekvd@type@choicespecial#1#2#3%
  {%
    \ekvd@ifnew{}{#2}%
      {%
        \ekvd@assert@twoargs{#3}%
          {%
            \ekvd@type@choice{#2}%
            \def\ekvd@set@choice{#2}%
            #1#3%
          }%
      }%
  }
%    \end{macrocode}
% \end{macro}
% \begin{macro}[internal]
%   {\ekvd@populate@choicestore,\ekvd@populate@choicestore@}
%   We initialise the storing macro if it doesn't yet exist, and then we loop
%   over the value list. The |\edef|s with |\unexpanded| are both necessary to
%   be able to store macro parameter tokens (the outer protects at define time,
%   the inner at use time).
%    \begin{macrocode}
\protected\long\def\ekvd@populate@choicestore#1%
  {%
    \ekvd@newlet#1\ekv@empty
    \ekvcsvloop{\ekvd@populate@choicestore@#1}%
  }
\protected\long\def\ekvd@populate@choicestore@#1#2%
  {%
    \protected\expandafter\edef
      \csname\ekvd@choice@name\ekvd@set\ekvd@set@choice{#2}\endcsname
      {\unexpanded{\edef#1{\unexpanded{#2}}}}%
  }
%    \end{macrocode}
% \end{macro}
% \begin{macro}[internal]
%   {\ekvd@populate@choiceenum,\ekvd@populate@choiceenum@}
%   This is similar to the population of a |choice-store| type, but instead of
%   storing the values in a macro this initialises a count and stores the
%   position of the value in the list inside that count (zero-indexed). The
%   space is necessary to terminate the number scanning, which is the reason we
%   use |\@firstofone| (so that the space after the macro name isn't gobbled by
%   \TeX).
%    \begin{macrocode}
\protected\long\def\ekvd@populate@choiceenum#1%
  {%
    \ekvd@newreg#1{count}%
    \def\ekvd@tmp{0}%
    \ekvcsvloop{\ekvd@populate@choiceenum@#1}%
  }
\protected\long\def\ekvd@populate@choiceenum@#1#2%
  {%
    \protected\expandafter\edef
      \csname\ekvd@choice@name\ekvd@set\ekvd@set@choice{#2}\endcsname
      {#1=\@firstofone{\ekvd@tmp} }%
    \edef\ekvd@tmp{\the\numexpr\ekvd@tmp+1\relax}%
  }
%    \end{macrocode}
% \end{macro}
% \end{macro}
%
% \begin{macro}[internal]{\ekvd@t@unknown-choice}
%    \begin{macrocode}
\protected\long\expandafter\def\csname ekvd@t@unknown-choice\endcsname#1#2%
  {%
    \ekvd@assert@new@for@name{\ekvd@unknown@choice@name\ekvd@set{#1}}%
      {%
        \ekvd@assert@arg
          {%
            \ekvd@assert@not@long
            \ekvd@assert@not@also
            \ekvd@prot\expandafter
            \def\csname\ekvd@unknown@choice@name\ekvd@set{#1}\endcsname##1{#2}%
          }%
      }%
  }
%    \end{macrocode}
% \end{macro}
%
% \begin{macro}[internal]
%   {\ekvd@t@unknown,\ekvd@type@unknown@code,\ekvd@type@unknown@noval}
%   The |unknown| type has different subtypes which would be the key names for
%   other types. It is first checked whether that subtype is defined, if it
%   isn't throw an error, else use that subtype.
%    \begin{macrocode}
\protected\long\def\ekvd@t@unknown#1#2%
  {%
    \ekv@ifdefined{ekvd@type@unknown@\detokenize{#1}}%
      {\csname ekvd@type@unknown@\detokenize{#1}\endcsname{#2}}%
      \ekvd@err@misused@unknown
  }
%    \end{macrocode}
%   The |unknown noval| type can use |\ekvdefunknownNoVal| directly (after
%   asserting some prefixes).
%    \begin{macrocode}
\protected\long\def\ekvd@type@unknown@noval#1%
  {%
    \ekvd@assert@new@for@name{\ekv@name\ekvd@set{}uN}%
      {%
        \ekvd@assert@arg
          {%
            \ekvd@assert@not@also
            \ekvd@assert@not@long
            \ekvd@prot\ekvdefunknownNoVal\ekvd@set{#1}%
          }%
      }%
  }
%    \end{macrocode}
%   The |unknown code| type uses some trickery during the definition in order to
%   swap out |#1| and |#2| in the user supplied definition. This is done via a
%   temporary macro that stores the definition but gets the parameter numbers
%   reversed while the real definition is done.
%    \begin{macrocode}
\protected\long\def\ekvd@type@unknown@code#1%
  {%
    \ekvd@assert@new@for@name{\ekv@name\ekvd@set{}u}%
      {%
        \ekvd@assert@arg
          {%
            \ekvd@assert@not@also
            \begingroup
              \def\ekvd@tmp##1##2{#1}%
              \ekv@exparg
              {%
                \endgroup
                \ekvd@prot\ekvd@long\ekvdefunknown\ekvd@set
              }%
              {\ekvd@tmp{##2}{##1}}%
          }%
      }%
  }
%    \end{macrocode}
% \end{macro}
%
% \begin{macro}[internal]
%   {
%     \ekvd@type@unknown@redirect,
%     \ekvd@type@unknown@redirect-code,
%     \ekvd@type@unknown@redirect-noval
%   }
%   The |unknown redirect| types also just forward to |\ekvredirectunknown|
%   after asserting some prefixes.
%    \begin{macrocode}
\protected\edef\ekvd@type@unknown@redirect#1%
  {%
    \expandafter\noexpand\csname ekvd@type@unknown@redirect-code\endcsname{#1}%
    \expandafter\noexpand\csname ekvd@type@unknown@redirect-noval\endcsname{#1}%
  }
\protected\expandafter\def\csname ekvd@type@unknown@redirect-code\endcsname#1%
  {%
    \ekvd@assert@new@for@name{\ekv@name\ekvd@set{}u}%
      {%
        \ekvd@assert@arg
          {%
            \ekvd@assert@not@also
            \ekvd@assert@not@protected
            \expandafter\ekvredirectunknown\expandafter{\ekvd@set}{#1}%
          }%
      }%
  }
\protected\expandafter\def\csname ekvd@type@unknown@redirect-noval\endcsname#1%
  {%
    \ekvd@assert@new@for@name{\ekv@name\ekvd@set{}uN}%
      {%
        \ekvd@assert@arg
          {%
            \ekvd@assert@not@also
            \ekvd@assert@not@protected
            \ekvd@assert@not@long
            \expandafter\ekvredirectunknownNoVal\expandafter{\ekvd@set}{#1}%
          }%
      }%
  }
%    \end{macrocode}
% \end{macro}
%
% \subsubsection{Key Type Helpers}
% There are some keys that might need helpers during their execution (not during
% their definition, which are gathered as |@type@| macros). These helpers are
% named |@h@|.
%
% \begin{macro}[internal]{\ekvd@h@choice,\ekvd@h@choice@}
% The |choice| helper will just test whether the given choice was defined, if
% not throw an error expandably, else call the macro which stores the code for
% this choice.
%    \begin{macrocode}
\def\ekvd@h@choice#1%
  {%
    \expandafter\ekvd@h@choice@
      \csname\ifcsname#1\endcsname#1\else relax\fi\endcsname
      {#1}%
  }
\def\ekvd@h@choice@#1#2%
  {%
    \ifx#1\relax
      \ekvd@err@choice@invalid{#2}%
      \expandafter\@gobble
    \fi
    #1%
  }
%    \end{macrocode}
% \end{macro}
%
%
% \subsubsection{Handling \texttt{also}}
%
% \begin{macro}[internal]
%   {\ekvd@add@val,\ekvd@add@noval,\ekvd@add@aux,\ekvd@add@aux@}
%    \begin{macrocode}
\protected\long\def\ekvd@add@val#1#2#3%
  {%
    \ekvd@assert@val{#1}%
      {%
        \expandafter\ekvd@add@aux\csname\ekv@name\ekvd@set{#1}\endcsname{{##1}}%
          {#1}{#2}{\ekvd@long\ekvdef}{#3}%
      }%
  }
\protected\long\def\ekvd@add@noval#1#2#3%
  {%
    \ekvd@assert@noval{#1}%
      {%
        \expandafter\ekvd@add@aux\csname\ekv@name\ekvd@set{#1}N\endcsname{}%
          {#1}{#2}\ekvdefNoVal{#3}%
      }%
  }
\protected\long\def\ekvd@add@aux#1#2%
  {%
    \ekvd@extract@prefixes#1%
    \expandafter\ekvd@add@aux@\expandafter{#1#2}%
  }
\protected\long\def\ekvd@add@aux@#1#2#3#4#5%
  {%
    #5%
    \ekvd@prot#4\ekvd@set{#2}{#1#3}%
  }
%    \end{macrocode}
% \end{macro}
%
% \begin{macro}[internal]
%   {
%     \ekvd@extract@prefixes,\ekvd@extract@prefixes@,
%     \ekvd@extract@prefixes@long,\ekvd@extract@prefixes@prot
%   }
% This macro checks which prefixes were used for the definition of a macro and
% sets |\ekvd@long| and |\ekvd@prot| accordingly.
%    \begin{macrocode}
\protected\def\ekvd@extract@prefixes#1%
  {%
    \expandafter\ekvd@extract@prefixes@\meaning#1\ekvd@stop
  }
%    \end{macrocode}
% In the following definition |#1| will get replaced by |macro:|, |#2| by
% |\long| and |#3| by |\protected| (in each, all tokens will have category
% other). This allows us to parse the |\meaning| of a macro for those strings.
%    \begin{macrocode}
\protected\def\ekvd@extract@prefixes@#1#2#3%
  {%
    \protected\def\ekvd@extract@prefixes@##1#1##2\ekvd@stop
      {%
        \ekvd@extract@prefixes@long
          ##1\ekvd@mark\@firstofone#2\ekvd@mark\@gobble\ekvd@stop
          {\let\ekvd@long\long}%
        \ekvd@extract@prefixes@prot
          ##1\ekvd@mark\@firstofone#3\ekvd@mark\@gobble\ekvd@stop
          {\let\ekvd@prot\protected}%
      }%
    \protected\def\ekvd@extract@prefixes@long##1#2##2\ekvd@mark##3##4\ekvd@stop
      {##3}%
    \protected\def\ekvd@extract@prefixes@prot##1#3##2\ekvd@mark##3##4\ekvd@stop
      {##3}%
  }
%    \end{macrocode}
% We use a temporary macro to expand the three arguments of
% |\ekvd@extract@prefixes@|, which will set up the real meaning of itself and
% the parsing for |\long| and |\protected|.
%    \begin{macrocode}
\begingroup
\edef\ekvd@tmp
  {%
    \endgroup
    \ekvd@extract@prefixes@
      {\detokenize{macro:}}%
      {\string\long}%
      {\string\protected}%
  }
\ekvd@tmp
%    \end{macrocode}
% \end{macro}
%
%
% \subsubsection{Tests}
%
% \begin{macro}[internal]{\ekvd@newlet,\ekvd@newreg}
% These macros test whether a control sequence is defined, if it isn't they
% define it, either via |\let| or via the correct \cs[no-index]{new\meta{reg}}.
%    \begin{macrocode}
\protected\def\ekvd@newlet#1#2%
  {%
    \ifdefined#1\ekv@fi@gobble\fi\@firstofone{\let#1#2}%
  }
\protected\def\ekvd@newreg#1#2%
  {%
    \ifdefined#1\ekv@fi@gobble\fi\@firstofone{\csname new#2\endcsname#1}%
  }
%    \end{macrocode}
% \end{macro}
%
% \begin{macro}[internal]
%   {\ekvd@assert@twoargs,\ekvd@ifnottwoargs,\ekvd@ifempty@gtwo}
% A test for exactly two tokens can be reduced for an empty-test after gobbling
% two tokens, in the case that there are fewer tokens than two in the argument,
% only macros will be gobbled that are needed for the true branch, which doesn't
% hurt, and if there are more this will not be empty.
%    \begin{macrocode}
\long\def\ekvd@assert@twoargs#1%
  {%
    \ekvd@ifnottwoargs{#1}{\ekvd@err@missing@definition}%
  }
\long\def\ekvd@ifnottwoargs#1%
  {%
    \ekvd@ifempty@gtwo#1\ekv@ifempty@B
      \ekv@ifempty@false\ekv@ifempty@A\ekv@ifempty@B\@firstoftwo
  }
\long\def\ekvd@ifempty@gtwo#1#2{\ekv@ifempty@\ekv@ifempty@A}
%    \end{macrocode}
% \end{macro}
%
% \begin{macro}[internal]
%   {
%     \ekvd@assert@val,\ekvd@assert@val@,\ekvd@assert@noval,\ekvd@assert@noval@,
%     \ekvd@extract@args,\ekvd@extracted@args,\ekvd@one@arg@string
%   }
% Assert that a given key is defined as a value taking key or a |NoVal| key with
% the correct argument structure, respectively.
%    \begin{macrocode}
\protected\def\ekvd@assert@val#1%
  {%
    \ekvifdefined\ekvd@set{#1}%
      {\expandafter\ekvd@assert@val@\csname\ekv@name\ekvd@set{#1}\endcsname}%
      {%
        \ekvifdefinedNoVal\ekvd@set{#1}%
          \ekvd@err@add@val@on@noval
          {\ekvd@err@undefined@key{#1}}%
        \@gobble
      }%
  }
\protected\def\ekvd@assert@val@#1%
  {%
    \expandafter\ekvd@extract@args\meaning#1\ekvd@stop
    \unless\ifx\ekvd@extracted@args\ekvd@one@arg@string
      \ekvd@err@unsupported@arg
    \fi
    \@firstofone
  }%
\protected\def\ekvd@assert@noval#1%
  {%
    \ekvifdefinedNoVal\ekvd@set{#1}%
      {\expandafter\ekvd@assert@noval@\csname\ekv@name\ekvd@set{#1}N\endcsname}%
      {%
        \ekvifdefined\ekvd@set{#1}%
          \ekvd@err@add@noval@on@val
          {\ekvd@err@undefined@key{#1}}%
        \@gobble
      }%
  }
\protected\def\ekvd@assert@noval@#1%
  {%
    \expandafter\ekvd@extract@args\meaning#1\ekvd@stop
    \unless\ifx\ekvd@extracted@args\ekv@empty
      \ekvd@err@unsupported@arg
    \fi
    \@firstofone
  }
\protected\def\ekvd@extract@args#1%
  {%
    \protected\def\ekvd@extract@args##1#1##2->##3\ekvd@stop
      {\def\ekvd@extracted@args{##2}}%
  }
\expandafter\ekvd@extract@args\expandafter{\detokenize{macro:}}
\edef\ekvd@one@arg@string{\string#1}
%    \end{macrocode}
% \end{macro}
%
% \begin{macro}[internal]{\ekvd@assert@arg,\ekvd@assert@arg@msg,\ekvd@ifnoarg}
% There is no need to actually define |\ekvd@ifnoarg| here, as it will be set by
% either |\ekvd@arg| or |\ekvd@noarg|.
%    \begin{macrocode}
\def\ekvd@assert@arg{\ekvd@ifnoarg\ekvd@err@missing@definition}
\long\def\ekvd@assert@arg@msg#1%
  {%
    \ekvd@ifnoarg{\ekvd@err@missing@definition@msg{#1}}%
  }
%    \end{macrocode}
% \end{macro}
%
% \begin{macro}[internal]{\ekvd@assert@filledarg,\ekvd@ifnoarg@or@empty}
%    \begin{macrocode}
\long\def\ekvd@assert@filledarg#1%
  {%
    \ekvd@ifnoarg@or@empty{#1}\ekvd@err@missing@definition
  }
\long\def\ekvd@ifnoarg@or@empty#1%
  {%
    \ekvd@ifnoarg
      \@firstoftwo
      {\ekv@ifempty{#1}}%
  }
%    \end{macrocode}
% \end{macro}
%
% \begin{macro}[internal]
%   {
%     \ekvd@assert@not@long,\ekvd@assert@not@protected,\ekvd@assert@not@also
%     \ekvd@assert@not@long@also,\ekvd@assert@not@protected@also,
%     \ekvd@assert@new,\ekvd@assert@not@new
%   }
% Some key-types don't want to be |also|, |\long| or |\protected|, so we provide
% macros to test this and throw an error, this could be silently ignored but now
% users will learn to not use unnecessary stuff which slows the compilation
% down.
%    \begin{macrocode}
\def\ekvd@assert@not@long{\ifx\ekvd@long\long\ekvd@err@no@prefix{long}\fi}
\def\ekvd@assert@not@protected
  {\ifx\ekvd@prot\protected\ekvd@err@no@prefix{protected}\fi}
\def\ekvd@assert@not@also{\ekvd@ifalso{\ekvd@err@no@prefix{also}}{}}
\def\ekvd@assert@not@long@also
  {\ifx\ekvd@long\long\ekvd@err@no@prefix@also{long}\fi}
\def\ekvd@assert@not@protected@also
  {\ifx\ekvd@prot\protected\ekvd@err@no@prefix@also{protected}\fi}
\def\ekvd@assert@new#1#2%
  {\csname ekvifdefined#1\endcsname\ekvd@set{#2}{\ekvd@err@not@new}}
\def\ekvd@assert@not@new
  {\ifx\ekvd@ifnew\ekvd@assert@new\ekvd@err@no@prefix{new}\fi}
\def\ekvd@assert@new@for@name#1%
  {%
    \ifx\ekvd@ifnew\ekvd@assert@new
      \ekv@fi@firstoftwo
    \fi
    \@secondoftwo
      {\ekv@ifdefined{#1}\ekvd@err@not@new}%
      \@firstofone
  }
%    \end{macrocode}
% \end{macro}
%
% \begin{macro}[internal]
%   {
%     \ekvd@if@not@already@choice, \ekvd@if@not@already@choice@a,
%     \ekvd@if@not@already@choice@b
%   }
% It is bad to use |also| on a key that already contains a |choice|, as both
% choices would share the same valid values and thus lead to each callback being
% used twice. The following is a rudimentary test against this.
%    \begin{macrocode}
\protected\def\ekvd@if@not@already@choice#1%
  {%
    \expandafter\ekvd@if@not@already@choice@a
      \csname\ekv@name\ekvd@set{#1}\endcsname
      {}\ekvd@h@choice\ekvd@stop
  }
\protected\def\ekvd@if@not@already@choice@a
  {%
    \expandafter\ekvd@if@not@already@choice@b
  }
\long\protected\def\ekvd@if@not@already@choice@b#1\ekvd@h@choice#2\ekvd@stop
  {%
    \ekv@ifempty{#2}\@firstofone\@gobble
  }
%    \end{macrocode}
% \end{macro}
%
% \begin{macro}[internal]{\ekvd@ifspace,\ekvd@ifspace@}
% Yet another test which can be reduced to an if-empty, this time by gobbling
% everything up to the first space.
%    \begin{macrocode}
\long\def\ekvd@ifspace#1%
  {%
    \ekvd@ifspace@#1 \ekv@ifempty@B
      \ekv@ifempty@false\ekv@ifempty@A\ekv@ifempty@B\@firstoftwo
  }
\long\def\ekvd@ifspace@#1 % keep this space
  {%
    \ekv@ifempty@\ekv@ifempty@A
  }
%    \end{macrocode}
% \end{macro}
%
% \subsubsection{Messages}
%
% Most messages of \expkvd\ are not expandable, since they only appear during
% key-definition, which is not expandable anyway.
%
% \begin{macro}[internal]
%   {
%     \ekvd@errm,\ekvd@err@missing@definition,\ekvd@err@missing@definition@msg,
%     \ekvd@err@missing@type,\ekvd@err@undefined@prefix,\ekvd@err@undefined@key,
%     \ekvd@err@no@prefix,\ekvd@err@no@prefix@msg,\ekvd@err@no@prefix@also,
%     \ekvd@err@add@val@on@noval,\ekvd@err@add@noval@on@val,
%     \ekvd@err@unsupported@arg,\ekvd@err@not@new
%   }
% The non-expandable error messages are boring, so here they are:
%    \begin{macrocode}
\protected\def\ekvd@errm#1{\errmessage{expkv-def Error: #1}}
\protected\def\ekvd@err@missing@definition
  {\ekvd@errm{Missing definition for key `\ekvd@cur'}}
\protected\def\ekvd@err@missing@definition@msg#1%
  {\ekvd@errm{Missing definition for key `\ekv@unexpanded{#1}'}}
\protected\def\ekvd@err@missing@type
  {\ekvd@errm{Missing type prefix for key `\ekvd@cur'}}
\protected\def\ekvd@err@undefined@prefix#1%
  {%
    \ekvd@errm
      {%
        Undefined prefix `\ekv@unexpanded{#1}' found while processing
        `\ekvd@cur'%
      }%
  }
\protected\def\ekvd@err@undefined@key#1%
  {%
    \ekvd@errm
      {Undefined key `\ekv@unexpanded{#1}' found while processing `\ekvd@cur'}%
  }
\protected\def\ekvd@err@undefined@noval#1%
  {%
    \ekvd@errm
      {%
        Undefined noval key `\unexpanded{#1}' found while processing
        `\ekvd@cur'%
      }%
  }
\protected\def\ekvd@err@no@prefix#1%
  {\ekvd@errm{prefix `#1' not accepted in `\ekvd@cur'}}
\protected\def\ekvd@err@no@prefix@msg#1#2%
  {\ekvd@errm{prefix `#2' not accepted in `\ekv@unexpanded{#1}'}}
\protected\def\ekvd@err@no@prefix@also#1%
  {\ekvd@errm{`\ekvd@cur' not allowed with a `#1' key}}
\protected\def\ekvd@err@add@val@on@noval
  {\ekvd@errm{`\ekvd@cur' not allowed with a NoVal key}}
\protected\def\ekvd@err@add@noval@on@val
  {\ekvd@errm{`\ekvd@cur' not allowed with a value taking key}}
\protected\def\ekvd@err@unsupported@arg\fi\@firstofone#1%
  {%
    \fi
    \ekvd@errm
      {%
        Existing key-macro has the unsupported argument string
        `\ekvd@extracted@args' for key `\ekvd@cur'%
      }%
  }
\protected\def\ekvd@err@not@new
  {\ekvd@errm{The key for `\ekvd@cur' is already defined}}
\protected\long\def\ekvd@err@misused@unknown
  {\ekvd@errm{Misuse of the unknown type found while processing `\ekvd@cur'}}
%    \end{macrocode}
% \end{macro}
%
% \begin{macro}[internal]
%   {
%     \ekvd@err@choice@invalid,\ekvd@err@choice@invalid@,\ekvd@choice@name,
%     \ekvd@unknown@choice@name
%   }
% |\ekvd@err@choice@invalid| will have to use this mechanism to throw its
% message. Also we have to retrieve the name parts of the choice in an easy way,
% so we use parentheses of catcode 8 here, which should suffice in most cases to
% allow for a correct separation.
%    \begin{macrocode}
\def\ekvd@err@choice@invalid#1%
  {%
    \ekvd@err@choice@invalid@#1%
  }
\begingroup
\catcode40=8
\catcode41=8
\@firstofone{\endgroup
\def\ekvd@choice@name#1#2#3%
  {%
    ekvd#1(#2)\detokenize{#3}%
  }
\def\ekvd@unknown@choice@name#1#2%
  {%
    ekvd:u:#1(#2)%
  }
\def\ekvd@err@choice@invalid@ ekvd#1(#2)\detokenize#3%
  {%
    \ekv@ifdefined{\ekvd@unknown@choice@name{#1}{#2}}%
      {\csname\ekvd@unknown@choice@name{#1}{#2}\endcsname{#3}}%
      {\ekvd@err{invalid choice `#3' for `#2' in set `#1'}}%
  }
}
%    \end{macrocode}
% \end{macro}
%
% \begin{macro}[internal]{\ekvd@err}
% The expandable error messages use |\ekvd@err|, which is just like |\ekv@err|
% from \expkv. It uses a runaway argument to start the error message.
%    \begin{macrocode}
\ekv@exparg{\long\def\ekvd@err#1}{\ekverr{expkv-def}{#1}}
%    \end{macrocode}
% \end{macro}
%
% Now everything that's left is to reset the category code of |@|.
%    \begin{macrocode}
\catcode`\@=\ekvd@tmp
%    \end{macrocode}
%
% \gobbledocstriptag
%</tex>
%^^A=<<
%
%^^A=<<
%
% \end{implementation}^^A=<<
%
% \clearpage
% \PrintIndex
%
\endinput
%
^^A vim: ft=tex fdm=marker fmr=>>=,=<<
}{}
\ProvidesFile{expkv-def.dtx}
  [\csname ekvdDate\endcsname\ a key-defining frontend for expkv]
\PassOptionsToPackage{full}{textcomp}
\documentclass{l3doc}
\RequirePackage[oldstylenums,nott]{kpfonts}
\input{glyphtounicode}
\pdfgentounicode=1
\RequirePackage{tcolorbox}
\newtcolorbox{exresult}[2][]
  {colback=expkvgrey!10!white,colframe=expkvgrey,fontupper=\small,width={#2},#1}
\newtcbox\exres[1][]
  {
     colback=expkvgrey!10!white
    ,colframe=expkvgrey
    ,size=small
    ,nobeforeafter
    ,tcbox raise base
    ,fontupper=\small
    ,#1
  }
\RequirePackage{listings}
\RequirePackage{booktabs}
\RequirePackage{array}
\RequirePackage{collcell}
\RequirePackage{siunitx}
\RequirePackage{xcolor}
\RequirePackage{caption}
\RequirePackage{microtype}
\RequirePackage{accsupp}
\lstdefinelanguage{expkvd}[primitive]{TeX}
  {
    ,moretexcs=[4]^^A e-TeX
      {
        expanded,
        numexpr,
        protected,
      }
    ,moretexcs=[5]^^A plain/LaTeX
      {
        approx,
        begin,
        empty,
        item,
        LaTeX,
        makeatletter,makeatother,
        newcommand,newdimen,
        RequirePackage,
        rule,
        TeX,
        textit,texttt,
        usepackage,
      }
    ,moretexcs=[6]^^A used packages
      {
        ^^A xfp
        fpeval,
        ^^A keyval
        setkeys,
        ^^A yax
        defactiveparameter,storevalue,setparameterlist
      }
    ,moretexcs=[2]^^A expkv-def macros
      {
        ekvdefinekeys
      }
    ,moretexcs=[3]^^A expkv-pkg macros
      {
        ^^A expkv
        ekvdef,ekvdefNoVal,^^A
        ekvlet,ekvletNoVal,ekvletkv,ekvletkvNoVal,^^A
        ekvdefunknown,ekvdefunknownNoVal,^^A
        ekvredirectunknown,ekvredirectunknownNoVal,^^A
        ekvset,ekvsetSneaked,^^A
        ekvsetdef,ekvsetSneakeddef,ekvsetdefSneaked,^^A
        ekvparse,^^A
        ekvVersion,ekvDate,^^A
        ekvifdefined,ekvifdefinedNoVal,ekvifdefinedset,^^A
        ekvbreak,ekvbreakPreSneak,ekvbreakPostSneak,^^A
        ekvsneak,ekvsneakPre,^^A
        ekvchangeset,^^A
        ekvoptarg,ekvoptargTF,^^A
        ekverr,
        ^^A expkv-cs
        ekvcSecondaryKeys,ekvcSplitAndForward,
      }
  }
\colorlet{codeparam}{cyan!65!black}
\def\mylstwd{.55em}
\lstset
  {
    ,language=expkvd
    ,flexiblecolumns=false
    ,basewidth=\mylstwd
    ,gobble=2
    ,basicstyle=\fontfamily{jkp}\itshape
    ,texcsstyle=*[1]{\bfseries\color{expkvgrey}} ^^A primitives
    ,texcsstyle=*[2]{\bfseries\color{expkvred}}  ^^A expkv
    ,texcsstyle=*[3]{\color{expkvred}}           ^^A expkv-pkg
    ,texcsstyle=*[4]{\bfseries\color{expkvgrey}} ^^A e-TeX
    ,texcsstyle=*[5]{\bfseries\color{expkvgrey}} ^^A plain/LaTeX macros
    ,texcsstyle=*[6]{}                           ^^A macros of other packages
    ,commentstyle=\color[gray]{0.4}
    ,literate=
              {\{} {{\CodeUpBf\{}}{1}
              {\}} {{\CodeUpBf\}}}{1}
              {$}  {{\CodeUpBf\$}}{1}
              {[}  {{\CodeUp[}}{1}
              {]}  {{\CodeUp]}}{1}
              {(}  {{\CodeUp(}}{1}
              {)}  {{\CodeUp)}}{1}
              {*}  {{$*$}}{1}
              {1}  {{\CodeUp{1}}}{1}
              {2}  {{\CodeUp{2}}}{1}
              {3}  {{\CodeUp{3}}}{1}
              {4}  {{\CodeUp{4}}}{1}
              {5}  {{\CodeUp{5}}}{1}
              {6}  {{\CodeUp{6}}}{1}
              {7}  {{\CodeUp{7}}}{1}
              {8}  {{\CodeUp{8}}}{1}
              {9}  {{\CodeUp{9}}}{1}
              {0}  {{\CodeUp{0}}}{1}
              {##} {{\CodeColored{codeparam}{1}{\#}}}{1}
              {##1}{{\CodeColored{codeparam}{2}{\#1}}}{2}
              {##2}{{\CodeColored{codeparam}{2}{\#2}}}{2}
              {##3}{{\CodeColored{codeparam}{2}{\#3}}}{2}
              {##4}{{\CodeColored{codeparam}{2}{\#4}}}{2}
              {##5}{{\CodeColored{codeparam}{2}{\#5}}}{2}
              {##6}{{\CodeColored{codeparam}{2}{\#6}}}{2}
              {##7}{{\CodeColored{codeparam}{2}{\#7}}}{2}
              {##8}{{\CodeColored{codeparam}{2}{\#8}}}{2}
              {##9}{{\CodeColored{codeparam}{2}{\#9}}}{2}
              {<key>}{{$\langle$}key{$\rangle$}}{5}
              {<set>}{{$\langle$}set{$\rangle$}}{5}
              {<set1>}{{$\langle$}set1{$\rangle$}}{6}
              {<set2>}{{$\langle$}set2{$\rangle$}}{6}
    ^^A,literate=*{<key>}{\key}{4}{<set>}{\set}{4}
  }
\newcommand*\CodeColored[3]{\textcolor{#1}{\makebox[\dimexpr\mylstwd*#2]{#3}}}
\newcommand*\CodeUpBf[1]{\makebox[\mylstwd]{\textup{\textbf{#1}}}}
\newcommand*\CodeUp[1]{\makebox[\mylstwd]{\textup{#1}}}
\RequirePackage{randtext}
\let\metaORIG\meta
\protected\def\meta #1{\texttt{\metaORIG{#1}}}
\renewcommand*\thefootnote{\fnsymbol{footnote}}
\definecolor{expkvred}{HTML}{9F393D}
\colorlet{expkvgrey}{black!75}
\makeatletter
\newcommand*\expkv
  {^^A
    \texorpdfstring
      {^^A
        \mbox
          {^^A
            \BeginAccSupp{ActualText=expkv}^^A
            \href{https://github.com/Skillmon/tex_expkv}
              {^^A
                \rmfamily
                \bfseries
                {\color{expkvgrey}e\kern-.05em x\kern-.05em}^^A
                \lower.493ex
                  \hbox{{\color{expkvgrey}P}\kern-.1em{\color{expkvred}k}}^^A
                \kern-.18em{\color{expkvred}v}^^A
              }^^A
            \EndAccSupp{}^^A
          }^^A
      }
      {expkv}^^A
  }
\newcommand*\expkvd
  {^^A
    \texorpdfstring
      {^^A
        \mbox
          {^^A
            \BeginAccSupp{ActualText=expkv-def}^^A
            \href{https://github.com/Skillmon/tex_expkv-def}
              {^^A
                \rmfamily
                \bfseries
                {\color{expkvgrey}e\kern-.05em x\kern-.05em}^^A
                \lower.493ex
                  \hbox{{\color{expkvgrey}P}\kern-.1em{\color{expkvred}k}}^^A
                \kern-.18em{\color{expkvred}v}^^A
                {\color{expkvgrey}^^A
                  \kern.05em\rule[-.1ex]{.08em}{1.2ex}\kern.05em\textsc{def}^^A
                }^^A
              }^^A
            \EndAccSupp{}^^A
          }^^A
      }
      {expkv-def}^^A
  }
\newcommand\kv{\meta{key}=\meta{value}}
\newcommand\key{\meta{key}}
\newcommand\val{\meta{value}}
\newcommand\set{\meta{set}}
\newcommand\enfprefix{\textcolor{black}}
\newcommand\allprefix{\textcolor{gray}}
\newcommand\notprefix{\textcolor{red!80!black}}
\newcommand\printprefix[2]
  {^^A
    \ifcase\numexpr#1\relax\or
      \enfprefix{#2}\or
      \allprefix{#2}\or
      \notprefix{#2}\fi
  }
\newcommand\prefixes[4]
  {^^A
    \hfill
    \printprefix{#1}{new}
    \printprefix{#2}{also}
    \printprefix{#3}{protected}
    \printprefix{#4}{long}^^A
  }
\hypersetup{linkcolor=red!80!black,urlcolor=purple!80!black}
\DoNotIndex{\def,\edef,\,,\=,\begingroup,\catcode,\chardef,\csname,\endcsname}
\DoNotIndex{\endgroup,\endinput,\errmessage,\expandafter,\input,\let,\long}
\DoNotIndex{\protected,\ProvidesFile,\ProvidesPackage,\relax,\space}
\DoNotIndex{\@,\unexpanded,\string,\expanded,\dimexpr,\global,\glueexpr,\hbox}
\DoNotIndex{\numexpr,\RequirePackage,\setbox,\the,\unless,\xdef,\@firstofone}
\DoNotIndex{\@firstoftwo,\@gobble,\@secondoftwo,\meaning,\detokenize}
\DoNotIndex{\romannumeral,\^}
\DoNotIndex{\ifcsname}
\DoNotIndex{\ifx}
\DoNotIndex{\ifdefined}
\DoNotIndex{\iffalse}
\DoNotIndex{\iftrue}
\DoNotIndex{\else}
\DoNotIndex{\fi}
\@gobble\fi ^^A ignoring \ifx, \ifcsname, etc., but only one \fi
\@gobble\fi ^^A ignoring \ifx, \ifcsname, etc., but only one \fi
\@gobble\fi ^^A ignoring \ifx, \ifcsname, etc., but only one \fi
\@gobble\fi ^^A ignoring \ifx, \ifcsname, etc., but only one \fi
\@ifdefinable\gobbledocstriptag{\def\gobbledocstriptag#1>{}}
\makeatother
\begin{document}
  \title
    {^^A
      \texorpdfstring
        {^^A
          \huge\expkvd
          \\[\medskipamount]
          \Large a key-defining frontend for \expkv
        }{expkv-def - a key-defining frontend for expkv}^^A
    }
  \date{\ekvdDate\space v\ekvdVersion}
  \author{Jonathan P. Spratte\thanks{\protect\randomize{jspratte@yahoo.de}}}
  \DocInput{expkv-def.dtx}
\end{document}
%</driver>^^A=<<
% \fi
%
% \maketitle
% \renewcommand*\thefootnote{\arabic{footnote}}
%
% \begin{abstract}
% \noindent\parfillskip=0pt
% \expkvd\ provides a small \kv\ interface to define keys for \expkv. Key-types
% are declared using prefixes, similar to static typed languages. The stylised
% name is \expkvd\ but the files use \file{expkv-def}, this is due to CTAN-rules
% which don't allow \string| in package names since that is the pipe symbol in
% *nix shells.
% \end{abstract}
%
% \tableofcontents
%
% \begin{documentation}^^A>>=
%
% \section{Documentation}
%
% Since the trend for the last couple of years goes to defining keys for a \kv\
% interface using a \kv\ interface, I thought that maybe providing such an
% interface for \expkv\ will make it more attractive for actual use, besides its
% unique selling points of being fully expandable, and fast and reliable. But at
% the same time I don't want to widen \expkv's initial scope. So here it is
% \expkvd, go define \kv\ interfaces with \kv\ interfaces.
%
% Unlike many of the other established \kv\ interfaces to define keys, \expkvd\
% works using prefixes instead of suffixes (\emph{e.g.}, |.tl_set:N| of
% \pkg{l3keys}) or directory like handlers (\emph{e.g.}, |/.store in| of
% \pkg{pgfkeys}). This was decided as a personal preference, more over in \TeX\
% parsing for the first space is way easier than parsing for the last one.
% \expkvd's prefixes are sorted into two categories: |p|-type, which are
% equivalent to \TeX's prefixes like |\long|, and |t|-type defining the type of
% the key. For a description of the available |p|-prefixes take a look at
% \autoref{sec:prefix:p}, the |t|-prefixes are described in
% \autoref{sec:prefix:t}.
%
% \expkvd\ is usable as generic code, as a \LaTeX\ package, and as a \ConTeXt\
% module. It'll automatically load \expkv\ in the same mode as well. To use it,
% just use one of
% \begin{lstlisting}
% \input expkv-def       % plainTeX
% \usepackage{expkv-def} % LaTeX
% \usemodule[expkv-def]  % ConTeXt
% \end{lstlisting}
%
% \subsection{Macros}
% Apart from version and date containers there is only a single user-facing
% macro, and that should be used to define keys.
%
% \begin{function}{\ekvdefinekeys}
%   \begin{syntax}
%     \cs{ekvdefinekeys}\marg{set}\{\kv, \ldots\}
%   \end{syntax}
%   In \set, define \key\ to have definition \val. The general syntax for \key\
%   should be
%   \begin{quote}\ttfamily\small
%     \meta{prefix} \meta{name}
%   \end{quote}
%   Where \meta{prefix} is a space separated list of optional |p|-type prefixes
%   followed by one |t|-type prefix. The syntax of \val\ is dependent on the
%   used |t|-prefix.
% \end{function}
%
% \begin{function}{\ekvdDate,\ekvdVersion}
%   These two macros store the version and date of the package.
% \end{function}
%
% \subsection{Prefixes}
% As already said there are |p|-prefixes and |t|-prefixes. Not every |p|-prefix
% is allowed for all |t|-prefixes.
%
% \subsubsection{\texttt{p}-Prefixes}\label{sec:prefix:p}
% The two |p|-type prefixes |long| and |protected| are pretty simple by nature,
% so their description is pretty simple. They affect the \key\ at use-time, so
% omitting |long| doesn't mean that a \meta{definition} can't contain a |\par|
% token, only that the \key\ will not accept a |\par| in \val.  On the other
% hand |new| and |also| might be simple on first sight as well, but their rules
% are a bit more complicated.
%
% \begin{function}{also}
%   The following key type will be \emph{added} to an existing \key's
%   definition. You can't add a type taking an argument at use time to an
%   existing key which doesn't take an argument and vice versa. Also you'll get
%   an error if you try to add an action which isn't allowed to be either |long|
%   or |protected| to a key which already is |long| or |protected| (the opposite
%   order would be suboptimal as well, but can't be really captured with the
%   current code).
%
%   A key already defined as |long| or |protected| will stay |long| or
%   |protected|, but you can as well add |long| or |protected| with the |also|
%   definition.
%
%   As a small example, suppose you want to create a boolean key, but
%   additionally to setting a boolean value you want to execute some more code
%   as well, you can use the following
% \begin{lstlisting}
% \ekvdefinekeys{also-example}
%   {
%      bool key      = \ifmybool
%     ,also code key = \domystuff{#1}
%   }
% \end{lstlisting}
%
%   If you use |also| on a |choice|, |bool|, |invbool|, or |boolpair| key it is
%   tried to determine if the key already is of one of those types. If this test
%   is true the declared choices will be added to the possible choices but the
%   key's definition will not be changed other than that. If that wouldn't have
%   been done, the callbacks of the different choices could get called multiple
%   times.
% \end{function}
%
% \begin{function}{protected,protect}
%   The following key will be defined |\protected|. Note that key-types which
%   can't be defined expandable will always use |\protected|.
% \end{function}
%
% \begin{function}{long}
%   The following key will be defined |\long|.
% \end{function}
%
% \begin{function}{new}
%   The following key must be new (so previously undefined). An error is thrown
%   if it is already defined and the new definition is ignored. |new| only
%   asserts that there are no conflicts between |NoVal| keys and other |NoVal|
%   keys or value taking keys and other value taking keys. For example you can
%   use the following without an error:
% \begin{lstlisting}
% \ekvdefinekeys{new-example}
%   {
%      code key       = \domystuffwitharg{#1}
%     ,new  noval key = \domystuffwithoutarg
%   }
% \end{lstlisting}
% \end{function}
%
% \subsubsection{\texttt{t}-Prefixes}\label{sec:prefix:t}
% Since the |p|-type prefixes apply to some of the |t|-prefixes automatically
% but sometimes one might be disallowed we need some way to highlight this
% behaviour. In the following an enforced prefix will be printed black
% (\texttt{\enfprefix{protected}}), allowed prefixes will be grey
% (\texttt{\allprefix{protected}}), and disallowed prefixes will be red
% (\texttt{\notprefix{protected}}). This will be put flush-right in the syntax
% showing line.
%
% \begin{function}{code,ecode}
%   \begin{syntax}
%     code \key\ = \marg{definition} \prefixes2222
%   \end{syntax}
%   Define \key\ to expand to \meta{definition}. The \key\ will require a \val\
%   for which you can use |#1| inside \meta{definition}. The |ecode| variant
%   will fully expand \meta{definition} inside an |\edef|.
% \end{function}
%
% \begin{function}{noval,enoval}
%   \begin{syntax}
%     noval \key\ = \marg{definition} \prefixes2223
%   \end{syntax}
%   The |noval| type defines \key\ to expand to \meta{definition}. The \key\
%   will not take a \val. |enoval| fully expands \meta{definition} inside an
%   |\edef|.
% \end{function}
%
% \begin{function}{default,qdefault,odefault,fdefault,edefault}
%   \begin{syntax}
%     default \key\ = \marg{definition} \prefixes3223
%   \end{syntax}
%   This serves to place a default \val\ for a \key\ that takes an argument, the
%   \key\ can be of any argument-grabbing kind, and when used without a \val\
%   it will be passed \meta{definition} instead. The |qdefault| variant will
%   expand the \key's code once, so will be slightly quicker, but not change if
%   you redefine \key. |odefault| is just another name for |qdefault|. The
%   |fdefault| version will expand the key code until a non-expandable token or
%   a space is found, a space would be gobbled.\footnotemark{} The |edefault| on
%   the other hand fully expands the \key-code with \meta{definition} as its
%   argument inside of an |\edef|.
% \end{function}
% \footnotetext
%   {^^A
%     For those familiar with \TeX-coding: This uses a
%     \cs[no-index]{romannumeral}-expansion.
%   }
%
% \begin{function}{initial,oinitial,finitial,einitial}
%   \begin{syntax}
%     initial \key\ = \{\val\} \prefixes3333
%     initial \key
%   \end{syntax}
%   With |initial| you can set an initial \val\ for an already defined argument
%   taking \key. It'll just call the key-macro of \key\ and pass it \val. The
%   |einitial| variant will expand \val\ using an |\edef| expansion prior to
%   passing it to the key-macro and the |oinitial| variant will expand the first
%   token in \val\ once. |finitial| will expand \val\ until a non-expandable
%   token or a space is found, a space would be
%   gobbled.\footnotemark
%
%   If you don't provide a value (and no equals sign) a |noval| \key\ of the
%   same name is called once (or, if you specified a |default| for a value
%   taking key that would be used).
% \end{function}
% \footnotetext{Again using \cs[no-index]{romannumeral}.}
%
% \begin{function}{bool,gbool,boolTF,gboolTF}
%   \begin{syntax}
%     bool \key\ = \meta{cs} \prefixes2223
%   \end{syntax}
%   The \meta{cs} should be a single control sequence, such as |\iffoo|.
%   This will define \key\ to be a boolean key, which only takes the values
%   |true| or |false| and will throw an error for other values. If the key is
%   used without a \val\ it'll have the same effect as if you use
%   \texttt{\key=true}. |bool| and |gbool| will behave like \TeX-ifs so either
%   be |\iftrue| or |\iffalse|. The |boolTF| and |gboolTF| variants will both
%   take two arguments and if true the first will be used else the second, so
%   they are always either |\@firstoftwo| or |\@secondoftwo|. The variants with
%   a leading |g| will set the control sequence globally, the others locally. If
%   \meta{cs} is not yet defined it'll be initialised as the |false| version.
%   Note that the initialisation is \emph{not} done with |\newif|, so you will
%   not be able to do |\footrue| outside of the \kv\ interface, but you could
%   use |\newif| yourself. Even if the \key\ will not be |\protected| the
%   commands which execute the |true| or |false| choice will be, so the usage
%   should be safe in an expansion context (\emph{e.g.}, you can use
%   \texttt{edefault \key\ = false} without an issue to change the default
%   behaviour to execute the |false| choice). Internally a |bool| \key\ is the
%   same as a |choice| key which is set up to handle |true| and |false| as
%   choices.
% \end{function}
%
% \begin{function}{invbool,ginvbool,invboolTF,ginvboolTF}
%   \begin{syntax}
%     bool \key\ = \meta{cs} \prefixes2223
%   \end{syntax}
%   These are inverse boolean keys, they behave like |bool| and friends but set
%   the opposite meaning to the macro \meta{cs} in each case. So if
%   \texttt{key=true} is used |invbool| will set \meta{cs} to |\iffalse| and
%   vice versa.
% \end{function}
%
% \begin{function}{boolpair,gboolpair,boolpairTF,gboolpairTF}
%   \begin{syntax}
%     boolpair \key\ = \meta{cs_1}\meta{cs_2} \prefixes2223
%   \end{syntax}
%   The |boolpair| key type behaves like both |bool| and |invbool|, the
%   \meta{cs_1} will be set to the meaning according to the rules of |bool|, and
%   \meta{cs_2} will be set to the opposite.
% \end{function}
%
% \begin{function}{store,estore,gstore,xstore}
%   \begin{syntax}
%     store \key\ = \meta{cs} \prefixes2212
%   \end{syntax}
%   The \meta{cs} should be a single control sequence, such as |\foo|.
%   This will define \key\ to store \val\ inside of the control sequence. If
%   \meta{cs} isn't yet defined it will be initialised as empty. The variants
%   behave similarly to their |\def|, |\edef|, |\gdef|, and |\xdef|
%   counterparts, but |store| and |gstore| will allow you to store macro
%   parameters inside of them by using |\unexpanded|.
% \end{function}
%
% \begin{function}{data,edata,gdata,xdata}
%   \begin{syntax}
%     data \key\ = \meta{cs} \prefixes2212
%   \end{syntax}
%   The \meta{cs} should be a single control sequence, such as |\foo|. This will
%   define \key\ to store \val\ inside of the control sequence. But unlike the
%   |store| type, the macro \meta{cs} will be a switch at the same time, it'll
%   take two arguments and if \key\ was used expands to the first argument
%   followed by \val\ in braces, if \key\ was not used \meta{cs} will expand to
%   the second argument (so behave like |\@secondoftwo|). The idea is that with
%   this type you can define a key which should be typeset formatted. The
%   |edata| and |xdata| variants will fully expand \val, the |gdata| and |xdata|
%   variants will store \val\ inside \meta{cs} globally. The |p|-prefixes will
%   only affect the key-macro, \meta{cs} will always be expandable and |\long|.
% \end{function}
%
% \begin{function}{dataT,edataT,gdataT,xdataT}
%   \begin{syntax}
%     dataT \key\ = \meta{cs} \prefixes2212
%   \end{syntax}
%   Just like |data|, but instead of \meta{cs} grabbing two arguments it'll only
%   grab one, so by default it'll behave like |\@gobble|, and if a \val\ was
%   given to \key\ the \meta{cs} will behave like |\@firstofone| appended by
%   \marg{value}.
% \end{function}
%
% \begin{function}{int,eint,gint,xint}
%   \begin{syntax}
%     int \key\ = \meta{cs} \prefixes2212
%   \end{syntax}
%   The \meta{cs} should be a single control sequence, such as |\foo|.
%   An |int| key will be a \TeX-count register. If \meta{cs} isn't defined yet,
%   |\newcount| will be used to initialise it. The |eint| and |xint| versions
%   will use |\numexpr| to allow basic computations in their \val. The |gint|
%   and |xint| variants set the register globally.
% \end{function}
%
% \begin{function}{dimen,edimen,gdimen,xdimen}
%   \begin{syntax}
%     dimen \key\ = \meta{cs} \prefixes2212
%   \end{syntax}
%   The \meta{cs} should be a single control sequence, such as |\foo|.
%   This is just like |int| but uses a dimen register, |\newdimen| and
%   |\dimexpr| instead.
% \end{function}
%
% \begin{function}{skip,eskip,gskip,xskip}
%   \begin{syntax}
%     skip \key\ = \meta{cs} \prefixes2212
%   \end{syntax}
%   The \meta{cs} should be a single control sequence, such as |\foo|.
%   This is just like |int| but uses a skip register, |\newskip| and |\glueexpr|
%   instead.
% \end{function}
%
% \begin{function}{toks,gtoks,apptoks,gapptoks,pretoks,gpretoks}
%   \begin{syntax}
%     toks \key\ = \meta{cs} \prefixes2212
%   \end{syntax}
%   The \meta{cs} should be a single control sequence, such as |\foo|.
%   Store \val\ inside of a toks-register. The |g| variants use |\global|, the
%   |app| variants append \val\ to the contents of that register, the |pre|
%   variants will prepend \val. If \meta{cs} is not yet defined it will be
%   initialised with |\newtoks|.
% \end{function}
%
% \begin{function}{box,gbox}
%   \begin{syntax}
%     box \key\ = \meta{cs} \prefixes2212
%   \end{syntax}
%   The \meta{cs} should be a single control sequence, such as |\foo|.
%   Typesets \val\ into a |\hbox| and stores the result in a box register. The
%   boxes are colour safe. \expkvd\ doesn't provide a |vbox| type.
% \end{function}
%
% \begin{function}{meta}
%   \begin{syntax}
%     meta \key\ = \{\kv, \ldots\} \prefixes2222
%   \end{syntax}
%   This key type can set other keys, you can access the \val\ which was passed
%   to \key\ inside the \kv\ list with |#1|. It works by calling a sub-|\ekvset|
%   on the \kv\ list, so a |set| key will only affect that \kv\ list and not the
%   current |\ekvset|. Since it runs in a separate |\ekvset| you can't use
%   |\ekvsneak| using keys or similar macros in the way you normally could.
% \end{function}
%
% \begin{function}{nmeta}
%   \begin{syntax}
%     nmeta \key\ = \{\kv, \ldots\} \prefixes2223
%   \end{syntax}
%   This key type can set other keys, the difference to |meta| is, that this key
%   doesn't take a value, so the \kv\ list is static.
% \end{function}
%
% \begin{function}{smeta}
%   \begin{syntax}
%     smeta \key\ = \marg{set}\{\kv, \ldots\} \prefixes2222
%   \end{syntax}
%   Yet another |meta| variant. An |smeta| key will take a \val\ which you can
%   access using |#1|, but it sets the \kv\ list inside of \set, so is equal to
%   \texttt{\cs[no-index]{ekvset}\marg{set}\{\kv, \ldots\}}.
% \end{function}
%
% \begin{function}{snmeta}
%   \begin{syntax}
%     snmeta \key\ = \marg{set}\{\kv, \ldots\} \prefixes2223
%   \end{syntax}
%   And the last |meta| variant. |snmeta| is a combination of |smeta| and
%   |nmeta|. It doesn't take an argument and sets the \kv\ list inside of \set.
% \end{function}
%
% \begin{function}{set}
%   \begin{syntax}
%     set \key\ = \marg{set} \prefixes2233
%   \end{syntax}
%   This will define \key\ to change the set of the current |\ekvset|
%   invocation to \set. You can omit \set\ (including the equals sign),
%   which is the same as using \texttt{set \key\ = \{\key\}}. The created |set|
%   key will not take a \val. Note that just like in \expkv\ it'll not be
%   checked whether \set\ is defined and you'll get a low-level \TeX\ error if
%   you use an undefined \set.
% \end{function}
%
% \begin{function}{choice}
%   \begin{syntax}
%     choice \key\ = \{\val=\meta{definition}, \ldots\} \prefixes2223
%   \end{syntax}
%   Defines \key\ to be a |choice| key, meaning it will only accept a limited
%   set of values. You should define each possible \val\ inside of the
%   \val=\meta{definition} list. If a defined \val\ is passed to \key\ the
%   \meta{definition} will be left in the input stream. You can make
%   individual values |protected| inside the \val=\meta{definition} list. By
%   default a |choice| key is expandable, an undefined \val\ will throw an error
%   in an expandable way (but see the |unknown-choice| prefix). You can add
%   additional choices after the \key\ was created by using |choice| again for
%   the same \key, redefining choices is possible the same way, but there is no
%   interface to remove certain choices.
% \end{function}
%
% \begin{function}{choice-store}
%   \begin{syntax}
%     choice-store \key\ = \meta{cs}\{\val, \ldots\} \prefixes2223
%   \end{syntax}
%   This is defines a special type of |choice| key that'll store the given
%   choice inside the macro \meta{cs} (so \meta{cs} should be a single control
%   sequence name such as |\foo|). Since storing inside a macro can't be done
%   expandably every choice-code is |\protected|, you might define the
%   |choice-store| key itself as |protected| as well if you want. Since the
%   definition of each choice is predefined with this key type the choice list
%   should just be a comma separated list of valid choices.
% \end{function}
% This means that the following |choice| and |choice-store| keys are equivalent
% at use time:
% \begin{lstlisting}
% \newcommand*\mya{}
% \ekvdefinekeys{example}
%   {
%      choice key1 = {a=\def\mya{a}, b=\def\mya{b}, c=\def\mya{c}}
%     ,choice-store key2 = \mya{a,b,c}
%   }
% \end{lstlisting}
%
% \begin{function}{choice-enum}
%   \begin{syntax}
%     choice-enum \key\ = \meta{cs}\{\val, \ldots\} \prefixes2223
%   \end{syntax}
%   This is similar to |choice-store|, the differences are: \meta{cs} should be
%   a count-register or is initialised as such if the \meta{cs} is undefined
%   (via |\newcount|); instead of the value the position of the value in the
%   given list is stored in this register (zero-based).
% \end{function}
% This means that the following |choice| and |choice-enum| keys are equivalent
% at use time:
% \begin{lstlisting}
% \newcount\myb
% \ekvdefinekeys{example}
%   {
%      choice key1 = {a={\myb=0 }, b={\myb=1 }, c={\myb=2 }}
%     ,choice-enum key2 = \myb{a,b,c}
%   }
% \end{lstlisting}
%
% \begin{function}{unknown-choice}
%   \begin{syntax}
%     unknown-choice \key\ = \marg{definition} \prefixes2323
%   \end{syntax}
%   By default an unknown \val\ passed to a |choice| or |bool| key will throw an
%   error.  However, with this prefix you can define an alternative action which
%   should be executed if \key\ received an unknown choice. In \meta{definition}
%   you can refer to the choice which was passed in with |#1|.
% \end{function}
%
% \begin{function}{unknown code}
%   \begin{syntax}
%     unknown code = \marg{definition} \prefixes2322
%   \end{syntax}
%   By default \expkv\ throws errors when it encounters unknown keys in a set.
%   With the |unknown| prefix you can define handlers that deal with undefined
%   keys, instead of a \key\ name you have to specify a subtype for this prefix,
%   here the subtype is |code|.\par
%   With |unknown code| the \meta{definition} is used for unknown keys which
%   were provided a value (so corresponds to |\ekvdefunknown|), you can access
%   the key name with |#1| and the value with |#2|.\footnotemark
% \end{function}
% \footnotetext
%   {^^A
%     There is some trickery involved to get this more intuitive argument order
%     without any performance hit if you compare this to
%     \cs[no-index]{ekvdefunknown} directly.^^A
%   }
%
% \begin{function}{unknown noval}
%   \begin{syntax}
%     unknown noval = \marg{definition} \prefixes2323
%   \end{syntax}
%   This is like |unknown code| but uses \meta{definition} for unknown keys to
%   which no value was passed (so corresponds to |\ekvdefunknownNoVal|). You can
%   access the key name with |#1|.
% \end{function}
%
% \begin{function}{unknown redirect-code}
%   \begin{syntax}
%     unknown redirect-code = \marg{set-list} \prefixes2331
%   \end{syntax}
%   This uses a predefined action for |unknown code|. Instead of throwing an
%   error, it is tried to find the \key\ in each \set\ in the comma separated
%   \meta{set-list}. The first found match will be used and the remaining
%   options from the list discarded. If the \key\ isn't found in any \set\ an
%   expandable error will be thrown eventually. Internally \expkv's
%   |\ekvredirectunknown| will be used.
% \end{function}
%
% \begin{function}{unknown redirect-noval}
%   \begin{syntax}
%     unknown redirect-noval = \marg{set-list} \prefixes2333
%   \end{syntax}
%   This behaves just like |unknown redirect-code| but will set up means to
%   forward keys for |unknown noval|. Internally \expkv's
%   |\ekvredirectunknownNoVal| will be used.
% \end{function}
%
% \begin{function}{unknown redirect}
%   \begin{syntax}
%     unknown redirect = \marg{set-list} \prefixes2333
%   \end{syntax}
%   This is a short cut to apply both, |unknown redirect-code| and
%   |unknown redirect-noval|, as a result you might get doubled error messages,
%   one from each.
% \end{function}
%
% \subsection{Bugs}
% I don't think there are any (but every developer says that), if you find some
% please let me know, either via the email address on the first page or on
% GitHub:
% \url{https://github.com/Skillmon/tex_expkv-def}
%
% \subsection{Example}
% The following is an example code defining each base key-type once. Please
% admire the very creative key-name examples.
%
% \begin{lstlisting}
% \ekvdefinekeys{example}
%   {
%      long code keyA = #1
%     ,noval     keyA = NoVal given
%     ,bool      keyB = \keyB
%     ,boolTF    keyC = \keyC
%     ,store     keyD = \keyD
%     ,data      keyE = \keyE
%     ,dataT     keyF = \keyF
%     ,int       keyG = \keyG
%     ,dimen     keyH = \keyH
%     ,skip      keyI = \keyI
%     ,toks      keyJ = \keyJ
%     ,default   keyJ = \empty test
%     ,new box   keyK = \keyK
%     ,qdefault  keyK = K
%     ,choice    keyL =
%       {
%          protected 1 = \texttt{a}
%         ,2 = b
%         ,3 = c
%         ,4 = d
%         ,5 = e
%       }
%     ,edefault  keyL = 2
%     ,meta      keyM = {keyA={#1},keyB=false}
%     ,invbool   keyN = \keyN
%     ,boolpair  keyO = \keyOa\keyOb
%   }
% \end{lstlisting}
%
% \bigskip
% Since the |data| type might be a bit strange, here is another usage example
% for it.
% \begin{lstlisting}
% \ekvdefinekeys{ex}
%   {
%      data  name  = \Pname
%     ,data  age   = \Page
%     ,dataT hobby = \Phobby
%   }
% \newcommand\Person[1]
%   {%
%     \begingroup
%     \ekvset{ex}{#1}%
%     \begin{description}
%       \item[\Pname{}{\errmessage{A person requires a name}}]
%       \item[Age] \Page{\textit}{\errmessage{A person requires an age}}
%       \Phobby{\item[Hobbies]}
%     \end{description}
%     \endgroup
%   }
% \Person{name=Jonathan P. Spratte, age=young, hobby=\TeX\ coding}
% \Person{name=Some User, age=unknown, hobby=Reading Documentation}
% \Person{name=Anybody, age=any}
% \end{lstlisting}
% In this example a person should have a name and an age, but doesn't have to
% have hobbies. The name will be displayed as the description item and the age
% in Italics. If a person has no hobbies the description item will be silently
% left out.  The result of the above code looks like this:
%
% \begin{exresult}{\linewidth}
% \ekvdefinekeys{ex}
%   {
%      data  name  = \Pname
%     ,data  age   = \Page
%     ,dataT hobby = \Phobby
%   }
% \newcommand\Person[1]
%   {%
%     \begingroup
%     \ekvset{ex}{#1}%
%     \begin{description}
%       \item[\Pname{}{\errmessage{A person requires a name}}]
%       \item[Age] \Page{\textit}{\errmessage{A person requires an age}}
%       \Phobby{\item[Hobbies]}
%     \end{description}
%     \endgroup
%   }
% \Person{name=Jonathan P. Spratte, age=young, hobby=\TeX\ coding}
% \Person{name=Some User, age=unknown, hobby=Reading Documentation}
% \Person{name=Anybody, age=any}
% \end{exresult}
%
% \subsection{License}
%
% Copyright \textcopyright\ 2020\unless\ifnum\year=2020--\the\year\fi\
% Jonathan P. Spratte
%
% \medskip\noindent
% This work may be distributed and/or modified under the conditions of the
% \LaTeX\ Project Public License (LPPL), either version 1.3c of this license or
% (at your option) any later version. The latest version of this license is in
% the file:
%
% \url{http://www.latex-project.org/lppl.txt}
%
% \noindent
% This work is ``maintained'' (as per LPPL maintenance status) by
%
% \mbox{Jonathan P. Spratte}.
%
% \end{documentation}^^A=<<
%
% \begin{implementation}^^A>>=
%
% \clearpage
%
% \section{Implementation}^^A>>=
%^^A the LaTeX package >>=
% \subsection{The \LaTeX\ Package}
% Just like for \expkv\ we provide a small \LaTeX\ package that sets up things
% such that we behave nicely on \LaTeX\ packages and files system. It'll
% |\input| the generic code which implements the functionality.
% \gobbledocstriptag
%<*pkg>
%    \begin{macrocode}
\RequirePackage{expkv}
\def\ekvd@tmp
  {%
    \ProvidesFile{expkv-def.tex}%
      [\ekvdDate\space v\ekvdVersion\space a key-defining frontend for expkv]%
  }
% \iffalse meta-comment
%
% File: expkv-def.dtx Copyright (C) 2020-2021 Jonathan P. Spratte
%
% This work  may be  distributed and/or  modified under  the conditions  of the
% LaTeX Project Public License (LPPL),  either version 1.3c  of this license or
% (at your option) any later version.  The latest version of this license is in
% the file:
%
%   http://www.latex-project.org/lppl.txt
%
% ------------------------------------------------------------------------------
%
%<*driver>^^A>>=
\def\nameofplainTeX{plain}
\ifx\fmtname\nameofplainTeX\else
  \expandafter\begingroup
\fi
\input l3docstrip.tex
\askforoverwritefalse
\preamble

--------------------------------------------------------------
expkv-def -- a key-defining frontend for expkv
E-mail: jspratte@yahoo.de
Released under the LaTeX Project Public License v1.3c or later
See http://www.latex-project.org/lppl.txt
--------------------------------------------------------------

Copyright (C) 2020-2021 Jonathan P. Spratte

This  work may be  distributed and/or  modified under  the conditions  of the
LaTeX Project Public License (LPPL),  either version 1.3c  of this license or
(at your option) any later version.  The latest version of this license is in
the file:

  http://www.latex-project.org/lppl.txt

This work is "maintained" (as per LPPL maintenance status) by
  Jonathan P. Spratte.

This work consists of the file  expkv-def.dtx
and the derived files           expkv-def.pdf
                                expkv-def.sty
                                expkv-def.tex
                                t-expkv-def.tex

\endpreamble
% stop docstrip adding \endinput
\postamble
\endpostamble
\generate{\file{expkv-def.sty}{\from{expkv-def.dtx}{pkg}}}
\generate{\file{expkv-def.tex}{\from{expkv-def.dtx}{tex}}}
\generate{\file{t-expkv-def.tex}{\from{expkv-def.dtx}{ctx}}}
\ifx\fmtname\nameofplainTeX
  \expandafter\endbatchfile
\else
  \expandafter\endgroup
\fi
%
\IfFileExists{expkv-def.tex}{\input{expkv-def.tex}}{}
\ProvidesFile{expkv-def.dtx}
  [\csname ekvdDate\endcsname\ a key-defining frontend for expkv]
\PassOptionsToPackage{full}{textcomp}
\documentclass{l3doc}
\RequirePackage[oldstylenums,nott]{kpfonts}
\input{glyphtounicode}
\pdfgentounicode=1
\RequirePackage{tcolorbox}
\newtcolorbox{exresult}[2][]
  {colback=expkvgrey!10!white,colframe=expkvgrey,fontupper=\small,width={#2},#1}
\newtcbox\exres[1][]
  {
     colback=expkvgrey!10!white
    ,colframe=expkvgrey
    ,size=small
    ,nobeforeafter
    ,tcbox raise base
    ,fontupper=\small
    ,#1
  }
\RequirePackage{listings}
\RequirePackage{booktabs}
\RequirePackage{array}
\RequirePackage{collcell}
\RequirePackage{siunitx}
\RequirePackage{xcolor}
\RequirePackage{caption}
\RequirePackage{microtype}
\RequirePackage{accsupp}
\lstdefinelanguage{expkvd}[primitive]{TeX}
  {
    ,moretexcs=[4]^^A e-TeX
      {
        expanded,
        numexpr,
        protected,
      }
    ,moretexcs=[5]^^A plain/LaTeX
      {
        approx,
        begin,
        empty,
        item,
        LaTeX,
        makeatletter,makeatother,
        newcommand,newdimen,
        RequirePackage,
        rule,
        TeX,
        textit,texttt,
        usepackage,
      }
    ,moretexcs=[6]^^A used packages
      {
        ^^A xfp
        fpeval,
        ^^A keyval
        setkeys,
        ^^A yax
        defactiveparameter,storevalue,setparameterlist
      }
    ,moretexcs=[2]^^A expkv-def macros
      {
        ekvdefinekeys
      }
    ,moretexcs=[3]^^A expkv-pkg macros
      {
        ^^A expkv
        ekvdef,ekvdefNoVal,^^A
        ekvlet,ekvletNoVal,ekvletkv,ekvletkvNoVal,^^A
        ekvdefunknown,ekvdefunknownNoVal,^^A
        ekvredirectunknown,ekvredirectunknownNoVal,^^A
        ekvset,ekvsetSneaked,^^A
        ekvsetdef,ekvsetSneakeddef,ekvsetdefSneaked,^^A
        ekvparse,^^A
        ekvVersion,ekvDate,^^A
        ekvifdefined,ekvifdefinedNoVal,ekvifdefinedset,^^A
        ekvbreak,ekvbreakPreSneak,ekvbreakPostSneak,^^A
        ekvsneak,ekvsneakPre,^^A
        ekvchangeset,^^A
        ekvoptarg,ekvoptargTF,^^A
        ekverr,
        ^^A expkv-cs
        ekvcSecondaryKeys,ekvcSplitAndForward,
      }
  }
\colorlet{codeparam}{cyan!65!black}
\def\mylstwd{.55em}
\lstset
  {
    ,language=expkvd
    ,flexiblecolumns=false
    ,basewidth=\mylstwd
    ,gobble=2
    ,basicstyle=\fontfamily{jkp}\itshape
    ,texcsstyle=*[1]{\bfseries\color{expkvgrey}} ^^A primitives
    ,texcsstyle=*[2]{\bfseries\color{expkvred}}  ^^A expkv
    ,texcsstyle=*[3]{\color{expkvred}}           ^^A expkv-pkg
    ,texcsstyle=*[4]{\bfseries\color{expkvgrey}} ^^A e-TeX
    ,texcsstyle=*[5]{\bfseries\color{expkvgrey}} ^^A plain/LaTeX macros
    ,texcsstyle=*[6]{}                           ^^A macros of other packages
    ,commentstyle=\color[gray]{0.4}
    ,literate=
              {\{} {{\CodeUpBf\{}}{1}
              {\}} {{\CodeUpBf\}}}{1}
              {$}  {{\CodeUpBf\$}}{1}
              {[}  {{\CodeUp[}}{1}
              {]}  {{\CodeUp]}}{1}
              {(}  {{\CodeUp(}}{1}
              {)}  {{\CodeUp)}}{1}
              {*}  {{$*$}}{1}
              {1}  {{\CodeUp{1}}}{1}
              {2}  {{\CodeUp{2}}}{1}
              {3}  {{\CodeUp{3}}}{1}
              {4}  {{\CodeUp{4}}}{1}
              {5}  {{\CodeUp{5}}}{1}
              {6}  {{\CodeUp{6}}}{1}
              {7}  {{\CodeUp{7}}}{1}
              {8}  {{\CodeUp{8}}}{1}
              {9}  {{\CodeUp{9}}}{1}
              {0}  {{\CodeUp{0}}}{1}
              {##} {{\CodeColored{codeparam}{1}{\#}}}{1}
              {##1}{{\CodeColored{codeparam}{2}{\#1}}}{2}
              {##2}{{\CodeColored{codeparam}{2}{\#2}}}{2}
              {##3}{{\CodeColored{codeparam}{2}{\#3}}}{2}
              {##4}{{\CodeColored{codeparam}{2}{\#4}}}{2}
              {##5}{{\CodeColored{codeparam}{2}{\#5}}}{2}
              {##6}{{\CodeColored{codeparam}{2}{\#6}}}{2}
              {##7}{{\CodeColored{codeparam}{2}{\#7}}}{2}
              {##8}{{\CodeColored{codeparam}{2}{\#8}}}{2}
              {##9}{{\CodeColored{codeparam}{2}{\#9}}}{2}
              {<key>}{{$\langle$}key{$\rangle$}}{5}
              {<set>}{{$\langle$}set{$\rangle$}}{5}
              {<set1>}{{$\langle$}set1{$\rangle$}}{6}
              {<set2>}{{$\langle$}set2{$\rangle$}}{6}
    ^^A,literate=*{<key>}{\key}{4}{<set>}{\set}{4}
  }
\newcommand*\CodeColored[3]{\textcolor{#1}{\makebox[\dimexpr\mylstwd*#2]{#3}}}
\newcommand*\CodeUpBf[1]{\makebox[\mylstwd]{\textup{\textbf{#1}}}}
\newcommand*\CodeUp[1]{\makebox[\mylstwd]{\textup{#1}}}
\RequirePackage{randtext}
\let\metaORIG\meta
\protected\def\meta #1{\texttt{\metaORIG{#1}}}
\renewcommand*\thefootnote{\fnsymbol{footnote}}
\definecolor{expkvred}{HTML}{9F393D}
\colorlet{expkvgrey}{black!75}
\makeatletter
\newcommand*\expkv
  {^^A
    \texorpdfstring
      {^^A
        \mbox
          {^^A
            \BeginAccSupp{ActualText=expkv}^^A
            \href{https://github.com/Skillmon/tex_expkv}
              {^^A
                \rmfamily
                \bfseries
                {\color{expkvgrey}e\kern-.05em x\kern-.05em}^^A
                \lower.493ex
                  \hbox{{\color{expkvgrey}P}\kern-.1em{\color{expkvred}k}}^^A
                \kern-.18em{\color{expkvred}v}^^A
              }^^A
            \EndAccSupp{}^^A
          }^^A
      }
      {expkv}^^A
  }
\newcommand*\expkvd
  {^^A
    \texorpdfstring
      {^^A
        \mbox
          {^^A
            \BeginAccSupp{ActualText=expkv-def}^^A
            \href{https://github.com/Skillmon/tex_expkv-def}
              {^^A
                \rmfamily
                \bfseries
                {\color{expkvgrey}e\kern-.05em x\kern-.05em}^^A
                \lower.493ex
                  \hbox{{\color{expkvgrey}P}\kern-.1em{\color{expkvred}k}}^^A
                \kern-.18em{\color{expkvred}v}^^A
                {\color{expkvgrey}^^A
                  \kern.05em\rule[-.1ex]{.08em}{1.2ex}\kern.05em\textsc{def}^^A
                }^^A
              }^^A
            \EndAccSupp{}^^A
          }^^A
      }
      {expkv-def}^^A
  }
\newcommand\kv{\meta{key}=\meta{value}}
\newcommand\key{\meta{key}}
\newcommand\val{\meta{value}}
\newcommand\set{\meta{set}}
\newcommand\enfprefix{\textcolor{black}}
\newcommand\allprefix{\textcolor{gray}}
\newcommand\notprefix{\textcolor{red!80!black}}
\newcommand\printprefix[2]
  {^^A
    \ifcase\numexpr#1\relax\or
      \enfprefix{#2}\or
      \allprefix{#2}\or
      \notprefix{#2}\fi
  }
\newcommand\prefixes[4]
  {^^A
    \hfill
    \printprefix{#1}{new}
    \printprefix{#2}{also}
    \printprefix{#3}{protected}
    \printprefix{#4}{long}^^A
  }
\hypersetup{linkcolor=red!80!black,urlcolor=purple!80!black}
\DoNotIndex{\def,\edef,\,,\=,\begingroup,\catcode,\chardef,\csname,\endcsname}
\DoNotIndex{\endgroup,\endinput,\errmessage,\expandafter,\input,\let,\long}
\DoNotIndex{\protected,\ProvidesFile,\ProvidesPackage,\relax,\space}
\DoNotIndex{\@,\unexpanded,\string,\expanded,\dimexpr,\global,\glueexpr,\hbox}
\DoNotIndex{\numexpr,\RequirePackage,\setbox,\the,\unless,\xdef,\@firstofone}
\DoNotIndex{\@firstoftwo,\@gobble,\@secondoftwo,\meaning,\detokenize}
\DoNotIndex{\romannumeral,\^}
\DoNotIndex{\ifcsname}
\DoNotIndex{\ifx}
\DoNotIndex{\ifdefined}
\DoNotIndex{\iffalse}
\DoNotIndex{\iftrue}
\DoNotIndex{\else}
\DoNotIndex{\fi}
\@gobble\fi ^^A ignoring \ifx, \ifcsname, etc., but only one \fi
\@gobble\fi ^^A ignoring \ifx, \ifcsname, etc., but only one \fi
\@gobble\fi ^^A ignoring \ifx, \ifcsname, etc., but only one \fi
\@gobble\fi ^^A ignoring \ifx, \ifcsname, etc., but only one \fi
\@ifdefinable\gobbledocstriptag{\def\gobbledocstriptag#1>{}}
\makeatother
\begin{document}
  \title
    {^^A
      \texorpdfstring
        {^^A
          \huge\expkvd
          \\[\medskipamount]
          \Large a key-defining frontend for \expkv
        }{expkv-def - a key-defining frontend for expkv}^^A
    }
  \date{\ekvdDate\space v\ekvdVersion}
  \author{Jonathan P. Spratte\thanks{\protect\randomize{jspratte@yahoo.de}}}
  \DocInput{expkv-def.dtx}
\end{document}
%</driver>^^A=<<
% \fi
%
% \maketitle
% \renewcommand*\thefootnote{\arabic{footnote}}
%
% \begin{abstract}
% \noindent\parfillskip=0pt
% \expkvd\ provides a small \kv\ interface to define keys for \expkv. Key-types
% are declared using prefixes, similar to static typed languages. The stylised
% name is \expkvd\ but the files use \file{expkv-def}, this is due to CTAN-rules
% which don't allow \string| in package names since that is the pipe symbol in
% *nix shells.
% \end{abstract}
%
% \tableofcontents
%
% \begin{documentation}^^A>>=
%
% \section{Documentation}
%
% Since the trend for the last couple of years goes to defining keys for a \kv\
% interface using a \kv\ interface, I thought that maybe providing such an
% interface for \expkv\ will make it more attractive for actual use, besides its
% unique selling points of being fully expandable, and fast and reliable. But at
% the same time I don't want to widen \expkv's initial scope. So here it is
% \expkvd, go define \kv\ interfaces with \kv\ interfaces.
%
% Unlike many of the other established \kv\ interfaces to define keys, \expkvd\
% works using prefixes instead of suffixes (\emph{e.g.}, |.tl_set:N| of
% \pkg{l3keys}) or directory like handlers (\emph{e.g.}, |/.store in| of
% \pkg{pgfkeys}). This was decided as a personal preference, more over in \TeX\
% parsing for the first space is way easier than parsing for the last one.
% \expkvd's prefixes are sorted into two categories: |p|-type, which are
% equivalent to \TeX's prefixes like |\long|, and |t|-type defining the type of
% the key. For a description of the available |p|-prefixes take a look at
% \autoref{sec:prefix:p}, the |t|-prefixes are described in
% \autoref{sec:prefix:t}.
%
% \expkvd\ is usable as generic code, as a \LaTeX\ package, and as a \ConTeXt\
% module. It'll automatically load \expkv\ in the same mode as well. To use it,
% just use one of
% \begin{lstlisting}
% \input expkv-def       % plainTeX
% \usepackage{expkv-def} % LaTeX
% \usemodule[expkv-def]  % ConTeXt
% \end{lstlisting}
%
% \subsection{Macros}
% Apart from version and date containers there is only a single user-facing
% macro, and that should be used to define keys.
%
% \begin{function}{\ekvdefinekeys}
%   \begin{syntax}
%     \cs{ekvdefinekeys}\marg{set}\{\kv, \ldots\}
%   \end{syntax}
%   In \set, define \key\ to have definition \val. The general syntax for \key\
%   should be
%   \begin{quote}\ttfamily\small
%     \meta{prefix} \meta{name}
%   \end{quote}
%   Where \meta{prefix} is a space separated list of optional |p|-type prefixes
%   followed by one |t|-type prefix. The syntax of \val\ is dependent on the
%   used |t|-prefix.
% \end{function}
%
% \begin{function}{\ekvdDate,\ekvdVersion}
%   These two macros store the version and date of the package.
% \end{function}
%
% \subsection{Prefixes}
% As already said there are |p|-prefixes and |t|-prefixes. Not every |p|-prefix
% is allowed for all |t|-prefixes.
%
% \subsubsection{\texttt{p}-Prefixes}\label{sec:prefix:p}
% The two |p|-type prefixes |long| and |protected| are pretty simple by nature,
% so their description is pretty simple. They affect the \key\ at use-time, so
% omitting |long| doesn't mean that a \meta{definition} can't contain a |\par|
% token, only that the \key\ will not accept a |\par| in \val.  On the other
% hand |new| and |also| might be simple on first sight as well, but their rules
% are a bit more complicated.
%
% \begin{function}{also}
%   The following key type will be \emph{added} to an existing \key's
%   definition. You can't add a type taking an argument at use time to an
%   existing key which doesn't take an argument and vice versa. Also you'll get
%   an error if you try to add an action which isn't allowed to be either |long|
%   or |protected| to a key which already is |long| or |protected| (the opposite
%   order would be suboptimal as well, but can't be really captured with the
%   current code).
%
%   A key already defined as |long| or |protected| will stay |long| or
%   |protected|, but you can as well add |long| or |protected| with the |also|
%   definition.
%
%   As a small example, suppose you want to create a boolean key, but
%   additionally to setting a boolean value you want to execute some more code
%   as well, you can use the following
% \begin{lstlisting}
% \ekvdefinekeys{also-example}
%   {
%      bool key      = \ifmybool
%     ,also code key = \domystuff{#1}
%   }
% \end{lstlisting}
%
%   If you use |also| on a |choice|, |bool|, |invbool|, or |boolpair| key it is
%   tried to determine if the key already is of one of those types. If this test
%   is true the declared choices will be added to the possible choices but the
%   key's definition will not be changed other than that. If that wouldn't have
%   been done, the callbacks of the different choices could get called multiple
%   times.
% \end{function}
%
% \begin{function}{protected,protect}
%   The following key will be defined |\protected|. Note that key-types which
%   can't be defined expandable will always use |\protected|.
% \end{function}
%
% \begin{function}{long}
%   The following key will be defined |\long|.
% \end{function}
%
% \begin{function}{new}
%   The following key must be new (so previously undefined). An error is thrown
%   if it is already defined and the new definition is ignored. |new| only
%   asserts that there are no conflicts between |NoVal| keys and other |NoVal|
%   keys or value taking keys and other value taking keys. For example you can
%   use the following without an error:
% \begin{lstlisting}
% \ekvdefinekeys{new-example}
%   {
%      code key       = \domystuffwitharg{#1}
%     ,new  noval key = \domystuffwithoutarg
%   }
% \end{lstlisting}
% \end{function}
%
% \subsubsection{\texttt{t}-Prefixes}\label{sec:prefix:t}
% Since the |p|-type prefixes apply to some of the |t|-prefixes automatically
% but sometimes one might be disallowed we need some way to highlight this
% behaviour. In the following an enforced prefix will be printed black
% (\texttt{\enfprefix{protected}}), allowed prefixes will be grey
% (\texttt{\allprefix{protected}}), and disallowed prefixes will be red
% (\texttt{\notprefix{protected}}). This will be put flush-right in the syntax
% showing line.
%
% \begin{function}{code,ecode}
%   \begin{syntax}
%     code \key\ = \marg{definition} \prefixes2222
%   \end{syntax}
%   Define \key\ to expand to \meta{definition}. The \key\ will require a \val\
%   for which you can use |#1| inside \meta{definition}. The |ecode| variant
%   will fully expand \meta{definition} inside an |\edef|.
% \end{function}
%
% \begin{function}{noval,enoval}
%   \begin{syntax}
%     noval \key\ = \marg{definition} \prefixes2223
%   \end{syntax}
%   The |noval| type defines \key\ to expand to \meta{definition}. The \key\
%   will not take a \val. |enoval| fully expands \meta{definition} inside an
%   |\edef|.
% \end{function}
%
% \begin{function}{default,qdefault,odefault,fdefault,edefault}
%   \begin{syntax}
%     default \key\ = \marg{definition} \prefixes3223
%   \end{syntax}
%   This serves to place a default \val\ for a \key\ that takes an argument, the
%   \key\ can be of any argument-grabbing kind, and when used without a \val\
%   it will be passed \meta{definition} instead. The |qdefault| variant will
%   expand the \key's code once, so will be slightly quicker, but not change if
%   you redefine \key. |odefault| is just another name for |qdefault|. The
%   |fdefault| version will expand the key code until a non-expandable token or
%   a space is found, a space would be gobbled.\footnotemark{} The |edefault| on
%   the other hand fully expands the \key-code with \meta{definition} as its
%   argument inside of an |\edef|.
% \end{function}
% \footnotetext
%   {^^A
%     For those familiar with \TeX-coding: This uses a
%     \cs[no-index]{romannumeral}-expansion.
%   }
%
% \begin{function}{initial,oinitial,finitial,einitial}
%   \begin{syntax}
%     initial \key\ = \{\val\} \prefixes3333
%     initial \key
%   \end{syntax}
%   With |initial| you can set an initial \val\ for an already defined argument
%   taking \key. It'll just call the key-macro of \key\ and pass it \val. The
%   |einitial| variant will expand \val\ using an |\edef| expansion prior to
%   passing it to the key-macro and the |oinitial| variant will expand the first
%   token in \val\ once. |finitial| will expand \val\ until a non-expandable
%   token or a space is found, a space would be
%   gobbled.\footnotemark
%
%   If you don't provide a value (and no equals sign) a |noval| \key\ of the
%   same name is called once (or, if you specified a |default| for a value
%   taking key that would be used).
% \end{function}
% \footnotetext{Again using \cs[no-index]{romannumeral}.}
%
% \begin{function}{bool,gbool,boolTF,gboolTF}
%   \begin{syntax}
%     bool \key\ = \meta{cs} \prefixes2223
%   \end{syntax}
%   The \meta{cs} should be a single control sequence, such as |\iffoo|.
%   This will define \key\ to be a boolean key, which only takes the values
%   |true| or |false| and will throw an error for other values. If the key is
%   used without a \val\ it'll have the same effect as if you use
%   \texttt{\key=true}. |bool| and |gbool| will behave like \TeX-ifs so either
%   be |\iftrue| or |\iffalse|. The |boolTF| and |gboolTF| variants will both
%   take two arguments and if true the first will be used else the second, so
%   they are always either |\@firstoftwo| or |\@secondoftwo|. The variants with
%   a leading |g| will set the control sequence globally, the others locally. If
%   \meta{cs} is not yet defined it'll be initialised as the |false| version.
%   Note that the initialisation is \emph{not} done with |\newif|, so you will
%   not be able to do |\footrue| outside of the \kv\ interface, but you could
%   use |\newif| yourself. Even if the \key\ will not be |\protected| the
%   commands which execute the |true| or |false| choice will be, so the usage
%   should be safe in an expansion context (\emph{e.g.}, you can use
%   \texttt{edefault \key\ = false} without an issue to change the default
%   behaviour to execute the |false| choice). Internally a |bool| \key\ is the
%   same as a |choice| key which is set up to handle |true| and |false| as
%   choices.
% \end{function}
%
% \begin{function}{invbool,ginvbool,invboolTF,ginvboolTF}
%   \begin{syntax}
%     bool \key\ = \meta{cs} \prefixes2223
%   \end{syntax}
%   These are inverse boolean keys, they behave like |bool| and friends but set
%   the opposite meaning to the macro \meta{cs} in each case. So if
%   \texttt{key=true} is used |invbool| will set \meta{cs} to |\iffalse| and
%   vice versa.
% \end{function}
%
% \begin{function}{boolpair,gboolpair,boolpairTF,gboolpairTF}
%   \begin{syntax}
%     boolpair \key\ = \meta{cs_1}\meta{cs_2} \prefixes2223
%   \end{syntax}
%   The |boolpair| key type behaves like both |bool| and |invbool|, the
%   \meta{cs_1} will be set to the meaning according to the rules of |bool|, and
%   \meta{cs_2} will be set to the opposite.
% \end{function}
%
% \begin{function}{store,estore,gstore,xstore}
%   \begin{syntax}
%     store \key\ = \meta{cs} \prefixes2212
%   \end{syntax}
%   The \meta{cs} should be a single control sequence, such as |\foo|.
%   This will define \key\ to store \val\ inside of the control sequence. If
%   \meta{cs} isn't yet defined it will be initialised as empty. The variants
%   behave similarly to their |\def|, |\edef|, |\gdef|, and |\xdef|
%   counterparts, but |store| and |gstore| will allow you to store macro
%   parameters inside of them by using |\unexpanded|.
% \end{function}
%
% \begin{function}{data,edata,gdata,xdata}
%   \begin{syntax}
%     data \key\ = \meta{cs} \prefixes2212
%   \end{syntax}
%   The \meta{cs} should be a single control sequence, such as |\foo|. This will
%   define \key\ to store \val\ inside of the control sequence. But unlike the
%   |store| type, the macro \meta{cs} will be a switch at the same time, it'll
%   take two arguments and if \key\ was used expands to the first argument
%   followed by \val\ in braces, if \key\ was not used \meta{cs} will expand to
%   the second argument (so behave like |\@secondoftwo|). The idea is that with
%   this type you can define a key which should be typeset formatted. The
%   |edata| and |xdata| variants will fully expand \val, the |gdata| and |xdata|
%   variants will store \val\ inside \meta{cs} globally. The |p|-prefixes will
%   only affect the key-macro, \meta{cs} will always be expandable and |\long|.
% \end{function}
%
% \begin{function}{dataT,edataT,gdataT,xdataT}
%   \begin{syntax}
%     dataT \key\ = \meta{cs} \prefixes2212
%   \end{syntax}
%   Just like |data|, but instead of \meta{cs} grabbing two arguments it'll only
%   grab one, so by default it'll behave like |\@gobble|, and if a \val\ was
%   given to \key\ the \meta{cs} will behave like |\@firstofone| appended by
%   \marg{value}.
% \end{function}
%
% \begin{function}{int,eint,gint,xint}
%   \begin{syntax}
%     int \key\ = \meta{cs} \prefixes2212
%   \end{syntax}
%   The \meta{cs} should be a single control sequence, such as |\foo|.
%   An |int| key will be a \TeX-count register. If \meta{cs} isn't defined yet,
%   |\newcount| will be used to initialise it. The |eint| and |xint| versions
%   will use |\numexpr| to allow basic computations in their \val. The |gint|
%   and |xint| variants set the register globally.
% \end{function}
%
% \begin{function}{dimen,edimen,gdimen,xdimen}
%   \begin{syntax}
%     dimen \key\ = \meta{cs} \prefixes2212
%   \end{syntax}
%   The \meta{cs} should be a single control sequence, such as |\foo|.
%   This is just like |int| but uses a dimen register, |\newdimen| and
%   |\dimexpr| instead.
% \end{function}
%
% \begin{function}{skip,eskip,gskip,xskip}
%   \begin{syntax}
%     skip \key\ = \meta{cs} \prefixes2212
%   \end{syntax}
%   The \meta{cs} should be a single control sequence, such as |\foo|.
%   This is just like |int| but uses a skip register, |\newskip| and |\glueexpr|
%   instead.
% \end{function}
%
% \begin{function}{toks,gtoks,apptoks,gapptoks,pretoks,gpretoks}
%   \begin{syntax}
%     toks \key\ = \meta{cs} \prefixes2212
%   \end{syntax}
%   The \meta{cs} should be a single control sequence, such as |\foo|.
%   Store \val\ inside of a toks-register. The |g| variants use |\global|, the
%   |app| variants append \val\ to the contents of that register, the |pre|
%   variants will prepend \val. If \meta{cs} is not yet defined it will be
%   initialised with |\newtoks|.
% \end{function}
%
% \begin{function}{box,gbox}
%   \begin{syntax}
%     box \key\ = \meta{cs} \prefixes2212
%   \end{syntax}
%   The \meta{cs} should be a single control sequence, such as |\foo|.
%   Typesets \val\ into a |\hbox| and stores the result in a box register. The
%   boxes are colour safe. \expkvd\ doesn't provide a |vbox| type.
% \end{function}
%
% \begin{function}{meta}
%   \begin{syntax}
%     meta \key\ = \{\kv, \ldots\} \prefixes2222
%   \end{syntax}
%   This key type can set other keys, you can access the \val\ which was passed
%   to \key\ inside the \kv\ list with |#1|. It works by calling a sub-|\ekvset|
%   on the \kv\ list, so a |set| key will only affect that \kv\ list and not the
%   current |\ekvset|. Since it runs in a separate |\ekvset| you can't use
%   |\ekvsneak| using keys or similar macros in the way you normally could.
% \end{function}
%
% \begin{function}{nmeta}
%   \begin{syntax}
%     nmeta \key\ = \{\kv, \ldots\} \prefixes2223
%   \end{syntax}
%   This key type can set other keys, the difference to |meta| is, that this key
%   doesn't take a value, so the \kv\ list is static.
% \end{function}
%
% \begin{function}{smeta}
%   \begin{syntax}
%     smeta \key\ = \marg{set}\{\kv, \ldots\} \prefixes2222
%   \end{syntax}
%   Yet another |meta| variant. An |smeta| key will take a \val\ which you can
%   access using |#1|, but it sets the \kv\ list inside of \set, so is equal to
%   \texttt{\cs[no-index]{ekvset}\marg{set}\{\kv, \ldots\}}.
% \end{function}
%
% \begin{function}{snmeta}
%   \begin{syntax}
%     snmeta \key\ = \marg{set}\{\kv, \ldots\} \prefixes2223
%   \end{syntax}
%   And the last |meta| variant. |snmeta| is a combination of |smeta| and
%   |nmeta|. It doesn't take an argument and sets the \kv\ list inside of \set.
% \end{function}
%
% \begin{function}{set}
%   \begin{syntax}
%     set \key\ = \marg{set} \prefixes2233
%   \end{syntax}
%   This will define \key\ to change the set of the current |\ekvset|
%   invocation to \set. You can omit \set\ (including the equals sign),
%   which is the same as using \texttt{set \key\ = \{\key\}}. The created |set|
%   key will not take a \val. Note that just like in \expkv\ it'll not be
%   checked whether \set\ is defined and you'll get a low-level \TeX\ error if
%   you use an undefined \set.
% \end{function}
%
% \begin{function}{choice}
%   \begin{syntax}
%     choice \key\ = \{\val=\meta{definition}, \ldots\} \prefixes2223
%   \end{syntax}
%   Defines \key\ to be a |choice| key, meaning it will only accept a limited
%   set of values. You should define each possible \val\ inside of the
%   \val=\meta{definition} list. If a defined \val\ is passed to \key\ the
%   \meta{definition} will be left in the input stream. You can make
%   individual values |protected| inside the \val=\meta{definition} list. By
%   default a |choice| key is expandable, an undefined \val\ will throw an error
%   in an expandable way (but see the |unknown-choice| prefix). You can add
%   additional choices after the \key\ was created by using |choice| again for
%   the same \key, redefining choices is possible the same way, but there is no
%   interface to remove certain choices.
% \end{function}
%
% \begin{function}{choice-store}
%   \begin{syntax}
%     choice-store \key\ = \meta{cs}\{\val, \ldots\} \prefixes2223
%   \end{syntax}
%   This is defines a special type of |choice| key that'll store the given
%   choice inside the macro \meta{cs} (so \meta{cs} should be a single control
%   sequence name such as |\foo|). Since storing inside a macro can't be done
%   expandably every choice-code is |\protected|, you might define the
%   |choice-store| key itself as |protected| as well if you want. Since the
%   definition of each choice is predefined with this key type the choice list
%   should just be a comma separated list of valid choices.
% \end{function}
% This means that the following |choice| and |choice-store| keys are equivalent
% at use time:
% \begin{lstlisting}
% \newcommand*\mya{}
% \ekvdefinekeys{example}
%   {
%      choice key1 = {a=\def\mya{a}, b=\def\mya{b}, c=\def\mya{c}}
%     ,choice-store key2 = \mya{a,b,c}
%   }
% \end{lstlisting}
%
% \begin{function}{choice-enum}
%   \begin{syntax}
%     choice-enum \key\ = \meta{cs}\{\val, \ldots\} \prefixes2223
%   \end{syntax}
%   This is similar to |choice-store|, the differences are: \meta{cs} should be
%   a count-register or is initialised as such if the \meta{cs} is undefined
%   (via |\newcount|); instead of the value the position of the value in the
%   given list is stored in this register (zero-based).
% \end{function}
% This means that the following |choice| and |choice-enum| keys are equivalent
% at use time:
% \begin{lstlisting}
% \newcount\myb
% \ekvdefinekeys{example}
%   {
%      choice key1 = {a={\myb=0 }, b={\myb=1 }, c={\myb=2 }}
%     ,choice-enum key2 = \myb{a,b,c}
%   }
% \end{lstlisting}
%
% \begin{function}{unknown-choice}
%   \begin{syntax}
%     unknown-choice \key\ = \marg{definition} \prefixes2323
%   \end{syntax}
%   By default an unknown \val\ passed to a |choice| or |bool| key will throw an
%   error.  However, with this prefix you can define an alternative action which
%   should be executed if \key\ received an unknown choice. In \meta{definition}
%   you can refer to the choice which was passed in with |#1|.
% \end{function}
%
% \begin{function}{unknown code}
%   \begin{syntax}
%     unknown code = \marg{definition} \prefixes2322
%   \end{syntax}
%   By default \expkv\ throws errors when it encounters unknown keys in a set.
%   With the |unknown| prefix you can define handlers that deal with undefined
%   keys, instead of a \key\ name you have to specify a subtype for this prefix,
%   here the subtype is |code|.\par
%   With |unknown code| the \meta{definition} is used for unknown keys which
%   were provided a value (so corresponds to |\ekvdefunknown|), you can access
%   the key name with |#1| and the value with |#2|.\footnotemark
% \end{function}
% \footnotetext
%   {^^A
%     There is some trickery involved to get this more intuitive argument order
%     without any performance hit if you compare this to
%     \cs[no-index]{ekvdefunknown} directly.^^A
%   }
%
% \begin{function}{unknown noval}
%   \begin{syntax}
%     unknown noval = \marg{definition} \prefixes2323
%   \end{syntax}
%   This is like |unknown code| but uses \meta{definition} for unknown keys to
%   which no value was passed (so corresponds to |\ekvdefunknownNoVal|). You can
%   access the key name with |#1|.
% \end{function}
%
% \begin{function}{unknown redirect-code}
%   \begin{syntax}
%     unknown redirect-code = \marg{set-list} \prefixes2331
%   \end{syntax}
%   This uses a predefined action for |unknown code|. Instead of throwing an
%   error, it is tried to find the \key\ in each \set\ in the comma separated
%   \meta{set-list}. The first found match will be used and the remaining
%   options from the list discarded. If the \key\ isn't found in any \set\ an
%   expandable error will be thrown eventually. Internally \expkv's
%   |\ekvredirectunknown| will be used.
% \end{function}
%
% \begin{function}{unknown redirect-noval}
%   \begin{syntax}
%     unknown redirect-noval = \marg{set-list} \prefixes2333
%   \end{syntax}
%   This behaves just like |unknown redirect-code| but will set up means to
%   forward keys for |unknown noval|. Internally \expkv's
%   |\ekvredirectunknownNoVal| will be used.
% \end{function}
%
% \begin{function}{unknown redirect}
%   \begin{syntax}
%     unknown redirect = \marg{set-list} \prefixes2333
%   \end{syntax}
%   This is a short cut to apply both, |unknown redirect-code| and
%   |unknown redirect-noval|, as a result you might get doubled error messages,
%   one from each.
% \end{function}
%
% \subsection{Bugs}
% I don't think there are any (but every developer says that), if you find some
% please let me know, either via the email address on the first page or on
% GitHub:
% \url{https://github.com/Skillmon/tex_expkv-def}
%
% \subsection{Example}
% The following is an example code defining each base key-type once. Please
% admire the very creative key-name examples.
%
% \begin{lstlisting}
% \ekvdefinekeys{example}
%   {
%      long code keyA = #1
%     ,noval     keyA = NoVal given
%     ,bool      keyB = \keyB
%     ,boolTF    keyC = \keyC
%     ,store     keyD = \keyD
%     ,data      keyE = \keyE
%     ,dataT     keyF = \keyF
%     ,int       keyG = \keyG
%     ,dimen     keyH = \keyH
%     ,skip      keyI = \keyI
%     ,toks      keyJ = \keyJ
%     ,default   keyJ = \empty test
%     ,new box   keyK = \keyK
%     ,qdefault  keyK = K
%     ,choice    keyL =
%       {
%          protected 1 = \texttt{a}
%         ,2 = b
%         ,3 = c
%         ,4 = d
%         ,5 = e
%       }
%     ,edefault  keyL = 2
%     ,meta      keyM = {keyA={#1},keyB=false}
%     ,invbool   keyN = \keyN
%     ,boolpair  keyO = \keyOa\keyOb
%   }
% \end{lstlisting}
%
% \bigskip
% Since the |data| type might be a bit strange, here is another usage example
% for it.
% \begin{lstlisting}
% \ekvdefinekeys{ex}
%   {
%      data  name  = \Pname
%     ,data  age   = \Page
%     ,dataT hobby = \Phobby
%   }
% \newcommand\Person[1]
%   {%
%     \begingroup
%     \ekvset{ex}{#1}%
%     \begin{description}
%       \item[\Pname{}{\errmessage{A person requires a name}}]
%       \item[Age] \Page{\textit}{\errmessage{A person requires an age}}
%       \Phobby{\item[Hobbies]}
%     \end{description}
%     \endgroup
%   }
% \Person{name=Jonathan P. Spratte, age=young, hobby=\TeX\ coding}
% \Person{name=Some User, age=unknown, hobby=Reading Documentation}
% \Person{name=Anybody, age=any}
% \end{lstlisting}
% In this example a person should have a name and an age, but doesn't have to
% have hobbies. The name will be displayed as the description item and the age
% in Italics. If a person has no hobbies the description item will be silently
% left out.  The result of the above code looks like this:
%
% \begin{exresult}{\linewidth}
% \ekvdefinekeys{ex}
%   {
%      data  name  = \Pname
%     ,data  age   = \Page
%     ,dataT hobby = \Phobby
%   }
% \newcommand\Person[1]
%   {%
%     \begingroup
%     \ekvset{ex}{#1}%
%     \begin{description}
%       \item[\Pname{}{\errmessage{A person requires a name}}]
%       \item[Age] \Page{\textit}{\errmessage{A person requires an age}}
%       \Phobby{\item[Hobbies]}
%     \end{description}
%     \endgroup
%   }
% \Person{name=Jonathan P. Spratte, age=young, hobby=\TeX\ coding}
% \Person{name=Some User, age=unknown, hobby=Reading Documentation}
% \Person{name=Anybody, age=any}
% \end{exresult}
%
% \subsection{License}
%
% Copyright \textcopyright\ 2020\unless\ifnum\year=2020--\the\year\fi\
% Jonathan P. Spratte
%
% \medskip\noindent
% This work may be distributed and/or modified under the conditions of the
% \LaTeX\ Project Public License (LPPL), either version 1.3c of this license or
% (at your option) any later version. The latest version of this license is in
% the file:
%
% \url{http://www.latex-project.org/lppl.txt}
%
% \noindent
% This work is ``maintained'' (as per LPPL maintenance status) by
%
% \mbox{Jonathan P. Spratte}.
%
% \end{documentation}^^A=<<
%
% \begin{implementation}^^A>>=
%
% \clearpage
%
% \section{Implementation}^^A>>=
%^^A the LaTeX package >>=
% \subsection{The \LaTeX\ Package}
% Just like for \expkv\ we provide a small \LaTeX\ package that sets up things
% such that we behave nicely on \LaTeX\ packages and files system. It'll
% |\input| the generic code which implements the functionality.
% \gobbledocstriptag
%<*pkg>
%    \begin{macrocode}
\RequirePackage{expkv}
\def\ekvd@tmp
  {%
    \ProvidesFile{expkv-def.tex}%
      [\ekvdDate\space v\ekvdVersion\space a key-defining frontend for expkv]%
  }
\input{expkv-def.tex}
\ProvidesPackage{expkv-def}%
  [\ekvdDate\space v\ekvdVersion\space a key-defining frontend for expkv]
%    \end{macrocode}
% \gobbledocstriptag
%</pkg>
%^^A=<<
%^^A the ConTeXt module >>=
% \subsection{The \ConTeXt\ module}
% \gobbledocstriptag
%<*ctx>
%    \begin{macrocode}
\writestatus{loading}{ConTeXt User Module / expkv-def}
\usemodule[expkv]
\unprotect
\input expkv-def.tex
\writestatus{loading}
  {ConTeXt User Module / expkv-def / Version \ekvdVersion\space loaded}
\protect\endinput
%    \end{macrocode}
% \gobbledocstriptag
%</ctx>
%^^A=<<
%^^A main file >>=
% \subsection{The Generic Code}
% The rest of this implementation will be the generic code.
% \gobbledocstriptag
%<*tex>
%
% Load \expkv\ if the package didn't already do so -- since \expkv\ has
% safeguards against being loaded twice this does no harm and the overhead
% isn't that big. Also we reuse some of the internals of \expkv\ to save us from
% retyping them.
%    \begin{macrocode}
\input expkv
%    \end{macrocode}
%
% We make sure that \file{expkv-def.tex} is only input once:
%    \begin{macrocode}
\expandafter\ifx\csname ekvdVersion\endcsname\relax
\else
  \expandafter\endinput
\fi
%    \end{macrocode}
%
% \begin{macro}{\ekvdVersion,\ekvdDate}
% We're on our first input, so lets store the version and date in a macro.
%    \begin{macrocode}
\def\ekvdVersion{0.8c}
\def\ekvdDate{2021-09-20}
%    \end{macrocode}
% \end{macro}
%
% If the \LaTeX\ format is loaded we want to be a good file and report back who
% we are, for this the package will have defined |\ekvd@tmp| to use
% |\ProvidesFile|, else this will expand to a |\relax| and do no harm.
%    \begin{macrocode}
\csname ekvd@tmp\endcsname
%    \end{macrocode}
%
% Store the category code of |@| to later be able to reset it and change it to
% 11 for now.
%    \begin{macrocode}
\expandafter\chardef\csname ekvd@tmp\endcsname=\catcode`\@
\catcode`\@=11
%    \end{macrocode}
% |\ekvd@tmp| will be reused later to handle expansion during the key
% defining. But we don't need it to ever store information long-term after
% \expkvd\ was initialized.
%
% \begin{macro}[internal]{\ekvd@ifprimitive}
%    \begin{macrocode}
\protected\long\def\ekvd@ifprimitive#1%
  {%
    \begingroup
      \edef\ekvd@tmpa{\string #1}%
      \edef\ekvd@tmpb{\meaning#1}%
      \expandafter
    \endgroup
    \ifx\ekvd@tmpa\ekvd@tmpb
      \ekv@fi@firstoftwo
    \fi
    \@secondoftwo
  }
%    \end{macrocode}
% \end{macro}
%
% \begin{macro}[internal]
%   {\ekvd@long,\ekvd@prot,\ekvd@clear@prefixes,\ekvd@ifalso}
% \expkvd\ will use |\ekvd@long|, |\ekvd@prot|, and |\ekvd@ifalso| to store
% whether a key should be defined as |\long| or |\protected| or adds an action
% to an existing key, and we have to clear them for every new key. By default
% |long| and |protected| will just be empty, |ifalso| will be
% |\@secondoftwo|, and |ifnew| will just use its third argument.
%    \begin{macrocode}
\protected\def\ekvd@clear@prefixes
  {%
    \let\ekvd@long\ekv@empty
    \let\ekvd@prot\ekv@empty
    \let\ekvd@ifalso\@secondoftwo
    \long\def\ekvd@ifnew##1##2##3{##3}%
  }
\ekvd@clear@prefixes
%    \end{macrocode}
% \end{macro}
%
% \begin{macro}{\ekvdefinekeys}
% This is the one front-facing macro which provides the interface to define
% keys. It's using |\ekvparse| to handle the \kv\ list, the interpretation will
% be done by |\ekvd@noarg| and |\ekvd@|. The \set\ for which the keys should be
% defined is stored in |\ekvd@set|.
%    \begin{macrocode}
\protected\def\ekvdefinekeys#1%
  {%
    \def\ekvd@set{#1}%
    \ekvparse\ekvd@noarg\ekvd@arg
  }
%    \end{macrocode}
% \end{macro}
%
% \begin{macro}[internal]{\ekvd@noarg,\ekvd@arg,\ekvd@handle}
% |\ekvd@noarg| and |\ekvd@arg| store whether there was a value in the \kv\
% pair. |\ekvd@handle| has to test whether there is a space inside the key and
% if so calls the prefix grabbing routine, else we throw an error and ignore the
% key.
%    \begin{macrocode}
\protected\long\def\ekvd@noarg#1%
  {%
    \let\ekvd@ifnoarg\@firstoftwo
    \expandafter\ekvd@handle\detokenize{#1}\ekvd@stop{}%
  }
\protected\long\def\ekvd@arg#1%
  {%
    \let\ekvd@ifnoarg\@secondoftwo
    \expandafter\ekvd@handle\detokenize{#1}\ekvd@stop
  }
\protected\long\def\ekvd@handle#1\ekvd@stop#2%
  {%
    \ekvd@clear@prefixes
    \def\ekvd@cur{#1}%
    \ekvd@ifspace{#1}%
      {\ekvd@prefix\ekv@mark#1\ekv@stop{#2}}%
      \ekvd@err@missing@type
  }
%    \end{macrocode}
% \end{macro}
%
% \begin{macro}[internal]{\ekvd@prefix,\ekvd@prefix@}
% \expkvd\ separates prefixes into two groups, the first being prefixes in the
% \TeX\ sense (|long| and |protected|) which use |@p@| in their name, the other
% being key-types (|code|, |int|, \emph{etc.}) which use |@t@| instead.
% |\ekvd@prefix| splits at the first space and checks whether its a |@p@|
% or |@t@| type prefix. If it is neither throw an error and gobble the
% definition (the value).
%    \begin{macrocode}
\protected\def\ekvd@prefix#1 {\ekv@strip{#1}\ekvd@prefix@\ekv@mark}
\protected\def\ekvd@prefix@#1#2\ekv@stop
  {%
    \ekv@ifdefined{ekvd@t@#1}%
      {\ekv@strip{#2}{\csname ekvd@t@#1\endcsname}}%
      {%
        \ekv@ifdefined{ekvd@p@#1}%
          {\csname ekvd@p@#1\endcsname\ekvd@prefix@after@p{#2}}%
          {\ekvd@err@undefined@prefix{#1}\@gobble}%
      }%
  }
%    \end{macrocode}
% \end{macro}
%
% \begin{macro}[internal]{\ekvd@prefix@after@p}
% The |@p@| type prefixes are all just modifying a following |@t@| type, so they
% will need to search for another prefix. This is true for all of them, so we
% use a macro to handle this. It'll throw an error if there is no other prefix.
%    \begin{macrocode}
\protected\def\ekvd@prefix@after@p#1%
  {%
    \ekvd@ifspace{#1}%
      {\ekvd@prefix#1\ekv@stop}%
      {\ekvd@err@missing@type\@gobble}%
  }
%    \end{macrocode}
% \end{macro}
%
% \begin{macro}[internal]
%   {\ekvd@p@long,\ekvd@p@protected,\ekvd@p@protect,\ekvd@p@also,\ekvd@p@new}
% Define the |@p@| type prefixes, they all just store some information in a
% temporary macro.
%    \begin{macrocode}
\protected\def\ekvd@p@long{\let\ekvd@long\long}
\protected\def\ekvd@p@protected{\let\ekvd@prot\protected}
\let\ekvd@p@protect\ekvd@p@protected
\protected\def\ekvd@p@also{\let\ekvd@ifalso\@firstoftwo}
\protected\def\ekvd@p@new{\let\ekvd@ifnew\ekvd@assert@new}
%    \end{macrocode}
% \end{macro}
%
% \subsubsection{Key Types}
%
% \begin{macro}[internal]{\ekvd@type@set,\ekvd@t@set}
% The |set| type is quite straight forward, just define a |NoVal| key to call
% |\ekvchangeset|.
%    \begin{macrocode}
\protected\def\ekvd@type@set#1#2%
  {%
    \ekvd@assert@not@long
    \ekvd@assert@not@protected
    \ekvd@ifnew{NoVal}{#1}%
      {%
        \ekv@ifempty{#2}%
          {\ekvd@err@missing@definition}%
          {%
            \ekvd@ifalso
              {%
                \ekv@expargtwice{\ekvd@add@noval{#1}}%
                  {\ekvchangeset{#2}}%
                  \ekvd@assert@not@protected@also
              }%
              {\ekv@expargtwice{\ekvdefNoVal\ekvd@set{#1}}{\ekvchangeset{#2}}}%
          }%
      }%
  }
\protected\def\ekvd@t@set#1#2%
  {%
    \ekvd@ifnoarg
      {\ekvd@type@set{#1}{#1}}%
      {\ekvd@type@set{#1}{#2}}%
  }
%    \end{macrocode}
% \end{macro}
%
% \begin{macro}[internal]{\ekvd@type@noval,\ekvd@t@noval,\ekvd@t@enoval}
% Another pretty simple type, |noval| just needs to assert that there is a
% definition and that |long| wasn't specified.
% There are types where the difference in the variants is so small, that we
% define a common handler for them, those common handlers are named with
% |@type@|. |noval| and |enoval| are so similar that we can use such a |@type@|
% macro, even if we could've done |noval| in a slightly faster way without it.
%    \begin{macrocode}
\protected\long\def\ekvd@type@noval#1#2#3%
  {%
    \ekvd@ifnew{NoVal}{#2}%
      {%
        \ekvd@assert@arg
          {%
            \ekvd@assert@not@long
            \ekvd@prot#1\ekvd@tmp{#3}%
            \ekvd@ifalso
              {\ekv@exparg{\ekvd@add@noval{#2}}\ekvd@tmp{}}%
              {\ekvletNoVal\ekvd@set{#2}\ekvd@tmp}%
          }%
      }%
  }
\protected\def\ekvd@t@noval{\ekvd@type@noval\def}
\protected\def\ekvd@t@enoval{\ekvd@type@noval\edef}
%    \end{macrocode}
% \end{macro}
%
% \begin{macro}[internal]{\ekvd@type@code,\ekvd@t@code,\ekvd@t@ecode}
% |code| is simple as well, |ecode| has to use |\edef| on a temporary macro,
% since \expkv\ doesn't provide an |\ekvedef|.
%    \begin{macrocode}
\protected\long\def\ekvd@type@code#1#2#3%
  {%
    \ekvd@ifnew{}{#2}%
      {%
        \ekvd@assert@arg
          {%
            \ekvd@prot\ekvd@long#1\ekvd@tmp##1{#3}%
            \ekvd@ifalso
              {\ekv@exparg{\ekvd@add@val{#2}}{\ekvd@tmp{##1}}{}}%
              {\ekvlet\ekvd@set{#2}\ekvd@tmp}%
          }%
      }%
  }
\protected\def\ekvd@t@code{\ekvd@type@code\def}
\protected\def\ekvd@t@ecode{\ekvd@type@code\edef}
%    \end{macrocode}
% \end{macro}
%
% \begin{macro}[internal]
%   {
%     \ekvd@type@default,\ekvd@t@default,\ekvd@t@qdefault,\ekvd@t@odefault,
%     \ekvd@t@fdefault
%   }
% |\ekvd@type@default| asserts there was an argument, also the key for which one
% wants to set a default has to be already defined (this is not so important for
% |default|, but |qdefault| requires is). If everything is good, |\edef| a
% temporary macro that expands |\ekvd@set| and the |\csname| for the key, and in
% the case of |qdefault| does the first expansion step of the key-macro.
%    \begin{macrocode}
\protected\long\def\ekvd@type@default#1#2#3#4%
  {%
    \ekvd@assert@arg
      {%
        \ekvifdefined\ekvd@set{#3}%
          {%
            \ekvd@assert@not@new
            \ekvd@assert@not@long
            \ekvd@prot\edef\ekvd@tmp
              {%
                \ekv@unexpanded\expandafter#1%
                  {#2\csname\ekv@name\ekvd@set{#3}\endcsname{#4}}%
              }%
            \ekvd@ifalso
              {\ekv@exparg{\ekvd@add@noval{#3}}\ekvd@tmp{}}%
              {\ekvletNoVal\ekvd@set{#3}\ekvd@tmp}%
          }%
          {\ekvd@err@undefined@key{#3}}%
      }%
  }
\protected\def\ekvd@t@default{\ekvd@type@default{}{}}
\protected\def\ekvd@t@qdefault{\ekvd@type@default{\expandafter\expandafter}{}}
\let\ekvd@t@odefault\ekvd@t@qdefault
\protected\def\ekvd@t@fdefault{\ekvd@type@default{}{\romannumeral`\^^@}}
%    \end{macrocode}
% \end{macro}
%
% \begin{macro}[internal]{\ekvd@t@edefault}
% |edefault| is too different from |default| and |qdefault| to reuse the
% |@type@| macro, as it doesn't need |\unexpanded| inside of |\edef|.
%    \begin{macrocode}
\protected\long\def\ekvd@t@edefault#1#2%
  {%
    \ekvd@assert@arg
      {%
        \ekvifdefined\ekvd@set{#1}%
          {%
            \ekvd@assert@not@new
            \ekvd@assert@not@long
            \ekvd@prot\edef\ekvd@tmp
              {\csname\ekv@name\ekvd@set{#1}\endcsname{#2}}%
            \ekvd@ifalso
              {\ekv@exparg{\ekvd@add@noval{#1}}\ekvd@tmp{}}%
              {\ekvletNoVal\ekvd@set{#1}\ekvd@tmp}%
          }%
          {\ekvd@err@undefined@key{#1}}%
      }%
  }
%    \end{macrocode}
% \end{macro}
%
% \begin{macro}[internal]
%   {\ekvd@t@initial,\ekvd@t@oinitial,\ekvd@t@finitial,\ekvd@t@einitial}
%    \begin{macrocode}
\long\def\ekvd@type@initial#1#2#3#4%
  {%
    \ekvd@assert@not@new
    \ekvd@assert@not@also
    \ekvd@assert@not@long
    \ekvd@assert@not@protected
    \ekvd@ifnoarg
      {%
        \ekvifdefinedNoVal\ekvd@set{#3}%
          {\csname\ekv@name\ekvd@set{#3}N\endcsname}%
          {\ekvd@err@undefined@noval{#3}}%
      }%
      {%
        \ekvifdefined\ekvd@set{#3}%
          {%
            #1{#2#4}%
            \csname\ekv@name\ekvd@set{#3}\expandafter\endcsname\expandafter
              {\ekvd@tmp}%
          }%
          {\ekvd@err@undefined@key{#3}}%
      }%
  }
\def\ekvd@t@initial{\ekvd@type@initial{\def\ekvd@tmp}{}}
\def\ekvd@t@oinitial{\ekvd@type@initial{\ekv@exparg{\def\ekvd@tmp}}{}}
\def\ekvd@t@einitial{\ekvd@type@initial{\edef\ekvd@tmp}{}}
\def\ekvd@t@finitial
  {\ekvd@type@initial{\ekv@exparg{\def\ekvd@tmp}}{\romannumeral`\^^@}}
%    \end{macrocode}
% \end{macro}
%
% \begin{macro}[internal]
%   {
%     \ekvd@type@bool,\ekvd@t@bool,\ekvd@t@gbool,\ekvd@t@boolTF,\ekvd@t@gboolTF,
%     \ekvd@t@invbool,\ekvd@t@ginvbool,\ekvd@t@invboolTF,\ekvd@t@ginvboolTF,
%   }
% The boolean types are a quicker version of a |choice| that accept |true| and
% |false|, and set up the |NoVal| action to be identical to \texttt{\key=true}.
% The |true| and |false| actions are always just |\let|ting the macro in |#7| to
% some other macro (\emph{e.g.}, \cs[no-index]{iftrue}).
%    \begin{macrocode}
\protected\def\ekvd@type@bool#1#2#3#4#5%
  {%
    \ekvd@ifnew{}{#4}%
      {%
        \ekvd@ifnew{NoVal}{#4}%
          {%
            \ekvd@assert@filledarg{#5}%
              {%
                \ekvd@newlet#5#3%
                \ekvd@type@choice{#4}%
                \protected\ekvdefNoVal\ekvd@set{#4}{#1\let#5#2}%
                \protected\expandafter\def
                  \csname\ekvd@choice@name\ekvd@set{#4}{true}\endcsname
                  {#1\let#5#2}%
                \protected\expandafter\def
                  \csname\ekvd@choice@name\ekvd@set{#4}{false}\endcsname
                  {#1\let#5#3}%
              }%
          }%
      }%
  }
\protected\def\ekvd@t@bool{\ekvd@type@bool{}\iftrue\iffalse}
\protected\def\ekvd@t@gbool{\ekvd@type@bool\global\iftrue\iffalse}
\protected\def\ekvd@t@boolTF{\ekvd@type@bool{}\@firstoftwo\@secondoftwo}
\protected\def\ekvd@t@gboolTF{\ekvd@type@bool\global\@firstoftwo\@secondoftwo}
\protected\def\ekvd@t@invbool{\ekvd@type@bool{}\iffalse\iftrue}
\protected\def\ekvd@t@ginvbool{\ekvd@type@bool\global\iffalse\iftrue}
\protected\def\ekvd@t@invboolTF{\ekvd@type@bool{}\@secondoftwo\@firstoftwo}
\protected\def\ekvd@t@ginvboolTF
  {\ekvd@type@bool\global\@secondoftwo\@firstoftwo}
%    \end{macrocode}
% \end{macro}
%
% \begin{macro}[internal]
%   {
%     \ekvd@type@boolpair,
%     \ekvd@t@boolpair,\ekvd@t@gboolpair,\ekvd@t@boolpairTF,\ekvd@t@gboolpairTF
%   }
% The boolean pair types are essentially the same as the boolean types, but set
% two macros instead of one.
%    \begin{macrocode}
\protected\def\ekvd@type@boolpair#1#2#3#4#5#6%
  {%
    \ekvd@ifnew{}{#4}%
      {%
        \ekvd@ifnew{NoVal}{#4}%
          {%
            \ekvd@newlet#5#3%
            \ekvd@newlet#6#2%
            \ekvd@type@choice{#4}%
            \protected\ekvdefNoVal\ekvd@set{#4}{#1\let#5#2#1\let#6#3}%
            \protected\expandafter\def
              \csname\ekvd@choice@name\ekvd@set{#4}{true}\endcsname
              {#1\let#5#2#1\let#6#3}%
            \protected\expandafter\def
              \csname\ekvd@choice@name\ekvd@set{#4}{false}\endcsname
              {#1\let#5#3#1\let#6#2}%
          }%
      }%
  }
\protected\def\ekvd@t@boolpair#1#2%
  {\ekvd@assert@twoargs{#2}{\ekvd@type@boolpair{}\iftrue\iffalse{#1}#2}}
\protected\def\ekvd@t@gboolpair#1#2%
  {\ekvd@assert@twoargs{#2}{\ekvd@type@boolpair\global\iftrue\iffalse{#1}#2}}
\protected\def\ekvd@t@boolpairTF#1#2%
  {%
    \ekvd@assert@twoargs{#2}%
      {\ekvd@type@boolpair{}\@firstoftwo\@secondoftwo{#1}#2}%
  }
\protected\def\ekvd@t@gboolpairTF#1#2%
  {%
    \ekvd@assert@twoargs{#2}%
      {\ekvd@type@boolpair\global\@firstoftwo\@secondoftwo{#1}#2}%
  }
%    \end{macrocode}
% \end{macro}
%
% \begin{macro}[internal]
%   {
%     \ekvd@type@data,\ekvd@t@data,\ekvd@t@gdata,\ekvd@t@dataT,\ekvd@t@gdataT
%   }
%    \begin{macrocode}
\protected\def\ekvd@type@data#1#2#3#4#5#6%
  {%
    \ekvd@ifnew{}{#5}%
      {%
        \ekvd@assert@filledarg{#6}%
          {%
            \ekvd@newlet#6#1%
            \ekvd@ifalso
              {%
                \let\ekvd@prot\protected
                \ekvd@add@val{#5}{\long#2#6####1#3{####1{#4}}}{}%
              }%
              {%
                \protected\ekvd@long\ekvdef\ekvd@set{#5}%
                  {\long#2#6####1#3{####1{#4}}}%
              }%
          }%
      }%
  }
\protected\def\ekvd@t@data
  {\ekvd@type@data\@secondoftwo\edef{####2}{\ekv@unexpanded{##1}}}
\protected\def\ekvd@t@edata{\ekvd@type@data\@secondoftwo\edef{####2}{##1}}
\protected\def\ekvd@t@gdata
  {\ekvd@type@data\@secondoftwo\xdef{####2}{\ekv@unexpanded{##1}}}
\protected\def\ekvd@t@xdata{\ekvd@type@data\@secondoftwo\xdef{####2}{##1}}
\protected\def\ekvd@t@dataT
  {\ekvd@type@data\@gobble\edef{}{\ekv@unexpanded{##1}}}
\protected\def\ekvd@t@edataT{\ekvd@type@data\@gobble\edef{}{##1}}
\protected\def\ekvd@t@gdataT
  {\ekvd@type@data\@gobble\xdef{}{\ekv@unexpanded{##1}}}
\protected\def\ekvd@t@xdataT{\ekvd@type@data\@gobble\xdef{}{##1}}
%    \end{macrocode}
% \end{macro}
%
% \begin{macro}[internal]{\ekvd@type@box,\ekvd@t@box,\ekvd@t@gbox}
% Set up our boxes. Though we're a generic package we want to be colour safe, so
% we put an additional grouping level inside the box contents, for the case that
% someone uses \pkg{color}. |\ekvd@newreg| is a small wrapper which tests
% whether the first argument is defined and if not does
% |\csname new#2\endcsname#1|.
%    \begin{macrocode}
\protected\def\ekvd@type@box#1#2#3%
  {%
    \ekvd@ifnew{}{#2}%
      {%
        \ekvd@assert@filledarg{#3}%
          {%
            \ekvd@newreg#3{box}%
            \ekvd@ifalso
              {%
                \let\ekvd@prot\protected
                \ekvd@add@val{#2}{#1\setbox#3=\hbox{\begingroup##1\endgroup}}{}%
              }%
              {%
                \protected\ekvd@long\ekvdef\ekvd@set{#2}%
                  {#1\setbox#3=\hbox{\begingroup##1\endgroup}}%
              }%
          }%
      }%
  }
\protected\def\ekvd@t@box{\ekvd@type@box{}}
\protected\def\ekvd@t@gbox{\ekvd@type@box\global}
%    \end{macrocode}
% \end{macro}
%
% \begin{macro}[internal]{\ekvd@type@toks,\ekvd@t@toks,\ekvd@t@gtoks}
% Similar to |box|, but set the |toks|.
%    \begin{macrocode}
\protected\def\ekvd@type@toks#1#2#3%
  {%
    \ekvd@ifnew{}{#2}%
      {%
        \ekvd@assert@filledarg{#3}%
          {%
            \ekvd@newreg#3{toks}%
            \ekvd@ifalso
              {%
                \let\ekvd@prot\protected
                \ekvd@add@val{#2}{#1#3={##1}}{}%
              }%
              {\protected\ekvd@long\ekvdef\ekvd@set{#2}{#1#3={##1}}}%
          }%
      }%
  }
\protected\def\ekvd@t@toks{\ekvd@type@toks{}}
\protected\def\ekvd@t@gtoks{\ekvd@type@toks\global}
%    \end{macrocode}
% \end{macro}
%
% \begin{macro}[internal]{\ekvd@type@preapptoks,\ekvd@t@apptoks,\ekvd@t@gapptoks}
% Just like |toks|, but expand the current contents of the |toks| register to
% append the new contents.
%    \begin{macrocode}
\ekvd@ifprimitive\toksapp
  {%
    \protected\def\ekvd@type@preapptoks#1#2#3%
      {%
        \ekvd@ifnew{}{#2}%
          {%
            \ekvd@assert@filledarg{#3}%
              {%
                \ekvd@newreg#3{toks}%
                \ekvd@ifalso
                  {%
                    \let\ekvd@prot\protected
                    \ekvd@add@val{#2}{#1#3{##1}}{}%
                  }%
                  {\protected\ekvd@long\ekvdef\ekvd@set{#2}{#1#3{##1}}}%
              }%
          }%
      }
    \protected\def\ekvd@t@apptoks{\ekvd@type@preapptoks\toksapp}
    \protected\def\ekvd@t@gapptoks{\ekvd@type@preapptoks\gtoksapp}
    \protected\def\ekvd@t@pretoks{\ekvd@type@preapptoks\tokspre}
    \protected\def\ekvd@t@gpretoks{\ekvd@type@preapptoks\gtokspre}
  }
  {%
    \protected\def\ekvd@type@apptoks#1#2#3%
      {%
        \ekvd@ifnew{}{#2}%
          {%
            \ekvd@assert@filledarg{#3}%
              {%
                \ekvd@newreg#3{toks}%
                \ekvd@ifalso
                  {%
                    \let\ekvd@prot\protected
                    \ekvd@add@val{#2}{#1#3=\expandafter{\the#3##1}}{}%
                  }%
                  {%
                    \protected\ekvd@long\ekvdef\ekvd@set{#2}%
                      {#1#3=\expandafter{\the#3##1}}%
                  }%
              }%
          }%
      }
    \protected\def\ekvd@t@apptoks{\ekvd@type@apptoks{}}
    \protected\def\ekvd@t@gapptoks{\ekvd@type@apptoks\global}
    \newtoks\ekvd@toks
    \protected\def\ekvd@type@pretoks#1#2#3%
      {%
        \ekvd@ifnew{}{#2}%
          {%
            \ekvd@assert@filledarg{#3}%
              {%
                \ekvd@newreg#3{toks}%
                \ekvd@ifalso
                  {%
                    \let\ekvd@prot\protected
                    \ekvd@add@val{#2}%
                      {%
                        \ekvd@toks={##1}%
                        #1#3=\expandafter{\the\expandafter\ekvd@toks\the#3}%
                      }%
                      {}%
                  }%
                  {%
                    \protected\ekvd@long\ekvdef\ekvd@set{#2}%
                      {%
                        \ekvd@toks={##1}%
                        #1#3=\expandafter{\the\expandafter\ekvd@toks\the#3}%
                      }%
                  }%
              }%
          }%
      }
    \protected\def\ekvd@t@pretoks{\ekvd@type@pretoks{}}
    \protected\def\ekvd@t@gpretoks{\ekvd@type@pretoks\global}
  }
%    \end{macrocode}
% \end{macro}
%
% \begin{macro}[internal]
%   {
%     \ekvd@type@reg,\ekvd@t@int,\ekvd@t@eint,\ekvd@t@gint,\ekvd@t@xint,
%     \ekvd@t@dimen,\ekvd@t@edimen,\ekvd@t@gdimen,\ekvd@t@xdimen,
%     \ekvd@t@skip,\ekvd@t@eskip,\ekvd@t@gskip,\ekvd@t@xskip
%   }
% The |\ekvd@type@reg| can handle all the types for which the assignment will
% just be \texttt{\meta{register}=\meta{value}}.
%    \begin{macrocode}
\protected\def\ekvd@type@reg#1#2#3#4#5#6%
  {%
    \ekvd@ifnew{}{#5}%
      {%
        \ekvd@assert@filledarg{#6}%
          {%
            \ekvd@newreg#6{#1}%
            \ekvd@ifalso
              {%
                \let\evkd@prot\protected
                \ekvd@add@val{#5}{#2#6=#3##1#4\relax}{}%
              }%
              {\protected\ekvd@long\ekvdef\ekvd@set{#5}{#2#6=#3##1#4\relax}}%
          }%
      }%
  }
\protected\def\ekvd@t@int{\ekvd@type@reg{count}{}{}{}}
\protected\def\ekvd@t@eint{\ekvd@type@reg{count}{}\numexpr\relax}
\protected\def\ekvd@t@gint{\ekvd@type@reg{count}\global{}{}}
\protected\def\ekvd@t@xint{\ekvd@type@reg{count}\global\numexpr\relax}
\protected\def\ekvd@t@dimen{\ekvd@type@reg{dimen}{}{}{}}
\protected\def\ekvd@t@edimen{\ekvd@type@reg{dimen}{}\dimexpr\relax}
\protected\def\ekvd@t@gdimen{\ekvd@type@reg{dimen}\global{}{}}
\protected\def\ekvd@t@xdimen{\ekvd@type@reg{dimen}\global\dimexpr\relax}
\protected\def\ekvd@t@skip{\ekvd@type@reg{skip}{}{}{}}
\protected\def\ekvd@t@eskip{\ekvd@type@reg{skip}{}\glueexpr\relax}
\protected\def\ekvd@t@gskip{\ekvd@type@reg{skip}\global{}{}}
\protected\def\ekvd@t@xskip{\ekvd@type@reg{skip}\global\glueexpr\relax}
%    \end{macrocode}
% \end{macro}
%
% \begin{macro}[internal]{\ekvd@type@store,\ekvd@t@store,\ekvd@t@gstore}
% The none-expanding |store| types use an |\edef| or |\xdef| and |\unexpanded|
% to be able to also store |#| easily.
%    \begin{macrocode}
\protected\def\ekvd@type@store#1#2#3#4%
  {%
    \ekvd@ifnew{}{#3}%
      {%
        \ekvd@assert@filledarg{#4}%
          {%
            \ekvd@newlet#4\ekv@empty
            \ekvd@ifalso
              {%
                \let\ekvd@prot\protected
                \ekvd@add@val{#3}{#1#4{#2}}{}%
              }%
              {\protected\ekvd@long\ekvdef\ekvd@set{#3}{#1#4{#2}}}%
          }%
      }%
  }
\protected\def\ekvd@t@store{\ekvd@type@store\edef{\ekv@unexpanded{##1}}}
\protected\def\ekvd@t@gstore{\ekvd@type@store\xdef{\ekv@unexpanded{##1}}}
\protected\def\ekvd@t@estore{\ekvd@type@store\edef{##1}}
\protected\def\ekvd@t@xstore{\ekvd@type@store\xdef{##1}}
%    \end{macrocode}
% \end{macro}
%
% \begin{macro}[internal]
%   {
%     \ekvd@type@meta,\ekvd@type@meta@a,\ekvd@type@meta@b,\ekvd@type@meta@c,
%     \ekvd@t@meta,\ekvd@t@nmeta
%   }
% |meta| sets up things such that another instance of |\ekvset| will be run on
% the argument, with the same \set.
%    \begin{macrocode}
\protected\long\def\ekvd@type@meta#1#2#3#4#5#6#7%
  {%
    \ekvd@ifnew{#1}{#6}%
      {%
        \ekvd@assert@filledarg{#7}%
          {%
            \edef\ekvd@tmp{\ekvd@set}%
            \expandafter\ekvd@type@meta@a\expandafter{\ekvd@tmp}{#7}{#2}%
            \ekvd@ifalso
              {\ekv@exparg{#3{#6}}{\ekvd@tmp#4}{#5}}%
              {\csname ekvlet#1\endcsname\ekvd@set{#6}\ekvd@tmp}%
          }%
      }%
  }
\protected\long\def\ekvd@type@meta@a#1#2%
  {%
    \expandafter\ekvd@type@meta@b\expandafter{\ekvset{#1}{#2}}%
  }
\protected\def\ekvd@type@meta@b
  {%
    \expandafter\ekvd@type@meta@c\expandafter
  }
\protected\long\def\ekvd@type@meta@c#1#2%
  {%
    \ekvd@prot\ekvd@long\def\ekvd@tmp#2{#1}%
  }
\protected\def\ekvd@t@meta{\ekvd@type@meta{}{##1}\ekvd@add@val{{##1}}{}}
\protected\def\ekvd@t@nmeta
  {%
    \ekvd@assert@not@long
    \ekvd@type@meta{NoVal}{}\ekvd@add@noval{}\ekvd@assert@not@long@also
  }
%    \end{macrocode}
% \end{macro}
%
% \begin{macro}[internal]
%   {\ekvd@type@smeta,\ekvd@type@smeta@,\ekvd@t@smeta,\ekvd@t@snmeta}
% |smeta| is pretty similar to |meta|, but needs two arguments inside of \val,
% such that the first is the \set\ for which the sub-|\ekvset| and the second is
% the \kv\ list.
%    \begin{macrocode}
\protected\long\def\ekvd@type@smeta#1#2#3#4#5#6#7%
  {%
    \ekvd@ifnew{#1}{#6}%
      {%
        \ekvd@assert@twoargs{#7}%
          {%
            \ekvd@type@meta@a#7{#2}%
            \ekvd@ifalso
              {\ekv@exparg{#3{#6}}{\ekvd@tmp#4}{#5}}%
              {\csname ekvlet#1\endcsname\ekvd@set{#6}\ekvd@tmp}%
          }%
      }%
  }
\protected\def\ekvd@t@smeta{\ekvd@type@smeta{}{##1}\ekvd@add@val{{##1}}{}}
\protected\def\ekvd@t@snmeta
  {%
    \ekvd@assert@not@long
    \ekvd@type@smeta{NoVal}{}\ekvd@add@noval{}\ekvd@assert@not@long@also
  }
%    \end{macrocode}
% \end{macro}
%
% \begin{macro}[internal]
%   {
%     \ekvd@type@choice,
%     \ekvd@populate@choice,\ekvd@populate@choice@,\ekvd@populate@choice@noarg,
%     \ekvd@choice@prefix,\ekvd@choice@prefix@,\ekvd@choice@prefix@done,
%     \ekvd@choice@p@protected,\ekvd@choice@p@protect,
%     \ekvd@choice@p@long,\ekvd@choice@p@long@,
%     \ekvd@t@choice
%   }
% The |choice| type is by far the most complex type, as we have to run a
% sub-parser on the choice-definition list, which should support the |@p@| type
% prefixes as well (but |long| will always throw an error, as they are not
% allowed to be long). |\ekvd@type@choice| will just define the choice-key, the
% handling of the choices definition will be done by |\ekvd@populate@choice|.
%    \begin{macrocode}
\protected\def\ekvd@type@choice#1%
  {%
    \ekvd@assert@not@long
    \ekv@exparg{\ekvd@prot\def\ekvd@tmp##1}%
      {%
        \expandafter\ekvd@h@choice\expandafter
          {\ekvd@choice@name\ekvd@set{#1}{##1}}%
      }%
    \ekvd@ifalso
      {%
        \ekvd@assert@val{#1}%
          {%
            \ekvd@if@not@already@choice{#1}%
              {%
                \ekv@exparg
                  {%
                    \expandafter\ekvd@add@aux
                      \csname\ekv@name\ekvd@set{#1}\endcsname{{##1}}{#1}%
                  }%
                  {\ekvd@tmp{##1}}%
                  {\ekvd@long\ekvdef}\ekvd@assert@not@long@also
              }%
          }%
      }%
      {\ekvlet\ekvd@set{#1}\ekvd@tmp}%
  }
%    \end{macrocode}
% |\ekvd@populate@choice| just uses |\ekvparse| and then gives control to
% |\ekvd@populate@choice@noarg|, which throws an error, and
% |\ekvd@populate@choice@|.
%    \begin{macrocode}
\protected\def\ekvd@populate@choice
  {%
    \ekvparse\ekvd@populate@choice@noarg\ekvd@populate@choice@
  }
\protected\long\def\ekvd@populate@choice@noarg#1%
  {%
    \expandafter\ekvd@err@missing@definition@msg\expandafter{\ekvd@cur : #1}%
  }
%    \end{macrocode}
% |\ekvd@populate@choice@| runs the prefix-test, if there is none we can
% directly define the choice, for that |\ekvd@set@choice| will expand to the
% current choice-key's name, which will have been defined by |\ekvd@t@choice|.
% If there is a prefix run the prefix grabbing routine, which was altered for
% |@type@choice|.
%    \begin{macrocode}
\protected\long\def\ekvd@populate@choice@#1#2%
  {%
    \ekvd@clear@prefixes
    \ekvd@ifspace{#1}%
      {\ekvd@choice@prefix{\ekv@mark#1}\ekv@mark#1\ekv@stop}%
      {%
        \expandafter\edef
          \csname\ekvd@choice@name\ekvd@set\ekvd@set@choice{#1}\endcsname
      }%
      {\unexpanded{#2}}%
  }
\protected\def\ekvd@choice@prefix#1#2
  {%
    \ekv@strip{#2}{\ekvd@choice@prefix@{#1}}\ekv@mark
  }
\protected\def\ekvd@choice@prefix@#1#2#3\ekv@stop
  {%
    \ekv@ifdefined{ekvd@choice@p@#2}%
      {%
        \csname ekvd@choice@p@#2\endcsname
        \ekvd@ifspace{#3}%
          {\ekvd@choice@prefix{#3}#3\ekv@stop}%
          {\ekvd@choice@prefix@done{#3}}%
      }%
      {\ekvd@choice@prefix@done{#1}}%
  }
\protected\def\ekvd@choice@prefix@done#1%
  {%
    \ekvd@prot\expandafter\edef
      \csname
        \ekv@strip{#1}{\ekvd@choice@name\ekvd@set\ekvd@set@choice}%
      \endcsname
  }
\protected\def\ekvd@choice@p@protected{\let\ekvd@prot\protected}
\let\ekvd@choice@p@protect\ekvd@choice@p@protected
\protected\def\ekvd@choice@invalid@p#1\ekvd@ifspace#2%
  {%
    \expandafter\ekvd@choice@invalid@p@\expandafter{\ekv@gobble@mark#2}{#1}%
    \ekvd@ifspace{#2}%
  }
\protected\def\ekvd@choice@invalid@p@#1#2%
  {%
    \expandafter\ekvd@err@no@prefix@msg\expandafter{\ekvd@cur : #2 #1}{#2}%
  }
\protected\def\ekvd@choice@p@long{\ekvd@choice@invalid@p{long}}%
\protected\def\ekvd@choice@p@also{\ekvd@choice@invalid@p{also}}%
\protected\def\ekvd@choice@p@new{\ekvd@choice@invalid@p{new}}%
%    \end{macrocode}
% Finally we're able to set up the |@t@choice| macro, which has to store the
% current choice-key's name, define the key, and parse the available choices.
%    \begin{macrocode}
\protected\long\def\ekvd@t@choice#1#2%
  {%
    \ekvd@ifnew{}{#1}%
      {%
        \ekvd@assert@arg
          {%
            \ekvd@type@choice{#1}%
            \def\ekvd@set@choice{#1}%
            \ekvd@populate@choice{#2}%
          }%
      }%
  }
%    \end{macrocode}
% \end{macro}
%
% \begin{macro}[internal]{\ekvd@t@choice-store,\ekvd@t@choice-enum}
%   These two types define a special kind of |choice| key and are quite similar,
%   the only difference is what the different choices do (hence they use a
%   shared initialisation which differs in the chosen |populate| step).
%    \begin{macrocode}
\protected\long\expandafter\def\csname ekvd@t@choice-store\endcsname
  {\ekvd@type@choicespecial\ekvd@populate@choicestore}
\protected\long\expandafter\def\csname ekvd@t@choice-enum\endcsname
  {\ekvd@type@choicespecial\ekvd@populate@choiceenum}
%    \end{macrocode}
% \begin{macro}[internal]{\ekvd@type@choicespecial}
%   Initialise similar to a |choice| key. The difference is that we require two
%   arguments (which we assert), a macro to store things in, and a |csv|-list
%   containing the allowed values. |#1| is the |populate| macro according to the
%   type used.
%    \begin{macrocode}
\protected\long\def\ekvd@type@choicespecial#1#2#3%
  {%
    \ekvd@ifnew{}{#2}%
      {%
        \ekvd@assert@twoargs{#3}%
          {%
            \ekvd@type@choice{#2}%
            \def\ekvd@set@choice{#2}%
            #1#3%
          }%
      }%
  }
%    \end{macrocode}
% \end{macro}
% \begin{macro}[internal]
%   {\ekvd@populate@choicestore,\ekvd@populate@choicestore@}
%   We initialise the storing macro if it doesn't yet exist, and then we loop
%   over the value list. The |\edef|s with |\unexpanded| are both necessary to
%   be able to store macro parameter tokens (the outer protects at define time,
%   the inner at use time).
%    \begin{macrocode}
\protected\long\def\ekvd@populate@choicestore#1%
  {%
    \ekvd@newlet#1\ekv@empty
    \ekvcsvloop{\ekvd@populate@choicestore@#1}%
  }
\protected\long\def\ekvd@populate@choicestore@#1#2%
  {%
    \protected\expandafter\edef
      \csname\ekvd@choice@name\ekvd@set\ekvd@set@choice{#2}\endcsname
      {\unexpanded{\edef#1{\unexpanded{#2}}}}%
  }
%    \end{macrocode}
% \end{macro}
% \begin{macro}[internal]
%   {\ekvd@populate@choiceenum,\ekvd@populate@choiceenum@}
%   This is similar to the population of a |choice-store| type, but instead of
%   storing the values in a macro this initialises a count and stores the
%   position of the value in the list inside that count (zero-indexed). The
%   space is necessary to terminate the number scanning, which is the reason we
%   use |\@firstofone| (so that the space after the macro name isn't gobbled by
%   \TeX).
%    \begin{macrocode}
\protected\long\def\ekvd@populate@choiceenum#1%
  {%
    \ekvd@newreg#1{count}%
    \def\ekvd@tmp{0}%
    \ekvcsvloop{\ekvd@populate@choiceenum@#1}%
  }
\protected\long\def\ekvd@populate@choiceenum@#1#2%
  {%
    \protected\expandafter\edef
      \csname\ekvd@choice@name\ekvd@set\ekvd@set@choice{#2}\endcsname
      {#1=\@firstofone{\ekvd@tmp} }%
    \edef\ekvd@tmp{\the\numexpr\ekvd@tmp+1\relax}%
  }
%    \end{macrocode}
% \end{macro}
% \end{macro}
%
% \begin{macro}[internal]{\ekvd@t@unknown-choice}
%    \begin{macrocode}
\protected\long\expandafter\def\csname ekvd@t@unknown-choice\endcsname#1#2%
  {%
    \ekvd@assert@new@for@name{\ekvd@unknown@choice@name\ekvd@set{#1}}%
      {%
        \ekvd@assert@arg
          {%
            \ekvd@assert@not@long
            \ekvd@assert@not@also
            \ekvd@prot\expandafter
            \def\csname\ekvd@unknown@choice@name\ekvd@set{#1}\endcsname##1{#2}%
          }%
      }%
  }
%    \end{macrocode}
% \end{macro}
%
% \begin{macro}[internal]
%   {\ekvd@t@unknown,\ekvd@type@unknown@code,\ekvd@type@unknown@noval}
%   The |unknown| type has different subtypes which would be the key names for
%   other types. It is first checked whether that subtype is defined, if it
%   isn't throw an error, else use that subtype.
%    \begin{macrocode}
\protected\long\def\ekvd@t@unknown#1#2%
  {%
    \ekv@ifdefined{ekvd@type@unknown@\detokenize{#1}}%
      {\csname ekvd@type@unknown@\detokenize{#1}\endcsname{#2}}%
      \ekvd@err@misused@unknown
  }
%    \end{macrocode}
%   The |unknown noval| type can use |\ekvdefunknownNoVal| directly (after
%   asserting some prefixes).
%    \begin{macrocode}
\protected\long\def\ekvd@type@unknown@noval#1%
  {%
    \ekvd@assert@new@for@name{\ekv@name\ekvd@set{}uN}%
      {%
        \ekvd@assert@arg
          {%
            \ekvd@assert@not@also
            \ekvd@assert@not@long
            \ekvd@prot\ekvdefunknownNoVal\ekvd@set{#1}%
          }%
      }%
  }
%    \end{macrocode}
%   The |unknown code| type uses some trickery during the definition in order to
%   swap out |#1| and |#2| in the user supplied definition. This is done via a
%   temporary macro that stores the definition but gets the parameter numbers
%   reversed while the real definition is done.
%    \begin{macrocode}
\protected\long\def\ekvd@type@unknown@code#1%
  {%
    \ekvd@assert@new@for@name{\ekv@name\ekvd@set{}u}%
      {%
        \ekvd@assert@arg
          {%
            \ekvd@assert@not@also
            \begingroup
              \def\ekvd@tmp##1##2{#1}%
              \ekv@exparg
              {%
                \endgroup
                \ekvd@prot\ekvd@long\ekvdefunknown\ekvd@set
              }%
              {\ekvd@tmp{##2}{##1}}%
          }%
      }%
  }
%    \end{macrocode}
% \end{macro}
%
% \begin{macro}[internal]
%   {
%     \ekvd@type@unknown@redirect,
%     \ekvd@type@unknown@redirect-code,
%     \ekvd@type@unknown@redirect-noval
%   }
%   The |unknown redirect| types also just forward to |\ekvredirectunknown|
%   after asserting some prefixes.
%    \begin{macrocode}
\protected\edef\ekvd@type@unknown@redirect#1%
  {%
    \expandafter\noexpand\csname ekvd@type@unknown@redirect-code\endcsname{#1}%
    \expandafter\noexpand\csname ekvd@type@unknown@redirect-noval\endcsname{#1}%
  }
\protected\expandafter\def\csname ekvd@type@unknown@redirect-code\endcsname#1%
  {%
    \ekvd@assert@new@for@name{\ekv@name\ekvd@set{}u}%
      {%
        \ekvd@assert@arg
          {%
            \ekvd@assert@not@also
            \ekvd@assert@not@protected
            \expandafter\ekvredirectunknown\expandafter{\ekvd@set}{#1}%
          }%
      }%
  }
\protected\expandafter\def\csname ekvd@type@unknown@redirect-noval\endcsname#1%
  {%
    \ekvd@assert@new@for@name{\ekv@name\ekvd@set{}uN}%
      {%
        \ekvd@assert@arg
          {%
            \ekvd@assert@not@also
            \ekvd@assert@not@protected
            \ekvd@assert@not@long
            \expandafter\ekvredirectunknownNoVal\expandafter{\ekvd@set}{#1}%
          }%
      }%
  }
%    \end{macrocode}
% \end{macro}
%
% \subsubsection{Key Type Helpers}
% There are some keys that might need helpers during their execution (not during
% their definition, which are gathered as |@type@| macros). These helpers are
% named |@h@|.
%
% \begin{macro}[internal]{\ekvd@h@choice,\ekvd@h@choice@}
% The |choice| helper will just test whether the given choice was defined, if
% not throw an error expandably, else call the macro which stores the code for
% this choice.
%    \begin{macrocode}
\def\ekvd@h@choice#1%
  {%
    \expandafter\ekvd@h@choice@
      \csname\ifcsname#1\endcsname#1\else relax\fi\endcsname
      {#1}%
  }
\def\ekvd@h@choice@#1#2%
  {%
    \ifx#1\relax
      \ekvd@err@choice@invalid{#2}%
      \expandafter\@gobble
    \fi
    #1%
  }
%    \end{macrocode}
% \end{macro}
%
%
% \subsubsection{Handling \texttt{also}}
%
% \begin{macro}[internal]
%   {\ekvd@add@val,\ekvd@add@noval,\ekvd@add@aux,\ekvd@add@aux@}
%    \begin{macrocode}
\protected\long\def\ekvd@add@val#1#2#3%
  {%
    \ekvd@assert@val{#1}%
      {%
        \expandafter\ekvd@add@aux\csname\ekv@name\ekvd@set{#1}\endcsname{{##1}}%
          {#1}{#2}{\ekvd@long\ekvdef}{#3}%
      }%
  }
\protected\long\def\ekvd@add@noval#1#2#3%
  {%
    \ekvd@assert@noval{#1}%
      {%
        \expandafter\ekvd@add@aux\csname\ekv@name\ekvd@set{#1}N\endcsname{}%
          {#1}{#2}\ekvdefNoVal{#3}%
      }%
  }
\protected\long\def\ekvd@add@aux#1#2%
  {%
    \ekvd@extract@prefixes#1%
    \expandafter\ekvd@add@aux@\expandafter{#1#2}%
  }
\protected\long\def\ekvd@add@aux@#1#2#3#4#5%
  {%
    #5%
    \ekvd@prot#4\ekvd@set{#2}{#1#3}%
  }
%    \end{macrocode}
% \end{macro}
%
% \begin{macro}[internal]
%   {
%     \ekvd@extract@prefixes,\ekvd@extract@prefixes@,
%     \ekvd@extract@prefixes@long,\ekvd@extract@prefixes@prot
%   }
% This macro checks which prefixes were used for the definition of a macro and
% sets |\ekvd@long| and |\ekvd@prot| accordingly.
%    \begin{macrocode}
\protected\def\ekvd@extract@prefixes#1%
  {%
    \expandafter\ekvd@extract@prefixes@\meaning#1\ekvd@stop
  }
%    \end{macrocode}
% In the following definition |#1| will get replaced by |macro:|, |#2| by
% |\long| and |#3| by |\protected| (in each, all tokens will have category
% other). This allows us to parse the |\meaning| of a macro for those strings.
%    \begin{macrocode}
\protected\def\ekvd@extract@prefixes@#1#2#3%
  {%
    \protected\def\ekvd@extract@prefixes@##1#1##2\ekvd@stop
      {%
        \ekvd@extract@prefixes@long
          ##1\ekvd@mark\@firstofone#2\ekvd@mark\@gobble\ekvd@stop
          {\let\ekvd@long\long}%
        \ekvd@extract@prefixes@prot
          ##1\ekvd@mark\@firstofone#3\ekvd@mark\@gobble\ekvd@stop
          {\let\ekvd@prot\protected}%
      }%
    \protected\def\ekvd@extract@prefixes@long##1#2##2\ekvd@mark##3##4\ekvd@stop
      {##3}%
    \protected\def\ekvd@extract@prefixes@prot##1#3##2\ekvd@mark##3##4\ekvd@stop
      {##3}%
  }
%    \end{macrocode}
% We use a temporary macro to expand the three arguments of
% |\ekvd@extract@prefixes@|, which will set up the real meaning of itself and
% the parsing for |\long| and |\protected|.
%    \begin{macrocode}
\begingroup
\edef\ekvd@tmp
  {%
    \endgroup
    \ekvd@extract@prefixes@
      {\detokenize{macro:}}%
      {\string\long}%
      {\string\protected}%
  }
\ekvd@tmp
%    \end{macrocode}
% \end{macro}
%
%
% \subsubsection{Tests}
%
% \begin{macro}[internal]{\ekvd@newlet,\ekvd@newreg}
% These macros test whether a control sequence is defined, if it isn't they
% define it, either via |\let| or via the correct \cs[no-index]{new\meta{reg}}.
%    \begin{macrocode}
\protected\def\ekvd@newlet#1#2%
  {%
    \ifdefined#1\ekv@fi@gobble\fi\@firstofone{\let#1#2}%
  }
\protected\def\ekvd@newreg#1#2%
  {%
    \ifdefined#1\ekv@fi@gobble\fi\@firstofone{\csname new#2\endcsname#1}%
  }
%    \end{macrocode}
% \end{macro}
%
% \begin{macro}[internal]
%   {\ekvd@assert@twoargs,\ekvd@ifnottwoargs,\ekvd@ifempty@gtwo}
% A test for exactly two tokens can be reduced for an empty-test after gobbling
% two tokens, in the case that there are fewer tokens than two in the argument,
% only macros will be gobbled that are needed for the true branch, which doesn't
% hurt, and if there are more this will not be empty.
%    \begin{macrocode}
\long\def\ekvd@assert@twoargs#1%
  {%
    \ekvd@ifnottwoargs{#1}{\ekvd@err@missing@definition}%
  }
\long\def\ekvd@ifnottwoargs#1%
  {%
    \ekvd@ifempty@gtwo#1\ekv@ifempty@B
      \ekv@ifempty@false\ekv@ifempty@A\ekv@ifempty@B\@firstoftwo
  }
\long\def\ekvd@ifempty@gtwo#1#2{\ekv@ifempty@\ekv@ifempty@A}
%    \end{macrocode}
% \end{macro}
%
% \begin{macro}[internal]
%   {
%     \ekvd@assert@val,\ekvd@assert@val@,\ekvd@assert@noval,\ekvd@assert@noval@,
%     \ekvd@extract@args,\ekvd@extracted@args,\ekvd@one@arg@string
%   }
% Assert that a given key is defined as a value taking key or a |NoVal| key with
% the correct argument structure, respectively.
%    \begin{macrocode}
\protected\def\ekvd@assert@val#1%
  {%
    \ekvifdefined\ekvd@set{#1}%
      {\expandafter\ekvd@assert@val@\csname\ekv@name\ekvd@set{#1}\endcsname}%
      {%
        \ekvifdefinedNoVal\ekvd@set{#1}%
          \ekvd@err@add@val@on@noval
          {\ekvd@err@undefined@key{#1}}%
        \@gobble
      }%
  }
\protected\def\ekvd@assert@val@#1%
  {%
    \expandafter\ekvd@extract@args\meaning#1\ekvd@stop
    \unless\ifx\ekvd@extracted@args\ekvd@one@arg@string
      \ekvd@err@unsupported@arg
    \fi
    \@firstofone
  }%
\protected\def\ekvd@assert@noval#1%
  {%
    \ekvifdefinedNoVal\ekvd@set{#1}%
      {\expandafter\ekvd@assert@noval@\csname\ekv@name\ekvd@set{#1}N\endcsname}%
      {%
        \ekvifdefined\ekvd@set{#1}%
          \ekvd@err@add@noval@on@val
          {\ekvd@err@undefined@key{#1}}%
        \@gobble
      }%
  }
\protected\def\ekvd@assert@noval@#1%
  {%
    \expandafter\ekvd@extract@args\meaning#1\ekvd@stop
    \unless\ifx\ekvd@extracted@args\ekv@empty
      \ekvd@err@unsupported@arg
    \fi
    \@firstofone
  }
\protected\def\ekvd@extract@args#1%
  {%
    \protected\def\ekvd@extract@args##1#1##2->##3\ekvd@stop
      {\def\ekvd@extracted@args{##2}}%
  }
\expandafter\ekvd@extract@args\expandafter{\detokenize{macro:}}
\edef\ekvd@one@arg@string{\string#1}
%    \end{macrocode}
% \end{macro}
%
% \begin{macro}[internal]{\ekvd@assert@arg,\ekvd@assert@arg@msg,\ekvd@ifnoarg}
% There is no need to actually define |\ekvd@ifnoarg| here, as it will be set by
% either |\ekvd@arg| or |\ekvd@noarg|.
%    \begin{macrocode}
\def\ekvd@assert@arg{\ekvd@ifnoarg\ekvd@err@missing@definition}
\long\def\ekvd@assert@arg@msg#1%
  {%
    \ekvd@ifnoarg{\ekvd@err@missing@definition@msg{#1}}%
  }
%    \end{macrocode}
% \end{macro}
%
% \begin{macro}[internal]{\ekvd@assert@filledarg,\ekvd@ifnoarg@or@empty}
%    \begin{macrocode}
\long\def\ekvd@assert@filledarg#1%
  {%
    \ekvd@ifnoarg@or@empty{#1}\ekvd@err@missing@definition
  }
\long\def\ekvd@ifnoarg@or@empty#1%
  {%
    \ekvd@ifnoarg
      \@firstoftwo
      {\ekv@ifempty{#1}}%
  }
%    \end{macrocode}
% \end{macro}
%
% \begin{macro}[internal]
%   {
%     \ekvd@assert@not@long,\ekvd@assert@not@protected,\ekvd@assert@not@also
%     \ekvd@assert@not@long@also,\ekvd@assert@not@protected@also,
%     \ekvd@assert@new,\ekvd@assert@not@new
%   }
% Some key-types don't want to be |also|, |\long| or |\protected|, so we provide
% macros to test this and throw an error, this could be silently ignored but now
% users will learn to not use unnecessary stuff which slows the compilation
% down.
%    \begin{macrocode}
\def\ekvd@assert@not@long{\ifx\ekvd@long\long\ekvd@err@no@prefix{long}\fi}
\def\ekvd@assert@not@protected
  {\ifx\ekvd@prot\protected\ekvd@err@no@prefix{protected}\fi}
\def\ekvd@assert@not@also{\ekvd@ifalso{\ekvd@err@no@prefix{also}}{}}
\def\ekvd@assert@not@long@also
  {\ifx\ekvd@long\long\ekvd@err@no@prefix@also{long}\fi}
\def\ekvd@assert@not@protected@also
  {\ifx\ekvd@prot\protected\ekvd@err@no@prefix@also{protected}\fi}
\def\ekvd@assert@new#1#2%
  {\csname ekvifdefined#1\endcsname\ekvd@set{#2}{\ekvd@err@not@new}}
\def\ekvd@assert@not@new
  {\ifx\ekvd@ifnew\ekvd@assert@new\ekvd@err@no@prefix{new}\fi}
\def\ekvd@assert@new@for@name#1%
  {%
    \ifx\ekvd@ifnew\ekvd@assert@new
      \ekv@fi@firstoftwo
    \fi
    \@secondoftwo
      {\ekv@ifdefined{#1}\ekvd@err@not@new}%
      \@firstofone
  }
%    \end{macrocode}
% \end{macro}
%
% \begin{macro}[internal]
%   {
%     \ekvd@if@not@already@choice, \ekvd@if@not@already@choice@a,
%     \ekvd@if@not@already@choice@b
%   }
% It is bad to use |also| on a key that already contains a |choice|, as both
% choices would share the same valid values and thus lead to each callback being
% used twice. The following is a rudimentary test against this.
%    \begin{macrocode}
\protected\def\ekvd@if@not@already@choice#1%
  {%
    \expandafter\ekvd@if@not@already@choice@a
      \csname\ekv@name\ekvd@set{#1}\endcsname
      {}\ekvd@h@choice\ekvd@stop
  }
\protected\def\ekvd@if@not@already@choice@a
  {%
    \expandafter\ekvd@if@not@already@choice@b
  }
\long\protected\def\ekvd@if@not@already@choice@b#1\ekvd@h@choice#2\ekvd@stop
  {%
    \ekv@ifempty{#2}\@firstofone\@gobble
  }
%    \end{macrocode}
% \end{macro}
%
% \begin{macro}[internal]{\ekvd@ifspace,\ekvd@ifspace@}
% Yet another test which can be reduced to an if-empty, this time by gobbling
% everything up to the first space.
%    \begin{macrocode}
\long\def\ekvd@ifspace#1%
  {%
    \ekvd@ifspace@#1 \ekv@ifempty@B
      \ekv@ifempty@false\ekv@ifempty@A\ekv@ifempty@B\@firstoftwo
  }
\long\def\ekvd@ifspace@#1 % keep this space
  {%
    \ekv@ifempty@\ekv@ifempty@A
  }
%    \end{macrocode}
% \end{macro}
%
% \subsubsection{Messages}
%
% Most messages of \expkvd\ are not expandable, since they only appear during
% key-definition, which is not expandable anyway.
%
% \begin{macro}[internal]
%   {
%     \ekvd@errm,\ekvd@err@missing@definition,\ekvd@err@missing@definition@msg,
%     \ekvd@err@missing@type,\ekvd@err@undefined@prefix,\ekvd@err@undefined@key,
%     \ekvd@err@no@prefix,\ekvd@err@no@prefix@msg,\ekvd@err@no@prefix@also,
%     \ekvd@err@add@val@on@noval,\ekvd@err@add@noval@on@val,
%     \ekvd@err@unsupported@arg,\ekvd@err@not@new
%   }
% The non-expandable error messages are boring, so here they are:
%    \begin{macrocode}
\protected\def\ekvd@errm#1{\errmessage{expkv-def Error: #1}}
\protected\def\ekvd@err@missing@definition
  {\ekvd@errm{Missing definition for key `\ekvd@cur'}}
\protected\def\ekvd@err@missing@definition@msg#1%
  {\ekvd@errm{Missing definition for key `\ekv@unexpanded{#1}'}}
\protected\def\ekvd@err@missing@type
  {\ekvd@errm{Missing type prefix for key `\ekvd@cur'}}
\protected\def\ekvd@err@undefined@prefix#1%
  {%
    \ekvd@errm
      {%
        Undefined prefix `\ekv@unexpanded{#1}' found while processing
        `\ekvd@cur'%
      }%
  }
\protected\def\ekvd@err@undefined@key#1%
  {%
    \ekvd@errm
      {Undefined key `\ekv@unexpanded{#1}' found while processing `\ekvd@cur'}%
  }
\protected\def\ekvd@err@undefined@noval#1%
  {%
    \ekvd@errm
      {%
        Undefined noval key `\unexpanded{#1}' found while processing
        `\ekvd@cur'%
      }%
  }
\protected\def\ekvd@err@no@prefix#1%
  {\ekvd@errm{prefix `#1' not accepted in `\ekvd@cur'}}
\protected\def\ekvd@err@no@prefix@msg#1#2%
  {\ekvd@errm{prefix `#2' not accepted in `\ekv@unexpanded{#1}'}}
\protected\def\ekvd@err@no@prefix@also#1%
  {\ekvd@errm{`\ekvd@cur' not allowed with a `#1' key}}
\protected\def\ekvd@err@add@val@on@noval
  {\ekvd@errm{`\ekvd@cur' not allowed with a NoVal key}}
\protected\def\ekvd@err@add@noval@on@val
  {\ekvd@errm{`\ekvd@cur' not allowed with a value taking key}}
\protected\def\ekvd@err@unsupported@arg\fi\@firstofone#1%
  {%
    \fi
    \ekvd@errm
      {%
        Existing key-macro has the unsupported argument string
        `\ekvd@extracted@args' for key `\ekvd@cur'%
      }%
  }
\protected\def\ekvd@err@not@new
  {\ekvd@errm{The key for `\ekvd@cur' is already defined}}
\protected\long\def\ekvd@err@misused@unknown
  {\ekvd@errm{Misuse of the unknown type found while processing `\ekvd@cur'}}
%    \end{macrocode}
% \end{macro}
%
% \begin{macro}[internal]
%   {
%     \ekvd@err@choice@invalid,\ekvd@err@choice@invalid@,\ekvd@choice@name,
%     \ekvd@unknown@choice@name
%   }
% |\ekvd@err@choice@invalid| will have to use this mechanism to throw its
% message. Also we have to retrieve the name parts of the choice in an easy way,
% so we use parentheses of catcode 8 here, which should suffice in most cases to
% allow for a correct separation.
%    \begin{macrocode}
\def\ekvd@err@choice@invalid#1%
  {%
    \ekvd@err@choice@invalid@#1%
  }
\begingroup
\catcode40=8
\catcode41=8
\@firstofone{\endgroup
\def\ekvd@choice@name#1#2#3%
  {%
    ekvd#1(#2)\detokenize{#3}%
  }
\def\ekvd@unknown@choice@name#1#2%
  {%
    ekvd:u:#1(#2)%
  }
\def\ekvd@err@choice@invalid@ ekvd#1(#2)\detokenize#3%
  {%
    \ekv@ifdefined{\ekvd@unknown@choice@name{#1}{#2}}%
      {\csname\ekvd@unknown@choice@name{#1}{#2}\endcsname{#3}}%
      {\ekvd@err{invalid choice `#3' for `#2' in set `#1'}}%
  }
}
%    \end{macrocode}
% \end{macro}
%
% \begin{macro}[internal]{\ekvd@err}
% The expandable error messages use |\ekvd@err|, which is just like |\ekv@err|
% from \expkv. It uses a runaway argument to start the error message.
%    \begin{macrocode}
\ekv@exparg{\long\def\ekvd@err#1}{\ekverr{expkv-def}{#1}}
%    \end{macrocode}
% \end{macro}
%
% Now everything that's left is to reset the category code of |@|.
%    \begin{macrocode}
\catcode`\@=\ekvd@tmp
%    \end{macrocode}
%
% \gobbledocstriptag
%</tex>
%^^A=<<
%
%^^A=<<
%
% \end{implementation}^^A=<<
%
% \clearpage
% \PrintIndex
%
\endinput
%
^^A vim: ft=tex fdm=marker fmr=>>=,=<<

\ProvidesPackage{expkv-def}%
  [\ekvdDate\space v\ekvdVersion\space a key-defining frontend for expkv]
%    \end{macrocode}
% \gobbledocstriptag
%</pkg>
%^^A=<<
%^^A the ConTeXt module >>=
% \subsection{The \ConTeXt\ module}
% \gobbledocstriptag
%<*ctx>
%    \begin{macrocode}
\writestatus{loading}{ConTeXt User Module / expkv-def}
\usemodule[expkv]
\unprotect
\input expkv-def.tex
\writestatus{loading}
  {ConTeXt User Module / expkv-def / Version \ekvdVersion\space loaded}
\protect\endinput
%    \end{macrocode}
% \gobbledocstriptag
%</ctx>
%^^A=<<
%^^A main file >>=
% \subsection{The Generic Code}
% The rest of this implementation will be the generic code.
% \gobbledocstriptag
%<*tex>
%
% Load \expkv\ if the package didn't already do so -- since \expkv\ has
% safeguards against being loaded twice this does no harm and the overhead
% isn't that big. Also we reuse some of the internals of \expkv\ to save us from
% retyping them.
%    \begin{macrocode}
\input expkv
%    \end{macrocode}
%
% We make sure that \file{expkv-def.tex} is only input once:
%    \begin{macrocode}
\expandafter\ifx\csname ekvdVersion\endcsname\relax
\else
  \expandafter\endinput
\fi
%    \end{macrocode}
%
% \begin{macro}{\ekvdVersion,\ekvdDate}
% We're on our first input, so lets store the version and date in a macro.
%    \begin{macrocode}
\def\ekvdVersion{0.8c}
\def\ekvdDate{2021-09-20}
%    \end{macrocode}
% \end{macro}
%
% If the \LaTeX\ format is loaded we want to be a good file and report back who
% we are, for this the package will have defined |\ekvd@tmp| to use
% |\ProvidesFile|, else this will expand to a |\relax| and do no harm.
%    \begin{macrocode}
\csname ekvd@tmp\endcsname
%    \end{macrocode}
%
% Store the category code of |@| to later be able to reset it and change it to
% 11 for now.
%    \begin{macrocode}
\expandafter\chardef\csname ekvd@tmp\endcsname=\catcode`\@
\catcode`\@=11
%    \end{macrocode}
% |\ekvd@tmp| will be reused later to handle expansion during the key
% defining. But we don't need it to ever store information long-term after
% \expkvd\ was initialized.
%
% \begin{macro}[internal]{\ekvd@ifprimitive}
%    \begin{macrocode}
\protected\long\def\ekvd@ifprimitive#1%
  {%
    \begingroup
      \edef\ekvd@tmpa{\string #1}%
      \edef\ekvd@tmpb{\meaning#1}%
      \expandafter
    \endgroup
    \ifx\ekvd@tmpa\ekvd@tmpb
      \ekv@fi@firstoftwo
    \fi
    \@secondoftwo
  }
%    \end{macrocode}
% \end{macro}
%
% \begin{macro}[internal]
%   {\ekvd@long,\ekvd@prot,\ekvd@clear@prefixes,\ekvd@ifalso}
% \expkvd\ will use |\ekvd@long|, |\ekvd@prot|, and |\ekvd@ifalso| to store
% whether a key should be defined as |\long| or |\protected| or adds an action
% to an existing key, and we have to clear them for every new key. By default
% |long| and |protected| will just be empty, |ifalso| will be
% |\@secondoftwo|, and |ifnew| will just use its third argument.
%    \begin{macrocode}
\protected\def\ekvd@clear@prefixes
  {%
    \let\ekvd@long\ekv@empty
    \let\ekvd@prot\ekv@empty
    \let\ekvd@ifalso\@secondoftwo
    \long\def\ekvd@ifnew##1##2##3{##3}%
  }
\ekvd@clear@prefixes
%    \end{macrocode}
% \end{macro}
%
% \begin{macro}{\ekvdefinekeys}
% This is the one front-facing macro which provides the interface to define
% keys. It's using |\ekvparse| to handle the \kv\ list, the interpretation will
% be done by |\ekvd@noarg| and |\ekvd@|. The \set\ for which the keys should be
% defined is stored in |\ekvd@set|.
%    \begin{macrocode}
\protected\def\ekvdefinekeys#1%
  {%
    \def\ekvd@set{#1}%
    \ekvparse\ekvd@noarg\ekvd@arg
  }
%    \end{macrocode}
% \end{macro}
%
% \begin{macro}[internal]{\ekvd@noarg,\ekvd@arg,\ekvd@handle}
% |\ekvd@noarg| and |\ekvd@arg| store whether there was a value in the \kv\
% pair. |\ekvd@handle| has to test whether there is a space inside the key and
% if so calls the prefix grabbing routine, else we throw an error and ignore the
% key.
%    \begin{macrocode}
\protected\long\def\ekvd@noarg#1%
  {%
    \let\ekvd@ifnoarg\@firstoftwo
    \expandafter\ekvd@handle\detokenize{#1}\ekvd@stop{}%
  }
\protected\long\def\ekvd@arg#1%
  {%
    \let\ekvd@ifnoarg\@secondoftwo
    \expandafter\ekvd@handle\detokenize{#1}\ekvd@stop
  }
\protected\long\def\ekvd@handle#1\ekvd@stop#2%
  {%
    \ekvd@clear@prefixes
    \def\ekvd@cur{#1}%
    \ekvd@ifspace{#1}%
      {\ekvd@prefix\ekv@mark#1\ekv@stop{#2}}%
      \ekvd@err@missing@type
  }
%    \end{macrocode}
% \end{macro}
%
% \begin{macro}[internal]{\ekvd@prefix,\ekvd@prefix@}
% \expkvd\ separates prefixes into two groups, the first being prefixes in the
% \TeX\ sense (|long| and |protected|) which use |@p@| in their name, the other
% being key-types (|code|, |int|, \emph{etc.}) which use |@t@| instead.
% |\ekvd@prefix| splits at the first space and checks whether its a |@p@|
% or |@t@| type prefix. If it is neither throw an error and gobble the
% definition (the value).
%    \begin{macrocode}
\protected\def\ekvd@prefix#1 {\ekv@strip{#1}\ekvd@prefix@\ekv@mark}
\protected\def\ekvd@prefix@#1#2\ekv@stop
  {%
    \ekv@ifdefined{ekvd@t@#1}%
      {\ekv@strip{#2}{\csname ekvd@t@#1\endcsname}}%
      {%
        \ekv@ifdefined{ekvd@p@#1}%
          {\csname ekvd@p@#1\endcsname\ekvd@prefix@after@p{#2}}%
          {\ekvd@err@undefined@prefix{#1}\@gobble}%
      }%
  }
%    \end{macrocode}
% \end{macro}
%
% \begin{macro}[internal]{\ekvd@prefix@after@p}
% The |@p@| type prefixes are all just modifying a following |@t@| type, so they
% will need to search for another prefix. This is true for all of them, so we
% use a macro to handle this. It'll throw an error if there is no other prefix.
%    \begin{macrocode}
\protected\def\ekvd@prefix@after@p#1%
  {%
    \ekvd@ifspace{#1}%
      {\ekvd@prefix#1\ekv@stop}%
      {\ekvd@err@missing@type\@gobble}%
  }
%    \end{macrocode}
% \end{macro}
%
% \begin{macro}[internal]
%   {\ekvd@p@long,\ekvd@p@protected,\ekvd@p@protect,\ekvd@p@also,\ekvd@p@new}
% Define the |@p@| type prefixes, they all just store some information in a
% temporary macro.
%    \begin{macrocode}
\protected\def\ekvd@p@long{\let\ekvd@long\long}
\protected\def\ekvd@p@protected{\let\ekvd@prot\protected}
\let\ekvd@p@protect\ekvd@p@protected
\protected\def\ekvd@p@also{\let\ekvd@ifalso\@firstoftwo}
\protected\def\ekvd@p@new{\let\ekvd@ifnew\ekvd@assert@new}
%    \end{macrocode}
% \end{macro}
%
% \subsubsection{Key Types}
%
% \begin{macro}[internal]{\ekvd@type@set,\ekvd@t@set}
% The |set| type is quite straight forward, just define a |NoVal| key to call
% |\ekvchangeset|.
%    \begin{macrocode}
\protected\def\ekvd@type@set#1#2%
  {%
    \ekvd@assert@not@long
    \ekvd@assert@not@protected
    \ekvd@ifnew{NoVal}{#1}%
      {%
        \ekv@ifempty{#2}%
          {\ekvd@err@missing@definition}%
          {%
            \ekvd@ifalso
              {%
                \ekv@expargtwice{\ekvd@add@noval{#1}}%
                  {\ekvchangeset{#2}}%
                  \ekvd@assert@not@protected@also
              }%
              {\ekv@expargtwice{\ekvdefNoVal\ekvd@set{#1}}{\ekvchangeset{#2}}}%
          }%
      }%
  }
\protected\def\ekvd@t@set#1#2%
  {%
    \ekvd@ifnoarg
      {\ekvd@type@set{#1}{#1}}%
      {\ekvd@type@set{#1}{#2}}%
  }
%    \end{macrocode}
% \end{macro}
%
% \begin{macro}[internal]{\ekvd@type@noval,\ekvd@t@noval,\ekvd@t@enoval}
% Another pretty simple type, |noval| just needs to assert that there is a
% definition and that |long| wasn't specified.
% There are types where the difference in the variants is so small, that we
% define a common handler for them, those common handlers are named with
% |@type@|. |noval| and |enoval| are so similar that we can use such a |@type@|
% macro, even if we could've done |noval| in a slightly faster way without it.
%    \begin{macrocode}
\protected\long\def\ekvd@type@noval#1#2#3%
  {%
    \ekvd@ifnew{NoVal}{#2}%
      {%
        \ekvd@assert@arg
          {%
            \ekvd@assert@not@long
            \ekvd@prot#1\ekvd@tmp{#3}%
            \ekvd@ifalso
              {\ekv@exparg{\ekvd@add@noval{#2}}\ekvd@tmp{}}%
              {\ekvletNoVal\ekvd@set{#2}\ekvd@tmp}%
          }%
      }%
  }
\protected\def\ekvd@t@noval{\ekvd@type@noval\def}
\protected\def\ekvd@t@enoval{\ekvd@type@noval\edef}
%    \end{macrocode}
% \end{macro}
%
% \begin{macro}[internal]{\ekvd@type@code,\ekvd@t@code,\ekvd@t@ecode}
% |code| is simple as well, |ecode| has to use |\edef| on a temporary macro,
% since \expkv\ doesn't provide an |\ekvedef|.
%    \begin{macrocode}
\protected\long\def\ekvd@type@code#1#2#3%
  {%
    \ekvd@ifnew{}{#2}%
      {%
        \ekvd@assert@arg
          {%
            \ekvd@prot\ekvd@long#1\ekvd@tmp##1{#3}%
            \ekvd@ifalso
              {\ekv@exparg{\ekvd@add@val{#2}}{\ekvd@tmp{##1}}{}}%
              {\ekvlet\ekvd@set{#2}\ekvd@tmp}%
          }%
      }%
  }
\protected\def\ekvd@t@code{\ekvd@type@code\def}
\protected\def\ekvd@t@ecode{\ekvd@type@code\edef}
%    \end{macrocode}
% \end{macro}
%
% \begin{macro}[internal]
%   {
%     \ekvd@type@default,\ekvd@t@default,\ekvd@t@qdefault,\ekvd@t@odefault,
%     \ekvd@t@fdefault
%   }
% |\ekvd@type@default| asserts there was an argument, also the key for which one
% wants to set a default has to be already defined (this is not so important for
% |default|, but |qdefault| requires is). If everything is good, |\edef| a
% temporary macro that expands |\ekvd@set| and the |\csname| for the key, and in
% the case of |qdefault| does the first expansion step of the key-macro.
%    \begin{macrocode}
\protected\long\def\ekvd@type@default#1#2#3#4%
  {%
    \ekvd@assert@arg
      {%
        \ekvifdefined\ekvd@set{#3}%
          {%
            \ekvd@assert@not@new
            \ekvd@assert@not@long
            \ekvd@prot\edef\ekvd@tmp
              {%
                \ekv@unexpanded\expandafter#1%
                  {#2\csname\ekv@name\ekvd@set{#3}\endcsname{#4}}%
              }%
            \ekvd@ifalso
              {\ekv@exparg{\ekvd@add@noval{#3}}\ekvd@tmp{}}%
              {\ekvletNoVal\ekvd@set{#3}\ekvd@tmp}%
          }%
          {\ekvd@err@undefined@key{#3}}%
      }%
  }
\protected\def\ekvd@t@default{\ekvd@type@default{}{}}
\protected\def\ekvd@t@qdefault{\ekvd@type@default{\expandafter\expandafter}{}}
\let\ekvd@t@odefault\ekvd@t@qdefault
\protected\def\ekvd@t@fdefault{\ekvd@type@default{}{\romannumeral`\^^@}}
%    \end{macrocode}
% \end{macro}
%
% \begin{macro}[internal]{\ekvd@t@edefault}
% |edefault| is too different from |default| and |qdefault| to reuse the
% |@type@| macro, as it doesn't need |\unexpanded| inside of |\edef|.
%    \begin{macrocode}
\protected\long\def\ekvd@t@edefault#1#2%
  {%
    \ekvd@assert@arg
      {%
        \ekvifdefined\ekvd@set{#1}%
          {%
            \ekvd@assert@not@new
            \ekvd@assert@not@long
            \ekvd@prot\edef\ekvd@tmp
              {\csname\ekv@name\ekvd@set{#1}\endcsname{#2}}%
            \ekvd@ifalso
              {\ekv@exparg{\ekvd@add@noval{#1}}\ekvd@tmp{}}%
              {\ekvletNoVal\ekvd@set{#1}\ekvd@tmp}%
          }%
          {\ekvd@err@undefined@key{#1}}%
      }%
  }
%    \end{macrocode}
% \end{macro}
%
% \begin{macro}[internal]
%   {\ekvd@t@initial,\ekvd@t@oinitial,\ekvd@t@finitial,\ekvd@t@einitial}
%    \begin{macrocode}
\long\def\ekvd@type@initial#1#2#3#4%
  {%
    \ekvd@assert@not@new
    \ekvd@assert@not@also
    \ekvd@assert@not@long
    \ekvd@assert@not@protected
    \ekvd@ifnoarg
      {%
        \ekvifdefinedNoVal\ekvd@set{#3}%
          {\csname\ekv@name\ekvd@set{#3}N\endcsname}%
          {\ekvd@err@undefined@noval{#3}}%
      }%
      {%
        \ekvifdefined\ekvd@set{#3}%
          {%
            #1{#2#4}%
            \csname\ekv@name\ekvd@set{#3}\expandafter\endcsname\expandafter
              {\ekvd@tmp}%
          }%
          {\ekvd@err@undefined@key{#3}}%
      }%
  }
\def\ekvd@t@initial{\ekvd@type@initial{\def\ekvd@tmp}{}}
\def\ekvd@t@oinitial{\ekvd@type@initial{\ekv@exparg{\def\ekvd@tmp}}{}}
\def\ekvd@t@einitial{\ekvd@type@initial{\edef\ekvd@tmp}{}}
\def\ekvd@t@finitial
  {\ekvd@type@initial{\ekv@exparg{\def\ekvd@tmp}}{\romannumeral`\^^@}}
%    \end{macrocode}
% \end{macro}
%
% \begin{macro}[internal]
%   {
%     \ekvd@type@bool,\ekvd@t@bool,\ekvd@t@gbool,\ekvd@t@boolTF,\ekvd@t@gboolTF,
%     \ekvd@t@invbool,\ekvd@t@ginvbool,\ekvd@t@invboolTF,\ekvd@t@ginvboolTF,
%   }
% The boolean types are a quicker version of a |choice| that accept |true| and
% |false|, and set up the |NoVal| action to be identical to \texttt{\key=true}.
% The |true| and |false| actions are always just |\let|ting the macro in |#7| to
% some other macro (\emph{e.g.}, \cs[no-index]{iftrue}).
%    \begin{macrocode}
\protected\def\ekvd@type@bool#1#2#3#4#5%
  {%
    \ekvd@ifnew{}{#4}%
      {%
        \ekvd@ifnew{NoVal}{#4}%
          {%
            \ekvd@assert@filledarg{#5}%
              {%
                \ekvd@newlet#5#3%
                \ekvd@type@choice{#4}%
                \protected\ekvdefNoVal\ekvd@set{#4}{#1\let#5#2}%
                \protected\expandafter\def
                  \csname\ekvd@choice@name\ekvd@set{#4}{true}\endcsname
                  {#1\let#5#2}%
                \protected\expandafter\def
                  \csname\ekvd@choice@name\ekvd@set{#4}{false}\endcsname
                  {#1\let#5#3}%
              }%
          }%
      }%
  }
\protected\def\ekvd@t@bool{\ekvd@type@bool{}\iftrue\iffalse}
\protected\def\ekvd@t@gbool{\ekvd@type@bool\global\iftrue\iffalse}
\protected\def\ekvd@t@boolTF{\ekvd@type@bool{}\@firstoftwo\@secondoftwo}
\protected\def\ekvd@t@gboolTF{\ekvd@type@bool\global\@firstoftwo\@secondoftwo}
\protected\def\ekvd@t@invbool{\ekvd@type@bool{}\iffalse\iftrue}
\protected\def\ekvd@t@ginvbool{\ekvd@type@bool\global\iffalse\iftrue}
\protected\def\ekvd@t@invboolTF{\ekvd@type@bool{}\@secondoftwo\@firstoftwo}
\protected\def\ekvd@t@ginvboolTF
  {\ekvd@type@bool\global\@secondoftwo\@firstoftwo}
%    \end{macrocode}
% \end{macro}
%
% \begin{macro}[internal]
%   {
%     \ekvd@type@boolpair,
%     \ekvd@t@boolpair,\ekvd@t@gboolpair,\ekvd@t@boolpairTF,\ekvd@t@gboolpairTF
%   }
% The boolean pair types are essentially the same as the boolean types, but set
% two macros instead of one.
%    \begin{macrocode}
\protected\def\ekvd@type@boolpair#1#2#3#4#5#6%
  {%
    \ekvd@ifnew{}{#4}%
      {%
        \ekvd@ifnew{NoVal}{#4}%
          {%
            \ekvd@newlet#5#3%
            \ekvd@newlet#6#2%
            \ekvd@type@choice{#4}%
            \protected\ekvdefNoVal\ekvd@set{#4}{#1\let#5#2#1\let#6#3}%
            \protected\expandafter\def
              \csname\ekvd@choice@name\ekvd@set{#4}{true}\endcsname
              {#1\let#5#2#1\let#6#3}%
            \protected\expandafter\def
              \csname\ekvd@choice@name\ekvd@set{#4}{false}\endcsname
              {#1\let#5#3#1\let#6#2}%
          }%
      }%
  }
\protected\def\ekvd@t@boolpair#1#2%
  {\ekvd@assert@twoargs{#2}{\ekvd@type@boolpair{}\iftrue\iffalse{#1}#2}}
\protected\def\ekvd@t@gboolpair#1#2%
  {\ekvd@assert@twoargs{#2}{\ekvd@type@boolpair\global\iftrue\iffalse{#1}#2}}
\protected\def\ekvd@t@boolpairTF#1#2%
  {%
    \ekvd@assert@twoargs{#2}%
      {\ekvd@type@boolpair{}\@firstoftwo\@secondoftwo{#1}#2}%
  }
\protected\def\ekvd@t@gboolpairTF#1#2%
  {%
    \ekvd@assert@twoargs{#2}%
      {\ekvd@type@boolpair\global\@firstoftwo\@secondoftwo{#1}#2}%
  }
%    \end{macrocode}
% \end{macro}
%
% \begin{macro}[internal]
%   {
%     \ekvd@type@data,\ekvd@t@data,\ekvd@t@gdata,\ekvd@t@dataT,\ekvd@t@gdataT
%   }
%    \begin{macrocode}
\protected\def\ekvd@type@data#1#2#3#4#5#6%
  {%
    \ekvd@ifnew{}{#5}%
      {%
        \ekvd@assert@filledarg{#6}%
          {%
            \ekvd@newlet#6#1%
            \ekvd@ifalso
              {%
                \let\ekvd@prot\protected
                \ekvd@add@val{#5}{\long#2#6####1#3{####1{#4}}}{}%
              }%
              {%
                \protected\ekvd@long\ekvdef\ekvd@set{#5}%
                  {\long#2#6####1#3{####1{#4}}}%
              }%
          }%
      }%
  }
\protected\def\ekvd@t@data
  {\ekvd@type@data\@secondoftwo\edef{####2}{\ekv@unexpanded{##1}}}
\protected\def\ekvd@t@edata{\ekvd@type@data\@secondoftwo\edef{####2}{##1}}
\protected\def\ekvd@t@gdata
  {\ekvd@type@data\@secondoftwo\xdef{####2}{\ekv@unexpanded{##1}}}
\protected\def\ekvd@t@xdata{\ekvd@type@data\@secondoftwo\xdef{####2}{##1}}
\protected\def\ekvd@t@dataT
  {\ekvd@type@data\@gobble\edef{}{\ekv@unexpanded{##1}}}
\protected\def\ekvd@t@edataT{\ekvd@type@data\@gobble\edef{}{##1}}
\protected\def\ekvd@t@gdataT
  {\ekvd@type@data\@gobble\xdef{}{\ekv@unexpanded{##1}}}
\protected\def\ekvd@t@xdataT{\ekvd@type@data\@gobble\xdef{}{##1}}
%    \end{macrocode}
% \end{macro}
%
% \begin{macro}[internal]{\ekvd@type@box,\ekvd@t@box,\ekvd@t@gbox}
% Set up our boxes. Though we're a generic package we want to be colour safe, so
% we put an additional grouping level inside the box contents, for the case that
% someone uses \pkg{color}. |\ekvd@newreg| is a small wrapper which tests
% whether the first argument is defined and if not does
% |\csname new#2\endcsname#1|.
%    \begin{macrocode}
\protected\def\ekvd@type@box#1#2#3%
  {%
    \ekvd@ifnew{}{#2}%
      {%
        \ekvd@assert@filledarg{#3}%
          {%
            \ekvd@newreg#3{box}%
            \ekvd@ifalso
              {%
                \let\ekvd@prot\protected
                \ekvd@add@val{#2}{#1\setbox#3=\hbox{\begingroup##1\endgroup}}{}%
              }%
              {%
                \protected\ekvd@long\ekvdef\ekvd@set{#2}%
                  {#1\setbox#3=\hbox{\begingroup##1\endgroup}}%
              }%
          }%
      }%
  }
\protected\def\ekvd@t@box{\ekvd@type@box{}}
\protected\def\ekvd@t@gbox{\ekvd@type@box\global}
%    \end{macrocode}
% \end{macro}
%
% \begin{macro}[internal]{\ekvd@type@toks,\ekvd@t@toks,\ekvd@t@gtoks}
% Similar to |box|, but set the |toks|.
%    \begin{macrocode}
\protected\def\ekvd@type@toks#1#2#3%
  {%
    \ekvd@ifnew{}{#2}%
      {%
        \ekvd@assert@filledarg{#3}%
          {%
            \ekvd@newreg#3{toks}%
            \ekvd@ifalso
              {%
                \let\ekvd@prot\protected
                \ekvd@add@val{#2}{#1#3={##1}}{}%
              }%
              {\protected\ekvd@long\ekvdef\ekvd@set{#2}{#1#3={##1}}}%
          }%
      }%
  }
\protected\def\ekvd@t@toks{\ekvd@type@toks{}}
\protected\def\ekvd@t@gtoks{\ekvd@type@toks\global}
%    \end{macrocode}
% \end{macro}
%
% \begin{macro}[internal]{\ekvd@type@preapptoks,\ekvd@t@apptoks,\ekvd@t@gapptoks}
% Just like |toks|, but expand the current contents of the |toks| register to
% append the new contents.
%    \begin{macrocode}
\ekvd@ifprimitive\toksapp
  {%
    \protected\def\ekvd@type@preapptoks#1#2#3%
      {%
        \ekvd@ifnew{}{#2}%
          {%
            \ekvd@assert@filledarg{#3}%
              {%
                \ekvd@newreg#3{toks}%
                \ekvd@ifalso
                  {%
                    \let\ekvd@prot\protected
                    \ekvd@add@val{#2}{#1#3{##1}}{}%
                  }%
                  {\protected\ekvd@long\ekvdef\ekvd@set{#2}{#1#3{##1}}}%
              }%
          }%
      }
    \protected\def\ekvd@t@apptoks{\ekvd@type@preapptoks\toksapp}
    \protected\def\ekvd@t@gapptoks{\ekvd@type@preapptoks\gtoksapp}
    \protected\def\ekvd@t@pretoks{\ekvd@type@preapptoks\tokspre}
    \protected\def\ekvd@t@gpretoks{\ekvd@type@preapptoks\gtokspre}
  }
  {%
    \protected\def\ekvd@type@apptoks#1#2#3%
      {%
        \ekvd@ifnew{}{#2}%
          {%
            \ekvd@assert@filledarg{#3}%
              {%
                \ekvd@newreg#3{toks}%
                \ekvd@ifalso
                  {%
                    \let\ekvd@prot\protected
                    \ekvd@add@val{#2}{#1#3=\expandafter{\the#3##1}}{}%
                  }%
                  {%
                    \protected\ekvd@long\ekvdef\ekvd@set{#2}%
                      {#1#3=\expandafter{\the#3##1}}%
                  }%
              }%
          }%
      }
    \protected\def\ekvd@t@apptoks{\ekvd@type@apptoks{}}
    \protected\def\ekvd@t@gapptoks{\ekvd@type@apptoks\global}
    \newtoks\ekvd@toks
    \protected\def\ekvd@type@pretoks#1#2#3%
      {%
        \ekvd@ifnew{}{#2}%
          {%
            \ekvd@assert@filledarg{#3}%
              {%
                \ekvd@newreg#3{toks}%
                \ekvd@ifalso
                  {%
                    \let\ekvd@prot\protected
                    \ekvd@add@val{#2}%
                      {%
                        \ekvd@toks={##1}%
                        #1#3=\expandafter{\the\expandafter\ekvd@toks\the#3}%
                      }%
                      {}%
                  }%
                  {%
                    \protected\ekvd@long\ekvdef\ekvd@set{#2}%
                      {%
                        \ekvd@toks={##1}%
                        #1#3=\expandafter{\the\expandafter\ekvd@toks\the#3}%
                      }%
                  }%
              }%
          }%
      }
    \protected\def\ekvd@t@pretoks{\ekvd@type@pretoks{}}
    \protected\def\ekvd@t@gpretoks{\ekvd@type@pretoks\global}
  }
%    \end{macrocode}
% \end{macro}
%
% \begin{macro}[internal]
%   {
%     \ekvd@type@reg,\ekvd@t@int,\ekvd@t@eint,\ekvd@t@gint,\ekvd@t@xint,
%     \ekvd@t@dimen,\ekvd@t@edimen,\ekvd@t@gdimen,\ekvd@t@xdimen,
%     \ekvd@t@skip,\ekvd@t@eskip,\ekvd@t@gskip,\ekvd@t@xskip
%   }
% The |\ekvd@type@reg| can handle all the types for which the assignment will
% just be \texttt{\meta{register}=\meta{value}}.
%    \begin{macrocode}
\protected\def\ekvd@type@reg#1#2#3#4#5#6%
  {%
    \ekvd@ifnew{}{#5}%
      {%
        \ekvd@assert@filledarg{#6}%
          {%
            \ekvd@newreg#6{#1}%
            \ekvd@ifalso
              {%
                \let\evkd@prot\protected
                \ekvd@add@val{#5}{#2#6=#3##1#4\relax}{}%
              }%
              {\protected\ekvd@long\ekvdef\ekvd@set{#5}{#2#6=#3##1#4\relax}}%
          }%
      }%
  }
\protected\def\ekvd@t@int{\ekvd@type@reg{count}{}{}{}}
\protected\def\ekvd@t@eint{\ekvd@type@reg{count}{}\numexpr\relax}
\protected\def\ekvd@t@gint{\ekvd@type@reg{count}\global{}{}}
\protected\def\ekvd@t@xint{\ekvd@type@reg{count}\global\numexpr\relax}
\protected\def\ekvd@t@dimen{\ekvd@type@reg{dimen}{}{}{}}
\protected\def\ekvd@t@edimen{\ekvd@type@reg{dimen}{}\dimexpr\relax}
\protected\def\ekvd@t@gdimen{\ekvd@type@reg{dimen}\global{}{}}
\protected\def\ekvd@t@xdimen{\ekvd@type@reg{dimen}\global\dimexpr\relax}
\protected\def\ekvd@t@skip{\ekvd@type@reg{skip}{}{}{}}
\protected\def\ekvd@t@eskip{\ekvd@type@reg{skip}{}\glueexpr\relax}
\protected\def\ekvd@t@gskip{\ekvd@type@reg{skip}\global{}{}}
\protected\def\ekvd@t@xskip{\ekvd@type@reg{skip}\global\glueexpr\relax}
%    \end{macrocode}
% \end{macro}
%
% \begin{macro}[internal]{\ekvd@type@store,\ekvd@t@store,\ekvd@t@gstore}
% The none-expanding |store| types use an |\edef| or |\xdef| and |\unexpanded|
% to be able to also store |#| easily.
%    \begin{macrocode}
\protected\def\ekvd@type@store#1#2#3#4%
  {%
    \ekvd@ifnew{}{#3}%
      {%
        \ekvd@assert@filledarg{#4}%
          {%
            \ekvd@newlet#4\ekv@empty
            \ekvd@ifalso
              {%
                \let\ekvd@prot\protected
                \ekvd@add@val{#3}{#1#4{#2}}{}%
              }%
              {\protected\ekvd@long\ekvdef\ekvd@set{#3}{#1#4{#2}}}%
          }%
      }%
  }
\protected\def\ekvd@t@store{\ekvd@type@store\edef{\ekv@unexpanded{##1}}}
\protected\def\ekvd@t@gstore{\ekvd@type@store\xdef{\ekv@unexpanded{##1}}}
\protected\def\ekvd@t@estore{\ekvd@type@store\edef{##1}}
\protected\def\ekvd@t@xstore{\ekvd@type@store\xdef{##1}}
%    \end{macrocode}
% \end{macro}
%
% \begin{macro}[internal]
%   {
%     \ekvd@type@meta,\ekvd@type@meta@a,\ekvd@type@meta@b,\ekvd@type@meta@c,
%     \ekvd@t@meta,\ekvd@t@nmeta
%   }
% |meta| sets up things such that another instance of |\ekvset| will be run on
% the argument, with the same \set.
%    \begin{macrocode}
\protected\long\def\ekvd@type@meta#1#2#3#4#5#6#7%
  {%
    \ekvd@ifnew{#1}{#6}%
      {%
        \ekvd@assert@filledarg{#7}%
          {%
            \edef\ekvd@tmp{\ekvd@set}%
            \expandafter\ekvd@type@meta@a\expandafter{\ekvd@tmp}{#7}{#2}%
            \ekvd@ifalso
              {\ekv@exparg{#3{#6}}{\ekvd@tmp#4}{#5}}%
              {\csname ekvlet#1\endcsname\ekvd@set{#6}\ekvd@tmp}%
          }%
      }%
  }
\protected\long\def\ekvd@type@meta@a#1#2%
  {%
    \expandafter\ekvd@type@meta@b\expandafter{\ekvset{#1}{#2}}%
  }
\protected\def\ekvd@type@meta@b
  {%
    \expandafter\ekvd@type@meta@c\expandafter
  }
\protected\long\def\ekvd@type@meta@c#1#2%
  {%
    \ekvd@prot\ekvd@long\def\ekvd@tmp#2{#1}%
  }
\protected\def\ekvd@t@meta{\ekvd@type@meta{}{##1}\ekvd@add@val{{##1}}{}}
\protected\def\ekvd@t@nmeta
  {%
    \ekvd@assert@not@long
    \ekvd@type@meta{NoVal}{}\ekvd@add@noval{}\ekvd@assert@not@long@also
  }
%    \end{macrocode}
% \end{macro}
%
% \begin{macro}[internal]
%   {\ekvd@type@smeta,\ekvd@type@smeta@,\ekvd@t@smeta,\ekvd@t@snmeta}
% |smeta| is pretty similar to |meta|, but needs two arguments inside of \val,
% such that the first is the \set\ for which the sub-|\ekvset| and the second is
% the \kv\ list.
%    \begin{macrocode}
\protected\long\def\ekvd@type@smeta#1#2#3#4#5#6#7%
  {%
    \ekvd@ifnew{#1}{#6}%
      {%
        \ekvd@assert@twoargs{#7}%
          {%
            \ekvd@type@meta@a#7{#2}%
            \ekvd@ifalso
              {\ekv@exparg{#3{#6}}{\ekvd@tmp#4}{#5}}%
              {\csname ekvlet#1\endcsname\ekvd@set{#6}\ekvd@tmp}%
          }%
      }%
  }
\protected\def\ekvd@t@smeta{\ekvd@type@smeta{}{##1}\ekvd@add@val{{##1}}{}}
\protected\def\ekvd@t@snmeta
  {%
    \ekvd@assert@not@long
    \ekvd@type@smeta{NoVal}{}\ekvd@add@noval{}\ekvd@assert@not@long@also
  }
%    \end{macrocode}
% \end{macro}
%
% \begin{macro}[internal]
%   {
%     \ekvd@type@choice,
%     \ekvd@populate@choice,\ekvd@populate@choice@,\ekvd@populate@choice@noarg,
%     \ekvd@choice@prefix,\ekvd@choice@prefix@,\ekvd@choice@prefix@done,
%     \ekvd@choice@p@protected,\ekvd@choice@p@protect,
%     \ekvd@choice@p@long,\ekvd@choice@p@long@,
%     \ekvd@t@choice
%   }
% The |choice| type is by far the most complex type, as we have to run a
% sub-parser on the choice-definition list, which should support the |@p@| type
% prefixes as well (but |long| will always throw an error, as they are not
% allowed to be long). |\ekvd@type@choice| will just define the choice-key, the
% handling of the choices definition will be done by |\ekvd@populate@choice|.
%    \begin{macrocode}
\protected\def\ekvd@type@choice#1%
  {%
    \ekvd@assert@not@long
    \ekv@exparg{\ekvd@prot\def\ekvd@tmp##1}%
      {%
        \expandafter\ekvd@h@choice\expandafter
          {\ekvd@choice@name\ekvd@set{#1}{##1}}%
      }%
    \ekvd@ifalso
      {%
        \ekvd@assert@val{#1}%
          {%
            \ekvd@if@not@already@choice{#1}%
              {%
                \ekv@exparg
                  {%
                    \expandafter\ekvd@add@aux
                      \csname\ekv@name\ekvd@set{#1}\endcsname{{##1}}{#1}%
                  }%
                  {\ekvd@tmp{##1}}%
                  {\ekvd@long\ekvdef}\ekvd@assert@not@long@also
              }%
          }%
      }%
      {\ekvlet\ekvd@set{#1}\ekvd@tmp}%
  }
%    \end{macrocode}
% |\ekvd@populate@choice| just uses |\ekvparse| and then gives control to
% |\ekvd@populate@choice@noarg|, which throws an error, and
% |\ekvd@populate@choice@|.
%    \begin{macrocode}
\protected\def\ekvd@populate@choice
  {%
    \ekvparse\ekvd@populate@choice@noarg\ekvd@populate@choice@
  }
\protected\long\def\ekvd@populate@choice@noarg#1%
  {%
    \expandafter\ekvd@err@missing@definition@msg\expandafter{\ekvd@cur : #1}%
  }
%    \end{macrocode}
% |\ekvd@populate@choice@| runs the prefix-test, if there is none we can
% directly define the choice, for that |\ekvd@set@choice| will expand to the
% current choice-key's name, which will have been defined by |\ekvd@t@choice|.
% If there is a prefix run the prefix grabbing routine, which was altered for
% |@type@choice|.
%    \begin{macrocode}
\protected\long\def\ekvd@populate@choice@#1#2%
  {%
    \ekvd@clear@prefixes
    \ekvd@ifspace{#1}%
      {\ekvd@choice@prefix{\ekv@mark#1}\ekv@mark#1\ekv@stop}%
      {%
        \expandafter\edef
          \csname\ekvd@choice@name\ekvd@set\ekvd@set@choice{#1}\endcsname
      }%
      {\unexpanded{#2}}%
  }
\protected\def\ekvd@choice@prefix#1#2
  {%
    \ekv@strip{#2}{\ekvd@choice@prefix@{#1}}\ekv@mark
  }
\protected\def\ekvd@choice@prefix@#1#2#3\ekv@stop
  {%
    \ekv@ifdefined{ekvd@choice@p@#2}%
      {%
        \csname ekvd@choice@p@#2\endcsname
        \ekvd@ifspace{#3}%
          {\ekvd@choice@prefix{#3}#3\ekv@stop}%
          {\ekvd@choice@prefix@done{#3}}%
      }%
      {\ekvd@choice@prefix@done{#1}}%
  }
\protected\def\ekvd@choice@prefix@done#1%
  {%
    \ekvd@prot\expandafter\edef
      \csname
        \ekv@strip{#1}{\ekvd@choice@name\ekvd@set\ekvd@set@choice}%
      \endcsname
  }
\protected\def\ekvd@choice@p@protected{\let\ekvd@prot\protected}
\let\ekvd@choice@p@protect\ekvd@choice@p@protected
\protected\def\ekvd@choice@invalid@p#1\ekvd@ifspace#2%
  {%
    \expandafter\ekvd@choice@invalid@p@\expandafter{\ekv@gobble@mark#2}{#1}%
    \ekvd@ifspace{#2}%
  }
\protected\def\ekvd@choice@invalid@p@#1#2%
  {%
    \expandafter\ekvd@err@no@prefix@msg\expandafter{\ekvd@cur : #2 #1}{#2}%
  }
\protected\def\ekvd@choice@p@long{\ekvd@choice@invalid@p{long}}%
\protected\def\ekvd@choice@p@also{\ekvd@choice@invalid@p{also}}%
\protected\def\ekvd@choice@p@new{\ekvd@choice@invalid@p{new}}%
%    \end{macrocode}
% Finally we're able to set up the |@t@choice| macro, which has to store the
% current choice-key's name, define the key, and parse the available choices.
%    \begin{macrocode}
\protected\long\def\ekvd@t@choice#1#2%
  {%
    \ekvd@ifnew{}{#1}%
      {%
        \ekvd@assert@arg
          {%
            \ekvd@type@choice{#1}%
            \def\ekvd@set@choice{#1}%
            \ekvd@populate@choice{#2}%
          }%
      }%
  }
%    \end{macrocode}
% \end{macro}
%
% \begin{macro}[internal]{\ekvd@t@choice-store,\ekvd@t@choice-enum}
%   These two types define a special kind of |choice| key and are quite similar,
%   the only difference is what the different choices do (hence they use a
%   shared initialisation which differs in the chosen |populate| step).
%    \begin{macrocode}
\protected\long\expandafter\def\csname ekvd@t@choice-store\endcsname
  {\ekvd@type@choicespecial\ekvd@populate@choicestore}
\protected\long\expandafter\def\csname ekvd@t@choice-enum\endcsname
  {\ekvd@type@choicespecial\ekvd@populate@choiceenum}
%    \end{macrocode}
% \begin{macro}[internal]{\ekvd@type@choicespecial}
%   Initialise similar to a |choice| key. The difference is that we require two
%   arguments (which we assert), a macro to store things in, and a |csv|-list
%   containing the allowed values. |#1| is the |populate| macro according to the
%   type used.
%    \begin{macrocode}
\protected\long\def\ekvd@type@choicespecial#1#2#3%
  {%
    \ekvd@ifnew{}{#2}%
      {%
        \ekvd@assert@twoargs{#3}%
          {%
            \ekvd@type@choice{#2}%
            \def\ekvd@set@choice{#2}%
            #1#3%
          }%
      }%
  }
%    \end{macrocode}
% \end{macro}
% \begin{macro}[internal]
%   {\ekvd@populate@choicestore,\ekvd@populate@choicestore@}
%   We initialise the storing macro if it doesn't yet exist, and then we loop
%   over the value list. The |\edef|s with |\unexpanded| are both necessary to
%   be able to store macro parameter tokens (the outer protects at define time,
%   the inner at use time).
%    \begin{macrocode}
\protected\long\def\ekvd@populate@choicestore#1%
  {%
    \ekvd@newlet#1\ekv@empty
    \ekvcsvloop{\ekvd@populate@choicestore@#1}%
  }
\protected\long\def\ekvd@populate@choicestore@#1#2%
  {%
    \protected\expandafter\edef
      \csname\ekvd@choice@name\ekvd@set\ekvd@set@choice{#2}\endcsname
      {\unexpanded{\edef#1{\unexpanded{#2}}}}%
  }
%    \end{macrocode}
% \end{macro}
% \begin{macro}[internal]
%   {\ekvd@populate@choiceenum,\ekvd@populate@choiceenum@}
%   This is similar to the population of a |choice-store| type, but instead of
%   storing the values in a macro this initialises a count and stores the
%   position of the value in the list inside that count (zero-indexed). The
%   space is necessary to terminate the number scanning, which is the reason we
%   use |\@firstofone| (so that the space after the macro name isn't gobbled by
%   \TeX).
%    \begin{macrocode}
\protected\long\def\ekvd@populate@choiceenum#1%
  {%
    \ekvd@newreg#1{count}%
    \def\ekvd@tmp{0}%
    \ekvcsvloop{\ekvd@populate@choiceenum@#1}%
  }
\protected\long\def\ekvd@populate@choiceenum@#1#2%
  {%
    \protected\expandafter\edef
      \csname\ekvd@choice@name\ekvd@set\ekvd@set@choice{#2}\endcsname
      {#1=\@firstofone{\ekvd@tmp} }%
    \edef\ekvd@tmp{\the\numexpr\ekvd@tmp+1\relax}%
  }
%    \end{macrocode}
% \end{macro}
% \end{macro}
%
% \begin{macro}[internal]{\ekvd@t@unknown-choice}
%    \begin{macrocode}
\protected\long\expandafter\def\csname ekvd@t@unknown-choice\endcsname#1#2%
  {%
    \ekvd@assert@new@for@name{\ekvd@unknown@choice@name\ekvd@set{#1}}%
      {%
        \ekvd@assert@arg
          {%
            \ekvd@assert@not@long
            \ekvd@assert@not@also
            \ekvd@prot\expandafter
            \def\csname\ekvd@unknown@choice@name\ekvd@set{#1}\endcsname##1{#2}%
          }%
      }%
  }
%    \end{macrocode}
% \end{macro}
%
% \begin{macro}[internal]
%   {\ekvd@t@unknown,\ekvd@type@unknown@code,\ekvd@type@unknown@noval}
%   The |unknown| type has different subtypes which would be the key names for
%   other types. It is first checked whether that subtype is defined, if it
%   isn't throw an error, else use that subtype.
%    \begin{macrocode}
\protected\long\def\ekvd@t@unknown#1#2%
  {%
    \ekv@ifdefined{ekvd@type@unknown@\detokenize{#1}}%
      {\csname ekvd@type@unknown@\detokenize{#1}\endcsname{#2}}%
      \ekvd@err@misused@unknown
  }
%    \end{macrocode}
%   The |unknown noval| type can use |\ekvdefunknownNoVal| directly (after
%   asserting some prefixes).
%    \begin{macrocode}
\protected\long\def\ekvd@type@unknown@noval#1%
  {%
    \ekvd@assert@new@for@name{\ekv@name\ekvd@set{}uN}%
      {%
        \ekvd@assert@arg
          {%
            \ekvd@assert@not@also
            \ekvd@assert@not@long
            \ekvd@prot\ekvdefunknownNoVal\ekvd@set{#1}%
          }%
      }%
  }
%    \end{macrocode}
%   The |unknown code| type uses some trickery during the definition in order to
%   swap out |#1| and |#2| in the user supplied definition. This is done via a
%   temporary macro that stores the definition but gets the parameter numbers
%   reversed while the real definition is done.
%    \begin{macrocode}
\protected\long\def\ekvd@type@unknown@code#1%
  {%
    \ekvd@assert@new@for@name{\ekv@name\ekvd@set{}u}%
      {%
        \ekvd@assert@arg
          {%
            \ekvd@assert@not@also
            \begingroup
              \def\ekvd@tmp##1##2{#1}%
              \ekv@exparg
              {%
                \endgroup
                \ekvd@prot\ekvd@long\ekvdefunknown\ekvd@set
              }%
              {\ekvd@tmp{##2}{##1}}%
          }%
      }%
  }
%    \end{macrocode}
% \end{macro}
%
% \begin{macro}[internal]
%   {
%     \ekvd@type@unknown@redirect,
%     \ekvd@type@unknown@redirect-code,
%     \ekvd@type@unknown@redirect-noval
%   }
%   The |unknown redirect| types also just forward to |\ekvredirectunknown|
%   after asserting some prefixes.
%    \begin{macrocode}
\protected\edef\ekvd@type@unknown@redirect#1%
  {%
    \expandafter\noexpand\csname ekvd@type@unknown@redirect-code\endcsname{#1}%
    \expandafter\noexpand\csname ekvd@type@unknown@redirect-noval\endcsname{#1}%
  }
\protected\expandafter\def\csname ekvd@type@unknown@redirect-code\endcsname#1%
  {%
    \ekvd@assert@new@for@name{\ekv@name\ekvd@set{}u}%
      {%
        \ekvd@assert@arg
          {%
            \ekvd@assert@not@also
            \ekvd@assert@not@protected
            \expandafter\ekvredirectunknown\expandafter{\ekvd@set}{#1}%
          }%
      }%
  }
\protected\expandafter\def\csname ekvd@type@unknown@redirect-noval\endcsname#1%
  {%
    \ekvd@assert@new@for@name{\ekv@name\ekvd@set{}uN}%
      {%
        \ekvd@assert@arg
          {%
            \ekvd@assert@not@also
            \ekvd@assert@not@protected
            \ekvd@assert@not@long
            \expandafter\ekvredirectunknownNoVal\expandafter{\ekvd@set}{#1}%
          }%
      }%
  }
%    \end{macrocode}
% \end{macro}
%
% \subsubsection{Key Type Helpers}
% There are some keys that might need helpers during their execution (not during
% their definition, which are gathered as |@type@| macros). These helpers are
% named |@h@|.
%
% \begin{macro}[internal]{\ekvd@h@choice,\ekvd@h@choice@}
% The |choice| helper will just test whether the given choice was defined, if
% not throw an error expandably, else call the macro which stores the code for
% this choice.
%    \begin{macrocode}
\def\ekvd@h@choice#1%
  {%
    \expandafter\ekvd@h@choice@
      \csname\ifcsname#1\endcsname#1\else relax\fi\endcsname
      {#1}%
  }
\def\ekvd@h@choice@#1#2%
  {%
    \ifx#1\relax
      \ekvd@err@choice@invalid{#2}%
      \expandafter\@gobble
    \fi
    #1%
  }
%    \end{macrocode}
% \end{macro}
%
%
% \subsubsection{Handling \texttt{also}}
%
% \begin{macro}[internal]
%   {\ekvd@add@val,\ekvd@add@noval,\ekvd@add@aux,\ekvd@add@aux@}
%    \begin{macrocode}
\protected\long\def\ekvd@add@val#1#2#3%
  {%
    \ekvd@assert@val{#1}%
      {%
        \expandafter\ekvd@add@aux\csname\ekv@name\ekvd@set{#1}\endcsname{{##1}}%
          {#1}{#2}{\ekvd@long\ekvdef}{#3}%
      }%
  }
\protected\long\def\ekvd@add@noval#1#2#3%
  {%
    \ekvd@assert@noval{#1}%
      {%
        \expandafter\ekvd@add@aux\csname\ekv@name\ekvd@set{#1}N\endcsname{}%
          {#1}{#2}\ekvdefNoVal{#3}%
      }%
  }
\protected\long\def\ekvd@add@aux#1#2%
  {%
    \ekvd@extract@prefixes#1%
    \expandafter\ekvd@add@aux@\expandafter{#1#2}%
  }
\protected\long\def\ekvd@add@aux@#1#2#3#4#5%
  {%
    #5%
    \ekvd@prot#4\ekvd@set{#2}{#1#3}%
  }
%    \end{macrocode}
% \end{macro}
%
% \begin{macro}[internal]
%   {
%     \ekvd@extract@prefixes,\ekvd@extract@prefixes@,
%     \ekvd@extract@prefixes@long,\ekvd@extract@prefixes@prot
%   }
% This macro checks which prefixes were used for the definition of a macro and
% sets |\ekvd@long| and |\ekvd@prot| accordingly.
%    \begin{macrocode}
\protected\def\ekvd@extract@prefixes#1%
  {%
    \expandafter\ekvd@extract@prefixes@\meaning#1\ekvd@stop
  }
%    \end{macrocode}
% In the following definition |#1| will get replaced by |macro:|, |#2| by
% |\long| and |#3| by |\protected| (in each, all tokens will have category
% other). This allows us to parse the |\meaning| of a macro for those strings.
%    \begin{macrocode}
\protected\def\ekvd@extract@prefixes@#1#2#3%
  {%
    \protected\def\ekvd@extract@prefixes@##1#1##2\ekvd@stop
      {%
        \ekvd@extract@prefixes@long
          ##1\ekvd@mark\@firstofone#2\ekvd@mark\@gobble\ekvd@stop
          {\let\ekvd@long\long}%
        \ekvd@extract@prefixes@prot
          ##1\ekvd@mark\@firstofone#3\ekvd@mark\@gobble\ekvd@stop
          {\let\ekvd@prot\protected}%
      }%
    \protected\def\ekvd@extract@prefixes@long##1#2##2\ekvd@mark##3##4\ekvd@stop
      {##3}%
    \protected\def\ekvd@extract@prefixes@prot##1#3##2\ekvd@mark##3##4\ekvd@stop
      {##3}%
  }
%    \end{macrocode}
% We use a temporary macro to expand the three arguments of
% |\ekvd@extract@prefixes@|, which will set up the real meaning of itself and
% the parsing for |\long| and |\protected|.
%    \begin{macrocode}
\begingroup
\edef\ekvd@tmp
  {%
    \endgroup
    \ekvd@extract@prefixes@
      {\detokenize{macro:}}%
      {\string\long}%
      {\string\protected}%
  }
\ekvd@tmp
%    \end{macrocode}
% \end{macro}
%
%
% \subsubsection{Tests}
%
% \begin{macro}[internal]{\ekvd@newlet,\ekvd@newreg}
% These macros test whether a control sequence is defined, if it isn't they
% define it, either via |\let| or via the correct \cs[no-index]{new\meta{reg}}.
%    \begin{macrocode}
\protected\def\ekvd@newlet#1#2%
  {%
    \ifdefined#1\ekv@fi@gobble\fi\@firstofone{\let#1#2}%
  }
\protected\def\ekvd@newreg#1#2%
  {%
    \ifdefined#1\ekv@fi@gobble\fi\@firstofone{\csname new#2\endcsname#1}%
  }
%    \end{macrocode}
% \end{macro}
%
% \begin{macro}[internal]
%   {\ekvd@assert@twoargs,\ekvd@ifnottwoargs,\ekvd@ifempty@gtwo}
% A test for exactly two tokens can be reduced for an empty-test after gobbling
% two tokens, in the case that there are fewer tokens than two in the argument,
% only macros will be gobbled that are needed for the true branch, which doesn't
% hurt, and if there are more this will not be empty.
%    \begin{macrocode}
\long\def\ekvd@assert@twoargs#1%
  {%
    \ekvd@ifnottwoargs{#1}{\ekvd@err@missing@definition}%
  }
\long\def\ekvd@ifnottwoargs#1%
  {%
    \ekvd@ifempty@gtwo#1\ekv@ifempty@B
      \ekv@ifempty@false\ekv@ifempty@A\ekv@ifempty@B\@firstoftwo
  }
\long\def\ekvd@ifempty@gtwo#1#2{\ekv@ifempty@\ekv@ifempty@A}
%    \end{macrocode}
% \end{macro}
%
% \begin{macro}[internal]
%   {
%     \ekvd@assert@val,\ekvd@assert@val@,\ekvd@assert@noval,\ekvd@assert@noval@,
%     \ekvd@extract@args,\ekvd@extracted@args,\ekvd@one@arg@string
%   }
% Assert that a given key is defined as a value taking key or a |NoVal| key with
% the correct argument structure, respectively.
%    \begin{macrocode}
\protected\def\ekvd@assert@val#1%
  {%
    \ekvifdefined\ekvd@set{#1}%
      {\expandafter\ekvd@assert@val@\csname\ekv@name\ekvd@set{#1}\endcsname}%
      {%
        \ekvifdefinedNoVal\ekvd@set{#1}%
          \ekvd@err@add@val@on@noval
          {\ekvd@err@undefined@key{#1}}%
        \@gobble
      }%
  }
\protected\def\ekvd@assert@val@#1%
  {%
    \expandafter\ekvd@extract@args\meaning#1\ekvd@stop
    \unless\ifx\ekvd@extracted@args\ekvd@one@arg@string
      \ekvd@err@unsupported@arg
    \fi
    \@firstofone
  }%
\protected\def\ekvd@assert@noval#1%
  {%
    \ekvifdefinedNoVal\ekvd@set{#1}%
      {\expandafter\ekvd@assert@noval@\csname\ekv@name\ekvd@set{#1}N\endcsname}%
      {%
        \ekvifdefined\ekvd@set{#1}%
          \ekvd@err@add@noval@on@val
          {\ekvd@err@undefined@key{#1}}%
        \@gobble
      }%
  }
\protected\def\ekvd@assert@noval@#1%
  {%
    \expandafter\ekvd@extract@args\meaning#1\ekvd@stop
    \unless\ifx\ekvd@extracted@args\ekv@empty
      \ekvd@err@unsupported@arg
    \fi
    \@firstofone
  }
\protected\def\ekvd@extract@args#1%
  {%
    \protected\def\ekvd@extract@args##1#1##2->##3\ekvd@stop
      {\def\ekvd@extracted@args{##2}}%
  }
\expandafter\ekvd@extract@args\expandafter{\detokenize{macro:}}
\edef\ekvd@one@arg@string{\string#1}
%    \end{macrocode}
% \end{macro}
%
% \begin{macro}[internal]{\ekvd@assert@arg,\ekvd@assert@arg@msg,\ekvd@ifnoarg}
% There is no need to actually define |\ekvd@ifnoarg| here, as it will be set by
% either |\ekvd@arg| or |\ekvd@noarg|.
%    \begin{macrocode}
\def\ekvd@assert@arg{\ekvd@ifnoarg\ekvd@err@missing@definition}
\long\def\ekvd@assert@arg@msg#1%
  {%
    \ekvd@ifnoarg{\ekvd@err@missing@definition@msg{#1}}%
  }
%    \end{macrocode}
% \end{macro}
%
% \begin{macro}[internal]{\ekvd@assert@filledarg,\ekvd@ifnoarg@or@empty}
%    \begin{macrocode}
\long\def\ekvd@assert@filledarg#1%
  {%
    \ekvd@ifnoarg@or@empty{#1}\ekvd@err@missing@definition
  }
\long\def\ekvd@ifnoarg@or@empty#1%
  {%
    \ekvd@ifnoarg
      \@firstoftwo
      {\ekv@ifempty{#1}}%
  }
%    \end{macrocode}
% \end{macro}
%
% \begin{macro}[internal]
%   {
%     \ekvd@assert@not@long,\ekvd@assert@not@protected,\ekvd@assert@not@also
%     \ekvd@assert@not@long@also,\ekvd@assert@not@protected@also,
%     \ekvd@assert@new,\ekvd@assert@not@new
%   }
% Some key-types don't want to be |also|, |\long| or |\protected|, so we provide
% macros to test this and throw an error, this could be silently ignored but now
% users will learn to not use unnecessary stuff which slows the compilation
% down.
%    \begin{macrocode}
\def\ekvd@assert@not@long{\ifx\ekvd@long\long\ekvd@err@no@prefix{long}\fi}
\def\ekvd@assert@not@protected
  {\ifx\ekvd@prot\protected\ekvd@err@no@prefix{protected}\fi}
\def\ekvd@assert@not@also{\ekvd@ifalso{\ekvd@err@no@prefix{also}}{}}
\def\ekvd@assert@not@long@also
  {\ifx\ekvd@long\long\ekvd@err@no@prefix@also{long}\fi}
\def\ekvd@assert@not@protected@also
  {\ifx\ekvd@prot\protected\ekvd@err@no@prefix@also{protected}\fi}
\def\ekvd@assert@new#1#2%
  {\csname ekvifdefined#1\endcsname\ekvd@set{#2}{\ekvd@err@not@new}}
\def\ekvd@assert@not@new
  {\ifx\ekvd@ifnew\ekvd@assert@new\ekvd@err@no@prefix{new}\fi}
\def\ekvd@assert@new@for@name#1%
  {%
    \ifx\ekvd@ifnew\ekvd@assert@new
      \ekv@fi@firstoftwo
    \fi
    \@secondoftwo
      {\ekv@ifdefined{#1}\ekvd@err@not@new}%
      \@firstofone
  }
%    \end{macrocode}
% \end{macro}
%
% \begin{macro}[internal]
%   {
%     \ekvd@if@not@already@choice, \ekvd@if@not@already@choice@a,
%     \ekvd@if@not@already@choice@b
%   }
% It is bad to use |also| on a key that already contains a |choice|, as both
% choices would share the same valid values and thus lead to each callback being
% used twice. The following is a rudimentary test against this.
%    \begin{macrocode}
\protected\def\ekvd@if@not@already@choice#1%
  {%
    \expandafter\ekvd@if@not@already@choice@a
      \csname\ekv@name\ekvd@set{#1}\endcsname
      {}\ekvd@h@choice\ekvd@stop
  }
\protected\def\ekvd@if@not@already@choice@a
  {%
    \expandafter\ekvd@if@not@already@choice@b
  }
\long\protected\def\ekvd@if@not@already@choice@b#1\ekvd@h@choice#2\ekvd@stop
  {%
    \ekv@ifempty{#2}\@firstofone\@gobble
  }
%    \end{macrocode}
% \end{macro}
%
% \begin{macro}[internal]{\ekvd@ifspace,\ekvd@ifspace@}
% Yet another test which can be reduced to an if-empty, this time by gobbling
% everything up to the first space.
%    \begin{macrocode}
\long\def\ekvd@ifspace#1%
  {%
    \ekvd@ifspace@#1 \ekv@ifempty@B
      \ekv@ifempty@false\ekv@ifempty@A\ekv@ifempty@B\@firstoftwo
  }
\long\def\ekvd@ifspace@#1 % keep this space
  {%
    \ekv@ifempty@\ekv@ifempty@A
  }
%    \end{macrocode}
% \end{macro}
%
% \subsubsection{Messages}
%
% Most messages of \expkvd\ are not expandable, since they only appear during
% key-definition, which is not expandable anyway.
%
% \begin{macro}[internal]
%   {
%     \ekvd@errm,\ekvd@err@missing@definition,\ekvd@err@missing@definition@msg,
%     \ekvd@err@missing@type,\ekvd@err@undefined@prefix,\ekvd@err@undefined@key,
%     \ekvd@err@no@prefix,\ekvd@err@no@prefix@msg,\ekvd@err@no@prefix@also,
%     \ekvd@err@add@val@on@noval,\ekvd@err@add@noval@on@val,
%     \ekvd@err@unsupported@arg,\ekvd@err@not@new
%   }
% The non-expandable error messages are boring, so here they are:
%    \begin{macrocode}
\protected\def\ekvd@errm#1{\errmessage{expkv-def Error: #1}}
\protected\def\ekvd@err@missing@definition
  {\ekvd@errm{Missing definition for key `\ekvd@cur'}}
\protected\def\ekvd@err@missing@definition@msg#1%
  {\ekvd@errm{Missing definition for key `\ekv@unexpanded{#1}'}}
\protected\def\ekvd@err@missing@type
  {\ekvd@errm{Missing type prefix for key `\ekvd@cur'}}
\protected\def\ekvd@err@undefined@prefix#1%
  {%
    \ekvd@errm
      {%
        Undefined prefix `\ekv@unexpanded{#1}' found while processing
        `\ekvd@cur'%
      }%
  }
\protected\def\ekvd@err@undefined@key#1%
  {%
    \ekvd@errm
      {Undefined key `\ekv@unexpanded{#1}' found while processing `\ekvd@cur'}%
  }
\protected\def\ekvd@err@undefined@noval#1%
  {%
    \ekvd@errm
      {%
        Undefined noval key `\unexpanded{#1}' found while processing
        `\ekvd@cur'%
      }%
  }
\protected\def\ekvd@err@no@prefix#1%
  {\ekvd@errm{prefix `#1' not accepted in `\ekvd@cur'}}
\protected\def\ekvd@err@no@prefix@msg#1#2%
  {\ekvd@errm{prefix `#2' not accepted in `\ekv@unexpanded{#1}'}}
\protected\def\ekvd@err@no@prefix@also#1%
  {\ekvd@errm{`\ekvd@cur' not allowed with a `#1' key}}
\protected\def\ekvd@err@add@val@on@noval
  {\ekvd@errm{`\ekvd@cur' not allowed with a NoVal key}}
\protected\def\ekvd@err@add@noval@on@val
  {\ekvd@errm{`\ekvd@cur' not allowed with a value taking key}}
\protected\def\ekvd@err@unsupported@arg\fi\@firstofone#1%
  {%
    \fi
    \ekvd@errm
      {%
        Existing key-macro has the unsupported argument string
        `\ekvd@extracted@args' for key `\ekvd@cur'%
      }%
  }
\protected\def\ekvd@err@not@new
  {\ekvd@errm{The key for `\ekvd@cur' is already defined}}
\protected\long\def\ekvd@err@misused@unknown
  {\ekvd@errm{Misuse of the unknown type found while processing `\ekvd@cur'}}
%    \end{macrocode}
% \end{macro}
%
% \begin{macro}[internal]
%   {
%     \ekvd@err@choice@invalid,\ekvd@err@choice@invalid@,\ekvd@choice@name,
%     \ekvd@unknown@choice@name
%   }
% |\ekvd@err@choice@invalid| will have to use this mechanism to throw its
% message. Also we have to retrieve the name parts of the choice in an easy way,
% so we use parentheses of catcode 8 here, which should suffice in most cases to
% allow for a correct separation.
%    \begin{macrocode}
\def\ekvd@err@choice@invalid#1%
  {%
    \ekvd@err@choice@invalid@#1%
  }
\begingroup
\catcode40=8
\catcode41=8
\@firstofone{\endgroup
\def\ekvd@choice@name#1#2#3%
  {%
    ekvd#1(#2)\detokenize{#3}%
  }
\def\ekvd@unknown@choice@name#1#2%
  {%
    ekvd:u:#1(#2)%
  }
\def\ekvd@err@choice@invalid@ ekvd#1(#2)\detokenize#3%
  {%
    \ekv@ifdefined{\ekvd@unknown@choice@name{#1}{#2}}%
      {\csname\ekvd@unknown@choice@name{#1}{#2}\endcsname{#3}}%
      {\ekvd@err{invalid choice `#3' for `#2' in set `#1'}}%
  }
}
%    \end{macrocode}
% \end{macro}
%
% \begin{macro}[internal]{\ekvd@err}
% The expandable error messages use |\ekvd@err|, which is just like |\ekv@err|
% from \expkv. It uses a runaway argument to start the error message.
%    \begin{macrocode}
\ekv@exparg{\long\def\ekvd@err#1}{\ekverr{expkv-def}{#1}}
%    \end{macrocode}
% \end{macro}
%
% Now everything that's left is to reset the category code of |@|.
%    \begin{macrocode}
\catcode`\@=\ekvd@tmp
%    \end{macrocode}
%
% \gobbledocstriptag
%</tex>
%^^A=<<
%
%^^A=<<
%
% \end{implementation}^^A=<<
%
% \clearpage
% \PrintIndex
%
\endinput
%
^^A vim: ft=tex fdm=marker fmr=>>=,=<<
}{}
\ProvidesFile{expkv-def.dtx}
  [\csname ekvdDate\endcsname\ a key-defining frontend for expkv]
\PassOptionsToPackage{full}{textcomp}
\documentclass{l3doc}
\RequirePackage[oldstylenums,nott]{kpfonts}
\input{glyphtounicode}
\pdfgentounicode=1
\RequirePackage{tcolorbox}
\newtcolorbox{exresult}[2][]
  {colback=expkvgrey!10!white,colframe=expkvgrey,fontupper=\small,width={#2},#1}
\newtcbox\exres[1][]
  {
     colback=expkvgrey!10!white
    ,colframe=expkvgrey
    ,size=small
    ,nobeforeafter
    ,tcbox raise base
    ,fontupper=\small
    ,#1
  }
\RequirePackage{listings}
\RequirePackage{booktabs}
\RequirePackage{array}
\RequirePackage{collcell}
\RequirePackage{siunitx}
\RequirePackage{xcolor}
\RequirePackage{caption}
\RequirePackage{microtype}
\RequirePackage{accsupp}
\lstdefinelanguage{expkvd}[primitive]{TeX}
  {
    ,moretexcs=[4]^^A e-TeX
      {
        expanded,
        numexpr,
        protected,
      }
    ,moretexcs=[5]^^A plain/LaTeX
      {
        approx,
        begin,
        empty,
        item,
        LaTeX,
        makeatletter,makeatother,
        newcommand,newdimen,
        RequirePackage,
        rule,
        TeX,
        textit,texttt,
        usepackage,
      }
    ,moretexcs=[6]^^A used packages
      {
        ^^A xfp
        fpeval,
        ^^A keyval
        setkeys,
        ^^A yax
        defactiveparameter,storevalue,setparameterlist
      }
    ,moretexcs=[2]^^A expkv-def macros
      {
        ekvdefinekeys
      }
    ,moretexcs=[3]^^A expkv-pkg macros
      {
        ^^A expkv
        ekvdef,ekvdefNoVal,^^A
        ekvlet,ekvletNoVal,ekvletkv,ekvletkvNoVal,^^A
        ekvdefunknown,ekvdefunknownNoVal,^^A
        ekvredirectunknown,ekvredirectunknownNoVal,^^A
        ekvset,ekvsetSneaked,^^A
        ekvsetdef,ekvsetSneakeddef,ekvsetdefSneaked,^^A
        ekvparse,^^A
        ekvVersion,ekvDate,^^A
        ekvifdefined,ekvifdefinedNoVal,ekvifdefinedset,^^A
        ekvbreak,ekvbreakPreSneak,ekvbreakPostSneak,^^A
        ekvsneak,ekvsneakPre,^^A
        ekvchangeset,^^A
        ekvoptarg,ekvoptargTF,^^A
        ekverr,
        ^^A expkv-cs
        ekvcSecondaryKeys,ekvcSplitAndForward,
      }
  }
\colorlet{codeparam}{cyan!65!black}
\def\mylstwd{.55em}
\lstset
  {
    ,language=expkvd
    ,flexiblecolumns=false
    ,basewidth=\mylstwd
    ,gobble=2
    ,basicstyle=\fontfamily{jkp}\itshape
    ,texcsstyle=*[1]{\bfseries\color{expkvgrey}} ^^A primitives
    ,texcsstyle=*[2]{\bfseries\color{expkvred}}  ^^A expkv
    ,texcsstyle=*[3]{\color{expkvred}}           ^^A expkv-pkg
    ,texcsstyle=*[4]{\bfseries\color{expkvgrey}} ^^A e-TeX
    ,texcsstyle=*[5]{\bfseries\color{expkvgrey}} ^^A plain/LaTeX macros
    ,texcsstyle=*[6]{}                           ^^A macros of other packages
    ,commentstyle=\color[gray]{0.4}
    ,literate=
              {\{} {{\CodeUpBf\{}}{1}
              {\}} {{\CodeUpBf\}}}{1}
              {$}  {{\CodeUpBf\$}}{1}
              {[}  {{\CodeUp[}}{1}
              {]}  {{\CodeUp]}}{1}
              {(}  {{\CodeUp(}}{1}
              {)}  {{\CodeUp)}}{1}
              {*}  {{$*$}}{1}
              {1}  {{\CodeUp{1}}}{1}
              {2}  {{\CodeUp{2}}}{1}
              {3}  {{\CodeUp{3}}}{1}
              {4}  {{\CodeUp{4}}}{1}
              {5}  {{\CodeUp{5}}}{1}
              {6}  {{\CodeUp{6}}}{1}
              {7}  {{\CodeUp{7}}}{1}
              {8}  {{\CodeUp{8}}}{1}
              {9}  {{\CodeUp{9}}}{1}
              {0}  {{\CodeUp{0}}}{1}
              {##} {{\CodeColored{codeparam}{1}{\#}}}{1}
              {##1}{{\CodeColored{codeparam}{2}{\#1}}}{2}
              {##2}{{\CodeColored{codeparam}{2}{\#2}}}{2}
              {##3}{{\CodeColored{codeparam}{2}{\#3}}}{2}
              {##4}{{\CodeColored{codeparam}{2}{\#4}}}{2}
              {##5}{{\CodeColored{codeparam}{2}{\#5}}}{2}
              {##6}{{\CodeColored{codeparam}{2}{\#6}}}{2}
              {##7}{{\CodeColored{codeparam}{2}{\#7}}}{2}
              {##8}{{\CodeColored{codeparam}{2}{\#8}}}{2}
              {##9}{{\CodeColored{codeparam}{2}{\#9}}}{2}
              {<key>}{{$\langle$}key{$\rangle$}}{5}
              {<set>}{{$\langle$}set{$\rangle$}}{5}
              {<set1>}{{$\langle$}set1{$\rangle$}}{6}
              {<set2>}{{$\langle$}set2{$\rangle$}}{6}
    ^^A,literate=*{<key>}{\key}{4}{<set>}{\set}{4}
  }
\newcommand*\CodeColored[3]{\textcolor{#1}{\makebox[\dimexpr\mylstwd*#2]{#3}}}
\newcommand*\CodeUpBf[1]{\makebox[\mylstwd]{\textup{\textbf{#1}}}}
\newcommand*\CodeUp[1]{\makebox[\mylstwd]{\textup{#1}}}
\RequirePackage{randtext}
\let\metaORIG\meta
\protected\def\meta #1{\texttt{\metaORIG{#1}}}
\renewcommand*\thefootnote{\fnsymbol{footnote}}
\definecolor{expkvred}{HTML}{9F393D}
\colorlet{expkvgrey}{black!75}
\makeatletter
\newcommand*\expkv
  {^^A
    \texorpdfstring
      {^^A
        \mbox
          {^^A
            \BeginAccSupp{ActualText=expkv}^^A
            \href{https://github.com/Skillmon/tex_expkv}
              {^^A
                \rmfamily
                \bfseries
                {\color{expkvgrey}e\kern-.05em x\kern-.05em}^^A
                \lower.493ex
                  \hbox{{\color{expkvgrey}P}\kern-.1em{\color{expkvred}k}}^^A
                \kern-.18em{\color{expkvred}v}^^A
              }^^A
            \EndAccSupp{}^^A
          }^^A
      }
      {expkv}^^A
  }
\newcommand*\expkvd
  {^^A
    \texorpdfstring
      {^^A
        \mbox
          {^^A
            \BeginAccSupp{ActualText=expkv-def}^^A
            \href{https://github.com/Skillmon/tex_expkv-def}
              {^^A
                \rmfamily
                \bfseries
                {\color{expkvgrey}e\kern-.05em x\kern-.05em}^^A
                \lower.493ex
                  \hbox{{\color{expkvgrey}P}\kern-.1em{\color{expkvred}k}}^^A
                \kern-.18em{\color{expkvred}v}^^A
                {\color{expkvgrey}^^A
                  \kern.05em\rule[-.1ex]{.08em}{1.2ex}\kern.05em\textsc{def}^^A
                }^^A
              }^^A
            \EndAccSupp{}^^A
          }^^A
      }
      {expkv-def}^^A
  }
\newcommand\kv{\meta{key}=\meta{value}}
\newcommand\key{\meta{key}}
\newcommand\val{\meta{value}}
\newcommand\set{\meta{set}}
\newcommand\enfprefix{\textcolor{black}}
\newcommand\allprefix{\textcolor{gray}}
\newcommand\notprefix{\textcolor{red!80!black}}
\newcommand\printprefix[2]
  {^^A
    \ifcase\numexpr#1\relax\or
      \enfprefix{#2}\or
      \allprefix{#2}\or
      \notprefix{#2}\fi
  }
\newcommand\prefixes[4]
  {^^A
    \hfill
    \printprefix{#1}{new}
    \printprefix{#2}{also}
    \printprefix{#3}{protected}
    \printprefix{#4}{long}^^A
  }
\hypersetup{linkcolor=red!80!black,urlcolor=purple!80!black}
\DoNotIndex{\def,\edef,\,,\=,\begingroup,\catcode,\chardef,\csname,\endcsname}
\DoNotIndex{\endgroup,\endinput,\errmessage,\expandafter,\input,\let,\long}
\DoNotIndex{\protected,\ProvidesFile,\ProvidesPackage,\relax,\space}
\DoNotIndex{\@,\unexpanded,\string,\expanded,\dimexpr,\global,\glueexpr,\hbox}
\DoNotIndex{\numexpr,\RequirePackage,\setbox,\the,\unless,\xdef,\@firstofone}
\DoNotIndex{\@firstoftwo,\@gobble,\@secondoftwo,\meaning,\detokenize}
\DoNotIndex{\romannumeral,\^}
\DoNotIndex{\ifcsname}
\DoNotIndex{\ifx}
\DoNotIndex{\ifdefined}
\DoNotIndex{\iffalse}
\DoNotIndex{\iftrue}
\DoNotIndex{\else}
\DoNotIndex{\fi}
\@gobble\fi ^^A ignoring \ifx, \ifcsname, etc., but only one \fi
\@gobble\fi ^^A ignoring \ifx, \ifcsname, etc., but only one \fi
\@gobble\fi ^^A ignoring \ifx, \ifcsname, etc., but only one \fi
\@gobble\fi ^^A ignoring \ifx, \ifcsname, etc., but only one \fi
\@ifdefinable\gobbledocstriptag{\def\gobbledocstriptag#1>{}}
\makeatother
\begin{document}
  \title
    {^^A
      \texorpdfstring
        {^^A
          \huge\expkvd
          \\[\medskipamount]
          \Large a key-defining frontend for \expkv
        }{expkv-def - a key-defining frontend for expkv}^^A
    }
  \date{\ekvdDate\space v\ekvdVersion}
  \author{Jonathan P. Spratte\thanks{\protect\randomize{jspratte@yahoo.de}}}
  \DocInput{expkv-def.dtx}
\end{document}
%</driver>^^A=<<
% \fi
%
% \maketitle
% \renewcommand*\thefootnote{\arabic{footnote}}
%
% \begin{abstract}
% \noindent\parfillskip=0pt
% \expkvd\ provides a small \kv\ interface to define keys for \expkv. Key-types
% are declared using prefixes, similar to static typed languages. The stylised
% name is \expkvd\ but the files use \file{expkv-def}, this is due to CTAN-rules
% which don't allow \string| in package names since that is the pipe symbol in
% *nix shells.
% \end{abstract}
%
% \tableofcontents
%
% \begin{documentation}^^A>>=
%
% \section{Documentation}
%
% Since the trend for the last couple of years goes to defining keys for a \kv\
% interface using a \kv\ interface, I thought that maybe providing such an
% interface for \expkv\ will make it more attractive for actual use, besides its
% unique selling points of being fully expandable, and fast and reliable. But at
% the same time I don't want to widen \expkv's initial scope. So here it is
% \expkvd, go define \kv\ interfaces with \kv\ interfaces.
%
% Unlike many of the other established \kv\ interfaces to define keys, \expkvd\
% works using prefixes instead of suffixes (\emph{e.g.}, |.tl_set:N| of
% \pkg{l3keys}) or directory like handlers (\emph{e.g.}, |/.store in| of
% \pkg{pgfkeys}). This was decided as a personal preference, more over in \TeX\
% parsing for the first space is way easier than parsing for the last one.
% \expkvd's prefixes are sorted into two categories: |p|-type, which are
% equivalent to \TeX's prefixes like |\long|, and |t|-type defining the type of
% the key. For a description of the available |p|-prefixes take a look at
% \autoref{sec:prefix:p}, the |t|-prefixes are described in
% \autoref{sec:prefix:t}.
%
% \expkvd\ is usable as generic code, as a \LaTeX\ package, and as a \ConTeXt\
% module. It'll automatically load \expkv\ in the same mode as well. To use it,
% just use one of
% \begin{lstlisting}
% \input expkv-def       % plainTeX
% \usepackage{expkv-def} % LaTeX
% \usemodule[expkv-def]  % ConTeXt
% \end{lstlisting}
%
% \subsection{Macros}
% Apart from version and date containers there is only a single user-facing
% macro, and that should be used to define keys.
%
% \begin{function}{\ekvdefinekeys}
%   \begin{syntax}
%     \cs{ekvdefinekeys}\marg{set}\{\kv, \ldots\}
%   \end{syntax}
%   In \set, define \key\ to have definition \val. The general syntax for \key\
%   should be
%   \begin{quote}\ttfamily\small
%     \meta{prefix} \meta{name}
%   \end{quote}
%   Where \meta{prefix} is a space separated list of optional |p|-type prefixes
%   followed by one |t|-type prefix. The syntax of \val\ is dependent on the
%   used |t|-prefix.
% \end{function}
%
% \begin{function}{\ekvdDate,\ekvdVersion}
%   These two macros store the version and date of the package.
% \end{function}
%
% \subsection{Prefixes}
% As already said there are |p|-prefixes and |t|-prefixes. Not every |p|-prefix
% is allowed for all |t|-prefixes.
%
% \subsubsection{\texttt{p}-Prefixes}\label{sec:prefix:p}
% The two |p|-type prefixes |long| and |protected| are pretty simple by nature,
% so their description is pretty simple. They affect the \key\ at use-time, so
% omitting |long| doesn't mean that a \meta{definition} can't contain a |\par|
% token, only that the \key\ will not accept a |\par| in \val.  On the other
% hand |new| and |also| might be simple on first sight as well, but their rules
% are a bit more complicated.
%
% \begin{function}{also}
%   The following key type will be \emph{added} to an existing \key's
%   definition. You can't add a type taking an argument at use time to an
%   existing key which doesn't take an argument and vice versa. Also you'll get
%   an error if you try to add an action which isn't allowed to be either |long|
%   or |protected| to a key which already is |long| or |protected| (the opposite
%   order would be suboptimal as well, but can't be really captured with the
%   current code).
%
%   A key already defined as |long| or |protected| will stay |long| or
%   |protected|, but you can as well add |long| or |protected| with the |also|
%   definition.
%
%   As a small example, suppose you want to create a boolean key, but
%   additionally to setting a boolean value you want to execute some more code
%   as well, you can use the following
% \begin{lstlisting}
% \ekvdefinekeys{also-example}
%   {
%      bool key      = \ifmybool
%     ,also code key = \domystuff{#1}
%   }
% \end{lstlisting}
%
%   If you use |also| on a |choice|, |bool|, |invbool|, or |boolpair| key it is
%   tried to determine if the key already is of one of those types. If this test
%   is true the declared choices will be added to the possible choices but the
%   key's definition will not be changed other than that. If that wouldn't have
%   been done, the callbacks of the different choices could get called multiple
%   times.
% \end{function}
%
% \begin{function}{protected,protect}
%   The following key will be defined |\protected|. Note that key-types which
%   can't be defined expandable will always use |\protected|.
% \end{function}
%
% \begin{function}{long}
%   The following key will be defined |\long|.
% \end{function}
%
% \begin{function}{new}
%   The following key must be new (so previously undefined). An error is thrown
%   if it is already defined and the new definition is ignored. |new| only
%   asserts that there are no conflicts between |NoVal| keys and other |NoVal|
%   keys or value taking keys and other value taking keys. For example you can
%   use the following without an error:
% \begin{lstlisting}
% \ekvdefinekeys{new-example}
%   {
%      code key       = \domystuffwitharg{#1}
%     ,new  noval key = \domystuffwithoutarg
%   }
% \end{lstlisting}
% \end{function}
%
% \subsubsection{\texttt{t}-Prefixes}\label{sec:prefix:t}
% Since the |p|-type prefixes apply to some of the |t|-prefixes automatically
% but sometimes one might be disallowed we need some way to highlight this
% behaviour. In the following an enforced prefix will be printed black
% (\texttt{\enfprefix{protected}}), allowed prefixes will be grey
% (\texttt{\allprefix{protected}}), and disallowed prefixes will be red
% (\texttt{\notprefix{protected}}). This will be put flush-right in the syntax
% showing line.
%
% \begin{function}{code,ecode}
%   \begin{syntax}
%     code \key\ = \marg{definition} \prefixes2222
%   \end{syntax}
%   Define \key\ to expand to \meta{definition}. The \key\ will require a \val\
%   for which you can use |#1| inside \meta{definition}. The |ecode| variant
%   will fully expand \meta{definition} inside an |\edef|.
% \end{function}
%
% \begin{function}{noval,enoval}
%   \begin{syntax}
%     noval \key\ = \marg{definition} \prefixes2223
%   \end{syntax}
%   The |noval| type defines \key\ to expand to \meta{definition}. The \key\
%   will not take a \val. |enoval| fully expands \meta{definition} inside an
%   |\edef|.
% \end{function}
%
% \begin{function}{default,qdefault,odefault,fdefault,edefault}
%   \begin{syntax}
%     default \key\ = \marg{definition} \prefixes3223
%   \end{syntax}
%   This serves to place a default \val\ for a \key\ that takes an argument, the
%   \key\ can be of any argument-grabbing kind, and when used without a \val\
%   it will be passed \meta{definition} instead. The |qdefault| variant will
%   expand the \key's code once, so will be slightly quicker, but not change if
%   you redefine \key. |odefault| is just another name for |qdefault|. The
%   |fdefault| version will expand the key code until a non-expandable token or
%   a space is found, a space would be gobbled.\footnotemark{} The |edefault| on
%   the other hand fully expands the \key-code with \meta{definition} as its
%   argument inside of an |\edef|.
% \end{function}
% \footnotetext
%   {^^A
%     For those familiar with \TeX-coding: This uses a
%     \cs[no-index]{romannumeral}-expansion.
%   }
%
% \begin{function}{initial,oinitial,finitial,einitial}
%   \begin{syntax}
%     initial \key\ = \{\val\} \prefixes3333
%     initial \key
%   \end{syntax}
%   With |initial| you can set an initial \val\ for an already defined argument
%   taking \key. It'll just call the key-macro of \key\ and pass it \val. The
%   |einitial| variant will expand \val\ using an |\edef| expansion prior to
%   passing it to the key-macro and the |oinitial| variant will expand the first
%   token in \val\ once. |finitial| will expand \val\ until a non-expandable
%   token or a space is found, a space would be
%   gobbled.\footnotemark
%
%   If you don't provide a value (and no equals sign) a |noval| \key\ of the
%   same name is called once (or, if you specified a |default| for a value
%   taking key that would be used).
% \end{function}
% \footnotetext{Again using \cs[no-index]{romannumeral}.}
%
% \begin{function}{bool,gbool,boolTF,gboolTF}
%   \begin{syntax}
%     bool \key\ = \meta{cs} \prefixes2223
%   \end{syntax}
%   The \meta{cs} should be a single control sequence, such as |\iffoo|.
%   This will define \key\ to be a boolean key, which only takes the values
%   |true| or |false| and will throw an error for other values. If the key is
%   used without a \val\ it'll have the same effect as if you use
%   \texttt{\key=true}. |bool| and |gbool| will behave like \TeX-ifs so either
%   be |\iftrue| or |\iffalse|. The |boolTF| and |gboolTF| variants will both
%   take two arguments and if true the first will be used else the second, so
%   they are always either |\@firstoftwo| or |\@secondoftwo|. The variants with
%   a leading |g| will set the control sequence globally, the others locally. If
%   \meta{cs} is not yet defined it'll be initialised as the |false| version.
%   Note that the initialisation is \emph{not} done with |\newif|, so you will
%   not be able to do |\footrue| outside of the \kv\ interface, but you could
%   use |\newif| yourself. Even if the \key\ will not be |\protected| the
%   commands which execute the |true| or |false| choice will be, so the usage
%   should be safe in an expansion context (\emph{e.g.}, you can use
%   \texttt{edefault \key\ = false} without an issue to change the default
%   behaviour to execute the |false| choice). Internally a |bool| \key\ is the
%   same as a |choice| key which is set up to handle |true| and |false| as
%   choices.
% \end{function}
%
% \begin{function}{invbool,ginvbool,invboolTF,ginvboolTF}
%   \begin{syntax}
%     bool \key\ = \meta{cs} \prefixes2223
%   \end{syntax}
%   These are inverse boolean keys, they behave like |bool| and friends but set
%   the opposite meaning to the macro \meta{cs} in each case. So if
%   \texttt{key=true} is used |invbool| will set \meta{cs} to |\iffalse| and
%   vice versa.
% \end{function}
%
% \begin{function}{boolpair,gboolpair,boolpairTF,gboolpairTF}
%   \begin{syntax}
%     boolpair \key\ = \meta{cs_1}\meta{cs_2} \prefixes2223
%   \end{syntax}
%   The |boolpair| key type behaves like both |bool| and |invbool|, the
%   \meta{cs_1} will be set to the meaning according to the rules of |bool|, and
%   \meta{cs_2} will be set to the opposite.
% \end{function}
%
% \begin{function}{store,estore,gstore,xstore}
%   \begin{syntax}
%     store \key\ = \meta{cs} \prefixes2212
%   \end{syntax}
%   The \meta{cs} should be a single control sequence, such as |\foo|.
%   This will define \key\ to store \val\ inside of the control sequence. If
%   \meta{cs} isn't yet defined it will be initialised as empty. The variants
%   behave similarly to their |\def|, |\edef|, |\gdef|, and |\xdef|
%   counterparts, but |store| and |gstore| will allow you to store macro
%   parameters inside of them by using |\unexpanded|.
% \end{function}
%
% \begin{function}{data,edata,gdata,xdata}
%   \begin{syntax}
%     data \key\ = \meta{cs} \prefixes2212
%   \end{syntax}
%   The \meta{cs} should be a single control sequence, such as |\foo|. This will
%   define \key\ to store \val\ inside of the control sequence. But unlike the
%   |store| type, the macro \meta{cs} will be a switch at the same time, it'll
%   take two arguments and if \key\ was used expands to the first argument
%   followed by \val\ in braces, if \key\ was not used \meta{cs} will expand to
%   the second argument (so behave like |\@secondoftwo|). The idea is that with
%   this type you can define a key which should be typeset formatted. The
%   |edata| and |xdata| variants will fully expand \val, the |gdata| and |xdata|
%   variants will store \val\ inside \meta{cs} globally. The |p|-prefixes will
%   only affect the key-macro, \meta{cs} will always be expandable and |\long|.
% \end{function}
%
% \begin{function}{dataT,edataT,gdataT,xdataT}
%   \begin{syntax}
%     dataT \key\ = \meta{cs} \prefixes2212
%   \end{syntax}
%   Just like |data|, but instead of \meta{cs} grabbing two arguments it'll only
%   grab one, so by default it'll behave like |\@gobble|, and if a \val\ was
%   given to \key\ the \meta{cs} will behave like |\@firstofone| appended by
%   \marg{value}.
% \end{function}
%
% \begin{function}{int,eint,gint,xint}
%   \begin{syntax}
%     int \key\ = \meta{cs} \prefixes2212
%   \end{syntax}
%   The \meta{cs} should be a single control sequence, such as |\foo|.
%   An |int| key will be a \TeX-count register. If \meta{cs} isn't defined yet,
%   |\newcount| will be used to initialise it. The |eint| and |xint| versions
%   will use |\numexpr| to allow basic computations in their \val. The |gint|
%   and |xint| variants set the register globally.
% \end{function}
%
% \begin{function}{dimen,edimen,gdimen,xdimen}
%   \begin{syntax}
%     dimen \key\ = \meta{cs} \prefixes2212
%   \end{syntax}
%   The \meta{cs} should be a single control sequence, such as |\foo|.
%   This is just like |int| but uses a dimen register, |\newdimen| and
%   |\dimexpr| instead.
% \end{function}
%
% \begin{function}{skip,eskip,gskip,xskip}
%   \begin{syntax}
%     skip \key\ = \meta{cs} \prefixes2212
%   \end{syntax}
%   The \meta{cs} should be a single control sequence, such as |\foo|.
%   This is just like |int| but uses a skip register, |\newskip| and |\glueexpr|
%   instead.
% \end{function}
%
% \begin{function}{toks,gtoks,apptoks,gapptoks,pretoks,gpretoks}
%   \begin{syntax}
%     toks \key\ = \meta{cs} \prefixes2212
%   \end{syntax}
%   The \meta{cs} should be a single control sequence, such as |\foo|.
%   Store \val\ inside of a toks-register. The |g| variants use |\global|, the
%   |app| variants append \val\ to the contents of that register, the |pre|
%   variants will prepend \val. If \meta{cs} is not yet defined it will be
%   initialised with |\newtoks|.
% \end{function}
%
% \begin{function}{box,gbox}
%   \begin{syntax}
%     box \key\ = \meta{cs} \prefixes2212
%   \end{syntax}
%   The \meta{cs} should be a single control sequence, such as |\foo|.
%   Typesets \val\ into a |\hbox| and stores the result in a box register. The
%   boxes are colour safe. \expkvd\ doesn't provide a |vbox| type.
% \end{function}
%
% \begin{function}{meta}
%   \begin{syntax}
%     meta \key\ = \{\kv, \ldots\} \prefixes2222
%   \end{syntax}
%   This key type can set other keys, you can access the \val\ which was passed
%   to \key\ inside the \kv\ list with |#1|. It works by calling a sub-|\ekvset|
%   on the \kv\ list, so a |set| key will only affect that \kv\ list and not the
%   current |\ekvset|. Since it runs in a separate |\ekvset| you can't use
%   |\ekvsneak| using keys or similar macros in the way you normally could.
% \end{function}
%
% \begin{function}{nmeta}
%   \begin{syntax}
%     nmeta \key\ = \{\kv, \ldots\} \prefixes2223
%   \end{syntax}
%   This key type can set other keys, the difference to |meta| is, that this key
%   doesn't take a value, so the \kv\ list is static.
% \end{function}
%
% \begin{function}{smeta}
%   \begin{syntax}
%     smeta \key\ = \marg{set}\{\kv, \ldots\} \prefixes2222
%   \end{syntax}
%   Yet another |meta| variant. An |smeta| key will take a \val\ which you can
%   access using |#1|, but it sets the \kv\ list inside of \set, so is equal to
%   \texttt{\cs[no-index]{ekvset}\marg{set}\{\kv, \ldots\}}.
% \end{function}
%
% \begin{function}{snmeta}
%   \begin{syntax}
%     snmeta \key\ = \marg{set}\{\kv, \ldots\} \prefixes2223
%   \end{syntax}
%   And the last |meta| variant. |snmeta| is a combination of |smeta| and
%   |nmeta|. It doesn't take an argument and sets the \kv\ list inside of \set.
% \end{function}
%
% \begin{function}{set}
%   \begin{syntax}
%     set \key\ = \marg{set} \prefixes2233
%   \end{syntax}
%   This will define \key\ to change the set of the current |\ekvset|
%   invocation to \set. You can omit \set\ (including the equals sign),
%   which is the same as using \texttt{set \key\ = \{\key\}}. The created |set|
%   key will not take a \val. Note that just like in \expkv\ it'll not be
%   checked whether \set\ is defined and you'll get a low-level \TeX\ error if
%   you use an undefined \set.
% \end{function}
%
% \begin{function}{choice}
%   \begin{syntax}
%     choice \key\ = \{\val=\meta{definition}, \ldots\} \prefixes2223
%   \end{syntax}
%   Defines \key\ to be a |choice| key, meaning it will only accept a limited
%   set of values. You should define each possible \val\ inside of the
%   \val=\meta{definition} list. If a defined \val\ is passed to \key\ the
%   \meta{definition} will be left in the input stream. You can make
%   individual values |protected| inside the \val=\meta{definition} list. By
%   default a |choice| key is expandable, an undefined \val\ will throw an error
%   in an expandable way (but see the |unknown-choice| prefix). You can add
%   additional choices after the \key\ was created by using |choice| again for
%   the same \key, redefining choices is possible the same way, but there is no
%   interface to remove certain choices.
% \end{function}
%
% \begin{function}{choice-store}
%   \begin{syntax}
%     choice-store \key\ = \meta{cs}\{\val, \ldots\} \prefixes2223
%   \end{syntax}
%   This is defines a special type of |choice| key that'll store the given
%   choice inside the macro \meta{cs} (so \meta{cs} should be a single control
%   sequence name such as |\foo|). Since storing inside a macro can't be done
%   expandably every choice-code is |\protected|, you might define the
%   |choice-store| key itself as |protected| as well if you want. Since the
%   definition of each choice is predefined with this key type the choice list
%   should just be a comma separated list of valid choices.
% \end{function}
% This means that the following |choice| and |choice-store| keys are equivalent
% at use time:
% \begin{lstlisting}
% \newcommand*\mya{}
% \ekvdefinekeys{example}
%   {
%      choice key1 = {a=\def\mya{a}, b=\def\mya{b}, c=\def\mya{c}}
%     ,choice-store key2 = \mya{a,b,c}
%   }
% \end{lstlisting}
%
% \begin{function}{choice-enum}
%   \begin{syntax}
%     choice-enum \key\ = \meta{cs}\{\val, \ldots\} \prefixes2223
%   \end{syntax}
%   This is similar to |choice-store|, the differences are: \meta{cs} should be
%   a count-register or is initialised as such if the \meta{cs} is undefined
%   (via |\newcount|); instead of the value the position of the value in the
%   given list is stored in this register (zero-based).
% \end{function}
% This means that the following |choice| and |choice-enum| keys are equivalent
% at use time:
% \begin{lstlisting}
% \newcount\myb
% \ekvdefinekeys{example}
%   {
%      choice key1 = {a={\myb=0 }, b={\myb=1 }, c={\myb=2 }}
%     ,choice-enum key2 = \myb{a,b,c}
%   }
% \end{lstlisting}
%
% \begin{function}{unknown-choice}
%   \begin{syntax}
%     unknown-choice \key\ = \marg{definition} \prefixes2323
%   \end{syntax}
%   By default an unknown \val\ passed to a |choice| or |bool| key will throw an
%   error.  However, with this prefix you can define an alternative action which
%   should be executed if \key\ received an unknown choice. In \meta{definition}
%   you can refer to the choice which was passed in with |#1|.
% \end{function}
%
% \begin{function}{unknown code}
%   \begin{syntax}
%     unknown code = \marg{definition} \prefixes2322
%   \end{syntax}
%   By default \expkv\ throws errors when it encounters unknown keys in a set.
%   With the |unknown| prefix you can define handlers that deal with undefined
%   keys, instead of a \key\ name you have to specify a subtype for this prefix,
%   here the subtype is |code|.\par
%   With |unknown code| the \meta{definition} is used for unknown keys which
%   were provided a value (so corresponds to |\ekvdefunknown|), you can access
%   the key name with |#1| and the value with |#2|.\footnotemark
% \end{function}
% \footnotetext
%   {^^A
%     There is some trickery involved to get this more intuitive argument order
%     without any performance hit if you compare this to
%     \cs[no-index]{ekvdefunknown} directly.^^A
%   }
%
% \begin{function}{unknown noval}
%   \begin{syntax}
%     unknown noval = \marg{definition} \prefixes2323
%   \end{syntax}
%   This is like |unknown code| but uses \meta{definition} for unknown keys to
%   which no value was passed (so corresponds to |\ekvdefunknownNoVal|). You can
%   access the key name with |#1|.
% \end{function}
%
% \begin{function}{unknown redirect-code}
%   \begin{syntax}
%     unknown redirect-code = \marg{set-list} \prefixes2331
%   \end{syntax}
%   This uses a predefined action for |unknown code|. Instead of throwing an
%   error, it is tried to find the \key\ in each \set\ in the comma separated
%   \meta{set-list}. The first found match will be used and the remaining
%   options from the list discarded. If the \key\ isn't found in any \set\ an
%   expandable error will be thrown eventually. Internally \expkv's
%   |\ekvredirectunknown| will be used.
% \end{function}
%
% \begin{function}{unknown redirect-noval}
%   \begin{syntax}
%     unknown redirect-noval = \marg{set-list} \prefixes2333
%   \end{syntax}
%   This behaves just like |unknown redirect-code| but will set up means to
%   forward keys for |unknown noval|. Internally \expkv's
%   |\ekvredirectunknownNoVal| will be used.
% \end{function}
%
% \begin{function}{unknown redirect}
%   \begin{syntax}
%     unknown redirect = \marg{set-list} \prefixes2333
%   \end{syntax}
%   This is a short cut to apply both, |unknown redirect-code| and
%   |unknown redirect-noval|, as a result you might get doubled error messages,
%   one from each.
% \end{function}
%
% \subsection{Bugs}
% I don't think there are any (but every developer says that), if you find some
% please let me know, either via the email address on the first page or on
% GitHub:
% \url{https://github.com/Skillmon/tex_expkv-def}
%
% \subsection{Example}
% The following is an example code defining each base key-type once. Please
% admire the very creative key-name examples.
%
% \begin{lstlisting}
% \ekvdefinekeys{example}
%   {
%      long code keyA = #1
%     ,noval     keyA = NoVal given
%     ,bool      keyB = \keyB
%     ,boolTF    keyC = \keyC
%     ,store     keyD = \keyD
%     ,data      keyE = \keyE
%     ,dataT     keyF = \keyF
%     ,int       keyG = \keyG
%     ,dimen     keyH = \keyH
%     ,skip      keyI = \keyI
%     ,toks      keyJ = \keyJ
%     ,default   keyJ = \empty test
%     ,new box   keyK = \keyK
%     ,qdefault  keyK = K
%     ,choice    keyL =
%       {
%          protected 1 = \texttt{a}
%         ,2 = b
%         ,3 = c
%         ,4 = d
%         ,5 = e
%       }
%     ,edefault  keyL = 2
%     ,meta      keyM = {keyA={#1},keyB=false}
%     ,invbool   keyN = \keyN
%     ,boolpair  keyO = \keyOa\keyOb
%   }
% \end{lstlisting}
%
% \bigskip
% Since the |data| type might be a bit strange, here is another usage example
% for it.
% \begin{lstlisting}
% \ekvdefinekeys{ex}
%   {
%      data  name  = \Pname
%     ,data  age   = \Page
%     ,dataT hobby = \Phobby
%   }
% \newcommand\Person[1]
%   {%
%     \begingroup
%     \ekvset{ex}{#1}%
%     \begin{description}
%       \item[\Pname{}{\errmessage{A person requires a name}}]
%       \item[Age] \Page{\textit}{\errmessage{A person requires an age}}
%       \Phobby{\item[Hobbies]}
%     \end{description}
%     \endgroup
%   }
% \Person{name=Jonathan P. Spratte, age=young, hobby=\TeX\ coding}
% \Person{name=Some User, age=unknown, hobby=Reading Documentation}
% \Person{name=Anybody, age=any}
% \end{lstlisting}
% In this example a person should have a name and an age, but doesn't have to
% have hobbies. The name will be displayed as the description item and the age
% in Italics. If a person has no hobbies the description item will be silently
% left out.  The result of the above code looks like this:
%
% \begin{exresult}{\linewidth}
% \ekvdefinekeys{ex}
%   {
%      data  name  = \Pname
%     ,data  age   = \Page
%     ,dataT hobby = \Phobby
%   }
% \newcommand\Person[1]
%   {%
%     \begingroup
%     \ekvset{ex}{#1}%
%     \begin{description}
%       \item[\Pname{}{\errmessage{A person requires a name}}]
%       \item[Age] \Page{\textit}{\errmessage{A person requires an age}}
%       \Phobby{\item[Hobbies]}
%     \end{description}
%     \endgroup
%   }
% \Person{name=Jonathan P. Spratte, age=young, hobby=\TeX\ coding}
% \Person{name=Some User, age=unknown, hobby=Reading Documentation}
% \Person{name=Anybody, age=any}
% \end{exresult}
%
% \subsection{License}
%
% Copyright \textcopyright\ 2020\unless\ifnum\year=2020--\the\year\fi\
% Jonathan P. Spratte
%
% \medskip\noindent
% This work may be distributed and/or modified under the conditions of the
% \LaTeX\ Project Public License (LPPL), either version 1.3c of this license or
% (at your option) any later version. The latest version of this license is in
% the file:
%
% \url{http://www.latex-project.org/lppl.txt}
%
% \noindent
% This work is ``maintained'' (as per LPPL maintenance status) by
%
% \mbox{Jonathan P. Spratte}.
%
% \end{documentation}^^A=<<
%
% \begin{implementation}^^A>>=
%
% \clearpage
%
% \section{Implementation}^^A>>=
%^^A the LaTeX package >>=
% \subsection{The \LaTeX\ Package}
% Just like for \expkv\ we provide a small \LaTeX\ package that sets up things
% such that we behave nicely on \LaTeX\ packages and files system. It'll
% |\input| the generic code which implements the functionality.
% \gobbledocstriptag
%<*pkg>
%    \begin{macrocode}
\RequirePackage{expkv}
\def\ekvd@tmp
  {%
    \ProvidesFile{expkv-def.tex}%
      [\ekvdDate\space v\ekvdVersion\space a key-defining frontend for expkv]%
  }
% \iffalse meta-comment
%
% File: expkv-def.dtx Copyright (C) 2020-2021 Jonathan P. Spratte
%
% This work  may be  distributed and/or  modified under  the conditions  of the
% LaTeX Project Public License (LPPL),  either version 1.3c  of this license or
% (at your option) any later version.  The latest version of this license is in
% the file:
%
%   http://www.latex-project.org/lppl.txt
%
% ------------------------------------------------------------------------------
%
%<*driver>^^A>>=
\def\nameofplainTeX{plain}
\ifx\fmtname\nameofplainTeX\else
  \expandafter\begingroup
\fi
\input l3docstrip.tex
\askforoverwritefalse
\preamble

--------------------------------------------------------------
expkv-def -- a key-defining frontend for expkv
E-mail: jspratte@yahoo.de
Released under the LaTeX Project Public License v1.3c or later
See http://www.latex-project.org/lppl.txt
--------------------------------------------------------------

Copyright (C) 2020-2021 Jonathan P. Spratte

This  work may be  distributed and/or  modified under  the conditions  of the
LaTeX Project Public License (LPPL),  either version 1.3c  of this license or
(at your option) any later version.  The latest version of this license is in
the file:

  http://www.latex-project.org/lppl.txt

This work is "maintained" (as per LPPL maintenance status) by
  Jonathan P. Spratte.

This work consists of the file  expkv-def.dtx
and the derived files           expkv-def.pdf
                                expkv-def.sty
                                expkv-def.tex
                                t-expkv-def.tex

\endpreamble
% stop docstrip adding \endinput
\postamble
\endpostamble
\generate{\file{expkv-def.sty}{\from{expkv-def.dtx}{pkg}}}
\generate{\file{expkv-def.tex}{\from{expkv-def.dtx}{tex}}}
\generate{\file{t-expkv-def.tex}{\from{expkv-def.dtx}{ctx}}}
\ifx\fmtname\nameofplainTeX
  \expandafter\endbatchfile
\else
  \expandafter\endgroup
\fi
%
\IfFileExists{expkv-def.tex}{% \iffalse meta-comment
%
% File: expkv-def.dtx Copyright (C) 2020-2021 Jonathan P. Spratte
%
% This work  may be  distributed and/or  modified under  the conditions  of the
% LaTeX Project Public License (LPPL),  either version 1.3c  of this license or
% (at your option) any later version.  The latest version of this license is in
% the file:
%
%   http://www.latex-project.org/lppl.txt
%
% ------------------------------------------------------------------------------
%
%<*driver>^^A>>=
\def\nameofplainTeX{plain}
\ifx\fmtname\nameofplainTeX\else
  \expandafter\begingroup
\fi
\input l3docstrip.tex
\askforoverwritefalse
\preamble

--------------------------------------------------------------
expkv-def -- a key-defining frontend for expkv
E-mail: jspratte@yahoo.de
Released under the LaTeX Project Public License v1.3c or later
See http://www.latex-project.org/lppl.txt
--------------------------------------------------------------

Copyright (C) 2020-2021 Jonathan P. Spratte

This  work may be  distributed and/or  modified under  the conditions  of the
LaTeX Project Public License (LPPL),  either version 1.3c  of this license or
(at your option) any later version.  The latest version of this license is in
the file:

  http://www.latex-project.org/lppl.txt

This work is "maintained" (as per LPPL maintenance status) by
  Jonathan P. Spratte.

This work consists of the file  expkv-def.dtx
and the derived files           expkv-def.pdf
                                expkv-def.sty
                                expkv-def.tex
                                t-expkv-def.tex

\endpreamble
% stop docstrip adding \endinput
\postamble
\endpostamble
\generate{\file{expkv-def.sty}{\from{expkv-def.dtx}{pkg}}}
\generate{\file{expkv-def.tex}{\from{expkv-def.dtx}{tex}}}
\generate{\file{t-expkv-def.tex}{\from{expkv-def.dtx}{ctx}}}
\ifx\fmtname\nameofplainTeX
  \expandafter\endbatchfile
\else
  \expandafter\endgroup
\fi
%
\IfFileExists{expkv-def.tex}{\input{expkv-def.tex}}{}
\ProvidesFile{expkv-def.dtx}
  [\csname ekvdDate\endcsname\ a key-defining frontend for expkv]
\PassOptionsToPackage{full}{textcomp}
\documentclass{l3doc}
\RequirePackage[oldstylenums,nott]{kpfonts}
\input{glyphtounicode}
\pdfgentounicode=1
\RequirePackage{tcolorbox}
\newtcolorbox{exresult}[2][]
  {colback=expkvgrey!10!white,colframe=expkvgrey,fontupper=\small,width={#2},#1}
\newtcbox\exres[1][]
  {
     colback=expkvgrey!10!white
    ,colframe=expkvgrey
    ,size=small
    ,nobeforeafter
    ,tcbox raise base
    ,fontupper=\small
    ,#1
  }
\RequirePackage{listings}
\RequirePackage{booktabs}
\RequirePackage{array}
\RequirePackage{collcell}
\RequirePackage{siunitx}
\RequirePackage{xcolor}
\RequirePackage{caption}
\RequirePackage{microtype}
\RequirePackage{accsupp}
\lstdefinelanguage{expkvd}[primitive]{TeX}
  {
    ,moretexcs=[4]^^A e-TeX
      {
        expanded,
        numexpr,
        protected,
      }
    ,moretexcs=[5]^^A plain/LaTeX
      {
        approx,
        begin,
        empty,
        item,
        LaTeX,
        makeatletter,makeatother,
        newcommand,newdimen,
        RequirePackage,
        rule,
        TeX,
        textit,texttt,
        usepackage,
      }
    ,moretexcs=[6]^^A used packages
      {
        ^^A xfp
        fpeval,
        ^^A keyval
        setkeys,
        ^^A yax
        defactiveparameter,storevalue,setparameterlist
      }
    ,moretexcs=[2]^^A expkv-def macros
      {
        ekvdefinekeys
      }
    ,moretexcs=[3]^^A expkv-pkg macros
      {
        ^^A expkv
        ekvdef,ekvdefNoVal,^^A
        ekvlet,ekvletNoVal,ekvletkv,ekvletkvNoVal,^^A
        ekvdefunknown,ekvdefunknownNoVal,^^A
        ekvredirectunknown,ekvredirectunknownNoVal,^^A
        ekvset,ekvsetSneaked,^^A
        ekvsetdef,ekvsetSneakeddef,ekvsetdefSneaked,^^A
        ekvparse,^^A
        ekvVersion,ekvDate,^^A
        ekvifdefined,ekvifdefinedNoVal,ekvifdefinedset,^^A
        ekvbreak,ekvbreakPreSneak,ekvbreakPostSneak,^^A
        ekvsneak,ekvsneakPre,^^A
        ekvchangeset,^^A
        ekvoptarg,ekvoptargTF,^^A
        ekverr,
        ^^A expkv-cs
        ekvcSecondaryKeys,ekvcSplitAndForward,
      }
  }
\colorlet{codeparam}{cyan!65!black}
\def\mylstwd{.55em}
\lstset
  {
    ,language=expkvd
    ,flexiblecolumns=false
    ,basewidth=\mylstwd
    ,gobble=2
    ,basicstyle=\fontfamily{jkp}\itshape
    ,texcsstyle=*[1]{\bfseries\color{expkvgrey}} ^^A primitives
    ,texcsstyle=*[2]{\bfseries\color{expkvred}}  ^^A expkv
    ,texcsstyle=*[3]{\color{expkvred}}           ^^A expkv-pkg
    ,texcsstyle=*[4]{\bfseries\color{expkvgrey}} ^^A e-TeX
    ,texcsstyle=*[5]{\bfseries\color{expkvgrey}} ^^A plain/LaTeX macros
    ,texcsstyle=*[6]{}                           ^^A macros of other packages
    ,commentstyle=\color[gray]{0.4}
    ,literate=
              {\{} {{\CodeUpBf\{}}{1}
              {\}} {{\CodeUpBf\}}}{1}
              {$}  {{\CodeUpBf\$}}{1}
              {[}  {{\CodeUp[}}{1}
              {]}  {{\CodeUp]}}{1}
              {(}  {{\CodeUp(}}{1}
              {)}  {{\CodeUp)}}{1}
              {*}  {{$*$}}{1}
              {1}  {{\CodeUp{1}}}{1}
              {2}  {{\CodeUp{2}}}{1}
              {3}  {{\CodeUp{3}}}{1}
              {4}  {{\CodeUp{4}}}{1}
              {5}  {{\CodeUp{5}}}{1}
              {6}  {{\CodeUp{6}}}{1}
              {7}  {{\CodeUp{7}}}{1}
              {8}  {{\CodeUp{8}}}{1}
              {9}  {{\CodeUp{9}}}{1}
              {0}  {{\CodeUp{0}}}{1}
              {##} {{\CodeColored{codeparam}{1}{\#}}}{1}
              {##1}{{\CodeColored{codeparam}{2}{\#1}}}{2}
              {##2}{{\CodeColored{codeparam}{2}{\#2}}}{2}
              {##3}{{\CodeColored{codeparam}{2}{\#3}}}{2}
              {##4}{{\CodeColored{codeparam}{2}{\#4}}}{2}
              {##5}{{\CodeColored{codeparam}{2}{\#5}}}{2}
              {##6}{{\CodeColored{codeparam}{2}{\#6}}}{2}
              {##7}{{\CodeColored{codeparam}{2}{\#7}}}{2}
              {##8}{{\CodeColored{codeparam}{2}{\#8}}}{2}
              {##9}{{\CodeColored{codeparam}{2}{\#9}}}{2}
              {<key>}{{$\langle$}key{$\rangle$}}{5}
              {<set>}{{$\langle$}set{$\rangle$}}{5}
              {<set1>}{{$\langle$}set1{$\rangle$}}{6}
              {<set2>}{{$\langle$}set2{$\rangle$}}{6}
    ^^A,literate=*{<key>}{\key}{4}{<set>}{\set}{4}
  }
\newcommand*\CodeColored[3]{\textcolor{#1}{\makebox[\dimexpr\mylstwd*#2]{#3}}}
\newcommand*\CodeUpBf[1]{\makebox[\mylstwd]{\textup{\textbf{#1}}}}
\newcommand*\CodeUp[1]{\makebox[\mylstwd]{\textup{#1}}}
\RequirePackage{randtext}
\let\metaORIG\meta
\protected\def\meta #1{\texttt{\metaORIG{#1}}}
\renewcommand*\thefootnote{\fnsymbol{footnote}}
\definecolor{expkvred}{HTML}{9F393D}
\colorlet{expkvgrey}{black!75}
\makeatletter
\newcommand*\expkv
  {^^A
    \texorpdfstring
      {^^A
        \mbox
          {^^A
            \BeginAccSupp{ActualText=expkv}^^A
            \href{https://github.com/Skillmon/tex_expkv}
              {^^A
                \rmfamily
                \bfseries
                {\color{expkvgrey}e\kern-.05em x\kern-.05em}^^A
                \lower.493ex
                  \hbox{{\color{expkvgrey}P}\kern-.1em{\color{expkvred}k}}^^A
                \kern-.18em{\color{expkvred}v}^^A
              }^^A
            \EndAccSupp{}^^A
          }^^A
      }
      {expkv}^^A
  }
\newcommand*\expkvd
  {^^A
    \texorpdfstring
      {^^A
        \mbox
          {^^A
            \BeginAccSupp{ActualText=expkv-def}^^A
            \href{https://github.com/Skillmon/tex_expkv-def}
              {^^A
                \rmfamily
                \bfseries
                {\color{expkvgrey}e\kern-.05em x\kern-.05em}^^A
                \lower.493ex
                  \hbox{{\color{expkvgrey}P}\kern-.1em{\color{expkvred}k}}^^A
                \kern-.18em{\color{expkvred}v}^^A
                {\color{expkvgrey}^^A
                  \kern.05em\rule[-.1ex]{.08em}{1.2ex}\kern.05em\textsc{def}^^A
                }^^A
              }^^A
            \EndAccSupp{}^^A
          }^^A
      }
      {expkv-def}^^A
  }
\newcommand\kv{\meta{key}=\meta{value}}
\newcommand\key{\meta{key}}
\newcommand\val{\meta{value}}
\newcommand\set{\meta{set}}
\newcommand\enfprefix{\textcolor{black}}
\newcommand\allprefix{\textcolor{gray}}
\newcommand\notprefix{\textcolor{red!80!black}}
\newcommand\printprefix[2]
  {^^A
    \ifcase\numexpr#1\relax\or
      \enfprefix{#2}\or
      \allprefix{#2}\or
      \notprefix{#2}\fi
  }
\newcommand\prefixes[4]
  {^^A
    \hfill
    \printprefix{#1}{new}
    \printprefix{#2}{also}
    \printprefix{#3}{protected}
    \printprefix{#4}{long}^^A
  }
\hypersetup{linkcolor=red!80!black,urlcolor=purple!80!black}
\DoNotIndex{\def,\edef,\,,\=,\begingroup,\catcode,\chardef,\csname,\endcsname}
\DoNotIndex{\endgroup,\endinput,\errmessage,\expandafter,\input,\let,\long}
\DoNotIndex{\protected,\ProvidesFile,\ProvidesPackage,\relax,\space}
\DoNotIndex{\@,\unexpanded,\string,\expanded,\dimexpr,\global,\glueexpr,\hbox}
\DoNotIndex{\numexpr,\RequirePackage,\setbox,\the,\unless,\xdef,\@firstofone}
\DoNotIndex{\@firstoftwo,\@gobble,\@secondoftwo,\meaning,\detokenize}
\DoNotIndex{\romannumeral,\^}
\DoNotIndex{\ifcsname}
\DoNotIndex{\ifx}
\DoNotIndex{\ifdefined}
\DoNotIndex{\iffalse}
\DoNotIndex{\iftrue}
\DoNotIndex{\else}
\DoNotIndex{\fi}
\@gobble\fi ^^A ignoring \ifx, \ifcsname, etc., but only one \fi
\@gobble\fi ^^A ignoring \ifx, \ifcsname, etc., but only one \fi
\@gobble\fi ^^A ignoring \ifx, \ifcsname, etc., but only one \fi
\@gobble\fi ^^A ignoring \ifx, \ifcsname, etc., but only one \fi
\@ifdefinable\gobbledocstriptag{\def\gobbledocstriptag#1>{}}
\makeatother
\begin{document}
  \title
    {^^A
      \texorpdfstring
        {^^A
          \huge\expkvd
          \\[\medskipamount]
          \Large a key-defining frontend for \expkv
        }{expkv-def - a key-defining frontend for expkv}^^A
    }
  \date{\ekvdDate\space v\ekvdVersion}
  \author{Jonathan P. Spratte\thanks{\protect\randomize{jspratte@yahoo.de}}}
  \DocInput{expkv-def.dtx}
\end{document}
%</driver>^^A=<<
% \fi
%
% \maketitle
% \renewcommand*\thefootnote{\arabic{footnote}}
%
% \begin{abstract}
% \noindent\parfillskip=0pt
% \expkvd\ provides a small \kv\ interface to define keys for \expkv. Key-types
% are declared using prefixes, similar to static typed languages. The stylised
% name is \expkvd\ but the files use \file{expkv-def}, this is due to CTAN-rules
% which don't allow \string| in package names since that is the pipe symbol in
% *nix shells.
% \end{abstract}
%
% \tableofcontents
%
% \begin{documentation}^^A>>=
%
% \section{Documentation}
%
% Since the trend for the last couple of years goes to defining keys for a \kv\
% interface using a \kv\ interface, I thought that maybe providing such an
% interface for \expkv\ will make it more attractive for actual use, besides its
% unique selling points of being fully expandable, and fast and reliable. But at
% the same time I don't want to widen \expkv's initial scope. So here it is
% \expkvd, go define \kv\ interfaces with \kv\ interfaces.
%
% Unlike many of the other established \kv\ interfaces to define keys, \expkvd\
% works using prefixes instead of suffixes (\emph{e.g.}, |.tl_set:N| of
% \pkg{l3keys}) or directory like handlers (\emph{e.g.}, |/.store in| of
% \pkg{pgfkeys}). This was decided as a personal preference, more over in \TeX\
% parsing for the first space is way easier than parsing for the last one.
% \expkvd's prefixes are sorted into two categories: |p|-type, which are
% equivalent to \TeX's prefixes like |\long|, and |t|-type defining the type of
% the key. For a description of the available |p|-prefixes take a look at
% \autoref{sec:prefix:p}, the |t|-prefixes are described in
% \autoref{sec:prefix:t}.
%
% \expkvd\ is usable as generic code, as a \LaTeX\ package, and as a \ConTeXt\
% module. It'll automatically load \expkv\ in the same mode as well. To use it,
% just use one of
% \begin{lstlisting}
% \input expkv-def       % plainTeX
% \usepackage{expkv-def} % LaTeX
% \usemodule[expkv-def]  % ConTeXt
% \end{lstlisting}
%
% \subsection{Macros}
% Apart from version and date containers there is only a single user-facing
% macro, and that should be used to define keys.
%
% \begin{function}{\ekvdefinekeys}
%   \begin{syntax}
%     \cs{ekvdefinekeys}\marg{set}\{\kv, \ldots\}
%   \end{syntax}
%   In \set, define \key\ to have definition \val. The general syntax for \key\
%   should be
%   \begin{quote}\ttfamily\small
%     \meta{prefix} \meta{name}
%   \end{quote}
%   Where \meta{prefix} is a space separated list of optional |p|-type prefixes
%   followed by one |t|-type prefix. The syntax of \val\ is dependent on the
%   used |t|-prefix.
% \end{function}
%
% \begin{function}{\ekvdDate,\ekvdVersion}
%   These two macros store the version and date of the package.
% \end{function}
%
% \subsection{Prefixes}
% As already said there are |p|-prefixes and |t|-prefixes. Not every |p|-prefix
% is allowed for all |t|-prefixes.
%
% \subsubsection{\texttt{p}-Prefixes}\label{sec:prefix:p}
% The two |p|-type prefixes |long| and |protected| are pretty simple by nature,
% so their description is pretty simple. They affect the \key\ at use-time, so
% omitting |long| doesn't mean that a \meta{definition} can't contain a |\par|
% token, only that the \key\ will not accept a |\par| in \val.  On the other
% hand |new| and |also| might be simple on first sight as well, but their rules
% are a bit more complicated.
%
% \begin{function}{also}
%   The following key type will be \emph{added} to an existing \key's
%   definition. You can't add a type taking an argument at use time to an
%   existing key which doesn't take an argument and vice versa. Also you'll get
%   an error if you try to add an action which isn't allowed to be either |long|
%   or |protected| to a key which already is |long| or |protected| (the opposite
%   order would be suboptimal as well, but can't be really captured with the
%   current code).
%
%   A key already defined as |long| or |protected| will stay |long| or
%   |protected|, but you can as well add |long| or |protected| with the |also|
%   definition.
%
%   As a small example, suppose you want to create a boolean key, but
%   additionally to setting a boolean value you want to execute some more code
%   as well, you can use the following
% \begin{lstlisting}
% \ekvdefinekeys{also-example}
%   {
%      bool key      = \ifmybool
%     ,also code key = \domystuff{#1}
%   }
% \end{lstlisting}
%
%   If you use |also| on a |choice|, |bool|, |invbool|, or |boolpair| key it is
%   tried to determine if the key already is of one of those types. If this test
%   is true the declared choices will be added to the possible choices but the
%   key's definition will not be changed other than that. If that wouldn't have
%   been done, the callbacks of the different choices could get called multiple
%   times.
% \end{function}
%
% \begin{function}{protected,protect}
%   The following key will be defined |\protected|. Note that key-types which
%   can't be defined expandable will always use |\protected|.
% \end{function}
%
% \begin{function}{long}
%   The following key will be defined |\long|.
% \end{function}
%
% \begin{function}{new}
%   The following key must be new (so previously undefined). An error is thrown
%   if it is already defined and the new definition is ignored. |new| only
%   asserts that there are no conflicts between |NoVal| keys and other |NoVal|
%   keys or value taking keys and other value taking keys. For example you can
%   use the following without an error:
% \begin{lstlisting}
% \ekvdefinekeys{new-example}
%   {
%      code key       = \domystuffwitharg{#1}
%     ,new  noval key = \domystuffwithoutarg
%   }
% \end{lstlisting}
% \end{function}
%
% \subsubsection{\texttt{t}-Prefixes}\label{sec:prefix:t}
% Since the |p|-type prefixes apply to some of the |t|-prefixes automatically
% but sometimes one might be disallowed we need some way to highlight this
% behaviour. In the following an enforced prefix will be printed black
% (\texttt{\enfprefix{protected}}), allowed prefixes will be grey
% (\texttt{\allprefix{protected}}), and disallowed prefixes will be red
% (\texttt{\notprefix{protected}}). This will be put flush-right in the syntax
% showing line.
%
% \begin{function}{code,ecode}
%   \begin{syntax}
%     code \key\ = \marg{definition} \prefixes2222
%   \end{syntax}
%   Define \key\ to expand to \meta{definition}. The \key\ will require a \val\
%   for which you can use |#1| inside \meta{definition}. The |ecode| variant
%   will fully expand \meta{definition} inside an |\edef|.
% \end{function}
%
% \begin{function}{noval,enoval}
%   \begin{syntax}
%     noval \key\ = \marg{definition} \prefixes2223
%   \end{syntax}
%   The |noval| type defines \key\ to expand to \meta{definition}. The \key\
%   will not take a \val. |enoval| fully expands \meta{definition} inside an
%   |\edef|.
% \end{function}
%
% \begin{function}{default,qdefault,odefault,fdefault,edefault}
%   \begin{syntax}
%     default \key\ = \marg{definition} \prefixes3223
%   \end{syntax}
%   This serves to place a default \val\ for a \key\ that takes an argument, the
%   \key\ can be of any argument-grabbing kind, and when used without a \val\
%   it will be passed \meta{definition} instead. The |qdefault| variant will
%   expand the \key's code once, so will be slightly quicker, but not change if
%   you redefine \key. |odefault| is just another name for |qdefault|. The
%   |fdefault| version will expand the key code until a non-expandable token or
%   a space is found, a space would be gobbled.\footnotemark{} The |edefault| on
%   the other hand fully expands the \key-code with \meta{definition} as its
%   argument inside of an |\edef|.
% \end{function}
% \footnotetext
%   {^^A
%     For those familiar with \TeX-coding: This uses a
%     \cs[no-index]{romannumeral}-expansion.
%   }
%
% \begin{function}{initial,oinitial,finitial,einitial}
%   \begin{syntax}
%     initial \key\ = \{\val\} \prefixes3333
%     initial \key
%   \end{syntax}
%   With |initial| you can set an initial \val\ for an already defined argument
%   taking \key. It'll just call the key-macro of \key\ and pass it \val. The
%   |einitial| variant will expand \val\ using an |\edef| expansion prior to
%   passing it to the key-macro and the |oinitial| variant will expand the first
%   token in \val\ once. |finitial| will expand \val\ until a non-expandable
%   token or a space is found, a space would be
%   gobbled.\footnotemark
%
%   If you don't provide a value (and no equals sign) a |noval| \key\ of the
%   same name is called once (or, if you specified a |default| for a value
%   taking key that would be used).
% \end{function}
% \footnotetext{Again using \cs[no-index]{romannumeral}.}
%
% \begin{function}{bool,gbool,boolTF,gboolTF}
%   \begin{syntax}
%     bool \key\ = \meta{cs} \prefixes2223
%   \end{syntax}
%   The \meta{cs} should be a single control sequence, such as |\iffoo|.
%   This will define \key\ to be a boolean key, which only takes the values
%   |true| or |false| and will throw an error for other values. If the key is
%   used without a \val\ it'll have the same effect as if you use
%   \texttt{\key=true}. |bool| and |gbool| will behave like \TeX-ifs so either
%   be |\iftrue| or |\iffalse|. The |boolTF| and |gboolTF| variants will both
%   take two arguments and if true the first will be used else the second, so
%   they are always either |\@firstoftwo| or |\@secondoftwo|. The variants with
%   a leading |g| will set the control sequence globally, the others locally. If
%   \meta{cs} is not yet defined it'll be initialised as the |false| version.
%   Note that the initialisation is \emph{not} done with |\newif|, so you will
%   not be able to do |\footrue| outside of the \kv\ interface, but you could
%   use |\newif| yourself. Even if the \key\ will not be |\protected| the
%   commands which execute the |true| or |false| choice will be, so the usage
%   should be safe in an expansion context (\emph{e.g.}, you can use
%   \texttt{edefault \key\ = false} without an issue to change the default
%   behaviour to execute the |false| choice). Internally a |bool| \key\ is the
%   same as a |choice| key which is set up to handle |true| and |false| as
%   choices.
% \end{function}
%
% \begin{function}{invbool,ginvbool,invboolTF,ginvboolTF}
%   \begin{syntax}
%     bool \key\ = \meta{cs} \prefixes2223
%   \end{syntax}
%   These are inverse boolean keys, they behave like |bool| and friends but set
%   the opposite meaning to the macro \meta{cs} in each case. So if
%   \texttt{key=true} is used |invbool| will set \meta{cs} to |\iffalse| and
%   vice versa.
% \end{function}
%
% \begin{function}{boolpair,gboolpair,boolpairTF,gboolpairTF}
%   \begin{syntax}
%     boolpair \key\ = \meta{cs_1}\meta{cs_2} \prefixes2223
%   \end{syntax}
%   The |boolpair| key type behaves like both |bool| and |invbool|, the
%   \meta{cs_1} will be set to the meaning according to the rules of |bool|, and
%   \meta{cs_2} will be set to the opposite.
% \end{function}
%
% \begin{function}{store,estore,gstore,xstore}
%   \begin{syntax}
%     store \key\ = \meta{cs} \prefixes2212
%   \end{syntax}
%   The \meta{cs} should be a single control sequence, such as |\foo|.
%   This will define \key\ to store \val\ inside of the control sequence. If
%   \meta{cs} isn't yet defined it will be initialised as empty. The variants
%   behave similarly to their |\def|, |\edef|, |\gdef|, and |\xdef|
%   counterparts, but |store| and |gstore| will allow you to store macro
%   parameters inside of them by using |\unexpanded|.
% \end{function}
%
% \begin{function}{data,edata,gdata,xdata}
%   \begin{syntax}
%     data \key\ = \meta{cs} \prefixes2212
%   \end{syntax}
%   The \meta{cs} should be a single control sequence, such as |\foo|. This will
%   define \key\ to store \val\ inside of the control sequence. But unlike the
%   |store| type, the macro \meta{cs} will be a switch at the same time, it'll
%   take two arguments and if \key\ was used expands to the first argument
%   followed by \val\ in braces, if \key\ was not used \meta{cs} will expand to
%   the second argument (so behave like |\@secondoftwo|). The idea is that with
%   this type you can define a key which should be typeset formatted. The
%   |edata| and |xdata| variants will fully expand \val, the |gdata| and |xdata|
%   variants will store \val\ inside \meta{cs} globally. The |p|-prefixes will
%   only affect the key-macro, \meta{cs} will always be expandable and |\long|.
% \end{function}
%
% \begin{function}{dataT,edataT,gdataT,xdataT}
%   \begin{syntax}
%     dataT \key\ = \meta{cs} \prefixes2212
%   \end{syntax}
%   Just like |data|, but instead of \meta{cs} grabbing two arguments it'll only
%   grab one, so by default it'll behave like |\@gobble|, and if a \val\ was
%   given to \key\ the \meta{cs} will behave like |\@firstofone| appended by
%   \marg{value}.
% \end{function}
%
% \begin{function}{int,eint,gint,xint}
%   \begin{syntax}
%     int \key\ = \meta{cs} \prefixes2212
%   \end{syntax}
%   The \meta{cs} should be a single control sequence, such as |\foo|.
%   An |int| key will be a \TeX-count register. If \meta{cs} isn't defined yet,
%   |\newcount| will be used to initialise it. The |eint| and |xint| versions
%   will use |\numexpr| to allow basic computations in their \val. The |gint|
%   and |xint| variants set the register globally.
% \end{function}
%
% \begin{function}{dimen,edimen,gdimen,xdimen}
%   \begin{syntax}
%     dimen \key\ = \meta{cs} \prefixes2212
%   \end{syntax}
%   The \meta{cs} should be a single control sequence, such as |\foo|.
%   This is just like |int| but uses a dimen register, |\newdimen| and
%   |\dimexpr| instead.
% \end{function}
%
% \begin{function}{skip,eskip,gskip,xskip}
%   \begin{syntax}
%     skip \key\ = \meta{cs} \prefixes2212
%   \end{syntax}
%   The \meta{cs} should be a single control sequence, such as |\foo|.
%   This is just like |int| but uses a skip register, |\newskip| and |\glueexpr|
%   instead.
% \end{function}
%
% \begin{function}{toks,gtoks,apptoks,gapptoks,pretoks,gpretoks}
%   \begin{syntax}
%     toks \key\ = \meta{cs} \prefixes2212
%   \end{syntax}
%   The \meta{cs} should be a single control sequence, such as |\foo|.
%   Store \val\ inside of a toks-register. The |g| variants use |\global|, the
%   |app| variants append \val\ to the contents of that register, the |pre|
%   variants will prepend \val. If \meta{cs} is not yet defined it will be
%   initialised with |\newtoks|.
% \end{function}
%
% \begin{function}{box,gbox}
%   \begin{syntax}
%     box \key\ = \meta{cs} \prefixes2212
%   \end{syntax}
%   The \meta{cs} should be a single control sequence, such as |\foo|.
%   Typesets \val\ into a |\hbox| and stores the result in a box register. The
%   boxes are colour safe. \expkvd\ doesn't provide a |vbox| type.
% \end{function}
%
% \begin{function}{meta}
%   \begin{syntax}
%     meta \key\ = \{\kv, \ldots\} \prefixes2222
%   \end{syntax}
%   This key type can set other keys, you can access the \val\ which was passed
%   to \key\ inside the \kv\ list with |#1|. It works by calling a sub-|\ekvset|
%   on the \kv\ list, so a |set| key will only affect that \kv\ list and not the
%   current |\ekvset|. Since it runs in a separate |\ekvset| you can't use
%   |\ekvsneak| using keys or similar macros in the way you normally could.
% \end{function}
%
% \begin{function}{nmeta}
%   \begin{syntax}
%     nmeta \key\ = \{\kv, \ldots\} \prefixes2223
%   \end{syntax}
%   This key type can set other keys, the difference to |meta| is, that this key
%   doesn't take a value, so the \kv\ list is static.
% \end{function}
%
% \begin{function}{smeta}
%   \begin{syntax}
%     smeta \key\ = \marg{set}\{\kv, \ldots\} \prefixes2222
%   \end{syntax}
%   Yet another |meta| variant. An |smeta| key will take a \val\ which you can
%   access using |#1|, but it sets the \kv\ list inside of \set, so is equal to
%   \texttt{\cs[no-index]{ekvset}\marg{set}\{\kv, \ldots\}}.
% \end{function}
%
% \begin{function}{snmeta}
%   \begin{syntax}
%     snmeta \key\ = \marg{set}\{\kv, \ldots\} \prefixes2223
%   \end{syntax}
%   And the last |meta| variant. |snmeta| is a combination of |smeta| and
%   |nmeta|. It doesn't take an argument and sets the \kv\ list inside of \set.
% \end{function}
%
% \begin{function}{set}
%   \begin{syntax}
%     set \key\ = \marg{set} \prefixes2233
%   \end{syntax}
%   This will define \key\ to change the set of the current |\ekvset|
%   invocation to \set. You can omit \set\ (including the equals sign),
%   which is the same as using \texttt{set \key\ = \{\key\}}. The created |set|
%   key will not take a \val. Note that just like in \expkv\ it'll not be
%   checked whether \set\ is defined and you'll get a low-level \TeX\ error if
%   you use an undefined \set.
% \end{function}
%
% \begin{function}{choice}
%   \begin{syntax}
%     choice \key\ = \{\val=\meta{definition}, \ldots\} \prefixes2223
%   \end{syntax}
%   Defines \key\ to be a |choice| key, meaning it will only accept a limited
%   set of values. You should define each possible \val\ inside of the
%   \val=\meta{definition} list. If a defined \val\ is passed to \key\ the
%   \meta{definition} will be left in the input stream. You can make
%   individual values |protected| inside the \val=\meta{definition} list. By
%   default a |choice| key is expandable, an undefined \val\ will throw an error
%   in an expandable way (but see the |unknown-choice| prefix). You can add
%   additional choices after the \key\ was created by using |choice| again for
%   the same \key, redefining choices is possible the same way, but there is no
%   interface to remove certain choices.
% \end{function}
%
% \begin{function}{choice-store}
%   \begin{syntax}
%     choice-store \key\ = \meta{cs}\{\val, \ldots\} \prefixes2223
%   \end{syntax}
%   This is defines a special type of |choice| key that'll store the given
%   choice inside the macro \meta{cs} (so \meta{cs} should be a single control
%   sequence name such as |\foo|). Since storing inside a macro can't be done
%   expandably every choice-code is |\protected|, you might define the
%   |choice-store| key itself as |protected| as well if you want. Since the
%   definition of each choice is predefined with this key type the choice list
%   should just be a comma separated list of valid choices.
% \end{function}
% This means that the following |choice| and |choice-store| keys are equivalent
% at use time:
% \begin{lstlisting}
% \newcommand*\mya{}
% \ekvdefinekeys{example}
%   {
%      choice key1 = {a=\def\mya{a}, b=\def\mya{b}, c=\def\mya{c}}
%     ,choice-store key2 = \mya{a,b,c}
%   }
% \end{lstlisting}
%
% \begin{function}{choice-enum}
%   \begin{syntax}
%     choice-enum \key\ = \meta{cs}\{\val, \ldots\} \prefixes2223
%   \end{syntax}
%   This is similar to |choice-store|, the differences are: \meta{cs} should be
%   a count-register or is initialised as such if the \meta{cs} is undefined
%   (via |\newcount|); instead of the value the position of the value in the
%   given list is stored in this register (zero-based).
% \end{function}
% This means that the following |choice| and |choice-enum| keys are equivalent
% at use time:
% \begin{lstlisting}
% \newcount\myb
% \ekvdefinekeys{example}
%   {
%      choice key1 = {a={\myb=0 }, b={\myb=1 }, c={\myb=2 }}
%     ,choice-enum key2 = \myb{a,b,c}
%   }
% \end{lstlisting}
%
% \begin{function}{unknown-choice}
%   \begin{syntax}
%     unknown-choice \key\ = \marg{definition} \prefixes2323
%   \end{syntax}
%   By default an unknown \val\ passed to a |choice| or |bool| key will throw an
%   error.  However, with this prefix you can define an alternative action which
%   should be executed if \key\ received an unknown choice. In \meta{definition}
%   you can refer to the choice which was passed in with |#1|.
% \end{function}
%
% \begin{function}{unknown code}
%   \begin{syntax}
%     unknown code = \marg{definition} \prefixes2322
%   \end{syntax}
%   By default \expkv\ throws errors when it encounters unknown keys in a set.
%   With the |unknown| prefix you can define handlers that deal with undefined
%   keys, instead of a \key\ name you have to specify a subtype for this prefix,
%   here the subtype is |code|.\par
%   With |unknown code| the \meta{definition} is used for unknown keys which
%   were provided a value (so corresponds to |\ekvdefunknown|), you can access
%   the key name with |#1| and the value with |#2|.\footnotemark
% \end{function}
% \footnotetext
%   {^^A
%     There is some trickery involved to get this more intuitive argument order
%     without any performance hit if you compare this to
%     \cs[no-index]{ekvdefunknown} directly.^^A
%   }
%
% \begin{function}{unknown noval}
%   \begin{syntax}
%     unknown noval = \marg{definition} \prefixes2323
%   \end{syntax}
%   This is like |unknown code| but uses \meta{definition} for unknown keys to
%   which no value was passed (so corresponds to |\ekvdefunknownNoVal|). You can
%   access the key name with |#1|.
% \end{function}
%
% \begin{function}{unknown redirect-code}
%   \begin{syntax}
%     unknown redirect-code = \marg{set-list} \prefixes2331
%   \end{syntax}
%   This uses a predefined action for |unknown code|. Instead of throwing an
%   error, it is tried to find the \key\ in each \set\ in the comma separated
%   \meta{set-list}. The first found match will be used and the remaining
%   options from the list discarded. If the \key\ isn't found in any \set\ an
%   expandable error will be thrown eventually. Internally \expkv's
%   |\ekvredirectunknown| will be used.
% \end{function}
%
% \begin{function}{unknown redirect-noval}
%   \begin{syntax}
%     unknown redirect-noval = \marg{set-list} \prefixes2333
%   \end{syntax}
%   This behaves just like |unknown redirect-code| but will set up means to
%   forward keys for |unknown noval|. Internally \expkv's
%   |\ekvredirectunknownNoVal| will be used.
% \end{function}
%
% \begin{function}{unknown redirect}
%   \begin{syntax}
%     unknown redirect = \marg{set-list} \prefixes2333
%   \end{syntax}
%   This is a short cut to apply both, |unknown redirect-code| and
%   |unknown redirect-noval|, as a result you might get doubled error messages,
%   one from each.
% \end{function}
%
% \subsection{Bugs}
% I don't think there are any (but every developer says that), if you find some
% please let me know, either via the email address on the first page or on
% GitHub:
% \url{https://github.com/Skillmon/tex_expkv-def}
%
% \subsection{Example}
% The following is an example code defining each base key-type once. Please
% admire the very creative key-name examples.
%
% \begin{lstlisting}
% \ekvdefinekeys{example}
%   {
%      long code keyA = #1
%     ,noval     keyA = NoVal given
%     ,bool      keyB = \keyB
%     ,boolTF    keyC = \keyC
%     ,store     keyD = \keyD
%     ,data      keyE = \keyE
%     ,dataT     keyF = \keyF
%     ,int       keyG = \keyG
%     ,dimen     keyH = \keyH
%     ,skip      keyI = \keyI
%     ,toks      keyJ = \keyJ
%     ,default   keyJ = \empty test
%     ,new box   keyK = \keyK
%     ,qdefault  keyK = K
%     ,choice    keyL =
%       {
%          protected 1 = \texttt{a}
%         ,2 = b
%         ,3 = c
%         ,4 = d
%         ,5 = e
%       }
%     ,edefault  keyL = 2
%     ,meta      keyM = {keyA={#1},keyB=false}
%     ,invbool   keyN = \keyN
%     ,boolpair  keyO = \keyOa\keyOb
%   }
% \end{lstlisting}
%
% \bigskip
% Since the |data| type might be a bit strange, here is another usage example
% for it.
% \begin{lstlisting}
% \ekvdefinekeys{ex}
%   {
%      data  name  = \Pname
%     ,data  age   = \Page
%     ,dataT hobby = \Phobby
%   }
% \newcommand\Person[1]
%   {%
%     \begingroup
%     \ekvset{ex}{#1}%
%     \begin{description}
%       \item[\Pname{}{\errmessage{A person requires a name}}]
%       \item[Age] \Page{\textit}{\errmessage{A person requires an age}}
%       \Phobby{\item[Hobbies]}
%     \end{description}
%     \endgroup
%   }
% \Person{name=Jonathan P. Spratte, age=young, hobby=\TeX\ coding}
% \Person{name=Some User, age=unknown, hobby=Reading Documentation}
% \Person{name=Anybody, age=any}
% \end{lstlisting}
% In this example a person should have a name and an age, but doesn't have to
% have hobbies. The name will be displayed as the description item and the age
% in Italics. If a person has no hobbies the description item will be silently
% left out.  The result of the above code looks like this:
%
% \begin{exresult}{\linewidth}
% \ekvdefinekeys{ex}
%   {
%      data  name  = \Pname
%     ,data  age   = \Page
%     ,dataT hobby = \Phobby
%   }
% \newcommand\Person[1]
%   {%
%     \begingroup
%     \ekvset{ex}{#1}%
%     \begin{description}
%       \item[\Pname{}{\errmessage{A person requires a name}}]
%       \item[Age] \Page{\textit}{\errmessage{A person requires an age}}
%       \Phobby{\item[Hobbies]}
%     \end{description}
%     \endgroup
%   }
% \Person{name=Jonathan P. Spratte, age=young, hobby=\TeX\ coding}
% \Person{name=Some User, age=unknown, hobby=Reading Documentation}
% \Person{name=Anybody, age=any}
% \end{exresult}
%
% \subsection{License}
%
% Copyright \textcopyright\ 2020\unless\ifnum\year=2020--\the\year\fi\
% Jonathan P. Spratte
%
% \medskip\noindent
% This work may be distributed and/or modified under the conditions of the
% \LaTeX\ Project Public License (LPPL), either version 1.3c of this license or
% (at your option) any later version. The latest version of this license is in
% the file:
%
% \url{http://www.latex-project.org/lppl.txt}
%
% \noindent
% This work is ``maintained'' (as per LPPL maintenance status) by
%
% \mbox{Jonathan P. Spratte}.
%
% \end{documentation}^^A=<<
%
% \begin{implementation}^^A>>=
%
% \clearpage
%
% \section{Implementation}^^A>>=
%^^A the LaTeX package >>=
% \subsection{The \LaTeX\ Package}
% Just like for \expkv\ we provide a small \LaTeX\ package that sets up things
% such that we behave nicely on \LaTeX\ packages and files system. It'll
% |\input| the generic code which implements the functionality.
% \gobbledocstriptag
%<*pkg>
%    \begin{macrocode}
\RequirePackage{expkv}
\def\ekvd@tmp
  {%
    \ProvidesFile{expkv-def.tex}%
      [\ekvdDate\space v\ekvdVersion\space a key-defining frontend for expkv]%
  }
\input{expkv-def.tex}
\ProvidesPackage{expkv-def}%
  [\ekvdDate\space v\ekvdVersion\space a key-defining frontend for expkv]
%    \end{macrocode}
% \gobbledocstriptag
%</pkg>
%^^A=<<
%^^A the ConTeXt module >>=
% \subsection{The \ConTeXt\ module}
% \gobbledocstriptag
%<*ctx>
%    \begin{macrocode}
\writestatus{loading}{ConTeXt User Module / expkv-def}
\usemodule[expkv]
\unprotect
\input expkv-def.tex
\writestatus{loading}
  {ConTeXt User Module / expkv-def / Version \ekvdVersion\space loaded}
\protect\endinput
%    \end{macrocode}
% \gobbledocstriptag
%</ctx>
%^^A=<<
%^^A main file >>=
% \subsection{The Generic Code}
% The rest of this implementation will be the generic code.
% \gobbledocstriptag
%<*tex>
%
% Load \expkv\ if the package didn't already do so -- since \expkv\ has
% safeguards against being loaded twice this does no harm and the overhead
% isn't that big. Also we reuse some of the internals of \expkv\ to save us from
% retyping them.
%    \begin{macrocode}
\input expkv
%    \end{macrocode}
%
% We make sure that \file{expkv-def.tex} is only input once:
%    \begin{macrocode}
\expandafter\ifx\csname ekvdVersion\endcsname\relax
\else
  \expandafter\endinput
\fi
%    \end{macrocode}
%
% \begin{macro}{\ekvdVersion,\ekvdDate}
% We're on our first input, so lets store the version and date in a macro.
%    \begin{macrocode}
\def\ekvdVersion{0.8c}
\def\ekvdDate{2021-09-20}
%    \end{macrocode}
% \end{macro}
%
% If the \LaTeX\ format is loaded we want to be a good file and report back who
% we are, for this the package will have defined |\ekvd@tmp| to use
% |\ProvidesFile|, else this will expand to a |\relax| and do no harm.
%    \begin{macrocode}
\csname ekvd@tmp\endcsname
%    \end{macrocode}
%
% Store the category code of |@| to later be able to reset it and change it to
% 11 for now.
%    \begin{macrocode}
\expandafter\chardef\csname ekvd@tmp\endcsname=\catcode`\@
\catcode`\@=11
%    \end{macrocode}
% |\ekvd@tmp| will be reused later to handle expansion during the key
% defining. But we don't need it to ever store information long-term after
% \expkvd\ was initialized.
%
% \begin{macro}[internal]{\ekvd@ifprimitive}
%    \begin{macrocode}
\protected\long\def\ekvd@ifprimitive#1%
  {%
    \begingroup
      \edef\ekvd@tmpa{\string #1}%
      \edef\ekvd@tmpb{\meaning#1}%
      \expandafter
    \endgroup
    \ifx\ekvd@tmpa\ekvd@tmpb
      \ekv@fi@firstoftwo
    \fi
    \@secondoftwo
  }
%    \end{macrocode}
% \end{macro}
%
% \begin{macro}[internal]
%   {\ekvd@long,\ekvd@prot,\ekvd@clear@prefixes,\ekvd@ifalso}
% \expkvd\ will use |\ekvd@long|, |\ekvd@prot|, and |\ekvd@ifalso| to store
% whether a key should be defined as |\long| or |\protected| or adds an action
% to an existing key, and we have to clear them for every new key. By default
% |long| and |protected| will just be empty, |ifalso| will be
% |\@secondoftwo|, and |ifnew| will just use its third argument.
%    \begin{macrocode}
\protected\def\ekvd@clear@prefixes
  {%
    \let\ekvd@long\ekv@empty
    \let\ekvd@prot\ekv@empty
    \let\ekvd@ifalso\@secondoftwo
    \long\def\ekvd@ifnew##1##2##3{##3}%
  }
\ekvd@clear@prefixes
%    \end{macrocode}
% \end{macro}
%
% \begin{macro}{\ekvdefinekeys}
% This is the one front-facing macro which provides the interface to define
% keys. It's using |\ekvparse| to handle the \kv\ list, the interpretation will
% be done by |\ekvd@noarg| and |\ekvd@|. The \set\ for which the keys should be
% defined is stored in |\ekvd@set|.
%    \begin{macrocode}
\protected\def\ekvdefinekeys#1%
  {%
    \def\ekvd@set{#1}%
    \ekvparse\ekvd@noarg\ekvd@arg
  }
%    \end{macrocode}
% \end{macro}
%
% \begin{macro}[internal]{\ekvd@noarg,\ekvd@arg,\ekvd@handle}
% |\ekvd@noarg| and |\ekvd@arg| store whether there was a value in the \kv\
% pair. |\ekvd@handle| has to test whether there is a space inside the key and
% if so calls the prefix grabbing routine, else we throw an error and ignore the
% key.
%    \begin{macrocode}
\protected\long\def\ekvd@noarg#1%
  {%
    \let\ekvd@ifnoarg\@firstoftwo
    \expandafter\ekvd@handle\detokenize{#1}\ekvd@stop{}%
  }
\protected\long\def\ekvd@arg#1%
  {%
    \let\ekvd@ifnoarg\@secondoftwo
    \expandafter\ekvd@handle\detokenize{#1}\ekvd@stop
  }
\protected\long\def\ekvd@handle#1\ekvd@stop#2%
  {%
    \ekvd@clear@prefixes
    \def\ekvd@cur{#1}%
    \ekvd@ifspace{#1}%
      {\ekvd@prefix\ekv@mark#1\ekv@stop{#2}}%
      \ekvd@err@missing@type
  }
%    \end{macrocode}
% \end{macro}
%
% \begin{macro}[internal]{\ekvd@prefix,\ekvd@prefix@}
% \expkvd\ separates prefixes into two groups, the first being prefixes in the
% \TeX\ sense (|long| and |protected|) which use |@p@| in their name, the other
% being key-types (|code|, |int|, \emph{etc.}) which use |@t@| instead.
% |\ekvd@prefix| splits at the first space and checks whether its a |@p@|
% or |@t@| type prefix. If it is neither throw an error and gobble the
% definition (the value).
%    \begin{macrocode}
\protected\def\ekvd@prefix#1 {\ekv@strip{#1}\ekvd@prefix@\ekv@mark}
\protected\def\ekvd@prefix@#1#2\ekv@stop
  {%
    \ekv@ifdefined{ekvd@t@#1}%
      {\ekv@strip{#2}{\csname ekvd@t@#1\endcsname}}%
      {%
        \ekv@ifdefined{ekvd@p@#1}%
          {\csname ekvd@p@#1\endcsname\ekvd@prefix@after@p{#2}}%
          {\ekvd@err@undefined@prefix{#1}\@gobble}%
      }%
  }
%    \end{macrocode}
% \end{macro}
%
% \begin{macro}[internal]{\ekvd@prefix@after@p}
% The |@p@| type prefixes are all just modifying a following |@t@| type, so they
% will need to search for another prefix. This is true for all of them, so we
% use a macro to handle this. It'll throw an error if there is no other prefix.
%    \begin{macrocode}
\protected\def\ekvd@prefix@after@p#1%
  {%
    \ekvd@ifspace{#1}%
      {\ekvd@prefix#1\ekv@stop}%
      {\ekvd@err@missing@type\@gobble}%
  }
%    \end{macrocode}
% \end{macro}
%
% \begin{macro}[internal]
%   {\ekvd@p@long,\ekvd@p@protected,\ekvd@p@protect,\ekvd@p@also,\ekvd@p@new}
% Define the |@p@| type prefixes, they all just store some information in a
% temporary macro.
%    \begin{macrocode}
\protected\def\ekvd@p@long{\let\ekvd@long\long}
\protected\def\ekvd@p@protected{\let\ekvd@prot\protected}
\let\ekvd@p@protect\ekvd@p@protected
\protected\def\ekvd@p@also{\let\ekvd@ifalso\@firstoftwo}
\protected\def\ekvd@p@new{\let\ekvd@ifnew\ekvd@assert@new}
%    \end{macrocode}
% \end{macro}
%
% \subsubsection{Key Types}
%
% \begin{macro}[internal]{\ekvd@type@set,\ekvd@t@set}
% The |set| type is quite straight forward, just define a |NoVal| key to call
% |\ekvchangeset|.
%    \begin{macrocode}
\protected\def\ekvd@type@set#1#2%
  {%
    \ekvd@assert@not@long
    \ekvd@assert@not@protected
    \ekvd@ifnew{NoVal}{#1}%
      {%
        \ekv@ifempty{#2}%
          {\ekvd@err@missing@definition}%
          {%
            \ekvd@ifalso
              {%
                \ekv@expargtwice{\ekvd@add@noval{#1}}%
                  {\ekvchangeset{#2}}%
                  \ekvd@assert@not@protected@also
              }%
              {\ekv@expargtwice{\ekvdefNoVal\ekvd@set{#1}}{\ekvchangeset{#2}}}%
          }%
      }%
  }
\protected\def\ekvd@t@set#1#2%
  {%
    \ekvd@ifnoarg
      {\ekvd@type@set{#1}{#1}}%
      {\ekvd@type@set{#1}{#2}}%
  }
%    \end{macrocode}
% \end{macro}
%
% \begin{macro}[internal]{\ekvd@type@noval,\ekvd@t@noval,\ekvd@t@enoval}
% Another pretty simple type, |noval| just needs to assert that there is a
% definition and that |long| wasn't specified.
% There are types where the difference in the variants is so small, that we
% define a common handler for them, those common handlers are named with
% |@type@|. |noval| and |enoval| are so similar that we can use such a |@type@|
% macro, even if we could've done |noval| in a slightly faster way without it.
%    \begin{macrocode}
\protected\long\def\ekvd@type@noval#1#2#3%
  {%
    \ekvd@ifnew{NoVal}{#2}%
      {%
        \ekvd@assert@arg
          {%
            \ekvd@assert@not@long
            \ekvd@prot#1\ekvd@tmp{#3}%
            \ekvd@ifalso
              {\ekv@exparg{\ekvd@add@noval{#2}}\ekvd@tmp{}}%
              {\ekvletNoVal\ekvd@set{#2}\ekvd@tmp}%
          }%
      }%
  }
\protected\def\ekvd@t@noval{\ekvd@type@noval\def}
\protected\def\ekvd@t@enoval{\ekvd@type@noval\edef}
%    \end{macrocode}
% \end{macro}
%
% \begin{macro}[internal]{\ekvd@type@code,\ekvd@t@code,\ekvd@t@ecode}
% |code| is simple as well, |ecode| has to use |\edef| on a temporary macro,
% since \expkv\ doesn't provide an |\ekvedef|.
%    \begin{macrocode}
\protected\long\def\ekvd@type@code#1#2#3%
  {%
    \ekvd@ifnew{}{#2}%
      {%
        \ekvd@assert@arg
          {%
            \ekvd@prot\ekvd@long#1\ekvd@tmp##1{#3}%
            \ekvd@ifalso
              {\ekv@exparg{\ekvd@add@val{#2}}{\ekvd@tmp{##1}}{}}%
              {\ekvlet\ekvd@set{#2}\ekvd@tmp}%
          }%
      }%
  }
\protected\def\ekvd@t@code{\ekvd@type@code\def}
\protected\def\ekvd@t@ecode{\ekvd@type@code\edef}
%    \end{macrocode}
% \end{macro}
%
% \begin{macro}[internal]
%   {
%     \ekvd@type@default,\ekvd@t@default,\ekvd@t@qdefault,\ekvd@t@odefault,
%     \ekvd@t@fdefault
%   }
% |\ekvd@type@default| asserts there was an argument, also the key for which one
% wants to set a default has to be already defined (this is not so important for
% |default|, but |qdefault| requires is). If everything is good, |\edef| a
% temporary macro that expands |\ekvd@set| and the |\csname| for the key, and in
% the case of |qdefault| does the first expansion step of the key-macro.
%    \begin{macrocode}
\protected\long\def\ekvd@type@default#1#2#3#4%
  {%
    \ekvd@assert@arg
      {%
        \ekvifdefined\ekvd@set{#3}%
          {%
            \ekvd@assert@not@new
            \ekvd@assert@not@long
            \ekvd@prot\edef\ekvd@tmp
              {%
                \ekv@unexpanded\expandafter#1%
                  {#2\csname\ekv@name\ekvd@set{#3}\endcsname{#4}}%
              }%
            \ekvd@ifalso
              {\ekv@exparg{\ekvd@add@noval{#3}}\ekvd@tmp{}}%
              {\ekvletNoVal\ekvd@set{#3}\ekvd@tmp}%
          }%
          {\ekvd@err@undefined@key{#3}}%
      }%
  }
\protected\def\ekvd@t@default{\ekvd@type@default{}{}}
\protected\def\ekvd@t@qdefault{\ekvd@type@default{\expandafter\expandafter}{}}
\let\ekvd@t@odefault\ekvd@t@qdefault
\protected\def\ekvd@t@fdefault{\ekvd@type@default{}{\romannumeral`\^^@}}
%    \end{macrocode}
% \end{macro}
%
% \begin{macro}[internal]{\ekvd@t@edefault}
% |edefault| is too different from |default| and |qdefault| to reuse the
% |@type@| macro, as it doesn't need |\unexpanded| inside of |\edef|.
%    \begin{macrocode}
\protected\long\def\ekvd@t@edefault#1#2%
  {%
    \ekvd@assert@arg
      {%
        \ekvifdefined\ekvd@set{#1}%
          {%
            \ekvd@assert@not@new
            \ekvd@assert@not@long
            \ekvd@prot\edef\ekvd@tmp
              {\csname\ekv@name\ekvd@set{#1}\endcsname{#2}}%
            \ekvd@ifalso
              {\ekv@exparg{\ekvd@add@noval{#1}}\ekvd@tmp{}}%
              {\ekvletNoVal\ekvd@set{#1}\ekvd@tmp}%
          }%
          {\ekvd@err@undefined@key{#1}}%
      }%
  }
%    \end{macrocode}
% \end{macro}
%
% \begin{macro}[internal]
%   {\ekvd@t@initial,\ekvd@t@oinitial,\ekvd@t@finitial,\ekvd@t@einitial}
%    \begin{macrocode}
\long\def\ekvd@type@initial#1#2#3#4%
  {%
    \ekvd@assert@not@new
    \ekvd@assert@not@also
    \ekvd@assert@not@long
    \ekvd@assert@not@protected
    \ekvd@ifnoarg
      {%
        \ekvifdefinedNoVal\ekvd@set{#3}%
          {\csname\ekv@name\ekvd@set{#3}N\endcsname}%
          {\ekvd@err@undefined@noval{#3}}%
      }%
      {%
        \ekvifdefined\ekvd@set{#3}%
          {%
            #1{#2#4}%
            \csname\ekv@name\ekvd@set{#3}\expandafter\endcsname\expandafter
              {\ekvd@tmp}%
          }%
          {\ekvd@err@undefined@key{#3}}%
      }%
  }
\def\ekvd@t@initial{\ekvd@type@initial{\def\ekvd@tmp}{}}
\def\ekvd@t@oinitial{\ekvd@type@initial{\ekv@exparg{\def\ekvd@tmp}}{}}
\def\ekvd@t@einitial{\ekvd@type@initial{\edef\ekvd@tmp}{}}
\def\ekvd@t@finitial
  {\ekvd@type@initial{\ekv@exparg{\def\ekvd@tmp}}{\romannumeral`\^^@}}
%    \end{macrocode}
% \end{macro}
%
% \begin{macro}[internal]
%   {
%     \ekvd@type@bool,\ekvd@t@bool,\ekvd@t@gbool,\ekvd@t@boolTF,\ekvd@t@gboolTF,
%     \ekvd@t@invbool,\ekvd@t@ginvbool,\ekvd@t@invboolTF,\ekvd@t@ginvboolTF,
%   }
% The boolean types are a quicker version of a |choice| that accept |true| and
% |false|, and set up the |NoVal| action to be identical to \texttt{\key=true}.
% The |true| and |false| actions are always just |\let|ting the macro in |#7| to
% some other macro (\emph{e.g.}, \cs[no-index]{iftrue}).
%    \begin{macrocode}
\protected\def\ekvd@type@bool#1#2#3#4#5%
  {%
    \ekvd@ifnew{}{#4}%
      {%
        \ekvd@ifnew{NoVal}{#4}%
          {%
            \ekvd@assert@filledarg{#5}%
              {%
                \ekvd@newlet#5#3%
                \ekvd@type@choice{#4}%
                \protected\ekvdefNoVal\ekvd@set{#4}{#1\let#5#2}%
                \protected\expandafter\def
                  \csname\ekvd@choice@name\ekvd@set{#4}{true}\endcsname
                  {#1\let#5#2}%
                \protected\expandafter\def
                  \csname\ekvd@choice@name\ekvd@set{#4}{false}\endcsname
                  {#1\let#5#3}%
              }%
          }%
      }%
  }
\protected\def\ekvd@t@bool{\ekvd@type@bool{}\iftrue\iffalse}
\protected\def\ekvd@t@gbool{\ekvd@type@bool\global\iftrue\iffalse}
\protected\def\ekvd@t@boolTF{\ekvd@type@bool{}\@firstoftwo\@secondoftwo}
\protected\def\ekvd@t@gboolTF{\ekvd@type@bool\global\@firstoftwo\@secondoftwo}
\protected\def\ekvd@t@invbool{\ekvd@type@bool{}\iffalse\iftrue}
\protected\def\ekvd@t@ginvbool{\ekvd@type@bool\global\iffalse\iftrue}
\protected\def\ekvd@t@invboolTF{\ekvd@type@bool{}\@secondoftwo\@firstoftwo}
\protected\def\ekvd@t@ginvboolTF
  {\ekvd@type@bool\global\@secondoftwo\@firstoftwo}
%    \end{macrocode}
% \end{macro}
%
% \begin{macro}[internal]
%   {
%     \ekvd@type@boolpair,
%     \ekvd@t@boolpair,\ekvd@t@gboolpair,\ekvd@t@boolpairTF,\ekvd@t@gboolpairTF
%   }
% The boolean pair types are essentially the same as the boolean types, but set
% two macros instead of one.
%    \begin{macrocode}
\protected\def\ekvd@type@boolpair#1#2#3#4#5#6%
  {%
    \ekvd@ifnew{}{#4}%
      {%
        \ekvd@ifnew{NoVal}{#4}%
          {%
            \ekvd@newlet#5#3%
            \ekvd@newlet#6#2%
            \ekvd@type@choice{#4}%
            \protected\ekvdefNoVal\ekvd@set{#4}{#1\let#5#2#1\let#6#3}%
            \protected\expandafter\def
              \csname\ekvd@choice@name\ekvd@set{#4}{true}\endcsname
              {#1\let#5#2#1\let#6#3}%
            \protected\expandafter\def
              \csname\ekvd@choice@name\ekvd@set{#4}{false}\endcsname
              {#1\let#5#3#1\let#6#2}%
          }%
      }%
  }
\protected\def\ekvd@t@boolpair#1#2%
  {\ekvd@assert@twoargs{#2}{\ekvd@type@boolpair{}\iftrue\iffalse{#1}#2}}
\protected\def\ekvd@t@gboolpair#1#2%
  {\ekvd@assert@twoargs{#2}{\ekvd@type@boolpair\global\iftrue\iffalse{#1}#2}}
\protected\def\ekvd@t@boolpairTF#1#2%
  {%
    \ekvd@assert@twoargs{#2}%
      {\ekvd@type@boolpair{}\@firstoftwo\@secondoftwo{#1}#2}%
  }
\protected\def\ekvd@t@gboolpairTF#1#2%
  {%
    \ekvd@assert@twoargs{#2}%
      {\ekvd@type@boolpair\global\@firstoftwo\@secondoftwo{#1}#2}%
  }
%    \end{macrocode}
% \end{macro}
%
% \begin{macro}[internal]
%   {
%     \ekvd@type@data,\ekvd@t@data,\ekvd@t@gdata,\ekvd@t@dataT,\ekvd@t@gdataT
%   }
%    \begin{macrocode}
\protected\def\ekvd@type@data#1#2#3#4#5#6%
  {%
    \ekvd@ifnew{}{#5}%
      {%
        \ekvd@assert@filledarg{#6}%
          {%
            \ekvd@newlet#6#1%
            \ekvd@ifalso
              {%
                \let\ekvd@prot\protected
                \ekvd@add@val{#5}{\long#2#6####1#3{####1{#4}}}{}%
              }%
              {%
                \protected\ekvd@long\ekvdef\ekvd@set{#5}%
                  {\long#2#6####1#3{####1{#4}}}%
              }%
          }%
      }%
  }
\protected\def\ekvd@t@data
  {\ekvd@type@data\@secondoftwo\edef{####2}{\ekv@unexpanded{##1}}}
\protected\def\ekvd@t@edata{\ekvd@type@data\@secondoftwo\edef{####2}{##1}}
\protected\def\ekvd@t@gdata
  {\ekvd@type@data\@secondoftwo\xdef{####2}{\ekv@unexpanded{##1}}}
\protected\def\ekvd@t@xdata{\ekvd@type@data\@secondoftwo\xdef{####2}{##1}}
\protected\def\ekvd@t@dataT
  {\ekvd@type@data\@gobble\edef{}{\ekv@unexpanded{##1}}}
\protected\def\ekvd@t@edataT{\ekvd@type@data\@gobble\edef{}{##1}}
\protected\def\ekvd@t@gdataT
  {\ekvd@type@data\@gobble\xdef{}{\ekv@unexpanded{##1}}}
\protected\def\ekvd@t@xdataT{\ekvd@type@data\@gobble\xdef{}{##1}}
%    \end{macrocode}
% \end{macro}
%
% \begin{macro}[internal]{\ekvd@type@box,\ekvd@t@box,\ekvd@t@gbox}
% Set up our boxes. Though we're a generic package we want to be colour safe, so
% we put an additional grouping level inside the box contents, for the case that
% someone uses \pkg{color}. |\ekvd@newreg| is a small wrapper which tests
% whether the first argument is defined and if not does
% |\csname new#2\endcsname#1|.
%    \begin{macrocode}
\protected\def\ekvd@type@box#1#2#3%
  {%
    \ekvd@ifnew{}{#2}%
      {%
        \ekvd@assert@filledarg{#3}%
          {%
            \ekvd@newreg#3{box}%
            \ekvd@ifalso
              {%
                \let\ekvd@prot\protected
                \ekvd@add@val{#2}{#1\setbox#3=\hbox{\begingroup##1\endgroup}}{}%
              }%
              {%
                \protected\ekvd@long\ekvdef\ekvd@set{#2}%
                  {#1\setbox#3=\hbox{\begingroup##1\endgroup}}%
              }%
          }%
      }%
  }
\protected\def\ekvd@t@box{\ekvd@type@box{}}
\protected\def\ekvd@t@gbox{\ekvd@type@box\global}
%    \end{macrocode}
% \end{macro}
%
% \begin{macro}[internal]{\ekvd@type@toks,\ekvd@t@toks,\ekvd@t@gtoks}
% Similar to |box|, but set the |toks|.
%    \begin{macrocode}
\protected\def\ekvd@type@toks#1#2#3%
  {%
    \ekvd@ifnew{}{#2}%
      {%
        \ekvd@assert@filledarg{#3}%
          {%
            \ekvd@newreg#3{toks}%
            \ekvd@ifalso
              {%
                \let\ekvd@prot\protected
                \ekvd@add@val{#2}{#1#3={##1}}{}%
              }%
              {\protected\ekvd@long\ekvdef\ekvd@set{#2}{#1#3={##1}}}%
          }%
      }%
  }
\protected\def\ekvd@t@toks{\ekvd@type@toks{}}
\protected\def\ekvd@t@gtoks{\ekvd@type@toks\global}
%    \end{macrocode}
% \end{macro}
%
% \begin{macro}[internal]{\ekvd@type@preapptoks,\ekvd@t@apptoks,\ekvd@t@gapptoks}
% Just like |toks|, but expand the current contents of the |toks| register to
% append the new contents.
%    \begin{macrocode}
\ekvd@ifprimitive\toksapp
  {%
    \protected\def\ekvd@type@preapptoks#1#2#3%
      {%
        \ekvd@ifnew{}{#2}%
          {%
            \ekvd@assert@filledarg{#3}%
              {%
                \ekvd@newreg#3{toks}%
                \ekvd@ifalso
                  {%
                    \let\ekvd@prot\protected
                    \ekvd@add@val{#2}{#1#3{##1}}{}%
                  }%
                  {\protected\ekvd@long\ekvdef\ekvd@set{#2}{#1#3{##1}}}%
              }%
          }%
      }
    \protected\def\ekvd@t@apptoks{\ekvd@type@preapptoks\toksapp}
    \protected\def\ekvd@t@gapptoks{\ekvd@type@preapptoks\gtoksapp}
    \protected\def\ekvd@t@pretoks{\ekvd@type@preapptoks\tokspre}
    \protected\def\ekvd@t@gpretoks{\ekvd@type@preapptoks\gtokspre}
  }
  {%
    \protected\def\ekvd@type@apptoks#1#2#3%
      {%
        \ekvd@ifnew{}{#2}%
          {%
            \ekvd@assert@filledarg{#3}%
              {%
                \ekvd@newreg#3{toks}%
                \ekvd@ifalso
                  {%
                    \let\ekvd@prot\protected
                    \ekvd@add@val{#2}{#1#3=\expandafter{\the#3##1}}{}%
                  }%
                  {%
                    \protected\ekvd@long\ekvdef\ekvd@set{#2}%
                      {#1#3=\expandafter{\the#3##1}}%
                  }%
              }%
          }%
      }
    \protected\def\ekvd@t@apptoks{\ekvd@type@apptoks{}}
    \protected\def\ekvd@t@gapptoks{\ekvd@type@apptoks\global}
    \newtoks\ekvd@toks
    \protected\def\ekvd@type@pretoks#1#2#3%
      {%
        \ekvd@ifnew{}{#2}%
          {%
            \ekvd@assert@filledarg{#3}%
              {%
                \ekvd@newreg#3{toks}%
                \ekvd@ifalso
                  {%
                    \let\ekvd@prot\protected
                    \ekvd@add@val{#2}%
                      {%
                        \ekvd@toks={##1}%
                        #1#3=\expandafter{\the\expandafter\ekvd@toks\the#3}%
                      }%
                      {}%
                  }%
                  {%
                    \protected\ekvd@long\ekvdef\ekvd@set{#2}%
                      {%
                        \ekvd@toks={##1}%
                        #1#3=\expandafter{\the\expandafter\ekvd@toks\the#3}%
                      }%
                  }%
              }%
          }%
      }
    \protected\def\ekvd@t@pretoks{\ekvd@type@pretoks{}}
    \protected\def\ekvd@t@gpretoks{\ekvd@type@pretoks\global}
  }
%    \end{macrocode}
% \end{macro}
%
% \begin{macro}[internal]
%   {
%     \ekvd@type@reg,\ekvd@t@int,\ekvd@t@eint,\ekvd@t@gint,\ekvd@t@xint,
%     \ekvd@t@dimen,\ekvd@t@edimen,\ekvd@t@gdimen,\ekvd@t@xdimen,
%     \ekvd@t@skip,\ekvd@t@eskip,\ekvd@t@gskip,\ekvd@t@xskip
%   }
% The |\ekvd@type@reg| can handle all the types for which the assignment will
% just be \texttt{\meta{register}=\meta{value}}.
%    \begin{macrocode}
\protected\def\ekvd@type@reg#1#2#3#4#5#6%
  {%
    \ekvd@ifnew{}{#5}%
      {%
        \ekvd@assert@filledarg{#6}%
          {%
            \ekvd@newreg#6{#1}%
            \ekvd@ifalso
              {%
                \let\evkd@prot\protected
                \ekvd@add@val{#5}{#2#6=#3##1#4\relax}{}%
              }%
              {\protected\ekvd@long\ekvdef\ekvd@set{#5}{#2#6=#3##1#4\relax}}%
          }%
      }%
  }
\protected\def\ekvd@t@int{\ekvd@type@reg{count}{}{}{}}
\protected\def\ekvd@t@eint{\ekvd@type@reg{count}{}\numexpr\relax}
\protected\def\ekvd@t@gint{\ekvd@type@reg{count}\global{}{}}
\protected\def\ekvd@t@xint{\ekvd@type@reg{count}\global\numexpr\relax}
\protected\def\ekvd@t@dimen{\ekvd@type@reg{dimen}{}{}{}}
\protected\def\ekvd@t@edimen{\ekvd@type@reg{dimen}{}\dimexpr\relax}
\protected\def\ekvd@t@gdimen{\ekvd@type@reg{dimen}\global{}{}}
\protected\def\ekvd@t@xdimen{\ekvd@type@reg{dimen}\global\dimexpr\relax}
\protected\def\ekvd@t@skip{\ekvd@type@reg{skip}{}{}{}}
\protected\def\ekvd@t@eskip{\ekvd@type@reg{skip}{}\glueexpr\relax}
\protected\def\ekvd@t@gskip{\ekvd@type@reg{skip}\global{}{}}
\protected\def\ekvd@t@xskip{\ekvd@type@reg{skip}\global\glueexpr\relax}
%    \end{macrocode}
% \end{macro}
%
% \begin{macro}[internal]{\ekvd@type@store,\ekvd@t@store,\ekvd@t@gstore}
% The none-expanding |store| types use an |\edef| or |\xdef| and |\unexpanded|
% to be able to also store |#| easily.
%    \begin{macrocode}
\protected\def\ekvd@type@store#1#2#3#4%
  {%
    \ekvd@ifnew{}{#3}%
      {%
        \ekvd@assert@filledarg{#4}%
          {%
            \ekvd@newlet#4\ekv@empty
            \ekvd@ifalso
              {%
                \let\ekvd@prot\protected
                \ekvd@add@val{#3}{#1#4{#2}}{}%
              }%
              {\protected\ekvd@long\ekvdef\ekvd@set{#3}{#1#4{#2}}}%
          }%
      }%
  }
\protected\def\ekvd@t@store{\ekvd@type@store\edef{\ekv@unexpanded{##1}}}
\protected\def\ekvd@t@gstore{\ekvd@type@store\xdef{\ekv@unexpanded{##1}}}
\protected\def\ekvd@t@estore{\ekvd@type@store\edef{##1}}
\protected\def\ekvd@t@xstore{\ekvd@type@store\xdef{##1}}
%    \end{macrocode}
% \end{macro}
%
% \begin{macro}[internal]
%   {
%     \ekvd@type@meta,\ekvd@type@meta@a,\ekvd@type@meta@b,\ekvd@type@meta@c,
%     \ekvd@t@meta,\ekvd@t@nmeta
%   }
% |meta| sets up things such that another instance of |\ekvset| will be run on
% the argument, with the same \set.
%    \begin{macrocode}
\protected\long\def\ekvd@type@meta#1#2#3#4#5#6#7%
  {%
    \ekvd@ifnew{#1}{#6}%
      {%
        \ekvd@assert@filledarg{#7}%
          {%
            \edef\ekvd@tmp{\ekvd@set}%
            \expandafter\ekvd@type@meta@a\expandafter{\ekvd@tmp}{#7}{#2}%
            \ekvd@ifalso
              {\ekv@exparg{#3{#6}}{\ekvd@tmp#4}{#5}}%
              {\csname ekvlet#1\endcsname\ekvd@set{#6}\ekvd@tmp}%
          }%
      }%
  }
\protected\long\def\ekvd@type@meta@a#1#2%
  {%
    \expandafter\ekvd@type@meta@b\expandafter{\ekvset{#1}{#2}}%
  }
\protected\def\ekvd@type@meta@b
  {%
    \expandafter\ekvd@type@meta@c\expandafter
  }
\protected\long\def\ekvd@type@meta@c#1#2%
  {%
    \ekvd@prot\ekvd@long\def\ekvd@tmp#2{#1}%
  }
\protected\def\ekvd@t@meta{\ekvd@type@meta{}{##1}\ekvd@add@val{{##1}}{}}
\protected\def\ekvd@t@nmeta
  {%
    \ekvd@assert@not@long
    \ekvd@type@meta{NoVal}{}\ekvd@add@noval{}\ekvd@assert@not@long@also
  }
%    \end{macrocode}
% \end{macro}
%
% \begin{macro}[internal]
%   {\ekvd@type@smeta,\ekvd@type@smeta@,\ekvd@t@smeta,\ekvd@t@snmeta}
% |smeta| is pretty similar to |meta|, but needs two arguments inside of \val,
% such that the first is the \set\ for which the sub-|\ekvset| and the second is
% the \kv\ list.
%    \begin{macrocode}
\protected\long\def\ekvd@type@smeta#1#2#3#4#5#6#7%
  {%
    \ekvd@ifnew{#1}{#6}%
      {%
        \ekvd@assert@twoargs{#7}%
          {%
            \ekvd@type@meta@a#7{#2}%
            \ekvd@ifalso
              {\ekv@exparg{#3{#6}}{\ekvd@tmp#4}{#5}}%
              {\csname ekvlet#1\endcsname\ekvd@set{#6}\ekvd@tmp}%
          }%
      }%
  }
\protected\def\ekvd@t@smeta{\ekvd@type@smeta{}{##1}\ekvd@add@val{{##1}}{}}
\protected\def\ekvd@t@snmeta
  {%
    \ekvd@assert@not@long
    \ekvd@type@smeta{NoVal}{}\ekvd@add@noval{}\ekvd@assert@not@long@also
  }
%    \end{macrocode}
% \end{macro}
%
% \begin{macro}[internal]
%   {
%     \ekvd@type@choice,
%     \ekvd@populate@choice,\ekvd@populate@choice@,\ekvd@populate@choice@noarg,
%     \ekvd@choice@prefix,\ekvd@choice@prefix@,\ekvd@choice@prefix@done,
%     \ekvd@choice@p@protected,\ekvd@choice@p@protect,
%     \ekvd@choice@p@long,\ekvd@choice@p@long@,
%     \ekvd@t@choice
%   }
% The |choice| type is by far the most complex type, as we have to run a
% sub-parser on the choice-definition list, which should support the |@p@| type
% prefixes as well (but |long| will always throw an error, as they are not
% allowed to be long). |\ekvd@type@choice| will just define the choice-key, the
% handling of the choices definition will be done by |\ekvd@populate@choice|.
%    \begin{macrocode}
\protected\def\ekvd@type@choice#1%
  {%
    \ekvd@assert@not@long
    \ekv@exparg{\ekvd@prot\def\ekvd@tmp##1}%
      {%
        \expandafter\ekvd@h@choice\expandafter
          {\ekvd@choice@name\ekvd@set{#1}{##1}}%
      }%
    \ekvd@ifalso
      {%
        \ekvd@assert@val{#1}%
          {%
            \ekvd@if@not@already@choice{#1}%
              {%
                \ekv@exparg
                  {%
                    \expandafter\ekvd@add@aux
                      \csname\ekv@name\ekvd@set{#1}\endcsname{{##1}}{#1}%
                  }%
                  {\ekvd@tmp{##1}}%
                  {\ekvd@long\ekvdef}\ekvd@assert@not@long@also
              }%
          }%
      }%
      {\ekvlet\ekvd@set{#1}\ekvd@tmp}%
  }
%    \end{macrocode}
% |\ekvd@populate@choice| just uses |\ekvparse| and then gives control to
% |\ekvd@populate@choice@noarg|, which throws an error, and
% |\ekvd@populate@choice@|.
%    \begin{macrocode}
\protected\def\ekvd@populate@choice
  {%
    \ekvparse\ekvd@populate@choice@noarg\ekvd@populate@choice@
  }
\protected\long\def\ekvd@populate@choice@noarg#1%
  {%
    \expandafter\ekvd@err@missing@definition@msg\expandafter{\ekvd@cur : #1}%
  }
%    \end{macrocode}
% |\ekvd@populate@choice@| runs the prefix-test, if there is none we can
% directly define the choice, for that |\ekvd@set@choice| will expand to the
% current choice-key's name, which will have been defined by |\ekvd@t@choice|.
% If there is a prefix run the prefix grabbing routine, which was altered for
% |@type@choice|.
%    \begin{macrocode}
\protected\long\def\ekvd@populate@choice@#1#2%
  {%
    \ekvd@clear@prefixes
    \ekvd@ifspace{#1}%
      {\ekvd@choice@prefix{\ekv@mark#1}\ekv@mark#1\ekv@stop}%
      {%
        \expandafter\edef
          \csname\ekvd@choice@name\ekvd@set\ekvd@set@choice{#1}\endcsname
      }%
      {\unexpanded{#2}}%
  }
\protected\def\ekvd@choice@prefix#1#2
  {%
    \ekv@strip{#2}{\ekvd@choice@prefix@{#1}}\ekv@mark
  }
\protected\def\ekvd@choice@prefix@#1#2#3\ekv@stop
  {%
    \ekv@ifdefined{ekvd@choice@p@#2}%
      {%
        \csname ekvd@choice@p@#2\endcsname
        \ekvd@ifspace{#3}%
          {\ekvd@choice@prefix{#3}#3\ekv@stop}%
          {\ekvd@choice@prefix@done{#3}}%
      }%
      {\ekvd@choice@prefix@done{#1}}%
  }
\protected\def\ekvd@choice@prefix@done#1%
  {%
    \ekvd@prot\expandafter\edef
      \csname
        \ekv@strip{#1}{\ekvd@choice@name\ekvd@set\ekvd@set@choice}%
      \endcsname
  }
\protected\def\ekvd@choice@p@protected{\let\ekvd@prot\protected}
\let\ekvd@choice@p@protect\ekvd@choice@p@protected
\protected\def\ekvd@choice@invalid@p#1\ekvd@ifspace#2%
  {%
    \expandafter\ekvd@choice@invalid@p@\expandafter{\ekv@gobble@mark#2}{#1}%
    \ekvd@ifspace{#2}%
  }
\protected\def\ekvd@choice@invalid@p@#1#2%
  {%
    \expandafter\ekvd@err@no@prefix@msg\expandafter{\ekvd@cur : #2 #1}{#2}%
  }
\protected\def\ekvd@choice@p@long{\ekvd@choice@invalid@p{long}}%
\protected\def\ekvd@choice@p@also{\ekvd@choice@invalid@p{also}}%
\protected\def\ekvd@choice@p@new{\ekvd@choice@invalid@p{new}}%
%    \end{macrocode}
% Finally we're able to set up the |@t@choice| macro, which has to store the
% current choice-key's name, define the key, and parse the available choices.
%    \begin{macrocode}
\protected\long\def\ekvd@t@choice#1#2%
  {%
    \ekvd@ifnew{}{#1}%
      {%
        \ekvd@assert@arg
          {%
            \ekvd@type@choice{#1}%
            \def\ekvd@set@choice{#1}%
            \ekvd@populate@choice{#2}%
          }%
      }%
  }
%    \end{macrocode}
% \end{macro}
%
% \begin{macro}[internal]{\ekvd@t@choice-store,\ekvd@t@choice-enum}
%   These two types define a special kind of |choice| key and are quite similar,
%   the only difference is what the different choices do (hence they use a
%   shared initialisation which differs in the chosen |populate| step).
%    \begin{macrocode}
\protected\long\expandafter\def\csname ekvd@t@choice-store\endcsname
  {\ekvd@type@choicespecial\ekvd@populate@choicestore}
\protected\long\expandafter\def\csname ekvd@t@choice-enum\endcsname
  {\ekvd@type@choicespecial\ekvd@populate@choiceenum}
%    \end{macrocode}
% \begin{macro}[internal]{\ekvd@type@choicespecial}
%   Initialise similar to a |choice| key. The difference is that we require two
%   arguments (which we assert), a macro to store things in, and a |csv|-list
%   containing the allowed values. |#1| is the |populate| macro according to the
%   type used.
%    \begin{macrocode}
\protected\long\def\ekvd@type@choicespecial#1#2#3%
  {%
    \ekvd@ifnew{}{#2}%
      {%
        \ekvd@assert@twoargs{#3}%
          {%
            \ekvd@type@choice{#2}%
            \def\ekvd@set@choice{#2}%
            #1#3%
          }%
      }%
  }
%    \end{macrocode}
% \end{macro}
% \begin{macro}[internal]
%   {\ekvd@populate@choicestore,\ekvd@populate@choicestore@}
%   We initialise the storing macro if it doesn't yet exist, and then we loop
%   over the value list. The |\edef|s with |\unexpanded| are both necessary to
%   be able to store macro parameter tokens (the outer protects at define time,
%   the inner at use time).
%    \begin{macrocode}
\protected\long\def\ekvd@populate@choicestore#1%
  {%
    \ekvd@newlet#1\ekv@empty
    \ekvcsvloop{\ekvd@populate@choicestore@#1}%
  }
\protected\long\def\ekvd@populate@choicestore@#1#2%
  {%
    \protected\expandafter\edef
      \csname\ekvd@choice@name\ekvd@set\ekvd@set@choice{#2}\endcsname
      {\unexpanded{\edef#1{\unexpanded{#2}}}}%
  }
%    \end{macrocode}
% \end{macro}
% \begin{macro}[internal]
%   {\ekvd@populate@choiceenum,\ekvd@populate@choiceenum@}
%   This is similar to the population of a |choice-store| type, but instead of
%   storing the values in a macro this initialises a count and stores the
%   position of the value in the list inside that count (zero-indexed). The
%   space is necessary to terminate the number scanning, which is the reason we
%   use |\@firstofone| (so that the space after the macro name isn't gobbled by
%   \TeX).
%    \begin{macrocode}
\protected\long\def\ekvd@populate@choiceenum#1%
  {%
    \ekvd@newreg#1{count}%
    \def\ekvd@tmp{0}%
    \ekvcsvloop{\ekvd@populate@choiceenum@#1}%
  }
\protected\long\def\ekvd@populate@choiceenum@#1#2%
  {%
    \protected\expandafter\edef
      \csname\ekvd@choice@name\ekvd@set\ekvd@set@choice{#2}\endcsname
      {#1=\@firstofone{\ekvd@tmp} }%
    \edef\ekvd@tmp{\the\numexpr\ekvd@tmp+1\relax}%
  }
%    \end{macrocode}
% \end{macro}
% \end{macro}
%
% \begin{macro}[internal]{\ekvd@t@unknown-choice}
%    \begin{macrocode}
\protected\long\expandafter\def\csname ekvd@t@unknown-choice\endcsname#1#2%
  {%
    \ekvd@assert@new@for@name{\ekvd@unknown@choice@name\ekvd@set{#1}}%
      {%
        \ekvd@assert@arg
          {%
            \ekvd@assert@not@long
            \ekvd@assert@not@also
            \ekvd@prot\expandafter
            \def\csname\ekvd@unknown@choice@name\ekvd@set{#1}\endcsname##1{#2}%
          }%
      }%
  }
%    \end{macrocode}
% \end{macro}
%
% \begin{macro}[internal]
%   {\ekvd@t@unknown,\ekvd@type@unknown@code,\ekvd@type@unknown@noval}
%   The |unknown| type has different subtypes which would be the key names for
%   other types. It is first checked whether that subtype is defined, if it
%   isn't throw an error, else use that subtype.
%    \begin{macrocode}
\protected\long\def\ekvd@t@unknown#1#2%
  {%
    \ekv@ifdefined{ekvd@type@unknown@\detokenize{#1}}%
      {\csname ekvd@type@unknown@\detokenize{#1}\endcsname{#2}}%
      \ekvd@err@misused@unknown
  }
%    \end{macrocode}
%   The |unknown noval| type can use |\ekvdefunknownNoVal| directly (after
%   asserting some prefixes).
%    \begin{macrocode}
\protected\long\def\ekvd@type@unknown@noval#1%
  {%
    \ekvd@assert@new@for@name{\ekv@name\ekvd@set{}uN}%
      {%
        \ekvd@assert@arg
          {%
            \ekvd@assert@not@also
            \ekvd@assert@not@long
            \ekvd@prot\ekvdefunknownNoVal\ekvd@set{#1}%
          }%
      }%
  }
%    \end{macrocode}
%   The |unknown code| type uses some trickery during the definition in order to
%   swap out |#1| and |#2| in the user supplied definition. This is done via a
%   temporary macro that stores the definition but gets the parameter numbers
%   reversed while the real definition is done.
%    \begin{macrocode}
\protected\long\def\ekvd@type@unknown@code#1%
  {%
    \ekvd@assert@new@for@name{\ekv@name\ekvd@set{}u}%
      {%
        \ekvd@assert@arg
          {%
            \ekvd@assert@not@also
            \begingroup
              \def\ekvd@tmp##1##2{#1}%
              \ekv@exparg
              {%
                \endgroup
                \ekvd@prot\ekvd@long\ekvdefunknown\ekvd@set
              }%
              {\ekvd@tmp{##2}{##1}}%
          }%
      }%
  }
%    \end{macrocode}
% \end{macro}
%
% \begin{macro}[internal]
%   {
%     \ekvd@type@unknown@redirect,
%     \ekvd@type@unknown@redirect-code,
%     \ekvd@type@unknown@redirect-noval
%   }
%   The |unknown redirect| types also just forward to |\ekvredirectunknown|
%   after asserting some prefixes.
%    \begin{macrocode}
\protected\edef\ekvd@type@unknown@redirect#1%
  {%
    \expandafter\noexpand\csname ekvd@type@unknown@redirect-code\endcsname{#1}%
    \expandafter\noexpand\csname ekvd@type@unknown@redirect-noval\endcsname{#1}%
  }
\protected\expandafter\def\csname ekvd@type@unknown@redirect-code\endcsname#1%
  {%
    \ekvd@assert@new@for@name{\ekv@name\ekvd@set{}u}%
      {%
        \ekvd@assert@arg
          {%
            \ekvd@assert@not@also
            \ekvd@assert@not@protected
            \expandafter\ekvredirectunknown\expandafter{\ekvd@set}{#1}%
          }%
      }%
  }
\protected\expandafter\def\csname ekvd@type@unknown@redirect-noval\endcsname#1%
  {%
    \ekvd@assert@new@for@name{\ekv@name\ekvd@set{}uN}%
      {%
        \ekvd@assert@arg
          {%
            \ekvd@assert@not@also
            \ekvd@assert@not@protected
            \ekvd@assert@not@long
            \expandafter\ekvredirectunknownNoVal\expandafter{\ekvd@set}{#1}%
          }%
      }%
  }
%    \end{macrocode}
% \end{macro}
%
% \subsubsection{Key Type Helpers}
% There are some keys that might need helpers during their execution (not during
% their definition, which are gathered as |@type@| macros). These helpers are
% named |@h@|.
%
% \begin{macro}[internal]{\ekvd@h@choice,\ekvd@h@choice@}
% The |choice| helper will just test whether the given choice was defined, if
% not throw an error expandably, else call the macro which stores the code for
% this choice.
%    \begin{macrocode}
\def\ekvd@h@choice#1%
  {%
    \expandafter\ekvd@h@choice@
      \csname\ifcsname#1\endcsname#1\else relax\fi\endcsname
      {#1}%
  }
\def\ekvd@h@choice@#1#2%
  {%
    \ifx#1\relax
      \ekvd@err@choice@invalid{#2}%
      \expandafter\@gobble
    \fi
    #1%
  }
%    \end{macrocode}
% \end{macro}
%
%
% \subsubsection{Handling \texttt{also}}
%
% \begin{macro}[internal]
%   {\ekvd@add@val,\ekvd@add@noval,\ekvd@add@aux,\ekvd@add@aux@}
%    \begin{macrocode}
\protected\long\def\ekvd@add@val#1#2#3%
  {%
    \ekvd@assert@val{#1}%
      {%
        \expandafter\ekvd@add@aux\csname\ekv@name\ekvd@set{#1}\endcsname{{##1}}%
          {#1}{#2}{\ekvd@long\ekvdef}{#3}%
      }%
  }
\protected\long\def\ekvd@add@noval#1#2#3%
  {%
    \ekvd@assert@noval{#1}%
      {%
        \expandafter\ekvd@add@aux\csname\ekv@name\ekvd@set{#1}N\endcsname{}%
          {#1}{#2}\ekvdefNoVal{#3}%
      }%
  }
\protected\long\def\ekvd@add@aux#1#2%
  {%
    \ekvd@extract@prefixes#1%
    \expandafter\ekvd@add@aux@\expandafter{#1#2}%
  }
\protected\long\def\ekvd@add@aux@#1#2#3#4#5%
  {%
    #5%
    \ekvd@prot#4\ekvd@set{#2}{#1#3}%
  }
%    \end{macrocode}
% \end{macro}
%
% \begin{macro}[internal]
%   {
%     \ekvd@extract@prefixes,\ekvd@extract@prefixes@,
%     \ekvd@extract@prefixes@long,\ekvd@extract@prefixes@prot
%   }
% This macro checks which prefixes were used for the definition of a macro and
% sets |\ekvd@long| and |\ekvd@prot| accordingly.
%    \begin{macrocode}
\protected\def\ekvd@extract@prefixes#1%
  {%
    \expandafter\ekvd@extract@prefixes@\meaning#1\ekvd@stop
  }
%    \end{macrocode}
% In the following definition |#1| will get replaced by |macro:|, |#2| by
% |\long| and |#3| by |\protected| (in each, all tokens will have category
% other). This allows us to parse the |\meaning| of a macro for those strings.
%    \begin{macrocode}
\protected\def\ekvd@extract@prefixes@#1#2#3%
  {%
    \protected\def\ekvd@extract@prefixes@##1#1##2\ekvd@stop
      {%
        \ekvd@extract@prefixes@long
          ##1\ekvd@mark\@firstofone#2\ekvd@mark\@gobble\ekvd@stop
          {\let\ekvd@long\long}%
        \ekvd@extract@prefixes@prot
          ##1\ekvd@mark\@firstofone#3\ekvd@mark\@gobble\ekvd@stop
          {\let\ekvd@prot\protected}%
      }%
    \protected\def\ekvd@extract@prefixes@long##1#2##2\ekvd@mark##3##4\ekvd@stop
      {##3}%
    \protected\def\ekvd@extract@prefixes@prot##1#3##2\ekvd@mark##3##4\ekvd@stop
      {##3}%
  }
%    \end{macrocode}
% We use a temporary macro to expand the three arguments of
% |\ekvd@extract@prefixes@|, which will set up the real meaning of itself and
% the parsing for |\long| and |\protected|.
%    \begin{macrocode}
\begingroup
\edef\ekvd@tmp
  {%
    \endgroup
    \ekvd@extract@prefixes@
      {\detokenize{macro:}}%
      {\string\long}%
      {\string\protected}%
  }
\ekvd@tmp
%    \end{macrocode}
% \end{macro}
%
%
% \subsubsection{Tests}
%
% \begin{macro}[internal]{\ekvd@newlet,\ekvd@newreg}
% These macros test whether a control sequence is defined, if it isn't they
% define it, either via |\let| or via the correct \cs[no-index]{new\meta{reg}}.
%    \begin{macrocode}
\protected\def\ekvd@newlet#1#2%
  {%
    \ifdefined#1\ekv@fi@gobble\fi\@firstofone{\let#1#2}%
  }
\protected\def\ekvd@newreg#1#2%
  {%
    \ifdefined#1\ekv@fi@gobble\fi\@firstofone{\csname new#2\endcsname#1}%
  }
%    \end{macrocode}
% \end{macro}
%
% \begin{macro}[internal]
%   {\ekvd@assert@twoargs,\ekvd@ifnottwoargs,\ekvd@ifempty@gtwo}
% A test for exactly two tokens can be reduced for an empty-test after gobbling
% two tokens, in the case that there are fewer tokens than two in the argument,
% only macros will be gobbled that are needed for the true branch, which doesn't
% hurt, and if there are more this will not be empty.
%    \begin{macrocode}
\long\def\ekvd@assert@twoargs#1%
  {%
    \ekvd@ifnottwoargs{#1}{\ekvd@err@missing@definition}%
  }
\long\def\ekvd@ifnottwoargs#1%
  {%
    \ekvd@ifempty@gtwo#1\ekv@ifempty@B
      \ekv@ifempty@false\ekv@ifempty@A\ekv@ifempty@B\@firstoftwo
  }
\long\def\ekvd@ifempty@gtwo#1#2{\ekv@ifempty@\ekv@ifempty@A}
%    \end{macrocode}
% \end{macro}
%
% \begin{macro}[internal]
%   {
%     \ekvd@assert@val,\ekvd@assert@val@,\ekvd@assert@noval,\ekvd@assert@noval@,
%     \ekvd@extract@args,\ekvd@extracted@args,\ekvd@one@arg@string
%   }
% Assert that a given key is defined as a value taking key or a |NoVal| key with
% the correct argument structure, respectively.
%    \begin{macrocode}
\protected\def\ekvd@assert@val#1%
  {%
    \ekvifdefined\ekvd@set{#1}%
      {\expandafter\ekvd@assert@val@\csname\ekv@name\ekvd@set{#1}\endcsname}%
      {%
        \ekvifdefinedNoVal\ekvd@set{#1}%
          \ekvd@err@add@val@on@noval
          {\ekvd@err@undefined@key{#1}}%
        \@gobble
      }%
  }
\protected\def\ekvd@assert@val@#1%
  {%
    \expandafter\ekvd@extract@args\meaning#1\ekvd@stop
    \unless\ifx\ekvd@extracted@args\ekvd@one@arg@string
      \ekvd@err@unsupported@arg
    \fi
    \@firstofone
  }%
\protected\def\ekvd@assert@noval#1%
  {%
    \ekvifdefinedNoVal\ekvd@set{#1}%
      {\expandafter\ekvd@assert@noval@\csname\ekv@name\ekvd@set{#1}N\endcsname}%
      {%
        \ekvifdefined\ekvd@set{#1}%
          \ekvd@err@add@noval@on@val
          {\ekvd@err@undefined@key{#1}}%
        \@gobble
      }%
  }
\protected\def\ekvd@assert@noval@#1%
  {%
    \expandafter\ekvd@extract@args\meaning#1\ekvd@stop
    \unless\ifx\ekvd@extracted@args\ekv@empty
      \ekvd@err@unsupported@arg
    \fi
    \@firstofone
  }
\protected\def\ekvd@extract@args#1%
  {%
    \protected\def\ekvd@extract@args##1#1##2->##3\ekvd@stop
      {\def\ekvd@extracted@args{##2}}%
  }
\expandafter\ekvd@extract@args\expandafter{\detokenize{macro:}}
\edef\ekvd@one@arg@string{\string#1}
%    \end{macrocode}
% \end{macro}
%
% \begin{macro}[internal]{\ekvd@assert@arg,\ekvd@assert@arg@msg,\ekvd@ifnoarg}
% There is no need to actually define |\ekvd@ifnoarg| here, as it will be set by
% either |\ekvd@arg| or |\ekvd@noarg|.
%    \begin{macrocode}
\def\ekvd@assert@arg{\ekvd@ifnoarg\ekvd@err@missing@definition}
\long\def\ekvd@assert@arg@msg#1%
  {%
    \ekvd@ifnoarg{\ekvd@err@missing@definition@msg{#1}}%
  }
%    \end{macrocode}
% \end{macro}
%
% \begin{macro}[internal]{\ekvd@assert@filledarg,\ekvd@ifnoarg@or@empty}
%    \begin{macrocode}
\long\def\ekvd@assert@filledarg#1%
  {%
    \ekvd@ifnoarg@or@empty{#1}\ekvd@err@missing@definition
  }
\long\def\ekvd@ifnoarg@or@empty#1%
  {%
    \ekvd@ifnoarg
      \@firstoftwo
      {\ekv@ifempty{#1}}%
  }
%    \end{macrocode}
% \end{macro}
%
% \begin{macro}[internal]
%   {
%     \ekvd@assert@not@long,\ekvd@assert@not@protected,\ekvd@assert@not@also
%     \ekvd@assert@not@long@also,\ekvd@assert@not@protected@also,
%     \ekvd@assert@new,\ekvd@assert@not@new
%   }
% Some key-types don't want to be |also|, |\long| or |\protected|, so we provide
% macros to test this and throw an error, this could be silently ignored but now
% users will learn to not use unnecessary stuff which slows the compilation
% down.
%    \begin{macrocode}
\def\ekvd@assert@not@long{\ifx\ekvd@long\long\ekvd@err@no@prefix{long}\fi}
\def\ekvd@assert@not@protected
  {\ifx\ekvd@prot\protected\ekvd@err@no@prefix{protected}\fi}
\def\ekvd@assert@not@also{\ekvd@ifalso{\ekvd@err@no@prefix{also}}{}}
\def\ekvd@assert@not@long@also
  {\ifx\ekvd@long\long\ekvd@err@no@prefix@also{long}\fi}
\def\ekvd@assert@not@protected@also
  {\ifx\ekvd@prot\protected\ekvd@err@no@prefix@also{protected}\fi}
\def\ekvd@assert@new#1#2%
  {\csname ekvifdefined#1\endcsname\ekvd@set{#2}{\ekvd@err@not@new}}
\def\ekvd@assert@not@new
  {\ifx\ekvd@ifnew\ekvd@assert@new\ekvd@err@no@prefix{new}\fi}
\def\ekvd@assert@new@for@name#1%
  {%
    \ifx\ekvd@ifnew\ekvd@assert@new
      \ekv@fi@firstoftwo
    \fi
    \@secondoftwo
      {\ekv@ifdefined{#1}\ekvd@err@not@new}%
      \@firstofone
  }
%    \end{macrocode}
% \end{macro}
%
% \begin{macro}[internal]
%   {
%     \ekvd@if@not@already@choice, \ekvd@if@not@already@choice@a,
%     \ekvd@if@not@already@choice@b
%   }
% It is bad to use |also| on a key that already contains a |choice|, as both
% choices would share the same valid values and thus lead to each callback being
% used twice. The following is a rudimentary test against this.
%    \begin{macrocode}
\protected\def\ekvd@if@not@already@choice#1%
  {%
    \expandafter\ekvd@if@not@already@choice@a
      \csname\ekv@name\ekvd@set{#1}\endcsname
      {}\ekvd@h@choice\ekvd@stop
  }
\protected\def\ekvd@if@not@already@choice@a
  {%
    \expandafter\ekvd@if@not@already@choice@b
  }
\long\protected\def\ekvd@if@not@already@choice@b#1\ekvd@h@choice#2\ekvd@stop
  {%
    \ekv@ifempty{#2}\@firstofone\@gobble
  }
%    \end{macrocode}
% \end{macro}
%
% \begin{macro}[internal]{\ekvd@ifspace,\ekvd@ifspace@}
% Yet another test which can be reduced to an if-empty, this time by gobbling
% everything up to the first space.
%    \begin{macrocode}
\long\def\ekvd@ifspace#1%
  {%
    \ekvd@ifspace@#1 \ekv@ifempty@B
      \ekv@ifempty@false\ekv@ifempty@A\ekv@ifempty@B\@firstoftwo
  }
\long\def\ekvd@ifspace@#1 % keep this space
  {%
    \ekv@ifempty@\ekv@ifempty@A
  }
%    \end{macrocode}
% \end{macro}
%
% \subsubsection{Messages}
%
% Most messages of \expkvd\ are not expandable, since they only appear during
% key-definition, which is not expandable anyway.
%
% \begin{macro}[internal]
%   {
%     \ekvd@errm,\ekvd@err@missing@definition,\ekvd@err@missing@definition@msg,
%     \ekvd@err@missing@type,\ekvd@err@undefined@prefix,\ekvd@err@undefined@key,
%     \ekvd@err@no@prefix,\ekvd@err@no@prefix@msg,\ekvd@err@no@prefix@also,
%     \ekvd@err@add@val@on@noval,\ekvd@err@add@noval@on@val,
%     \ekvd@err@unsupported@arg,\ekvd@err@not@new
%   }
% The non-expandable error messages are boring, so here they are:
%    \begin{macrocode}
\protected\def\ekvd@errm#1{\errmessage{expkv-def Error: #1}}
\protected\def\ekvd@err@missing@definition
  {\ekvd@errm{Missing definition for key `\ekvd@cur'}}
\protected\def\ekvd@err@missing@definition@msg#1%
  {\ekvd@errm{Missing definition for key `\ekv@unexpanded{#1}'}}
\protected\def\ekvd@err@missing@type
  {\ekvd@errm{Missing type prefix for key `\ekvd@cur'}}
\protected\def\ekvd@err@undefined@prefix#1%
  {%
    \ekvd@errm
      {%
        Undefined prefix `\ekv@unexpanded{#1}' found while processing
        `\ekvd@cur'%
      }%
  }
\protected\def\ekvd@err@undefined@key#1%
  {%
    \ekvd@errm
      {Undefined key `\ekv@unexpanded{#1}' found while processing `\ekvd@cur'}%
  }
\protected\def\ekvd@err@undefined@noval#1%
  {%
    \ekvd@errm
      {%
        Undefined noval key `\unexpanded{#1}' found while processing
        `\ekvd@cur'%
      }%
  }
\protected\def\ekvd@err@no@prefix#1%
  {\ekvd@errm{prefix `#1' not accepted in `\ekvd@cur'}}
\protected\def\ekvd@err@no@prefix@msg#1#2%
  {\ekvd@errm{prefix `#2' not accepted in `\ekv@unexpanded{#1}'}}
\protected\def\ekvd@err@no@prefix@also#1%
  {\ekvd@errm{`\ekvd@cur' not allowed with a `#1' key}}
\protected\def\ekvd@err@add@val@on@noval
  {\ekvd@errm{`\ekvd@cur' not allowed with a NoVal key}}
\protected\def\ekvd@err@add@noval@on@val
  {\ekvd@errm{`\ekvd@cur' not allowed with a value taking key}}
\protected\def\ekvd@err@unsupported@arg\fi\@firstofone#1%
  {%
    \fi
    \ekvd@errm
      {%
        Existing key-macro has the unsupported argument string
        `\ekvd@extracted@args' for key `\ekvd@cur'%
      }%
  }
\protected\def\ekvd@err@not@new
  {\ekvd@errm{The key for `\ekvd@cur' is already defined}}
\protected\long\def\ekvd@err@misused@unknown
  {\ekvd@errm{Misuse of the unknown type found while processing `\ekvd@cur'}}
%    \end{macrocode}
% \end{macro}
%
% \begin{macro}[internal]
%   {
%     \ekvd@err@choice@invalid,\ekvd@err@choice@invalid@,\ekvd@choice@name,
%     \ekvd@unknown@choice@name
%   }
% |\ekvd@err@choice@invalid| will have to use this mechanism to throw its
% message. Also we have to retrieve the name parts of the choice in an easy way,
% so we use parentheses of catcode 8 here, which should suffice in most cases to
% allow for a correct separation.
%    \begin{macrocode}
\def\ekvd@err@choice@invalid#1%
  {%
    \ekvd@err@choice@invalid@#1%
  }
\begingroup
\catcode40=8
\catcode41=8
\@firstofone{\endgroup
\def\ekvd@choice@name#1#2#3%
  {%
    ekvd#1(#2)\detokenize{#3}%
  }
\def\ekvd@unknown@choice@name#1#2%
  {%
    ekvd:u:#1(#2)%
  }
\def\ekvd@err@choice@invalid@ ekvd#1(#2)\detokenize#3%
  {%
    \ekv@ifdefined{\ekvd@unknown@choice@name{#1}{#2}}%
      {\csname\ekvd@unknown@choice@name{#1}{#2}\endcsname{#3}}%
      {\ekvd@err{invalid choice `#3' for `#2' in set `#1'}}%
  }
}
%    \end{macrocode}
% \end{macro}
%
% \begin{macro}[internal]{\ekvd@err}
% The expandable error messages use |\ekvd@err|, which is just like |\ekv@err|
% from \expkv. It uses a runaway argument to start the error message.
%    \begin{macrocode}
\ekv@exparg{\long\def\ekvd@err#1}{\ekverr{expkv-def}{#1}}
%    \end{macrocode}
% \end{macro}
%
% Now everything that's left is to reset the category code of |@|.
%    \begin{macrocode}
\catcode`\@=\ekvd@tmp
%    \end{macrocode}
%
% \gobbledocstriptag
%</tex>
%^^A=<<
%
%^^A=<<
%
% \end{implementation}^^A=<<
%
% \clearpage
% \PrintIndex
%
\endinput
%
^^A vim: ft=tex fdm=marker fmr=>>=,=<<
}{}
\ProvidesFile{expkv-def.dtx}
  [\csname ekvdDate\endcsname\ a key-defining frontend for expkv]
\PassOptionsToPackage{full}{textcomp}
\documentclass{l3doc}
\RequirePackage[oldstylenums,nott]{kpfonts}
\input{glyphtounicode}
\pdfgentounicode=1
\RequirePackage{tcolorbox}
\newtcolorbox{exresult}[2][]
  {colback=expkvgrey!10!white,colframe=expkvgrey,fontupper=\small,width={#2},#1}
\newtcbox\exres[1][]
  {
     colback=expkvgrey!10!white
    ,colframe=expkvgrey
    ,size=small
    ,nobeforeafter
    ,tcbox raise base
    ,fontupper=\small
    ,#1
  }
\RequirePackage{listings}
\RequirePackage{booktabs}
\RequirePackage{array}
\RequirePackage{collcell}
\RequirePackage{siunitx}
\RequirePackage{xcolor}
\RequirePackage{caption}
\RequirePackage{microtype}
\RequirePackage{accsupp}
\lstdefinelanguage{expkvd}[primitive]{TeX}
  {
    ,moretexcs=[4]^^A e-TeX
      {
        expanded,
        numexpr,
        protected,
      }
    ,moretexcs=[5]^^A plain/LaTeX
      {
        approx,
        begin,
        empty,
        item,
        LaTeX,
        makeatletter,makeatother,
        newcommand,newdimen,
        RequirePackage,
        rule,
        TeX,
        textit,texttt,
        usepackage,
      }
    ,moretexcs=[6]^^A used packages
      {
        ^^A xfp
        fpeval,
        ^^A keyval
        setkeys,
        ^^A yax
        defactiveparameter,storevalue,setparameterlist
      }
    ,moretexcs=[2]^^A expkv-def macros
      {
        ekvdefinekeys
      }
    ,moretexcs=[3]^^A expkv-pkg macros
      {
        ^^A expkv
        ekvdef,ekvdefNoVal,^^A
        ekvlet,ekvletNoVal,ekvletkv,ekvletkvNoVal,^^A
        ekvdefunknown,ekvdefunknownNoVal,^^A
        ekvredirectunknown,ekvredirectunknownNoVal,^^A
        ekvset,ekvsetSneaked,^^A
        ekvsetdef,ekvsetSneakeddef,ekvsetdefSneaked,^^A
        ekvparse,^^A
        ekvVersion,ekvDate,^^A
        ekvifdefined,ekvifdefinedNoVal,ekvifdefinedset,^^A
        ekvbreak,ekvbreakPreSneak,ekvbreakPostSneak,^^A
        ekvsneak,ekvsneakPre,^^A
        ekvchangeset,^^A
        ekvoptarg,ekvoptargTF,^^A
        ekverr,
        ^^A expkv-cs
        ekvcSecondaryKeys,ekvcSplitAndForward,
      }
  }
\colorlet{codeparam}{cyan!65!black}
\def\mylstwd{.55em}
\lstset
  {
    ,language=expkvd
    ,flexiblecolumns=false
    ,basewidth=\mylstwd
    ,gobble=2
    ,basicstyle=\fontfamily{jkp}\itshape
    ,texcsstyle=*[1]{\bfseries\color{expkvgrey}} ^^A primitives
    ,texcsstyle=*[2]{\bfseries\color{expkvred}}  ^^A expkv
    ,texcsstyle=*[3]{\color{expkvred}}           ^^A expkv-pkg
    ,texcsstyle=*[4]{\bfseries\color{expkvgrey}} ^^A e-TeX
    ,texcsstyle=*[5]{\bfseries\color{expkvgrey}} ^^A plain/LaTeX macros
    ,texcsstyle=*[6]{}                           ^^A macros of other packages
    ,commentstyle=\color[gray]{0.4}
    ,literate=
              {\{} {{\CodeUpBf\{}}{1}
              {\}} {{\CodeUpBf\}}}{1}
              {$}  {{\CodeUpBf\$}}{1}
              {[}  {{\CodeUp[}}{1}
              {]}  {{\CodeUp]}}{1}
              {(}  {{\CodeUp(}}{1}
              {)}  {{\CodeUp)}}{1}
              {*}  {{$*$}}{1}
              {1}  {{\CodeUp{1}}}{1}
              {2}  {{\CodeUp{2}}}{1}
              {3}  {{\CodeUp{3}}}{1}
              {4}  {{\CodeUp{4}}}{1}
              {5}  {{\CodeUp{5}}}{1}
              {6}  {{\CodeUp{6}}}{1}
              {7}  {{\CodeUp{7}}}{1}
              {8}  {{\CodeUp{8}}}{1}
              {9}  {{\CodeUp{9}}}{1}
              {0}  {{\CodeUp{0}}}{1}
              {##} {{\CodeColored{codeparam}{1}{\#}}}{1}
              {##1}{{\CodeColored{codeparam}{2}{\#1}}}{2}
              {##2}{{\CodeColored{codeparam}{2}{\#2}}}{2}
              {##3}{{\CodeColored{codeparam}{2}{\#3}}}{2}
              {##4}{{\CodeColored{codeparam}{2}{\#4}}}{2}
              {##5}{{\CodeColored{codeparam}{2}{\#5}}}{2}
              {##6}{{\CodeColored{codeparam}{2}{\#6}}}{2}
              {##7}{{\CodeColored{codeparam}{2}{\#7}}}{2}
              {##8}{{\CodeColored{codeparam}{2}{\#8}}}{2}
              {##9}{{\CodeColored{codeparam}{2}{\#9}}}{2}
              {<key>}{{$\langle$}key{$\rangle$}}{5}
              {<set>}{{$\langle$}set{$\rangle$}}{5}
              {<set1>}{{$\langle$}set1{$\rangle$}}{6}
              {<set2>}{{$\langle$}set2{$\rangle$}}{6}
    ^^A,literate=*{<key>}{\key}{4}{<set>}{\set}{4}
  }
\newcommand*\CodeColored[3]{\textcolor{#1}{\makebox[\dimexpr\mylstwd*#2]{#3}}}
\newcommand*\CodeUpBf[1]{\makebox[\mylstwd]{\textup{\textbf{#1}}}}
\newcommand*\CodeUp[1]{\makebox[\mylstwd]{\textup{#1}}}
\RequirePackage{randtext}
\let\metaORIG\meta
\protected\def\meta #1{\texttt{\metaORIG{#1}}}
\renewcommand*\thefootnote{\fnsymbol{footnote}}
\definecolor{expkvred}{HTML}{9F393D}
\colorlet{expkvgrey}{black!75}
\makeatletter
\newcommand*\expkv
  {^^A
    \texorpdfstring
      {^^A
        \mbox
          {^^A
            \BeginAccSupp{ActualText=expkv}^^A
            \href{https://github.com/Skillmon/tex_expkv}
              {^^A
                \rmfamily
                \bfseries
                {\color{expkvgrey}e\kern-.05em x\kern-.05em}^^A
                \lower.493ex
                  \hbox{{\color{expkvgrey}P}\kern-.1em{\color{expkvred}k}}^^A
                \kern-.18em{\color{expkvred}v}^^A
              }^^A
            \EndAccSupp{}^^A
          }^^A
      }
      {expkv}^^A
  }
\newcommand*\expkvd
  {^^A
    \texorpdfstring
      {^^A
        \mbox
          {^^A
            \BeginAccSupp{ActualText=expkv-def}^^A
            \href{https://github.com/Skillmon/tex_expkv-def}
              {^^A
                \rmfamily
                \bfseries
                {\color{expkvgrey}e\kern-.05em x\kern-.05em}^^A
                \lower.493ex
                  \hbox{{\color{expkvgrey}P}\kern-.1em{\color{expkvred}k}}^^A
                \kern-.18em{\color{expkvred}v}^^A
                {\color{expkvgrey}^^A
                  \kern.05em\rule[-.1ex]{.08em}{1.2ex}\kern.05em\textsc{def}^^A
                }^^A
              }^^A
            \EndAccSupp{}^^A
          }^^A
      }
      {expkv-def}^^A
  }
\newcommand\kv{\meta{key}=\meta{value}}
\newcommand\key{\meta{key}}
\newcommand\val{\meta{value}}
\newcommand\set{\meta{set}}
\newcommand\enfprefix{\textcolor{black}}
\newcommand\allprefix{\textcolor{gray}}
\newcommand\notprefix{\textcolor{red!80!black}}
\newcommand\printprefix[2]
  {^^A
    \ifcase\numexpr#1\relax\or
      \enfprefix{#2}\or
      \allprefix{#2}\or
      \notprefix{#2}\fi
  }
\newcommand\prefixes[4]
  {^^A
    \hfill
    \printprefix{#1}{new}
    \printprefix{#2}{also}
    \printprefix{#3}{protected}
    \printprefix{#4}{long}^^A
  }
\hypersetup{linkcolor=red!80!black,urlcolor=purple!80!black}
\DoNotIndex{\def,\edef,\,,\=,\begingroup,\catcode,\chardef,\csname,\endcsname}
\DoNotIndex{\endgroup,\endinput,\errmessage,\expandafter,\input,\let,\long}
\DoNotIndex{\protected,\ProvidesFile,\ProvidesPackage,\relax,\space}
\DoNotIndex{\@,\unexpanded,\string,\expanded,\dimexpr,\global,\glueexpr,\hbox}
\DoNotIndex{\numexpr,\RequirePackage,\setbox,\the,\unless,\xdef,\@firstofone}
\DoNotIndex{\@firstoftwo,\@gobble,\@secondoftwo,\meaning,\detokenize}
\DoNotIndex{\romannumeral,\^}
\DoNotIndex{\ifcsname}
\DoNotIndex{\ifx}
\DoNotIndex{\ifdefined}
\DoNotIndex{\iffalse}
\DoNotIndex{\iftrue}
\DoNotIndex{\else}
\DoNotIndex{\fi}
\@gobble\fi ^^A ignoring \ifx, \ifcsname, etc., but only one \fi
\@gobble\fi ^^A ignoring \ifx, \ifcsname, etc., but only one \fi
\@gobble\fi ^^A ignoring \ifx, \ifcsname, etc., but only one \fi
\@gobble\fi ^^A ignoring \ifx, \ifcsname, etc., but only one \fi
\@ifdefinable\gobbledocstriptag{\def\gobbledocstriptag#1>{}}
\makeatother
\begin{document}
  \title
    {^^A
      \texorpdfstring
        {^^A
          \huge\expkvd
          \\[\medskipamount]
          \Large a key-defining frontend for \expkv
        }{expkv-def - a key-defining frontend for expkv}^^A
    }
  \date{\ekvdDate\space v\ekvdVersion}
  \author{Jonathan P. Spratte\thanks{\protect\randomize{jspratte@yahoo.de}}}
  \DocInput{expkv-def.dtx}
\end{document}
%</driver>^^A=<<
% \fi
%
% \maketitle
% \renewcommand*\thefootnote{\arabic{footnote}}
%
% \begin{abstract}
% \noindent\parfillskip=0pt
% \expkvd\ provides a small \kv\ interface to define keys for \expkv. Key-types
% are declared using prefixes, similar to static typed languages. The stylised
% name is \expkvd\ but the files use \file{expkv-def}, this is due to CTAN-rules
% which don't allow \string| in package names since that is the pipe symbol in
% *nix shells.
% \end{abstract}
%
% \tableofcontents
%
% \begin{documentation}^^A>>=
%
% \section{Documentation}
%
% Since the trend for the last couple of years goes to defining keys for a \kv\
% interface using a \kv\ interface, I thought that maybe providing such an
% interface for \expkv\ will make it more attractive for actual use, besides its
% unique selling points of being fully expandable, and fast and reliable. But at
% the same time I don't want to widen \expkv's initial scope. So here it is
% \expkvd, go define \kv\ interfaces with \kv\ interfaces.
%
% Unlike many of the other established \kv\ interfaces to define keys, \expkvd\
% works using prefixes instead of suffixes (\emph{e.g.}, |.tl_set:N| of
% \pkg{l3keys}) or directory like handlers (\emph{e.g.}, |/.store in| of
% \pkg{pgfkeys}). This was decided as a personal preference, more over in \TeX\
% parsing for the first space is way easier than parsing for the last one.
% \expkvd's prefixes are sorted into two categories: |p|-type, which are
% equivalent to \TeX's prefixes like |\long|, and |t|-type defining the type of
% the key. For a description of the available |p|-prefixes take a look at
% \autoref{sec:prefix:p}, the |t|-prefixes are described in
% \autoref{sec:prefix:t}.
%
% \expkvd\ is usable as generic code, as a \LaTeX\ package, and as a \ConTeXt\
% module. It'll automatically load \expkv\ in the same mode as well. To use it,
% just use one of
% \begin{lstlisting}
% \input expkv-def       % plainTeX
% \usepackage{expkv-def} % LaTeX
% \usemodule[expkv-def]  % ConTeXt
% \end{lstlisting}
%
% \subsection{Macros}
% Apart from version and date containers there is only a single user-facing
% macro, and that should be used to define keys.
%
% \begin{function}{\ekvdefinekeys}
%   \begin{syntax}
%     \cs{ekvdefinekeys}\marg{set}\{\kv, \ldots\}
%   \end{syntax}
%   In \set, define \key\ to have definition \val. The general syntax for \key\
%   should be
%   \begin{quote}\ttfamily\small
%     \meta{prefix} \meta{name}
%   \end{quote}
%   Where \meta{prefix} is a space separated list of optional |p|-type prefixes
%   followed by one |t|-type prefix. The syntax of \val\ is dependent on the
%   used |t|-prefix.
% \end{function}
%
% \begin{function}{\ekvdDate,\ekvdVersion}
%   These two macros store the version and date of the package.
% \end{function}
%
% \subsection{Prefixes}
% As already said there are |p|-prefixes and |t|-prefixes. Not every |p|-prefix
% is allowed for all |t|-prefixes.
%
% \subsubsection{\texttt{p}-Prefixes}\label{sec:prefix:p}
% The two |p|-type prefixes |long| and |protected| are pretty simple by nature,
% so their description is pretty simple. They affect the \key\ at use-time, so
% omitting |long| doesn't mean that a \meta{definition} can't contain a |\par|
% token, only that the \key\ will not accept a |\par| in \val.  On the other
% hand |new| and |also| might be simple on first sight as well, but their rules
% are a bit more complicated.
%
% \begin{function}{also}
%   The following key type will be \emph{added} to an existing \key's
%   definition. You can't add a type taking an argument at use time to an
%   existing key which doesn't take an argument and vice versa. Also you'll get
%   an error if you try to add an action which isn't allowed to be either |long|
%   or |protected| to a key which already is |long| or |protected| (the opposite
%   order would be suboptimal as well, but can't be really captured with the
%   current code).
%
%   A key already defined as |long| or |protected| will stay |long| or
%   |protected|, but you can as well add |long| or |protected| with the |also|
%   definition.
%
%   As a small example, suppose you want to create a boolean key, but
%   additionally to setting a boolean value you want to execute some more code
%   as well, you can use the following
% \begin{lstlisting}
% \ekvdefinekeys{also-example}
%   {
%      bool key      = \ifmybool
%     ,also code key = \domystuff{#1}
%   }
% \end{lstlisting}
%
%   If you use |also| on a |choice|, |bool|, |invbool|, or |boolpair| key it is
%   tried to determine if the key already is of one of those types. If this test
%   is true the declared choices will be added to the possible choices but the
%   key's definition will not be changed other than that. If that wouldn't have
%   been done, the callbacks of the different choices could get called multiple
%   times.
% \end{function}
%
% \begin{function}{protected,protect}
%   The following key will be defined |\protected|. Note that key-types which
%   can't be defined expandable will always use |\protected|.
% \end{function}
%
% \begin{function}{long}
%   The following key will be defined |\long|.
% \end{function}
%
% \begin{function}{new}
%   The following key must be new (so previously undefined). An error is thrown
%   if it is already defined and the new definition is ignored. |new| only
%   asserts that there are no conflicts between |NoVal| keys and other |NoVal|
%   keys or value taking keys and other value taking keys. For example you can
%   use the following without an error:
% \begin{lstlisting}
% \ekvdefinekeys{new-example}
%   {
%      code key       = \domystuffwitharg{#1}
%     ,new  noval key = \domystuffwithoutarg
%   }
% \end{lstlisting}
% \end{function}
%
% \subsubsection{\texttt{t}-Prefixes}\label{sec:prefix:t}
% Since the |p|-type prefixes apply to some of the |t|-prefixes automatically
% but sometimes one might be disallowed we need some way to highlight this
% behaviour. In the following an enforced prefix will be printed black
% (\texttt{\enfprefix{protected}}), allowed prefixes will be grey
% (\texttt{\allprefix{protected}}), and disallowed prefixes will be red
% (\texttt{\notprefix{protected}}). This will be put flush-right in the syntax
% showing line.
%
% \begin{function}{code,ecode}
%   \begin{syntax}
%     code \key\ = \marg{definition} \prefixes2222
%   \end{syntax}
%   Define \key\ to expand to \meta{definition}. The \key\ will require a \val\
%   for which you can use |#1| inside \meta{definition}. The |ecode| variant
%   will fully expand \meta{definition} inside an |\edef|.
% \end{function}
%
% \begin{function}{noval,enoval}
%   \begin{syntax}
%     noval \key\ = \marg{definition} \prefixes2223
%   \end{syntax}
%   The |noval| type defines \key\ to expand to \meta{definition}. The \key\
%   will not take a \val. |enoval| fully expands \meta{definition} inside an
%   |\edef|.
% \end{function}
%
% \begin{function}{default,qdefault,odefault,fdefault,edefault}
%   \begin{syntax}
%     default \key\ = \marg{definition} \prefixes3223
%   \end{syntax}
%   This serves to place a default \val\ for a \key\ that takes an argument, the
%   \key\ can be of any argument-grabbing kind, and when used without a \val\
%   it will be passed \meta{definition} instead. The |qdefault| variant will
%   expand the \key's code once, so will be slightly quicker, but not change if
%   you redefine \key. |odefault| is just another name for |qdefault|. The
%   |fdefault| version will expand the key code until a non-expandable token or
%   a space is found, a space would be gobbled.\footnotemark{} The |edefault| on
%   the other hand fully expands the \key-code with \meta{definition} as its
%   argument inside of an |\edef|.
% \end{function}
% \footnotetext
%   {^^A
%     For those familiar with \TeX-coding: This uses a
%     \cs[no-index]{romannumeral}-expansion.
%   }
%
% \begin{function}{initial,oinitial,finitial,einitial}
%   \begin{syntax}
%     initial \key\ = \{\val\} \prefixes3333
%     initial \key
%   \end{syntax}
%   With |initial| you can set an initial \val\ for an already defined argument
%   taking \key. It'll just call the key-macro of \key\ and pass it \val. The
%   |einitial| variant will expand \val\ using an |\edef| expansion prior to
%   passing it to the key-macro and the |oinitial| variant will expand the first
%   token in \val\ once. |finitial| will expand \val\ until a non-expandable
%   token or a space is found, a space would be
%   gobbled.\footnotemark
%
%   If you don't provide a value (and no equals sign) a |noval| \key\ of the
%   same name is called once (or, if you specified a |default| for a value
%   taking key that would be used).
% \end{function}
% \footnotetext{Again using \cs[no-index]{romannumeral}.}
%
% \begin{function}{bool,gbool,boolTF,gboolTF}
%   \begin{syntax}
%     bool \key\ = \meta{cs} \prefixes2223
%   \end{syntax}
%   The \meta{cs} should be a single control sequence, such as |\iffoo|.
%   This will define \key\ to be a boolean key, which only takes the values
%   |true| or |false| and will throw an error for other values. If the key is
%   used without a \val\ it'll have the same effect as if you use
%   \texttt{\key=true}. |bool| and |gbool| will behave like \TeX-ifs so either
%   be |\iftrue| or |\iffalse|. The |boolTF| and |gboolTF| variants will both
%   take two arguments and if true the first will be used else the second, so
%   they are always either |\@firstoftwo| or |\@secondoftwo|. The variants with
%   a leading |g| will set the control sequence globally, the others locally. If
%   \meta{cs} is not yet defined it'll be initialised as the |false| version.
%   Note that the initialisation is \emph{not} done with |\newif|, so you will
%   not be able to do |\footrue| outside of the \kv\ interface, but you could
%   use |\newif| yourself. Even if the \key\ will not be |\protected| the
%   commands which execute the |true| or |false| choice will be, so the usage
%   should be safe in an expansion context (\emph{e.g.}, you can use
%   \texttt{edefault \key\ = false} without an issue to change the default
%   behaviour to execute the |false| choice). Internally a |bool| \key\ is the
%   same as a |choice| key which is set up to handle |true| and |false| as
%   choices.
% \end{function}
%
% \begin{function}{invbool,ginvbool,invboolTF,ginvboolTF}
%   \begin{syntax}
%     bool \key\ = \meta{cs} \prefixes2223
%   \end{syntax}
%   These are inverse boolean keys, they behave like |bool| and friends but set
%   the opposite meaning to the macro \meta{cs} in each case. So if
%   \texttt{key=true} is used |invbool| will set \meta{cs} to |\iffalse| and
%   vice versa.
% \end{function}
%
% \begin{function}{boolpair,gboolpair,boolpairTF,gboolpairTF}
%   \begin{syntax}
%     boolpair \key\ = \meta{cs_1}\meta{cs_2} \prefixes2223
%   \end{syntax}
%   The |boolpair| key type behaves like both |bool| and |invbool|, the
%   \meta{cs_1} will be set to the meaning according to the rules of |bool|, and
%   \meta{cs_2} will be set to the opposite.
% \end{function}
%
% \begin{function}{store,estore,gstore,xstore}
%   \begin{syntax}
%     store \key\ = \meta{cs} \prefixes2212
%   \end{syntax}
%   The \meta{cs} should be a single control sequence, such as |\foo|.
%   This will define \key\ to store \val\ inside of the control sequence. If
%   \meta{cs} isn't yet defined it will be initialised as empty. The variants
%   behave similarly to their |\def|, |\edef|, |\gdef|, and |\xdef|
%   counterparts, but |store| and |gstore| will allow you to store macro
%   parameters inside of them by using |\unexpanded|.
% \end{function}
%
% \begin{function}{data,edata,gdata,xdata}
%   \begin{syntax}
%     data \key\ = \meta{cs} \prefixes2212
%   \end{syntax}
%   The \meta{cs} should be a single control sequence, such as |\foo|. This will
%   define \key\ to store \val\ inside of the control sequence. But unlike the
%   |store| type, the macro \meta{cs} will be a switch at the same time, it'll
%   take two arguments and if \key\ was used expands to the first argument
%   followed by \val\ in braces, if \key\ was not used \meta{cs} will expand to
%   the second argument (so behave like |\@secondoftwo|). The idea is that with
%   this type you can define a key which should be typeset formatted. The
%   |edata| and |xdata| variants will fully expand \val, the |gdata| and |xdata|
%   variants will store \val\ inside \meta{cs} globally. The |p|-prefixes will
%   only affect the key-macro, \meta{cs} will always be expandable and |\long|.
% \end{function}
%
% \begin{function}{dataT,edataT,gdataT,xdataT}
%   \begin{syntax}
%     dataT \key\ = \meta{cs} \prefixes2212
%   \end{syntax}
%   Just like |data|, but instead of \meta{cs} grabbing two arguments it'll only
%   grab one, so by default it'll behave like |\@gobble|, and if a \val\ was
%   given to \key\ the \meta{cs} will behave like |\@firstofone| appended by
%   \marg{value}.
% \end{function}
%
% \begin{function}{int,eint,gint,xint}
%   \begin{syntax}
%     int \key\ = \meta{cs} \prefixes2212
%   \end{syntax}
%   The \meta{cs} should be a single control sequence, such as |\foo|.
%   An |int| key will be a \TeX-count register. If \meta{cs} isn't defined yet,
%   |\newcount| will be used to initialise it. The |eint| and |xint| versions
%   will use |\numexpr| to allow basic computations in their \val. The |gint|
%   and |xint| variants set the register globally.
% \end{function}
%
% \begin{function}{dimen,edimen,gdimen,xdimen}
%   \begin{syntax}
%     dimen \key\ = \meta{cs} \prefixes2212
%   \end{syntax}
%   The \meta{cs} should be a single control sequence, such as |\foo|.
%   This is just like |int| but uses a dimen register, |\newdimen| and
%   |\dimexpr| instead.
% \end{function}
%
% \begin{function}{skip,eskip,gskip,xskip}
%   \begin{syntax}
%     skip \key\ = \meta{cs} \prefixes2212
%   \end{syntax}
%   The \meta{cs} should be a single control sequence, such as |\foo|.
%   This is just like |int| but uses a skip register, |\newskip| and |\glueexpr|
%   instead.
% \end{function}
%
% \begin{function}{toks,gtoks,apptoks,gapptoks,pretoks,gpretoks}
%   \begin{syntax}
%     toks \key\ = \meta{cs} \prefixes2212
%   \end{syntax}
%   The \meta{cs} should be a single control sequence, such as |\foo|.
%   Store \val\ inside of a toks-register. The |g| variants use |\global|, the
%   |app| variants append \val\ to the contents of that register, the |pre|
%   variants will prepend \val. If \meta{cs} is not yet defined it will be
%   initialised with |\newtoks|.
% \end{function}
%
% \begin{function}{box,gbox}
%   \begin{syntax}
%     box \key\ = \meta{cs} \prefixes2212
%   \end{syntax}
%   The \meta{cs} should be a single control sequence, such as |\foo|.
%   Typesets \val\ into a |\hbox| and stores the result in a box register. The
%   boxes are colour safe. \expkvd\ doesn't provide a |vbox| type.
% \end{function}
%
% \begin{function}{meta}
%   \begin{syntax}
%     meta \key\ = \{\kv, \ldots\} \prefixes2222
%   \end{syntax}
%   This key type can set other keys, you can access the \val\ which was passed
%   to \key\ inside the \kv\ list with |#1|. It works by calling a sub-|\ekvset|
%   on the \kv\ list, so a |set| key will only affect that \kv\ list and not the
%   current |\ekvset|. Since it runs in a separate |\ekvset| you can't use
%   |\ekvsneak| using keys or similar macros in the way you normally could.
% \end{function}
%
% \begin{function}{nmeta}
%   \begin{syntax}
%     nmeta \key\ = \{\kv, \ldots\} \prefixes2223
%   \end{syntax}
%   This key type can set other keys, the difference to |meta| is, that this key
%   doesn't take a value, so the \kv\ list is static.
% \end{function}
%
% \begin{function}{smeta}
%   \begin{syntax}
%     smeta \key\ = \marg{set}\{\kv, \ldots\} \prefixes2222
%   \end{syntax}
%   Yet another |meta| variant. An |smeta| key will take a \val\ which you can
%   access using |#1|, but it sets the \kv\ list inside of \set, so is equal to
%   \texttt{\cs[no-index]{ekvset}\marg{set}\{\kv, \ldots\}}.
% \end{function}
%
% \begin{function}{snmeta}
%   \begin{syntax}
%     snmeta \key\ = \marg{set}\{\kv, \ldots\} \prefixes2223
%   \end{syntax}
%   And the last |meta| variant. |snmeta| is a combination of |smeta| and
%   |nmeta|. It doesn't take an argument and sets the \kv\ list inside of \set.
% \end{function}
%
% \begin{function}{set}
%   \begin{syntax}
%     set \key\ = \marg{set} \prefixes2233
%   \end{syntax}
%   This will define \key\ to change the set of the current |\ekvset|
%   invocation to \set. You can omit \set\ (including the equals sign),
%   which is the same as using \texttt{set \key\ = \{\key\}}. The created |set|
%   key will not take a \val. Note that just like in \expkv\ it'll not be
%   checked whether \set\ is defined and you'll get a low-level \TeX\ error if
%   you use an undefined \set.
% \end{function}
%
% \begin{function}{choice}
%   \begin{syntax}
%     choice \key\ = \{\val=\meta{definition}, \ldots\} \prefixes2223
%   \end{syntax}
%   Defines \key\ to be a |choice| key, meaning it will only accept a limited
%   set of values. You should define each possible \val\ inside of the
%   \val=\meta{definition} list. If a defined \val\ is passed to \key\ the
%   \meta{definition} will be left in the input stream. You can make
%   individual values |protected| inside the \val=\meta{definition} list. By
%   default a |choice| key is expandable, an undefined \val\ will throw an error
%   in an expandable way (but see the |unknown-choice| prefix). You can add
%   additional choices after the \key\ was created by using |choice| again for
%   the same \key, redefining choices is possible the same way, but there is no
%   interface to remove certain choices.
% \end{function}
%
% \begin{function}{choice-store}
%   \begin{syntax}
%     choice-store \key\ = \meta{cs}\{\val, \ldots\} \prefixes2223
%   \end{syntax}
%   This is defines a special type of |choice| key that'll store the given
%   choice inside the macro \meta{cs} (so \meta{cs} should be a single control
%   sequence name such as |\foo|). Since storing inside a macro can't be done
%   expandably every choice-code is |\protected|, you might define the
%   |choice-store| key itself as |protected| as well if you want. Since the
%   definition of each choice is predefined with this key type the choice list
%   should just be a comma separated list of valid choices.
% \end{function}
% This means that the following |choice| and |choice-store| keys are equivalent
% at use time:
% \begin{lstlisting}
% \newcommand*\mya{}
% \ekvdefinekeys{example}
%   {
%      choice key1 = {a=\def\mya{a}, b=\def\mya{b}, c=\def\mya{c}}
%     ,choice-store key2 = \mya{a,b,c}
%   }
% \end{lstlisting}
%
% \begin{function}{choice-enum}
%   \begin{syntax}
%     choice-enum \key\ = \meta{cs}\{\val, \ldots\} \prefixes2223
%   \end{syntax}
%   This is similar to |choice-store|, the differences are: \meta{cs} should be
%   a count-register or is initialised as such if the \meta{cs} is undefined
%   (via |\newcount|); instead of the value the position of the value in the
%   given list is stored in this register (zero-based).
% \end{function}
% This means that the following |choice| and |choice-enum| keys are equivalent
% at use time:
% \begin{lstlisting}
% \newcount\myb
% \ekvdefinekeys{example}
%   {
%      choice key1 = {a={\myb=0 }, b={\myb=1 }, c={\myb=2 }}
%     ,choice-enum key2 = \myb{a,b,c}
%   }
% \end{lstlisting}
%
% \begin{function}{unknown-choice}
%   \begin{syntax}
%     unknown-choice \key\ = \marg{definition} \prefixes2323
%   \end{syntax}
%   By default an unknown \val\ passed to a |choice| or |bool| key will throw an
%   error.  However, with this prefix you can define an alternative action which
%   should be executed if \key\ received an unknown choice. In \meta{definition}
%   you can refer to the choice which was passed in with |#1|.
% \end{function}
%
% \begin{function}{unknown code}
%   \begin{syntax}
%     unknown code = \marg{definition} \prefixes2322
%   \end{syntax}
%   By default \expkv\ throws errors when it encounters unknown keys in a set.
%   With the |unknown| prefix you can define handlers that deal with undefined
%   keys, instead of a \key\ name you have to specify a subtype for this prefix,
%   here the subtype is |code|.\par
%   With |unknown code| the \meta{definition} is used for unknown keys which
%   were provided a value (so corresponds to |\ekvdefunknown|), you can access
%   the key name with |#1| and the value with |#2|.\footnotemark
% \end{function}
% \footnotetext
%   {^^A
%     There is some trickery involved to get this more intuitive argument order
%     without any performance hit if you compare this to
%     \cs[no-index]{ekvdefunknown} directly.^^A
%   }
%
% \begin{function}{unknown noval}
%   \begin{syntax}
%     unknown noval = \marg{definition} \prefixes2323
%   \end{syntax}
%   This is like |unknown code| but uses \meta{definition} for unknown keys to
%   which no value was passed (so corresponds to |\ekvdefunknownNoVal|). You can
%   access the key name with |#1|.
% \end{function}
%
% \begin{function}{unknown redirect-code}
%   \begin{syntax}
%     unknown redirect-code = \marg{set-list} \prefixes2331
%   \end{syntax}
%   This uses a predefined action for |unknown code|. Instead of throwing an
%   error, it is tried to find the \key\ in each \set\ in the comma separated
%   \meta{set-list}. The first found match will be used and the remaining
%   options from the list discarded. If the \key\ isn't found in any \set\ an
%   expandable error will be thrown eventually. Internally \expkv's
%   |\ekvredirectunknown| will be used.
% \end{function}
%
% \begin{function}{unknown redirect-noval}
%   \begin{syntax}
%     unknown redirect-noval = \marg{set-list} \prefixes2333
%   \end{syntax}
%   This behaves just like |unknown redirect-code| but will set up means to
%   forward keys for |unknown noval|. Internally \expkv's
%   |\ekvredirectunknownNoVal| will be used.
% \end{function}
%
% \begin{function}{unknown redirect}
%   \begin{syntax}
%     unknown redirect = \marg{set-list} \prefixes2333
%   \end{syntax}
%   This is a short cut to apply both, |unknown redirect-code| and
%   |unknown redirect-noval|, as a result you might get doubled error messages,
%   one from each.
% \end{function}
%
% \subsection{Bugs}
% I don't think there are any (but every developer says that), if you find some
% please let me know, either via the email address on the first page or on
% GitHub:
% \url{https://github.com/Skillmon/tex_expkv-def}
%
% \subsection{Example}
% The following is an example code defining each base key-type once. Please
% admire the very creative key-name examples.
%
% \begin{lstlisting}
% \ekvdefinekeys{example}
%   {
%      long code keyA = #1
%     ,noval     keyA = NoVal given
%     ,bool      keyB = \keyB
%     ,boolTF    keyC = \keyC
%     ,store     keyD = \keyD
%     ,data      keyE = \keyE
%     ,dataT     keyF = \keyF
%     ,int       keyG = \keyG
%     ,dimen     keyH = \keyH
%     ,skip      keyI = \keyI
%     ,toks      keyJ = \keyJ
%     ,default   keyJ = \empty test
%     ,new box   keyK = \keyK
%     ,qdefault  keyK = K
%     ,choice    keyL =
%       {
%          protected 1 = \texttt{a}
%         ,2 = b
%         ,3 = c
%         ,4 = d
%         ,5 = e
%       }
%     ,edefault  keyL = 2
%     ,meta      keyM = {keyA={#1},keyB=false}
%     ,invbool   keyN = \keyN
%     ,boolpair  keyO = \keyOa\keyOb
%   }
% \end{lstlisting}
%
% \bigskip
% Since the |data| type might be a bit strange, here is another usage example
% for it.
% \begin{lstlisting}
% \ekvdefinekeys{ex}
%   {
%      data  name  = \Pname
%     ,data  age   = \Page
%     ,dataT hobby = \Phobby
%   }
% \newcommand\Person[1]
%   {%
%     \begingroup
%     \ekvset{ex}{#1}%
%     \begin{description}
%       \item[\Pname{}{\errmessage{A person requires a name}}]
%       \item[Age] \Page{\textit}{\errmessage{A person requires an age}}
%       \Phobby{\item[Hobbies]}
%     \end{description}
%     \endgroup
%   }
% \Person{name=Jonathan P. Spratte, age=young, hobby=\TeX\ coding}
% \Person{name=Some User, age=unknown, hobby=Reading Documentation}
% \Person{name=Anybody, age=any}
% \end{lstlisting}
% In this example a person should have a name and an age, but doesn't have to
% have hobbies. The name will be displayed as the description item and the age
% in Italics. If a person has no hobbies the description item will be silently
% left out.  The result of the above code looks like this:
%
% \begin{exresult}{\linewidth}
% \ekvdefinekeys{ex}
%   {
%      data  name  = \Pname
%     ,data  age   = \Page
%     ,dataT hobby = \Phobby
%   }
% \newcommand\Person[1]
%   {%
%     \begingroup
%     \ekvset{ex}{#1}%
%     \begin{description}
%       \item[\Pname{}{\errmessage{A person requires a name}}]
%       \item[Age] \Page{\textit}{\errmessage{A person requires an age}}
%       \Phobby{\item[Hobbies]}
%     \end{description}
%     \endgroup
%   }
% \Person{name=Jonathan P. Spratte, age=young, hobby=\TeX\ coding}
% \Person{name=Some User, age=unknown, hobby=Reading Documentation}
% \Person{name=Anybody, age=any}
% \end{exresult}
%
% \subsection{License}
%
% Copyright \textcopyright\ 2020\unless\ifnum\year=2020--\the\year\fi\
% Jonathan P. Spratte
%
% \medskip\noindent
% This work may be distributed and/or modified under the conditions of the
% \LaTeX\ Project Public License (LPPL), either version 1.3c of this license or
% (at your option) any later version. The latest version of this license is in
% the file:
%
% \url{http://www.latex-project.org/lppl.txt}
%
% \noindent
% This work is ``maintained'' (as per LPPL maintenance status) by
%
% \mbox{Jonathan P. Spratte}.
%
% \end{documentation}^^A=<<
%
% \begin{implementation}^^A>>=
%
% \clearpage
%
% \section{Implementation}^^A>>=
%^^A the LaTeX package >>=
% \subsection{The \LaTeX\ Package}
% Just like for \expkv\ we provide a small \LaTeX\ package that sets up things
% such that we behave nicely on \LaTeX\ packages and files system. It'll
% |\input| the generic code which implements the functionality.
% \gobbledocstriptag
%<*pkg>
%    \begin{macrocode}
\RequirePackage{expkv}
\def\ekvd@tmp
  {%
    \ProvidesFile{expkv-def.tex}%
      [\ekvdDate\space v\ekvdVersion\space a key-defining frontend for expkv]%
  }
% \iffalse meta-comment
%
% File: expkv-def.dtx Copyright (C) 2020-2021 Jonathan P. Spratte
%
% This work  may be  distributed and/or  modified under  the conditions  of the
% LaTeX Project Public License (LPPL),  either version 1.3c  of this license or
% (at your option) any later version.  The latest version of this license is in
% the file:
%
%   http://www.latex-project.org/lppl.txt
%
% ------------------------------------------------------------------------------
%
%<*driver>^^A>>=
\def\nameofplainTeX{plain}
\ifx\fmtname\nameofplainTeX\else
  \expandafter\begingroup
\fi
\input l3docstrip.tex
\askforoverwritefalse
\preamble

--------------------------------------------------------------
expkv-def -- a key-defining frontend for expkv
E-mail: jspratte@yahoo.de
Released under the LaTeX Project Public License v1.3c or later
See http://www.latex-project.org/lppl.txt
--------------------------------------------------------------

Copyright (C) 2020-2021 Jonathan P. Spratte

This  work may be  distributed and/or  modified under  the conditions  of the
LaTeX Project Public License (LPPL),  either version 1.3c  of this license or
(at your option) any later version.  The latest version of this license is in
the file:

  http://www.latex-project.org/lppl.txt

This work is "maintained" (as per LPPL maintenance status) by
  Jonathan P. Spratte.

This work consists of the file  expkv-def.dtx
and the derived files           expkv-def.pdf
                                expkv-def.sty
                                expkv-def.tex
                                t-expkv-def.tex

\endpreamble
% stop docstrip adding \endinput
\postamble
\endpostamble
\generate{\file{expkv-def.sty}{\from{expkv-def.dtx}{pkg}}}
\generate{\file{expkv-def.tex}{\from{expkv-def.dtx}{tex}}}
\generate{\file{t-expkv-def.tex}{\from{expkv-def.dtx}{ctx}}}
\ifx\fmtname\nameofplainTeX
  \expandafter\endbatchfile
\else
  \expandafter\endgroup
\fi
%
\IfFileExists{expkv-def.tex}{\input{expkv-def.tex}}{}
\ProvidesFile{expkv-def.dtx}
  [\csname ekvdDate\endcsname\ a key-defining frontend for expkv]
\PassOptionsToPackage{full}{textcomp}
\documentclass{l3doc}
\RequirePackage[oldstylenums,nott]{kpfonts}
\input{glyphtounicode}
\pdfgentounicode=1
\RequirePackage{tcolorbox}
\newtcolorbox{exresult}[2][]
  {colback=expkvgrey!10!white,colframe=expkvgrey,fontupper=\small,width={#2},#1}
\newtcbox\exres[1][]
  {
     colback=expkvgrey!10!white
    ,colframe=expkvgrey
    ,size=small
    ,nobeforeafter
    ,tcbox raise base
    ,fontupper=\small
    ,#1
  }
\RequirePackage{listings}
\RequirePackage{booktabs}
\RequirePackage{array}
\RequirePackage{collcell}
\RequirePackage{siunitx}
\RequirePackage{xcolor}
\RequirePackage{caption}
\RequirePackage{microtype}
\RequirePackage{accsupp}
\lstdefinelanguage{expkvd}[primitive]{TeX}
  {
    ,moretexcs=[4]^^A e-TeX
      {
        expanded,
        numexpr,
        protected,
      }
    ,moretexcs=[5]^^A plain/LaTeX
      {
        approx,
        begin,
        empty,
        item,
        LaTeX,
        makeatletter,makeatother,
        newcommand,newdimen,
        RequirePackage,
        rule,
        TeX,
        textit,texttt,
        usepackage,
      }
    ,moretexcs=[6]^^A used packages
      {
        ^^A xfp
        fpeval,
        ^^A keyval
        setkeys,
        ^^A yax
        defactiveparameter,storevalue,setparameterlist
      }
    ,moretexcs=[2]^^A expkv-def macros
      {
        ekvdefinekeys
      }
    ,moretexcs=[3]^^A expkv-pkg macros
      {
        ^^A expkv
        ekvdef,ekvdefNoVal,^^A
        ekvlet,ekvletNoVal,ekvletkv,ekvletkvNoVal,^^A
        ekvdefunknown,ekvdefunknownNoVal,^^A
        ekvredirectunknown,ekvredirectunknownNoVal,^^A
        ekvset,ekvsetSneaked,^^A
        ekvsetdef,ekvsetSneakeddef,ekvsetdefSneaked,^^A
        ekvparse,^^A
        ekvVersion,ekvDate,^^A
        ekvifdefined,ekvifdefinedNoVal,ekvifdefinedset,^^A
        ekvbreak,ekvbreakPreSneak,ekvbreakPostSneak,^^A
        ekvsneak,ekvsneakPre,^^A
        ekvchangeset,^^A
        ekvoptarg,ekvoptargTF,^^A
        ekverr,
        ^^A expkv-cs
        ekvcSecondaryKeys,ekvcSplitAndForward,
      }
  }
\colorlet{codeparam}{cyan!65!black}
\def\mylstwd{.55em}
\lstset
  {
    ,language=expkvd
    ,flexiblecolumns=false
    ,basewidth=\mylstwd
    ,gobble=2
    ,basicstyle=\fontfamily{jkp}\itshape
    ,texcsstyle=*[1]{\bfseries\color{expkvgrey}} ^^A primitives
    ,texcsstyle=*[2]{\bfseries\color{expkvred}}  ^^A expkv
    ,texcsstyle=*[3]{\color{expkvred}}           ^^A expkv-pkg
    ,texcsstyle=*[4]{\bfseries\color{expkvgrey}} ^^A e-TeX
    ,texcsstyle=*[5]{\bfseries\color{expkvgrey}} ^^A plain/LaTeX macros
    ,texcsstyle=*[6]{}                           ^^A macros of other packages
    ,commentstyle=\color[gray]{0.4}
    ,literate=
              {\{} {{\CodeUpBf\{}}{1}
              {\}} {{\CodeUpBf\}}}{1}
              {$}  {{\CodeUpBf\$}}{1}
              {[}  {{\CodeUp[}}{1}
              {]}  {{\CodeUp]}}{1}
              {(}  {{\CodeUp(}}{1}
              {)}  {{\CodeUp)}}{1}
              {*}  {{$*$}}{1}
              {1}  {{\CodeUp{1}}}{1}
              {2}  {{\CodeUp{2}}}{1}
              {3}  {{\CodeUp{3}}}{1}
              {4}  {{\CodeUp{4}}}{1}
              {5}  {{\CodeUp{5}}}{1}
              {6}  {{\CodeUp{6}}}{1}
              {7}  {{\CodeUp{7}}}{1}
              {8}  {{\CodeUp{8}}}{1}
              {9}  {{\CodeUp{9}}}{1}
              {0}  {{\CodeUp{0}}}{1}
              {##} {{\CodeColored{codeparam}{1}{\#}}}{1}
              {##1}{{\CodeColored{codeparam}{2}{\#1}}}{2}
              {##2}{{\CodeColored{codeparam}{2}{\#2}}}{2}
              {##3}{{\CodeColored{codeparam}{2}{\#3}}}{2}
              {##4}{{\CodeColored{codeparam}{2}{\#4}}}{2}
              {##5}{{\CodeColored{codeparam}{2}{\#5}}}{2}
              {##6}{{\CodeColored{codeparam}{2}{\#6}}}{2}
              {##7}{{\CodeColored{codeparam}{2}{\#7}}}{2}
              {##8}{{\CodeColored{codeparam}{2}{\#8}}}{2}
              {##9}{{\CodeColored{codeparam}{2}{\#9}}}{2}
              {<key>}{{$\langle$}key{$\rangle$}}{5}
              {<set>}{{$\langle$}set{$\rangle$}}{5}
              {<set1>}{{$\langle$}set1{$\rangle$}}{6}
              {<set2>}{{$\langle$}set2{$\rangle$}}{6}
    ^^A,literate=*{<key>}{\key}{4}{<set>}{\set}{4}
  }
\newcommand*\CodeColored[3]{\textcolor{#1}{\makebox[\dimexpr\mylstwd*#2]{#3}}}
\newcommand*\CodeUpBf[1]{\makebox[\mylstwd]{\textup{\textbf{#1}}}}
\newcommand*\CodeUp[1]{\makebox[\mylstwd]{\textup{#1}}}
\RequirePackage{randtext}
\let\metaORIG\meta
\protected\def\meta #1{\texttt{\metaORIG{#1}}}
\renewcommand*\thefootnote{\fnsymbol{footnote}}
\definecolor{expkvred}{HTML}{9F393D}
\colorlet{expkvgrey}{black!75}
\makeatletter
\newcommand*\expkv
  {^^A
    \texorpdfstring
      {^^A
        \mbox
          {^^A
            \BeginAccSupp{ActualText=expkv}^^A
            \href{https://github.com/Skillmon/tex_expkv}
              {^^A
                \rmfamily
                \bfseries
                {\color{expkvgrey}e\kern-.05em x\kern-.05em}^^A
                \lower.493ex
                  \hbox{{\color{expkvgrey}P}\kern-.1em{\color{expkvred}k}}^^A
                \kern-.18em{\color{expkvred}v}^^A
              }^^A
            \EndAccSupp{}^^A
          }^^A
      }
      {expkv}^^A
  }
\newcommand*\expkvd
  {^^A
    \texorpdfstring
      {^^A
        \mbox
          {^^A
            \BeginAccSupp{ActualText=expkv-def}^^A
            \href{https://github.com/Skillmon/tex_expkv-def}
              {^^A
                \rmfamily
                \bfseries
                {\color{expkvgrey}e\kern-.05em x\kern-.05em}^^A
                \lower.493ex
                  \hbox{{\color{expkvgrey}P}\kern-.1em{\color{expkvred}k}}^^A
                \kern-.18em{\color{expkvred}v}^^A
                {\color{expkvgrey}^^A
                  \kern.05em\rule[-.1ex]{.08em}{1.2ex}\kern.05em\textsc{def}^^A
                }^^A
              }^^A
            \EndAccSupp{}^^A
          }^^A
      }
      {expkv-def}^^A
  }
\newcommand\kv{\meta{key}=\meta{value}}
\newcommand\key{\meta{key}}
\newcommand\val{\meta{value}}
\newcommand\set{\meta{set}}
\newcommand\enfprefix{\textcolor{black}}
\newcommand\allprefix{\textcolor{gray}}
\newcommand\notprefix{\textcolor{red!80!black}}
\newcommand\printprefix[2]
  {^^A
    \ifcase\numexpr#1\relax\or
      \enfprefix{#2}\or
      \allprefix{#2}\or
      \notprefix{#2}\fi
  }
\newcommand\prefixes[4]
  {^^A
    \hfill
    \printprefix{#1}{new}
    \printprefix{#2}{also}
    \printprefix{#3}{protected}
    \printprefix{#4}{long}^^A
  }
\hypersetup{linkcolor=red!80!black,urlcolor=purple!80!black}
\DoNotIndex{\def,\edef,\,,\=,\begingroup,\catcode,\chardef,\csname,\endcsname}
\DoNotIndex{\endgroup,\endinput,\errmessage,\expandafter,\input,\let,\long}
\DoNotIndex{\protected,\ProvidesFile,\ProvidesPackage,\relax,\space}
\DoNotIndex{\@,\unexpanded,\string,\expanded,\dimexpr,\global,\glueexpr,\hbox}
\DoNotIndex{\numexpr,\RequirePackage,\setbox,\the,\unless,\xdef,\@firstofone}
\DoNotIndex{\@firstoftwo,\@gobble,\@secondoftwo,\meaning,\detokenize}
\DoNotIndex{\romannumeral,\^}
\DoNotIndex{\ifcsname}
\DoNotIndex{\ifx}
\DoNotIndex{\ifdefined}
\DoNotIndex{\iffalse}
\DoNotIndex{\iftrue}
\DoNotIndex{\else}
\DoNotIndex{\fi}
\@gobble\fi ^^A ignoring \ifx, \ifcsname, etc., but only one \fi
\@gobble\fi ^^A ignoring \ifx, \ifcsname, etc., but only one \fi
\@gobble\fi ^^A ignoring \ifx, \ifcsname, etc., but only one \fi
\@gobble\fi ^^A ignoring \ifx, \ifcsname, etc., but only one \fi
\@ifdefinable\gobbledocstriptag{\def\gobbledocstriptag#1>{}}
\makeatother
\begin{document}
  \title
    {^^A
      \texorpdfstring
        {^^A
          \huge\expkvd
          \\[\medskipamount]
          \Large a key-defining frontend for \expkv
        }{expkv-def - a key-defining frontend for expkv}^^A
    }
  \date{\ekvdDate\space v\ekvdVersion}
  \author{Jonathan P. Spratte\thanks{\protect\randomize{jspratte@yahoo.de}}}
  \DocInput{expkv-def.dtx}
\end{document}
%</driver>^^A=<<
% \fi
%
% \maketitle
% \renewcommand*\thefootnote{\arabic{footnote}}
%
% \begin{abstract}
% \noindent\parfillskip=0pt
% \expkvd\ provides a small \kv\ interface to define keys for \expkv. Key-types
% are declared using prefixes, similar to static typed languages. The stylised
% name is \expkvd\ but the files use \file{expkv-def}, this is due to CTAN-rules
% which don't allow \string| in package names since that is the pipe symbol in
% *nix shells.
% \end{abstract}
%
% \tableofcontents
%
% \begin{documentation}^^A>>=
%
% \section{Documentation}
%
% Since the trend for the last couple of years goes to defining keys for a \kv\
% interface using a \kv\ interface, I thought that maybe providing such an
% interface for \expkv\ will make it more attractive for actual use, besides its
% unique selling points of being fully expandable, and fast and reliable. But at
% the same time I don't want to widen \expkv's initial scope. So here it is
% \expkvd, go define \kv\ interfaces with \kv\ interfaces.
%
% Unlike many of the other established \kv\ interfaces to define keys, \expkvd\
% works using prefixes instead of suffixes (\emph{e.g.}, |.tl_set:N| of
% \pkg{l3keys}) or directory like handlers (\emph{e.g.}, |/.store in| of
% \pkg{pgfkeys}). This was decided as a personal preference, more over in \TeX\
% parsing for the first space is way easier than parsing for the last one.
% \expkvd's prefixes are sorted into two categories: |p|-type, which are
% equivalent to \TeX's prefixes like |\long|, and |t|-type defining the type of
% the key. For a description of the available |p|-prefixes take a look at
% \autoref{sec:prefix:p}, the |t|-prefixes are described in
% \autoref{sec:prefix:t}.
%
% \expkvd\ is usable as generic code, as a \LaTeX\ package, and as a \ConTeXt\
% module. It'll automatically load \expkv\ in the same mode as well. To use it,
% just use one of
% \begin{lstlisting}
% \input expkv-def       % plainTeX
% \usepackage{expkv-def} % LaTeX
% \usemodule[expkv-def]  % ConTeXt
% \end{lstlisting}
%
% \subsection{Macros}
% Apart from version and date containers there is only a single user-facing
% macro, and that should be used to define keys.
%
% \begin{function}{\ekvdefinekeys}
%   \begin{syntax}
%     \cs{ekvdefinekeys}\marg{set}\{\kv, \ldots\}
%   \end{syntax}
%   In \set, define \key\ to have definition \val. The general syntax for \key\
%   should be
%   \begin{quote}\ttfamily\small
%     \meta{prefix} \meta{name}
%   \end{quote}
%   Where \meta{prefix} is a space separated list of optional |p|-type prefixes
%   followed by one |t|-type prefix. The syntax of \val\ is dependent on the
%   used |t|-prefix.
% \end{function}
%
% \begin{function}{\ekvdDate,\ekvdVersion}
%   These two macros store the version and date of the package.
% \end{function}
%
% \subsection{Prefixes}
% As already said there are |p|-prefixes and |t|-prefixes. Not every |p|-prefix
% is allowed for all |t|-prefixes.
%
% \subsubsection{\texttt{p}-Prefixes}\label{sec:prefix:p}
% The two |p|-type prefixes |long| and |protected| are pretty simple by nature,
% so their description is pretty simple. They affect the \key\ at use-time, so
% omitting |long| doesn't mean that a \meta{definition} can't contain a |\par|
% token, only that the \key\ will not accept a |\par| in \val.  On the other
% hand |new| and |also| might be simple on first sight as well, but their rules
% are a bit more complicated.
%
% \begin{function}{also}
%   The following key type will be \emph{added} to an existing \key's
%   definition. You can't add a type taking an argument at use time to an
%   existing key which doesn't take an argument and vice versa. Also you'll get
%   an error if you try to add an action which isn't allowed to be either |long|
%   or |protected| to a key which already is |long| or |protected| (the opposite
%   order would be suboptimal as well, but can't be really captured with the
%   current code).
%
%   A key already defined as |long| or |protected| will stay |long| or
%   |protected|, but you can as well add |long| or |protected| with the |also|
%   definition.
%
%   As a small example, suppose you want to create a boolean key, but
%   additionally to setting a boolean value you want to execute some more code
%   as well, you can use the following
% \begin{lstlisting}
% \ekvdefinekeys{also-example}
%   {
%      bool key      = \ifmybool
%     ,also code key = \domystuff{#1}
%   }
% \end{lstlisting}
%
%   If you use |also| on a |choice|, |bool|, |invbool|, or |boolpair| key it is
%   tried to determine if the key already is of one of those types. If this test
%   is true the declared choices will be added to the possible choices but the
%   key's definition will not be changed other than that. If that wouldn't have
%   been done, the callbacks of the different choices could get called multiple
%   times.
% \end{function}
%
% \begin{function}{protected,protect}
%   The following key will be defined |\protected|. Note that key-types which
%   can't be defined expandable will always use |\protected|.
% \end{function}
%
% \begin{function}{long}
%   The following key will be defined |\long|.
% \end{function}
%
% \begin{function}{new}
%   The following key must be new (so previously undefined). An error is thrown
%   if it is already defined and the new definition is ignored. |new| only
%   asserts that there are no conflicts between |NoVal| keys and other |NoVal|
%   keys or value taking keys and other value taking keys. For example you can
%   use the following without an error:
% \begin{lstlisting}
% \ekvdefinekeys{new-example}
%   {
%      code key       = \domystuffwitharg{#1}
%     ,new  noval key = \domystuffwithoutarg
%   }
% \end{lstlisting}
% \end{function}
%
% \subsubsection{\texttt{t}-Prefixes}\label{sec:prefix:t}
% Since the |p|-type prefixes apply to some of the |t|-prefixes automatically
% but sometimes one might be disallowed we need some way to highlight this
% behaviour. In the following an enforced prefix will be printed black
% (\texttt{\enfprefix{protected}}), allowed prefixes will be grey
% (\texttt{\allprefix{protected}}), and disallowed prefixes will be red
% (\texttt{\notprefix{protected}}). This will be put flush-right in the syntax
% showing line.
%
% \begin{function}{code,ecode}
%   \begin{syntax}
%     code \key\ = \marg{definition} \prefixes2222
%   \end{syntax}
%   Define \key\ to expand to \meta{definition}. The \key\ will require a \val\
%   for which you can use |#1| inside \meta{definition}. The |ecode| variant
%   will fully expand \meta{definition} inside an |\edef|.
% \end{function}
%
% \begin{function}{noval,enoval}
%   \begin{syntax}
%     noval \key\ = \marg{definition} \prefixes2223
%   \end{syntax}
%   The |noval| type defines \key\ to expand to \meta{definition}. The \key\
%   will not take a \val. |enoval| fully expands \meta{definition} inside an
%   |\edef|.
% \end{function}
%
% \begin{function}{default,qdefault,odefault,fdefault,edefault}
%   \begin{syntax}
%     default \key\ = \marg{definition} \prefixes3223
%   \end{syntax}
%   This serves to place a default \val\ for a \key\ that takes an argument, the
%   \key\ can be of any argument-grabbing kind, and when used without a \val\
%   it will be passed \meta{definition} instead. The |qdefault| variant will
%   expand the \key's code once, so will be slightly quicker, but not change if
%   you redefine \key. |odefault| is just another name for |qdefault|. The
%   |fdefault| version will expand the key code until a non-expandable token or
%   a space is found, a space would be gobbled.\footnotemark{} The |edefault| on
%   the other hand fully expands the \key-code with \meta{definition} as its
%   argument inside of an |\edef|.
% \end{function}
% \footnotetext
%   {^^A
%     For those familiar with \TeX-coding: This uses a
%     \cs[no-index]{romannumeral}-expansion.
%   }
%
% \begin{function}{initial,oinitial,finitial,einitial}
%   \begin{syntax}
%     initial \key\ = \{\val\} \prefixes3333
%     initial \key
%   \end{syntax}
%   With |initial| you can set an initial \val\ for an already defined argument
%   taking \key. It'll just call the key-macro of \key\ and pass it \val. The
%   |einitial| variant will expand \val\ using an |\edef| expansion prior to
%   passing it to the key-macro and the |oinitial| variant will expand the first
%   token in \val\ once. |finitial| will expand \val\ until a non-expandable
%   token or a space is found, a space would be
%   gobbled.\footnotemark
%
%   If you don't provide a value (and no equals sign) a |noval| \key\ of the
%   same name is called once (or, if you specified a |default| for a value
%   taking key that would be used).
% \end{function}
% \footnotetext{Again using \cs[no-index]{romannumeral}.}
%
% \begin{function}{bool,gbool,boolTF,gboolTF}
%   \begin{syntax}
%     bool \key\ = \meta{cs} \prefixes2223
%   \end{syntax}
%   The \meta{cs} should be a single control sequence, such as |\iffoo|.
%   This will define \key\ to be a boolean key, which only takes the values
%   |true| or |false| and will throw an error for other values. If the key is
%   used without a \val\ it'll have the same effect as if you use
%   \texttt{\key=true}. |bool| and |gbool| will behave like \TeX-ifs so either
%   be |\iftrue| or |\iffalse|. The |boolTF| and |gboolTF| variants will both
%   take two arguments and if true the first will be used else the second, so
%   they are always either |\@firstoftwo| or |\@secondoftwo|. The variants with
%   a leading |g| will set the control sequence globally, the others locally. If
%   \meta{cs} is not yet defined it'll be initialised as the |false| version.
%   Note that the initialisation is \emph{not} done with |\newif|, so you will
%   not be able to do |\footrue| outside of the \kv\ interface, but you could
%   use |\newif| yourself. Even if the \key\ will not be |\protected| the
%   commands which execute the |true| or |false| choice will be, so the usage
%   should be safe in an expansion context (\emph{e.g.}, you can use
%   \texttt{edefault \key\ = false} without an issue to change the default
%   behaviour to execute the |false| choice). Internally a |bool| \key\ is the
%   same as a |choice| key which is set up to handle |true| and |false| as
%   choices.
% \end{function}
%
% \begin{function}{invbool,ginvbool,invboolTF,ginvboolTF}
%   \begin{syntax}
%     bool \key\ = \meta{cs} \prefixes2223
%   \end{syntax}
%   These are inverse boolean keys, they behave like |bool| and friends but set
%   the opposite meaning to the macro \meta{cs} in each case. So if
%   \texttt{key=true} is used |invbool| will set \meta{cs} to |\iffalse| and
%   vice versa.
% \end{function}
%
% \begin{function}{boolpair,gboolpair,boolpairTF,gboolpairTF}
%   \begin{syntax}
%     boolpair \key\ = \meta{cs_1}\meta{cs_2} \prefixes2223
%   \end{syntax}
%   The |boolpair| key type behaves like both |bool| and |invbool|, the
%   \meta{cs_1} will be set to the meaning according to the rules of |bool|, and
%   \meta{cs_2} will be set to the opposite.
% \end{function}
%
% \begin{function}{store,estore,gstore,xstore}
%   \begin{syntax}
%     store \key\ = \meta{cs} \prefixes2212
%   \end{syntax}
%   The \meta{cs} should be a single control sequence, such as |\foo|.
%   This will define \key\ to store \val\ inside of the control sequence. If
%   \meta{cs} isn't yet defined it will be initialised as empty. The variants
%   behave similarly to their |\def|, |\edef|, |\gdef|, and |\xdef|
%   counterparts, but |store| and |gstore| will allow you to store macro
%   parameters inside of them by using |\unexpanded|.
% \end{function}
%
% \begin{function}{data,edata,gdata,xdata}
%   \begin{syntax}
%     data \key\ = \meta{cs} \prefixes2212
%   \end{syntax}
%   The \meta{cs} should be a single control sequence, such as |\foo|. This will
%   define \key\ to store \val\ inside of the control sequence. But unlike the
%   |store| type, the macro \meta{cs} will be a switch at the same time, it'll
%   take two arguments and if \key\ was used expands to the first argument
%   followed by \val\ in braces, if \key\ was not used \meta{cs} will expand to
%   the second argument (so behave like |\@secondoftwo|). The idea is that with
%   this type you can define a key which should be typeset formatted. The
%   |edata| and |xdata| variants will fully expand \val, the |gdata| and |xdata|
%   variants will store \val\ inside \meta{cs} globally. The |p|-prefixes will
%   only affect the key-macro, \meta{cs} will always be expandable and |\long|.
% \end{function}
%
% \begin{function}{dataT,edataT,gdataT,xdataT}
%   \begin{syntax}
%     dataT \key\ = \meta{cs} \prefixes2212
%   \end{syntax}
%   Just like |data|, but instead of \meta{cs} grabbing two arguments it'll only
%   grab one, so by default it'll behave like |\@gobble|, and if a \val\ was
%   given to \key\ the \meta{cs} will behave like |\@firstofone| appended by
%   \marg{value}.
% \end{function}
%
% \begin{function}{int,eint,gint,xint}
%   \begin{syntax}
%     int \key\ = \meta{cs} \prefixes2212
%   \end{syntax}
%   The \meta{cs} should be a single control sequence, such as |\foo|.
%   An |int| key will be a \TeX-count register. If \meta{cs} isn't defined yet,
%   |\newcount| will be used to initialise it. The |eint| and |xint| versions
%   will use |\numexpr| to allow basic computations in their \val. The |gint|
%   and |xint| variants set the register globally.
% \end{function}
%
% \begin{function}{dimen,edimen,gdimen,xdimen}
%   \begin{syntax}
%     dimen \key\ = \meta{cs} \prefixes2212
%   \end{syntax}
%   The \meta{cs} should be a single control sequence, such as |\foo|.
%   This is just like |int| but uses a dimen register, |\newdimen| and
%   |\dimexpr| instead.
% \end{function}
%
% \begin{function}{skip,eskip,gskip,xskip}
%   \begin{syntax}
%     skip \key\ = \meta{cs} \prefixes2212
%   \end{syntax}
%   The \meta{cs} should be a single control sequence, such as |\foo|.
%   This is just like |int| but uses a skip register, |\newskip| and |\glueexpr|
%   instead.
% \end{function}
%
% \begin{function}{toks,gtoks,apptoks,gapptoks,pretoks,gpretoks}
%   \begin{syntax}
%     toks \key\ = \meta{cs} \prefixes2212
%   \end{syntax}
%   The \meta{cs} should be a single control sequence, such as |\foo|.
%   Store \val\ inside of a toks-register. The |g| variants use |\global|, the
%   |app| variants append \val\ to the contents of that register, the |pre|
%   variants will prepend \val. If \meta{cs} is not yet defined it will be
%   initialised with |\newtoks|.
% \end{function}
%
% \begin{function}{box,gbox}
%   \begin{syntax}
%     box \key\ = \meta{cs} \prefixes2212
%   \end{syntax}
%   The \meta{cs} should be a single control sequence, such as |\foo|.
%   Typesets \val\ into a |\hbox| and stores the result in a box register. The
%   boxes are colour safe. \expkvd\ doesn't provide a |vbox| type.
% \end{function}
%
% \begin{function}{meta}
%   \begin{syntax}
%     meta \key\ = \{\kv, \ldots\} \prefixes2222
%   \end{syntax}
%   This key type can set other keys, you can access the \val\ which was passed
%   to \key\ inside the \kv\ list with |#1|. It works by calling a sub-|\ekvset|
%   on the \kv\ list, so a |set| key will only affect that \kv\ list and not the
%   current |\ekvset|. Since it runs in a separate |\ekvset| you can't use
%   |\ekvsneak| using keys or similar macros in the way you normally could.
% \end{function}
%
% \begin{function}{nmeta}
%   \begin{syntax}
%     nmeta \key\ = \{\kv, \ldots\} \prefixes2223
%   \end{syntax}
%   This key type can set other keys, the difference to |meta| is, that this key
%   doesn't take a value, so the \kv\ list is static.
% \end{function}
%
% \begin{function}{smeta}
%   \begin{syntax}
%     smeta \key\ = \marg{set}\{\kv, \ldots\} \prefixes2222
%   \end{syntax}
%   Yet another |meta| variant. An |smeta| key will take a \val\ which you can
%   access using |#1|, but it sets the \kv\ list inside of \set, so is equal to
%   \texttt{\cs[no-index]{ekvset}\marg{set}\{\kv, \ldots\}}.
% \end{function}
%
% \begin{function}{snmeta}
%   \begin{syntax}
%     snmeta \key\ = \marg{set}\{\kv, \ldots\} \prefixes2223
%   \end{syntax}
%   And the last |meta| variant. |snmeta| is a combination of |smeta| and
%   |nmeta|. It doesn't take an argument and sets the \kv\ list inside of \set.
% \end{function}
%
% \begin{function}{set}
%   \begin{syntax}
%     set \key\ = \marg{set} \prefixes2233
%   \end{syntax}
%   This will define \key\ to change the set of the current |\ekvset|
%   invocation to \set. You can omit \set\ (including the equals sign),
%   which is the same as using \texttt{set \key\ = \{\key\}}. The created |set|
%   key will not take a \val. Note that just like in \expkv\ it'll not be
%   checked whether \set\ is defined and you'll get a low-level \TeX\ error if
%   you use an undefined \set.
% \end{function}
%
% \begin{function}{choice}
%   \begin{syntax}
%     choice \key\ = \{\val=\meta{definition}, \ldots\} \prefixes2223
%   \end{syntax}
%   Defines \key\ to be a |choice| key, meaning it will only accept a limited
%   set of values. You should define each possible \val\ inside of the
%   \val=\meta{definition} list. If a defined \val\ is passed to \key\ the
%   \meta{definition} will be left in the input stream. You can make
%   individual values |protected| inside the \val=\meta{definition} list. By
%   default a |choice| key is expandable, an undefined \val\ will throw an error
%   in an expandable way (but see the |unknown-choice| prefix). You can add
%   additional choices after the \key\ was created by using |choice| again for
%   the same \key, redefining choices is possible the same way, but there is no
%   interface to remove certain choices.
% \end{function}
%
% \begin{function}{choice-store}
%   \begin{syntax}
%     choice-store \key\ = \meta{cs}\{\val, \ldots\} \prefixes2223
%   \end{syntax}
%   This is defines a special type of |choice| key that'll store the given
%   choice inside the macro \meta{cs} (so \meta{cs} should be a single control
%   sequence name such as |\foo|). Since storing inside a macro can't be done
%   expandably every choice-code is |\protected|, you might define the
%   |choice-store| key itself as |protected| as well if you want. Since the
%   definition of each choice is predefined with this key type the choice list
%   should just be a comma separated list of valid choices.
% \end{function}
% This means that the following |choice| and |choice-store| keys are equivalent
% at use time:
% \begin{lstlisting}
% \newcommand*\mya{}
% \ekvdefinekeys{example}
%   {
%      choice key1 = {a=\def\mya{a}, b=\def\mya{b}, c=\def\mya{c}}
%     ,choice-store key2 = \mya{a,b,c}
%   }
% \end{lstlisting}
%
% \begin{function}{choice-enum}
%   \begin{syntax}
%     choice-enum \key\ = \meta{cs}\{\val, \ldots\} \prefixes2223
%   \end{syntax}
%   This is similar to |choice-store|, the differences are: \meta{cs} should be
%   a count-register or is initialised as such if the \meta{cs} is undefined
%   (via |\newcount|); instead of the value the position of the value in the
%   given list is stored in this register (zero-based).
% \end{function}
% This means that the following |choice| and |choice-enum| keys are equivalent
% at use time:
% \begin{lstlisting}
% \newcount\myb
% \ekvdefinekeys{example}
%   {
%      choice key1 = {a={\myb=0 }, b={\myb=1 }, c={\myb=2 }}
%     ,choice-enum key2 = \myb{a,b,c}
%   }
% \end{lstlisting}
%
% \begin{function}{unknown-choice}
%   \begin{syntax}
%     unknown-choice \key\ = \marg{definition} \prefixes2323
%   \end{syntax}
%   By default an unknown \val\ passed to a |choice| or |bool| key will throw an
%   error.  However, with this prefix you can define an alternative action which
%   should be executed if \key\ received an unknown choice. In \meta{definition}
%   you can refer to the choice which was passed in with |#1|.
% \end{function}
%
% \begin{function}{unknown code}
%   \begin{syntax}
%     unknown code = \marg{definition} \prefixes2322
%   \end{syntax}
%   By default \expkv\ throws errors when it encounters unknown keys in a set.
%   With the |unknown| prefix you can define handlers that deal with undefined
%   keys, instead of a \key\ name you have to specify a subtype for this prefix,
%   here the subtype is |code|.\par
%   With |unknown code| the \meta{definition} is used for unknown keys which
%   were provided a value (so corresponds to |\ekvdefunknown|), you can access
%   the key name with |#1| and the value with |#2|.\footnotemark
% \end{function}
% \footnotetext
%   {^^A
%     There is some trickery involved to get this more intuitive argument order
%     without any performance hit if you compare this to
%     \cs[no-index]{ekvdefunknown} directly.^^A
%   }
%
% \begin{function}{unknown noval}
%   \begin{syntax}
%     unknown noval = \marg{definition} \prefixes2323
%   \end{syntax}
%   This is like |unknown code| but uses \meta{definition} for unknown keys to
%   which no value was passed (so corresponds to |\ekvdefunknownNoVal|). You can
%   access the key name with |#1|.
% \end{function}
%
% \begin{function}{unknown redirect-code}
%   \begin{syntax}
%     unknown redirect-code = \marg{set-list} \prefixes2331
%   \end{syntax}
%   This uses a predefined action for |unknown code|. Instead of throwing an
%   error, it is tried to find the \key\ in each \set\ in the comma separated
%   \meta{set-list}. The first found match will be used and the remaining
%   options from the list discarded. If the \key\ isn't found in any \set\ an
%   expandable error will be thrown eventually. Internally \expkv's
%   |\ekvredirectunknown| will be used.
% \end{function}
%
% \begin{function}{unknown redirect-noval}
%   \begin{syntax}
%     unknown redirect-noval = \marg{set-list} \prefixes2333
%   \end{syntax}
%   This behaves just like |unknown redirect-code| but will set up means to
%   forward keys for |unknown noval|. Internally \expkv's
%   |\ekvredirectunknownNoVal| will be used.
% \end{function}
%
% \begin{function}{unknown redirect}
%   \begin{syntax}
%     unknown redirect = \marg{set-list} \prefixes2333
%   \end{syntax}
%   This is a short cut to apply both, |unknown redirect-code| and
%   |unknown redirect-noval|, as a result you might get doubled error messages,
%   one from each.
% \end{function}
%
% \subsection{Bugs}
% I don't think there are any (but every developer says that), if you find some
% please let me know, either via the email address on the first page or on
% GitHub:
% \url{https://github.com/Skillmon/tex_expkv-def}
%
% \subsection{Example}
% The following is an example code defining each base key-type once. Please
% admire the very creative key-name examples.
%
% \begin{lstlisting}
% \ekvdefinekeys{example}
%   {
%      long code keyA = #1
%     ,noval     keyA = NoVal given
%     ,bool      keyB = \keyB
%     ,boolTF    keyC = \keyC
%     ,store     keyD = \keyD
%     ,data      keyE = \keyE
%     ,dataT     keyF = \keyF
%     ,int       keyG = \keyG
%     ,dimen     keyH = \keyH
%     ,skip      keyI = \keyI
%     ,toks      keyJ = \keyJ
%     ,default   keyJ = \empty test
%     ,new box   keyK = \keyK
%     ,qdefault  keyK = K
%     ,choice    keyL =
%       {
%          protected 1 = \texttt{a}
%         ,2 = b
%         ,3 = c
%         ,4 = d
%         ,5 = e
%       }
%     ,edefault  keyL = 2
%     ,meta      keyM = {keyA={#1},keyB=false}
%     ,invbool   keyN = \keyN
%     ,boolpair  keyO = \keyOa\keyOb
%   }
% \end{lstlisting}
%
% \bigskip
% Since the |data| type might be a bit strange, here is another usage example
% for it.
% \begin{lstlisting}
% \ekvdefinekeys{ex}
%   {
%      data  name  = \Pname
%     ,data  age   = \Page
%     ,dataT hobby = \Phobby
%   }
% \newcommand\Person[1]
%   {%
%     \begingroup
%     \ekvset{ex}{#1}%
%     \begin{description}
%       \item[\Pname{}{\errmessage{A person requires a name}}]
%       \item[Age] \Page{\textit}{\errmessage{A person requires an age}}
%       \Phobby{\item[Hobbies]}
%     \end{description}
%     \endgroup
%   }
% \Person{name=Jonathan P. Spratte, age=young, hobby=\TeX\ coding}
% \Person{name=Some User, age=unknown, hobby=Reading Documentation}
% \Person{name=Anybody, age=any}
% \end{lstlisting}
% In this example a person should have a name and an age, but doesn't have to
% have hobbies. The name will be displayed as the description item and the age
% in Italics. If a person has no hobbies the description item will be silently
% left out.  The result of the above code looks like this:
%
% \begin{exresult}{\linewidth}
% \ekvdefinekeys{ex}
%   {
%      data  name  = \Pname
%     ,data  age   = \Page
%     ,dataT hobby = \Phobby
%   }
% \newcommand\Person[1]
%   {%
%     \begingroup
%     \ekvset{ex}{#1}%
%     \begin{description}
%       \item[\Pname{}{\errmessage{A person requires a name}}]
%       \item[Age] \Page{\textit}{\errmessage{A person requires an age}}
%       \Phobby{\item[Hobbies]}
%     \end{description}
%     \endgroup
%   }
% \Person{name=Jonathan P. Spratte, age=young, hobby=\TeX\ coding}
% \Person{name=Some User, age=unknown, hobby=Reading Documentation}
% \Person{name=Anybody, age=any}
% \end{exresult}
%
% \subsection{License}
%
% Copyright \textcopyright\ 2020\unless\ifnum\year=2020--\the\year\fi\
% Jonathan P. Spratte
%
% \medskip\noindent
% This work may be distributed and/or modified under the conditions of the
% \LaTeX\ Project Public License (LPPL), either version 1.3c of this license or
% (at your option) any later version. The latest version of this license is in
% the file:
%
% \url{http://www.latex-project.org/lppl.txt}
%
% \noindent
% This work is ``maintained'' (as per LPPL maintenance status) by
%
% \mbox{Jonathan P. Spratte}.
%
% \end{documentation}^^A=<<
%
% \begin{implementation}^^A>>=
%
% \clearpage
%
% \section{Implementation}^^A>>=
%^^A the LaTeX package >>=
% \subsection{The \LaTeX\ Package}
% Just like for \expkv\ we provide a small \LaTeX\ package that sets up things
% such that we behave nicely on \LaTeX\ packages and files system. It'll
% |\input| the generic code which implements the functionality.
% \gobbledocstriptag
%<*pkg>
%    \begin{macrocode}
\RequirePackage{expkv}
\def\ekvd@tmp
  {%
    \ProvidesFile{expkv-def.tex}%
      [\ekvdDate\space v\ekvdVersion\space a key-defining frontend for expkv]%
  }
\input{expkv-def.tex}
\ProvidesPackage{expkv-def}%
  [\ekvdDate\space v\ekvdVersion\space a key-defining frontend for expkv]
%    \end{macrocode}
% \gobbledocstriptag
%</pkg>
%^^A=<<
%^^A the ConTeXt module >>=
% \subsection{The \ConTeXt\ module}
% \gobbledocstriptag
%<*ctx>
%    \begin{macrocode}
\writestatus{loading}{ConTeXt User Module / expkv-def}
\usemodule[expkv]
\unprotect
\input expkv-def.tex
\writestatus{loading}
  {ConTeXt User Module / expkv-def / Version \ekvdVersion\space loaded}
\protect\endinput
%    \end{macrocode}
% \gobbledocstriptag
%</ctx>
%^^A=<<
%^^A main file >>=
% \subsection{The Generic Code}
% The rest of this implementation will be the generic code.
% \gobbledocstriptag
%<*tex>
%
% Load \expkv\ if the package didn't already do so -- since \expkv\ has
% safeguards against being loaded twice this does no harm and the overhead
% isn't that big. Also we reuse some of the internals of \expkv\ to save us from
% retyping them.
%    \begin{macrocode}
\input expkv
%    \end{macrocode}
%
% We make sure that \file{expkv-def.tex} is only input once:
%    \begin{macrocode}
\expandafter\ifx\csname ekvdVersion\endcsname\relax
\else
  \expandafter\endinput
\fi
%    \end{macrocode}
%
% \begin{macro}{\ekvdVersion,\ekvdDate}
% We're on our first input, so lets store the version and date in a macro.
%    \begin{macrocode}
\def\ekvdVersion{0.8c}
\def\ekvdDate{2021-09-20}
%    \end{macrocode}
% \end{macro}
%
% If the \LaTeX\ format is loaded we want to be a good file and report back who
% we are, for this the package will have defined |\ekvd@tmp| to use
% |\ProvidesFile|, else this will expand to a |\relax| and do no harm.
%    \begin{macrocode}
\csname ekvd@tmp\endcsname
%    \end{macrocode}
%
% Store the category code of |@| to later be able to reset it and change it to
% 11 for now.
%    \begin{macrocode}
\expandafter\chardef\csname ekvd@tmp\endcsname=\catcode`\@
\catcode`\@=11
%    \end{macrocode}
% |\ekvd@tmp| will be reused later to handle expansion during the key
% defining. But we don't need it to ever store information long-term after
% \expkvd\ was initialized.
%
% \begin{macro}[internal]{\ekvd@ifprimitive}
%    \begin{macrocode}
\protected\long\def\ekvd@ifprimitive#1%
  {%
    \begingroup
      \edef\ekvd@tmpa{\string #1}%
      \edef\ekvd@tmpb{\meaning#1}%
      \expandafter
    \endgroup
    \ifx\ekvd@tmpa\ekvd@tmpb
      \ekv@fi@firstoftwo
    \fi
    \@secondoftwo
  }
%    \end{macrocode}
% \end{macro}
%
% \begin{macro}[internal]
%   {\ekvd@long,\ekvd@prot,\ekvd@clear@prefixes,\ekvd@ifalso}
% \expkvd\ will use |\ekvd@long|, |\ekvd@prot|, and |\ekvd@ifalso| to store
% whether a key should be defined as |\long| or |\protected| or adds an action
% to an existing key, and we have to clear them for every new key. By default
% |long| and |protected| will just be empty, |ifalso| will be
% |\@secondoftwo|, and |ifnew| will just use its third argument.
%    \begin{macrocode}
\protected\def\ekvd@clear@prefixes
  {%
    \let\ekvd@long\ekv@empty
    \let\ekvd@prot\ekv@empty
    \let\ekvd@ifalso\@secondoftwo
    \long\def\ekvd@ifnew##1##2##3{##3}%
  }
\ekvd@clear@prefixes
%    \end{macrocode}
% \end{macro}
%
% \begin{macro}{\ekvdefinekeys}
% This is the one front-facing macro which provides the interface to define
% keys. It's using |\ekvparse| to handle the \kv\ list, the interpretation will
% be done by |\ekvd@noarg| and |\ekvd@|. The \set\ for which the keys should be
% defined is stored in |\ekvd@set|.
%    \begin{macrocode}
\protected\def\ekvdefinekeys#1%
  {%
    \def\ekvd@set{#1}%
    \ekvparse\ekvd@noarg\ekvd@arg
  }
%    \end{macrocode}
% \end{macro}
%
% \begin{macro}[internal]{\ekvd@noarg,\ekvd@arg,\ekvd@handle}
% |\ekvd@noarg| and |\ekvd@arg| store whether there was a value in the \kv\
% pair. |\ekvd@handle| has to test whether there is a space inside the key and
% if so calls the prefix grabbing routine, else we throw an error and ignore the
% key.
%    \begin{macrocode}
\protected\long\def\ekvd@noarg#1%
  {%
    \let\ekvd@ifnoarg\@firstoftwo
    \expandafter\ekvd@handle\detokenize{#1}\ekvd@stop{}%
  }
\protected\long\def\ekvd@arg#1%
  {%
    \let\ekvd@ifnoarg\@secondoftwo
    \expandafter\ekvd@handle\detokenize{#1}\ekvd@stop
  }
\protected\long\def\ekvd@handle#1\ekvd@stop#2%
  {%
    \ekvd@clear@prefixes
    \def\ekvd@cur{#1}%
    \ekvd@ifspace{#1}%
      {\ekvd@prefix\ekv@mark#1\ekv@stop{#2}}%
      \ekvd@err@missing@type
  }
%    \end{macrocode}
% \end{macro}
%
% \begin{macro}[internal]{\ekvd@prefix,\ekvd@prefix@}
% \expkvd\ separates prefixes into two groups, the first being prefixes in the
% \TeX\ sense (|long| and |protected|) which use |@p@| in their name, the other
% being key-types (|code|, |int|, \emph{etc.}) which use |@t@| instead.
% |\ekvd@prefix| splits at the first space and checks whether its a |@p@|
% or |@t@| type prefix. If it is neither throw an error and gobble the
% definition (the value).
%    \begin{macrocode}
\protected\def\ekvd@prefix#1 {\ekv@strip{#1}\ekvd@prefix@\ekv@mark}
\protected\def\ekvd@prefix@#1#2\ekv@stop
  {%
    \ekv@ifdefined{ekvd@t@#1}%
      {\ekv@strip{#2}{\csname ekvd@t@#1\endcsname}}%
      {%
        \ekv@ifdefined{ekvd@p@#1}%
          {\csname ekvd@p@#1\endcsname\ekvd@prefix@after@p{#2}}%
          {\ekvd@err@undefined@prefix{#1}\@gobble}%
      }%
  }
%    \end{macrocode}
% \end{macro}
%
% \begin{macro}[internal]{\ekvd@prefix@after@p}
% The |@p@| type prefixes are all just modifying a following |@t@| type, so they
% will need to search for another prefix. This is true for all of them, so we
% use a macro to handle this. It'll throw an error if there is no other prefix.
%    \begin{macrocode}
\protected\def\ekvd@prefix@after@p#1%
  {%
    \ekvd@ifspace{#1}%
      {\ekvd@prefix#1\ekv@stop}%
      {\ekvd@err@missing@type\@gobble}%
  }
%    \end{macrocode}
% \end{macro}
%
% \begin{macro}[internal]
%   {\ekvd@p@long,\ekvd@p@protected,\ekvd@p@protect,\ekvd@p@also,\ekvd@p@new}
% Define the |@p@| type prefixes, they all just store some information in a
% temporary macro.
%    \begin{macrocode}
\protected\def\ekvd@p@long{\let\ekvd@long\long}
\protected\def\ekvd@p@protected{\let\ekvd@prot\protected}
\let\ekvd@p@protect\ekvd@p@protected
\protected\def\ekvd@p@also{\let\ekvd@ifalso\@firstoftwo}
\protected\def\ekvd@p@new{\let\ekvd@ifnew\ekvd@assert@new}
%    \end{macrocode}
% \end{macro}
%
% \subsubsection{Key Types}
%
% \begin{macro}[internal]{\ekvd@type@set,\ekvd@t@set}
% The |set| type is quite straight forward, just define a |NoVal| key to call
% |\ekvchangeset|.
%    \begin{macrocode}
\protected\def\ekvd@type@set#1#2%
  {%
    \ekvd@assert@not@long
    \ekvd@assert@not@protected
    \ekvd@ifnew{NoVal}{#1}%
      {%
        \ekv@ifempty{#2}%
          {\ekvd@err@missing@definition}%
          {%
            \ekvd@ifalso
              {%
                \ekv@expargtwice{\ekvd@add@noval{#1}}%
                  {\ekvchangeset{#2}}%
                  \ekvd@assert@not@protected@also
              }%
              {\ekv@expargtwice{\ekvdefNoVal\ekvd@set{#1}}{\ekvchangeset{#2}}}%
          }%
      }%
  }
\protected\def\ekvd@t@set#1#2%
  {%
    \ekvd@ifnoarg
      {\ekvd@type@set{#1}{#1}}%
      {\ekvd@type@set{#1}{#2}}%
  }
%    \end{macrocode}
% \end{macro}
%
% \begin{macro}[internal]{\ekvd@type@noval,\ekvd@t@noval,\ekvd@t@enoval}
% Another pretty simple type, |noval| just needs to assert that there is a
% definition and that |long| wasn't specified.
% There are types where the difference in the variants is so small, that we
% define a common handler for them, those common handlers are named with
% |@type@|. |noval| and |enoval| are so similar that we can use such a |@type@|
% macro, even if we could've done |noval| in a slightly faster way without it.
%    \begin{macrocode}
\protected\long\def\ekvd@type@noval#1#2#3%
  {%
    \ekvd@ifnew{NoVal}{#2}%
      {%
        \ekvd@assert@arg
          {%
            \ekvd@assert@not@long
            \ekvd@prot#1\ekvd@tmp{#3}%
            \ekvd@ifalso
              {\ekv@exparg{\ekvd@add@noval{#2}}\ekvd@tmp{}}%
              {\ekvletNoVal\ekvd@set{#2}\ekvd@tmp}%
          }%
      }%
  }
\protected\def\ekvd@t@noval{\ekvd@type@noval\def}
\protected\def\ekvd@t@enoval{\ekvd@type@noval\edef}
%    \end{macrocode}
% \end{macro}
%
% \begin{macro}[internal]{\ekvd@type@code,\ekvd@t@code,\ekvd@t@ecode}
% |code| is simple as well, |ecode| has to use |\edef| on a temporary macro,
% since \expkv\ doesn't provide an |\ekvedef|.
%    \begin{macrocode}
\protected\long\def\ekvd@type@code#1#2#3%
  {%
    \ekvd@ifnew{}{#2}%
      {%
        \ekvd@assert@arg
          {%
            \ekvd@prot\ekvd@long#1\ekvd@tmp##1{#3}%
            \ekvd@ifalso
              {\ekv@exparg{\ekvd@add@val{#2}}{\ekvd@tmp{##1}}{}}%
              {\ekvlet\ekvd@set{#2}\ekvd@tmp}%
          }%
      }%
  }
\protected\def\ekvd@t@code{\ekvd@type@code\def}
\protected\def\ekvd@t@ecode{\ekvd@type@code\edef}
%    \end{macrocode}
% \end{macro}
%
% \begin{macro}[internal]
%   {
%     \ekvd@type@default,\ekvd@t@default,\ekvd@t@qdefault,\ekvd@t@odefault,
%     \ekvd@t@fdefault
%   }
% |\ekvd@type@default| asserts there was an argument, also the key for which one
% wants to set a default has to be already defined (this is not so important for
% |default|, but |qdefault| requires is). If everything is good, |\edef| a
% temporary macro that expands |\ekvd@set| and the |\csname| for the key, and in
% the case of |qdefault| does the first expansion step of the key-macro.
%    \begin{macrocode}
\protected\long\def\ekvd@type@default#1#2#3#4%
  {%
    \ekvd@assert@arg
      {%
        \ekvifdefined\ekvd@set{#3}%
          {%
            \ekvd@assert@not@new
            \ekvd@assert@not@long
            \ekvd@prot\edef\ekvd@tmp
              {%
                \ekv@unexpanded\expandafter#1%
                  {#2\csname\ekv@name\ekvd@set{#3}\endcsname{#4}}%
              }%
            \ekvd@ifalso
              {\ekv@exparg{\ekvd@add@noval{#3}}\ekvd@tmp{}}%
              {\ekvletNoVal\ekvd@set{#3}\ekvd@tmp}%
          }%
          {\ekvd@err@undefined@key{#3}}%
      }%
  }
\protected\def\ekvd@t@default{\ekvd@type@default{}{}}
\protected\def\ekvd@t@qdefault{\ekvd@type@default{\expandafter\expandafter}{}}
\let\ekvd@t@odefault\ekvd@t@qdefault
\protected\def\ekvd@t@fdefault{\ekvd@type@default{}{\romannumeral`\^^@}}
%    \end{macrocode}
% \end{macro}
%
% \begin{macro}[internal]{\ekvd@t@edefault}
% |edefault| is too different from |default| and |qdefault| to reuse the
% |@type@| macro, as it doesn't need |\unexpanded| inside of |\edef|.
%    \begin{macrocode}
\protected\long\def\ekvd@t@edefault#1#2%
  {%
    \ekvd@assert@arg
      {%
        \ekvifdefined\ekvd@set{#1}%
          {%
            \ekvd@assert@not@new
            \ekvd@assert@not@long
            \ekvd@prot\edef\ekvd@tmp
              {\csname\ekv@name\ekvd@set{#1}\endcsname{#2}}%
            \ekvd@ifalso
              {\ekv@exparg{\ekvd@add@noval{#1}}\ekvd@tmp{}}%
              {\ekvletNoVal\ekvd@set{#1}\ekvd@tmp}%
          }%
          {\ekvd@err@undefined@key{#1}}%
      }%
  }
%    \end{macrocode}
% \end{macro}
%
% \begin{macro}[internal]
%   {\ekvd@t@initial,\ekvd@t@oinitial,\ekvd@t@finitial,\ekvd@t@einitial}
%    \begin{macrocode}
\long\def\ekvd@type@initial#1#2#3#4%
  {%
    \ekvd@assert@not@new
    \ekvd@assert@not@also
    \ekvd@assert@not@long
    \ekvd@assert@not@protected
    \ekvd@ifnoarg
      {%
        \ekvifdefinedNoVal\ekvd@set{#3}%
          {\csname\ekv@name\ekvd@set{#3}N\endcsname}%
          {\ekvd@err@undefined@noval{#3}}%
      }%
      {%
        \ekvifdefined\ekvd@set{#3}%
          {%
            #1{#2#4}%
            \csname\ekv@name\ekvd@set{#3}\expandafter\endcsname\expandafter
              {\ekvd@tmp}%
          }%
          {\ekvd@err@undefined@key{#3}}%
      }%
  }
\def\ekvd@t@initial{\ekvd@type@initial{\def\ekvd@tmp}{}}
\def\ekvd@t@oinitial{\ekvd@type@initial{\ekv@exparg{\def\ekvd@tmp}}{}}
\def\ekvd@t@einitial{\ekvd@type@initial{\edef\ekvd@tmp}{}}
\def\ekvd@t@finitial
  {\ekvd@type@initial{\ekv@exparg{\def\ekvd@tmp}}{\romannumeral`\^^@}}
%    \end{macrocode}
% \end{macro}
%
% \begin{macro}[internal]
%   {
%     \ekvd@type@bool,\ekvd@t@bool,\ekvd@t@gbool,\ekvd@t@boolTF,\ekvd@t@gboolTF,
%     \ekvd@t@invbool,\ekvd@t@ginvbool,\ekvd@t@invboolTF,\ekvd@t@ginvboolTF,
%   }
% The boolean types are a quicker version of a |choice| that accept |true| and
% |false|, and set up the |NoVal| action to be identical to \texttt{\key=true}.
% The |true| and |false| actions are always just |\let|ting the macro in |#7| to
% some other macro (\emph{e.g.}, \cs[no-index]{iftrue}).
%    \begin{macrocode}
\protected\def\ekvd@type@bool#1#2#3#4#5%
  {%
    \ekvd@ifnew{}{#4}%
      {%
        \ekvd@ifnew{NoVal}{#4}%
          {%
            \ekvd@assert@filledarg{#5}%
              {%
                \ekvd@newlet#5#3%
                \ekvd@type@choice{#4}%
                \protected\ekvdefNoVal\ekvd@set{#4}{#1\let#5#2}%
                \protected\expandafter\def
                  \csname\ekvd@choice@name\ekvd@set{#4}{true}\endcsname
                  {#1\let#5#2}%
                \protected\expandafter\def
                  \csname\ekvd@choice@name\ekvd@set{#4}{false}\endcsname
                  {#1\let#5#3}%
              }%
          }%
      }%
  }
\protected\def\ekvd@t@bool{\ekvd@type@bool{}\iftrue\iffalse}
\protected\def\ekvd@t@gbool{\ekvd@type@bool\global\iftrue\iffalse}
\protected\def\ekvd@t@boolTF{\ekvd@type@bool{}\@firstoftwo\@secondoftwo}
\protected\def\ekvd@t@gboolTF{\ekvd@type@bool\global\@firstoftwo\@secondoftwo}
\protected\def\ekvd@t@invbool{\ekvd@type@bool{}\iffalse\iftrue}
\protected\def\ekvd@t@ginvbool{\ekvd@type@bool\global\iffalse\iftrue}
\protected\def\ekvd@t@invboolTF{\ekvd@type@bool{}\@secondoftwo\@firstoftwo}
\protected\def\ekvd@t@ginvboolTF
  {\ekvd@type@bool\global\@secondoftwo\@firstoftwo}
%    \end{macrocode}
% \end{macro}
%
% \begin{macro}[internal]
%   {
%     \ekvd@type@boolpair,
%     \ekvd@t@boolpair,\ekvd@t@gboolpair,\ekvd@t@boolpairTF,\ekvd@t@gboolpairTF
%   }
% The boolean pair types are essentially the same as the boolean types, but set
% two macros instead of one.
%    \begin{macrocode}
\protected\def\ekvd@type@boolpair#1#2#3#4#5#6%
  {%
    \ekvd@ifnew{}{#4}%
      {%
        \ekvd@ifnew{NoVal}{#4}%
          {%
            \ekvd@newlet#5#3%
            \ekvd@newlet#6#2%
            \ekvd@type@choice{#4}%
            \protected\ekvdefNoVal\ekvd@set{#4}{#1\let#5#2#1\let#6#3}%
            \protected\expandafter\def
              \csname\ekvd@choice@name\ekvd@set{#4}{true}\endcsname
              {#1\let#5#2#1\let#6#3}%
            \protected\expandafter\def
              \csname\ekvd@choice@name\ekvd@set{#4}{false}\endcsname
              {#1\let#5#3#1\let#6#2}%
          }%
      }%
  }
\protected\def\ekvd@t@boolpair#1#2%
  {\ekvd@assert@twoargs{#2}{\ekvd@type@boolpair{}\iftrue\iffalse{#1}#2}}
\protected\def\ekvd@t@gboolpair#1#2%
  {\ekvd@assert@twoargs{#2}{\ekvd@type@boolpair\global\iftrue\iffalse{#1}#2}}
\protected\def\ekvd@t@boolpairTF#1#2%
  {%
    \ekvd@assert@twoargs{#2}%
      {\ekvd@type@boolpair{}\@firstoftwo\@secondoftwo{#1}#2}%
  }
\protected\def\ekvd@t@gboolpairTF#1#2%
  {%
    \ekvd@assert@twoargs{#2}%
      {\ekvd@type@boolpair\global\@firstoftwo\@secondoftwo{#1}#2}%
  }
%    \end{macrocode}
% \end{macro}
%
% \begin{macro}[internal]
%   {
%     \ekvd@type@data,\ekvd@t@data,\ekvd@t@gdata,\ekvd@t@dataT,\ekvd@t@gdataT
%   }
%    \begin{macrocode}
\protected\def\ekvd@type@data#1#2#3#4#5#6%
  {%
    \ekvd@ifnew{}{#5}%
      {%
        \ekvd@assert@filledarg{#6}%
          {%
            \ekvd@newlet#6#1%
            \ekvd@ifalso
              {%
                \let\ekvd@prot\protected
                \ekvd@add@val{#5}{\long#2#6####1#3{####1{#4}}}{}%
              }%
              {%
                \protected\ekvd@long\ekvdef\ekvd@set{#5}%
                  {\long#2#6####1#3{####1{#4}}}%
              }%
          }%
      }%
  }
\protected\def\ekvd@t@data
  {\ekvd@type@data\@secondoftwo\edef{####2}{\ekv@unexpanded{##1}}}
\protected\def\ekvd@t@edata{\ekvd@type@data\@secondoftwo\edef{####2}{##1}}
\protected\def\ekvd@t@gdata
  {\ekvd@type@data\@secondoftwo\xdef{####2}{\ekv@unexpanded{##1}}}
\protected\def\ekvd@t@xdata{\ekvd@type@data\@secondoftwo\xdef{####2}{##1}}
\protected\def\ekvd@t@dataT
  {\ekvd@type@data\@gobble\edef{}{\ekv@unexpanded{##1}}}
\protected\def\ekvd@t@edataT{\ekvd@type@data\@gobble\edef{}{##1}}
\protected\def\ekvd@t@gdataT
  {\ekvd@type@data\@gobble\xdef{}{\ekv@unexpanded{##1}}}
\protected\def\ekvd@t@xdataT{\ekvd@type@data\@gobble\xdef{}{##1}}
%    \end{macrocode}
% \end{macro}
%
% \begin{macro}[internal]{\ekvd@type@box,\ekvd@t@box,\ekvd@t@gbox}
% Set up our boxes. Though we're a generic package we want to be colour safe, so
% we put an additional grouping level inside the box contents, for the case that
% someone uses \pkg{color}. |\ekvd@newreg| is a small wrapper which tests
% whether the first argument is defined and if not does
% |\csname new#2\endcsname#1|.
%    \begin{macrocode}
\protected\def\ekvd@type@box#1#2#3%
  {%
    \ekvd@ifnew{}{#2}%
      {%
        \ekvd@assert@filledarg{#3}%
          {%
            \ekvd@newreg#3{box}%
            \ekvd@ifalso
              {%
                \let\ekvd@prot\protected
                \ekvd@add@val{#2}{#1\setbox#3=\hbox{\begingroup##1\endgroup}}{}%
              }%
              {%
                \protected\ekvd@long\ekvdef\ekvd@set{#2}%
                  {#1\setbox#3=\hbox{\begingroup##1\endgroup}}%
              }%
          }%
      }%
  }
\protected\def\ekvd@t@box{\ekvd@type@box{}}
\protected\def\ekvd@t@gbox{\ekvd@type@box\global}
%    \end{macrocode}
% \end{macro}
%
% \begin{macro}[internal]{\ekvd@type@toks,\ekvd@t@toks,\ekvd@t@gtoks}
% Similar to |box|, but set the |toks|.
%    \begin{macrocode}
\protected\def\ekvd@type@toks#1#2#3%
  {%
    \ekvd@ifnew{}{#2}%
      {%
        \ekvd@assert@filledarg{#3}%
          {%
            \ekvd@newreg#3{toks}%
            \ekvd@ifalso
              {%
                \let\ekvd@prot\protected
                \ekvd@add@val{#2}{#1#3={##1}}{}%
              }%
              {\protected\ekvd@long\ekvdef\ekvd@set{#2}{#1#3={##1}}}%
          }%
      }%
  }
\protected\def\ekvd@t@toks{\ekvd@type@toks{}}
\protected\def\ekvd@t@gtoks{\ekvd@type@toks\global}
%    \end{macrocode}
% \end{macro}
%
% \begin{macro}[internal]{\ekvd@type@preapptoks,\ekvd@t@apptoks,\ekvd@t@gapptoks}
% Just like |toks|, but expand the current contents of the |toks| register to
% append the new contents.
%    \begin{macrocode}
\ekvd@ifprimitive\toksapp
  {%
    \protected\def\ekvd@type@preapptoks#1#2#3%
      {%
        \ekvd@ifnew{}{#2}%
          {%
            \ekvd@assert@filledarg{#3}%
              {%
                \ekvd@newreg#3{toks}%
                \ekvd@ifalso
                  {%
                    \let\ekvd@prot\protected
                    \ekvd@add@val{#2}{#1#3{##1}}{}%
                  }%
                  {\protected\ekvd@long\ekvdef\ekvd@set{#2}{#1#3{##1}}}%
              }%
          }%
      }
    \protected\def\ekvd@t@apptoks{\ekvd@type@preapptoks\toksapp}
    \protected\def\ekvd@t@gapptoks{\ekvd@type@preapptoks\gtoksapp}
    \protected\def\ekvd@t@pretoks{\ekvd@type@preapptoks\tokspre}
    \protected\def\ekvd@t@gpretoks{\ekvd@type@preapptoks\gtokspre}
  }
  {%
    \protected\def\ekvd@type@apptoks#1#2#3%
      {%
        \ekvd@ifnew{}{#2}%
          {%
            \ekvd@assert@filledarg{#3}%
              {%
                \ekvd@newreg#3{toks}%
                \ekvd@ifalso
                  {%
                    \let\ekvd@prot\protected
                    \ekvd@add@val{#2}{#1#3=\expandafter{\the#3##1}}{}%
                  }%
                  {%
                    \protected\ekvd@long\ekvdef\ekvd@set{#2}%
                      {#1#3=\expandafter{\the#3##1}}%
                  }%
              }%
          }%
      }
    \protected\def\ekvd@t@apptoks{\ekvd@type@apptoks{}}
    \protected\def\ekvd@t@gapptoks{\ekvd@type@apptoks\global}
    \newtoks\ekvd@toks
    \protected\def\ekvd@type@pretoks#1#2#3%
      {%
        \ekvd@ifnew{}{#2}%
          {%
            \ekvd@assert@filledarg{#3}%
              {%
                \ekvd@newreg#3{toks}%
                \ekvd@ifalso
                  {%
                    \let\ekvd@prot\protected
                    \ekvd@add@val{#2}%
                      {%
                        \ekvd@toks={##1}%
                        #1#3=\expandafter{\the\expandafter\ekvd@toks\the#3}%
                      }%
                      {}%
                  }%
                  {%
                    \protected\ekvd@long\ekvdef\ekvd@set{#2}%
                      {%
                        \ekvd@toks={##1}%
                        #1#3=\expandafter{\the\expandafter\ekvd@toks\the#3}%
                      }%
                  }%
              }%
          }%
      }
    \protected\def\ekvd@t@pretoks{\ekvd@type@pretoks{}}
    \protected\def\ekvd@t@gpretoks{\ekvd@type@pretoks\global}
  }
%    \end{macrocode}
% \end{macro}
%
% \begin{macro}[internal]
%   {
%     \ekvd@type@reg,\ekvd@t@int,\ekvd@t@eint,\ekvd@t@gint,\ekvd@t@xint,
%     \ekvd@t@dimen,\ekvd@t@edimen,\ekvd@t@gdimen,\ekvd@t@xdimen,
%     \ekvd@t@skip,\ekvd@t@eskip,\ekvd@t@gskip,\ekvd@t@xskip
%   }
% The |\ekvd@type@reg| can handle all the types for which the assignment will
% just be \texttt{\meta{register}=\meta{value}}.
%    \begin{macrocode}
\protected\def\ekvd@type@reg#1#2#3#4#5#6%
  {%
    \ekvd@ifnew{}{#5}%
      {%
        \ekvd@assert@filledarg{#6}%
          {%
            \ekvd@newreg#6{#1}%
            \ekvd@ifalso
              {%
                \let\evkd@prot\protected
                \ekvd@add@val{#5}{#2#6=#3##1#4\relax}{}%
              }%
              {\protected\ekvd@long\ekvdef\ekvd@set{#5}{#2#6=#3##1#4\relax}}%
          }%
      }%
  }
\protected\def\ekvd@t@int{\ekvd@type@reg{count}{}{}{}}
\protected\def\ekvd@t@eint{\ekvd@type@reg{count}{}\numexpr\relax}
\protected\def\ekvd@t@gint{\ekvd@type@reg{count}\global{}{}}
\protected\def\ekvd@t@xint{\ekvd@type@reg{count}\global\numexpr\relax}
\protected\def\ekvd@t@dimen{\ekvd@type@reg{dimen}{}{}{}}
\protected\def\ekvd@t@edimen{\ekvd@type@reg{dimen}{}\dimexpr\relax}
\protected\def\ekvd@t@gdimen{\ekvd@type@reg{dimen}\global{}{}}
\protected\def\ekvd@t@xdimen{\ekvd@type@reg{dimen}\global\dimexpr\relax}
\protected\def\ekvd@t@skip{\ekvd@type@reg{skip}{}{}{}}
\protected\def\ekvd@t@eskip{\ekvd@type@reg{skip}{}\glueexpr\relax}
\protected\def\ekvd@t@gskip{\ekvd@type@reg{skip}\global{}{}}
\protected\def\ekvd@t@xskip{\ekvd@type@reg{skip}\global\glueexpr\relax}
%    \end{macrocode}
% \end{macro}
%
% \begin{macro}[internal]{\ekvd@type@store,\ekvd@t@store,\ekvd@t@gstore}
% The none-expanding |store| types use an |\edef| or |\xdef| and |\unexpanded|
% to be able to also store |#| easily.
%    \begin{macrocode}
\protected\def\ekvd@type@store#1#2#3#4%
  {%
    \ekvd@ifnew{}{#3}%
      {%
        \ekvd@assert@filledarg{#4}%
          {%
            \ekvd@newlet#4\ekv@empty
            \ekvd@ifalso
              {%
                \let\ekvd@prot\protected
                \ekvd@add@val{#3}{#1#4{#2}}{}%
              }%
              {\protected\ekvd@long\ekvdef\ekvd@set{#3}{#1#4{#2}}}%
          }%
      }%
  }
\protected\def\ekvd@t@store{\ekvd@type@store\edef{\ekv@unexpanded{##1}}}
\protected\def\ekvd@t@gstore{\ekvd@type@store\xdef{\ekv@unexpanded{##1}}}
\protected\def\ekvd@t@estore{\ekvd@type@store\edef{##1}}
\protected\def\ekvd@t@xstore{\ekvd@type@store\xdef{##1}}
%    \end{macrocode}
% \end{macro}
%
% \begin{macro}[internal]
%   {
%     \ekvd@type@meta,\ekvd@type@meta@a,\ekvd@type@meta@b,\ekvd@type@meta@c,
%     \ekvd@t@meta,\ekvd@t@nmeta
%   }
% |meta| sets up things such that another instance of |\ekvset| will be run on
% the argument, with the same \set.
%    \begin{macrocode}
\protected\long\def\ekvd@type@meta#1#2#3#4#5#6#7%
  {%
    \ekvd@ifnew{#1}{#6}%
      {%
        \ekvd@assert@filledarg{#7}%
          {%
            \edef\ekvd@tmp{\ekvd@set}%
            \expandafter\ekvd@type@meta@a\expandafter{\ekvd@tmp}{#7}{#2}%
            \ekvd@ifalso
              {\ekv@exparg{#3{#6}}{\ekvd@tmp#4}{#5}}%
              {\csname ekvlet#1\endcsname\ekvd@set{#6}\ekvd@tmp}%
          }%
      }%
  }
\protected\long\def\ekvd@type@meta@a#1#2%
  {%
    \expandafter\ekvd@type@meta@b\expandafter{\ekvset{#1}{#2}}%
  }
\protected\def\ekvd@type@meta@b
  {%
    \expandafter\ekvd@type@meta@c\expandafter
  }
\protected\long\def\ekvd@type@meta@c#1#2%
  {%
    \ekvd@prot\ekvd@long\def\ekvd@tmp#2{#1}%
  }
\protected\def\ekvd@t@meta{\ekvd@type@meta{}{##1}\ekvd@add@val{{##1}}{}}
\protected\def\ekvd@t@nmeta
  {%
    \ekvd@assert@not@long
    \ekvd@type@meta{NoVal}{}\ekvd@add@noval{}\ekvd@assert@not@long@also
  }
%    \end{macrocode}
% \end{macro}
%
% \begin{macro}[internal]
%   {\ekvd@type@smeta,\ekvd@type@smeta@,\ekvd@t@smeta,\ekvd@t@snmeta}
% |smeta| is pretty similar to |meta|, but needs two arguments inside of \val,
% such that the first is the \set\ for which the sub-|\ekvset| and the second is
% the \kv\ list.
%    \begin{macrocode}
\protected\long\def\ekvd@type@smeta#1#2#3#4#5#6#7%
  {%
    \ekvd@ifnew{#1}{#6}%
      {%
        \ekvd@assert@twoargs{#7}%
          {%
            \ekvd@type@meta@a#7{#2}%
            \ekvd@ifalso
              {\ekv@exparg{#3{#6}}{\ekvd@tmp#4}{#5}}%
              {\csname ekvlet#1\endcsname\ekvd@set{#6}\ekvd@tmp}%
          }%
      }%
  }
\protected\def\ekvd@t@smeta{\ekvd@type@smeta{}{##1}\ekvd@add@val{{##1}}{}}
\protected\def\ekvd@t@snmeta
  {%
    \ekvd@assert@not@long
    \ekvd@type@smeta{NoVal}{}\ekvd@add@noval{}\ekvd@assert@not@long@also
  }
%    \end{macrocode}
% \end{macro}
%
% \begin{macro}[internal]
%   {
%     \ekvd@type@choice,
%     \ekvd@populate@choice,\ekvd@populate@choice@,\ekvd@populate@choice@noarg,
%     \ekvd@choice@prefix,\ekvd@choice@prefix@,\ekvd@choice@prefix@done,
%     \ekvd@choice@p@protected,\ekvd@choice@p@protect,
%     \ekvd@choice@p@long,\ekvd@choice@p@long@,
%     \ekvd@t@choice
%   }
% The |choice| type is by far the most complex type, as we have to run a
% sub-parser on the choice-definition list, which should support the |@p@| type
% prefixes as well (but |long| will always throw an error, as they are not
% allowed to be long). |\ekvd@type@choice| will just define the choice-key, the
% handling of the choices definition will be done by |\ekvd@populate@choice|.
%    \begin{macrocode}
\protected\def\ekvd@type@choice#1%
  {%
    \ekvd@assert@not@long
    \ekv@exparg{\ekvd@prot\def\ekvd@tmp##1}%
      {%
        \expandafter\ekvd@h@choice\expandafter
          {\ekvd@choice@name\ekvd@set{#1}{##1}}%
      }%
    \ekvd@ifalso
      {%
        \ekvd@assert@val{#1}%
          {%
            \ekvd@if@not@already@choice{#1}%
              {%
                \ekv@exparg
                  {%
                    \expandafter\ekvd@add@aux
                      \csname\ekv@name\ekvd@set{#1}\endcsname{{##1}}{#1}%
                  }%
                  {\ekvd@tmp{##1}}%
                  {\ekvd@long\ekvdef}\ekvd@assert@not@long@also
              }%
          }%
      }%
      {\ekvlet\ekvd@set{#1}\ekvd@tmp}%
  }
%    \end{macrocode}
% |\ekvd@populate@choice| just uses |\ekvparse| and then gives control to
% |\ekvd@populate@choice@noarg|, which throws an error, and
% |\ekvd@populate@choice@|.
%    \begin{macrocode}
\protected\def\ekvd@populate@choice
  {%
    \ekvparse\ekvd@populate@choice@noarg\ekvd@populate@choice@
  }
\protected\long\def\ekvd@populate@choice@noarg#1%
  {%
    \expandafter\ekvd@err@missing@definition@msg\expandafter{\ekvd@cur : #1}%
  }
%    \end{macrocode}
% |\ekvd@populate@choice@| runs the prefix-test, if there is none we can
% directly define the choice, for that |\ekvd@set@choice| will expand to the
% current choice-key's name, which will have been defined by |\ekvd@t@choice|.
% If there is a prefix run the prefix grabbing routine, which was altered for
% |@type@choice|.
%    \begin{macrocode}
\protected\long\def\ekvd@populate@choice@#1#2%
  {%
    \ekvd@clear@prefixes
    \ekvd@ifspace{#1}%
      {\ekvd@choice@prefix{\ekv@mark#1}\ekv@mark#1\ekv@stop}%
      {%
        \expandafter\edef
          \csname\ekvd@choice@name\ekvd@set\ekvd@set@choice{#1}\endcsname
      }%
      {\unexpanded{#2}}%
  }
\protected\def\ekvd@choice@prefix#1#2
  {%
    \ekv@strip{#2}{\ekvd@choice@prefix@{#1}}\ekv@mark
  }
\protected\def\ekvd@choice@prefix@#1#2#3\ekv@stop
  {%
    \ekv@ifdefined{ekvd@choice@p@#2}%
      {%
        \csname ekvd@choice@p@#2\endcsname
        \ekvd@ifspace{#3}%
          {\ekvd@choice@prefix{#3}#3\ekv@stop}%
          {\ekvd@choice@prefix@done{#3}}%
      }%
      {\ekvd@choice@prefix@done{#1}}%
  }
\protected\def\ekvd@choice@prefix@done#1%
  {%
    \ekvd@prot\expandafter\edef
      \csname
        \ekv@strip{#1}{\ekvd@choice@name\ekvd@set\ekvd@set@choice}%
      \endcsname
  }
\protected\def\ekvd@choice@p@protected{\let\ekvd@prot\protected}
\let\ekvd@choice@p@protect\ekvd@choice@p@protected
\protected\def\ekvd@choice@invalid@p#1\ekvd@ifspace#2%
  {%
    \expandafter\ekvd@choice@invalid@p@\expandafter{\ekv@gobble@mark#2}{#1}%
    \ekvd@ifspace{#2}%
  }
\protected\def\ekvd@choice@invalid@p@#1#2%
  {%
    \expandafter\ekvd@err@no@prefix@msg\expandafter{\ekvd@cur : #2 #1}{#2}%
  }
\protected\def\ekvd@choice@p@long{\ekvd@choice@invalid@p{long}}%
\protected\def\ekvd@choice@p@also{\ekvd@choice@invalid@p{also}}%
\protected\def\ekvd@choice@p@new{\ekvd@choice@invalid@p{new}}%
%    \end{macrocode}
% Finally we're able to set up the |@t@choice| macro, which has to store the
% current choice-key's name, define the key, and parse the available choices.
%    \begin{macrocode}
\protected\long\def\ekvd@t@choice#1#2%
  {%
    \ekvd@ifnew{}{#1}%
      {%
        \ekvd@assert@arg
          {%
            \ekvd@type@choice{#1}%
            \def\ekvd@set@choice{#1}%
            \ekvd@populate@choice{#2}%
          }%
      }%
  }
%    \end{macrocode}
% \end{macro}
%
% \begin{macro}[internal]{\ekvd@t@choice-store,\ekvd@t@choice-enum}
%   These two types define a special kind of |choice| key and are quite similar,
%   the only difference is what the different choices do (hence they use a
%   shared initialisation which differs in the chosen |populate| step).
%    \begin{macrocode}
\protected\long\expandafter\def\csname ekvd@t@choice-store\endcsname
  {\ekvd@type@choicespecial\ekvd@populate@choicestore}
\protected\long\expandafter\def\csname ekvd@t@choice-enum\endcsname
  {\ekvd@type@choicespecial\ekvd@populate@choiceenum}
%    \end{macrocode}
% \begin{macro}[internal]{\ekvd@type@choicespecial}
%   Initialise similar to a |choice| key. The difference is that we require two
%   arguments (which we assert), a macro to store things in, and a |csv|-list
%   containing the allowed values. |#1| is the |populate| macro according to the
%   type used.
%    \begin{macrocode}
\protected\long\def\ekvd@type@choicespecial#1#2#3%
  {%
    \ekvd@ifnew{}{#2}%
      {%
        \ekvd@assert@twoargs{#3}%
          {%
            \ekvd@type@choice{#2}%
            \def\ekvd@set@choice{#2}%
            #1#3%
          }%
      }%
  }
%    \end{macrocode}
% \end{macro}
% \begin{macro}[internal]
%   {\ekvd@populate@choicestore,\ekvd@populate@choicestore@}
%   We initialise the storing macro if it doesn't yet exist, and then we loop
%   over the value list. The |\edef|s with |\unexpanded| are both necessary to
%   be able to store macro parameter tokens (the outer protects at define time,
%   the inner at use time).
%    \begin{macrocode}
\protected\long\def\ekvd@populate@choicestore#1%
  {%
    \ekvd@newlet#1\ekv@empty
    \ekvcsvloop{\ekvd@populate@choicestore@#1}%
  }
\protected\long\def\ekvd@populate@choicestore@#1#2%
  {%
    \protected\expandafter\edef
      \csname\ekvd@choice@name\ekvd@set\ekvd@set@choice{#2}\endcsname
      {\unexpanded{\edef#1{\unexpanded{#2}}}}%
  }
%    \end{macrocode}
% \end{macro}
% \begin{macro}[internal]
%   {\ekvd@populate@choiceenum,\ekvd@populate@choiceenum@}
%   This is similar to the population of a |choice-store| type, but instead of
%   storing the values in a macro this initialises a count and stores the
%   position of the value in the list inside that count (zero-indexed). The
%   space is necessary to terminate the number scanning, which is the reason we
%   use |\@firstofone| (so that the space after the macro name isn't gobbled by
%   \TeX).
%    \begin{macrocode}
\protected\long\def\ekvd@populate@choiceenum#1%
  {%
    \ekvd@newreg#1{count}%
    \def\ekvd@tmp{0}%
    \ekvcsvloop{\ekvd@populate@choiceenum@#1}%
  }
\protected\long\def\ekvd@populate@choiceenum@#1#2%
  {%
    \protected\expandafter\edef
      \csname\ekvd@choice@name\ekvd@set\ekvd@set@choice{#2}\endcsname
      {#1=\@firstofone{\ekvd@tmp} }%
    \edef\ekvd@tmp{\the\numexpr\ekvd@tmp+1\relax}%
  }
%    \end{macrocode}
% \end{macro}
% \end{macro}
%
% \begin{macro}[internal]{\ekvd@t@unknown-choice}
%    \begin{macrocode}
\protected\long\expandafter\def\csname ekvd@t@unknown-choice\endcsname#1#2%
  {%
    \ekvd@assert@new@for@name{\ekvd@unknown@choice@name\ekvd@set{#1}}%
      {%
        \ekvd@assert@arg
          {%
            \ekvd@assert@not@long
            \ekvd@assert@not@also
            \ekvd@prot\expandafter
            \def\csname\ekvd@unknown@choice@name\ekvd@set{#1}\endcsname##1{#2}%
          }%
      }%
  }
%    \end{macrocode}
% \end{macro}
%
% \begin{macro}[internal]
%   {\ekvd@t@unknown,\ekvd@type@unknown@code,\ekvd@type@unknown@noval}
%   The |unknown| type has different subtypes which would be the key names for
%   other types. It is first checked whether that subtype is defined, if it
%   isn't throw an error, else use that subtype.
%    \begin{macrocode}
\protected\long\def\ekvd@t@unknown#1#2%
  {%
    \ekv@ifdefined{ekvd@type@unknown@\detokenize{#1}}%
      {\csname ekvd@type@unknown@\detokenize{#1}\endcsname{#2}}%
      \ekvd@err@misused@unknown
  }
%    \end{macrocode}
%   The |unknown noval| type can use |\ekvdefunknownNoVal| directly (after
%   asserting some prefixes).
%    \begin{macrocode}
\protected\long\def\ekvd@type@unknown@noval#1%
  {%
    \ekvd@assert@new@for@name{\ekv@name\ekvd@set{}uN}%
      {%
        \ekvd@assert@arg
          {%
            \ekvd@assert@not@also
            \ekvd@assert@not@long
            \ekvd@prot\ekvdefunknownNoVal\ekvd@set{#1}%
          }%
      }%
  }
%    \end{macrocode}
%   The |unknown code| type uses some trickery during the definition in order to
%   swap out |#1| and |#2| in the user supplied definition. This is done via a
%   temporary macro that stores the definition but gets the parameter numbers
%   reversed while the real definition is done.
%    \begin{macrocode}
\protected\long\def\ekvd@type@unknown@code#1%
  {%
    \ekvd@assert@new@for@name{\ekv@name\ekvd@set{}u}%
      {%
        \ekvd@assert@arg
          {%
            \ekvd@assert@not@also
            \begingroup
              \def\ekvd@tmp##1##2{#1}%
              \ekv@exparg
              {%
                \endgroup
                \ekvd@prot\ekvd@long\ekvdefunknown\ekvd@set
              }%
              {\ekvd@tmp{##2}{##1}}%
          }%
      }%
  }
%    \end{macrocode}
% \end{macro}
%
% \begin{macro}[internal]
%   {
%     \ekvd@type@unknown@redirect,
%     \ekvd@type@unknown@redirect-code,
%     \ekvd@type@unknown@redirect-noval
%   }
%   The |unknown redirect| types also just forward to |\ekvredirectunknown|
%   after asserting some prefixes.
%    \begin{macrocode}
\protected\edef\ekvd@type@unknown@redirect#1%
  {%
    \expandafter\noexpand\csname ekvd@type@unknown@redirect-code\endcsname{#1}%
    \expandafter\noexpand\csname ekvd@type@unknown@redirect-noval\endcsname{#1}%
  }
\protected\expandafter\def\csname ekvd@type@unknown@redirect-code\endcsname#1%
  {%
    \ekvd@assert@new@for@name{\ekv@name\ekvd@set{}u}%
      {%
        \ekvd@assert@arg
          {%
            \ekvd@assert@not@also
            \ekvd@assert@not@protected
            \expandafter\ekvredirectunknown\expandafter{\ekvd@set}{#1}%
          }%
      }%
  }
\protected\expandafter\def\csname ekvd@type@unknown@redirect-noval\endcsname#1%
  {%
    \ekvd@assert@new@for@name{\ekv@name\ekvd@set{}uN}%
      {%
        \ekvd@assert@arg
          {%
            \ekvd@assert@not@also
            \ekvd@assert@not@protected
            \ekvd@assert@not@long
            \expandafter\ekvredirectunknownNoVal\expandafter{\ekvd@set}{#1}%
          }%
      }%
  }
%    \end{macrocode}
% \end{macro}
%
% \subsubsection{Key Type Helpers}
% There are some keys that might need helpers during their execution (not during
% their definition, which are gathered as |@type@| macros). These helpers are
% named |@h@|.
%
% \begin{macro}[internal]{\ekvd@h@choice,\ekvd@h@choice@}
% The |choice| helper will just test whether the given choice was defined, if
% not throw an error expandably, else call the macro which stores the code for
% this choice.
%    \begin{macrocode}
\def\ekvd@h@choice#1%
  {%
    \expandafter\ekvd@h@choice@
      \csname\ifcsname#1\endcsname#1\else relax\fi\endcsname
      {#1}%
  }
\def\ekvd@h@choice@#1#2%
  {%
    \ifx#1\relax
      \ekvd@err@choice@invalid{#2}%
      \expandafter\@gobble
    \fi
    #1%
  }
%    \end{macrocode}
% \end{macro}
%
%
% \subsubsection{Handling \texttt{also}}
%
% \begin{macro}[internal]
%   {\ekvd@add@val,\ekvd@add@noval,\ekvd@add@aux,\ekvd@add@aux@}
%    \begin{macrocode}
\protected\long\def\ekvd@add@val#1#2#3%
  {%
    \ekvd@assert@val{#1}%
      {%
        \expandafter\ekvd@add@aux\csname\ekv@name\ekvd@set{#1}\endcsname{{##1}}%
          {#1}{#2}{\ekvd@long\ekvdef}{#3}%
      }%
  }
\protected\long\def\ekvd@add@noval#1#2#3%
  {%
    \ekvd@assert@noval{#1}%
      {%
        \expandafter\ekvd@add@aux\csname\ekv@name\ekvd@set{#1}N\endcsname{}%
          {#1}{#2}\ekvdefNoVal{#3}%
      }%
  }
\protected\long\def\ekvd@add@aux#1#2%
  {%
    \ekvd@extract@prefixes#1%
    \expandafter\ekvd@add@aux@\expandafter{#1#2}%
  }
\protected\long\def\ekvd@add@aux@#1#2#3#4#5%
  {%
    #5%
    \ekvd@prot#4\ekvd@set{#2}{#1#3}%
  }
%    \end{macrocode}
% \end{macro}
%
% \begin{macro}[internal]
%   {
%     \ekvd@extract@prefixes,\ekvd@extract@prefixes@,
%     \ekvd@extract@prefixes@long,\ekvd@extract@prefixes@prot
%   }
% This macro checks which prefixes were used for the definition of a macro and
% sets |\ekvd@long| and |\ekvd@prot| accordingly.
%    \begin{macrocode}
\protected\def\ekvd@extract@prefixes#1%
  {%
    \expandafter\ekvd@extract@prefixes@\meaning#1\ekvd@stop
  }
%    \end{macrocode}
% In the following definition |#1| will get replaced by |macro:|, |#2| by
% |\long| and |#3| by |\protected| (in each, all tokens will have category
% other). This allows us to parse the |\meaning| of a macro for those strings.
%    \begin{macrocode}
\protected\def\ekvd@extract@prefixes@#1#2#3%
  {%
    \protected\def\ekvd@extract@prefixes@##1#1##2\ekvd@stop
      {%
        \ekvd@extract@prefixes@long
          ##1\ekvd@mark\@firstofone#2\ekvd@mark\@gobble\ekvd@stop
          {\let\ekvd@long\long}%
        \ekvd@extract@prefixes@prot
          ##1\ekvd@mark\@firstofone#3\ekvd@mark\@gobble\ekvd@stop
          {\let\ekvd@prot\protected}%
      }%
    \protected\def\ekvd@extract@prefixes@long##1#2##2\ekvd@mark##3##4\ekvd@stop
      {##3}%
    \protected\def\ekvd@extract@prefixes@prot##1#3##2\ekvd@mark##3##4\ekvd@stop
      {##3}%
  }
%    \end{macrocode}
% We use a temporary macro to expand the three arguments of
% |\ekvd@extract@prefixes@|, which will set up the real meaning of itself and
% the parsing for |\long| and |\protected|.
%    \begin{macrocode}
\begingroup
\edef\ekvd@tmp
  {%
    \endgroup
    \ekvd@extract@prefixes@
      {\detokenize{macro:}}%
      {\string\long}%
      {\string\protected}%
  }
\ekvd@tmp
%    \end{macrocode}
% \end{macro}
%
%
% \subsubsection{Tests}
%
% \begin{macro}[internal]{\ekvd@newlet,\ekvd@newreg}
% These macros test whether a control sequence is defined, if it isn't they
% define it, either via |\let| or via the correct \cs[no-index]{new\meta{reg}}.
%    \begin{macrocode}
\protected\def\ekvd@newlet#1#2%
  {%
    \ifdefined#1\ekv@fi@gobble\fi\@firstofone{\let#1#2}%
  }
\protected\def\ekvd@newreg#1#2%
  {%
    \ifdefined#1\ekv@fi@gobble\fi\@firstofone{\csname new#2\endcsname#1}%
  }
%    \end{macrocode}
% \end{macro}
%
% \begin{macro}[internal]
%   {\ekvd@assert@twoargs,\ekvd@ifnottwoargs,\ekvd@ifempty@gtwo}
% A test for exactly two tokens can be reduced for an empty-test after gobbling
% two tokens, in the case that there are fewer tokens than two in the argument,
% only macros will be gobbled that are needed for the true branch, which doesn't
% hurt, and if there are more this will not be empty.
%    \begin{macrocode}
\long\def\ekvd@assert@twoargs#1%
  {%
    \ekvd@ifnottwoargs{#1}{\ekvd@err@missing@definition}%
  }
\long\def\ekvd@ifnottwoargs#1%
  {%
    \ekvd@ifempty@gtwo#1\ekv@ifempty@B
      \ekv@ifempty@false\ekv@ifempty@A\ekv@ifempty@B\@firstoftwo
  }
\long\def\ekvd@ifempty@gtwo#1#2{\ekv@ifempty@\ekv@ifempty@A}
%    \end{macrocode}
% \end{macro}
%
% \begin{macro}[internal]
%   {
%     \ekvd@assert@val,\ekvd@assert@val@,\ekvd@assert@noval,\ekvd@assert@noval@,
%     \ekvd@extract@args,\ekvd@extracted@args,\ekvd@one@arg@string
%   }
% Assert that a given key is defined as a value taking key or a |NoVal| key with
% the correct argument structure, respectively.
%    \begin{macrocode}
\protected\def\ekvd@assert@val#1%
  {%
    \ekvifdefined\ekvd@set{#1}%
      {\expandafter\ekvd@assert@val@\csname\ekv@name\ekvd@set{#1}\endcsname}%
      {%
        \ekvifdefinedNoVal\ekvd@set{#1}%
          \ekvd@err@add@val@on@noval
          {\ekvd@err@undefined@key{#1}}%
        \@gobble
      }%
  }
\protected\def\ekvd@assert@val@#1%
  {%
    \expandafter\ekvd@extract@args\meaning#1\ekvd@stop
    \unless\ifx\ekvd@extracted@args\ekvd@one@arg@string
      \ekvd@err@unsupported@arg
    \fi
    \@firstofone
  }%
\protected\def\ekvd@assert@noval#1%
  {%
    \ekvifdefinedNoVal\ekvd@set{#1}%
      {\expandafter\ekvd@assert@noval@\csname\ekv@name\ekvd@set{#1}N\endcsname}%
      {%
        \ekvifdefined\ekvd@set{#1}%
          \ekvd@err@add@noval@on@val
          {\ekvd@err@undefined@key{#1}}%
        \@gobble
      }%
  }
\protected\def\ekvd@assert@noval@#1%
  {%
    \expandafter\ekvd@extract@args\meaning#1\ekvd@stop
    \unless\ifx\ekvd@extracted@args\ekv@empty
      \ekvd@err@unsupported@arg
    \fi
    \@firstofone
  }
\protected\def\ekvd@extract@args#1%
  {%
    \protected\def\ekvd@extract@args##1#1##2->##3\ekvd@stop
      {\def\ekvd@extracted@args{##2}}%
  }
\expandafter\ekvd@extract@args\expandafter{\detokenize{macro:}}
\edef\ekvd@one@arg@string{\string#1}
%    \end{macrocode}
% \end{macro}
%
% \begin{macro}[internal]{\ekvd@assert@arg,\ekvd@assert@arg@msg,\ekvd@ifnoarg}
% There is no need to actually define |\ekvd@ifnoarg| here, as it will be set by
% either |\ekvd@arg| or |\ekvd@noarg|.
%    \begin{macrocode}
\def\ekvd@assert@arg{\ekvd@ifnoarg\ekvd@err@missing@definition}
\long\def\ekvd@assert@arg@msg#1%
  {%
    \ekvd@ifnoarg{\ekvd@err@missing@definition@msg{#1}}%
  }
%    \end{macrocode}
% \end{macro}
%
% \begin{macro}[internal]{\ekvd@assert@filledarg,\ekvd@ifnoarg@or@empty}
%    \begin{macrocode}
\long\def\ekvd@assert@filledarg#1%
  {%
    \ekvd@ifnoarg@or@empty{#1}\ekvd@err@missing@definition
  }
\long\def\ekvd@ifnoarg@or@empty#1%
  {%
    \ekvd@ifnoarg
      \@firstoftwo
      {\ekv@ifempty{#1}}%
  }
%    \end{macrocode}
% \end{macro}
%
% \begin{macro}[internal]
%   {
%     \ekvd@assert@not@long,\ekvd@assert@not@protected,\ekvd@assert@not@also
%     \ekvd@assert@not@long@also,\ekvd@assert@not@protected@also,
%     \ekvd@assert@new,\ekvd@assert@not@new
%   }
% Some key-types don't want to be |also|, |\long| or |\protected|, so we provide
% macros to test this and throw an error, this could be silently ignored but now
% users will learn to not use unnecessary stuff which slows the compilation
% down.
%    \begin{macrocode}
\def\ekvd@assert@not@long{\ifx\ekvd@long\long\ekvd@err@no@prefix{long}\fi}
\def\ekvd@assert@not@protected
  {\ifx\ekvd@prot\protected\ekvd@err@no@prefix{protected}\fi}
\def\ekvd@assert@not@also{\ekvd@ifalso{\ekvd@err@no@prefix{also}}{}}
\def\ekvd@assert@not@long@also
  {\ifx\ekvd@long\long\ekvd@err@no@prefix@also{long}\fi}
\def\ekvd@assert@not@protected@also
  {\ifx\ekvd@prot\protected\ekvd@err@no@prefix@also{protected}\fi}
\def\ekvd@assert@new#1#2%
  {\csname ekvifdefined#1\endcsname\ekvd@set{#2}{\ekvd@err@not@new}}
\def\ekvd@assert@not@new
  {\ifx\ekvd@ifnew\ekvd@assert@new\ekvd@err@no@prefix{new}\fi}
\def\ekvd@assert@new@for@name#1%
  {%
    \ifx\ekvd@ifnew\ekvd@assert@new
      \ekv@fi@firstoftwo
    \fi
    \@secondoftwo
      {\ekv@ifdefined{#1}\ekvd@err@not@new}%
      \@firstofone
  }
%    \end{macrocode}
% \end{macro}
%
% \begin{macro}[internal]
%   {
%     \ekvd@if@not@already@choice, \ekvd@if@not@already@choice@a,
%     \ekvd@if@not@already@choice@b
%   }
% It is bad to use |also| on a key that already contains a |choice|, as both
% choices would share the same valid values and thus lead to each callback being
% used twice. The following is a rudimentary test against this.
%    \begin{macrocode}
\protected\def\ekvd@if@not@already@choice#1%
  {%
    \expandafter\ekvd@if@not@already@choice@a
      \csname\ekv@name\ekvd@set{#1}\endcsname
      {}\ekvd@h@choice\ekvd@stop
  }
\protected\def\ekvd@if@not@already@choice@a
  {%
    \expandafter\ekvd@if@not@already@choice@b
  }
\long\protected\def\ekvd@if@not@already@choice@b#1\ekvd@h@choice#2\ekvd@stop
  {%
    \ekv@ifempty{#2}\@firstofone\@gobble
  }
%    \end{macrocode}
% \end{macro}
%
% \begin{macro}[internal]{\ekvd@ifspace,\ekvd@ifspace@}
% Yet another test which can be reduced to an if-empty, this time by gobbling
% everything up to the first space.
%    \begin{macrocode}
\long\def\ekvd@ifspace#1%
  {%
    \ekvd@ifspace@#1 \ekv@ifempty@B
      \ekv@ifempty@false\ekv@ifempty@A\ekv@ifempty@B\@firstoftwo
  }
\long\def\ekvd@ifspace@#1 % keep this space
  {%
    \ekv@ifempty@\ekv@ifempty@A
  }
%    \end{macrocode}
% \end{macro}
%
% \subsubsection{Messages}
%
% Most messages of \expkvd\ are not expandable, since they only appear during
% key-definition, which is not expandable anyway.
%
% \begin{macro}[internal]
%   {
%     \ekvd@errm,\ekvd@err@missing@definition,\ekvd@err@missing@definition@msg,
%     \ekvd@err@missing@type,\ekvd@err@undefined@prefix,\ekvd@err@undefined@key,
%     \ekvd@err@no@prefix,\ekvd@err@no@prefix@msg,\ekvd@err@no@prefix@also,
%     \ekvd@err@add@val@on@noval,\ekvd@err@add@noval@on@val,
%     \ekvd@err@unsupported@arg,\ekvd@err@not@new
%   }
% The non-expandable error messages are boring, so here they are:
%    \begin{macrocode}
\protected\def\ekvd@errm#1{\errmessage{expkv-def Error: #1}}
\protected\def\ekvd@err@missing@definition
  {\ekvd@errm{Missing definition for key `\ekvd@cur'}}
\protected\def\ekvd@err@missing@definition@msg#1%
  {\ekvd@errm{Missing definition for key `\ekv@unexpanded{#1}'}}
\protected\def\ekvd@err@missing@type
  {\ekvd@errm{Missing type prefix for key `\ekvd@cur'}}
\protected\def\ekvd@err@undefined@prefix#1%
  {%
    \ekvd@errm
      {%
        Undefined prefix `\ekv@unexpanded{#1}' found while processing
        `\ekvd@cur'%
      }%
  }
\protected\def\ekvd@err@undefined@key#1%
  {%
    \ekvd@errm
      {Undefined key `\ekv@unexpanded{#1}' found while processing `\ekvd@cur'}%
  }
\protected\def\ekvd@err@undefined@noval#1%
  {%
    \ekvd@errm
      {%
        Undefined noval key `\unexpanded{#1}' found while processing
        `\ekvd@cur'%
      }%
  }
\protected\def\ekvd@err@no@prefix#1%
  {\ekvd@errm{prefix `#1' not accepted in `\ekvd@cur'}}
\protected\def\ekvd@err@no@prefix@msg#1#2%
  {\ekvd@errm{prefix `#2' not accepted in `\ekv@unexpanded{#1}'}}
\protected\def\ekvd@err@no@prefix@also#1%
  {\ekvd@errm{`\ekvd@cur' not allowed with a `#1' key}}
\protected\def\ekvd@err@add@val@on@noval
  {\ekvd@errm{`\ekvd@cur' not allowed with a NoVal key}}
\protected\def\ekvd@err@add@noval@on@val
  {\ekvd@errm{`\ekvd@cur' not allowed with a value taking key}}
\protected\def\ekvd@err@unsupported@arg\fi\@firstofone#1%
  {%
    \fi
    \ekvd@errm
      {%
        Existing key-macro has the unsupported argument string
        `\ekvd@extracted@args' for key `\ekvd@cur'%
      }%
  }
\protected\def\ekvd@err@not@new
  {\ekvd@errm{The key for `\ekvd@cur' is already defined}}
\protected\long\def\ekvd@err@misused@unknown
  {\ekvd@errm{Misuse of the unknown type found while processing `\ekvd@cur'}}
%    \end{macrocode}
% \end{macro}
%
% \begin{macro}[internal]
%   {
%     \ekvd@err@choice@invalid,\ekvd@err@choice@invalid@,\ekvd@choice@name,
%     \ekvd@unknown@choice@name
%   }
% |\ekvd@err@choice@invalid| will have to use this mechanism to throw its
% message. Also we have to retrieve the name parts of the choice in an easy way,
% so we use parentheses of catcode 8 here, which should suffice in most cases to
% allow for a correct separation.
%    \begin{macrocode}
\def\ekvd@err@choice@invalid#1%
  {%
    \ekvd@err@choice@invalid@#1%
  }
\begingroup
\catcode40=8
\catcode41=8
\@firstofone{\endgroup
\def\ekvd@choice@name#1#2#3%
  {%
    ekvd#1(#2)\detokenize{#3}%
  }
\def\ekvd@unknown@choice@name#1#2%
  {%
    ekvd:u:#1(#2)%
  }
\def\ekvd@err@choice@invalid@ ekvd#1(#2)\detokenize#3%
  {%
    \ekv@ifdefined{\ekvd@unknown@choice@name{#1}{#2}}%
      {\csname\ekvd@unknown@choice@name{#1}{#2}\endcsname{#3}}%
      {\ekvd@err{invalid choice `#3' for `#2' in set `#1'}}%
  }
}
%    \end{macrocode}
% \end{macro}
%
% \begin{macro}[internal]{\ekvd@err}
% The expandable error messages use |\ekvd@err|, which is just like |\ekv@err|
% from \expkv. It uses a runaway argument to start the error message.
%    \begin{macrocode}
\ekv@exparg{\long\def\ekvd@err#1}{\ekverr{expkv-def}{#1}}
%    \end{macrocode}
% \end{macro}
%
% Now everything that's left is to reset the category code of |@|.
%    \begin{macrocode}
\catcode`\@=\ekvd@tmp
%    \end{macrocode}
%
% \gobbledocstriptag
%</tex>
%^^A=<<
%
%^^A=<<
%
% \end{implementation}^^A=<<
%
% \clearpage
% \PrintIndex
%
\endinput
%
^^A vim: ft=tex fdm=marker fmr=>>=,=<<

\ProvidesPackage{expkv-def}%
  [\ekvdDate\space v\ekvdVersion\space a key-defining frontend for expkv]
%    \end{macrocode}
% \gobbledocstriptag
%</pkg>
%^^A=<<
%^^A the ConTeXt module >>=
% \subsection{The \ConTeXt\ module}
% \gobbledocstriptag
%<*ctx>
%    \begin{macrocode}
\writestatus{loading}{ConTeXt User Module / expkv-def}
\usemodule[expkv]
\unprotect
\input expkv-def.tex
\writestatus{loading}
  {ConTeXt User Module / expkv-def / Version \ekvdVersion\space loaded}
\protect\endinput
%    \end{macrocode}
% \gobbledocstriptag
%</ctx>
%^^A=<<
%^^A main file >>=
% \subsection{The Generic Code}
% The rest of this implementation will be the generic code.
% \gobbledocstriptag
%<*tex>
%
% Load \expkv\ if the package didn't already do so -- since \expkv\ has
% safeguards against being loaded twice this does no harm and the overhead
% isn't that big. Also we reuse some of the internals of \expkv\ to save us from
% retyping them.
%    \begin{macrocode}
\input expkv
%    \end{macrocode}
%
% We make sure that \file{expkv-def.tex} is only input once:
%    \begin{macrocode}
\expandafter\ifx\csname ekvdVersion\endcsname\relax
\else
  \expandafter\endinput
\fi
%    \end{macrocode}
%
% \begin{macro}{\ekvdVersion,\ekvdDate}
% We're on our first input, so lets store the version and date in a macro.
%    \begin{macrocode}
\def\ekvdVersion{0.8c}
\def\ekvdDate{2021-09-20}
%    \end{macrocode}
% \end{macro}
%
% If the \LaTeX\ format is loaded we want to be a good file and report back who
% we are, for this the package will have defined |\ekvd@tmp| to use
% |\ProvidesFile|, else this will expand to a |\relax| and do no harm.
%    \begin{macrocode}
\csname ekvd@tmp\endcsname
%    \end{macrocode}
%
% Store the category code of |@| to later be able to reset it and change it to
% 11 for now.
%    \begin{macrocode}
\expandafter\chardef\csname ekvd@tmp\endcsname=\catcode`\@
\catcode`\@=11
%    \end{macrocode}
% |\ekvd@tmp| will be reused later to handle expansion during the key
% defining. But we don't need it to ever store information long-term after
% \expkvd\ was initialized.
%
% \begin{macro}[internal]{\ekvd@ifprimitive}
%    \begin{macrocode}
\protected\long\def\ekvd@ifprimitive#1%
  {%
    \begingroup
      \edef\ekvd@tmpa{\string #1}%
      \edef\ekvd@tmpb{\meaning#1}%
      \expandafter
    \endgroup
    \ifx\ekvd@tmpa\ekvd@tmpb
      \ekv@fi@firstoftwo
    \fi
    \@secondoftwo
  }
%    \end{macrocode}
% \end{macro}
%
% \begin{macro}[internal]
%   {\ekvd@long,\ekvd@prot,\ekvd@clear@prefixes,\ekvd@ifalso}
% \expkvd\ will use |\ekvd@long|, |\ekvd@prot|, and |\ekvd@ifalso| to store
% whether a key should be defined as |\long| or |\protected| or adds an action
% to an existing key, and we have to clear them for every new key. By default
% |long| and |protected| will just be empty, |ifalso| will be
% |\@secondoftwo|, and |ifnew| will just use its third argument.
%    \begin{macrocode}
\protected\def\ekvd@clear@prefixes
  {%
    \let\ekvd@long\ekv@empty
    \let\ekvd@prot\ekv@empty
    \let\ekvd@ifalso\@secondoftwo
    \long\def\ekvd@ifnew##1##2##3{##3}%
  }
\ekvd@clear@prefixes
%    \end{macrocode}
% \end{macro}
%
% \begin{macro}{\ekvdefinekeys}
% This is the one front-facing macro which provides the interface to define
% keys. It's using |\ekvparse| to handle the \kv\ list, the interpretation will
% be done by |\ekvd@noarg| and |\ekvd@|. The \set\ for which the keys should be
% defined is stored in |\ekvd@set|.
%    \begin{macrocode}
\protected\def\ekvdefinekeys#1%
  {%
    \def\ekvd@set{#1}%
    \ekvparse\ekvd@noarg\ekvd@arg
  }
%    \end{macrocode}
% \end{macro}
%
% \begin{macro}[internal]{\ekvd@noarg,\ekvd@arg,\ekvd@handle}
% |\ekvd@noarg| and |\ekvd@arg| store whether there was a value in the \kv\
% pair. |\ekvd@handle| has to test whether there is a space inside the key and
% if so calls the prefix grabbing routine, else we throw an error and ignore the
% key.
%    \begin{macrocode}
\protected\long\def\ekvd@noarg#1%
  {%
    \let\ekvd@ifnoarg\@firstoftwo
    \expandafter\ekvd@handle\detokenize{#1}\ekvd@stop{}%
  }
\protected\long\def\ekvd@arg#1%
  {%
    \let\ekvd@ifnoarg\@secondoftwo
    \expandafter\ekvd@handle\detokenize{#1}\ekvd@stop
  }
\protected\long\def\ekvd@handle#1\ekvd@stop#2%
  {%
    \ekvd@clear@prefixes
    \def\ekvd@cur{#1}%
    \ekvd@ifspace{#1}%
      {\ekvd@prefix\ekv@mark#1\ekv@stop{#2}}%
      \ekvd@err@missing@type
  }
%    \end{macrocode}
% \end{macro}
%
% \begin{macro}[internal]{\ekvd@prefix,\ekvd@prefix@}
% \expkvd\ separates prefixes into two groups, the first being prefixes in the
% \TeX\ sense (|long| and |protected|) which use |@p@| in their name, the other
% being key-types (|code|, |int|, \emph{etc.}) which use |@t@| instead.
% |\ekvd@prefix| splits at the first space and checks whether its a |@p@|
% or |@t@| type prefix. If it is neither throw an error and gobble the
% definition (the value).
%    \begin{macrocode}
\protected\def\ekvd@prefix#1 {\ekv@strip{#1}\ekvd@prefix@\ekv@mark}
\protected\def\ekvd@prefix@#1#2\ekv@stop
  {%
    \ekv@ifdefined{ekvd@t@#1}%
      {\ekv@strip{#2}{\csname ekvd@t@#1\endcsname}}%
      {%
        \ekv@ifdefined{ekvd@p@#1}%
          {\csname ekvd@p@#1\endcsname\ekvd@prefix@after@p{#2}}%
          {\ekvd@err@undefined@prefix{#1}\@gobble}%
      }%
  }
%    \end{macrocode}
% \end{macro}
%
% \begin{macro}[internal]{\ekvd@prefix@after@p}
% The |@p@| type prefixes are all just modifying a following |@t@| type, so they
% will need to search for another prefix. This is true for all of them, so we
% use a macro to handle this. It'll throw an error if there is no other prefix.
%    \begin{macrocode}
\protected\def\ekvd@prefix@after@p#1%
  {%
    \ekvd@ifspace{#1}%
      {\ekvd@prefix#1\ekv@stop}%
      {\ekvd@err@missing@type\@gobble}%
  }
%    \end{macrocode}
% \end{macro}
%
% \begin{macro}[internal]
%   {\ekvd@p@long,\ekvd@p@protected,\ekvd@p@protect,\ekvd@p@also,\ekvd@p@new}
% Define the |@p@| type prefixes, they all just store some information in a
% temporary macro.
%    \begin{macrocode}
\protected\def\ekvd@p@long{\let\ekvd@long\long}
\protected\def\ekvd@p@protected{\let\ekvd@prot\protected}
\let\ekvd@p@protect\ekvd@p@protected
\protected\def\ekvd@p@also{\let\ekvd@ifalso\@firstoftwo}
\protected\def\ekvd@p@new{\let\ekvd@ifnew\ekvd@assert@new}
%    \end{macrocode}
% \end{macro}
%
% \subsubsection{Key Types}
%
% \begin{macro}[internal]{\ekvd@type@set,\ekvd@t@set}
% The |set| type is quite straight forward, just define a |NoVal| key to call
% |\ekvchangeset|.
%    \begin{macrocode}
\protected\def\ekvd@type@set#1#2%
  {%
    \ekvd@assert@not@long
    \ekvd@assert@not@protected
    \ekvd@ifnew{NoVal}{#1}%
      {%
        \ekv@ifempty{#2}%
          {\ekvd@err@missing@definition}%
          {%
            \ekvd@ifalso
              {%
                \ekv@expargtwice{\ekvd@add@noval{#1}}%
                  {\ekvchangeset{#2}}%
                  \ekvd@assert@not@protected@also
              }%
              {\ekv@expargtwice{\ekvdefNoVal\ekvd@set{#1}}{\ekvchangeset{#2}}}%
          }%
      }%
  }
\protected\def\ekvd@t@set#1#2%
  {%
    \ekvd@ifnoarg
      {\ekvd@type@set{#1}{#1}}%
      {\ekvd@type@set{#1}{#2}}%
  }
%    \end{macrocode}
% \end{macro}
%
% \begin{macro}[internal]{\ekvd@type@noval,\ekvd@t@noval,\ekvd@t@enoval}
% Another pretty simple type, |noval| just needs to assert that there is a
% definition and that |long| wasn't specified.
% There are types where the difference in the variants is so small, that we
% define a common handler for them, those common handlers are named with
% |@type@|. |noval| and |enoval| are so similar that we can use such a |@type@|
% macro, even if we could've done |noval| in a slightly faster way without it.
%    \begin{macrocode}
\protected\long\def\ekvd@type@noval#1#2#3%
  {%
    \ekvd@ifnew{NoVal}{#2}%
      {%
        \ekvd@assert@arg
          {%
            \ekvd@assert@not@long
            \ekvd@prot#1\ekvd@tmp{#3}%
            \ekvd@ifalso
              {\ekv@exparg{\ekvd@add@noval{#2}}\ekvd@tmp{}}%
              {\ekvletNoVal\ekvd@set{#2}\ekvd@tmp}%
          }%
      }%
  }
\protected\def\ekvd@t@noval{\ekvd@type@noval\def}
\protected\def\ekvd@t@enoval{\ekvd@type@noval\edef}
%    \end{macrocode}
% \end{macro}
%
% \begin{macro}[internal]{\ekvd@type@code,\ekvd@t@code,\ekvd@t@ecode}
% |code| is simple as well, |ecode| has to use |\edef| on a temporary macro,
% since \expkv\ doesn't provide an |\ekvedef|.
%    \begin{macrocode}
\protected\long\def\ekvd@type@code#1#2#3%
  {%
    \ekvd@ifnew{}{#2}%
      {%
        \ekvd@assert@arg
          {%
            \ekvd@prot\ekvd@long#1\ekvd@tmp##1{#3}%
            \ekvd@ifalso
              {\ekv@exparg{\ekvd@add@val{#2}}{\ekvd@tmp{##1}}{}}%
              {\ekvlet\ekvd@set{#2}\ekvd@tmp}%
          }%
      }%
  }
\protected\def\ekvd@t@code{\ekvd@type@code\def}
\protected\def\ekvd@t@ecode{\ekvd@type@code\edef}
%    \end{macrocode}
% \end{macro}
%
% \begin{macro}[internal]
%   {
%     \ekvd@type@default,\ekvd@t@default,\ekvd@t@qdefault,\ekvd@t@odefault,
%     \ekvd@t@fdefault
%   }
% |\ekvd@type@default| asserts there was an argument, also the key for which one
% wants to set a default has to be already defined (this is not so important for
% |default|, but |qdefault| requires is). If everything is good, |\edef| a
% temporary macro that expands |\ekvd@set| and the |\csname| for the key, and in
% the case of |qdefault| does the first expansion step of the key-macro.
%    \begin{macrocode}
\protected\long\def\ekvd@type@default#1#2#3#4%
  {%
    \ekvd@assert@arg
      {%
        \ekvifdefined\ekvd@set{#3}%
          {%
            \ekvd@assert@not@new
            \ekvd@assert@not@long
            \ekvd@prot\edef\ekvd@tmp
              {%
                \ekv@unexpanded\expandafter#1%
                  {#2\csname\ekv@name\ekvd@set{#3}\endcsname{#4}}%
              }%
            \ekvd@ifalso
              {\ekv@exparg{\ekvd@add@noval{#3}}\ekvd@tmp{}}%
              {\ekvletNoVal\ekvd@set{#3}\ekvd@tmp}%
          }%
          {\ekvd@err@undefined@key{#3}}%
      }%
  }
\protected\def\ekvd@t@default{\ekvd@type@default{}{}}
\protected\def\ekvd@t@qdefault{\ekvd@type@default{\expandafter\expandafter}{}}
\let\ekvd@t@odefault\ekvd@t@qdefault
\protected\def\ekvd@t@fdefault{\ekvd@type@default{}{\romannumeral`\^^@}}
%    \end{macrocode}
% \end{macro}
%
% \begin{macro}[internal]{\ekvd@t@edefault}
% |edefault| is too different from |default| and |qdefault| to reuse the
% |@type@| macro, as it doesn't need |\unexpanded| inside of |\edef|.
%    \begin{macrocode}
\protected\long\def\ekvd@t@edefault#1#2%
  {%
    \ekvd@assert@arg
      {%
        \ekvifdefined\ekvd@set{#1}%
          {%
            \ekvd@assert@not@new
            \ekvd@assert@not@long
            \ekvd@prot\edef\ekvd@tmp
              {\csname\ekv@name\ekvd@set{#1}\endcsname{#2}}%
            \ekvd@ifalso
              {\ekv@exparg{\ekvd@add@noval{#1}}\ekvd@tmp{}}%
              {\ekvletNoVal\ekvd@set{#1}\ekvd@tmp}%
          }%
          {\ekvd@err@undefined@key{#1}}%
      }%
  }
%    \end{macrocode}
% \end{macro}
%
% \begin{macro}[internal]
%   {\ekvd@t@initial,\ekvd@t@oinitial,\ekvd@t@finitial,\ekvd@t@einitial}
%    \begin{macrocode}
\long\def\ekvd@type@initial#1#2#3#4%
  {%
    \ekvd@assert@not@new
    \ekvd@assert@not@also
    \ekvd@assert@not@long
    \ekvd@assert@not@protected
    \ekvd@ifnoarg
      {%
        \ekvifdefinedNoVal\ekvd@set{#3}%
          {\csname\ekv@name\ekvd@set{#3}N\endcsname}%
          {\ekvd@err@undefined@noval{#3}}%
      }%
      {%
        \ekvifdefined\ekvd@set{#3}%
          {%
            #1{#2#4}%
            \csname\ekv@name\ekvd@set{#3}\expandafter\endcsname\expandafter
              {\ekvd@tmp}%
          }%
          {\ekvd@err@undefined@key{#3}}%
      }%
  }
\def\ekvd@t@initial{\ekvd@type@initial{\def\ekvd@tmp}{}}
\def\ekvd@t@oinitial{\ekvd@type@initial{\ekv@exparg{\def\ekvd@tmp}}{}}
\def\ekvd@t@einitial{\ekvd@type@initial{\edef\ekvd@tmp}{}}
\def\ekvd@t@finitial
  {\ekvd@type@initial{\ekv@exparg{\def\ekvd@tmp}}{\romannumeral`\^^@}}
%    \end{macrocode}
% \end{macro}
%
% \begin{macro}[internal]
%   {
%     \ekvd@type@bool,\ekvd@t@bool,\ekvd@t@gbool,\ekvd@t@boolTF,\ekvd@t@gboolTF,
%     \ekvd@t@invbool,\ekvd@t@ginvbool,\ekvd@t@invboolTF,\ekvd@t@ginvboolTF,
%   }
% The boolean types are a quicker version of a |choice| that accept |true| and
% |false|, and set up the |NoVal| action to be identical to \texttt{\key=true}.
% The |true| and |false| actions are always just |\let|ting the macro in |#7| to
% some other macro (\emph{e.g.}, \cs[no-index]{iftrue}).
%    \begin{macrocode}
\protected\def\ekvd@type@bool#1#2#3#4#5%
  {%
    \ekvd@ifnew{}{#4}%
      {%
        \ekvd@ifnew{NoVal}{#4}%
          {%
            \ekvd@assert@filledarg{#5}%
              {%
                \ekvd@newlet#5#3%
                \ekvd@type@choice{#4}%
                \protected\ekvdefNoVal\ekvd@set{#4}{#1\let#5#2}%
                \protected\expandafter\def
                  \csname\ekvd@choice@name\ekvd@set{#4}{true}\endcsname
                  {#1\let#5#2}%
                \protected\expandafter\def
                  \csname\ekvd@choice@name\ekvd@set{#4}{false}\endcsname
                  {#1\let#5#3}%
              }%
          }%
      }%
  }
\protected\def\ekvd@t@bool{\ekvd@type@bool{}\iftrue\iffalse}
\protected\def\ekvd@t@gbool{\ekvd@type@bool\global\iftrue\iffalse}
\protected\def\ekvd@t@boolTF{\ekvd@type@bool{}\@firstoftwo\@secondoftwo}
\protected\def\ekvd@t@gboolTF{\ekvd@type@bool\global\@firstoftwo\@secondoftwo}
\protected\def\ekvd@t@invbool{\ekvd@type@bool{}\iffalse\iftrue}
\protected\def\ekvd@t@ginvbool{\ekvd@type@bool\global\iffalse\iftrue}
\protected\def\ekvd@t@invboolTF{\ekvd@type@bool{}\@secondoftwo\@firstoftwo}
\protected\def\ekvd@t@ginvboolTF
  {\ekvd@type@bool\global\@secondoftwo\@firstoftwo}
%    \end{macrocode}
% \end{macro}
%
% \begin{macro}[internal]
%   {
%     \ekvd@type@boolpair,
%     \ekvd@t@boolpair,\ekvd@t@gboolpair,\ekvd@t@boolpairTF,\ekvd@t@gboolpairTF
%   }
% The boolean pair types are essentially the same as the boolean types, but set
% two macros instead of one.
%    \begin{macrocode}
\protected\def\ekvd@type@boolpair#1#2#3#4#5#6%
  {%
    \ekvd@ifnew{}{#4}%
      {%
        \ekvd@ifnew{NoVal}{#4}%
          {%
            \ekvd@newlet#5#3%
            \ekvd@newlet#6#2%
            \ekvd@type@choice{#4}%
            \protected\ekvdefNoVal\ekvd@set{#4}{#1\let#5#2#1\let#6#3}%
            \protected\expandafter\def
              \csname\ekvd@choice@name\ekvd@set{#4}{true}\endcsname
              {#1\let#5#2#1\let#6#3}%
            \protected\expandafter\def
              \csname\ekvd@choice@name\ekvd@set{#4}{false}\endcsname
              {#1\let#5#3#1\let#6#2}%
          }%
      }%
  }
\protected\def\ekvd@t@boolpair#1#2%
  {\ekvd@assert@twoargs{#2}{\ekvd@type@boolpair{}\iftrue\iffalse{#1}#2}}
\protected\def\ekvd@t@gboolpair#1#2%
  {\ekvd@assert@twoargs{#2}{\ekvd@type@boolpair\global\iftrue\iffalse{#1}#2}}
\protected\def\ekvd@t@boolpairTF#1#2%
  {%
    \ekvd@assert@twoargs{#2}%
      {\ekvd@type@boolpair{}\@firstoftwo\@secondoftwo{#1}#2}%
  }
\protected\def\ekvd@t@gboolpairTF#1#2%
  {%
    \ekvd@assert@twoargs{#2}%
      {\ekvd@type@boolpair\global\@firstoftwo\@secondoftwo{#1}#2}%
  }
%    \end{macrocode}
% \end{macro}
%
% \begin{macro}[internal]
%   {
%     \ekvd@type@data,\ekvd@t@data,\ekvd@t@gdata,\ekvd@t@dataT,\ekvd@t@gdataT
%   }
%    \begin{macrocode}
\protected\def\ekvd@type@data#1#2#3#4#5#6%
  {%
    \ekvd@ifnew{}{#5}%
      {%
        \ekvd@assert@filledarg{#6}%
          {%
            \ekvd@newlet#6#1%
            \ekvd@ifalso
              {%
                \let\ekvd@prot\protected
                \ekvd@add@val{#5}{\long#2#6####1#3{####1{#4}}}{}%
              }%
              {%
                \protected\ekvd@long\ekvdef\ekvd@set{#5}%
                  {\long#2#6####1#3{####1{#4}}}%
              }%
          }%
      }%
  }
\protected\def\ekvd@t@data
  {\ekvd@type@data\@secondoftwo\edef{####2}{\ekv@unexpanded{##1}}}
\protected\def\ekvd@t@edata{\ekvd@type@data\@secondoftwo\edef{####2}{##1}}
\protected\def\ekvd@t@gdata
  {\ekvd@type@data\@secondoftwo\xdef{####2}{\ekv@unexpanded{##1}}}
\protected\def\ekvd@t@xdata{\ekvd@type@data\@secondoftwo\xdef{####2}{##1}}
\protected\def\ekvd@t@dataT
  {\ekvd@type@data\@gobble\edef{}{\ekv@unexpanded{##1}}}
\protected\def\ekvd@t@edataT{\ekvd@type@data\@gobble\edef{}{##1}}
\protected\def\ekvd@t@gdataT
  {\ekvd@type@data\@gobble\xdef{}{\ekv@unexpanded{##1}}}
\protected\def\ekvd@t@xdataT{\ekvd@type@data\@gobble\xdef{}{##1}}
%    \end{macrocode}
% \end{macro}
%
% \begin{macro}[internal]{\ekvd@type@box,\ekvd@t@box,\ekvd@t@gbox}
% Set up our boxes. Though we're a generic package we want to be colour safe, so
% we put an additional grouping level inside the box contents, for the case that
% someone uses \pkg{color}. |\ekvd@newreg| is a small wrapper which tests
% whether the first argument is defined and if not does
% |\csname new#2\endcsname#1|.
%    \begin{macrocode}
\protected\def\ekvd@type@box#1#2#3%
  {%
    \ekvd@ifnew{}{#2}%
      {%
        \ekvd@assert@filledarg{#3}%
          {%
            \ekvd@newreg#3{box}%
            \ekvd@ifalso
              {%
                \let\ekvd@prot\protected
                \ekvd@add@val{#2}{#1\setbox#3=\hbox{\begingroup##1\endgroup}}{}%
              }%
              {%
                \protected\ekvd@long\ekvdef\ekvd@set{#2}%
                  {#1\setbox#3=\hbox{\begingroup##1\endgroup}}%
              }%
          }%
      }%
  }
\protected\def\ekvd@t@box{\ekvd@type@box{}}
\protected\def\ekvd@t@gbox{\ekvd@type@box\global}
%    \end{macrocode}
% \end{macro}
%
% \begin{macro}[internal]{\ekvd@type@toks,\ekvd@t@toks,\ekvd@t@gtoks}
% Similar to |box|, but set the |toks|.
%    \begin{macrocode}
\protected\def\ekvd@type@toks#1#2#3%
  {%
    \ekvd@ifnew{}{#2}%
      {%
        \ekvd@assert@filledarg{#3}%
          {%
            \ekvd@newreg#3{toks}%
            \ekvd@ifalso
              {%
                \let\ekvd@prot\protected
                \ekvd@add@val{#2}{#1#3={##1}}{}%
              }%
              {\protected\ekvd@long\ekvdef\ekvd@set{#2}{#1#3={##1}}}%
          }%
      }%
  }
\protected\def\ekvd@t@toks{\ekvd@type@toks{}}
\protected\def\ekvd@t@gtoks{\ekvd@type@toks\global}
%    \end{macrocode}
% \end{macro}
%
% \begin{macro}[internal]{\ekvd@type@preapptoks,\ekvd@t@apptoks,\ekvd@t@gapptoks}
% Just like |toks|, but expand the current contents of the |toks| register to
% append the new contents.
%    \begin{macrocode}
\ekvd@ifprimitive\toksapp
  {%
    \protected\def\ekvd@type@preapptoks#1#2#3%
      {%
        \ekvd@ifnew{}{#2}%
          {%
            \ekvd@assert@filledarg{#3}%
              {%
                \ekvd@newreg#3{toks}%
                \ekvd@ifalso
                  {%
                    \let\ekvd@prot\protected
                    \ekvd@add@val{#2}{#1#3{##1}}{}%
                  }%
                  {\protected\ekvd@long\ekvdef\ekvd@set{#2}{#1#3{##1}}}%
              }%
          }%
      }
    \protected\def\ekvd@t@apptoks{\ekvd@type@preapptoks\toksapp}
    \protected\def\ekvd@t@gapptoks{\ekvd@type@preapptoks\gtoksapp}
    \protected\def\ekvd@t@pretoks{\ekvd@type@preapptoks\tokspre}
    \protected\def\ekvd@t@gpretoks{\ekvd@type@preapptoks\gtokspre}
  }
  {%
    \protected\def\ekvd@type@apptoks#1#2#3%
      {%
        \ekvd@ifnew{}{#2}%
          {%
            \ekvd@assert@filledarg{#3}%
              {%
                \ekvd@newreg#3{toks}%
                \ekvd@ifalso
                  {%
                    \let\ekvd@prot\protected
                    \ekvd@add@val{#2}{#1#3=\expandafter{\the#3##1}}{}%
                  }%
                  {%
                    \protected\ekvd@long\ekvdef\ekvd@set{#2}%
                      {#1#3=\expandafter{\the#3##1}}%
                  }%
              }%
          }%
      }
    \protected\def\ekvd@t@apptoks{\ekvd@type@apptoks{}}
    \protected\def\ekvd@t@gapptoks{\ekvd@type@apptoks\global}
    \newtoks\ekvd@toks
    \protected\def\ekvd@type@pretoks#1#2#3%
      {%
        \ekvd@ifnew{}{#2}%
          {%
            \ekvd@assert@filledarg{#3}%
              {%
                \ekvd@newreg#3{toks}%
                \ekvd@ifalso
                  {%
                    \let\ekvd@prot\protected
                    \ekvd@add@val{#2}%
                      {%
                        \ekvd@toks={##1}%
                        #1#3=\expandafter{\the\expandafter\ekvd@toks\the#3}%
                      }%
                      {}%
                  }%
                  {%
                    \protected\ekvd@long\ekvdef\ekvd@set{#2}%
                      {%
                        \ekvd@toks={##1}%
                        #1#3=\expandafter{\the\expandafter\ekvd@toks\the#3}%
                      }%
                  }%
              }%
          }%
      }
    \protected\def\ekvd@t@pretoks{\ekvd@type@pretoks{}}
    \protected\def\ekvd@t@gpretoks{\ekvd@type@pretoks\global}
  }
%    \end{macrocode}
% \end{macro}
%
% \begin{macro}[internal]
%   {
%     \ekvd@type@reg,\ekvd@t@int,\ekvd@t@eint,\ekvd@t@gint,\ekvd@t@xint,
%     \ekvd@t@dimen,\ekvd@t@edimen,\ekvd@t@gdimen,\ekvd@t@xdimen,
%     \ekvd@t@skip,\ekvd@t@eskip,\ekvd@t@gskip,\ekvd@t@xskip
%   }
% The |\ekvd@type@reg| can handle all the types for which the assignment will
% just be \texttt{\meta{register}=\meta{value}}.
%    \begin{macrocode}
\protected\def\ekvd@type@reg#1#2#3#4#5#6%
  {%
    \ekvd@ifnew{}{#5}%
      {%
        \ekvd@assert@filledarg{#6}%
          {%
            \ekvd@newreg#6{#1}%
            \ekvd@ifalso
              {%
                \let\evkd@prot\protected
                \ekvd@add@val{#5}{#2#6=#3##1#4\relax}{}%
              }%
              {\protected\ekvd@long\ekvdef\ekvd@set{#5}{#2#6=#3##1#4\relax}}%
          }%
      }%
  }
\protected\def\ekvd@t@int{\ekvd@type@reg{count}{}{}{}}
\protected\def\ekvd@t@eint{\ekvd@type@reg{count}{}\numexpr\relax}
\protected\def\ekvd@t@gint{\ekvd@type@reg{count}\global{}{}}
\protected\def\ekvd@t@xint{\ekvd@type@reg{count}\global\numexpr\relax}
\protected\def\ekvd@t@dimen{\ekvd@type@reg{dimen}{}{}{}}
\protected\def\ekvd@t@edimen{\ekvd@type@reg{dimen}{}\dimexpr\relax}
\protected\def\ekvd@t@gdimen{\ekvd@type@reg{dimen}\global{}{}}
\protected\def\ekvd@t@xdimen{\ekvd@type@reg{dimen}\global\dimexpr\relax}
\protected\def\ekvd@t@skip{\ekvd@type@reg{skip}{}{}{}}
\protected\def\ekvd@t@eskip{\ekvd@type@reg{skip}{}\glueexpr\relax}
\protected\def\ekvd@t@gskip{\ekvd@type@reg{skip}\global{}{}}
\protected\def\ekvd@t@xskip{\ekvd@type@reg{skip}\global\glueexpr\relax}
%    \end{macrocode}
% \end{macro}
%
% \begin{macro}[internal]{\ekvd@type@store,\ekvd@t@store,\ekvd@t@gstore}
% The none-expanding |store| types use an |\edef| or |\xdef| and |\unexpanded|
% to be able to also store |#| easily.
%    \begin{macrocode}
\protected\def\ekvd@type@store#1#2#3#4%
  {%
    \ekvd@ifnew{}{#3}%
      {%
        \ekvd@assert@filledarg{#4}%
          {%
            \ekvd@newlet#4\ekv@empty
            \ekvd@ifalso
              {%
                \let\ekvd@prot\protected
                \ekvd@add@val{#3}{#1#4{#2}}{}%
              }%
              {\protected\ekvd@long\ekvdef\ekvd@set{#3}{#1#4{#2}}}%
          }%
      }%
  }
\protected\def\ekvd@t@store{\ekvd@type@store\edef{\ekv@unexpanded{##1}}}
\protected\def\ekvd@t@gstore{\ekvd@type@store\xdef{\ekv@unexpanded{##1}}}
\protected\def\ekvd@t@estore{\ekvd@type@store\edef{##1}}
\protected\def\ekvd@t@xstore{\ekvd@type@store\xdef{##1}}
%    \end{macrocode}
% \end{macro}
%
% \begin{macro}[internal]
%   {
%     \ekvd@type@meta,\ekvd@type@meta@a,\ekvd@type@meta@b,\ekvd@type@meta@c,
%     \ekvd@t@meta,\ekvd@t@nmeta
%   }
% |meta| sets up things such that another instance of |\ekvset| will be run on
% the argument, with the same \set.
%    \begin{macrocode}
\protected\long\def\ekvd@type@meta#1#2#3#4#5#6#7%
  {%
    \ekvd@ifnew{#1}{#6}%
      {%
        \ekvd@assert@filledarg{#7}%
          {%
            \edef\ekvd@tmp{\ekvd@set}%
            \expandafter\ekvd@type@meta@a\expandafter{\ekvd@tmp}{#7}{#2}%
            \ekvd@ifalso
              {\ekv@exparg{#3{#6}}{\ekvd@tmp#4}{#5}}%
              {\csname ekvlet#1\endcsname\ekvd@set{#6}\ekvd@tmp}%
          }%
      }%
  }
\protected\long\def\ekvd@type@meta@a#1#2%
  {%
    \expandafter\ekvd@type@meta@b\expandafter{\ekvset{#1}{#2}}%
  }
\protected\def\ekvd@type@meta@b
  {%
    \expandafter\ekvd@type@meta@c\expandafter
  }
\protected\long\def\ekvd@type@meta@c#1#2%
  {%
    \ekvd@prot\ekvd@long\def\ekvd@tmp#2{#1}%
  }
\protected\def\ekvd@t@meta{\ekvd@type@meta{}{##1}\ekvd@add@val{{##1}}{}}
\protected\def\ekvd@t@nmeta
  {%
    \ekvd@assert@not@long
    \ekvd@type@meta{NoVal}{}\ekvd@add@noval{}\ekvd@assert@not@long@also
  }
%    \end{macrocode}
% \end{macro}
%
% \begin{macro}[internal]
%   {\ekvd@type@smeta,\ekvd@type@smeta@,\ekvd@t@smeta,\ekvd@t@snmeta}
% |smeta| is pretty similar to |meta|, but needs two arguments inside of \val,
% such that the first is the \set\ for which the sub-|\ekvset| and the second is
% the \kv\ list.
%    \begin{macrocode}
\protected\long\def\ekvd@type@smeta#1#2#3#4#5#6#7%
  {%
    \ekvd@ifnew{#1}{#6}%
      {%
        \ekvd@assert@twoargs{#7}%
          {%
            \ekvd@type@meta@a#7{#2}%
            \ekvd@ifalso
              {\ekv@exparg{#3{#6}}{\ekvd@tmp#4}{#5}}%
              {\csname ekvlet#1\endcsname\ekvd@set{#6}\ekvd@tmp}%
          }%
      }%
  }
\protected\def\ekvd@t@smeta{\ekvd@type@smeta{}{##1}\ekvd@add@val{{##1}}{}}
\protected\def\ekvd@t@snmeta
  {%
    \ekvd@assert@not@long
    \ekvd@type@smeta{NoVal}{}\ekvd@add@noval{}\ekvd@assert@not@long@also
  }
%    \end{macrocode}
% \end{macro}
%
% \begin{macro}[internal]
%   {
%     \ekvd@type@choice,
%     \ekvd@populate@choice,\ekvd@populate@choice@,\ekvd@populate@choice@noarg,
%     \ekvd@choice@prefix,\ekvd@choice@prefix@,\ekvd@choice@prefix@done,
%     \ekvd@choice@p@protected,\ekvd@choice@p@protect,
%     \ekvd@choice@p@long,\ekvd@choice@p@long@,
%     \ekvd@t@choice
%   }
% The |choice| type is by far the most complex type, as we have to run a
% sub-parser on the choice-definition list, which should support the |@p@| type
% prefixes as well (but |long| will always throw an error, as they are not
% allowed to be long). |\ekvd@type@choice| will just define the choice-key, the
% handling of the choices definition will be done by |\ekvd@populate@choice|.
%    \begin{macrocode}
\protected\def\ekvd@type@choice#1%
  {%
    \ekvd@assert@not@long
    \ekv@exparg{\ekvd@prot\def\ekvd@tmp##1}%
      {%
        \expandafter\ekvd@h@choice\expandafter
          {\ekvd@choice@name\ekvd@set{#1}{##1}}%
      }%
    \ekvd@ifalso
      {%
        \ekvd@assert@val{#1}%
          {%
            \ekvd@if@not@already@choice{#1}%
              {%
                \ekv@exparg
                  {%
                    \expandafter\ekvd@add@aux
                      \csname\ekv@name\ekvd@set{#1}\endcsname{{##1}}{#1}%
                  }%
                  {\ekvd@tmp{##1}}%
                  {\ekvd@long\ekvdef}\ekvd@assert@not@long@also
              }%
          }%
      }%
      {\ekvlet\ekvd@set{#1}\ekvd@tmp}%
  }
%    \end{macrocode}
% |\ekvd@populate@choice| just uses |\ekvparse| and then gives control to
% |\ekvd@populate@choice@noarg|, which throws an error, and
% |\ekvd@populate@choice@|.
%    \begin{macrocode}
\protected\def\ekvd@populate@choice
  {%
    \ekvparse\ekvd@populate@choice@noarg\ekvd@populate@choice@
  }
\protected\long\def\ekvd@populate@choice@noarg#1%
  {%
    \expandafter\ekvd@err@missing@definition@msg\expandafter{\ekvd@cur : #1}%
  }
%    \end{macrocode}
% |\ekvd@populate@choice@| runs the prefix-test, if there is none we can
% directly define the choice, for that |\ekvd@set@choice| will expand to the
% current choice-key's name, which will have been defined by |\ekvd@t@choice|.
% If there is a prefix run the prefix grabbing routine, which was altered for
% |@type@choice|.
%    \begin{macrocode}
\protected\long\def\ekvd@populate@choice@#1#2%
  {%
    \ekvd@clear@prefixes
    \ekvd@ifspace{#1}%
      {\ekvd@choice@prefix{\ekv@mark#1}\ekv@mark#1\ekv@stop}%
      {%
        \expandafter\edef
          \csname\ekvd@choice@name\ekvd@set\ekvd@set@choice{#1}\endcsname
      }%
      {\unexpanded{#2}}%
  }
\protected\def\ekvd@choice@prefix#1#2
  {%
    \ekv@strip{#2}{\ekvd@choice@prefix@{#1}}\ekv@mark
  }
\protected\def\ekvd@choice@prefix@#1#2#3\ekv@stop
  {%
    \ekv@ifdefined{ekvd@choice@p@#2}%
      {%
        \csname ekvd@choice@p@#2\endcsname
        \ekvd@ifspace{#3}%
          {\ekvd@choice@prefix{#3}#3\ekv@stop}%
          {\ekvd@choice@prefix@done{#3}}%
      }%
      {\ekvd@choice@prefix@done{#1}}%
  }
\protected\def\ekvd@choice@prefix@done#1%
  {%
    \ekvd@prot\expandafter\edef
      \csname
        \ekv@strip{#1}{\ekvd@choice@name\ekvd@set\ekvd@set@choice}%
      \endcsname
  }
\protected\def\ekvd@choice@p@protected{\let\ekvd@prot\protected}
\let\ekvd@choice@p@protect\ekvd@choice@p@protected
\protected\def\ekvd@choice@invalid@p#1\ekvd@ifspace#2%
  {%
    \expandafter\ekvd@choice@invalid@p@\expandafter{\ekv@gobble@mark#2}{#1}%
    \ekvd@ifspace{#2}%
  }
\protected\def\ekvd@choice@invalid@p@#1#2%
  {%
    \expandafter\ekvd@err@no@prefix@msg\expandafter{\ekvd@cur : #2 #1}{#2}%
  }
\protected\def\ekvd@choice@p@long{\ekvd@choice@invalid@p{long}}%
\protected\def\ekvd@choice@p@also{\ekvd@choice@invalid@p{also}}%
\protected\def\ekvd@choice@p@new{\ekvd@choice@invalid@p{new}}%
%    \end{macrocode}
% Finally we're able to set up the |@t@choice| macro, which has to store the
% current choice-key's name, define the key, and parse the available choices.
%    \begin{macrocode}
\protected\long\def\ekvd@t@choice#1#2%
  {%
    \ekvd@ifnew{}{#1}%
      {%
        \ekvd@assert@arg
          {%
            \ekvd@type@choice{#1}%
            \def\ekvd@set@choice{#1}%
            \ekvd@populate@choice{#2}%
          }%
      }%
  }
%    \end{macrocode}
% \end{macro}
%
% \begin{macro}[internal]{\ekvd@t@choice-store,\ekvd@t@choice-enum}
%   These two types define a special kind of |choice| key and are quite similar,
%   the only difference is what the different choices do (hence they use a
%   shared initialisation which differs in the chosen |populate| step).
%    \begin{macrocode}
\protected\long\expandafter\def\csname ekvd@t@choice-store\endcsname
  {\ekvd@type@choicespecial\ekvd@populate@choicestore}
\protected\long\expandafter\def\csname ekvd@t@choice-enum\endcsname
  {\ekvd@type@choicespecial\ekvd@populate@choiceenum}
%    \end{macrocode}
% \begin{macro}[internal]{\ekvd@type@choicespecial}
%   Initialise similar to a |choice| key. The difference is that we require two
%   arguments (which we assert), a macro to store things in, and a |csv|-list
%   containing the allowed values. |#1| is the |populate| macro according to the
%   type used.
%    \begin{macrocode}
\protected\long\def\ekvd@type@choicespecial#1#2#3%
  {%
    \ekvd@ifnew{}{#2}%
      {%
        \ekvd@assert@twoargs{#3}%
          {%
            \ekvd@type@choice{#2}%
            \def\ekvd@set@choice{#2}%
            #1#3%
          }%
      }%
  }
%    \end{macrocode}
% \end{macro}
% \begin{macro}[internal]
%   {\ekvd@populate@choicestore,\ekvd@populate@choicestore@}
%   We initialise the storing macro if it doesn't yet exist, and then we loop
%   over the value list. The |\edef|s with |\unexpanded| are both necessary to
%   be able to store macro parameter tokens (the outer protects at define time,
%   the inner at use time).
%    \begin{macrocode}
\protected\long\def\ekvd@populate@choicestore#1%
  {%
    \ekvd@newlet#1\ekv@empty
    \ekvcsvloop{\ekvd@populate@choicestore@#1}%
  }
\protected\long\def\ekvd@populate@choicestore@#1#2%
  {%
    \protected\expandafter\edef
      \csname\ekvd@choice@name\ekvd@set\ekvd@set@choice{#2}\endcsname
      {\unexpanded{\edef#1{\unexpanded{#2}}}}%
  }
%    \end{macrocode}
% \end{macro}
% \begin{macro}[internal]
%   {\ekvd@populate@choiceenum,\ekvd@populate@choiceenum@}
%   This is similar to the population of a |choice-store| type, but instead of
%   storing the values in a macro this initialises a count and stores the
%   position of the value in the list inside that count (zero-indexed). The
%   space is necessary to terminate the number scanning, which is the reason we
%   use |\@firstofone| (so that the space after the macro name isn't gobbled by
%   \TeX).
%    \begin{macrocode}
\protected\long\def\ekvd@populate@choiceenum#1%
  {%
    \ekvd@newreg#1{count}%
    \def\ekvd@tmp{0}%
    \ekvcsvloop{\ekvd@populate@choiceenum@#1}%
  }
\protected\long\def\ekvd@populate@choiceenum@#1#2%
  {%
    \protected\expandafter\edef
      \csname\ekvd@choice@name\ekvd@set\ekvd@set@choice{#2}\endcsname
      {#1=\@firstofone{\ekvd@tmp} }%
    \edef\ekvd@tmp{\the\numexpr\ekvd@tmp+1\relax}%
  }
%    \end{macrocode}
% \end{macro}
% \end{macro}
%
% \begin{macro}[internal]{\ekvd@t@unknown-choice}
%    \begin{macrocode}
\protected\long\expandafter\def\csname ekvd@t@unknown-choice\endcsname#1#2%
  {%
    \ekvd@assert@new@for@name{\ekvd@unknown@choice@name\ekvd@set{#1}}%
      {%
        \ekvd@assert@arg
          {%
            \ekvd@assert@not@long
            \ekvd@assert@not@also
            \ekvd@prot\expandafter
            \def\csname\ekvd@unknown@choice@name\ekvd@set{#1}\endcsname##1{#2}%
          }%
      }%
  }
%    \end{macrocode}
% \end{macro}
%
% \begin{macro}[internal]
%   {\ekvd@t@unknown,\ekvd@type@unknown@code,\ekvd@type@unknown@noval}
%   The |unknown| type has different subtypes which would be the key names for
%   other types. It is first checked whether that subtype is defined, if it
%   isn't throw an error, else use that subtype.
%    \begin{macrocode}
\protected\long\def\ekvd@t@unknown#1#2%
  {%
    \ekv@ifdefined{ekvd@type@unknown@\detokenize{#1}}%
      {\csname ekvd@type@unknown@\detokenize{#1}\endcsname{#2}}%
      \ekvd@err@misused@unknown
  }
%    \end{macrocode}
%   The |unknown noval| type can use |\ekvdefunknownNoVal| directly (after
%   asserting some prefixes).
%    \begin{macrocode}
\protected\long\def\ekvd@type@unknown@noval#1%
  {%
    \ekvd@assert@new@for@name{\ekv@name\ekvd@set{}uN}%
      {%
        \ekvd@assert@arg
          {%
            \ekvd@assert@not@also
            \ekvd@assert@not@long
            \ekvd@prot\ekvdefunknownNoVal\ekvd@set{#1}%
          }%
      }%
  }
%    \end{macrocode}
%   The |unknown code| type uses some trickery during the definition in order to
%   swap out |#1| and |#2| in the user supplied definition. This is done via a
%   temporary macro that stores the definition but gets the parameter numbers
%   reversed while the real definition is done.
%    \begin{macrocode}
\protected\long\def\ekvd@type@unknown@code#1%
  {%
    \ekvd@assert@new@for@name{\ekv@name\ekvd@set{}u}%
      {%
        \ekvd@assert@arg
          {%
            \ekvd@assert@not@also
            \begingroup
              \def\ekvd@tmp##1##2{#1}%
              \ekv@exparg
              {%
                \endgroup
                \ekvd@prot\ekvd@long\ekvdefunknown\ekvd@set
              }%
              {\ekvd@tmp{##2}{##1}}%
          }%
      }%
  }
%    \end{macrocode}
% \end{macro}
%
% \begin{macro}[internal]
%   {
%     \ekvd@type@unknown@redirect,
%     \ekvd@type@unknown@redirect-code,
%     \ekvd@type@unknown@redirect-noval
%   }
%   The |unknown redirect| types also just forward to |\ekvredirectunknown|
%   after asserting some prefixes.
%    \begin{macrocode}
\protected\edef\ekvd@type@unknown@redirect#1%
  {%
    \expandafter\noexpand\csname ekvd@type@unknown@redirect-code\endcsname{#1}%
    \expandafter\noexpand\csname ekvd@type@unknown@redirect-noval\endcsname{#1}%
  }
\protected\expandafter\def\csname ekvd@type@unknown@redirect-code\endcsname#1%
  {%
    \ekvd@assert@new@for@name{\ekv@name\ekvd@set{}u}%
      {%
        \ekvd@assert@arg
          {%
            \ekvd@assert@not@also
            \ekvd@assert@not@protected
            \expandafter\ekvredirectunknown\expandafter{\ekvd@set}{#1}%
          }%
      }%
  }
\protected\expandafter\def\csname ekvd@type@unknown@redirect-noval\endcsname#1%
  {%
    \ekvd@assert@new@for@name{\ekv@name\ekvd@set{}uN}%
      {%
        \ekvd@assert@arg
          {%
            \ekvd@assert@not@also
            \ekvd@assert@not@protected
            \ekvd@assert@not@long
            \expandafter\ekvredirectunknownNoVal\expandafter{\ekvd@set}{#1}%
          }%
      }%
  }
%    \end{macrocode}
% \end{macro}
%
% \subsubsection{Key Type Helpers}
% There are some keys that might need helpers during their execution (not during
% their definition, which are gathered as |@type@| macros). These helpers are
% named |@h@|.
%
% \begin{macro}[internal]{\ekvd@h@choice,\ekvd@h@choice@}
% The |choice| helper will just test whether the given choice was defined, if
% not throw an error expandably, else call the macro which stores the code for
% this choice.
%    \begin{macrocode}
\def\ekvd@h@choice#1%
  {%
    \expandafter\ekvd@h@choice@
      \csname\ifcsname#1\endcsname#1\else relax\fi\endcsname
      {#1}%
  }
\def\ekvd@h@choice@#1#2%
  {%
    \ifx#1\relax
      \ekvd@err@choice@invalid{#2}%
      \expandafter\@gobble
    \fi
    #1%
  }
%    \end{macrocode}
% \end{macro}
%
%
% \subsubsection{Handling \texttt{also}}
%
% \begin{macro}[internal]
%   {\ekvd@add@val,\ekvd@add@noval,\ekvd@add@aux,\ekvd@add@aux@}
%    \begin{macrocode}
\protected\long\def\ekvd@add@val#1#2#3%
  {%
    \ekvd@assert@val{#1}%
      {%
        \expandafter\ekvd@add@aux\csname\ekv@name\ekvd@set{#1}\endcsname{{##1}}%
          {#1}{#2}{\ekvd@long\ekvdef}{#3}%
      }%
  }
\protected\long\def\ekvd@add@noval#1#2#3%
  {%
    \ekvd@assert@noval{#1}%
      {%
        \expandafter\ekvd@add@aux\csname\ekv@name\ekvd@set{#1}N\endcsname{}%
          {#1}{#2}\ekvdefNoVal{#3}%
      }%
  }
\protected\long\def\ekvd@add@aux#1#2%
  {%
    \ekvd@extract@prefixes#1%
    \expandafter\ekvd@add@aux@\expandafter{#1#2}%
  }
\protected\long\def\ekvd@add@aux@#1#2#3#4#5%
  {%
    #5%
    \ekvd@prot#4\ekvd@set{#2}{#1#3}%
  }
%    \end{macrocode}
% \end{macro}
%
% \begin{macro}[internal]
%   {
%     \ekvd@extract@prefixes,\ekvd@extract@prefixes@,
%     \ekvd@extract@prefixes@long,\ekvd@extract@prefixes@prot
%   }
% This macro checks which prefixes were used for the definition of a macro and
% sets |\ekvd@long| and |\ekvd@prot| accordingly.
%    \begin{macrocode}
\protected\def\ekvd@extract@prefixes#1%
  {%
    \expandafter\ekvd@extract@prefixes@\meaning#1\ekvd@stop
  }
%    \end{macrocode}
% In the following definition |#1| will get replaced by |macro:|, |#2| by
% |\long| and |#3| by |\protected| (in each, all tokens will have category
% other). This allows us to parse the |\meaning| of a macro for those strings.
%    \begin{macrocode}
\protected\def\ekvd@extract@prefixes@#1#2#3%
  {%
    \protected\def\ekvd@extract@prefixes@##1#1##2\ekvd@stop
      {%
        \ekvd@extract@prefixes@long
          ##1\ekvd@mark\@firstofone#2\ekvd@mark\@gobble\ekvd@stop
          {\let\ekvd@long\long}%
        \ekvd@extract@prefixes@prot
          ##1\ekvd@mark\@firstofone#3\ekvd@mark\@gobble\ekvd@stop
          {\let\ekvd@prot\protected}%
      }%
    \protected\def\ekvd@extract@prefixes@long##1#2##2\ekvd@mark##3##4\ekvd@stop
      {##3}%
    \protected\def\ekvd@extract@prefixes@prot##1#3##2\ekvd@mark##3##4\ekvd@stop
      {##3}%
  }
%    \end{macrocode}
% We use a temporary macro to expand the three arguments of
% |\ekvd@extract@prefixes@|, which will set up the real meaning of itself and
% the parsing for |\long| and |\protected|.
%    \begin{macrocode}
\begingroup
\edef\ekvd@tmp
  {%
    \endgroup
    \ekvd@extract@prefixes@
      {\detokenize{macro:}}%
      {\string\long}%
      {\string\protected}%
  }
\ekvd@tmp
%    \end{macrocode}
% \end{macro}
%
%
% \subsubsection{Tests}
%
% \begin{macro}[internal]{\ekvd@newlet,\ekvd@newreg}
% These macros test whether a control sequence is defined, if it isn't they
% define it, either via |\let| or via the correct \cs[no-index]{new\meta{reg}}.
%    \begin{macrocode}
\protected\def\ekvd@newlet#1#2%
  {%
    \ifdefined#1\ekv@fi@gobble\fi\@firstofone{\let#1#2}%
  }
\protected\def\ekvd@newreg#1#2%
  {%
    \ifdefined#1\ekv@fi@gobble\fi\@firstofone{\csname new#2\endcsname#1}%
  }
%    \end{macrocode}
% \end{macro}
%
% \begin{macro}[internal]
%   {\ekvd@assert@twoargs,\ekvd@ifnottwoargs,\ekvd@ifempty@gtwo}
% A test for exactly two tokens can be reduced for an empty-test after gobbling
% two tokens, in the case that there are fewer tokens than two in the argument,
% only macros will be gobbled that are needed for the true branch, which doesn't
% hurt, and if there are more this will not be empty.
%    \begin{macrocode}
\long\def\ekvd@assert@twoargs#1%
  {%
    \ekvd@ifnottwoargs{#1}{\ekvd@err@missing@definition}%
  }
\long\def\ekvd@ifnottwoargs#1%
  {%
    \ekvd@ifempty@gtwo#1\ekv@ifempty@B
      \ekv@ifempty@false\ekv@ifempty@A\ekv@ifempty@B\@firstoftwo
  }
\long\def\ekvd@ifempty@gtwo#1#2{\ekv@ifempty@\ekv@ifempty@A}
%    \end{macrocode}
% \end{macro}
%
% \begin{macro}[internal]
%   {
%     \ekvd@assert@val,\ekvd@assert@val@,\ekvd@assert@noval,\ekvd@assert@noval@,
%     \ekvd@extract@args,\ekvd@extracted@args,\ekvd@one@arg@string
%   }
% Assert that a given key is defined as a value taking key or a |NoVal| key with
% the correct argument structure, respectively.
%    \begin{macrocode}
\protected\def\ekvd@assert@val#1%
  {%
    \ekvifdefined\ekvd@set{#1}%
      {\expandafter\ekvd@assert@val@\csname\ekv@name\ekvd@set{#1}\endcsname}%
      {%
        \ekvifdefinedNoVal\ekvd@set{#1}%
          \ekvd@err@add@val@on@noval
          {\ekvd@err@undefined@key{#1}}%
        \@gobble
      }%
  }
\protected\def\ekvd@assert@val@#1%
  {%
    \expandafter\ekvd@extract@args\meaning#1\ekvd@stop
    \unless\ifx\ekvd@extracted@args\ekvd@one@arg@string
      \ekvd@err@unsupported@arg
    \fi
    \@firstofone
  }%
\protected\def\ekvd@assert@noval#1%
  {%
    \ekvifdefinedNoVal\ekvd@set{#1}%
      {\expandafter\ekvd@assert@noval@\csname\ekv@name\ekvd@set{#1}N\endcsname}%
      {%
        \ekvifdefined\ekvd@set{#1}%
          \ekvd@err@add@noval@on@val
          {\ekvd@err@undefined@key{#1}}%
        \@gobble
      }%
  }
\protected\def\ekvd@assert@noval@#1%
  {%
    \expandafter\ekvd@extract@args\meaning#1\ekvd@stop
    \unless\ifx\ekvd@extracted@args\ekv@empty
      \ekvd@err@unsupported@arg
    \fi
    \@firstofone
  }
\protected\def\ekvd@extract@args#1%
  {%
    \protected\def\ekvd@extract@args##1#1##2->##3\ekvd@stop
      {\def\ekvd@extracted@args{##2}}%
  }
\expandafter\ekvd@extract@args\expandafter{\detokenize{macro:}}
\edef\ekvd@one@arg@string{\string#1}
%    \end{macrocode}
% \end{macro}
%
% \begin{macro}[internal]{\ekvd@assert@arg,\ekvd@assert@arg@msg,\ekvd@ifnoarg}
% There is no need to actually define |\ekvd@ifnoarg| here, as it will be set by
% either |\ekvd@arg| or |\ekvd@noarg|.
%    \begin{macrocode}
\def\ekvd@assert@arg{\ekvd@ifnoarg\ekvd@err@missing@definition}
\long\def\ekvd@assert@arg@msg#1%
  {%
    \ekvd@ifnoarg{\ekvd@err@missing@definition@msg{#1}}%
  }
%    \end{macrocode}
% \end{macro}
%
% \begin{macro}[internal]{\ekvd@assert@filledarg,\ekvd@ifnoarg@or@empty}
%    \begin{macrocode}
\long\def\ekvd@assert@filledarg#1%
  {%
    \ekvd@ifnoarg@or@empty{#1}\ekvd@err@missing@definition
  }
\long\def\ekvd@ifnoarg@or@empty#1%
  {%
    \ekvd@ifnoarg
      \@firstoftwo
      {\ekv@ifempty{#1}}%
  }
%    \end{macrocode}
% \end{macro}
%
% \begin{macro}[internal]
%   {
%     \ekvd@assert@not@long,\ekvd@assert@not@protected,\ekvd@assert@not@also
%     \ekvd@assert@not@long@also,\ekvd@assert@not@protected@also,
%     \ekvd@assert@new,\ekvd@assert@not@new
%   }
% Some key-types don't want to be |also|, |\long| or |\protected|, so we provide
% macros to test this and throw an error, this could be silently ignored but now
% users will learn to not use unnecessary stuff which slows the compilation
% down.
%    \begin{macrocode}
\def\ekvd@assert@not@long{\ifx\ekvd@long\long\ekvd@err@no@prefix{long}\fi}
\def\ekvd@assert@not@protected
  {\ifx\ekvd@prot\protected\ekvd@err@no@prefix{protected}\fi}
\def\ekvd@assert@not@also{\ekvd@ifalso{\ekvd@err@no@prefix{also}}{}}
\def\ekvd@assert@not@long@also
  {\ifx\ekvd@long\long\ekvd@err@no@prefix@also{long}\fi}
\def\ekvd@assert@not@protected@also
  {\ifx\ekvd@prot\protected\ekvd@err@no@prefix@also{protected}\fi}
\def\ekvd@assert@new#1#2%
  {\csname ekvifdefined#1\endcsname\ekvd@set{#2}{\ekvd@err@not@new}}
\def\ekvd@assert@not@new
  {\ifx\ekvd@ifnew\ekvd@assert@new\ekvd@err@no@prefix{new}\fi}
\def\ekvd@assert@new@for@name#1%
  {%
    \ifx\ekvd@ifnew\ekvd@assert@new
      \ekv@fi@firstoftwo
    \fi
    \@secondoftwo
      {\ekv@ifdefined{#1}\ekvd@err@not@new}%
      \@firstofone
  }
%    \end{macrocode}
% \end{macro}
%
% \begin{macro}[internal]
%   {
%     \ekvd@if@not@already@choice, \ekvd@if@not@already@choice@a,
%     \ekvd@if@not@already@choice@b
%   }
% It is bad to use |also| on a key that already contains a |choice|, as both
% choices would share the same valid values and thus lead to each callback being
% used twice. The following is a rudimentary test against this.
%    \begin{macrocode}
\protected\def\ekvd@if@not@already@choice#1%
  {%
    \expandafter\ekvd@if@not@already@choice@a
      \csname\ekv@name\ekvd@set{#1}\endcsname
      {}\ekvd@h@choice\ekvd@stop
  }
\protected\def\ekvd@if@not@already@choice@a
  {%
    \expandafter\ekvd@if@not@already@choice@b
  }
\long\protected\def\ekvd@if@not@already@choice@b#1\ekvd@h@choice#2\ekvd@stop
  {%
    \ekv@ifempty{#2}\@firstofone\@gobble
  }
%    \end{macrocode}
% \end{macro}
%
% \begin{macro}[internal]{\ekvd@ifspace,\ekvd@ifspace@}
% Yet another test which can be reduced to an if-empty, this time by gobbling
% everything up to the first space.
%    \begin{macrocode}
\long\def\ekvd@ifspace#1%
  {%
    \ekvd@ifspace@#1 \ekv@ifempty@B
      \ekv@ifempty@false\ekv@ifempty@A\ekv@ifempty@B\@firstoftwo
  }
\long\def\ekvd@ifspace@#1 % keep this space
  {%
    \ekv@ifempty@\ekv@ifempty@A
  }
%    \end{macrocode}
% \end{macro}
%
% \subsubsection{Messages}
%
% Most messages of \expkvd\ are not expandable, since they only appear during
% key-definition, which is not expandable anyway.
%
% \begin{macro}[internal]
%   {
%     \ekvd@errm,\ekvd@err@missing@definition,\ekvd@err@missing@definition@msg,
%     \ekvd@err@missing@type,\ekvd@err@undefined@prefix,\ekvd@err@undefined@key,
%     \ekvd@err@no@prefix,\ekvd@err@no@prefix@msg,\ekvd@err@no@prefix@also,
%     \ekvd@err@add@val@on@noval,\ekvd@err@add@noval@on@val,
%     \ekvd@err@unsupported@arg,\ekvd@err@not@new
%   }
% The non-expandable error messages are boring, so here they are:
%    \begin{macrocode}
\protected\def\ekvd@errm#1{\errmessage{expkv-def Error: #1}}
\protected\def\ekvd@err@missing@definition
  {\ekvd@errm{Missing definition for key `\ekvd@cur'}}
\protected\def\ekvd@err@missing@definition@msg#1%
  {\ekvd@errm{Missing definition for key `\ekv@unexpanded{#1}'}}
\protected\def\ekvd@err@missing@type
  {\ekvd@errm{Missing type prefix for key `\ekvd@cur'}}
\protected\def\ekvd@err@undefined@prefix#1%
  {%
    \ekvd@errm
      {%
        Undefined prefix `\ekv@unexpanded{#1}' found while processing
        `\ekvd@cur'%
      }%
  }
\protected\def\ekvd@err@undefined@key#1%
  {%
    \ekvd@errm
      {Undefined key `\ekv@unexpanded{#1}' found while processing `\ekvd@cur'}%
  }
\protected\def\ekvd@err@undefined@noval#1%
  {%
    \ekvd@errm
      {%
        Undefined noval key `\unexpanded{#1}' found while processing
        `\ekvd@cur'%
      }%
  }
\protected\def\ekvd@err@no@prefix#1%
  {\ekvd@errm{prefix `#1' not accepted in `\ekvd@cur'}}
\protected\def\ekvd@err@no@prefix@msg#1#2%
  {\ekvd@errm{prefix `#2' not accepted in `\ekv@unexpanded{#1}'}}
\protected\def\ekvd@err@no@prefix@also#1%
  {\ekvd@errm{`\ekvd@cur' not allowed with a `#1' key}}
\protected\def\ekvd@err@add@val@on@noval
  {\ekvd@errm{`\ekvd@cur' not allowed with a NoVal key}}
\protected\def\ekvd@err@add@noval@on@val
  {\ekvd@errm{`\ekvd@cur' not allowed with a value taking key}}
\protected\def\ekvd@err@unsupported@arg\fi\@firstofone#1%
  {%
    \fi
    \ekvd@errm
      {%
        Existing key-macro has the unsupported argument string
        `\ekvd@extracted@args' for key `\ekvd@cur'%
      }%
  }
\protected\def\ekvd@err@not@new
  {\ekvd@errm{The key for `\ekvd@cur' is already defined}}
\protected\long\def\ekvd@err@misused@unknown
  {\ekvd@errm{Misuse of the unknown type found while processing `\ekvd@cur'}}
%    \end{macrocode}
% \end{macro}
%
% \begin{macro}[internal]
%   {
%     \ekvd@err@choice@invalid,\ekvd@err@choice@invalid@,\ekvd@choice@name,
%     \ekvd@unknown@choice@name
%   }
% |\ekvd@err@choice@invalid| will have to use this mechanism to throw its
% message. Also we have to retrieve the name parts of the choice in an easy way,
% so we use parentheses of catcode 8 here, which should suffice in most cases to
% allow for a correct separation.
%    \begin{macrocode}
\def\ekvd@err@choice@invalid#1%
  {%
    \ekvd@err@choice@invalid@#1%
  }
\begingroup
\catcode40=8
\catcode41=8
\@firstofone{\endgroup
\def\ekvd@choice@name#1#2#3%
  {%
    ekvd#1(#2)\detokenize{#3}%
  }
\def\ekvd@unknown@choice@name#1#2%
  {%
    ekvd:u:#1(#2)%
  }
\def\ekvd@err@choice@invalid@ ekvd#1(#2)\detokenize#3%
  {%
    \ekv@ifdefined{\ekvd@unknown@choice@name{#1}{#2}}%
      {\csname\ekvd@unknown@choice@name{#1}{#2}\endcsname{#3}}%
      {\ekvd@err{invalid choice `#3' for `#2' in set `#1'}}%
  }
}
%    \end{macrocode}
% \end{macro}
%
% \begin{macro}[internal]{\ekvd@err}
% The expandable error messages use |\ekvd@err|, which is just like |\ekv@err|
% from \expkv. It uses a runaway argument to start the error message.
%    \begin{macrocode}
\ekv@exparg{\long\def\ekvd@err#1}{\ekverr{expkv-def}{#1}}
%    \end{macrocode}
% \end{macro}
%
% Now everything that's left is to reset the category code of |@|.
%    \begin{macrocode}
\catcode`\@=\ekvd@tmp
%    \end{macrocode}
%
% \gobbledocstriptag
%</tex>
%^^A=<<
%
%^^A=<<
%
% \end{implementation}^^A=<<
%
% \clearpage
% \PrintIndex
%
\endinput
%
^^A vim: ft=tex fdm=marker fmr=>>=,=<<

\ProvidesPackage{expkv-def}%
  [\ekvdDate\space v\ekvdVersion\space a key-defining frontend for expkv]
%    \end{macrocode}
% \gobbledocstriptag
%</pkg>
%^^A=<<
%^^A the ConTeXt module >>=
% \subsection{The \ConTeXt\ module}
% \gobbledocstriptag
%<*ctx>
%    \begin{macrocode}
\writestatus{loading}{ConTeXt User Module / expkv-def}
\usemodule[expkv]
\unprotect
\input expkv-def.tex
\writestatus{loading}
  {ConTeXt User Module / expkv-def / Version \ekvdVersion\space loaded}
\protect\endinput
%    \end{macrocode}
% \gobbledocstriptag
%</ctx>
%^^A=<<
%^^A main file >>=
% \subsection{The Generic Code}
% The rest of this implementation will be the generic code.
% \gobbledocstriptag
%<*tex>
%
% Load \expkv\ if the package didn't already do so -- since \expkv\ has
% safeguards against being loaded twice this does no harm and the overhead
% isn't that big. Also we reuse some of the internals of \expkv\ to save us from
% retyping them.
%    \begin{macrocode}
\input expkv
%    \end{macrocode}
%
% We make sure that \file{expkv-def.tex} is only input once:
%    \begin{macrocode}
\expandafter\ifx\csname ekvdVersion\endcsname\relax
\else
  \expandafter\endinput
\fi
%    \end{macrocode}
%
% \begin{macro}{\ekvdVersion,\ekvdDate}
% We're on our first input, so lets store the version and date in a macro.
%    \begin{macrocode}
\def\ekvdVersion{0.8c}
\def\ekvdDate{2021-09-20}
%    \end{macrocode}
% \end{macro}
%
% If the \LaTeX\ format is loaded we want to be a good file and report back who
% we are, for this the package will have defined |\ekvd@tmp| to use
% |\ProvidesFile|, else this will expand to a |\relax| and do no harm.
%    \begin{macrocode}
\csname ekvd@tmp\endcsname
%    \end{macrocode}
%
% Store the category code of |@| to later be able to reset it and change it to
% 11 for now.
%    \begin{macrocode}
\expandafter\chardef\csname ekvd@tmp\endcsname=\catcode`\@
\catcode`\@=11
%    \end{macrocode}
% |\ekvd@tmp| will be reused later to handle expansion during the key
% defining. But we don't need it to ever store information long-term after
% \expkvd\ was initialized.
%
% \begin{macro}[internal]{\ekvd@ifprimitive}
%    \begin{macrocode}
\protected\long\def\ekvd@ifprimitive#1%
  {%
    \begingroup
      \edef\ekvd@tmpa{\string #1}%
      \edef\ekvd@tmpb{\meaning#1}%
      \expandafter
    \endgroup
    \ifx\ekvd@tmpa\ekvd@tmpb
      \ekv@fi@firstoftwo
    \fi
    \@secondoftwo
  }
%    \end{macrocode}
% \end{macro}
%
% \begin{macro}[internal]
%   {\ekvd@long,\ekvd@prot,\ekvd@clear@prefixes,\ekvd@ifalso}
% \expkvd\ will use |\ekvd@long|, |\ekvd@prot|, and |\ekvd@ifalso| to store
% whether a key should be defined as |\long| or |\protected| or adds an action
% to an existing key, and we have to clear them for every new key. By default
% |long| and |protected| will just be empty, |ifalso| will be
% |\@secondoftwo|, and |ifnew| will just use its third argument.
%    \begin{macrocode}
\protected\def\ekvd@clear@prefixes
  {%
    \let\ekvd@long\ekv@empty
    \let\ekvd@prot\ekv@empty
    \let\ekvd@ifalso\@secondoftwo
    \long\def\ekvd@ifnew##1##2##3{##3}%
  }
\ekvd@clear@prefixes
%    \end{macrocode}
% \end{macro}
%
% \begin{macro}{\ekvdefinekeys}
% This is the one front-facing macro which provides the interface to define
% keys. It's using |\ekvparse| to handle the \kv\ list, the interpretation will
% be done by |\ekvd@noarg| and |\ekvd@|. The \set\ for which the keys should be
% defined is stored in |\ekvd@set|.
%    \begin{macrocode}
\protected\def\ekvdefinekeys#1%
  {%
    \def\ekvd@set{#1}%
    \ekvparse\ekvd@noarg\ekvd@arg
  }
%    \end{macrocode}
% \end{macro}
%
% \begin{macro}[internal]{\ekvd@noarg,\ekvd@arg,\ekvd@handle}
% |\ekvd@noarg| and |\ekvd@arg| store whether there was a value in the \kv\
% pair. |\ekvd@handle| has to test whether there is a space inside the key and
% if so calls the prefix grabbing routine, else we throw an error and ignore the
% key.
%    \begin{macrocode}
\protected\long\def\ekvd@noarg#1%
  {%
    \let\ekvd@ifnoarg\@firstoftwo
    \expandafter\ekvd@handle\detokenize{#1}\ekvd@stop{}%
  }
\protected\long\def\ekvd@arg#1%
  {%
    \let\ekvd@ifnoarg\@secondoftwo
    \expandafter\ekvd@handle\detokenize{#1}\ekvd@stop
  }
\protected\long\def\ekvd@handle#1\ekvd@stop#2%
  {%
    \ekvd@clear@prefixes
    \def\ekvd@cur{#1}%
    \ekvd@ifspace{#1}%
      {\ekvd@prefix\ekv@mark#1\ekv@stop{#2}}%
      \ekvd@err@missing@type
  }
%    \end{macrocode}
% \end{macro}
%
% \begin{macro}[internal]{\ekvd@prefix,\ekvd@prefix@}
% \expkvd\ separates prefixes into two groups, the first being prefixes in the
% \TeX\ sense (|long| and |protected|) which use |@p@| in their name, the other
% being key-types (|code|, |int|, \emph{etc.}) which use |@t@| instead.
% |\ekvd@prefix| splits at the first space and checks whether its a |@p@|
% or |@t@| type prefix. If it is neither throw an error and gobble the
% definition (the value).
%    \begin{macrocode}
\protected\def\ekvd@prefix#1 {\ekv@strip{#1}\ekvd@prefix@\ekv@mark}
\protected\def\ekvd@prefix@#1#2\ekv@stop
  {%
    \ekv@ifdefined{ekvd@t@#1}%
      {\ekv@strip{#2}{\csname ekvd@t@#1\endcsname}}%
      {%
        \ekv@ifdefined{ekvd@p@#1}%
          {\csname ekvd@p@#1\endcsname\ekvd@prefix@after@p{#2}}%
          {\ekvd@err@undefined@prefix{#1}\@gobble}%
      }%
  }
%    \end{macrocode}
% \end{macro}
%
% \begin{macro}[internal]{\ekvd@prefix@after@p}
% The |@p@| type prefixes are all just modifying a following |@t@| type, so they
% will need to search for another prefix. This is true for all of them, so we
% use a macro to handle this. It'll throw an error if there is no other prefix.
%    \begin{macrocode}
\protected\def\ekvd@prefix@after@p#1%
  {%
    \ekvd@ifspace{#1}%
      {\ekvd@prefix#1\ekv@stop}%
      {\ekvd@err@missing@type\@gobble}%
  }
%    \end{macrocode}
% \end{macro}
%
% \begin{macro}[internal]
%   {\ekvd@p@long,\ekvd@p@protected,\ekvd@p@protect,\ekvd@p@also,\ekvd@p@new}
% Define the |@p@| type prefixes, they all just store some information in a
% temporary macro.
%    \begin{macrocode}
\protected\def\ekvd@p@long{\let\ekvd@long\long}
\protected\def\ekvd@p@protected{\let\ekvd@prot\protected}
\let\ekvd@p@protect\ekvd@p@protected
\protected\def\ekvd@p@also{\let\ekvd@ifalso\@firstoftwo}
\protected\def\ekvd@p@new{\let\ekvd@ifnew\ekvd@assert@new}
%    \end{macrocode}
% \end{macro}
%
% \subsubsection{Key Types}
%
% \begin{macro}[internal]{\ekvd@type@set,\ekvd@t@set}
% The |set| type is quite straight forward, just define a |NoVal| key to call
% |\ekvchangeset|.
%    \begin{macrocode}
\protected\def\ekvd@type@set#1#2%
  {%
    \ekvd@assert@not@long
    \ekvd@assert@not@protected
    \ekvd@ifnew{NoVal}{#1}%
      {%
        \ekv@ifempty{#2}%
          {\ekvd@err@missing@definition}%
          {%
            \ekvd@ifalso
              {%
                \ekv@expargtwice{\ekvd@add@noval{#1}}%
                  {\ekvchangeset{#2}}%
                  \ekvd@assert@not@protected@also
              }%
              {\ekv@expargtwice{\ekvdefNoVal\ekvd@set{#1}}{\ekvchangeset{#2}}}%
          }%
      }%
  }
\protected\def\ekvd@t@set#1#2%
  {%
    \ekvd@ifnoarg
      {\ekvd@type@set{#1}{#1}}%
      {\ekvd@type@set{#1}{#2}}%
  }
%    \end{macrocode}
% \end{macro}
%
% \begin{macro}[internal]{\ekvd@type@noval,\ekvd@t@noval,\ekvd@t@enoval}
% Another pretty simple type, |noval| just needs to assert that there is a
% definition and that |long| wasn't specified.
% There are types where the difference in the variants is so small, that we
% define a common handler for them, those common handlers are named with
% |@type@|. |noval| and |enoval| are so similar that we can use such a |@type@|
% macro, even if we could've done |noval| in a slightly faster way without it.
%    \begin{macrocode}
\protected\long\def\ekvd@type@noval#1#2#3%
  {%
    \ekvd@ifnew{NoVal}{#2}%
      {%
        \ekvd@assert@arg
          {%
            \ekvd@assert@not@long
            \ekvd@prot#1\ekvd@tmp{#3}%
            \ekvd@ifalso
              {\ekv@exparg{\ekvd@add@noval{#2}}\ekvd@tmp{}}%
              {\ekvletNoVal\ekvd@set{#2}\ekvd@tmp}%
          }%
      }%
  }
\protected\def\ekvd@t@noval{\ekvd@type@noval\def}
\protected\def\ekvd@t@enoval{\ekvd@type@noval\edef}
%    \end{macrocode}
% \end{macro}
%
% \begin{macro}[internal]{\ekvd@type@code,\ekvd@t@code,\ekvd@t@ecode}
% |code| is simple as well, |ecode| has to use |\edef| on a temporary macro,
% since \expkv\ doesn't provide an |\ekvedef|.
%    \begin{macrocode}
\protected\long\def\ekvd@type@code#1#2#3%
  {%
    \ekvd@ifnew{}{#2}%
      {%
        \ekvd@assert@arg
          {%
            \ekvd@prot\ekvd@long#1\ekvd@tmp##1{#3}%
            \ekvd@ifalso
              {\ekv@exparg{\ekvd@add@val{#2}}{\ekvd@tmp{##1}}{}}%
              {\ekvlet\ekvd@set{#2}\ekvd@tmp}%
          }%
      }%
  }
\protected\def\ekvd@t@code{\ekvd@type@code\def}
\protected\def\ekvd@t@ecode{\ekvd@type@code\edef}
%    \end{macrocode}
% \end{macro}
%
% \begin{macro}[internal]
%   {
%     \ekvd@type@default,\ekvd@t@default,\ekvd@t@qdefault,\ekvd@t@odefault,
%     \ekvd@t@fdefault
%   }
% |\ekvd@type@default| asserts there was an argument, also the key for which one
% wants to set a default has to be already defined (this is not so important for
% |default|, but |qdefault| requires is). If everything is good, |\edef| a
% temporary macro that expands |\ekvd@set| and the |\csname| for the key, and in
% the case of |qdefault| does the first expansion step of the key-macro.
%    \begin{macrocode}
\protected\long\def\ekvd@type@default#1#2#3#4%
  {%
    \ekvd@assert@arg
      {%
        \ekvifdefined\ekvd@set{#3}%
          {%
            \ekvd@assert@not@new
            \ekvd@assert@not@long
            \ekvd@prot\edef\ekvd@tmp
              {%
                \ekv@unexpanded\expandafter#1%
                  {#2\csname\ekv@name\ekvd@set{#3}\endcsname{#4}}%
              }%
            \ekvd@ifalso
              {\ekv@exparg{\ekvd@add@noval{#3}}\ekvd@tmp{}}%
              {\ekvletNoVal\ekvd@set{#3}\ekvd@tmp}%
          }%
          {\ekvd@err@undefined@key{#3}}%
      }%
  }
\protected\def\ekvd@t@default{\ekvd@type@default{}{}}
\protected\def\ekvd@t@qdefault{\ekvd@type@default{\expandafter\expandafter}{}}
\let\ekvd@t@odefault\ekvd@t@qdefault
\protected\def\ekvd@t@fdefault{\ekvd@type@default{}{\romannumeral`\^^@}}
%    \end{macrocode}
% \end{macro}
%
% \begin{macro}[internal]{\ekvd@t@edefault}
% |edefault| is too different from |default| and |qdefault| to reuse the
% |@type@| macro, as it doesn't need |\unexpanded| inside of |\edef|.
%    \begin{macrocode}
\protected\long\def\ekvd@t@edefault#1#2%
  {%
    \ekvd@assert@arg
      {%
        \ekvifdefined\ekvd@set{#1}%
          {%
            \ekvd@assert@not@new
            \ekvd@assert@not@long
            \ekvd@prot\edef\ekvd@tmp
              {\csname\ekv@name\ekvd@set{#1}\endcsname{#2}}%
            \ekvd@ifalso
              {\ekv@exparg{\ekvd@add@noval{#1}}\ekvd@tmp{}}%
              {\ekvletNoVal\ekvd@set{#1}\ekvd@tmp}%
          }%
          {\ekvd@err@undefined@key{#1}}%
      }%
  }
%    \end{macrocode}
% \end{macro}
%
% \begin{macro}[internal]
%   {\ekvd@t@initial,\ekvd@t@oinitial,\ekvd@t@finitial,\ekvd@t@einitial}
%    \begin{macrocode}
\long\def\ekvd@type@initial#1#2#3#4%
  {%
    \ekvd@assert@not@new
    \ekvd@assert@not@also
    \ekvd@assert@not@long
    \ekvd@assert@not@protected
    \ekvd@ifnoarg
      {%
        \ekvifdefinedNoVal\ekvd@set{#3}%
          {\csname\ekv@name\ekvd@set{#3}N\endcsname}%
          {\ekvd@err@undefined@noval{#3}}%
      }%
      {%
        \ekvifdefined\ekvd@set{#3}%
          {%
            #1{#2#4}%
            \csname\ekv@name\ekvd@set{#3}\expandafter\endcsname\expandafter
              {\ekvd@tmp}%
          }%
          {\ekvd@err@undefined@key{#3}}%
      }%
  }
\def\ekvd@t@initial{\ekvd@type@initial{\def\ekvd@tmp}{}}
\def\ekvd@t@oinitial{\ekvd@type@initial{\ekv@exparg{\def\ekvd@tmp}}{}}
\def\ekvd@t@einitial{\ekvd@type@initial{\edef\ekvd@tmp}{}}
\def\ekvd@t@finitial
  {\ekvd@type@initial{\ekv@exparg{\def\ekvd@tmp}}{\romannumeral`\^^@}}
%    \end{macrocode}
% \end{macro}
%
% \begin{macro}[internal]
%   {
%     \ekvd@type@bool,\ekvd@t@bool,\ekvd@t@gbool,\ekvd@t@boolTF,\ekvd@t@gboolTF,
%     \ekvd@t@invbool,\ekvd@t@ginvbool,\ekvd@t@invboolTF,\ekvd@t@ginvboolTF,
%   }
% The boolean types are a quicker version of a |choice| that accept |true| and
% |false|, and set up the |NoVal| action to be identical to \texttt{\key=true}.
% The |true| and |false| actions are always just |\let|ting the macro in |#7| to
% some other macro (\emph{e.g.}, \cs[no-index]{iftrue}).
%    \begin{macrocode}
\protected\def\ekvd@type@bool#1#2#3#4#5%
  {%
    \ekvd@ifnew{}{#4}%
      {%
        \ekvd@ifnew{NoVal}{#4}%
          {%
            \ekvd@assert@filledarg{#5}%
              {%
                \ekvd@newlet#5#3%
                \ekvd@type@choice{#4}%
                \protected\ekvdefNoVal\ekvd@set{#4}{#1\let#5#2}%
                \protected\expandafter\def
                  \csname\ekvd@choice@name\ekvd@set{#4}{true}\endcsname
                  {#1\let#5#2}%
                \protected\expandafter\def
                  \csname\ekvd@choice@name\ekvd@set{#4}{false}\endcsname
                  {#1\let#5#3}%
              }%
          }%
      }%
  }
\protected\def\ekvd@t@bool{\ekvd@type@bool{}\iftrue\iffalse}
\protected\def\ekvd@t@gbool{\ekvd@type@bool\global\iftrue\iffalse}
\protected\def\ekvd@t@boolTF{\ekvd@type@bool{}\@firstoftwo\@secondoftwo}
\protected\def\ekvd@t@gboolTF{\ekvd@type@bool\global\@firstoftwo\@secondoftwo}
\protected\def\ekvd@t@invbool{\ekvd@type@bool{}\iffalse\iftrue}
\protected\def\ekvd@t@ginvbool{\ekvd@type@bool\global\iffalse\iftrue}
\protected\def\ekvd@t@invboolTF{\ekvd@type@bool{}\@secondoftwo\@firstoftwo}
\protected\def\ekvd@t@ginvboolTF
  {\ekvd@type@bool\global\@secondoftwo\@firstoftwo}
%    \end{macrocode}
% \end{macro}
%
% \begin{macro}[internal]
%   {
%     \ekvd@type@boolpair,
%     \ekvd@t@boolpair,\ekvd@t@gboolpair,\ekvd@t@boolpairTF,\ekvd@t@gboolpairTF
%   }
% The boolean pair types are essentially the same as the boolean types, but set
% two macros instead of one.
%    \begin{macrocode}
\protected\def\ekvd@type@boolpair#1#2#3#4#5#6%
  {%
    \ekvd@ifnew{}{#4}%
      {%
        \ekvd@ifnew{NoVal}{#4}%
          {%
            \ekvd@newlet#5#3%
            \ekvd@newlet#6#2%
            \ekvd@type@choice{#4}%
            \protected\ekvdefNoVal\ekvd@set{#4}{#1\let#5#2#1\let#6#3}%
            \protected\expandafter\def
              \csname\ekvd@choice@name\ekvd@set{#4}{true}\endcsname
              {#1\let#5#2#1\let#6#3}%
            \protected\expandafter\def
              \csname\ekvd@choice@name\ekvd@set{#4}{false}\endcsname
              {#1\let#5#3#1\let#6#2}%
          }%
      }%
  }
\protected\def\ekvd@t@boolpair#1#2%
  {\ekvd@assert@twoargs{#2}{\ekvd@type@boolpair{}\iftrue\iffalse{#1}#2}}
\protected\def\ekvd@t@gboolpair#1#2%
  {\ekvd@assert@twoargs{#2}{\ekvd@type@boolpair\global\iftrue\iffalse{#1}#2}}
\protected\def\ekvd@t@boolpairTF#1#2%
  {%
    \ekvd@assert@twoargs{#2}%
      {\ekvd@type@boolpair{}\@firstoftwo\@secondoftwo{#1}#2}%
  }
\protected\def\ekvd@t@gboolpairTF#1#2%
  {%
    \ekvd@assert@twoargs{#2}%
      {\ekvd@type@boolpair\global\@firstoftwo\@secondoftwo{#1}#2}%
  }
%    \end{macrocode}
% \end{macro}
%
% \begin{macro}[internal]
%   {
%     \ekvd@type@data,\ekvd@t@data,\ekvd@t@gdata,\ekvd@t@dataT,\ekvd@t@gdataT
%   }
%    \begin{macrocode}
\protected\def\ekvd@type@data#1#2#3#4#5#6%
  {%
    \ekvd@ifnew{}{#5}%
      {%
        \ekvd@assert@filledarg{#6}%
          {%
            \ekvd@newlet#6#1%
            \ekvd@ifalso
              {%
                \let\ekvd@prot\protected
                \ekvd@add@val{#5}{\long#2#6####1#3{####1{#4}}}{}%
              }%
              {%
                \protected\ekvd@long\ekvdef\ekvd@set{#5}%
                  {\long#2#6####1#3{####1{#4}}}%
              }%
          }%
      }%
  }
\protected\def\ekvd@t@data
  {\ekvd@type@data\@secondoftwo\edef{####2}{\ekv@unexpanded{##1}}}
\protected\def\ekvd@t@edata{\ekvd@type@data\@secondoftwo\edef{####2}{##1}}
\protected\def\ekvd@t@gdata
  {\ekvd@type@data\@secondoftwo\xdef{####2}{\ekv@unexpanded{##1}}}
\protected\def\ekvd@t@xdata{\ekvd@type@data\@secondoftwo\xdef{####2}{##1}}
\protected\def\ekvd@t@dataT
  {\ekvd@type@data\@gobble\edef{}{\ekv@unexpanded{##1}}}
\protected\def\ekvd@t@edataT{\ekvd@type@data\@gobble\edef{}{##1}}
\protected\def\ekvd@t@gdataT
  {\ekvd@type@data\@gobble\xdef{}{\ekv@unexpanded{##1}}}
\protected\def\ekvd@t@xdataT{\ekvd@type@data\@gobble\xdef{}{##1}}
%    \end{macrocode}
% \end{macro}
%
% \begin{macro}[internal]{\ekvd@type@box,\ekvd@t@box,\ekvd@t@gbox}
% Set up our boxes. Though we're a generic package we want to be colour safe, so
% we put an additional grouping level inside the box contents, for the case that
% someone uses \pkg{color}. |\ekvd@newreg| is a small wrapper which tests
% whether the first argument is defined and if not does
% |\csname new#2\endcsname#1|.
%    \begin{macrocode}
\protected\def\ekvd@type@box#1#2#3%
  {%
    \ekvd@ifnew{}{#2}%
      {%
        \ekvd@assert@filledarg{#3}%
          {%
            \ekvd@newreg#3{box}%
            \ekvd@ifalso
              {%
                \let\ekvd@prot\protected
                \ekvd@add@val{#2}{#1\setbox#3=\hbox{\begingroup##1\endgroup}}{}%
              }%
              {%
                \protected\ekvd@long\ekvdef\ekvd@set{#2}%
                  {#1\setbox#3=\hbox{\begingroup##1\endgroup}}%
              }%
          }%
      }%
  }
\protected\def\ekvd@t@box{\ekvd@type@box{}}
\protected\def\ekvd@t@gbox{\ekvd@type@box\global}
%    \end{macrocode}
% \end{macro}
%
% \begin{macro}[internal]{\ekvd@type@toks,\ekvd@t@toks,\ekvd@t@gtoks}
% Similar to |box|, but set the |toks|.
%    \begin{macrocode}
\protected\def\ekvd@type@toks#1#2#3%
  {%
    \ekvd@ifnew{}{#2}%
      {%
        \ekvd@assert@filledarg{#3}%
          {%
            \ekvd@newreg#3{toks}%
            \ekvd@ifalso
              {%
                \let\ekvd@prot\protected
                \ekvd@add@val{#2}{#1#3={##1}}{}%
              }%
              {\protected\ekvd@long\ekvdef\ekvd@set{#2}{#1#3={##1}}}%
          }%
      }%
  }
\protected\def\ekvd@t@toks{\ekvd@type@toks{}}
\protected\def\ekvd@t@gtoks{\ekvd@type@toks\global}
%    \end{macrocode}
% \end{macro}
%
% \begin{macro}[internal]{\ekvd@type@preapptoks,\ekvd@t@apptoks,\ekvd@t@gapptoks}
% Just like |toks|, but expand the current contents of the |toks| register to
% append the new contents.
%    \begin{macrocode}
\ekvd@ifprimitive\toksapp
  {%
    \protected\def\ekvd@type@preapptoks#1#2#3%
      {%
        \ekvd@ifnew{}{#2}%
          {%
            \ekvd@assert@filledarg{#3}%
              {%
                \ekvd@newreg#3{toks}%
                \ekvd@ifalso
                  {%
                    \let\ekvd@prot\protected
                    \ekvd@add@val{#2}{#1#3{##1}}{}%
                  }%
                  {\protected\ekvd@long\ekvdef\ekvd@set{#2}{#1#3{##1}}}%
              }%
          }%
      }
    \protected\def\ekvd@t@apptoks{\ekvd@type@preapptoks\toksapp}
    \protected\def\ekvd@t@gapptoks{\ekvd@type@preapptoks\gtoksapp}
    \protected\def\ekvd@t@pretoks{\ekvd@type@preapptoks\tokspre}
    \protected\def\ekvd@t@gpretoks{\ekvd@type@preapptoks\gtokspre}
  }
  {%
    \protected\def\ekvd@type@apptoks#1#2#3%
      {%
        \ekvd@ifnew{}{#2}%
          {%
            \ekvd@assert@filledarg{#3}%
              {%
                \ekvd@newreg#3{toks}%
                \ekvd@ifalso
                  {%
                    \let\ekvd@prot\protected
                    \ekvd@add@val{#2}{#1#3=\expandafter{\the#3##1}}{}%
                  }%
                  {%
                    \protected\ekvd@long\ekvdef\ekvd@set{#2}%
                      {#1#3=\expandafter{\the#3##1}}%
                  }%
              }%
          }%
      }
    \protected\def\ekvd@t@apptoks{\ekvd@type@apptoks{}}
    \protected\def\ekvd@t@gapptoks{\ekvd@type@apptoks\global}
    \newtoks\ekvd@toks
    \protected\def\ekvd@type@pretoks#1#2#3%
      {%
        \ekvd@ifnew{}{#2}%
          {%
            \ekvd@assert@filledarg{#3}%
              {%
                \ekvd@newreg#3{toks}%
                \ekvd@ifalso
                  {%
                    \let\ekvd@prot\protected
                    \ekvd@add@val{#2}%
                      {%
                        \ekvd@toks={##1}%
                        #1#3=\expandafter{\the\expandafter\ekvd@toks\the#3}%
                      }%
                      {}%
                  }%
                  {%
                    \protected\ekvd@long\ekvdef\ekvd@set{#2}%
                      {%
                        \ekvd@toks={##1}%
                        #1#3=\expandafter{\the\expandafter\ekvd@toks\the#3}%
                      }%
                  }%
              }%
          }%
      }
    \protected\def\ekvd@t@pretoks{\ekvd@type@pretoks{}}
    \protected\def\ekvd@t@gpretoks{\ekvd@type@pretoks\global}
  }
%    \end{macrocode}
% \end{macro}
%
% \begin{macro}[internal]
%   {
%     \ekvd@type@reg,\ekvd@t@int,\ekvd@t@eint,\ekvd@t@gint,\ekvd@t@xint,
%     \ekvd@t@dimen,\ekvd@t@edimen,\ekvd@t@gdimen,\ekvd@t@xdimen,
%     \ekvd@t@skip,\ekvd@t@eskip,\ekvd@t@gskip,\ekvd@t@xskip
%   }
% The |\ekvd@type@reg| can handle all the types for which the assignment will
% just be \texttt{\meta{register}=\meta{value}}.
%    \begin{macrocode}
\protected\def\ekvd@type@reg#1#2#3#4#5#6%
  {%
    \ekvd@ifnew{}{#5}%
      {%
        \ekvd@assert@filledarg{#6}%
          {%
            \ekvd@newreg#6{#1}%
            \ekvd@ifalso
              {%
                \let\evkd@prot\protected
                \ekvd@add@val{#5}{#2#6=#3##1#4\relax}{}%
              }%
              {\protected\ekvd@long\ekvdef\ekvd@set{#5}{#2#6=#3##1#4\relax}}%
          }%
      }%
  }
\protected\def\ekvd@t@int{\ekvd@type@reg{count}{}{}{}}
\protected\def\ekvd@t@eint{\ekvd@type@reg{count}{}\numexpr\relax}
\protected\def\ekvd@t@gint{\ekvd@type@reg{count}\global{}{}}
\protected\def\ekvd@t@xint{\ekvd@type@reg{count}\global\numexpr\relax}
\protected\def\ekvd@t@dimen{\ekvd@type@reg{dimen}{}{}{}}
\protected\def\ekvd@t@edimen{\ekvd@type@reg{dimen}{}\dimexpr\relax}
\protected\def\ekvd@t@gdimen{\ekvd@type@reg{dimen}\global{}{}}
\protected\def\ekvd@t@xdimen{\ekvd@type@reg{dimen}\global\dimexpr\relax}
\protected\def\ekvd@t@skip{\ekvd@type@reg{skip}{}{}{}}
\protected\def\ekvd@t@eskip{\ekvd@type@reg{skip}{}\glueexpr\relax}
\protected\def\ekvd@t@gskip{\ekvd@type@reg{skip}\global{}{}}
\protected\def\ekvd@t@xskip{\ekvd@type@reg{skip}\global\glueexpr\relax}
%    \end{macrocode}
% \end{macro}
%
% \begin{macro}[internal]{\ekvd@type@store,\ekvd@t@store,\ekvd@t@gstore}
% The none-expanding |store| types use an |\edef| or |\xdef| and |\unexpanded|
% to be able to also store |#| easily.
%    \begin{macrocode}
\protected\def\ekvd@type@store#1#2#3#4%
  {%
    \ekvd@ifnew{}{#3}%
      {%
        \ekvd@assert@filledarg{#4}%
          {%
            \ekvd@newlet#4\ekv@empty
            \ekvd@ifalso
              {%
                \let\ekvd@prot\protected
                \ekvd@add@val{#3}{#1#4{#2}}{}%
              }%
              {\protected\ekvd@long\ekvdef\ekvd@set{#3}{#1#4{#2}}}%
          }%
      }%
  }
\protected\def\ekvd@t@store{\ekvd@type@store\edef{\ekv@unexpanded{##1}}}
\protected\def\ekvd@t@gstore{\ekvd@type@store\xdef{\ekv@unexpanded{##1}}}
\protected\def\ekvd@t@estore{\ekvd@type@store\edef{##1}}
\protected\def\ekvd@t@xstore{\ekvd@type@store\xdef{##1}}
%    \end{macrocode}
% \end{macro}
%
% \begin{macro}[internal]
%   {
%     \ekvd@type@meta,\ekvd@type@meta@a,\ekvd@type@meta@b,\ekvd@type@meta@c,
%     \ekvd@t@meta,\ekvd@t@nmeta
%   }
% |meta| sets up things such that another instance of |\ekvset| will be run on
% the argument, with the same \set.
%    \begin{macrocode}
\protected\long\def\ekvd@type@meta#1#2#3#4#5#6#7%
  {%
    \ekvd@ifnew{#1}{#6}%
      {%
        \ekvd@assert@filledarg{#7}%
          {%
            \edef\ekvd@tmp{\ekvd@set}%
            \expandafter\ekvd@type@meta@a\expandafter{\ekvd@tmp}{#7}{#2}%
            \ekvd@ifalso
              {\ekv@exparg{#3{#6}}{\ekvd@tmp#4}{#5}}%
              {\csname ekvlet#1\endcsname\ekvd@set{#6}\ekvd@tmp}%
          }%
      }%
  }
\protected\long\def\ekvd@type@meta@a#1#2%
  {%
    \expandafter\ekvd@type@meta@b\expandafter{\ekvset{#1}{#2}}%
  }
\protected\def\ekvd@type@meta@b
  {%
    \expandafter\ekvd@type@meta@c\expandafter
  }
\protected\long\def\ekvd@type@meta@c#1#2%
  {%
    \ekvd@prot\ekvd@long\def\ekvd@tmp#2{#1}%
  }
\protected\def\ekvd@t@meta{\ekvd@type@meta{}{##1}\ekvd@add@val{{##1}}{}}
\protected\def\ekvd@t@nmeta
  {%
    \ekvd@assert@not@long
    \ekvd@type@meta{NoVal}{}\ekvd@add@noval{}\ekvd@assert@not@long@also
  }
%    \end{macrocode}
% \end{macro}
%
% \begin{macro}[internal]
%   {\ekvd@type@smeta,\ekvd@type@smeta@,\ekvd@t@smeta,\ekvd@t@snmeta}
% |smeta| is pretty similar to |meta|, but needs two arguments inside of \val,
% such that the first is the \set\ for which the sub-|\ekvset| and the second is
% the \kv\ list.
%    \begin{macrocode}
\protected\long\def\ekvd@type@smeta#1#2#3#4#5#6#7%
  {%
    \ekvd@ifnew{#1}{#6}%
      {%
        \ekvd@assert@twoargs{#7}%
          {%
            \ekvd@type@meta@a#7{#2}%
            \ekvd@ifalso
              {\ekv@exparg{#3{#6}}{\ekvd@tmp#4}{#5}}%
              {\csname ekvlet#1\endcsname\ekvd@set{#6}\ekvd@tmp}%
          }%
      }%
  }
\protected\def\ekvd@t@smeta{\ekvd@type@smeta{}{##1}\ekvd@add@val{{##1}}{}}
\protected\def\ekvd@t@snmeta
  {%
    \ekvd@assert@not@long
    \ekvd@type@smeta{NoVal}{}\ekvd@add@noval{}\ekvd@assert@not@long@also
  }
%    \end{macrocode}
% \end{macro}
%
% \begin{macro}[internal]
%   {
%     \ekvd@type@choice,
%     \ekvd@populate@choice,\ekvd@populate@choice@,\ekvd@populate@choice@noarg,
%     \ekvd@choice@prefix,\ekvd@choice@prefix@,\ekvd@choice@prefix@done,
%     \ekvd@choice@p@protected,\ekvd@choice@p@protect,
%     \ekvd@choice@p@long,\ekvd@choice@p@long@,
%     \ekvd@t@choice
%   }
% The |choice| type is by far the most complex type, as we have to run a
% sub-parser on the choice-definition list, which should support the |@p@| type
% prefixes as well (but |long| will always throw an error, as they are not
% allowed to be long). |\ekvd@type@choice| will just define the choice-key, the
% handling of the choices definition will be done by |\ekvd@populate@choice|.
%    \begin{macrocode}
\protected\def\ekvd@type@choice#1%
  {%
    \ekvd@assert@not@long
    \ekv@exparg{\ekvd@prot\def\ekvd@tmp##1}%
      {%
        \expandafter\ekvd@h@choice\expandafter
          {\ekvd@choice@name\ekvd@set{#1}{##1}}%
      }%
    \ekvd@ifalso
      {%
        \ekvd@assert@val{#1}%
          {%
            \ekvd@if@not@already@choice{#1}%
              {%
                \ekv@exparg
                  {%
                    \expandafter\ekvd@add@aux
                      \csname\ekv@name\ekvd@set{#1}\endcsname{{##1}}{#1}%
                  }%
                  {\ekvd@tmp{##1}}%
                  {\ekvd@long\ekvdef}\ekvd@assert@not@long@also
              }%
          }%
      }%
      {\ekvlet\ekvd@set{#1}\ekvd@tmp}%
  }
%    \end{macrocode}
% |\ekvd@populate@choice| just uses |\ekvparse| and then gives control to
% |\ekvd@populate@choice@noarg|, which throws an error, and
% |\ekvd@populate@choice@|.
%    \begin{macrocode}
\protected\def\ekvd@populate@choice
  {%
    \ekvparse\ekvd@populate@choice@noarg\ekvd@populate@choice@
  }
\protected\long\def\ekvd@populate@choice@noarg#1%
  {%
    \expandafter\ekvd@err@missing@definition@msg\expandafter{\ekvd@cur : #1}%
  }
%    \end{macrocode}
% |\ekvd@populate@choice@| runs the prefix-test, if there is none we can
% directly define the choice, for that |\ekvd@set@choice| will expand to the
% current choice-key's name, which will have been defined by |\ekvd@t@choice|.
% If there is a prefix run the prefix grabbing routine, which was altered for
% |@type@choice|.
%    \begin{macrocode}
\protected\long\def\ekvd@populate@choice@#1#2%
  {%
    \ekvd@clear@prefixes
    \ekvd@ifspace{#1}%
      {\ekvd@choice@prefix{\ekv@mark#1}\ekv@mark#1\ekv@stop}%
      {%
        \expandafter\edef
          \csname\ekvd@choice@name\ekvd@set\ekvd@set@choice{#1}\endcsname
      }%
      {\unexpanded{#2}}%
  }
\protected\def\ekvd@choice@prefix#1#2
  {%
    \ekv@strip{#2}{\ekvd@choice@prefix@{#1}}\ekv@mark
  }
\protected\def\ekvd@choice@prefix@#1#2#3\ekv@stop
  {%
    \ekv@ifdefined{ekvd@choice@p@#2}%
      {%
        \csname ekvd@choice@p@#2\endcsname
        \ekvd@ifspace{#3}%
          {\ekvd@choice@prefix{#3}#3\ekv@stop}%
          {\ekvd@choice@prefix@done{#3}}%
      }%
      {\ekvd@choice@prefix@done{#1}}%
  }
\protected\def\ekvd@choice@prefix@done#1%
  {%
    \ekvd@prot\expandafter\edef
      \csname
        \ekv@strip{#1}{\ekvd@choice@name\ekvd@set\ekvd@set@choice}%
      \endcsname
  }
\protected\def\ekvd@choice@p@protected{\let\ekvd@prot\protected}
\let\ekvd@choice@p@protect\ekvd@choice@p@protected
\protected\def\ekvd@choice@invalid@p#1\ekvd@ifspace#2%
  {%
    \expandafter\ekvd@choice@invalid@p@\expandafter{\ekv@gobble@mark#2}{#1}%
    \ekvd@ifspace{#2}%
  }
\protected\def\ekvd@choice@invalid@p@#1#2%
  {%
    \expandafter\ekvd@err@no@prefix@msg\expandafter{\ekvd@cur : #2 #1}{#2}%
  }
\protected\def\ekvd@choice@p@long{\ekvd@choice@invalid@p{long}}%
\protected\def\ekvd@choice@p@also{\ekvd@choice@invalid@p{also}}%
\protected\def\ekvd@choice@p@new{\ekvd@choice@invalid@p{new}}%
%    \end{macrocode}
% Finally we're able to set up the |@t@choice| macro, which has to store the
% current choice-key's name, define the key, and parse the available choices.
%    \begin{macrocode}
\protected\long\def\ekvd@t@choice#1#2%
  {%
    \ekvd@ifnew{}{#1}%
      {%
        \ekvd@assert@arg
          {%
            \ekvd@type@choice{#1}%
            \def\ekvd@set@choice{#1}%
            \ekvd@populate@choice{#2}%
          }%
      }%
  }
%    \end{macrocode}
% \end{macro}
%
% \begin{macro}[internal]{\ekvd@t@choice-store,\ekvd@t@choice-enum}
%   These two types define a special kind of |choice| key and are quite similar,
%   the only difference is what the different choices do (hence they use a
%   shared initialisation which differs in the chosen |populate| step).
%    \begin{macrocode}
\protected\long\expandafter\def\csname ekvd@t@choice-store\endcsname
  {\ekvd@type@choicespecial\ekvd@populate@choicestore}
\protected\long\expandafter\def\csname ekvd@t@choice-enum\endcsname
  {\ekvd@type@choicespecial\ekvd@populate@choiceenum}
%    \end{macrocode}
% \begin{macro}[internal]{\ekvd@type@choicespecial}
%   Initialise similar to a |choice| key. The difference is that we require two
%   arguments (which we assert), a macro to store things in, and a |csv|-list
%   containing the allowed values. |#1| is the |populate| macro according to the
%   type used.
%    \begin{macrocode}
\protected\long\def\ekvd@type@choicespecial#1#2#3%
  {%
    \ekvd@ifnew{}{#2}%
      {%
        \ekvd@assert@twoargs{#3}%
          {%
            \ekvd@type@choice{#2}%
            \def\ekvd@set@choice{#2}%
            #1#3%
          }%
      }%
  }
%    \end{macrocode}
% \end{macro}
% \begin{macro}[internal]
%   {\ekvd@populate@choicestore,\ekvd@populate@choicestore@}
%   We initialise the storing macro if it doesn't yet exist, and then we loop
%   over the value list. The |\edef|s with |\unexpanded| are both necessary to
%   be able to store macro parameter tokens (the outer protects at define time,
%   the inner at use time).
%    \begin{macrocode}
\protected\long\def\ekvd@populate@choicestore#1%
  {%
    \ekvd@newlet#1\ekv@empty
    \ekvcsvloop{\ekvd@populate@choicestore@#1}%
  }
\protected\long\def\ekvd@populate@choicestore@#1#2%
  {%
    \protected\expandafter\edef
      \csname\ekvd@choice@name\ekvd@set\ekvd@set@choice{#2}\endcsname
      {\unexpanded{\edef#1{\unexpanded{#2}}}}%
  }
%    \end{macrocode}
% \end{macro}
% \begin{macro}[internal]
%   {\ekvd@populate@choiceenum,\ekvd@populate@choiceenum@}
%   This is similar to the population of a |choice-store| type, but instead of
%   storing the values in a macro this initialises a count and stores the
%   position of the value in the list inside that count (zero-indexed). The
%   space is necessary to terminate the number scanning, which is the reason we
%   use |\@firstofone| (so that the space after the macro name isn't gobbled by
%   \TeX).
%    \begin{macrocode}
\protected\long\def\ekvd@populate@choiceenum#1%
  {%
    \ekvd@newreg#1{count}%
    \def\ekvd@tmp{0}%
    \ekvcsvloop{\ekvd@populate@choiceenum@#1}%
  }
\protected\long\def\ekvd@populate@choiceenum@#1#2%
  {%
    \protected\expandafter\edef
      \csname\ekvd@choice@name\ekvd@set\ekvd@set@choice{#2}\endcsname
      {#1=\@firstofone{\ekvd@tmp} }%
    \edef\ekvd@tmp{\the\numexpr\ekvd@tmp+1\relax}%
  }
%    \end{macrocode}
% \end{macro}
% \end{macro}
%
% \begin{macro}[internal]{\ekvd@t@unknown-choice}
%    \begin{macrocode}
\protected\long\expandafter\def\csname ekvd@t@unknown-choice\endcsname#1#2%
  {%
    \ekvd@assert@new@for@name{\ekvd@unknown@choice@name\ekvd@set{#1}}%
      {%
        \ekvd@assert@arg
          {%
            \ekvd@assert@not@long
            \ekvd@assert@not@also
            \ekvd@prot\expandafter
            \def\csname\ekvd@unknown@choice@name\ekvd@set{#1}\endcsname##1{#2}%
          }%
      }%
  }
%    \end{macrocode}
% \end{macro}
%
% \begin{macro}[internal]
%   {\ekvd@t@unknown,\ekvd@type@unknown@code,\ekvd@type@unknown@noval}
%   The |unknown| type has different subtypes which would be the key names for
%   other types. It is first checked whether that subtype is defined, if it
%   isn't throw an error, else use that subtype.
%    \begin{macrocode}
\protected\long\def\ekvd@t@unknown#1#2%
  {%
    \ekv@ifdefined{ekvd@type@unknown@\detokenize{#1}}%
      {\csname ekvd@type@unknown@\detokenize{#1}\endcsname{#2}}%
      \ekvd@err@misused@unknown
  }
%    \end{macrocode}
%   The |unknown noval| type can use |\ekvdefunknownNoVal| directly (after
%   asserting some prefixes).
%    \begin{macrocode}
\protected\long\def\ekvd@type@unknown@noval#1%
  {%
    \ekvd@assert@new@for@name{\ekv@name\ekvd@set{}uN}%
      {%
        \ekvd@assert@arg
          {%
            \ekvd@assert@not@also
            \ekvd@assert@not@long
            \ekvd@prot\ekvdefunknownNoVal\ekvd@set{#1}%
          }%
      }%
  }
%    \end{macrocode}
%   The |unknown code| type uses some trickery during the definition in order to
%   swap out |#1| and |#2| in the user supplied definition. This is done via a
%   temporary macro that stores the definition but gets the parameter numbers
%   reversed while the real definition is done.
%    \begin{macrocode}
\protected\long\def\ekvd@type@unknown@code#1%
  {%
    \ekvd@assert@new@for@name{\ekv@name\ekvd@set{}u}%
      {%
        \ekvd@assert@arg
          {%
            \ekvd@assert@not@also
            \begingroup
              \def\ekvd@tmp##1##2{#1}%
              \ekv@exparg
              {%
                \endgroup
                \ekvd@prot\ekvd@long\ekvdefunknown\ekvd@set
              }%
              {\ekvd@tmp{##2}{##1}}%
          }%
      }%
  }
%    \end{macrocode}
% \end{macro}
%
% \begin{macro}[internal]
%   {
%     \ekvd@type@unknown@redirect,
%     \ekvd@type@unknown@redirect-code,
%     \ekvd@type@unknown@redirect-noval
%   }
%   The |unknown redirect| types also just forward to |\ekvredirectunknown|
%   after asserting some prefixes.
%    \begin{macrocode}
\protected\edef\ekvd@type@unknown@redirect#1%
  {%
    \expandafter\noexpand\csname ekvd@type@unknown@redirect-code\endcsname{#1}%
    \expandafter\noexpand\csname ekvd@type@unknown@redirect-noval\endcsname{#1}%
  }
\protected\expandafter\def\csname ekvd@type@unknown@redirect-code\endcsname#1%
  {%
    \ekvd@assert@new@for@name{\ekv@name\ekvd@set{}u}%
      {%
        \ekvd@assert@arg
          {%
            \ekvd@assert@not@also
            \ekvd@assert@not@protected
            \expandafter\ekvredirectunknown\expandafter{\ekvd@set}{#1}%
          }%
      }%
  }
\protected\expandafter\def\csname ekvd@type@unknown@redirect-noval\endcsname#1%
  {%
    \ekvd@assert@new@for@name{\ekv@name\ekvd@set{}uN}%
      {%
        \ekvd@assert@arg
          {%
            \ekvd@assert@not@also
            \ekvd@assert@not@protected
            \ekvd@assert@not@long
            \expandafter\ekvredirectunknownNoVal\expandafter{\ekvd@set}{#1}%
          }%
      }%
  }
%    \end{macrocode}
% \end{macro}
%
% \subsubsection{Key Type Helpers}
% There are some keys that might need helpers during their execution (not during
% their definition, which are gathered as |@type@| macros). These helpers are
% named |@h@|.
%
% \begin{macro}[internal]{\ekvd@h@choice,\ekvd@h@choice@}
% The |choice| helper will just test whether the given choice was defined, if
% not throw an error expandably, else call the macro which stores the code for
% this choice.
%    \begin{macrocode}
\def\ekvd@h@choice#1%
  {%
    \expandafter\ekvd@h@choice@
      \csname\ifcsname#1\endcsname#1\else relax\fi\endcsname
      {#1}%
  }
\def\ekvd@h@choice@#1#2%
  {%
    \ifx#1\relax
      \ekvd@err@choice@invalid{#2}%
      \expandafter\@gobble
    \fi
    #1%
  }
%    \end{macrocode}
% \end{macro}
%
%
% \subsubsection{Handling \texttt{also}}
%
% \begin{macro}[internal]
%   {\ekvd@add@val,\ekvd@add@noval,\ekvd@add@aux,\ekvd@add@aux@}
%    \begin{macrocode}
\protected\long\def\ekvd@add@val#1#2#3%
  {%
    \ekvd@assert@val{#1}%
      {%
        \expandafter\ekvd@add@aux\csname\ekv@name\ekvd@set{#1}\endcsname{{##1}}%
          {#1}{#2}{\ekvd@long\ekvdef}{#3}%
      }%
  }
\protected\long\def\ekvd@add@noval#1#2#3%
  {%
    \ekvd@assert@noval{#1}%
      {%
        \expandafter\ekvd@add@aux\csname\ekv@name\ekvd@set{#1}N\endcsname{}%
          {#1}{#2}\ekvdefNoVal{#3}%
      }%
  }
\protected\long\def\ekvd@add@aux#1#2%
  {%
    \ekvd@extract@prefixes#1%
    \expandafter\ekvd@add@aux@\expandafter{#1#2}%
  }
\protected\long\def\ekvd@add@aux@#1#2#3#4#5%
  {%
    #5%
    \ekvd@prot#4\ekvd@set{#2}{#1#3}%
  }
%    \end{macrocode}
% \end{macro}
%
% \begin{macro}[internal]
%   {
%     \ekvd@extract@prefixes,\ekvd@extract@prefixes@,
%     \ekvd@extract@prefixes@long,\ekvd@extract@prefixes@prot
%   }
% This macro checks which prefixes were used for the definition of a macro and
% sets |\ekvd@long| and |\ekvd@prot| accordingly.
%    \begin{macrocode}
\protected\def\ekvd@extract@prefixes#1%
  {%
    \expandafter\ekvd@extract@prefixes@\meaning#1\ekvd@stop
  }
%    \end{macrocode}
% In the following definition |#1| will get replaced by |macro:|, |#2| by
% |\long| and |#3| by |\protected| (in each, all tokens will have category
% other). This allows us to parse the |\meaning| of a macro for those strings.
%    \begin{macrocode}
\protected\def\ekvd@extract@prefixes@#1#2#3%
  {%
    \protected\def\ekvd@extract@prefixes@##1#1##2\ekvd@stop
      {%
        \ekvd@extract@prefixes@long
          ##1\ekvd@mark\@firstofone#2\ekvd@mark\@gobble\ekvd@stop
          {\let\ekvd@long\long}%
        \ekvd@extract@prefixes@prot
          ##1\ekvd@mark\@firstofone#3\ekvd@mark\@gobble\ekvd@stop
          {\let\ekvd@prot\protected}%
      }%
    \protected\def\ekvd@extract@prefixes@long##1#2##2\ekvd@mark##3##4\ekvd@stop
      {##3}%
    \protected\def\ekvd@extract@prefixes@prot##1#3##2\ekvd@mark##3##4\ekvd@stop
      {##3}%
  }
%    \end{macrocode}
% We use a temporary macro to expand the three arguments of
% |\ekvd@extract@prefixes@|, which will set up the real meaning of itself and
% the parsing for |\long| and |\protected|.
%    \begin{macrocode}
\begingroup
\edef\ekvd@tmp
  {%
    \endgroup
    \ekvd@extract@prefixes@
      {\detokenize{macro:}}%
      {\string\long}%
      {\string\protected}%
  }
\ekvd@tmp
%    \end{macrocode}
% \end{macro}
%
%
% \subsubsection{Tests}
%
% \begin{macro}[internal]{\ekvd@newlet,\ekvd@newreg}
% These macros test whether a control sequence is defined, if it isn't they
% define it, either via |\let| or via the correct \cs[no-index]{new\meta{reg}}.
%    \begin{macrocode}
\protected\def\ekvd@newlet#1#2%
  {%
    \ifdefined#1\ekv@fi@gobble\fi\@firstofone{\let#1#2}%
  }
\protected\def\ekvd@newreg#1#2%
  {%
    \ifdefined#1\ekv@fi@gobble\fi\@firstofone{\csname new#2\endcsname#1}%
  }
%    \end{macrocode}
% \end{macro}
%
% \begin{macro}[internal]
%   {\ekvd@assert@twoargs,\ekvd@ifnottwoargs,\ekvd@ifempty@gtwo}
% A test for exactly two tokens can be reduced for an empty-test after gobbling
% two tokens, in the case that there are fewer tokens than two in the argument,
% only macros will be gobbled that are needed for the true branch, which doesn't
% hurt, and if there are more this will not be empty.
%    \begin{macrocode}
\long\def\ekvd@assert@twoargs#1%
  {%
    \ekvd@ifnottwoargs{#1}{\ekvd@err@missing@definition}%
  }
\long\def\ekvd@ifnottwoargs#1%
  {%
    \ekvd@ifempty@gtwo#1\ekv@ifempty@B
      \ekv@ifempty@false\ekv@ifempty@A\ekv@ifempty@B\@firstoftwo
  }
\long\def\ekvd@ifempty@gtwo#1#2{\ekv@ifempty@\ekv@ifempty@A}
%    \end{macrocode}
% \end{macro}
%
% \begin{macro}[internal]
%   {
%     \ekvd@assert@val,\ekvd@assert@val@,\ekvd@assert@noval,\ekvd@assert@noval@,
%     \ekvd@extract@args,\ekvd@extracted@args,\ekvd@one@arg@string
%   }
% Assert that a given key is defined as a value taking key or a |NoVal| key with
% the correct argument structure, respectively.
%    \begin{macrocode}
\protected\def\ekvd@assert@val#1%
  {%
    \ekvifdefined\ekvd@set{#1}%
      {\expandafter\ekvd@assert@val@\csname\ekv@name\ekvd@set{#1}\endcsname}%
      {%
        \ekvifdefinedNoVal\ekvd@set{#1}%
          \ekvd@err@add@val@on@noval
          {\ekvd@err@undefined@key{#1}}%
        \@gobble
      }%
  }
\protected\def\ekvd@assert@val@#1%
  {%
    \expandafter\ekvd@extract@args\meaning#1\ekvd@stop
    \unless\ifx\ekvd@extracted@args\ekvd@one@arg@string
      \ekvd@err@unsupported@arg
    \fi
    \@firstofone
  }%
\protected\def\ekvd@assert@noval#1%
  {%
    \ekvifdefinedNoVal\ekvd@set{#1}%
      {\expandafter\ekvd@assert@noval@\csname\ekv@name\ekvd@set{#1}N\endcsname}%
      {%
        \ekvifdefined\ekvd@set{#1}%
          \ekvd@err@add@noval@on@val
          {\ekvd@err@undefined@key{#1}}%
        \@gobble
      }%
  }
\protected\def\ekvd@assert@noval@#1%
  {%
    \expandafter\ekvd@extract@args\meaning#1\ekvd@stop
    \unless\ifx\ekvd@extracted@args\ekv@empty
      \ekvd@err@unsupported@arg
    \fi
    \@firstofone
  }
\protected\def\ekvd@extract@args#1%
  {%
    \protected\def\ekvd@extract@args##1#1##2->##3\ekvd@stop
      {\def\ekvd@extracted@args{##2}}%
  }
\expandafter\ekvd@extract@args\expandafter{\detokenize{macro:}}
\edef\ekvd@one@arg@string{\string#1}
%    \end{macrocode}
% \end{macro}
%
% \begin{macro}[internal]{\ekvd@assert@arg,\ekvd@assert@arg@msg,\ekvd@ifnoarg}
% There is no need to actually define |\ekvd@ifnoarg| here, as it will be set by
% either |\ekvd@arg| or |\ekvd@noarg|.
%    \begin{macrocode}
\def\ekvd@assert@arg{\ekvd@ifnoarg\ekvd@err@missing@definition}
\long\def\ekvd@assert@arg@msg#1%
  {%
    \ekvd@ifnoarg{\ekvd@err@missing@definition@msg{#1}}%
  }
%    \end{macrocode}
% \end{macro}
%
% \begin{macro}[internal]{\ekvd@assert@filledarg,\ekvd@ifnoarg@or@empty}
%    \begin{macrocode}
\long\def\ekvd@assert@filledarg#1%
  {%
    \ekvd@ifnoarg@or@empty{#1}\ekvd@err@missing@definition
  }
\long\def\ekvd@ifnoarg@or@empty#1%
  {%
    \ekvd@ifnoarg
      \@firstoftwo
      {\ekv@ifempty{#1}}%
  }
%    \end{macrocode}
% \end{macro}
%
% \begin{macro}[internal]
%   {
%     \ekvd@assert@not@long,\ekvd@assert@not@protected,\ekvd@assert@not@also
%     \ekvd@assert@not@long@also,\ekvd@assert@not@protected@also,
%     \ekvd@assert@new,\ekvd@assert@not@new
%   }
% Some key-types don't want to be |also|, |\long| or |\protected|, so we provide
% macros to test this and throw an error, this could be silently ignored but now
% users will learn to not use unnecessary stuff which slows the compilation
% down.
%    \begin{macrocode}
\def\ekvd@assert@not@long{\ifx\ekvd@long\long\ekvd@err@no@prefix{long}\fi}
\def\ekvd@assert@not@protected
  {\ifx\ekvd@prot\protected\ekvd@err@no@prefix{protected}\fi}
\def\ekvd@assert@not@also{\ekvd@ifalso{\ekvd@err@no@prefix{also}}{}}
\def\ekvd@assert@not@long@also
  {\ifx\ekvd@long\long\ekvd@err@no@prefix@also{long}\fi}
\def\ekvd@assert@not@protected@also
  {\ifx\ekvd@prot\protected\ekvd@err@no@prefix@also{protected}\fi}
\def\ekvd@assert@new#1#2%
  {\csname ekvifdefined#1\endcsname\ekvd@set{#2}{\ekvd@err@not@new}}
\def\ekvd@assert@not@new
  {\ifx\ekvd@ifnew\ekvd@assert@new\ekvd@err@no@prefix{new}\fi}
\def\ekvd@assert@new@for@name#1%
  {%
    \ifx\ekvd@ifnew\ekvd@assert@new
      \ekv@fi@firstoftwo
    \fi
    \@secondoftwo
      {\ekv@ifdefined{#1}\ekvd@err@not@new}%
      \@firstofone
  }
%    \end{macrocode}
% \end{macro}
%
% \begin{macro}[internal]
%   {
%     \ekvd@if@not@already@choice, \ekvd@if@not@already@choice@a,
%     \ekvd@if@not@already@choice@b
%   }
% It is bad to use |also| on a key that already contains a |choice|, as both
% choices would share the same valid values and thus lead to each callback being
% used twice. The following is a rudimentary test against this.
%    \begin{macrocode}
\protected\def\ekvd@if@not@already@choice#1%
  {%
    \expandafter\ekvd@if@not@already@choice@a
      \csname\ekv@name\ekvd@set{#1}\endcsname
      {}\ekvd@h@choice\ekvd@stop
  }
\protected\def\ekvd@if@not@already@choice@a
  {%
    \expandafter\ekvd@if@not@already@choice@b
  }
\long\protected\def\ekvd@if@not@already@choice@b#1\ekvd@h@choice#2\ekvd@stop
  {%
    \ekv@ifempty{#2}\@firstofone\@gobble
  }
%    \end{macrocode}
% \end{macro}
%
% \begin{macro}[internal]{\ekvd@ifspace,\ekvd@ifspace@}
% Yet another test which can be reduced to an if-empty, this time by gobbling
% everything up to the first space.
%    \begin{macrocode}
\long\def\ekvd@ifspace#1%
  {%
    \ekvd@ifspace@#1 \ekv@ifempty@B
      \ekv@ifempty@false\ekv@ifempty@A\ekv@ifempty@B\@firstoftwo
  }
\long\def\ekvd@ifspace@#1 % keep this space
  {%
    \ekv@ifempty@\ekv@ifempty@A
  }
%    \end{macrocode}
% \end{macro}
%
% \subsubsection{Messages}
%
% Most messages of \expkvd\ are not expandable, since they only appear during
% key-definition, which is not expandable anyway.
%
% \begin{macro}[internal]
%   {
%     \ekvd@errm,\ekvd@err@missing@definition,\ekvd@err@missing@definition@msg,
%     \ekvd@err@missing@type,\ekvd@err@undefined@prefix,\ekvd@err@undefined@key,
%     \ekvd@err@no@prefix,\ekvd@err@no@prefix@msg,\ekvd@err@no@prefix@also,
%     \ekvd@err@add@val@on@noval,\ekvd@err@add@noval@on@val,
%     \ekvd@err@unsupported@arg,\ekvd@err@not@new
%   }
% The non-expandable error messages are boring, so here they are:
%    \begin{macrocode}
\protected\def\ekvd@errm#1{\errmessage{expkv-def Error: #1}}
\protected\def\ekvd@err@missing@definition
  {\ekvd@errm{Missing definition for key `\ekvd@cur'}}
\protected\def\ekvd@err@missing@definition@msg#1%
  {\ekvd@errm{Missing definition for key `\ekv@unexpanded{#1}'}}
\protected\def\ekvd@err@missing@type
  {\ekvd@errm{Missing type prefix for key `\ekvd@cur'}}
\protected\def\ekvd@err@undefined@prefix#1%
  {%
    \ekvd@errm
      {%
        Undefined prefix `\ekv@unexpanded{#1}' found while processing
        `\ekvd@cur'%
      }%
  }
\protected\def\ekvd@err@undefined@key#1%
  {%
    \ekvd@errm
      {Undefined key `\ekv@unexpanded{#1}' found while processing `\ekvd@cur'}%
  }
\protected\def\ekvd@err@undefined@noval#1%
  {%
    \ekvd@errm
      {%
        Undefined noval key `\unexpanded{#1}' found while processing
        `\ekvd@cur'%
      }%
  }
\protected\def\ekvd@err@no@prefix#1%
  {\ekvd@errm{prefix `#1' not accepted in `\ekvd@cur'}}
\protected\def\ekvd@err@no@prefix@msg#1#2%
  {\ekvd@errm{prefix `#2' not accepted in `\ekv@unexpanded{#1}'}}
\protected\def\ekvd@err@no@prefix@also#1%
  {\ekvd@errm{`\ekvd@cur' not allowed with a `#1' key}}
\protected\def\ekvd@err@add@val@on@noval
  {\ekvd@errm{`\ekvd@cur' not allowed with a NoVal key}}
\protected\def\ekvd@err@add@noval@on@val
  {\ekvd@errm{`\ekvd@cur' not allowed with a value taking key}}
\protected\def\ekvd@err@unsupported@arg\fi\@firstofone#1%
  {%
    \fi
    \ekvd@errm
      {%
        Existing key-macro has the unsupported argument string
        `\ekvd@extracted@args' for key `\ekvd@cur'%
      }%
  }
\protected\def\ekvd@err@not@new
  {\ekvd@errm{The key for `\ekvd@cur' is already defined}}
\protected\long\def\ekvd@err@misused@unknown
  {\ekvd@errm{Misuse of the unknown type found while processing `\ekvd@cur'}}
%    \end{macrocode}
% \end{macro}
%
% \begin{macro}[internal]
%   {
%     \ekvd@err@choice@invalid,\ekvd@err@choice@invalid@,\ekvd@choice@name,
%     \ekvd@unknown@choice@name
%   }
% |\ekvd@err@choice@invalid| will have to use this mechanism to throw its
% message. Also we have to retrieve the name parts of the choice in an easy way,
% so we use parentheses of catcode 8 here, which should suffice in most cases to
% allow for a correct separation.
%    \begin{macrocode}
\def\ekvd@err@choice@invalid#1%
  {%
    \ekvd@err@choice@invalid@#1%
  }
\begingroup
\catcode40=8
\catcode41=8
\@firstofone{\endgroup
\def\ekvd@choice@name#1#2#3%
  {%
    ekvd#1(#2)\detokenize{#3}%
  }
\def\ekvd@unknown@choice@name#1#2%
  {%
    ekvd:u:#1(#2)%
  }
\def\ekvd@err@choice@invalid@ ekvd#1(#2)\detokenize#3%
  {%
    \ekv@ifdefined{\ekvd@unknown@choice@name{#1}{#2}}%
      {\csname\ekvd@unknown@choice@name{#1}{#2}\endcsname{#3}}%
      {\ekvd@err{invalid choice `#3' for `#2' in set `#1'}}%
  }
}
%    \end{macrocode}
% \end{macro}
%
% \begin{macro}[internal]{\ekvd@err}
% The expandable error messages use |\ekvd@err|, which is just like |\ekv@err|
% from \expkv. It uses a runaway argument to start the error message.
%    \begin{macrocode}
\ekv@exparg{\long\def\ekvd@err#1}{\ekverr{expkv-def}{#1}}
%    \end{macrocode}
% \end{macro}
%
% Now everything that's left is to reset the category code of |@|.
%    \begin{macrocode}
\catcode`\@=\ekvd@tmp
%    \end{macrocode}
%
% \gobbledocstriptag
%</tex>
%^^A=<<
%
%^^A=<<
%
% \end{implementation}^^A=<<
%
% \clearpage
% \PrintIndex
%
\endinput
%
^^A vim: ft=tex fdm=marker fmr=>>=,=<<
}{}
\ProvidesFile{expkv-def.dtx}
  [\csname ekvdDate\endcsname\ a key-defining frontend for expkv]
\PassOptionsToPackage{full}{textcomp}
\documentclass{l3doc}
\RequirePackage[oldstylenums,nott]{kpfonts}
\input{glyphtounicode}
\pdfgentounicode=1
\RequirePackage{listings}
\RequirePackage{booktabs}
\RequirePackage{array}
\RequirePackage{collcell}
\RequirePackage{siunitx}
\RequirePackage{xcolor}
\RequirePackage{caption}
\RequirePackage{microtype}
\RequirePackage{accsupp}
\lstset
  {
    ,flexiblecolumns=false
    ,basewidth=.53em
    ,gobble=2
    ,basicstyle=\fontfamily{jkp}\itshape
    ,morekeywords=^^A
      {^^A
        \ekvdefinekeys
      }
    ,morecomment=[l]\%
    ,commentstyle=\color[gray]{0.4}
    ,literate={\{}{{\CodeSymbol\{}}{1}
              {\}}{{\CodeSymbol\}}}{1}
    ^^A,literate=*{<key>}{\key}{4}{<set>}{\set}{4}
  }
\newcommand*\CodeSymbol[1]{\textbf{#1}}
\RequirePackage{randtext}
\let\metaORIG\meta
\protected\def\meta #1{\texttt{\metaORIG{#1}}}
\renewcommand*\thefootnote{\fnsymbol{footnote}}
\definecolor{expkvred}{HTML}{9F393D}
\colorlet{expkvgrey}{black!75}
\makeatletter
\newcommand*\expkv
  {^^A
    \texorpdfstring
      {^^A
        \mbox
          {^^A
            \BeginAccSupp{ActualText=expkv}^^A
            \href{https://github.com/Skillmon/tex_expkv}
              {^^A
                \rmfamily
                \bfseries
                {\color{expkvgrey}e\kern-.05em x\kern-.05em}^^A
                \lower.493ex
                  \hbox{{\color{expkvgrey}P}\kern-.1em{\color{expkvred}k}}^^A
                \kern-.18em{\color{expkvred}v}^^A
              }^^A
            \EndAccSupp{}^^A
          }^^A
      }
      {expkv}^^A
  }
\newcommand*\expkvd
  {^^A
    \texorpdfstring
      {^^A
        \mbox
          {^^A
            \BeginAccSupp{ActualText=expkv-def}^^A
            \href{https://github.com/Skillmon/tex_expkv-def}
              {^^A
                \rmfamily
                \bfseries
                {\color{expkvgrey}e\kern-.05em x\kern-.05em}^^A
                \lower.493ex
                  \hbox{{\color{expkvgrey}P}\kern-.1em{\color{expkvred}k}}^^A
                \kern-.18em{\color{expkvred}v}^^A
                {\color{expkvgrey}^^A
                  \kern.05em\rule[-.1ex]{.08em}{1.2ex}\kern.05em\textsc{def}^^A
                }^^A
              }^^A
            \EndAccSupp{}^^A
          }^^A
      }
      {expkv-def}^^A
  }
\newcommand\kv{\meta{key}=\meta{value}}
\newcommand\key{\meta{key}}
\newcommand\val{\meta{value}}
\newcommand\set{\meta{set}}
\newcommand\enfprot{\textcolor{black}{protected}}
\newcommand\allprot{\textcolor{gray}{protected}}
\newcommand\notprot{\textcolor{red!80!black}{protected}}
\newcommand\enflong{\textcolor{black}{long}}
\newcommand\alllong{\textcolor{gray}{long}}
\newcommand\notlong{\textcolor{red!80!black}{long}}
\newcommand\prefixes[2]
  {^^A
    \hfill
    \ifcase\numexpr#1\relax\or
      \enfprot\or
      \allprot\or
      \notprot\fi\space
    \ifcase\numexpr#2\relax\or
      \enflong\or
      \alllong\or
      \notlong\fi
  }
\hypersetup{linkcolor=red!80!black,urlcolor=purple!80!black}
\DoNotIndex{\def,\edef,\,,\=,\begingroup,\catcode,\chardef,\csname,\endcsname}
\DoNotIndex{\endgroup,\endinput,\errmessage,\expandafter,\input,\let,\long}
\DoNotIndex{\noexpand,\protected,\ProvidesFile,\ProvidesPackage,\relax,\space}
\DoNotIndex{\@,\unexpanded,\string,\expanded,\dimexpr,\global,\glueexpr,\hbox}
\DoNotIndex{\numexpr,\RequirePackage,\setbox,\the,\unless,\xdef}
\DoNotIndex{\ifcsname}
\DoNotIndex{\ifx}
\DoNotIndex{\ifdefined}
\DoNotIndex{\iffalse}
\DoNotIndex{\iftrue}
\DoNotIndex{\else}
\DoNotIndex{\fi}
\@gobble\fi ^^A ignoring \ifx, \ifcsname, etc., but only one \fi
\@gobble\fi ^^A ignoring \ifx, \ifcsname, etc., but only one \fi
\@gobble\fi ^^A ignoring \ifx, \ifcsname, etc., but only one \fi
\@gobble\fi ^^A ignoring \ifx, \ifcsname, etc., but only one \fi
\@ifdefinable\gobbledocstriptag{\def\gobbledocstriptag#1>{}}
\makeatother
\begin{document}
  \title
    {^^A
      \texorpdfstring
        {^^A
          \huge\expkvd
          \\[\medskipamount]
          \Large a key-defining frontend for \expkv
        }{expkv-def - a key-defining frontend for expkv}^^A
    }
  \date{\ekvdDate\space v\ekvdVersion}
  \author{Jonathan P. Spratte\thanks{\protect\randomize{jspratte@yahoo.de}}}
  \DocInput{expkv-def.dtx}
\end{document}
%</driver>^^A=<<
% \fi
%
% \maketitle
% \renewcommand*\thefootnote{\arabic{footnote}}
%
% \begin{abstract}
% \noindent\parfillskip=0pt
% \expkvd\ provides a small \kv\ interface to define keys for \expkv. Key-types
% are declared using prefixes, similar to static typed languages. The stylised
% name is \expkvd\ but the files use \file{expkv-def}, this is due to CTAN-rules
% which don't allow \string| in package names since that is the pipe symbol in *nix
% shells.
% \end{abstract}
%
% \tableofcontents
%
% \begin{documentation}^^A>>=
%
% \section{Documentation}
%
% Since the trend for the last couple of years goes to defining keys for a \kv\
% interface using a \kv\ interface, I thought that maybe providing such an
% interface for \expkv\ will make it more attractive for actual use, besides its
% unique selling points of being fully expandable, and fast and reliable. But at
% the same time I don't want to widen \expkv's initial scope. So here it is
% \expkvd, go define \kv\ interfaces with \kv\ interfaces.
%
% Unlike many of the other established \kv\ interfaces to define keys, \expkvd\
% works using prefixes instead of suffixes (\emph{e.g.}, |.tl_set:N| of
% \pkg{l3keys}) or directory like handlers (\emph{e.g.}, |/.store in| of
% \pkg{pgfkeys}). This was decided as a personal preference, more over in \TeX\
% parsing for the first space is way easier than parsing for the last one.
% \expkvd's prefixes are sorted into two categories |p|-type, which are
% equivalent to \TeX's prefixes like |\long|, and |t|-type defining the type of
% the key. For a description of the available |p|-prefixes take a look at
% \autoref{sec:prefix:p}, the |t|-prefixes are described in
% \autoref{sec:prefix:t}.
%
% \expkvd\ is usable as generic code and as a \LaTeX\ package. It'll
% automatically load \expkv\ in the same mode as well. To use it, just use one
% of
% \begin{lstlisting}
% \usepackage{expkv-def} % LaTeX
% \input expkv-def       % plainTeX
% \end{lstlisting}
%
% \subsection{Macros}
% Apart from version and date containers there is only a single user-facing
% macro, and that should be used to define keys.
%
% \begin{function}{\ekvdefinekeys}
%   \begin{syntax}
%     \cs{ekvdefinekeys}\marg{set}\{\kv, \ldots\}
%   \end{syntax}
%   In \set, define \key\ to have definition \val. The general syntax for \key\
%   should be
%   \begin{quote}\ttfamily\small
%     \meta{prefix} \meta{name}
%   \end{quote}
%   Where \meta{prefix} is a space separated list of optional |p|-type prefixes
%   followed by one |t|-type prefix. The syntax of \val\ is dependent on the
%   used |t|-prefix.
% \end{function}
%
% \begin{function}{\ekvdDate,\ekvdVersion}
%   These two macros store the version and date of the package.
% \end{function}
%
% \subsection{Prefixes}
% As already said there are |p|-prefixes and |t|-prefixes. Not every |p|-prefix
% is allowed for all |t|-prefixes.
%
% \subsubsection{\texttt{p}-Prefixes}\label{sec:prefix:p}
% The |p|-type prefixes are pretty simple by nature, so their description is
% pretty simple. They affect the \key\ at use-time, so omitting |long| doesn't
% mean that a \meta{definition} can't contain a |\par| token, only that the
% \key\ will not accept a |\par| in \val.
% \begin{function}{protected,protect}
%   The following key will be defined |\protected|. Note that key-types which
%   can't be defined expandable will always use |\protected|.
% \end{function}
%
% \begin{function}{long}
%   The following key will be defined |\long|.
% \end{function}
%
% \subsubsection{\texttt{t}-Prefixes}\label{sec:prefix:t}
% Since the |p|-type prefixes apply to some of the |t|-prefixes automatically
% but sometimes one might be disallowed we need some way to highlight this
% behaviour. In the following an enforced prefix will be printed black
% (\texttt{\enfprot}), allowed prefixes will be grey (\texttt{\allprot}), and
% disallowed prefixes will be red (\texttt{\notprot}). This will be put
% flush-right in the syntax showing line.
%
% \begin{function}{code,ecode}
%   \begin{syntax}
%     code \key\ = \marg{definition} \prefixes22
%   \end{syntax}
%   Define \key\ to expand to \meta{definition}. The \key\ will require a \val\
%   for which you can use |#1| inside \meta{definition}. The |ecode| variant
%   will fully expand \meta{definition} inside an |\edef|.
% \end{function}
%
% \begin{function}{noval,enoval}
%   \begin{syntax}
%     noval \key\ = \marg{definition} \prefixes23
%   \end{syntax}
%   The |noval| type defines \key\ to expand to \meta{definition}. The \key\
%   will not take a \val. |enoval| fully expands \meta{definition} inside an
%   |\edef|.
% \end{function}
%
% \begin{function}{default,qdefault,edefault}
%   \begin{syntax}
%     default \key\ = \marg{definition} \prefixes23
%   \end{syntax}
%   This serves to place a default \val\ for a \key\ that takes an argument, the
%   \key\ can be of any argument-grabbing kind, and when used without a \val\
%   it will be passed \meta{definition} instead. The |qdefault| variant will
%   expand the \key's code once, so will be slightly quicker, but not change if
%   you redefine \key. The |edefault| on the other hand fully expands the
%   \key-code with \meta{definition} as its argument inside of an |\edef|.
% \end{function}
%
% \begin{function}{initial}
%   \begin{syntax}
%     initial \key\ = \{\val\} \prefixes33
%   \end{syntax}
%   With |initial| you can set an initial \val\ for an already defined argument
%   taking \key. It'll just call the key-macro of \key\ and pass it \val.
% \end{function}
%
% \begin{function}{bool,gbool,boolTF,gboolTF}
%   \begin{syntax}
%     bool \key\ = \meta{cs} \prefixes23
%   \end{syntax}
%   The \meta{cs} should be a single control sequence, such as |\iffoo|.
%   This will define \key\ to be a boolean key, which only takes the values
%   |true| or |false| and will throw an error for other values. If the key is
%   used without a \val\ it'll have the same effect as if you use
%   \texttt{\key=true}. |bool| and |gbool| will behave like \TeX-ifs so either
%   be |\iftrue| or |\iffalse|. The |boolTF| and |gboolTF| variants will both
%   take two arguments and if true the first will be used else the second, so
%   they are always either |\@firstoftwo| or |\@secondoftwo|. The variants with
%   a leading |g| will set the control sequence globally, the others locally. If
%   \meta{cs} is not yet defined it'll be initialised as the |false| version.
%   Note that the initialisation is \emph{not} done with |\newif|, so you will
%   not be able to do |\footrue| outside of the \kv\ interface, but you could
%   use |\newif| yourself. Even if the \key\ will not be |\protected| the
%   commands which execute the |true| or |false| choice will be, so the usage
%   should be safe in an expansion context (\emph{e.g.}, you can use
%   \texttt{edefault \key\ = false} without an issue to change the default
%   behaviour to execute the |false| choice).
% \end{function}
%
% \begin{function}{store,estore,gstore,xstore}
%   \begin{syntax}
%     store \key\ = \meta{cs} \prefixes12
%   \end{syntax}
%   The \meta{cs} should be a single control sequence, such as |\foo|.
%   This will define \key\ to store \val\ inside of the control sequence. If
%   \meta{cs} isn't yet defined it will be initialised as empty. The variants
%   behave similarly to their |\def|, |\edef|, |\gdef|, and |\xdef|
%   counterparts, but |store| and |gstore| will allow you to store macro
%   parameters inside of them by using |\unexpanded|.
% \end{function}
%
% \begin{function}{int,eint,gint,xint}
%   \begin{syntax}
%     int \key\ = \meta{cs} \prefixes12
%   \end{syntax}
%   The \meta{cs} should be a single control sequence, such as |\foo|.
%   An |int| key will be a \TeX-count register. If \meta{cs} isn't defined yet,
%   |\newcount| will be used to initialise it. The |eint| and |xint| versions
%   will use |\numexpr| to allow basic computations in their \val. The |gint|
%   and |xint| variants set the register globally.
% \end{function}
%
% \begin{function}{dimen,edimen,gdimen,xdimen}
%   \begin{syntax}
%     dimen \key\ = \meta{cs} \prefixes12
%   \end{syntax}
%   The \meta{cs} should be a single control sequence, such as |\foo|.
%   This is just like |int| but uses a dimen register, |\newdimen| and
%   |\dimexpr| instead.
% \end{function}
%
% \begin{function}{skip,eskip,gskip,xskip}
%   \begin{syntax}
%     dimen \key\ = \meta{cs} \prefixes12
%   \end{syntax}
%   The \meta{cs} should be a single control sequence, such as |\foo|.
%   This is just like |int| but uses a skip register, |\newskip| and |\glueexpr|
%   instead.
% \end{function}
%
% \begin{function}{toks,gtoks,apptoks,gapptoks}
%   \begin{syntax}
%     dimen \key\ = \meta{cs} \prefixes12
%   \end{syntax}
%   The \meta{cs} should be a single control sequence, such as |\foo|. 
%   Store \val\ inside of a toks-register. The |g| variants use |\global|, the
%   |app| variants append \val\ to the contents of that register. If \meta{cs}
%   is not yet defined it will be initialised with |\newtoks|.
% \end{function}
%
% \begin{function}{box,gbox}
%   \begin{syntax}
%     box \key\ = \meta{cs} \prefixes12
%   \end{syntax}
%   The \meta{cs} should be a single control sequence, such as |\foo|.
%   Typesets \val\ into a |\hbox| and stores the result in a box register. The
%   boxes are colour safe. \expkvd\ doesn't provide a |vbox| type.
% \end{function}
%
% \begin{function}{meta}
%   \begin{syntax}
%     meta \key\ = \{\kv, \ldots\} \prefixes22
%   \end{syntax}
%   This key type can set other keys, you can access the \val\ which was passed
%   to \key\ inside the \kv\ list with |#1|. It works by calling a sub-|\ekvset|
%   on the \kv\ list, so a |set| key will only affect that \kv\ list and not the
%   current |\ekvset|.
% \end{function}
%
% \begin{function}{nmeta}
%   \begin{syntax}
%     nmeta \key\ = \{\kv, \ldots\} \prefixes23
%   \end{syntax}
%   This key type can set other keys, the difference to |meta| is, that this key
%   doesn't take a value, so the \kv\ list is static.
% \end{function}
%
% \begin{function}{smeta}
%   \begin{syntax}
%     smeta \key\ = \marg{set}\{\kv, \ldots\} \prefixes22
%   \end{syntax}
%   Yet another |meta| variant. An |smeta| key will take a \val\ which you can
%   access using |#1|, but it sets the \kv\ list inside of \set, so is equal to
%   \texttt{\cs[no-index]{ekvset}\marg{set}\{\kv, \ldots\}}.
% \end{function}
%
% \begin{function}{snmeta}
%   \begin{syntax}
%     snmeta \key\ = \marg{set}\{\kv, \ldots\} \prefixes23
%   \end{syntax}
%   And the last |meta| variant. |snmeta| is a combination of |smeta| and
%   |nmeta|. It doesn't take an argument and sets the \kv\ list inside of \set.
% \end{function}
%
% \begin{function}{set}
%   \begin{syntax}
%     set \key\ = \marg{set} \prefixes33
%   \end{syntax}
%   This will define |\key| to change the set of the current |\ekvset|
%   invocation to \set. You can omit \set\ (including the equals sign),
%   which is the same as using \texttt{set \key\ = \{\key\}}. The created |set|
%   key will not take a \val. Note that just like in \expkv\ it'll not be
%   checked whether \set\ is defined and you'll get a low-level \TeX\ error if
%   you use an undefined \set.
% \end{function}
%
% \begin{function}{choice}
%   \begin{syntax}
%     choice \key\ = \{\val=\meta{definition}, \ldots\} \prefixes23
%   \end{syntax}
%   Defines \key\ to be a |choice| key, meaning it will only accept a limited
%   set of values. You should define each possible \val\ inside of the
%   \val=\meta{definition} list. If a defined \val\ is passed to \key\ the
%   \meta{dafinition} will be left in the input stream. You can make
%   individual values |protected| inside the \val=\meta{definition} list. By
%   default a |choice| key is expandable, an undefined \val\ will throw an error
%   in an expandable way.
% \end{function}
%
% \subsection{Bugs}
% I don't think there are any (but every developer says that), if you find some
% please let me know, either via the email address on the first page or on
% GitHub:
% \url{https://github.com/Skillmon/tex_expkv-def}
%
% \subsection{Example}
% The following is an example code defining each base key-type once. Please
% admire the very creative key-name examples.
%
% \begin{lstlisting}
% \ekvdefinekeys{example}
%   {
%     ,long code keyA = #1
%     ,noval     keyA = NoVal given
%     ,bool      keyB = \keyB
%     ,boolTF    keyC = \keyC
%     ,store     keyD = \keyD
%     ,int       keyE = \keyE
%     ,dimen     keyF = \keyF
%     ,skip      keyG = \keyG
%     ,toks      keyH = \keyH
%     ,default   keyH = \empty test
%     ,box       keyI = \keyI
%     ,qdefault  keyI = text
%     ,choice    keyJ =
%       {
%         ,protected 1 = \texttt{a}
%         ,2 = b
%         ,3 = c
%         ,4 = d
%         ,5 = e
%       }
%     ,edefault  keyJ = 2
%     ,meta      keyK = {keyA={#1},keyB=false}
%     ,set       setB = B
%   }
% \end{lstlisting}
%
% \subsection{License}
%
% Copyright \textcopyright\ 2020\unless\ifnum\year=2020--\the\year\fi\
% Jonathan P. Spratte
%
% \medskip\noindent
% This work may be distributed and/or modified under the conditions of the
% \LaTeX\ Project Public License (LPPL), either version 1.3c of this license or
% (at your option) any later version. The latest version of this license is in
% the file:
%
% \url{http://www.latex-project.org/lppl.txt}
%
% \noindent
% This work is ``maintained'' (as per LPPL maintenance status) by
%
% \mbox{Jonathan P. Spratte}.
%
% \end{documentation}^^A=<<
%
% \begin{implementation}^^A>>=
%
% \clearpage
%
% \section{Implementation}^^A>>=
%^^A the LaTeX package >>=
% \subsection{The \LaTeX\ Package}
% Just like for \expkv\ we provide a small \LaTeX\ package that sets up things
% such that we behave nicely on \LaTeX\ packages and files system. It'll
% |\input| the generic code which implements the functionality.
% \gobbledocstriptag
%<*pkg>
%    \begin{macrocode}
\RequirePackage{expkv}
\def\ekvd@tmp
  {%
    \ProvidesFile{expkv-def.tex}%
      [\ekvdDate\space v\ekvdVersion\space a key-defining frontend for expkv]%
  }
% \iffalse meta-comment
%
% File: expkv-def.dtx Copyright (C) 2020-2021 Jonathan P. Spratte
%
% This work  may be  distributed and/or  modified under  the conditions  of the
% LaTeX Project Public License (LPPL),  either version 1.3c  of this license or
% (at your option) any later version.  The latest version of this license is in
% the file:
%
%   http://www.latex-project.org/lppl.txt
%
% ------------------------------------------------------------------------------
%
%<*driver>^^A>>=
\def\nameofplainTeX{plain}
\ifx\fmtname\nameofplainTeX\else
  \expandafter\begingroup
\fi
\input l3docstrip.tex
\askforoverwritefalse
\preamble

--------------------------------------------------------------
expkv-def -- a key-defining frontend for expkv
E-mail: jspratte@yahoo.de
Released under the LaTeX Project Public License v1.3c or later
See http://www.latex-project.org/lppl.txt
--------------------------------------------------------------

Copyright (C) 2020-2021 Jonathan P. Spratte

This  work may be  distributed and/or  modified under  the conditions  of the
LaTeX Project Public License (LPPL),  either version 1.3c  of this license or
(at your option) any later version.  The latest version of this license is in
the file:

  http://www.latex-project.org/lppl.txt

This work is "maintained" (as per LPPL maintenance status) by
  Jonathan P. Spratte.

This work consists of the file  expkv-def.dtx
and the derived files           expkv-def.pdf
                                expkv-def.sty
                                expkv-def.tex
                                t-expkv-def.tex

\endpreamble
% stop docstrip adding \endinput
\postamble
\endpostamble
\generate{\file{expkv-def.sty}{\from{expkv-def.dtx}{pkg}}}
\generate{\file{expkv-def.tex}{\from{expkv-def.dtx}{tex}}}
\generate{\file{t-expkv-def.tex}{\from{expkv-def.dtx}{ctx}}}
\ifx\fmtname\nameofplainTeX
  \expandafter\endbatchfile
\else
  \expandafter\endgroup
\fi
%
\IfFileExists{expkv-def.tex}{% \iffalse meta-comment
%
% File: expkv-def.dtx Copyright (C) 2020-2021 Jonathan P. Spratte
%
% This work  may be  distributed and/or  modified under  the conditions  of the
% LaTeX Project Public License (LPPL),  either version 1.3c  of this license or
% (at your option) any later version.  The latest version of this license is in
% the file:
%
%   http://www.latex-project.org/lppl.txt
%
% ------------------------------------------------------------------------------
%
%<*driver>^^A>>=
\def\nameofplainTeX{plain}
\ifx\fmtname\nameofplainTeX\else
  \expandafter\begingroup
\fi
\input l3docstrip.tex
\askforoverwritefalse
\preamble

--------------------------------------------------------------
expkv-def -- a key-defining frontend for expkv
E-mail: jspratte@yahoo.de
Released under the LaTeX Project Public License v1.3c or later
See http://www.latex-project.org/lppl.txt
--------------------------------------------------------------

Copyright (C) 2020-2021 Jonathan P. Spratte

This  work may be  distributed and/or  modified under  the conditions  of the
LaTeX Project Public License (LPPL),  either version 1.3c  of this license or
(at your option) any later version.  The latest version of this license is in
the file:

  http://www.latex-project.org/lppl.txt

This work is "maintained" (as per LPPL maintenance status) by
  Jonathan P. Spratte.

This work consists of the file  expkv-def.dtx
and the derived files           expkv-def.pdf
                                expkv-def.sty
                                expkv-def.tex
                                t-expkv-def.tex

\endpreamble
% stop docstrip adding \endinput
\postamble
\endpostamble
\generate{\file{expkv-def.sty}{\from{expkv-def.dtx}{pkg}}}
\generate{\file{expkv-def.tex}{\from{expkv-def.dtx}{tex}}}
\generate{\file{t-expkv-def.tex}{\from{expkv-def.dtx}{ctx}}}
\ifx\fmtname\nameofplainTeX
  \expandafter\endbatchfile
\else
  \expandafter\endgroup
\fi
%
\IfFileExists{expkv-def.tex}{% \iffalse meta-comment
%
% File: expkv-def.dtx Copyright (C) 2020-2021 Jonathan P. Spratte
%
% This work  may be  distributed and/or  modified under  the conditions  of the
% LaTeX Project Public License (LPPL),  either version 1.3c  of this license or
% (at your option) any later version.  The latest version of this license is in
% the file:
%
%   http://www.latex-project.org/lppl.txt
%
% ------------------------------------------------------------------------------
%
%<*driver>^^A>>=
\def\nameofplainTeX{plain}
\ifx\fmtname\nameofplainTeX\else
  \expandafter\begingroup
\fi
\input l3docstrip.tex
\askforoverwritefalse
\preamble

--------------------------------------------------------------
expkv-def -- a key-defining frontend for expkv
E-mail: jspratte@yahoo.de
Released under the LaTeX Project Public License v1.3c or later
See http://www.latex-project.org/lppl.txt
--------------------------------------------------------------

Copyright (C) 2020-2021 Jonathan P. Spratte

This  work may be  distributed and/or  modified under  the conditions  of the
LaTeX Project Public License (LPPL),  either version 1.3c  of this license or
(at your option) any later version.  The latest version of this license is in
the file:

  http://www.latex-project.org/lppl.txt

This work is "maintained" (as per LPPL maintenance status) by
  Jonathan P. Spratte.

This work consists of the file  expkv-def.dtx
and the derived files           expkv-def.pdf
                                expkv-def.sty
                                expkv-def.tex
                                t-expkv-def.tex

\endpreamble
% stop docstrip adding \endinput
\postamble
\endpostamble
\generate{\file{expkv-def.sty}{\from{expkv-def.dtx}{pkg}}}
\generate{\file{expkv-def.tex}{\from{expkv-def.dtx}{tex}}}
\generate{\file{t-expkv-def.tex}{\from{expkv-def.dtx}{ctx}}}
\ifx\fmtname\nameofplainTeX
  \expandafter\endbatchfile
\else
  \expandafter\endgroup
\fi
%
\IfFileExists{expkv-def.tex}{\input{expkv-def.tex}}{}
\ProvidesFile{expkv-def.dtx}
  [\csname ekvdDate\endcsname\ a key-defining frontend for expkv]
\PassOptionsToPackage{full}{textcomp}
\documentclass{l3doc}
\RequirePackage[oldstylenums,nott]{kpfonts}
\input{glyphtounicode}
\pdfgentounicode=1
\RequirePackage{tcolorbox}
\newtcolorbox{exresult}[2][]
  {colback=expkvgrey!10!white,colframe=expkvgrey,fontupper=\small,width={#2},#1}
\newtcbox\exres[1][]
  {
     colback=expkvgrey!10!white
    ,colframe=expkvgrey
    ,size=small
    ,nobeforeafter
    ,tcbox raise base
    ,fontupper=\small
    ,#1
  }
\RequirePackage{listings}
\RequirePackage{booktabs}
\RequirePackage{array}
\RequirePackage{collcell}
\RequirePackage{siunitx}
\RequirePackage{xcolor}
\RequirePackage{caption}
\RequirePackage{microtype}
\RequirePackage{accsupp}
\lstdefinelanguage{expkvd}[primitive]{TeX}
  {
    ,moretexcs=[4]^^A e-TeX
      {
        expanded,
        numexpr,
        protected,
      }
    ,moretexcs=[5]^^A plain/LaTeX
      {
        approx,
        begin,
        empty,
        item,
        LaTeX,
        makeatletter,makeatother,
        newcommand,newdimen,
        RequirePackage,
        rule,
        TeX,
        textit,texttt,
        usepackage,
      }
    ,moretexcs=[6]^^A used packages
      {
        ^^A xfp
        fpeval,
        ^^A keyval
        setkeys,
        ^^A yax
        defactiveparameter,storevalue,setparameterlist
      }
    ,moretexcs=[2]^^A expkv-def macros
      {
        ekvdefinekeys
      }
    ,moretexcs=[3]^^A expkv-pkg macros
      {
        ^^A expkv
        ekvdef,ekvdefNoVal,^^A
        ekvlet,ekvletNoVal,ekvletkv,ekvletkvNoVal,^^A
        ekvdefunknown,ekvdefunknownNoVal,^^A
        ekvredirectunknown,ekvredirectunknownNoVal,^^A
        ekvset,ekvsetSneaked,^^A
        ekvsetdef,ekvsetSneakeddef,ekvsetdefSneaked,^^A
        ekvparse,^^A
        ekvVersion,ekvDate,^^A
        ekvifdefined,ekvifdefinedNoVal,ekvifdefinedset,^^A
        ekvbreak,ekvbreakPreSneak,ekvbreakPostSneak,^^A
        ekvsneak,ekvsneakPre,^^A
        ekvchangeset,^^A
        ekvoptarg,ekvoptargTF,^^A
        ekverr,
        ^^A expkv-cs
        ekvcSecondaryKeys,ekvcSplitAndForward,
      }
  }
\colorlet{codeparam}{cyan!65!black}
\def\mylstwd{.55em}
\lstset
  {
    ,language=expkvd
    ,flexiblecolumns=false
    ,basewidth=\mylstwd
    ,gobble=2
    ,basicstyle=\fontfamily{jkp}\itshape
    ,texcsstyle=*[1]{\bfseries\color{expkvgrey}} ^^A primitives
    ,texcsstyle=*[2]{\bfseries\color{expkvred}}  ^^A expkv
    ,texcsstyle=*[3]{\color{expkvred}}           ^^A expkv-pkg
    ,texcsstyle=*[4]{\bfseries\color{expkvgrey}} ^^A e-TeX
    ,texcsstyle=*[5]{\bfseries\color{expkvgrey}} ^^A plain/LaTeX macros
    ,texcsstyle=*[6]{}                           ^^A macros of other packages
    ,commentstyle=\color[gray]{0.4}
    ,literate=
              {\{} {{\CodeUpBf\{}}{1}
              {\}} {{\CodeUpBf\}}}{1}
              {$}  {{\CodeUpBf\$}}{1}
              {[}  {{\CodeUp[}}{1}
              {]}  {{\CodeUp]}}{1}
              {(}  {{\CodeUp(}}{1}
              {)}  {{\CodeUp)}}{1}
              {*}  {{$*$}}{1}
              {1}  {{\CodeUp{1}}}{1}
              {2}  {{\CodeUp{2}}}{1}
              {3}  {{\CodeUp{3}}}{1}
              {4}  {{\CodeUp{4}}}{1}
              {5}  {{\CodeUp{5}}}{1}
              {6}  {{\CodeUp{6}}}{1}
              {7}  {{\CodeUp{7}}}{1}
              {8}  {{\CodeUp{8}}}{1}
              {9}  {{\CodeUp{9}}}{1}
              {0}  {{\CodeUp{0}}}{1}
              {##} {{\CodeColored{codeparam}{1}{\#}}}{1}
              {##1}{{\CodeColored{codeparam}{2}{\#1}}}{2}
              {##2}{{\CodeColored{codeparam}{2}{\#2}}}{2}
              {##3}{{\CodeColored{codeparam}{2}{\#3}}}{2}
              {##4}{{\CodeColored{codeparam}{2}{\#4}}}{2}
              {##5}{{\CodeColored{codeparam}{2}{\#5}}}{2}
              {##6}{{\CodeColored{codeparam}{2}{\#6}}}{2}
              {##7}{{\CodeColored{codeparam}{2}{\#7}}}{2}
              {##8}{{\CodeColored{codeparam}{2}{\#8}}}{2}
              {##9}{{\CodeColored{codeparam}{2}{\#9}}}{2}
              {<key>}{{$\langle$}key{$\rangle$}}{5}
              {<set>}{{$\langle$}set{$\rangle$}}{5}
              {<set1>}{{$\langle$}set1{$\rangle$}}{6}
              {<set2>}{{$\langle$}set2{$\rangle$}}{6}
    ^^A,literate=*{<key>}{\key}{4}{<set>}{\set}{4}
  }
\newcommand*\CodeColored[3]{\textcolor{#1}{\makebox[\dimexpr\mylstwd*#2]{#3}}}
\newcommand*\CodeUpBf[1]{\makebox[\mylstwd]{\textup{\textbf{#1}}}}
\newcommand*\CodeUp[1]{\makebox[\mylstwd]{\textup{#1}}}
\RequirePackage{randtext}
\let\metaORIG\meta
\protected\def\meta #1{\texttt{\metaORIG{#1}}}
\renewcommand*\thefootnote{\fnsymbol{footnote}}
\definecolor{expkvred}{HTML}{9F393D}
\colorlet{expkvgrey}{black!75}
\makeatletter
\newcommand*\expkv
  {^^A
    \texorpdfstring
      {^^A
        \mbox
          {^^A
            \BeginAccSupp{ActualText=expkv}^^A
            \href{https://github.com/Skillmon/tex_expkv}
              {^^A
                \rmfamily
                \bfseries
                {\color{expkvgrey}e\kern-.05em x\kern-.05em}^^A
                \lower.493ex
                  \hbox{{\color{expkvgrey}P}\kern-.1em{\color{expkvred}k}}^^A
                \kern-.18em{\color{expkvred}v}^^A
              }^^A
            \EndAccSupp{}^^A
          }^^A
      }
      {expkv}^^A
  }
\newcommand*\expkvd
  {^^A
    \texorpdfstring
      {^^A
        \mbox
          {^^A
            \BeginAccSupp{ActualText=expkv-def}^^A
            \href{https://github.com/Skillmon/tex_expkv-def}
              {^^A
                \rmfamily
                \bfseries
                {\color{expkvgrey}e\kern-.05em x\kern-.05em}^^A
                \lower.493ex
                  \hbox{{\color{expkvgrey}P}\kern-.1em{\color{expkvred}k}}^^A
                \kern-.18em{\color{expkvred}v}^^A
                {\color{expkvgrey}^^A
                  \kern.05em\rule[-.1ex]{.08em}{1.2ex}\kern.05em\textsc{def}^^A
                }^^A
              }^^A
            \EndAccSupp{}^^A
          }^^A
      }
      {expkv-def}^^A
  }
\newcommand\kv{\meta{key}=\meta{value}}
\newcommand\key{\meta{key}}
\newcommand\val{\meta{value}}
\newcommand\set{\meta{set}}
\newcommand\enfprefix{\textcolor{black}}
\newcommand\allprefix{\textcolor{gray}}
\newcommand\notprefix{\textcolor{red!80!black}}
\newcommand\printprefix[2]
  {^^A
    \ifcase\numexpr#1\relax\or
      \enfprefix{#2}\or
      \allprefix{#2}\or
      \notprefix{#2}\fi
  }
\newcommand\prefixes[4]
  {^^A
    \hfill
    \printprefix{#1}{new}
    \printprefix{#2}{also}
    \printprefix{#3}{protected}
    \printprefix{#4}{long}^^A
  }
\hypersetup{linkcolor=red!80!black,urlcolor=purple!80!black}
\DoNotIndex{\def,\edef,\,,\=,\begingroup,\catcode,\chardef,\csname,\endcsname}
\DoNotIndex{\endgroup,\endinput,\errmessage,\expandafter,\input,\let,\long}
\DoNotIndex{\protected,\ProvidesFile,\ProvidesPackage,\relax,\space}
\DoNotIndex{\@,\unexpanded,\string,\expanded,\dimexpr,\global,\glueexpr,\hbox}
\DoNotIndex{\numexpr,\RequirePackage,\setbox,\the,\unless,\xdef,\@firstofone}
\DoNotIndex{\@firstoftwo,\@gobble,\@secondoftwo,\meaning,\detokenize}
\DoNotIndex{\romannumeral,\^}
\DoNotIndex{\ifcsname}
\DoNotIndex{\ifx}
\DoNotIndex{\ifdefined}
\DoNotIndex{\iffalse}
\DoNotIndex{\iftrue}
\DoNotIndex{\else}
\DoNotIndex{\fi}
\@gobble\fi ^^A ignoring \ifx, \ifcsname, etc., but only one \fi
\@gobble\fi ^^A ignoring \ifx, \ifcsname, etc., but only one \fi
\@gobble\fi ^^A ignoring \ifx, \ifcsname, etc., but only one \fi
\@gobble\fi ^^A ignoring \ifx, \ifcsname, etc., but only one \fi
\@ifdefinable\gobbledocstriptag{\def\gobbledocstriptag#1>{}}
\makeatother
\begin{document}
  \title
    {^^A
      \texorpdfstring
        {^^A
          \huge\expkvd
          \\[\medskipamount]
          \Large a key-defining frontend for \expkv
        }{expkv-def - a key-defining frontend for expkv}^^A
    }
  \date{\ekvdDate\space v\ekvdVersion}
  \author{Jonathan P. Spratte\thanks{\protect\randomize{jspratte@yahoo.de}}}
  \DocInput{expkv-def.dtx}
\end{document}
%</driver>^^A=<<
% \fi
%
% \maketitle
% \renewcommand*\thefootnote{\arabic{footnote}}
%
% \begin{abstract}
% \noindent\parfillskip=0pt
% \expkvd\ provides a small \kv\ interface to define keys for \expkv. Key-types
% are declared using prefixes, similar to static typed languages. The stylised
% name is \expkvd\ but the files use \file{expkv-def}, this is due to CTAN-rules
% which don't allow \string| in package names since that is the pipe symbol in
% *nix shells.
% \end{abstract}
%
% \tableofcontents
%
% \begin{documentation}^^A>>=
%
% \section{Documentation}
%
% Since the trend for the last couple of years goes to defining keys for a \kv\
% interface using a \kv\ interface, I thought that maybe providing such an
% interface for \expkv\ will make it more attractive for actual use, besides its
% unique selling points of being fully expandable, and fast and reliable. But at
% the same time I don't want to widen \expkv's initial scope. So here it is
% \expkvd, go define \kv\ interfaces with \kv\ interfaces.
%
% Unlike many of the other established \kv\ interfaces to define keys, \expkvd\
% works using prefixes instead of suffixes (\emph{e.g.}, |.tl_set:N| of
% \pkg{l3keys}) or directory like handlers (\emph{e.g.}, |/.store in| of
% \pkg{pgfkeys}). This was decided as a personal preference, more over in \TeX\
% parsing for the first space is way easier than parsing for the last one.
% \expkvd's prefixes are sorted into two categories: |p|-type, which are
% equivalent to \TeX's prefixes like |\long|, and |t|-type defining the type of
% the key. For a description of the available |p|-prefixes take a look at
% \autoref{sec:prefix:p}, the |t|-prefixes are described in
% \autoref{sec:prefix:t}.
%
% \expkvd\ is usable as generic code, as a \LaTeX\ package, and as a \ConTeXt\
% module. It'll automatically load \expkv\ in the same mode as well. To use it,
% just use one of
% \begin{lstlisting}
% \input expkv-def       % plainTeX
% \usepackage{expkv-def} % LaTeX
% \usemodule[expkv-def]  % ConTeXt
% \end{lstlisting}
%
% \subsection{Macros}
% Apart from version and date containers there is only a single user-facing
% macro, and that should be used to define keys.
%
% \begin{function}{\ekvdefinekeys}
%   \begin{syntax}
%     \cs{ekvdefinekeys}\marg{set}\{\kv, \ldots\}
%   \end{syntax}
%   In \set, define \key\ to have definition \val. The general syntax for \key\
%   should be
%   \begin{quote}\ttfamily\small
%     \meta{prefix} \meta{name}
%   \end{quote}
%   Where \meta{prefix} is a space separated list of optional |p|-type prefixes
%   followed by one |t|-type prefix. The syntax of \val\ is dependent on the
%   used |t|-prefix.
% \end{function}
%
% \begin{function}{\ekvdDate,\ekvdVersion}
%   These two macros store the version and date of the package.
% \end{function}
%
% \subsection{Prefixes}
% As already said there are |p|-prefixes and |t|-prefixes. Not every |p|-prefix
% is allowed for all |t|-prefixes.
%
% \subsubsection{\texttt{p}-Prefixes}\label{sec:prefix:p}
% The two |p|-type prefixes |long| and |protected| are pretty simple by nature,
% so their description is pretty simple. They affect the \key\ at use-time, so
% omitting |long| doesn't mean that a \meta{definition} can't contain a |\par|
% token, only that the \key\ will not accept a |\par| in \val.  On the other
% hand |new| and |also| might be simple on first sight as well, but their rules
% are a bit more complicated.
%
% \begin{function}{also}
%   The following key type will be \emph{added} to an existing \key's
%   definition. You can't add a type taking an argument at use time to an
%   existing key which doesn't take an argument and vice versa. Also you'll get
%   an error if you try to add an action which isn't allowed to be either |long|
%   or |protected| to a key which already is |long| or |protected| (the opposite
%   order would be suboptimal as well, but can't be really captured with the
%   current code).
%
%   A key already defined as |long| or |protected| will stay |long| or
%   |protected|, but you can as well add |long| or |protected| with the |also|
%   definition.
%
%   As a small example, suppose you want to create a boolean key, but
%   additionally to setting a boolean value you want to execute some more code
%   as well, you can use the following
% \begin{lstlisting}
% \ekvdefinekeys{also-example}
%   {
%      bool key      = \ifmybool
%     ,also code key = \domystuff{#1}
%   }
% \end{lstlisting}
%
%   If you use |also| on a |choice|, |bool|, |invbool|, or |boolpair| key it is
%   tried to determine if the key already is of one of those types. If this test
%   is true the declared choices will be added to the possible choices but the
%   key's definition will not be changed other than that. If that wouldn't have
%   been done, the callbacks of the different choices could get called multiple
%   times.
% \end{function}
%
% \begin{function}{protected,protect}
%   The following key will be defined |\protected|. Note that key-types which
%   can't be defined expandable will always use |\protected|.
% \end{function}
%
% \begin{function}{long}
%   The following key will be defined |\long|.
% \end{function}
%
% \begin{function}{new}
%   The following key must be new (so previously undefined). An error is thrown
%   if it is already defined and the new definition is ignored. |new| only
%   asserts that there are no conflicts between |NoVal| keys and other |NoVal|
%   keys or value taking keys and other value taking keys. For example you can
%   use the following without an error:
% \begin{lstlisting}
% \ekvdefinekeys{new-example}
%   {
%      code key       = \domystuffwitharg{#1}
%     ,new  noval key = \domystuffwithoutarg
%   }
% \end{lstlisting}
% \end{function}
%
% \subsubsection{\texttt{t}-Prefixes}\label{sec:prefix:t}
% Since the |p|-type prefixes apply to some of the |t|-prefixes automatically
% but sometimes one might be disallowed we need some way to highlight this
% behaviour. In the following an enforced prefix will be printed black
% (\texttt{\enfprefix{protected}}), allowed prefixes will be grey
% (\texttt{\allprefix{protected}}), and disallowed prefixes will be red
% (\texttt{\notprefix{protected}}). This will be put flush-right in the syntax
% showing line.
%
% \begin{function}{code,ecode}
%   \begin{syntax}
%     code \key\ = \marg{definition} \prefixes2222
%   \end{syntax}
%   Define \key\ to expand to \meta{definition}. The \key\ will require a \val\
%   for which you can use |#1| inside \meta{definition}. The |ecode| variant
%   will fully expand \meta{definition} inside an |\edef|.
% \end{function}
%
% \begin{function}{noval,enoval}
%   \begin{syntax}
%     noval \key\ = \marg{definition} \prefixes2223
%   \end{syntax}
%   The |noval| type defines \key\ to expand to \meta{definition}. The \key\
%   will not take a \val. |enoval| fully expands \meta{definition} inside an
%   |\edef|.
% \end{function}
%
% \begin{function}{default,qdefault,odefault,fdefault,edefault}
%   \begin{syntax}
%     default \key\ = \marg{definition} \prefixes3223
%   \end{syntax}
%   This serves to place a default \val\ for a \key\ that takes an argument, the
%   \key\ can be of any argument-grabbing kind, and when used without a \val\
%   it will be passed \meta{definition} instead. The |qdefault| variant will
%   expand the \key's code once, so will be slightly quicker, but not change if
%   you redefine \key. |odefault| is just another name for |qdefault|. The
%   |fdefault| version will expand the key code until a non-expandable token or
%   a space is found, a space would be gobbled.\footnotemark{} The |edefault| on
%   the other hand fully expands the \key-code with \meta{definition} as its
%   argument inside of an |\edef|.
% \end{function}
% \footnotetext
%   {^^A
%     For those familiar with \TeX-coding: This uses a
%     \cs[no-index]{romannumeral}-expansion.
%   }
%
% \begin{function}{initial,oinitial,finitial,einitial}
%   \begin{syntax}
%     initial \key\ = \{\val\} \prefixes3333
%     initial \key
%   \end{syntax}
%   With |initial| you can set an initial \val\ for an already defined argument
%   taking \key. It'll just call the key-macro of \key\ and pass it \val. The
%   |einitial| variant will expand \val\ using an |\edef| expansion prior to
%   passing it to the key-macro and the |oinitial| variant will expand the first
%   token in \val\ once. |finitial| will expand \val\ until a non-expandable
%   token or a space is found, a space would be
%   gobbled.\footnotemark
%
%   If you don't provide a value (and no equals sign) a |noval| \key\ of the
%   same name is called once (or, if you specified a |default| for a value
%   taking key that would be used).
% \end{function}
% \footnotetext{Again using \cs[no-index]{romannumeral}.}
%
% \begin{function}{bool,gbool,boolTF,gboolTF}
%   \begin{syntax}
%     bool \key\ = \meta{cs} \prefixes2223
%   \end{syntax}
%   The \meta{cs} should be a single control sequence, such as |\iffoo|.
%   This will define \key\ to be a boolean key, which only takes the values
%   |true| or |false| and will throw an error for other values. If the key is
%   used without a \val\ it'll have the same effect as if you use
%   \texttt{\key=true}. |bool| and |gbool| will behave like \TeX-ifs so either
%   be |\iftrue| or |\iffalse|. The |boolTF| and |gboolTF| variants will both
%   take two arguments and if true the first will be used else the second, so
%   they are always either |\@firstoftwo| or |\@secondoftwo|. The variants with
%   a leading |g| will set the control sequence globally, the others locally. If
%   \meta{cs} is not yet defined it'll be initialised as the |false| version.
%   Note that the initialisation is \emph{not} done with |\newif|, so you will
%   not be able to do |\footrue| outside of the \kv\ interface, but you could
%   use |\newif| yourself. Even if the \key\ will not be |\protected| the
%   commands which execute the |true| or |false| choice will be, so the usage
%   should be safe in an expansion context (\emph{e.g.}, you can use
%   \texttt{edefault \key\ = false} without an issue to change the default
%   behaviour to execute the |false| choice). Internally a |bool| \key\ is the
%   same as a |choice| key which is set up to handle |true| and |false| as
%   choices.
% \end{function}
%
% \begin{function}{invbool,ginvbool,invboolTF,ginvboolTF}
%   \begin{syntax}
%     bool \key\ = \meta{cs} \prefixes2223
%   \end{syntax}
%   These are inverse boolean keys, they behave like |bool| and friends but set
%   the opposite meaning to the macro \meta{cs} in each case. So if
%   \texttt{key=true} is used |invbool| will set \meta{cs} to |\iffalse| and
%   vice versa.
% \end{function}
%
% \begin{function}{boolpair,gboolpair,boolpairTF,gboolpairTF}
%   \begin{syntax}
%     boolpair \key\ = \meta{cs_1}\meta{cs_2} \prefixes2223
%   \end{syntax}
%   The |boolpair| key type behaves like both |bool| and |invbool|, the
%   \meta{cs_1} will be set to the meaning according to the rules of |bool|, and
%   \meta{cs_2} will be set to the opposite.
% \end{function}
%
% \begin{function}{store,estore,gstore,xstore}
%   \begin{syntax}
%     store \key\ = \meta{cs} \prefixes2212
%   \end{syntax}
%   The \meta{cs} should be a single control sequence, such as |\foo|.
%   This will define \key\ to store \val\ inside of the control sequence. If
%   \meta{cs} isn't yet defined it will be initialised as empty. The variants
%   behave similarly to their |\def|, |\edef|, |\gdef|, and |\xdef|
%   counterparts, but |store| and |gstore| will allow you to store macro
%   parameters inside of them by using |\unexpanded|.
% \end{function}
%
% \begin{function}{data,edata,gdata,xdata}
%   \begin{syntax}
%     data \key\ = \meta{cs} \prefixes2212
%   \end{syntax}
%   The \meta{cs} should be a single control sequence, such as |\foo|. This will
%   define \key\ to store \val\ inside of the control sequence. But unlike the
%   |store| type, the macro \meta{cs} will be a switch at the same time, it'll
%   take two arguments and if \key\ was used expands to the first argument
%   followed by \val\ in braces, if \key\ was not used \meta{cs} will expand to
%   the second argument (so behave like |\@secondoftwo|). The idea is that with
%   this type you can define a key which should be typeset formatted. The
%   |edata| and |xdata| variants will fully expand \val, the |gdata| and |xdata|
%   variants will store \val\ inside \meta{cs} globally. The |p|-prefixes will
%   only affect the key-macro, \meta{cs} will always be expandable and |\long|.
% \end{function}
%
% \begin{function}{dataT,edataT,gdataT,xdataT}
%   \begin{syntax}
%     dataT \key\ = \meta{cs} \prefixes2212
%   \end{syntax}
%   Just like |data|, but instead of \meta{cs} grabbing two arguments it'll only
%   grab one, so by default it'll behave like |\@gobble|, and if a \val\ was
%   given to \key\ the \meta{cs} will behave like |\@firstofone| appended by
%   \marg{value}.
% \end{function}
%
% \begin{function}{int,eint,gint,xint}
%   \begin{syntax}
%     int \key\ = \meta{cs} \prefixes2212
%   \end{syntax}
%   The \meta{cs} should be a single control sequence, such as |\foo|.
%   An |int| key will be a \TeX-count register. If \meta{cs} isn't defined yet,
%   |\newcount| will be used to initialise it. The |eint| and |xint| versions
%   will use |\numexpr| to allow basic computations in their \val. The |gint|
%   and |xint| variants set the register globally.
% \end{function}
%
% \begin{function}{dimen,edimen,gdimen,xdimen}
%   \begin{syntax}
%     dimen \key\ = \meta{cs} \prefixes2212
%   \end{syntax}
%   The \meta{cs} should be a single control sequence, such as |\foo|.
%   This is just like |int| but uses a dimen register, |\newdimen| and
%   |\dimexpr| instead.
% \end{function}
%
% \begin{function}{skip,eskip,gskip,xskip}
%   \begin{syntax}
%     skip \key\ = \meta{cs} \prefixes2212
%   \end{syntax}
%   The \meta{cs} should be a single control sequence, such as |\foo|.
%   This is just like |int| but uses a skip register, |\newskip| and |\glueexpr|
%   instead.
% \end{function}
%
% \begin{function}{toks,gtoks,apptoks,gapptoks,pretoks,gpretoks}
%   \begin{syntax}
%     toks \key\ = \meta{cs} \prefixes2212
%   \end{syntax}
%   The \meta{cs} should be a single control sequence, such as |\foo|.
%   Store \val\ inside of a toks-register. The |g| variants use |\global|, the
%   |app| variants append \val\ to the contents of that register, the |pre|
%   variants will prepend \val. If \meta{cs} is not yet defined it will be
%   initialised with |\newtoks|.
% \end{function}
%
% \begin{function}{box,gbox}
%   \begin{syntax}
%     box \key\ = \meta{cs} \prefixes2212
%   \end{syntax}
%   The \meta{cs} should be a single control sequence, such as |\foo|.
%   Typesets \val\ into a |\hbox| and stores the result in a box register. The
%   boxes are colour safe. \expkvd\ doesn't provide a |vbox| type.
% \end{function}
%
% \begin{function}{meta}
%   \begin{syntax}
%     meta \key\ = \{\kv, \ldots\} \prefixes2222
%   \end{syntax}
%   This key type can set other keys, you can access the \val\ which was passed
%   to \key\ inside the \kv\ list with |#1|. It works by calling a sub-|\ekvset|
%   on the \kv\ list, so a |set| key will only affect that \kv\ list and not the
%   current |\ekvset|. Since it runs in a separate |\ekvset| you can't use
%   |\ekvsneak| using keys or similar macros in the way you normally could.
% \end{function}
%
% \begin{function}{nmeta}
%   \begin{syntax}
%     nmeta \key\ = \{\kv, \ldots\} \prefixes2223
%   \end{syntax}
%   This key type can set other keys, the difference to |meta| is, that this key
%   doesn't take a value, so the \kv\ list is static.
% \end{function}
%
% \begin{function}{smeta}
%   \begin{syntax}
%     smeta \key\ = \marg{set}\{\kv, \ldots\} \prefixes2222
%   \end{syntax}
%   Yet another |meta| variant. An |smeta| key will take a \val\ which you can
%   access using |#1|, but it sets the \kv\ list inside of \set, so is equal to
%   \texttt{\cs[no-index]{ekvset}\marg{set}\{\kv, \ldots\}}.
% \end{function}
%
% \begin{function}{snmeta}
%   \begin{syntax}
%     snmeta \key\ = \marg{set}\{\kv, \ldots\} \prefixes2223
%   \end{syntax}
%   And the last |meta| variant. |snmeta| is a combination of |smeta| and
%   |nmeta|. It doesn't take an argument and sets the \kv\ list inside of \set.
% \end{function}
%
% \begin{function}{set}
%   \begin{syntax}
%     set \key\ = \marg{set} \prefixes2233
%   \end{syntax}
%   This will define \key\ to change the set of the current |\ekvset|
%   invocation to \set. You can omit \set\ (including the equals sign),
%   which is the same as using \texttt{set \key\ = \{\key\}}. The created |set|
%   key will not take a \val. Note that just like in \expkv\ it'll not be
%   checked whether \set\ is defined and you'll get a low-level \TeX\ error if
%   you use an undefined \set.
% \end{function}
%
% \begin{function}{choice}
%   \begin{syntax}
%     choice \key\ = \{\val=\meta{definition}, \ldots\} \prefixes2223
%   \end{syntax}
%   Defines \key\ to be a |choice| key, meaning it will only accept a limited
%   set of values. You should define each possible \val\ inside of the
%   \val=\meta{definition} list. If a defined \val\ is passed to \key\ the
%   \meta{definition} will be left in the input stream. You can make
%   individual values |protected| inside the \val=\meta{definition} list. By
%   default a |choice| key is expandable, an undefined \val\ will throw an error
%   in an expandable way (but see the |unknown-choice| prefix). You can add
%   additional choices after the \key\ was created by using |choice| again for
%   the same \key, redefining choices is possible the same way, but there is no
%   interface to remove certain choices.
% \end{function}
%
% \begin{function}{choice-store}
%   \begin{syntax}
%     choice-store \key\ = \meta{cs}\{\val, \ldots\} \prefixes2223
%   \end{syntax}
%   This is defines a special type of |choice| key that'll store the given
%   choice inside the macro \meta{cs} (so \meta{cs} should be a single control
%   sequence name such as |\foo|). Since storing inside a macro can't be done
%   expandably every choice-code is |\protected|, you might define the
%   |choice-store| key itself as |protected| as well if you want. Since the
%   definition of each choice is predefined with this key type the choice list
%   should just be a comma separated list of valid choices.
% \end{function}
% This means that the following |choice| and |choice-store| keys are equivalent
% at use time:
% \begin{lstlisting}
% \newcommand*\mya{}
% \ekvdefinekeys{example}
%   {
%      choice key1 = {a=\def\mya{a}, b=\def\mya{b}, c=\def\mya{c}}
%     ,choice-store key2 = \mya{a,b,c}
%   }
% \end{lstlisting}
%
% \begin{function}{choice-enum}
%   \begin{syntax}
%     choice-enum \key\ = \meta{cs}\{\val, \ldots\} \prefixes2223
%   \end{syntax}
%   This is similar to |choice-store|, the differences are: \meta{cs} should be
%   a count-register or is initialised as such if the \meta{cs} is undefined
%   (via |\newcount|); instead of the value the position of the value in the
%   given list is stored in this register (zero-based).
% \end{function}
% This means that the following |choice| and |choice-enum| keys are equivalent
% at use time:
% \begin{lstlisting}
% \newcount\myb
% \ekvdefinekeys{example}
%   {
%      choice key1 = {a={\myb=0 }, b={\myb=1 }, c={\myb=2 }}
%     ,choice-enum key2 = \myb{a,b,c}
%   }
% \end{lstlisting}
%
% \begin{function}{unknown-choice}
%   \begin{syntax}
%     unknown-choice \key\ = \marg{definition} \prefixes2323
%   \end{syntax}
%   By default an unknown \val\ passed to a |choice| or |bool| key will throw an
%   error.  However, with this prefix you can define an alternative action which
%   should be executed if \key\ received an unknown choice. In \meta{definition}
%   you can refer to the choice which was passed in with |#1|.
% \end{function}
%
% \begin{function}{unknown code}
%   \begin{syntax}
%     unknown code = \marg{definition} \prefixes2322
%   \end{syntax}
%   By default \expkv\ throws errors when it encounters unknown keys in a set.
%   With the |unknown| prefix you can define handlers that deal with undefined
%   keys, instead of a \key\ name you have to specify a subtype for this prefix,
%   here the subtype is |code|.\par
%   With |unknown code| the \meta{definition} is used for unknown keys which
%   were provided a value (so corresponds to |\ekvdefunknown|), you can access
%   the key name with |#1| and the value with |#2|.\footnotemark
% \end{function}
% \footnotetext
%   {^^A
%     There is some trickery involved to get this more intuitive argument order
%     without any performance hit if you compare this to
%     \cs[no-index]{ekvdefunknown} directly.^^A
%   }
%
% \begin{function}{unknown noval}
%   \begin{syntax}
%     unknown noval = \marg{definition} \prefixes2323
%   \end{syntax}
%   This is like |unknown code| but uses \meta{definition} for unknown keys to
%   which no value was passed (so corresponds to |\ekvdefunknownNoVal|). You can
%   access the key name with |#1|.
% \end{function}
%
% \begin{function}{unknown redirect-code}
%   \begin{syntax}
%     unknown redirect-code = \marg{set-list} \prefixes2331
%   \end{syntax}
%   This uses a predefined action for |unknown code|. Instead of throwing an
%   error, it is tried to find the \key\ in each \set\ in the comma separated
%   \meta{set-list}. The first found match will be used and the remaining
%   options from the list discarded. If the \key\ isn't found in any \set\ an
%   expandable error will be thrown eventually. Internally \expkv's
%   |\ekvredirectunknown| will be used.
% \end{function}
%
% \begin{function}{unknown redirect-noval}
%   \begin{syntax}
%     unknown redirect-noval = \marg{set-list} \prefixes2333
%   \end{syntax}
%   This behaves just like |unknown redirect-code| but will set up means to
%   forward keys for |unknown noval|. Internally \expkv's
%   |\ekvredirectunknownNoVal| will be used.
% \end{function}
%
% \begin{function}{unknown redirect}
%   \begin{syntax}
%     unknown redirect = \marg{set-list} \prefixes2333
%   \end{syntax}
%   This is a short cut to apply both, |unknown redirect-code| and
%   |unknown redirect-noval|, as a result you might get doubled error messages,
%   one from each.
% \end{function}
%
% \subsection{Bugs}
% I don't think there are any (but every developer says that), if you find some
% please let me know, either via the email address on the first page or on
% GitHub:
% \url{https://github.com/Skillmon/tex_expkv-def}
%
% \subsection{Example}
% The following is an example code defining each base key-type once. Please
% admire the very creative key-name examples.
%
% \begin{lstlisting}
% \ekvdefinekeys{example}
%   {
%      long code keyA = #1
%     ,noval     keyA = NoVal given
%     ,bool      keyB = \keyB
%     ,boolTF    keyC = \keyC
%     ,store     keyD = \keyD
%     ,data      keyE = \keyE
%     ,dataT     keyF = \keyF
%     ,int       keyG = \keyG
%     ,dimen     keyH = \keyH
%     ,skip      keyI = \keyI
%     ,toks      keyJ = \keyJ
%     ,default   keyJ = \empty test
%     ,new box   keyK = \keyK
%     ,qdefault  keyK = K
%     ,choice    keyL =
%       {
%          protected 1 = \texttt{a}
%         ,2 = b
%         ,3 = c
%         ,4 = d
%         ,5 = e
%       }
%     ,edefault  keyL = 2
%     ,meta      keyM = {keyA={#1},keyB=false}
%     ,invbool   keyN = \keyN
%     ,boolpair  keyO = \keyOa\keyOb
%   }
% \end{lstlisting}
%
% \bigskip
% Since the |data| type might be a bit strange, here is another usage example
% for it.
% \begin{lstlisting}
% \ekvdefinekeys{ex}
%   {
%      data  name  = \Pname
%     ,data  age   = \Page
%     ,dataT hobby = \Phobby
%   }
% \newcommand\Person[1]
%   {%
%     \begingroup
%     \ekvset{ex}{#1}%
%     \begin{description}
%       \item[\Pname{}{\errmessage{A person requires a name}}]
%       \item[Age] \Page{\textit}{\errmessage{A person requires an age}}
%       \Phobby{\item[Hobbies]}
%     \end{description}
%     \endgroup
%   }
% \Person{name=Jonathan P. Spratte, age=young, hobby=\TeX\ coding}
% \Person{name=Some User, age=unknown, hobby=Reading Documentation}
% \Person{name=Anybody, age=any}
% \end{lstlisting}
% In this example a person should have a name and an age, but doesn't have to
% have hobbies. The name will be displayed as the description item and the age
% in Italics. If a person has no hobbies the description item will be silently
% left out.  The result of the above code looks like this:
%
% \begin{exresult}{\linewidth}
% \ekvdefinekeys{ex}
%   {
%      data  name  = \Pname
%     ,data  age   = \Page
%     ,dataT hobby = \Phobby
%   }
% \newcommand\Person[1]
%   {%
%     \begingroup
%     \ekvset{ex}{#1}%
%     \begin{description}
%       \item[\Pname{}{\errmessage{A person requires a name}}]
%       \item[Age] \Page{\textit}{\errmessage{A person requires an age}}
%       \Phobby{\item[Hobbies]}
%     \end{description}
%     \endgroup
%   }
% \Person{name=Jonathan P. Spratte, age=young, hobby=\TeX\ coding}
% \Person{name=Some User, age=unknown, hobby=Reading Documentation}
% \Person{name=Anybody, age=any}
% \end{exresult}
%
% \subsection{License}
%
% Copyright \textcopyright\ 2020\unless\ifnum\year=2020--\the\year\fi\
% Jonathan P. Spratte
%
% \medskip\noindent
% This work may be distributed and/or modified under the conditions of the
% \LaTeX\ Project Public License (LPPL), either version 1.3c of this license or
% (at your option) any later version. The latest version of this license is in
% the file:
%
% \url{http://www.latex-project.org/lppl.txt}
%
% \noindent
% This work is ``maintained'' (as per LPPL maintenance status) by
%
% \mbox{Jonathan P. Spratte}.
%
% \end{documentation}^^A=<<
%
% \begin{implementation}^^A>>=
%
% \clearpage
%
% \section{Implementation}^^A>>=
%^^A the LaTeX package >>=
% \subsection{The \LaTeX\ Package}
% Just like for \expkv\ we provide a small \LaTeX\ package that sets up things
% such that we behave nicely on \LaTeX\ packages and files system. It'll
% |\input| the generic code which implements the functionality.
% \gobbledocstriptag
%<*pkg>
%    \begin{macrocode}
\RequirePackage{expkv}
\def\ekvd@tmp
  {%
    \ProvidesFile{expkv-def.tex}%
      [\ekvdDate\space v\ekvdVersion\space a key-defining frontend for expkv]%
  }
\input{expkv-def.tex}
\ProvidesPackage{expkv-def}%
  [\ekvdDate\space v\ekvdVersion\space a key-defining frontend for expkv]
%    \end{macrocode}
% \gobbledocstriptag
%</pkg>
%^^A=<<
%^^A the ConTeXt module >>=
% \subsection{The \ConTeXt\ module}
% \gobbledocstriptag
%<*ctx>
%    \begin{macrocode}
\writestatus{loading}{ConTeXt User Module / expkv-def}
\usemodule[expkv]
\unprotect
\input expkv-def.tex
\writestatus{loading}
  {ConTeXt User Module / expkv-def / Version \ekvdVersion\space loaded}
\protect\endinput
%    \end{macrocode}
% \gobbledocstriptag
%</ctx>
%^^A=<<
%^^A main file >>=
% \subsection{The Generic Code}
% The rest of this implementation will be the generic code.
% \gobbledocstriptag
%<*tex>
%
% Load \expkv\ if the package didn't already do so -- since \expkv\ has
% safeguards against being loaded twice this does no harm and the overhead
% isn't that big. Also we reuse some of the internals of \expkv\ to save us from
% retyping them.
%    \begin{macrocode}
\input expkv
%    \end{macrocode}
%
% We make sure that \file{expkv-def.tex} is only input once:
%    \begin{macrocode}
\expandafter\ifx\csname ekvdVersion\endcsname\relax
\else
  \expandafter\endinput
\fi
%    \end{macrocode}
%
% \begin{macro}{\ekvdVersion,\ekvdDate}
% We're on our first input, so lets store the version and date in a macro.
%    \begin{macrocode}
\def\ekvdVersion{0.8c}
\def\ekvdDate{2021-09-20}
%    \end{macrocode}
% \end{macro}
%
% If the \LaTeX\ format is loaded we want to be a good file and report back who
% we are, for this the package will have defined |\ekvd@tmp| to use
% |\ProvidesFile|, else this will expand to a |\relax| and do no harm.
%    \begin{macrocode}
\csname ekvd@tmp\endcsname
%    \end{macrocode}
%
% Store the category code of |@| to later be able to reset it and change it to
% 11 for now.
%    \begin{macrocode}
\expandafter\chardef\csname ekvd@tmp\endcsname=\catcode`\@
\catcode`\@=11
%    \end{macrocode}
% |\ekvd@tmp| will be reused later to handle expansion during the key
% defining. But we don't need it to ever store information long-term after
% \expkvd\ was initialized.
%
% \begin{macro}[internal]{\ekvd@ifprimitive}
%    \begin{macrocode}
\protected\long\def\ekvd@ifprimitive#1%
  {%
    \begingroup
      \edef\ekvd@tmpa{\string #1}%
      \edef\ekvd@tmpb{\meaning#1}%
      \expandafter
    \endgroup
    \ifx\ekvd@tmpa\ekvd@tmpb
      \ekv@fi@firstoftwo
    \fi
    \@secondoftwo
  }
%    \end{macrocode}
% \end{macro}
%
% \begin{macro}[internal]
%   {\ekvd@long,\ekvd@prot,\ekvd@clear@prefixes,\ekvd@ifalso}
% \expkvd\ will use |\ekvd@long|, |\ekvd@prot|, and |\ekvd@ifalso| to store
% whether a key should be defined as |\long| or |\protected| or adds an action
% to an existing key, and we have to clear them for every new key. By default
% |long| and |protected| will just be empty, |ifalso| will be
% |\@secondoftwo|, and |ifnew| will just use its third argument.
%    \begin{macrocode}
\protected\def\ekvd@clear@prefixes
  {%
    \let\ekvd@long\ekv@empty
    \let\ekvd@prot\ekv@empty
    \let\ekvd@ifalso\@secondoftwo
    \long\def\ekvd@ifnew##1##2##3{##3}%
  }
\ekvd@clear@prefixes
%    \end{macrocode}
% \end{macro}
%
% \begin{macro}{\ekvdefinekeys}
% This is the one front-facing macro which provides the interface to define
% keys. It's using |\ekvparse| to handle the \kv\ list, the interpretation will
% be done by |\ekvd@noarg| and |\ekvd@|. The \set\ for which the keys should be
% defined is stored in |\ekvd@set|.
%    \begin{macrocode}
\protected\def\ekvdefinekeys#1%
  {%
    \def\ekvd@set{#1}%
    \ekvparse\ekvd@noarg\ekvd@arg
  }
%    \end{macrocode}
% \end{macro}
%
% \begin{macro}[internal]{\ekvd@noarg,\ekvd@arg,\ekvd@handle}
% |\ekvd@noarg| and |\ekvd@arg| store whether there was a value in the \kv\
% pair. |\ekvd@handle| has to test whether there is a space inside the key and
% if so calls the prefix grabbing routine, else we throw an error and ignore the
% key.
%    \begin{macrocode}
\protected\long\def\ekvd@noarg#1%
  {%
    \let\ekvd@ifnoarg\@firstoftwo
    \expandafter\ekvd@handle\detokenize{#1}\ekvd@stop{}%
  }
\protected\long\def\ekvd@arg#1%
  {%
    \let\ekvd@ifnoarg\@secondoftwo
    \expandafter\ekvd@handle\detokenize{#1}\ekvd@stop
  }
\protected\long\def\ekvd@handle#1\ekvd@stop#2%
  {%
    \ekvd@clear@prefixes
    \def\ekvd@cur{#1}%
    \ekvd@ifspace{#1}%
      {\ekvd@prefix\ekv@mark#1\ekv@stop{#2}}%
      \ekvd@err@missing@type
  }
%    \end{macrocode}
% \end{macro}
%
% \begin{macro}[internal]{\ekvd@prefix,\ekvd@prefix@}
% \expkvd\ separates prefixes into two groups, the first being prefixes in the
% \TeX\ sense (|long| and |protected|) which use |@p@| in their name, the other
% being key-types (|code|, |int|, \emph{etc.}) which use |@t@| instead.
% |\ekvd@prefix| splits at the first space and checks whether its a |@p@|
% or |@t@| type prefix. If it is neither throw an error and gobble the
% definition (the value).
%    \begin{macrocode}
\protected\def\ekvd@prefix#1 {\ekv@strip{#1}\ekvd@prefix@\ekv@mark}
\protected\def\ekvd@prefix@#1#2\ekv@stop
  {%
    \ekv@ifdefined{ekvd@t@#1}%
      {\ekv@strip{#2}{\csname ekvd@t@#1\endcsname}}%
      {%
        \ekv@ifdefined{ekvd@p@#1}%
          {\csname ekvd@p@#1\endcsname\ekvd@prefix@after@p{#2}}%
          {\ekvd@err@undefined@prefix{#1}\@gobble}%
      }%
  }
%    \end{macrocode}
% \end{macro}
%
% \begin{macro}[internal]{\ekvd@prefix@after@p}
% The |@p@| type prefixes are all just modifying a following |@t@| type, so they
% will need to search for another prefix. This is true for all of them, so we
% use a macro to handle this. It'll throw an error if there is no other prefix.
%    \begin{macrocode}
\protected\def\ekvd@prefix@after@p#1%
  {%
    \ekvd@ifspace{#1}%
      {\ekvd@prefix#1\ekv@stop}%
      {\ekvd@err@missing@type\@gobble}%
  }
%    \end{macrocode}
% \end{macro}
%
% \begin{macro}[internal]
%   {\ekvd@p@long,\ekvd@p@protected,\ekvd@p@protect,\ekvd@p@also,\ekvd@p@new}
% Define the |@p@| type prefixes, they all just store some information in a
% temporary macro.
%    \begin{macrocode}
\protected\def\ekvd@p@long{\let\ekvd@long\long}
\protected\def\ekvd@p@protected{\let\ekvd@prot\protected}
\let\ekvd@p@protect\ekvd@p@protected
\protected\def\ekvd@p@also{\let\ekvd@ifalso\@firstoftwo}
\protected\def\ekvd@p@new{\let\ekvd@ifnew\ekvd@assert@new}
%    \end{macrocode}
% \end{macro}
%
% \subsubsection{Key Types}
%
% \begin{macro}[internal]{\ekvd@type@set,\ekvd@t@set}
% The |set| type is quite straight forward, just define a |NoVal| key to call
% |\ekvchangeset|.
%    \begin{macrocode}
\protected\def\ekvd@type@set#1#2%
  {%
    \ekvd@assert@not@long
    \ekvd@assert@not@protected
    \ekvd@ifnew{NoVal}{#1}%
      {%
        \ekv@ifempty{#2}%
          {\ekvd@err@missing@definition}%
          {%
            \ekvd@ifalso
              {%
                \ekv@expargtwice{\ekvd@add@noval{#1}}%
                  {\ekvchangeset{#2}}%
                  \ekvd@assert@not@protected@also
              }%
              {\ekv@expargtwice{\ekvdefNoVal\ekvd@set{#1}}{\ekvchangeset{#2}}}%
          }%
      }%
  }
\protected\def\ekvd@t@set#1#2%
  {%
    \ekvd@ifnoarg
      {\ekvd@type@set{#1}{#1}}%
      {\ekvd@type@set{#1}{#2}}%
  }
%    \end{macrocode}
% \end{macro}
%
% \begin{macro}[internal]{\ekvd@type@noval,\ekvd@t@noval,\ekvd@t@enoval}
% Another pretty simple type, |noval| just needs to assert that there is a
% definition and that |long| wasn't specified.
% There are types where the difference in the variants is so small, that we
% define a common handler for them, those common handlers are named with
% |@type@|. |noval| and |enoval| are so similar that we can use such a |@type@|
% macro, even if we could've done |noval| in a slightly faster way without it.
%    \begin{macrocode}
\protected\long\def\ekvd@type@noval#1#2#3%
  {%
    \ekvd@ifnew{NoVal}{#2}%
      {%
        \ekvd@assert@arg
          {%
            \ekvd@assert@not@long
            \ekvd@prot#1\ekvd@tmp{#3}%
            \ekvd@ifalso
              {\ekv@exparg{\ekvd@add@noval{#2}}\ekvd@tmp{}}%
              {\ekvletNoVal\ekvd@set{#2}\ekvd@tmp}%
          }%
      }%
  }
\protected\def\ekvd@t@noval{\ekvd@type@noval\def}
\protected\def\ekvd@t@enoval{\ekvd@type@noval\edef}
%    \end{macrocode}
% \end{macro}
%
% \begin{macro}[internal]{\ekvd@type@code,\ekvd@t@code,\ekvd@t@ecode}
% |code| is simple as well, |ecode| has to use |\edef| on a temporary macro,
% since \expkv\ doesn't provide an |\ekvedef|.
%    \begin{macrocode}
\protected\long\def\ekvd@type@code#1#2#3%
  {%
    \ekvd@ifnew{}{#2}%
      {%
        \ekvd@assert@arg
          {%
            \ekvd@prot\ekvd@long#1\ekvd@tmp##1{#3}%
            \ekvd@ifalso
              {\ekv@exparg{\ekvd@add@val{#2}}{\ekvd@tmp{##1}}{}}%
              {\ekvlet\ekvd@set{#2}\ekvd@tmp}%
          }%
      }%
  }
\protected\def\ekvd@t@code{\ekvd@type@code\def}
\protected\def\ekvd@t@ecode{\ekvd@type@code\edef}
%    \end{macrocode}
% \end{macro}
%
% \begin{macro}[internal]
%   {
%     \ekvd@type@default,\ekvd@t@default,\ekvd@t@qdefault,\ekvd@t@odefault,
%     \ekvd@t@fdefault
%   }
% |\ekvd@type@default| asserts there was an argument, also the key for which one
% wants to set a default has to be already defined (this is not so important for
% |default|, but |qdefault| requires is). If everything is good, |\edef| a
% temporary macro that expands |\ekvd@set| and the |\csname| for the key, and in
% the case of |qdefault| does the first expansion step of the key-macro.
%    \begin{macrocode}
\protected\long\def\ekvd@type@default#1#2#3#4%
  {%
    \ekvd@assert@arg
      {%
        \ekvifdefined\ekvd@set{#3}%
          {%
            \ekvd@assert@not@new
            \ekvd@assert@not@long
            \ekvd@prot\edef\ekvd@tmp
              {%
                \ekv@unexpanded\expandafter#1%
                  {#2\csname\ekv@name\ekvd@set{#3}\endcsname{#4}}%
              }%
            \ekvd@ifalso
              {\ekv@exparg{\ekvd@add@noval{#3}}\ekvd@tmp{}}%
              {\ekvletNoVal\ekvd@set{#3}\ekvd@tmp}%
          }%
          {\ekvd@err@undefined@key{#3}}%
      }%
  }
\protected\def\ekvd@t@default{\ekvd@type@default{}{}}
\protected\def\ekvd@t@qdefault{\ekvd@type@default{\expandafter\expandafter}{}}
\let\ekvd@t@odefault\ekvd@t@qdefault
\protected\def\ekvd@t@fdefault{\ekvd@type@default{}{\romannumeral`\^^@}}
%    \end{macrocode}
% \end{macro}
%
% \begin{macro}[internal]{\ekvd@t@edefault}
% |edefault| is too different from |default| and |qdefault| to reuse the
% |@type@| macro, as it doesn't need |\unexpanded| inside of |\edef|.
%    \begin{macrocode}
\protected\long\def\ekvd@t@edefault#1#2%
  {%
    \ekvd@assert@arg
      {%
        \ekvifdefined\ekvd@set{#1}%
          {%
            \ekvd@assert@not@new
            \ekvd@assert@not@long
            \ekvd@prot\edef\ekvd@tmp
              {\csname\ekv@name\ekvd@set{#1}\endcsname{#2}}%
            \ekvd@ifalso
              {\ekv@exparg{\ekvd@add@noval{#1}}\ekvd@tmp{}}%
              {\ekvletNoVal\ekvd@set{#1}\ekvd@tmp}%
          }%
          {\ekvd@err@undefined@key{#1}}%
      }%
  }
%    \end{macrocode}
% \end{macro}
%
% \begin{macro}[internal]
%   {\ekvd@t@initial,\ekvd@t@oinitial,\ekvd@t@finitial,\ekvd@t@einitial}
%    \begin{macrocode}
\long\def\ekvd@type@initial#1#2#3#4%
  {%
    \ekvd@assert@not@new
    \ekvd@assert@not@also
    \ekvd@assert@not@long
    \ekvd@assert@not@protected
    \ekvd@ifnoarg
      {%
        \ekvifdefinedNoVal\ekvd@set{#3}%
          {\csname\ekv@name\ekvd@set{#3}N\endcsname}%
          {\ekvd@err@undefined@noval{#3}}%
      }%
      {%
        \ekvifdefined\ekvd@set{#3}%
          {%
            #1{#2#4}%
            \csname\ekv@name\ekvd@set{#3}\expandafter\endcsname\expandafter
              {\ekvd@tmp}%
          }%
          {\ekvd@err@undefined@key{#3}}%
      }%
  }
\def\ekvd@t@initial{\ekvd@type@initial{\def\ekvd@tmp}{}}
\def\ekvd@t@oinitial{\ekvd@type@initial{\ekv@exparg{\def\ekvd@tmp}}{}}
\def\ekvd@t@einitial{\ekvd@type@initial{\edef\ekvd@tmp}{}}
\def\ekvd@t@finitial
  {\ekvd@type@initial{\ekv@exparg{\def\ekvd@tmp}}{\romannumeral`\^^@}}
%    \end{macrocode}
% \end{macro}
%
% \begin{macro}[internal]
%   {
%     \ekvd@type@bool,\ekvd@t@bool,\ekvd@t@gbool,\ekvd@t@boolTF,\ekvd@t@gboolTF,
%     \ekvd@t@invbool,\ekvd@t@ginvbool,\ekvd@t@invboolTF,\ekvd@t@ginvboolTF,
%   }
% The boolean types are a quicker version of a |choice| that accept |true| and
% |false|, and set up the |NoVal| action to be identical to \texttt{\key=true}.
% The |true| and |false| actions are always just |\let|ting the macro in |#7| to
% some other macro (\emph{e.g.}, \cs[no-index]{iftrue}).
%    \begin{macrocode}
\protected\def\ekvd@type@bool#1#2#3#4#5%
  {%
    \ekvd@ifnew{}{#4}%
      {%
        \ekvd@ifnew{NoVal}{#4}%
          {%
            \ekvd@assert@filledarg{#5}%
              {%
                \ekvd@newlet#5#3%
                \ekvd@type@choice{#4}%
                \protected\ekvdefNoVal\ekvd@set{#4}{#1\let#5#2}%
                \protected\expandafter\def
                  \csname\ekvd@choice@name\ekvd@set{#4}{true}\endcsname
                  {#1\let#5#2}%
                \protected\expandafter\def
                  \csname\ekvd@choice@name\ekvd@set{#4}{false}\endcsname
                  {#1\let#5#3}%
              }%
          }%
      }%
  }
\protected\def\ekvd@t@bool{\ekvd@type@bool{}\iftrue\iffalse}
\protected\def\ekvd@t@gbool{\ekvd@type@bool\global\iftrue\iffalse}
\protected\def\ekvd@t@boolTF{\ekvd@type@bool{}\@firstoftwo\@secondoftwo}
\protected\def\ekvd@t@gboolTF{\ekvd@type@bool\global\@firstoftwo\@secondoftwo}
\protected\def\ekvd@t@invbool{\ekvd@type@bool{}\iffalse\iftrue}
\protected\def\ekvd@t@ginvbool{\ekvd@type@bool\global\iffalse\iftrue}
\protected\def\ekvd@t@invboolTF{\ekvd@type@bool{}\@secondoftwo\@firstoftwo}
\protected\def\ekvd@t@ginvboolTF
  {\ekvd@type@bool\global\@secondoftwo\@firstoftwo}
%    \end{macrocode}
% \end{macro}
%
% \begin{macro}[internal]
%   {
%     \ekvd@type@boolpair,
%     \ekvd@t@boolpair,\ekvd@t@gboolpair,\ekvd@t@boolpairTF,\ekvd@t@gboolpairTF
%   }
% The boolean pair types are essentially the same as the boolean types, but set
% two macros instead of one.
%    \begin{macrocode}
\protected\def\ekvd@type@boolpair#1#2#3#4#5#6%
  {%
    \ekvd@ifnew{}{#4}%
      {%
        \ekvd@ifnew{NoVal}{#4}%
          {%
            \ekvd@newlet#5#3%
            \ekvd@newlet#6#2%
            \ekvd@type@choice{#4}%
            \protected\ekvdefNoVal\ekvd@set{#4}{#1\let#5#2#1\let#6#3}%
            \protected\expandafter\def
              \csname\ekvd@choice@name\ekvd@set{#4}{true}\endcsname
              {#1\let#5#2#1\let#6#3}%
            \protected\expandafter\def
              \csname\ekvd@choice@name\ekvd@set{#4}{false}\endcsname
              {#1\let#5#3#1\let#6#2}%
          }%
      }%
  }
\protected\def\ekvd@t@boolpair#1#2%
  {\ekvd@assert@twoargs{#2}{\ekvd@type@boolpair{}\iftrue\iffalse{#1}#2}}
\protected\def\ekvd@t@gboolpair#1#2%
  {\ekvd@assert@twoargs{#2}{\ekvd@type@boolpair\global\iftrue\iffalse{#1}#2}}
\protected\def\ekvd@t@boolpairTF#1#2%
  {%
    \ekvd@assert@twoargs{#2}%
      {\ekvd@type@boolpair{}\@firstoftwo\@secondoftwo{#1}#2}%
  }
\protected\def\ekvd@t@gboolpairTF#1#2%
  {%
    \ekvd@assert@twoargs{#2}%
      {\ekvd@type@boolpair\global\@firstoftwo\@secondoftwo{#1}#2}%
  }
%    \end{macrocode}
% \end{macro}
%
% \begin{macro}[internal]
%   {
%     \ekvd@type@data,\ekvd@t@data,\ekvd@t@gdata,\ekvd@t@dataT,\ekvd@t@gdataT
%   }
%    \begin{macrocode}
\protected\def\ekvd@type@data#1#2#3#4#5#6%
  {%
    \ekvd@ifnew{}{#5}%
      {%
        \ekvd@assert@filledarg{#6}%
          {%
            \ekvd@newlet#6#1%
            \ekvd@ifalso
              {%
                \let\ekvd@prot\protected
                \ekvd@add@val{#5}{\long#2#6####1#3{####1{#4}}}{}%
              }%
              {%
                \protected\ekvd@long\ekvdef\ekvd@set{#5}%
                  {\long#2#6####1#3{####1{#4}}}%
              }%
          }%
      }%
  }
\protected\def\ekvd@t@data
  {\ekvd@type@data\@secondoftwo\edef{####2}{\ekv@unexpanded{##1}}}
\protected\def\ekvd@t@edata{\ekvd@type@data\@secondoftwo\edef{####2}{##1}}
\protected\def\ekvd@t@gdata
  {\ekvd@type@data\@secondoftwo\xdef{####2}{\ekv@unexpanded{##1}}}
\protected\def\ekvd@t@xdata{\ekvd@type@data\@secondoftwo\xdef{####2}{##1}}
\protected\def\ekvd@t@dataT
  {\ekvd@type@data\@gobble\edef{}{\ekv@unexpanded{##1}}}
\protected\def\ekvd@t@edataT{\ekvd@type@data\@gobble\edef{}{##1}}
\protected\def\ekvd@t@gdataT
  {\ekvd@type@data\@gobble\xdef{}{\ekv@unexpanded{##1}}}
\protected\def\ekvd@t@xdataT{\ekvd@type@data\@gobble\xdef{}{##1}}
%    \end{macrocode}
% \end{macro}
%
% \begin{macro}[internal]{\ekvd@type@box,\ekvd@t@box,\ekvd@t@gbox}
% Set up our boxes. Though we're a generic package we want to be colour safe, so
% we put an additional grouping level inside the box contents, for the case that
% someone uses \pkg{color}. |\ekvd@newreg| is a small wrapper which tests
% whether the first argument is defined and if not does
% |\csname new#2\endcsname#1|.
%    \begin{macrocode}
\protected\def\ekvd@type@box#1#2#3%
  {%
    \ekvd@ifnew{}{#2}%
      {%
        \ekvd@assert@filledarg{#3}%
          {%
            \ekvd@newreg#3{box}%
            \ekvd@ifalso
              {%
                \let\ekvd@prot\protected
                \ekvd@add@val{#2}{#1\setbox#3=\hbox{\begingroup##1\endgroup}}{}%
              }%
              {%
                \protected\ekvd@long\ekvdef\ekvd@set{#2}%
                  {#1\setbox#3=\hbox{\begingroup##1\endgroup}}%
              }%
          }%
      }%
  }
\protected\def\ekvd@t@box{\ekvd@type@box{}}
\protected\def\ekvd@t@gbox{\ekvd@type@box\global}
%    \end{macrocode}
% \end{macro}
%
% \begin{macro}[internal]{\ekvd@type@toks,\ekvd@t@toks,\ekvd@t@gtoks}
% Similar to |box|, but set the |toks|.
%    \begin{macrocode}
\protected\def\ekvd@type@toks#1#2#3%
  {%
    \ekvd@ifnew{}{#2}%
      {%
        \ekvd@assert@filledarg{#3}%
          {%
            \ekvd@newreg#3{toks}%
            \ekvd@ifalso
              {%
                \let\ekvd@prot\protected
                \ekvd@add@val{#2}{#1#3={##1}}{}%
              }%
              {\protected\ekvd@long\ekvdef\ekvd@set{#2}{#1#3={##1}}}%
          }%
      }%
  }
\protected\def\ekvd@t@toks{\ekvd@type@toks{}}
\protected\def\ekvd@t@gtoks{\ekvd@type@toks\global}
%    \end{macrocode}
% \end{macro}
%
% \begin{macro}[internal]{\ekvd@type@preapptoks,\ekvd@t@apptoks,\ekvd@t@gapptoks}
% Just like |toks|, but expand the current contents of the |toks| register to
% append the new contents.
%    \begin{macrocode}
\ekvd@ifprimitive\toksapp
  {%
    \protected\def\ekvd@type@preapptoks#1#2#3%
      {%
        \ekvd@ifnew{}{#2}%
          {%
            \ekvd@assert@filledarg{#3}%
              {%
                \ekvd@newreg#3{toks}%
                \ekvd@ifalso
                  {%
                    \let\ekvd@prot\protected
                    \ekvd@add@val{#2}{#1#3{##1}}{}%
                  }%
                  {\protected\ekvd@long\ekvdef\ekvd@set{#2}{#1#3{##1}}}%
              }%
          }%
      }
    \protected\def\ekvd@t@apptoks{\ekvd@type@preapptoks\toksapp}
    \protected\def\ekvd@t@gapptoks{\ekvd@type@preapptoks\gtoksapp}
    \protected\def\ekvd@t@pretoks{\ekvd@type@preapptoks\tokspre}
    \protected\def\ekvd@t@gpretoks{\ekvd@type@preapptoks\gtokspre}
  }
  {%
    \protected\def\ekvd@type@apptoks#1#2#3%
      {%
        \ekvd@ifnew{}{#2}%
          {%
            \ekvd@assert@filledarg{#3}%
              {%
                \ekvd@newreg#3{toks}%
                \ekvd@ifalso
                  {%
                    \let\ekvd@prot\protected
                    \ekvd@add@val{#2}{#1#3=\expandafter{\the#3##1}}{}%
                  }%
                  {%
                    \protected\ekvd@long\ekvdef\ekvd@set{#2}%
                      {#1#3=\expandafter{\the#3##1}}%
                  }%
              }%
          }%
      }
    \protected\def\ekvd@t@apptoks{\ekvd@type@apptoks{}}
    \protected\def\ekvd@t@gapptoks{\ekvd@type@apptoks\global}
    \newtoks\ekvd@toks
    \protected\def\ekvd@type@pretoks#1#2#3%
      {%
        \ekvd@ifnew{}{#2}%
          {%
            \ekvd@assert@filledarg{#3}%
              {%
                \ekvd@newreg#3{toks}%
                \ekvd@ifalso
                  {%
                    \let\ekvd@prot\protected
                    \ekvd@add@val{#2}%
                      {%
                        \ekvd@toks={##1}%
                        #1#3=\expandafter{\the\expandafter\ekvd@toks\the#3}%
                      }%
                      {}%
                  }%
                  {%
                    \protected\ekvd@long\ekvdef\ekvd@set{#2}%
                      {%
                        \ekvd@toks={##1}%
                        #1#3=\expandafter{\the\expandafter\ekvd@toks\the#3}%
                      }%
                  }%
              }%
          }%
      }
    \protected\def\ekvd@t@pretoks{\ekvd@type@pretoks{}}
    \protected\def\ekvd@t@gpretoks{\ekvd@type@pretoks\global}
  }
%    \end{macrocode}
% \end{macro}
%
% \begin{macro}[internal]
%   {
%     \ekvd@type@reg,\ekvd@t@int,\ekvd@t@eint,\ekvd@t@gint,\ekvd@t@xint,
%     \ekvd@t@dimen,\ekvd@t@edimen,\ekvd@t@gdimen,\ekvd@t@xdimen,
%     \ekvd@t@skip,\ekvd@t@eskip,\ekvd@t@gskip,\ekvd@t@xskip
%   }
% The |\ekvd@type@reg| can handle all the types for which the assignment will
% just be \texttt{\meta{register}=\meta{value}}.
%    \begin{macrocode}
\protected\def\ekvd@type@reg#1#2#3#4#5#6%
  {%
    \ekvd@ifnew{}{#5}%
      {%
        \ekvd@assert@filledarg{#6}%
          {%
            \ekvd@newreg#6{#1}%
            \ekvd@ifalso
              {%
                \let\evkd@prot\protected
                \ekvd@add@val{#5}{#2#6=#3##1#4\relax}{}%
              }%
              {\protected\ekvd@long\ekvdef\ekvd@set{#5}{#2#6=#3##1#4\relax}}%
          }%
      }%
  }
\protected\def\ekvd@t@int{\ekvd@type@reg{count}{}{}{}}
\protected\def\ekvd@t@eint{\ekvd@type@reg{count}{}\numexpr\relax}
\protected\def\ekvd@t@gint{\ekvd@type@reg{count}\global{}{}}
\protected\def\ekvd@t@xint{\ekvd@type@reg{count}\global\numexpr\relax}
\protected\def\ekvd@t@dimen{\ekvd@type@reg{dimen}{}{}{}}
\protected\def\ekvd@t@edimen{\ekvd@type@reg{dimen}{}\dimexpr\relax}
\protected\def\ekvd@t@gdimen{\ekvd@type@reg{dimen}\global{}{}}
\protected\def\ekvd@t@xdimen{\ekvd@type@reg{dimen}\global\dimexpr\relax}
\protected\def\ekvd@t@skip{\ekvd@type@reg{skip}{}{}{}}
\protected\def\ekvd@t@eskip{\ekvd@type@reg{skip}{}\glueexpr\relax}
\protected\def\ekvd@t@gskip{\ekvd@type@reg{skip}\global{}{}}
\protected\def\ekvd@t@xskip{\ekvd@type@reg{skip}\global\glueexpr\relax}
%    \end{macrocode}
% \end{macro}
%
% \begin{macro}[internal]{\ekvd@type@store,\ekvd@t@store,\ekvd@t@gstore}
% The none-expanding |store| types use an |\edef| or |\xdef| and |\unexpanded|
% to be able to also store |#| easily.
%    \begin{macrocode}
\protected\def\ekvd@type@store#1#2#3#4%
  {%
    \ekvd@ifnew{}{#3}%
      {%
        \ekvd@assert@filledarg{#4}%
          {%
            \ekvd@newlet#4\ekv@empty
            \ekvd@ifalso
              {%
                \let\ekvd@prot\protected
                \ekvd@add@val{#3}{#1#4{#2}}{}%
              }%
              {\protected\ekvd@long\ekvdef\ekvd@set{#3}{#1#4{#2}}}%
          }%
      }%
  }
\protected\def\ekvd@t@store{\ekvd@type@store\edef{\ekv@unexpanded{##1}}}
\protected\def\ekvd@t@gstore{\ekvd@type@store\xdef{\ekv@unexpanded{##1}}}
\protected\def\ekvd@t@estore{\ekvd@type@store\edef{##1}}
\protected\def\ekvd@t@xstore{\ekvd@type@store\xdef{##1}}
%    \end{macrocode}
% \end{macro}
%
% \begin{macro}[internal]
%   {
%     \ekvd@type@meta,\ekvd@type@meta@a,\ekvd@type@meta@b,\ekvd@type@meta@c,
%     \ekvd@t@meta,\ekvd@t@nmeta
%   }
% |meta| sets up things such that another instance of |\ekvset| will be run on
% the argument, with the same \set.
%    \begin{macrocode}
\protected\long\def\ekvd@type@meta#1#2#3#4#5#6#7%
  {%
    \ekvd@ifnew{#1}{#6}%
      {%
        \ekvd@assert@filledarg{#7}%
          {%
            \edef\ekvd@tmp{\ekvd@set}%
            \expandafter\ekvd@type@meta@a\expandafter{\ekvd@tmp}{#7}{#2}%
            \ekvd@ifalso
              {\ekv@exparg{#3{#6}}{\ekvd@tmp#4}{#5}}%
              {\csname ekvlet#1\endcsname\ekvd@set{#6}\ekvd@tmp}%
          }%
      }%
  }
\protected\long\def\ekvd@type@meta@a#1#2%
  {%
    \expandafter\ekvd@type@meta@b\expandafter{\ekvset{#1}{#2}}%
  }
\protected\def\ekvd@type@meta@b
  {%
    \expandafter\ekvd@type@meta@c\expandafter
  }
\protected\long\def\ekvd@type@meta@c#1#2%
  {%
    \ekvd@prot\ekvd@long\def\ekvd@tmp#2{#1}%
  }
\protected\def\ekvd@t@meta{\ekvd@type@meta{}{##1}\ekvd@add@val{{##1}}{}}
\protected\def\ekvd@t@nmeta
  {%
    \ekvd@assert@not@long
    \ekvd@type@meta{NoVal}{}\ekvd@add@noval{}\ekvd@assert@not@long@also
  }
%    \end{macrocode}
% \end{macro}
%
% \begin{macro}[internal]
%   {\ekvd@type@smeta,\ekvd@type@smeta@,\ekvd@t@smeta,\ekvd@t@snmeta}
% |smeta| is pretty similar to |meta|, but needs two arguments inside of \val,
% such that the first is the \set\ for which the sub-|\ekvset| and the second is
% the \kv\ list.
%    \begin{macrocode}
\protected\long\def\ekvd@type@smeta#1#2#3#4#5#6#7%
  {%
    \ekvd@ifnew{#1}{#6}%
      {%
        \ekvd@assert@twoargs{#7}%
          {%
            \ekvd@type@meta@a#7{#2}%
            \ekvd@ifalso
              {\ekv@exparg{#3{#6}}{\ekvd@tmp#4}{#5}}%
              {\csname ekvlet#1\endcsname\ekvd@set{#6}\ekvd@tmp}%
          }%
      }%
  }
\protected\def\ekvd@t@smeta{\ekvd@type@smeta{}{##1}\ekvd@add@val{{##1}}{}}
\protected\def\ekvd@t@snmeta
  {%
    \ekvd@assert@not@long
    \ekvd@type@smeta{NoVal}{}\ekvd@add@noval{}\ekvd@assert@not@long@also
  }
%    \end{macrocode}
% \end{macro}
%
% \begin{macro}[internal]
%   {
%     \ekvd@type@choice,
%     \ekvd@populate@choice,\ekvd@populate@choice@,\ekvd@populate@choice@noarg,
%     \ekvd@choice@prefix,\ekvd@choice@prefix@,\ekvd@choice@prefix@done,
%     \ekvd@choice@p@protected,\ekvd@choice@p@protect,
%     \ekvd@choice@p@long,\ekvd@choice@p@long@,
%     \ekvd@t@choice
%   }
% The |choice| type is by far the most complex type, as we have to run a
% sub-parser on the choice-definition list, which should support the |@p@| type
% prefixes as well (but |long| will always throw an error, as they are not
% allowed to be long). |\ekvd@type@choice| will just define the choice-key, the
% handling of the choices definition will be done by |\ekvd@populate@choice|.
%    \begin{macrocode}
\protected\def\ekvd@type@choice#1%
  {%
    \ekvd@assert@not@long
    \ekv@exparg{\ekvd@prot\def\ekvd@tmp##1}%
      {%
        \expandafter\ekvd@h@choice\expandafter
          {\ekvd@choice@name\ekvd@set{#1}{##1}}%
      }%
    \ekvd@ifalso
      {%
        \ekvd@assert@val{#1}%
          {%
            \ekvd@if@not@already@choice{#1}%
              {%
                \ekv@exparg
                  {%
                    \expandafter\ekvd@add@aux
                      \csname\ekv@name\ekvd@set{#1}\endcsname{{##1}}{#1}%
                  }%
                  {\ekvd@tmp{##1}}%
                  {\ekvd@long\ekvdef}\ekvd@assert@not@long@also
              }%
          }%
      }%
      {\ekvlet\ekvd@set{#1}\ekvd@tmp}%
  }
%    \end{macrocode}
% |\ekvd@populate@choice| just uses |\ekvparse| and then gives control to
% |\ekvd@populate@choice@noarg|, which throws an error, and
% |\ekvd@populate@choice@|.
%    \begin{macrocode}
\protected\def\ekvd@populate@choice
  {%
    \ekvparse\ekvd@populate@choice@noarg\ekvd@populate@choice@
  }
\protected\long\def\ekvd@populate@choice@noarg#1%
  {%
    \expandafter\ekvd@err@missing@definition@msg\expandafter{\ekvd@cur : #1}%
  }
%    \end{macrocode}
% |\ekvd@populate@choice@| runs the prefix-test, if there is none we can
% directly define the choice, for that |\ekvd@set@choice| will expand to the
% current choice-key's name, which will have been defined by |\ekvd@t@choice|.
% If there is a prefix run the prefix grabbing routine, which was altered for
% |@type@choice|.
%    \begin{macrocode}
\protected\long\def\ekvd@populate@choice@#1#2%
  {%
    \ekvd@clear@prefixes
    \ekvd@ifspace{#1}%
      {\ekvd@choice@prefix{\ekv@mark#1}\ekv@mark#1\ekv@stop}%
      {%
        \expandafter\edef
          \csname\ekvd@choice@name\ekvd@set\ekvd@set@choice{#1}\endcsname
      }%
      {\unexpanded{#2}}%
  }
\protected\def\ekvd@choice@prefix#1#2
  {%
    \ekv@strip{#2}{\ekvd@choice@prefix@{#1}}\ekv@mark
  }
\protected\def\ekvd@choice@prefix@#1#2#3\ekv@stop
  {%
    \ekv@ifdefined{ekvd@choice@p@#2}%
      {%
        \csname ekvd@choice@p@#2\endcsname
        \ekvd@ifspace{#3}%
          {\ekvd@choice@prefix{#3}#3\ekv@stop}%
          {\ekvd@choice@prefix@done{#3}}%
      }%
      {\ekvd@choice@prefix@done{#1}}%
  }
\protected\def\ekvd@choice@prefix@done#1%
  {%
    \ekvd@prot\expandafter\edef
      \csname
        \ekv@strip{#1}{\ekvd@choice@name\ekvd@set\ekvd@set@choice}%
      \endcsname
  }
\protected\def\ekvd@choice@p@protected{\let\ekvd@prot\protected}
\let\ekvd@choice@p@protect\ekvd@choice@p@protected
\protected\def\ekvd@choice@invalid@p#1\ekvd@ifspace#2%
  {%
    \expandafter\ekvd@choice@invalid@p@\expandafter{\ekv@gobble@mark#2}{#1}%
    \ekvd@ifspace{#2}%
  }
\protected\def\ekvd@choice@invalid@p@#1#2%
  {%
    \expandafter\ekvd@err@no@prefix@msg\expandafter{\ekvd@cur : #2 #1}{#2}%
  }
\protected\def\ekvd@choice@p@long{\ekvd@choice@invalid@p{long}}%
\protected\def\ekvd@choice@p@also{\ekvd@choice@invalid@p{also}}%
\protected\def\ekvd@choice@p@new{\ekvd@choice@invalid@p{new}}%
%    \end{macrocode}
% Finally we're able to set up the |@t@choice| macro, which has to store the
% current choice-key's name, define the key, and parse the available choices.
%    \begin{macrocode}
\protected\long\def\ekvd@t@choice#1#2%
  {%
    \ekvd@ifnew{}{#1}%
      {%
        \ekvd@assert@arg
          {%
            \ekvd@type@choice{#1}%
            \def\ekvd@set@choice{#1}%
            \ekvd@populate@choice{#2}%
          }%
      }%
  }
%    \end{macrocode}
% \end{macro}
%
% \begin{macro}[internal]{\ekvd@t@choice-store,\ekvd@t@choice-enum}
%   These two types define a special kind of |choice| key and are quite similar,
%   the only difference is what the different choices do (hence they use a
%   shared initialisation which differs in the chosen |populate| step).
%    \begin{macrocode}
\protected\long\expandafter\def\csname ekvd@t@choice-store\endcsname
  {\ekvd@type@choicespecial\ekvd@populate@choicestore}
\protected\long\expandafter\def\csname ekvd@t@choice-enum\endcsname
  {\ekvd@type@choicespecial\ekvd@populate@choiceenum}
%    \end{macrocode}
% \begin{macro}[internal]{\ekvd@type@choicespecial}
%   Initialise similar to a |choice| key. The difference is that we require two
%   arguments (which we assert), a macro to store things in, and a |csv|-list
%   containing the allowed values. |#1| is the |populate| macro according to the
%   type used.
%    \begin{macrocode}
\protected\long\def\ekvd@type@choicespecial#1#2#3%
  {%
    \ekvd@ifnew{}{#2}%
      {%
        \ekvd@assert@twoargs{#3}%
          {%
            \ekvd@type@choice{#2}%
            \def\ekvd@set@choice{#2}%
            #1#3%
          }%
      }%
  }
%    \end{macrocode}
% \end{macro}
% \begin{macro}[internal]
%   {\ekvd@populate@choicestore,\ekvd@populate@choicestore@}
%   We initialise the storing macro if it doesn't yet exist, and then we loop
%   over the value list. The |\edef|s with |\unexpanded| are both necessary to
%   be able to store macro parameter tokens (the outer protects at define time,
%   the inner at use time).
%    \begin{macrocode}
\protected\long\def\ekvd@populate@choicestore#1%
  {%
    \ekvd@newlet#1\ekv@empty
    \ekvcsvloop{\ekvd@populate@choicestore@#1}%
  }
\protected\long\def\ekvd@populate@choicestore@#1#2%
  {%
    \protected\expandafter\edef
      \csname\ekvd@choice@name\ekvd@set\ekvd@set@choice{#2}\endcsname
      {\unexpanded{\edef#1{\unexpanded{#2}}}}%
  }
%    \end{macrocode}
% \end{macro}
% \begin{macro}[internal]
%   {\ekvd@populate@choiceenum,\ekvd@populate@choiceenum@}
%   This is similar to the population of a |choice-store| type, but instead of
%   storing the values in a macro this initialises a count and stores the
%   position of the value in the list inside that count (zero-indexed). The
%   space is necessary to terminate the number scanning, which is the reason we
%   use |\@firstofone| (so that the space after the macro name isn't gobbled by
%   \TeX).
%    \begin{macrocode}
\protected\long\def\ekvd@populate@choiceenum#1%
  {%
    \ekvd@newreg#1{count}%
    \def\ekvd@tmp{0}%
    \ekvcsvloop{\ekvd@populate@choiceenum@#1}%
  }
\protected\long\def\ekvd@populate@choiceenum@#1#2%
  {%
    \protected\expandafter\edef
      \csname\ekvd@choice@name\ekvd@set\ekvd@set@choice{#2}\endcsname
      {#1=\@firstofone{\ekvd@tmp} }%
    \edef\ekvd@tmp{\the\numexpr\ekvd@tmp+1\relax}%
  }
%    \end{macrocode}
% \end{macro}
% \end{macro}
%
% \begin{macro}[internal]{\ekvd@t@unknown-choice}
%    \begin{macrocode}
\protected\long\expandafter\def\csname ekvd@t@unknown-choice\endcsname#1#2%
  {%
    \ekvd@assert@new@for@name{\ekvd@unknown@choice@name\ekvd@set{#1}}%
      {%
        \ekvd@assert@arg
          {%
            \ekvd@assert@not@long
            \ekvd@assert@not@also
            \ekvd@prot\expandafter
            \def\csname\ekvd@unknown@choice@name\ekvd@set{#1}\endcsname##1{#2}%
          }%
      }%
  }
%    \end{macrocode}
% \end{macro}
%
% \begin{macro}[internal]
%   {\ekvd@t@unknown,\ekvd@type@unknown@code,\ekvd@type@unknown@noval}
%   The |unknown| type has different subtypes which would be the key names for
%   other types. It is first checked whether that subtype is defined, if it
%   isn't throw an error, else use that subtype.
%    \begin{macrocode}
\protected\long\def\ekvd@t@unknown#1#2%
  {%
    \ekv@ifdefined{ekvd@type@unknown@\detokenize{#1}}%
      {\csname ekvd@type@unknown@\detokenize{#1}\endcsname{#2}}%
      \ekvd@err@misused@unknown
  }
%    \end{macrocode}
%   The |unknown noval| type can use |\ekvdefunknownNoVal| directly (after
%   asserting some prefixes).
%    \begin{macrocode}
\protected\long\def\ekvd@type@unknown@noval#1%
  {%
    \ekvd@assert@new@for@name{\ekv@name\ekvd@set{}uN}%
      {%
        \ekvd@assert@arg
          {%
            \ekvd@assert@not@also
            \ekvd@assert@not@long
            \ekvd@prot\ekvdefunknownNoVal\ekvd@set{#1}%
          }%
      }%
  }
%    \end{macrocode}
%   The |unknown code| type uses some trickery during the definition in order to
%   swap out |#1| and |#2| in the user supplied definition. This is done via a
%   temporary macro that stores the definition but gets the parameter numbers
%   reversed while the real definition is done.
%    \begin{macrocode}
\protected\long\def\ekvd@type@unknown@code#1%
  {%
    \ekvd@assert@new@for@name{\ekv@name\ekvd@set{}u}%
      {%
        \ekvd@assert@arg
          {%
            \ekvd@assert@not@also
            \begingroup
              \def\ekvd@tmp##1##2{#1}%
              \ekv@exparg
              {%
                \endgroup
                \ekvd@prot\ekvd@long\ekvdefunknown\ekvd@set
              }%
              {\ekvd@tmp{##2}{##1}}%
          }%
      }%
  }
%    \end{macrocode}
% \end{macro}
%
% \begin{macro}[internal]
%   {
%     \ekvd@type@unknown@redirect,
%     \ekvd@type@unknown@redirect-code,
%     \ekvd@type@unknown@redirect-noval
%   }
%   The |unknown redirect| types also just forward to |\ekvredirectunknown|
%   after asserting some prefixes.
%    \begin{macrocode}
\protected\edef\ekvd@type@unknown@redirect#1%
  {%
    \expandafter\noexpand\csname ekvd@type@unknown@redirect-code\endcsname{#1}%
    \expandafter\noexpand\csname ekvd@type@unknown@redirect-noval\endcsname{#1}%
  }
\protected\expandafter\def\csname ekvd@type@unknown@redirect-code\endcsname#1%
  {%
    \ekvd@assert@new@for@name{\ekv@name\ekvd@set{}u}%
      {%
        \ekvd@assert@arg
          {%
            \ekvd@assert@not@also
            \ekvd@assert@not@protected
            \expandafter\ekvredirectunknown\expandafter{\ekvd@set}{#1}%
          }%
      }%
  }
\protected\expandafter\def\csname ekvd@type@unknown@redirect-noval\endcsname#1%
  {%
    \ekvd@assert@new@for@name{\ekv@name\ekvd@set{}uN}%
      {%
        \ekvd@assert@arg
          {%
            \ekvd@assert@not@also
            \ekvd@assert@not@protected
            \ekvd@assert@not@long
            \expandafter\ekvredirectunknownNoVal\expandafter{\ekvd@set}{#1}%
          }%
      }%
  }
%    \end{macrocode}
% \end{macro}
%
% \subsubsection{Key Type Helpers}
% There are some keys that might need helpers during their execution (not during
% their definition, which are gathered as |@type@| macros). These helpers are
% named |@h@|.
%
% \begin{macro}[internal]{\ekvd@h@choice,\ekvd@h@choice@}
% The |choice| helper will just test whether the given choice was defined, if
% not throw an error expandably, else call the macro which stores the code for
% this choice.
%    \begin{macrocode}
\def\ekvd@h@choice#1%
  {%
    \expandafter\ekvd@h@choice@
      \csname\ifcsname#1\endcsname#1\else relax\fi\endcsname
      {#1}%
  }
\def\ekvd@h@choice@#1#2%
  {%
    \ifx#1\relax
      \ekvd@err@choice@invalid{#2}%
      \expandafter\@gobble
    \fi
    #1%
  }
%    \end{macrocode}
% \end{macro}
%
%
% \subsubsection{Handling \texttt{also}}
%
% \begin{macro}[internal]
%   {\ekvd@add@val,\ekvd@add@noval,\ekvd@add@aux,\ekvd@add@aux@}
%    \begin{macrocode}
\protected\long\def\ekvd@add@val#1#2#3%
  {%
    \ekvd@assert@val{#1}%
      {%
        \expandafter\ekvd@add@aux\csname\ekv@name\ekvd@set{#1}\endcsname{{##1}}%
          {#1}{#2}{\ekvd@long\ekvdef}{#3}%
      }%
  }
\protected\long\def\ekvd@add@noval#1#2#3%
  {%
    \ekvd@assert@noval{#1}%
      {%
        \expandafter\ekvd@add@aux\csname\ekv@name\ekvd@set{#1}N\endcsname{}%
          {#1}{#2}\ekvdefNoVal{#3}%
      }%
  }
\protected\long\def\ekvd@add@aux#1#2%
  {%
    \ekvd@extract@prefixes#1%
    \expandafter\ekvd@add@aux@\expandafter{#1#2}%
  }
\protected\long\def\ekvd@add@aux@#1#2#3#4#5%
  {%
    #5%
    \ekvd@prot#4\ekvd@set{#2}{#1#3}%
  }
%    \end{macrocode}
% \end{macro}
%
% \begin{macro}[internal]
%   {
%     \ekvd@extract@prefixes,\ekvd@extract@prefixes@,
%     \ekvd@extract@prefixes@long,\ekvd@extract@prefixes@prot
%   }
% This macro checks which prefixes were used for the definition of a macro and
% sets |\ekvd@long| and |\ekvd@prot| accordingly.
%    \begin{macrocode}
\protected\def\ekvd@extract@prefixes#1%
  {%
    \expandafter\ekvd@extract@prefixes@\meaning#1\ekvd@stop
  }
%    \end{macrocode}
% In the following definition |#1| will get replaced by |macro:|, |#2| by
% |\long| and |#3| by |\protected| (in each, all tokens will have category
% other). This allows us to parse the |\meaning| of a macro for those strings.
%    \begin{macrocode}
\protected\def\ekvd@extract@prefixes@#1#2#3%
  {%
    \protected\def\ekvd@extract@prefixes@##1#1##2\ekvd@stop
      {%
        \ekvd@extract@prefixes@long
          ##1\ekvd@mark\@firstofone#2\ekvd@mark\@gobble\ekvd@stop
          {\let\ekvd@long\long}%
        \ekvd@extract@prefixes@prot
          ##1\ekvd@mark\@firstofone#3\ekvd@mark\@gobble\ekvd@stop
          {\let\ekvd@prot\protected}%
      }%
    \protected\def\ekvd@extract@prefixes@long##1#2##2\ekvd@mark##3##4\ekvd@stop
      {##3}%
    \protected\def\ekvd@extract@prefixes@prot##1#3##2\ekvd@mark##3##4\ekvd@stop
      {##3}%
  }
%    \end{macrocode}
% We use a temporary macro to expand the three arguments of
% |\ekvd@extract@prefixes@|, which will set up the real meaning of itself and
% the parsing for |\long| and |\protected|.
%    \begin{macrocode}
\begingroup
\edef\ekvd@tmp
  {%
    \endgroup
    \ekvd@extract@prefixes@
      {\detokenize{macro:}}%
      {\string\long}%
      {\string\protected}%
  }
\ekvd@tmp
%    \end{macrocode}
% \end{macro}
%
%
% \subsubsection{Tests}
%
% \begin{macro}[internal]{\ekvd@newlet,\ekvd@newreg}
% These macros test whether a control sequence is defined, if it isn't they
% define it, either via |\let| or via the correct \cs[no-index]{new\meta{reg}}.
%    \begin{macrocode}
\protected\def\ekvd@newlet#1#2%
  {%
    \ifdefined#1\ekv@fi@gobble\fi\@firstofone{\let#1#2}%
  }
\protected\def\ekvd@newreg#1#2%
  {%
    \ifdefined#1\ekv@fi@gobble\fi\@firstofone{\csname new#2\endcsname#1}%
  }
%    \end{macrocode}
% \end{macro}
%
% \begin{macro}[internal]
%   {\ekvd@assert@twoargs,\ekvd@ifnottwoargs,\ekvd@ifempty@gtwo}
% A test for exactly two tokens can be reduced for an empty-test after gobbling
% two tokens, in the case that there are fewer tokens than two in the argument,
% only macros will be gobbled that are needed for the true branch, which doesn't
% hurt, and if there are more this will not be empty.
%    \begin{macrocode}
\long\def\ekvd@assert@twoargs#1%
  {%
    \ekvd@ifnottwoargs{#1}{\ekvd@err@missing@definition}%
  }
\long\def\ekvd@ifnottwoargs#1%
  {%
    \ekvd@ifempty@gtwo#1\ekv@ifempty@B
      \ekv@ifempty@false\ekv@ifempty@A\ekv@ifempty@B\@firstoftwo
  }
\long\def\ekvd@ifempty@gtwo#1#2{\ekv@ifempty@\ekv@ifempty@A}
%    \end{macrocode}
% \end{macro}
%
% \begin{macro}[internal]
%   {
%     \ekvd@assert@val,\ekvd@assert@val@,\ekvd@assert@noval,\ekvd@assert@noval@,
%     \ekvd@extract@args,\ekvd@extracted@args,\ekvd@one@arg@string
%   }
% Assert that a given key is defined as a value taking key or a |NoVal| key with
% the correct argument structure, respectively.
%    \begin{macrocode}
\protected\def\ekvd@assert@val#1%
  {%
    \ekvifdefined\ekvd@set{#1}%
      {\expandafter\ekvd@assert@val@\csname\ekv@name\ekvd@set{#1}\endcsname}%
      {%
        \ekvifdefinedNoVal\ekvd@set{#1}%
          \ekvd@err@add@val@on@noval
          {\ekvd@err@undefined@key{#1}}%
        \@gobble
      }%
  }
\protected\def\ekvd@assert@val@#1%
  {%
    \expandafter\ekvd@extract@args\meaning#1\ekvd@stop
    \unless\ifx\ekvd@extracted@args\ekvd@one@arg@string
      \ekvd@err@unsupported@arg
    \fi
    \@firstofone
  }%
\protected\def\ekvd@assert@noval#1%
  {%
    \ekvifdefinedNoVal\ekvd@set{#1}%
      {\expandafter\ekvd@assert@noval@\csname\ekv@name\ekvd@set{#1}N\endcsname}%
      {%
        \ekvifdefined\ekvd@set{#1}%
          \ekvd@err@add@noval@on@val
          {\ekvd@err@undefined@key{#1}}%
        \@gobble
      }%
  }
\protected\def\ekvd@assert@noval@#1%
  {%
    \expandafter\ekvd@extract@args\meaning#1\ekvd@stop
    \unless\ifx\ekvd@extracted@args\ekv@empty
      \ekvd@err@unsupported@arg
    \fi
    \@firstofone
  }
\protected\def\ekvd@extract@args#1%
  {%
    \protected\def\ekvd@extract@args##1#1##2->##3\ekvd@stop
      {\def\ekvd@extracted@args{##2}}%
  }
\expandafter\ekvd@extract@args\expandafter{\detokenize{macro:}}
\edef\ekvd@one@arg@string{\string#1}
%    \end{macrocode}
% \end{macro}
%
% \begin{macro}[internal]{\ekvd@assert@arg,\ekvd@assert@arg@msg,\ekvd@ifnoarg}
% There is no need to actually define |\ekvd@ifnoarg| here, as it will be set by
% either |\ekvd@arg| or |\ekvd@noarg|.
%    \begin{macrocode}
\def\ekvd@assert@arg{\ekvd@ifnoarg\ekvd@err@missing@definition}
\long\def\ekvd@assert@arg@msg#1%
  {%
    \ekvd@ifnoarg{\ekvd@err@missing@definition@msg{#1}}%
  }
%    \end{macrocode}
% \end{macro}
%
% \begin{macro}[internal]{\ekvd@assert@filledarg,\ekvd@ifnoarg@or@empty}
%    \begin{macrocode}
\long\def\ekvd@assert@filledarg#1%
  {%
    \ekvd@ifnoarg@or@empty{#1}\ekvd@err@missing@definition
  }
\long\def\ekvd@ifnoarg@or@empty#1%
  {%
    \ekvd@ifnoarg
      \@firstoftwo
      {\ekv@ifempty{#1}}%
  }
%    \end{macrocode}
% \end{macro}
%
% \begin{macro}[internal]
%   {
%     \ekvd@assert@not@long,\ekvd@assert@not@protected,\ekvd@assert@not@also
%     \ekvd@assert@not@long@also,\ekvd@assert@not@protected@also,
%     \ekvd@assert@new,\ekvd@assert@not@new
%   }
% Some key-types don't want to be |also|, |\long| or |\protected|, so we provide
% macros to test this and throw an error, this could be silently ignored but now
% users will learn to not use unnecessary stuff which slows the compilation
% down.
%    \begin{macrocode}
\def\ekvd@assert@not@long{\ifx\ekvd@long\long\ekvd@err@no@prefix{long}\fi}
\def\ekvd@assert@not@protected
  {\ifx\ekvd@prot\protected\ekvd@err@no@prefix{protected}\fi}
\def\ekvd@assert@not@also{\ekvd@ifalso{\ekvd@err@no@prefix{also}}{}}
\def\ekvd@assert@not@long@also
  {\ifx\ekvd@long\long\ekvd@err@no@prefix@also{long}\fi}
\def\ekvd@assert@not@protected@also
  {\ifx\ekvd@prot\protected\ekvd@err@no@prefix@also{protected}\fi}
\def\ekvd@assert@new#1#2%
  {\csname ekvifdefined#1\endcsname\ekvd@set{#2}{\ekvd@err@not@new}}
\def\ekvd@assert@not@new
  {\ifx\ekvd@ifnew\ekvd@assert@new\ekvd@err@no@prefix{new}\fi}
\def\ekvd@assert@new@for@name#1%
  {%
    \ifx\ekvd@ifnew\ekvd@assert@new
      \ekv@fi@firstoftwo
    \fi
    \@secondoftwo
      {\ekv@ifdefined{#1}\ekvd@err@not@new}%
      \@firstofone
  }
%    \end{macrocode}
% \end{macro}
%
% \begin{macro}[internal]
%   {
%     \ekvd@if@not@already@choice, \ekvd@if@not@already@choice@a,
%     \ekvd@if@not@already@choice@b
%   }
% It is bad to use |also| on a key that already contains a |choice|, as both
% choices would share the same valid values and thus lead to each callback being
% used twice. The following is a rudimentary test against this.
%    \begin{macrocode}
\protected\def\ekvd@if@not@already@choice#1%
  {%
    \expandafter\ekvd@if@not@already@choice@a
      \csname\ekv@name\ekvd@set{#1}\endcsname
      {}\ekvd@h@choice\ekvd@stop
  }
\protected\def\ekvd@if@not@already@choice@a
  {%
    \expandafter\ekvd@if@not@already@choice@b
  }
\long\protected\def\ekvd@if@not@already@choice@b#1\ekvd@h@choice#2\ekvd@stop
  {%
    \ekv@ifempty{#2}\@firstofone\@gobble
  }
%    \end{macrocode}
% \end{macro}
%
% \begin{macro}[internal]{\ekvd@ifspace,\ekvd@ifspace@}
% Yet another test which can be reduced to an if-empty, this time by gobbling
% everything up to the first space.
%    \begin{macrocode}
\long\def\ekvd@ifspace#1%
  {%
    \ekvd@ifspace@#1 \ekv@ifempty@B
      \ekv@ifempty@false\ekv@ifempty@A\ekv@ifempty@B\@firstoftwo
  }
\long\def\ekvd@ifspace@#1 % keep this space
  {%
    \ekv@ifempty@\ekv@ifempty@A
  }
%    \end{macrocode}
% \end{macro}
%
% \subsubsection{Messages}
%
% Most messages of \expkvd\ are not expandable, since they only appear during
% key-definition, which is not expandable anyway.
%
% \begin{macro}[internal]
%   {
%     \ekvd@errm,\ekvd@err@missing@definition,\ekvd@err@missing@definition@msg,
%     \ekvd@err@missing@type,\ekvd@err@undefined@prefix,\ekvd@err@undefined@key,
%     \ekvd@err@no@prefix,\ekvd@err@no@prefix@msg,\ekvd@err@no@prefix@also,
%     \ekvd@err@add@val@on@noval,\ekvd@err@add@noval@on@val,
%     \ekvd@err@unsupported@arg,\ekvd@err@not@new
%   }
% The non-expandable error messages are boring, so here they are:
%    \begin{macrocode}
\protected\def\ekvd@errm#1{\errmessage{expkv-def Error: #1}}
\protected\def\ekvd@err@missing@definition
  {\ekvd@errm{Missing definition for key `\ekvd@cur'}}
\protected\def\ekvd@err@missing@definition@msg#1%
  {\ekvd@errm{Missing definition for key `\ekv@unexpanded{#1}'}}
\protected\def\ekvd@err@missing@type
  {\ekvd@errm{Missing type prefix for key `\ekvd@cur'}}
\protected\def\ekvd@err@undefined@prefix#1%
  {%
    \ekvd@errm
      {%
        Undefined prefix `\ekv@unexpanded{#1}' found while processing
        `\ekvd@cur'%
      }%
  }
\protected\def\ekvd@err@undefined@key#1%
  {%
    \ekvd@errm
      {Undefined key `\ekv@unexpanded{#1}' found while processing `\ekvd@cur'}%
  }
\protected\def\ekvd@err@undefined@noval#1%
  {%
    \ekvd@errm
      {%
        Undefined noval key `\unexpanded{#1}' found while processing
        `\ekvd@cur'%
      }%
  }
\protected\def\ekvd@err@no@prefix#1%
  {\ekvd@errm{prefix `#1' not accepted in `\ekvd@cur'}}
\protected\def\ekvd@err@no@prefix@msg#1#2%
  {\ekvd@errm{prefix `#2' not accepted in `\ekv@unexpanded{#1}'}}
\protected\def\ekvd@err@no@prefix@also#1%
  {\ekvd@errm{`\ekvd@cur' not allowed with a `#1' key}}
\protected\def\ekvd@err@add@val@on@noval
  {\ekvd@errm{`\ekvd@cur' not allowed with a NoVal key}}
\protected\def\ekvd@err@add@noval@on@val
  {\ekvd@errm{`\ekvd@cur' not allowed with a value taking key}}
\protected\def\ekvd@err@unsupported@arg\fi\@firstofone#1%
  {%
    \fi
    \ekvd@errm
      {%
        Existing key-macro has the unsupported argument string
        `\ekvd@extracted@args' for key `\ekvd@cur'%
      }%
  }
\protected\def\ekvd@err@not@new
  {\ekvd@errm{The key for `\ekvd@cur' is already defined}}
\protected\long\def\ekvd@err@misused@unknown
  {\ekvd@errm{Misuse of the unknown type found while processing `\ekvd@cur'}}
%    \end{macrocode}
% \end{macro}
%
% \begin{macro}[internal]
%   {
%     \ekvd@err@choice@invalid,\ekvd@err@choice@invalid@,\ekvd@choice@name,
%     \ekvd@unknown@choice@name
%   }
% |\ekvd@err@choice@invalid| will have to use this mechanism to throw its
% message. Also we have to retrieve the name parts of the choice in an easy way,
% so we use parentheses of catcode 8 here, which should suffice in most cases to
% allow for a correct separation.
%    \begin{macrocode}
\def\ekvd@err@choice@invalid#1%
  {%
    \ekvd@err@choice@invalid@#1%
  }
\begingroup
\catcode40=8
\catcode41=8
\@firstofone{\endgroup
\def\ekvd@choice@name#1#2#3%
  {%
    ekvd#1(#2)\detokenize{#3}%
  }
\def\ekvd@unknown@choice@name#1#2%
  {%
    ekvd:u:#1(#2)%
  }
\def\ekvd@err@choice@invalid@ ekvd#1(#2)\detokenize#3%
  {%
    \ekv@ifdefined{\ekvd@unknown@choice@name{#1}{#2}}%
      {\csname\ekvd@unknown@choice@name{#1}{#2}\endcsname{#3}}%
      {\ekvd@err{invalid choice `#3' for `#2' in set `#1'}}%
  }
}
%    \end{macrocode}
% \end{macro}
%
% \begin{macro}[internal]{\ekvd@err}
% The expandable error messages use |\ekvd@err|, which is just like |\ekv@err|
% from \expkv. It uses a runaway argument to start the error message.
%    \begin{macrocode}
\ekv@exparg{\long\def\ekvd@err#1}{\ekverr{expkv-def}{#1}}
%    \end{macrocode}
% \end{macro}
%
% Now everything that's left is to reset the category code of |@|.
%    \begin{macrocode}
\catcode`\@=\ekvd@tmp
%    \end{macrocode}
%
% \gobbledocstriptag
%</tex>
%^^A=<<
%
%^^A=<<
%
% \end{implementation}^^A=<<
%
% \clearpage
% \PrintIndex
%
\endinput
%
^^A vim: ft=tex fdm=marker fmr=>>=,=<<
}{}
\ProvidesFile{expkv-def.dtx}
  [\csname ekvdDate\endcsname\ a key-defining frontend for expkv]
\PassOptionsToPackage{full}{textcomp}
\documentclass{l3doc}
\RequirePackage[oldstylenums,nott]{kpfonts}
\input{glyphtounicode}
\pdfgentounicode=1
\RequirePackage{tcolorbox}
\newtcolorbox{exresult}[2][]
  {colback=expkvgrey!10!white,colframe=expkvgrey,fontupper=\small,width={#2},#1}
\newtcbox\exres[1][]
  {
     colback=expkvgrey!10!white
    ,colframe=expkvgrey
    ,size=small
    ,nobeforeafter
    ,tcbox raise base
    ,fontupper=\small
    ,#1
  }
\RequirePackage{listings}
\RequirePackage{booktabs}
\RequirePackage{array}
\RequirePackage{collcell}
\RequirePackage{siunitx}
\RequirePackage{xcolor}
\RequirePackage{caption}
\RequirePackage{microtype}
\RequirePackage{accsupp}
\lstdefinelanguage{expkvd}[primitive]{TeX}
  {
    ,moretexcs=[4]^^A e-TeX
      {
        expanded,
        numexpr,
        protected,
      }
    ,moretexcs=[5]^^A plain/LaTeX
      {
        approx,
        begin,
        empty,
        item,
        LaTeX,
        makeatletter,makeatother,
        newcommand,newdimen,
        RequirePackage,
        rule,
        TeX,
        textit,texttt,
        usepackage,
      }
    ,moretexcs=[6]^^A used packages
      {
        ^^A xfp
        fpeval,
        ^^A keyval
        setkeys,
        ^^A yax
        defactiveparameter,storevalue,setparameterlist
      }
    ,moretexcs=[2]^^A expkv-def macros
      {
        ekvdefinekeys
      }
    ,moretexcs=[3]^^A expkv-pkg macros
      {
        ^^A expkv
        ekvdef,ekvdefNoVal,^^A
        ekvlet,ekvletNoVal,ekvletkv,ekvletkvNoVal,^^A
        ekvdefunknown,ekvdefunknownNoVal,^^A
        ekvredirectunknown,ekvredirectunknownNoVal,^^A
        ekvset,ekvsetSneaked,^^A
        ekvsetdef,ekvsetSneakeddef,ekvsetdefSneaked,^^A
        ekvparse,^^A
        ekvVersion,ekvDate,^^A
        ekvifdefined,ekvifdefinedNoVal,ekvifdefinedset,^^A
        ekvbreak,ekvbreakPreSneak,ekvbreakPostSneak,^^A
        ekvsneak,ekvsneakPre,^^A
        ekvchangeset,^^A
        ekvoptarg,ekvoptargTF,^^A
        ekverr,
        ^^A expkv-cs
        ekvcSecondaryKeys,ekvcSplitAndForward,
      }
  }
\colorlet{codeparam}{cyan!65!black}
\def\mylstwd{.55em}
\lstset
  {
    ,language=expkvd
    ,flexiblecolumns=false
    ,basewidth=\mylstwd
    ,gobble=2
    ,basicstyle=\fontfamily{jkp}\itshape
    ,texcsstyle=*[1]{\bfseries\color{expkvgrey}} ^^A primitives
    ,texcsstyle=*[2]{\bfseries\color{expkvred}}  ^^A expkv
    ,texcsstyle=*[3]{\color{expkvred}}           ^^A expkv-pkg
    ,texcsstyle=*[4]{\bfseries\color{expkvgrey}} ^^A e-TeX
    ,texcsstyle=*[5]{\bfseries\color{expkvgrey}} ^^A plain/LaTeX macros
    ,texcsstyle=*[6]{}                           ^^A macros of other packages
    ,commentstyle=\color[gray]{0.4}
    ,literate=
              {\{} {{\CodeUpBf\{}}{1}
              {\}} {{\CodeUpBf\}}}{1}
              {$}  {{\CodeUpBf\$}}{1}
              {[}  {{\CodeUp[}}{1}
              {]}  {{\CodeUp]}}{1}
              {(}  {{\CodeUp(}}{1}
              {)}  {{\CodeUp)}}{1}
              {*}  {{$*$}}{1}
              {1}  {{\CodeUp{1}}}{1}
              {2}  {{\CodeUp{2}}}{1}
              {3}  {{\CodeUp{3}}}{1}
              {4}  {{\CodeUp{4}}}{1}
              {5}  {{\CodeUp{5}}}{1}
              {6}  {{\CodeUp{6}}}{1}
              {7}  {{\CodeUp{7}}}{1}
              {8}  {{\CodeUp{8}}}{1}
              {9}  {{\CodeUp{9}}}{1}
              {0}  {{\CodeUp{0}}}{1}
              {##} {{\CodeColored{codeparam}{1}{\#}}}{1}
              {##1}{{\CodeColored{codeparam}{2}{\#1}}}{2}
              {##2}{{\CodeColored{codeparam}{2}{\#2}}}{2}
              {##3}{{\CodeColored{codeparam}{2}{\#3}}}{2}
              {##4}{{\CodeColored{codeparam}{2}{\#4}}}{2}
              {##5}{{\CodeColored{codeparam}{2}{\#5}}}{2}
              {##6}{{\CodeColored{codeparam}{2}{\#6}}}{2}
              {##7}{{\CodeColored{codeparam}{2}{\#7}}}{2}
              {##8}{{\CodeColored{codeparam}{2}{\#8}}}{2}
              {##9}{{\CodeColored{codeparam}{2}{\#9}}}{2}
              {<key>}{{$\langle$}key{$\rangle$}}{5}
              {<set>}{{$\langle$}set{$\rangle$}}{5}
              {<set1>}{{$\langle$}set1{$\rangle$}}{6}
              {<set2>}{{$\langle$}set2{$\rangle$}}{6}
    ^^A,literate=*{<key>}{\key}{4}{<set>}{\set}{4}
  }
\newcommand*\CodeColored[3]{\textcolor{#1}{\makebox[\dimexpr\mylstwd*#2]{#3}}}
\newcommand*\CodeUpBf[1]{\makebox[\mylstwd]{\textup{\textbf{#1}}}}
\newcommand*\CodeUp[1]{\makebox[\mylstwd]{\textup{#1}}}
\RequirePackage{randtext}
\let\metaORIG\meta
\protected\def\meta #1{\texttt{\metaORIG{#1}}}
\renewcommand*\thefootnote{\fnsymbol{footnote}}
\definecolor{expkvred}{HTML}{9F393D}
\colorlet{expkvgrey}{black!75}
\makeatletter
\newcommand*\expkv
  {^^A
    \texorpdfstring
      {^^A
        \mbox
          {^^A
            \BeginAccSupp{ActualText=expkv}^^A
            \href{https://github.com/Skillmon/tex_expkv}
              {^^A
                \rmfamily
                \bfseries
                {\color{expkvgrey}e\kern-.05em x\kern-.05em}^^A
                \lower.493ex
                  \hbox{{\color{expkvgrey}P}\kern-.1em{\color{expkvred}k}}^^A
                \kern-.18em{\color{expkvred}v}^^A
              }^^A
            \EndAccSupp{}^^A
          }^^A
      }
      {expkv}^^A
  }
\newcommand*\expkvd
  {^^A
    \texorpdfstring
      {^^A
        \mbox
          {^^A
            \BeginAccSupp{ActualText=expkv-def}^^A
            \href{https://github.com/Skillmon/tex_expkv-def}
              {^^A
                \rmfamily
                \bfseries
                {\color{expkvgrey}e\kern-.05em x\kern-.05em}^^A
                \lower.493ex
                  \hbox{{\color{expkvgrey}P}\kern-.1em{\color{expkvred}k}}^^A
                \kern-.18em{\color{expkvred}v}^^A
                {\color{expkvgrey}^^A
                  \kern.05em\rule[-.1ex]{.08em}{1.2ex}\kern.05em\textsc{def}^^A
                }^^A
              }^^A
            \EndAccSupp{}^^A
          }^^A
      }
      {expkv-def}^^A
  }
\newcommand\kv{\meta{key}=\meta{value}}
\newcommand\key{\meta{key}}
\newcommand\val{\meta{value}}
\newcommand\set{\meta{set}}
\newcommand\enfprefix{\textcolor{black}}
\newcommand\allprefix{\textcolor{gray}}
\newcommand\notprefix{\textcolor{red!80!black}}
\newcommand\printprefix[2]
  {^^A
    \ifcase\numexpr#1\relax\or
      \enfprefix{#2}\or
      \allprefix{#2}\or
      \notprefix{#2}\fi
  }
\newcommand\prefixes[4]
  {^^A
    \hfill
    \printprefix{#1}{new}
    \printprefix{#2}{also}
    \printprefix{#3}{protected}
    \printprefix{#4}{long}^^A
  }
\hypersetup{linkcolor=red!80!black,urlcolor=purple!80!black}
\DoNotIndex{\def,\edef,\,,\=,\begingroup,\catcode,\chardef,\csname,\endcsname}
\DoNotIndex{\endgroup,\endinput,\errmessage,\expandafter,\input,\let,\long}
\DoNotIndex{\protected,\ProvidesFile,\ProvidesPackage,\relax,\space}
\DoNotIndex{\@,\unexpanded,\string,\expanded,\dimexpr,\global,\glueexpr,\hbox}
\DoNotIndex{\numexpr,\RequirePackage,\setbox,\the,\unless,\xdef,\@firstofone}
\DoNotIndex{\@firstoftwo,\@gobble,\@secondoftwo,\meaning,\detokenize}
\DoNotIndex{\romannumeral,\^}
\DoNotIndex{\ifcsname}
\DoNotIndex{\ifx}
\DoNotIndex{\ifdefined}
\DoNotIndex{\iffalse}
\DoNotIndex{\iftrue}
\DoNotIndex{\else}
\DoNotIndex{\fi}
\@gobble\fi ^^A ignoring \ifx, \ifcsname, etc., but only one \fi
\@gobble\fi ^^A ignoring \ifx, \ifcsname, etc., but only one \fi
\@gobble\fi ^^A ignoring \ifx, \ifcsname, etc., but only one \fi
\@gobble\fi ^^A ignoring \ifx, \ifcsname, etc., but only one \fi
\@ifdefinable\gobbledocstriptag{\def\gobbledocstriptag#1>{}}
\makeatother
\begin{document}
  \title
    {^^A
      \texorpdfstring
        {^^A
          \huge\expkvd
          \\[\medskipamount]
          \Large a key-defining frontend for \expkv
        }{expkv-def - a key-defining frontend for expkv}^^A
    }
  \date{\ekvdDate\space v\ekvdVersion}
  \author{Jonathan P. Spratte\thanks{\protect\randomize{jspratte@yahoo.de}}}
  \DocInput{expkv-def.dtx}
\end{document}
%</driver>^^A=<<
% \fi
%
% \maketitle
% \renewcommand*\thefootnote{\arabic{footnote}}
%
% \begin{abstract}
% \noindent\parfillskip=0pt
% \expkvd\ provides a small \kv\ interface to define keys for \expkv. Key-types
% are declared using prefixes, similar to static typed languages. The stylised
% name is \expkvd\ but the files use \file{expkv-def}, this is due to CTAN-rules
% which don't allow \string| in package names since that is the pipe symbol in
% *nix shells.
% \end{abstract}
%
% \tableofcontents
%
% \begin{documentation}^^A>>=
%
% \section{Documentation}
%
% Since the trend for the last couple of years goes to defining keys for a \kv\
% interface using a \kv\ interface, I thought that maybe providing such an
% interface for \expkv\ will make it more attractive for actual use, besides its
% unique selling points of being fully expandable, and fast and reliable. But at
% the same time I don't want to widen \expkv's initial scope. So here it is
% \expkvd, go define \kv\ interfaces with \kv\ interfaces.
%
% Unlike many of the other established \kv\ interfaces to define keys, \expkvd\
% works using prefixes instead of suffixes (\emph{e.g.}, |.tl_set:N| of
% \pkg{l3keys}) or directory like handlers (\emph{e.g.}, |/.store in| of
% \pkg{pgfkeys}). This was decided as a personal preference, more over in \TeX\
% parsing for the first space is way easier than parsing for the last one.
% \expkvd's prefixes are sorted into two categories: |p|-type, which are
% equivalent to \TeX's prefixes like |\long|, and |t|-type defining the type of
% the key. For a description of the available |p|-prefixes take a look at
% \autoref{sec:prefix:p}, the |t|-prefixes are described in
% \autoref{sec:prefix:t}.
%
% \expkvd\ is usable as generic code, as a \LaTeX\ package, and as a \ConTeXt\
% module. It'll automatically load \expkv\ in the same mode as well. To use it,
% just use one of
% \begin{lstlisting}
% \input expkv-def       % plainTeX
% \usepackage{expkv-def} % LaTeX
% \usemodule[expkv-def]  % ConTeXt
% \end{lstlisting}
%
% \subsection{Macros}
% Apart from version and date containers there is only a single user-facing
% macro, and that should be used to define keys.
%
% \begin{function}{\ekvdefinekeys}
%   \begin{syntax}
%     \cs{ekvdefinekeys}\marg{set}\{\kv, \ldots\}
%   \end{syntax}
%   In \set, define \key\ to have definition \val. The general syntax for \key\
%   should be
%   \begin{quote}\ttfamily\small
%     \meta{prefix} \meta{name}
%   \end{quote}
%   Where \meta{prefix} is a space separated list of optional |p|-type prefixes
%   followed by one |t|-type prefix. The syntax of \val\ is dependent on the
%   used |t|-prefix.
% \end{function}
%
% \begin{function}{\ekvdDate,\ekvdVersion}
%   These two macros store the version and date of the package.
% \end{function}
%
% \subsection{Prefixes}
% As already said there are |p|-prefixes and |t|-prefixes. Not every |p|-prefix
% is allowed for all |t|-prefixes.
%
% \subsubsection{\texttt{p}-Prefixes}\label{sec:prefix:p}
% The two |p|-type prefixes |long| and |protected| are pretty simple by nature,
% so their description is pretty simple. They affect the \key\ at use-time, so
% omitting |long| doesn't mean that a \meta{definition} can't contain a |\par|
% token, only that the \key\ will not accept a |\par| in \val.  On the other
% hand |new| and |also| might be simple on first sight as well, but their rules
% are a bit more complicated.
%
% \begin{function}{also}
%   The following key type will be \emph{added} to an existing \key's
%   definition. You can't add a type taking an argument at use time to an
%   existing key which doesn't take an argument and vice versa. Also you'll get
%   an error if you try to add an action which isn't allowed to be either |long|
%   or |protected| to a key which already is |long| or |protected| (the opposite
%   order would be suboptimal as well, but can't be really captured with the
%   current code).
%
%   A key already defined as |long| or |protected| will stay |long| or
%   |protected|, but you can as well add |long| or |protected| with the |also|
%   definition.
%
%   As a small example, suppose you want to create a boolean key, but
%   additionally to setting a boolean value you want to execute some more code
%   as well, you can use the following
% \begin{lstlisting}
% \ekvdefinekeys{also-example}
%   {
%      bool key      = \ifmybool
%     ,also code key = \domystuff{#1}
%   }
% \end{lstlisting}
%
%   If you use |also| on a |choice|, |bool|, |invbool|, or |boolpair| key it is
%   tried to determine if the key already is of one of those types. If this test
%   is true the declared choices will be added to the possible choices but the
%   key's definition will not be changed other than that. If that wouldn't have
%   been done, the callbacks of the different choices could get called multiple
%   times.
% \end{function}
%
% \begin{function}{protected,protect}
%   The following key will be defined |\protected|. Note that key-types which
%   can't be defined expandable will always use |\protected|.
% \end{function}
%
% \begin{function}{long}
%   The following key will be defined |\long|.
% \end{function}
%
% \begin{function}{new}
%   The following key must be new (so previously undefined). An error is thrown
%   if it is already defined and the new definition is ignored. |new| only
%   asserts that there are no conflicts between |NoVal| keys and other |NoVal|
%   keys or value taking keys and other value taking keys. For example you can
%   use the following without an error:
% \begin{lstlisting}
% \ekvdefinekeys{new-example}
%   {
%      code key       = \domystuffwitharg{#1}
%     ,new  noval key = \domystuffwithoutarg
%   }
% \end{lstlisting}
% \end{function}
%
% \subsubsection{\texttt{t}-Prefixes}\label{sec:prefix:t}
% Since the |p|-type prefixes apply to some of the |t|-prefixes automatically
% but sometimes one might be disallowed we need some way to highlight this
% behaviour. In the following an enforced prefix will be printed black
% (\texttt{\enfprefix{protected}}), allowed prefixes will be grey
% (\texttt{\allprefix{protected}}), and disallowed prefixes will be red
% (\texttt{\notprefix{protected}}). This will be put flush-right in the syntax
% showing line.
%
% \begin{function}{code,ecode}
%   \begin{syntax}
%     code \key\ = \marg{definition} \prefixes2222
%   \end{syntax}
%   Define \key\ to expand to \meta{definition}. The \key\ will require a \val\
%   for which you can use |#1| inside \meta{definition}. The |ecode| variant
%   will fully expand \meta{definition} inside an |\edef|.
% \end{function}
%
% \begin{function}{noval,enoval}
%   \begin{syntax}
%     noval \key\ = \marg{definition} \prefixes2223
%   \end{syntax}
%   The |noval| type defines \key\ to expand to \meta{definition}. The \key\
%   will not take a \val. |enoval| fully expands \meta{definition} inside an
%   |\edef|.
% \end{function}
%
% \begin{function}{default,qdefault,odefault,fdefault,edefault}
%   \begin{syntax}
%     default \key\ = \marg{definition} \prefixes3223
%   \end{syntax}
%   This serves to place a default \val\ for a \key\ that takes an argument, the
%   \key\ can be of any argument-grabbing kind, and when used without a \val\
%   it will be passed \meta{definition} instead. The |qdefault| variant will
%   expand the \key's code once, so will be slightly quicker, but not change if
%   you redefine \key. |odefault| is just another name for |qdefault|. The
%   |fdefault| version will expand the key code until a non-expandable token or
%   a space is found, a space would be gobbled.\footnotemark{} The |edefault| on
%   the other hand fully expands the \key-code with \meta{definition} as its
%   argument inside of an |\edef|.
% \end{function}
% \footnotetext
%   {^^A
%     For those familiar with \TeX-coding: This uses a
%     \cs[no-index]{romannumeral}-expansion.
%   }
%
% \begin{function}{initial,oinitial,finitial,einitial}
%   \begin{syntax}
%     initial \key\ = \{\val\} \prefixes3333
%     initial \key
%   \end{syntax}
%   With |initial| you can set an initial \val\ for an already defined argument
%   taking \key. It'll just call the key-macro of \key\ and pass it \val. The
%   |einitial| variant will expand \val\ using an |\edef| expansion prior to
%   passing it to the key-macro and the |oinitial| variant will expand the first
%   token in \val\ once. |finitial| will expand \val\ until a non-expandable
%   token or a space is found, a space would be
%   gobbled.\footnotemark
%
%   If you don't provide a value (and no equals sign) a |noval| \key\ of the
%   same name is called once (or, if you specified a |default| for a value
%   taking key that would be used).
% \end{function}
% \footnotetext{Again using \cs[no-index]{romannumeral}.}
%
% \begin{function}{bool,gbool,boolTF,gboolTF}
%   \begin{syntax}
%     bool \key\ = \meta{cs} \prefixes2223
%   \end{syntax}
%   The \meta{cs} should be a single control sequence, such as |\iffoo|.
%   This will define \key\ to be a boolean key, which only takes the values
%   |true| or |false| and will throw an error for other values. If the key is
%   used without a \val\ it'll have the same effect as if you use
%   \texttt{\key=true}. |bool| and |gbool| will behave like \TeX-ifs so either
%   be |\iftrue| or |\iffalse|. The |boolTF| and |gboolTF| variants will both
%   take two arguments and if true the first will be used else the second, so
%   they are always either |\@firstoftwo| or |\@secondoftwo|. The variants with
%   a leading |g| will set the control sequence globally, the others locally. If
%   \meta{cs} is not yet defined it'll be initialised as the |false| version.
%   Note that the initialisation is \emph{not} done with |\newif|, so you will
%   not be able to do |\footrue| outside of the \kv\ interface, but you could
%   use |\newif| yourself. Even if the \key\ will not be |\protected| the
%   commands which execute the |true| or |false| choice will be, so the usage
%   should be safe in an expansion context (\emph{e.g.}, you can use
%   \texttt{edefault \key\ = false} without an issue to change the default
%   behaviour to execute the |false| choice). Internally a |bool| \key\ is the
%   same as a |choice| key which is set up to handle |true| and |false| as
%   choices.
% \end{function}
%
% \begin{function}{invbool,ginvbool,invboolTF,ginvboolTF}
%   \begin{syntax}
%     bool \key\ = \meta{cs} \prefixes2223
%   \end{syntax}
%   These are inverse boolean keys, they behave like |bool| and friends but set
%   the opposite meaning to the macro \meta{cs} in each case. So if
%   \texttt{key=true} is used |invbool| will set \meta{cs} to |\iffalse| and
%   vice versa.
% \end{function}
%
% \begin{function}{boolpair,gboolpair,boolpairTF,gboolpairTF}
%   \begin{syntax}
%     boolpair \key\ = \meta{cs_1}\meta{cs_2} \prefixes2223
%   \end{syntax}
%   The |boolpair| key type behaves like both |bool| and |invbool|, the
%   \meta{cs_1} will be set to the meaning according to the rules of |bool|, and
%   \meta{cs_2} will be set to the opposite.
% \end{function}
%
% \begin{function}{store,estore,gstore,xstore}
%   \begin{syntax}
%     store \key\ = \meta{cs} \prefixes2212
%   \end{syntax}
%   The \meta{cs} should be a single control sequence, such as |\foo|.
%   This will define \key\ to store \val\ inside of the control sequence. If
%   \meta{cs} isn't yet defined it will be initialised as empty. The variants
%   behave similarly to their |\def|, |\edef|, |\gdef|, and |\xdef|
%   counterparts, but |store| and |gstore| will allow you to store macro
%   parameters inside of them by using |\unexpanded|.
% \end{function}
%
% \begin{function}{data,edata,gdata,xdata}
%   \begin{syntax}
%     data \key\ = \meta{cs} \prefixes2212
%   \end{syntax}
%   The \meta{cs} should be a single control sequence, such as |\foo|. This will
%   define \key\ to store \val\ inside of the control sequence. But unlike the
%   |store| type, the macro \meta{cs} will be a switch at the same time, it'll
%   take two arguments and if \key\ was used expands to the first argument
%   followed by \val\ in braces, if \key\ was not used \meta{cs} will expand to
%   the second argument (so behave like |\@secondoftwo|). The idea is that with
%   this type you can define a key which should be typeset formatted. The
%   |edata| and |xdata| variants will fully expand \val, the |gdata| and |xdata|
%   variants will store \val\ inside \meta{cs} globally. The |p|-prefixes will
%   only affect the key-macro, \meta{cs} will always be expandable and |\long|.
% \end{function}
%
% \begin{function}{dataT,edataT,gdataT,xdataT}
%   \begin{syntax}
%     dataT \key\ = \meta{cs} \prefixes2212
%   \end{syntax}
%   Just like |data|, but instead of \meta{cs} grabbing two arguments it'll only
%   grab one, so by default it'll behave like |\@gobble|, and if a \val\ was
%   given to \key\ the \meta{cs} will behave like |\@firstofone| appended by
%   \marg{value}.
% \end{function}
%
% \begin{function}{int,eint,gint,xint}
%   \begin{syntax}
%     int \key\ = \meta{cs} \prefixes2212
%   \end{syntax}
%   The \meta{cs} should be a single control sequence, such as |\foo|.
%   An |int| key will be a \TeX-count register. If \meta{cs} isn't defined yet,
%   |\newcount| will be used to initialise it. The |eint| and |xint| versions
%   will use |\numexpr| to allow basic computations in their \val. The |gint|
%   and |xint| variants set the register globally.
% \end{function}
%
% \begin{function}{dimen,edimen,gdimen,xdimen}
%   \begin{syntax}
%     dimen \key\ = \meta{cs} \prefixes2212
%   \end{syntax}
%   The \meta{cs} should be a single control sequence, such as |\foo|.
%   This is just like |int| but uses a dimen register, |\newdimen| and
%   |\dimexpr| instead.
% \end{function}
%
% \begin{function}{skip,eskip,gskip,xskip}
%   \begin{syntax}
%     skip \key\ = \meta{cs} \prefixes2212
%   \end{syntax}
%   The \meta{cs} should be a single control sequence, such as |\foo|.
%   This is just like |int| but uses a skip register, |\newskip| and |\glueexpr|
%   instead.
% \end{function}
%
% \begin{function}{toks,gtoks,apptoks,gapptoks,pretoks,gpretoks}
%   \begin{syntax}
%     toks \key\ = \meta{cs} \prefixes2212
%   \end{syntax}
%   The \meta{cs} should be a single control sequence, such as |\foo|.
%   Store \val\ inside of a toks-register. The |g| variants use |\global|, the
%   |app| variants append \val\ to the contents of that register, the |pre|
%   variants will prepend \val. If \meta{cs} is not yet defined it will be
%   initialised with |\newtoks|.
% \end{function}
%
% \begin{function}{box,gbox}
%   \begin{syntax}
%     box \key\ = \meta{cs} \prefixes2212
%   \end{syntax}
%   The \meta{cs} should be a single control sequence, such as |\foo|.
%   Typesets \val\ into a |\hbox| and stores the result in a box register. The
%   boxes are colour safe. \expkvd\ doesn't provide a |vbox| type.
% \end{function}
%
% \begin{function}{meta}
%   \begin{syntax}
%     meta \key\ = \{\kv, \ldots\} \prefixes2222
%   \end{syntax}
%   This key type can set other keys, you can access the \val\ which was passed
%   to \key\ inside the \kv\ list with |#1|. It works by calling a sub-|\ekvset|
%   on the \kv\ list, so a |set| key will only affect that \kv\ list and not the
%   current |\ekvset|. Since it runs in a separate |\ekvset| you can't use
%   |\ekvsneak| using keys or similar macros in the way you normally could.
% \end{function}
%
% \begin{function}{nmeta}
%   \begin{syntax}
%     nmeta \key\ = \{\kv, \ldots\} \prefixes2223
%   \end{syntax}
%   This key type can set other keys, the difference to |meta| is, that this key
%   doesn't take a value, so the \kv\ list is static.
% \end{function}
%
% \begin{function}{smeta}
%   \begin{syntax}
%     smeta \key\ = \marg{set}\{\kv, \ldots\} \prefixes2222
%   \end{syntax}
%   Yet another |meta| variant. An |smeta| key will take a \val\ which you can
%   access using |#1|, but it sets the \kv\ list inside of \set, so is equal to
%   \texttt{\cs[no-index]{ekvset}\marg{set}\{\kv, \ldots\}}.
% \end{function}
%
% \begin{function}{snmeta}
%   \begin{syntax}
%     snmeta \key\ = \marg{set}\{\kv, \ldots\} \prefixes2223
%   \end{syntax}
%   And the last |meta| variant. |snmeta| is a combination of |smeta| and
%   |nmeta|. It doesn't take an argument and sets the \kv\ list inside of \set.
% \end{function}
%
% \begin{function}{set}
%   \begin{syntax}
%     set \key\ = \marg{set} \prefixes2233
%   \end{syntax}
%   This will define \key\ to change the set of the current |\ekvset|
%   invocation to \set. You can omit \set\ (including the equals sign),
%   which is the same as using \texttt{set \key\ = \{\key\}}. The created |set|
%   key will not take a \val. Note that just like in \expkv\ it'll not be
%   checked whether \set\ is defined and you'll get a low-level \TeX\ error if
%   you use an undefined \set.
% \end{function}
%
% \begin{function}{choice}
%   \begin{syntax}
%     choice \key\ = \{\val=\meta{definition}, \ldots\} \prefixes2223
%   \end{syntax}
%   Defines \key\ to be a |choice| key, meaning it will only accept a limited
%   set of values. You should define each possible \val\ inside of the
%   \val=\meta{definition} list. If a defined \val\ is passed to \key\ the
%   \meta{definition} will be left in the input stream. You can make
%   individual values |protected| inside the \val=\meta{definition} list. By
%   default a |choice| key is expandable, an undefined \val\ will throw an error
%   in an expandable way (but see the |unknown-choice| prefix). You can add
%   additional choices after the \key\ was created by using |choice| again for
%   the same \key, redefining choices is possible the same way, but there is no
%   interface to remove certain choices.
% \end{function}
%
% \begin{function}{choice-store}
%   \begin{syntax}
%     choice-store \key\ = \meta{cs}\{\val, \ldots\} \prefixes2223
%   \end{syntax}
%   This is defines a special type of |choice| key that'll store the given
%   choice inside the macro \meta{cs} (so \meta{cs} should be a single control
%   sequence name such as |\foo|). Since storing inside a macro can't be done
%   expandably every choice-code is |\protected|, you might define the
%   |choice-store| key itself as |protected| as well if you want. Since the
%   definition of each choice is predefined with this key type the choice list
%   should just be a comma separated list of valid choices.
% \end{function}
% This means that the following |choice| and |choice-store| keys are equivalent
% at use time:
% \begin{lstlisting}
% \newcommand*\mya{}
% \ekvdefinekeys{example}
%   {
%      choice key1 = {a=\def\mya{a}, b=\def\mya{b}, c=\def\mya{c}}
%     ,choice-store key2 = \mya{a,b,c}
%   }
% \end{lstlisting}
%
% \begin{function}{choice-enum}
%   \begin{syntax}
%     choice-enum \key\ = \meta{cs}\{\val, \ldots\} \prefixes2223
%   \end{syntax}
%   This is similar to |choice-store|, the differences are: \meta{cs} should be
%   a count-register or is initialised as such if the \meta{cs} is undefined
%   (via |\newcount|); instead of the value the position of the value in the
%   given list is stored in this register (zero-based).
% \end{function}
% This means that the following |choice| and |choice-enum| keys are equivalent
% at use time:
% \begin{lstlisting}
% \newcount\myb
% \ekvdefinekeys{example}
%   {
%      choice key1 = {a={\myb=0 }, b={\myb=1 }, c={\myb=2 }}
%     ,choice-enum key2 = \myb{a,b,c}
%   }
% \end{lstlisting}
%
% \begin{function}{unknown-choice}
%   \begin{syntax}
%     unknown-choice \key\ = \marg{definition} \prefixes2323
%   \end{syntax}
%   By default an unknown \val\ passed to a |choice| or |bool| key will throw an
%   error.  However, with this prefix you can define an alternative action which
%   should be executed if \key\ received an unknown choice. In \meta{definition}
%   you can refer to the choice which was passed in with |#1|.
% \end{function}
%
% \begin{function}{unknown code}
%   \begin{syntax}
%     unknown code = \marg{definition} \prefixes2322
%   \end{syntax}
%   By default \expkv\ throws errors when it encounters unknown keys in a set.
%   With the |unknown| prefix you can define handlers that deal with undefined
%   keys, instead of a \key\ name you have to specify a subtype for this prefix,
%   here the subtype is |code|.\par
%   With |unknown code| the \meta{definition} is used for unknown keys which
%   were provided a value (so corresponds to |\ekvdefunknown|), you can access
%   the key name with |#1| and the value with |#2|.\footnotemark
% \end{function}
% \footnotetext
%   {^^A
%     There is some trickery involved to get this more intuitive argument order
%     without any performance hit if you compare this to
%     \cs[no-index]{ekvdefunknown} directly.^^A
%   }
%
% \begin{function}{unknown noval}
%   \begin{syntax}
%     unknown noval = \marg{definition} \prefixes2323
%   \end{syntax}
%   This is like |unknown code| but uses \meta{definition} for unknown keys to
%   which no value was passed (so corresponds to |\ekvdefunknownNoVal|). You can
%   access the key name with |#1|.
% \end{function}
%
% \begin{function}{unknown redirect-code}
%   \begin{syntax}
%     unknown redirect-code = \marg{set-list} \prefixes2331
%   \end{syntax}
%   This uses a predefined action for |unknown code|. Instead of throwing an
%   error, it is tried to find the \key\ in each \set\ in the comma separated
%   \meta{set-list}. The first found match will be used and the remaining
%   options from the list discarded. If the \key\ isn't found in any \set\ an
%   expandable error will be thrown eventually. Internally \expkv's
%   |\ekvredirectunknown| will be used.
% \end{function}
%
% \begin{function}{unknown redirect-noval}
%   \begin{syntax}
%     unknown redirect-noval = \marg{set-list} \prefixes2333
%   \end{syntax}
%   This behaves just like |unknown redirect-code| but will set up means to
%   forward keys for |unknown noval|. Internally \expkv's
%   |\ekvredirectunknownNoVal| will be used.
% \end{function}
%
% \begin{function}{unknown redirect}
%   \begin{syntax}
%     unknown redirect = \marg{set-list} \prefixes2333
%   \end{syntax}
%   This is a short cut to apply both, |unknown redirect-code| and
%   |unknown redirect-noval|, as a result you might get doubled error messages,
%   one from each.
% \end{function}
%
% \subsection{Bugs}
% I don't think there are any (but every developer says that), if you find some
% please let me know, either via the email address on the first page or on
% GitHub:
% \url{https://github.com/Skillmon/tex_expkv-def}
%
% \subsection{Example}
% The following is an example code defining each base key-type once. Please
% admire the very creative key-name examples.
%
% \begin{lstlisting}
% \ekvdefinekeys{example}
%   {
%      long code keyA = #1
%     ,noval     keyA = NoVal given
%     ,bool      keyB = \keyB
%     ,boolTF    keyC = \keyC
%     ,store     keyD = \keyD
%     ,data      keyE = \keyE
%     ,dataT     keyF = \keyF
%     ,int       keyG = \keyG
%     ,dimen     keyH = \keyH
%     ,skip      keyI = \keyI
%     ,toks      keyJ = \keyJ
%     ,default   keyJ = \empty test
%     ,new box   keyK = \keyK
%     ,qdefault  keyK = K
%     ,choice    keyL =
%       {
%          protected 1 = \texttt{a}
%         ,2 = b
%         ,3 = c
%         ,4 = d
%         ,5 = e
%       }
%     ,edefault  keyL = 2
%     ,meta      keyM = {keyA={#1},keyB=false}
%     ,invbool   keyN = \keyN
%     ,boolpair  keyO = \keyOa\keyOb
%   }
% \end{lstlisting}
%
% \bigskip
% Since the |data| type might be a bit strange, here is another usage example
% for it.
% \begin{lstlisting}
% \ekvdefinekeys{ex}
%   {
%      data  name  = \Pname
%     ,data  age   = \Page
%     ,dataT hobby = \Phobby
%   }
% \newcommand\Person[1]
%   {%
%     \begingroup
%     \ekvset{ex}{#1}%
%     \begin{description}
%       \item[\Pname{}{\errmessage{A person requires a name}}]
%       \item[Age] \Page{\textit}{\errmessage{A person requires an age}}
%       \Phobby{\item[Hobbies]}
%     \end{description}
%     \endgroup
%   }
% \Person{name=Jonathan P. Spratte, age=young, hobby=\TeX\ coding}
% \Person{name=Some User, age=unknown, hobby=Reading Documentation}
% \Person{name=Anybody, age=any}
% \end{lstlisting}
% In this example a person should have a name and an age, but doesn't have to
% have hobbies. The name will be displayed as the description item and the age
% in Italics. If a person has no hobbies the description item will be silently
% left out.  The result of the above code looks like this:
%
% \begin{exresult}{\linewidth}
% \ekvdefinekeys{ex}
%   {
%      data  name  = \Pname
%     ,data  age   = \Page
%     ,dataT hobby = \Phobby
%   }
% \newcommand\Person[1]
%   {%
%     \begingroup
%     \ekvset{ex}{#1}%
%     \begin{description}
%       \item[\Pname{}{\errmessage{A person requires a name}}]
%       \item[Age] \Page{\textit}{\errmessage{A person requires an age}}
%       \Phobby{\item[Hobbies]}
%     \end{description}
%     \endgroup
%   }
% \Person{name=Jonathan P. Spratte, age=young, hobby=\TeX\ coding}
% \Person{name=Some User, age=unknown, hobby=Reading Documentation}
% \Person{name=Anybody, age=any}
% \end{exresult}
%
% \subsection{License}
%
% Copyright \textcopyright\ 2020\unless\ifnum\year=2020--\the\year\fi\
% Jonathan P. Spratte
%
% \medskip\noindent
% This work may be distributed and/or modified under the conditions of the
% \LaTeX\ Project Public License (LPPL), either version 1.3c of this license or
% (at your option) any later version. The latest version of this license is in
% the file:
%
% \url{http://www.latex-project.org/lppl.txt}
%
% \noindent
% This work is ``maintained'' (as per LPPL maintenance status) by
%
% \mbox{Jonathan P. Spratte}.
%
% \end{documentation}^^A=<<
%
% \begin{implementation}^^A>>=
%
% \clearpage
%
% \section{Implementation}^^A>>=
%^^A the LaTeX package >>=
% \subsection{The \LaTeX\ Package}
% Just like for \expkv\ we provide a small \LaTeX\ package that sets up things
% such that we behave nicely on \LaTeX\ packages and files system. It'll
% |\input| the generic code which implements the functionality.
% \gobbledocstriptag
%<*pkg>
%    \begin{macrocode}
\RequirePackage{expkv}
\def\ekvd@tmp
  {%
    \ProvidesFile{expkv-def.tex}%
      [\ekvdDate\space v\ekvdVersion\space a key-defining frontend for expkv]%
  }
% \iffalse meta-comment
%
% File: expkv-def.dtx Copyright (C) 2020-2021 Jonathan P. Spratte
%
% This work  may be  distributed and/or  modified under  the conditions  of the
% LaTeX Project Public License (LPPL),  either version 1.3c  of this license or
% (at your option) any later version.  The latest version of this license is in
% the file:
%
%   http://www.latex-project.org/lppl.txt
%
% ------------------------------------------------------------------------------
%
%<*driver>^^A>>=
\def\nameofplainTeX{plain}
\ifx\fmtname\nameofplainTeX\else
  \expandafter\begingroup
\fi
\input l3docstrip.tex
\askforoverwritefalse
\preamble

--------------------------------------------------------------
expkv-def -- a key-defining frontend for expkv
E-mail: jspratte@yahoo.de
Released under the LaTeX Project Public License v1.3c or later
See http://www.latex-project.org/lppl.txt
--------------------------------------------------------------

Copyright (C) 2020-2021 Jonathan P. Spratte

This  work may be  distributed and/or  modified under  the conditions  of the
LaTeX Project Public License (LPPL),  either version 1.3c  of this license or
(at your option) any later version.  The latest version of this license is in
the file:

  http://www.latex-project.org/lppl.txt

This work is "maintained" (as per LPPL maintenance status) by
  Jonathan P. Spratte.

This work consists of the file  expkv-def.dtx
and the derived files           expkv-def.pdf
                                expkv-def.sty
                                expkv-def.tex
                                t-expkv-def.tex

\endpreamble
% stop docstrip adding \endinput
\postamble
\endpostamble
\generate{\file{expkv-def.sty}{\from{expkv-def.dtx}{pkg}}}
\generate{\file{expkv-def.tex}{\from{expkv-def.dtx}{tex}}}
\generate{\file{t-expkv-def.tex}{\from{expkv-def.dtx}{ctx}}}
\ifx\fmtname\nameofplainTeX
  \expandafter\endbatchfile
\else
  \expandafter\endgroup
\fi
%
\IfFileExists{expkv-def.tex}{\input{expkv-def.tex}}{}
\ProvidesFile{expkv-def.dtx}
  [\csname ekvdDate\endcsname\ a key-defining frontend for expkv]
\PassOptionsToPackage{full}{textcomp}
\documentclass{l3doc}
\RequirePackage[oldstylenums,nott]{kpfonts}
\input{glyphtounicode}
\pdfgentounicode=1
\RequirePackage{tcolorbox}
\newtcolorbox{exresult}[2][]
  {colback=expkvgrey!10!white,colframe=expkvgrey,fontupper=\small,width={#2},#1}
\newtcbox\exres[1][]
  {
     colback=expkvgrey!10!white
    ,colframe=expkvgrey
    ,size=small
    ,nobeforeafter
    ,tcbox raise base
    ,fontupper=\small
    ,#1
  }
\RequirePackage{listings}
\RequirePackage{booktabs}
\RequirePackage{array}
\RequirePackage{collcell}
\RequirePackage{siunitx}
\RequirePackage{xcolor}
\RequirePackage{caption}
\RequirePackage{microtype}
\RequirePackage{accsupp}
\lstdefinelanguage{expkvd}[primitive]{TeX}
  {
    ,moretexcs=[4]^^A e-TeX
      {
        expanded,
        numexpr,
        protected,
      }
    ,moretexcs=[5]^^A plain/LaTeX
      {
        approx,
        begin,
        empty,
        item,
        LaTeX,
        makeatletter,makeatother,
        newcommand,newdimen,
        RequirePackage,
        rule,
        TeX,
        textit,texttt,
        usepackage,
      }
    ,moretexcs=[6]^^A used packages
      {
        ^^A xfp
        fpeval,
        ^^A keyval
        setkeys,
        ^^A yax
        defactiveparameter,storevalue,setparameterlist
      }
    ,moretexcs=[2]^^A expkv-def macros
      {
        ekvdefinekeys
      }
    ,moretexcs=[3]^^A expkv-pkg macros
      {
        ^^A expkv
        ekvdef,ekvdefNoVal,^^A
        ekvlet,ekvletNoVal,ekvletkv,ekvletkvNoVal,^^A
        ekvdefunknown,ekvdefunknownNoVal,^^A
        ekvredirectunknown,ekvredirectunknownNoVal,^^A
        ekvset,ekvsetSneaked,^^A
        ekvsetdef,ekvsetSneakeddef,ekvsetdefSneaked,^^A
        ekvparse,^^A
        ekvVersion,ekvDate,^^A
        ekvifdefined,ekvifdefinedNoVal,ekvifdefinedset,^^A
        ekvbreak,ekvbreakPreSneak,ekvbreakPostSneak,^^A
        ekvsneak,ekvsneakPre,^^A
        ekvchangeset,^^A
        ekvoptarg,ekvoptargTF,^^A
        ekverr,
        ^^A expkv-cs
        ekvcSecondaryKeys,ekvcSplitAndForward,
      }
  }
\colorlet{codeparam}{cyan!65!black}
\def\mylstwd{.55em}
\lstset
  {
    ,language=expkvd
    ,flexiblecolumns=false
    ,basewidth=\mylstwd
    ,gobble=2
    ,basicstyle=\fontfamily{jkp}\itshape
    ,texcsstyle=*[1]{\bfseries\color{expkvgrey}} ^^A primitives
    ,texcsstyle=*[2]{\bfseries\color{expkvred}}  ^^A expkv
    ,texcsstyle=*[3]{\color{expkvred}}           ^^A expkv-pkg
    ,texcsstyle=*[4]{\bfseries\color{expkvgrey}} ^^A e-TeX
    ,texcsstyle=*[5]{\bfseries\color{expkvgrey}} ^^A plain/LaTeX macros
    ,texcsstyle=*[6]{}                           ^^A macros of other packages
    ,commentstyle=\color[gray]{0.4}
    ,literate=
              {\{} {{\CodeUpBf\{}}{1}
              {\}} {{\CodeUpBf\}}}{1}
              {$}  {{\CodeUpBf\$}}{1}
              {[}  {{\CodeUp[}}{1}
              {]}  {{\CodeUp]}}{1}
              {(}  {{\CodeUp(}}{1}
              {)}  {{\CodeUp)}}{1}
              {*}  {{$*$}}{1}
              {1}  {{\CodeUp{1}}}{1}
              {2}  {{\CodeUp{2}}}{1}
              {3}  {{\CodeUp{3}}}{1}
              {4}  {{\CodeUp{4}}}{1}
              {5}  {{\CodeUp{5}}}{1}
              {6}  {{\CodeUp{6}}}{1}
              {7}  {{\CodeUp{7}}}{1}
              {8}  {{\CodeUp{8}}}{1}
              {9}  {{\CodeUp{9}}}{1}
              {0}  {{\CodeUp{0}}}{1}
              {##} {{\CodeColored{codeparam}{1}{\#}}}{1}
              {##1}{{\CodeColored{codeparam}{2}{\#1}}}{2}
              {##2}{{\CodeColored{codeparam}{2}{\#2}}}{2}
              {##3}{{\CodeColored{codeparam}{2}{\#3}}}{2}
              {##4}{{\CodeColored{codeparam}{2}{\#4}}}{2}
              {##5}{{\CodeColored{codeparam}{2}{\#5}}}{2}
              {##6}{{\CodeColored{codeparam}{2}{\#6}}}{2}
              {##7}{{\CodeColored{codeparam}{2}{\#7}}}{2}
              {##8}{{\CodeColored{codeparam}{2}{\#8}}}{2}
              {##9}{{\CodeColored{codeparam}{2}{\#9}}}{2}
              {<key>}{{$\langle$}key{$\rangle$}}{5}
              {<set>}{{$\langle$}set{$\rangle$}}{5}
              {<set1>}{{$\langle$}set1{$\rangle$}}{6}
              {<set2>}{{$\langle$}set2{$\rangle$}}{6}
    ^^A,literate=*{<key>}{\key}{4}{<set>}{\set}{4}
  }
\newcommand*\CodeColored[3]{\textcolor{#1}{\makebox[\dimexpr\mylstwd*#2]{#3}}}
\newcommand*\CodeUpBf[1]{\makebox[\mylstwd]{\textup{\textbf{#1}}}}
\newcommand*\CodeUp[1]{\makebox[\mylstwd]{\textup{#1}}}
\RequirePackage{randtext}
\let\metaORIG\meta
\protected\def\meta #1{\texttt{\metaORIG{#1}}}
\renewcommand*\thefootnote{\fnsymbol{footnote}}
\definecolor{expkvred}{HTML}{9F393D}
\colorlet{expkvgrey}{black!75}
\makeatletter
\newcommand*\expkv
  {^^A
    \texorpdfstring
      {^^A
        \mbox
          {^^A
            \BeginAccSupp{ActualText=expkv}^^A
            \href{https://github.com/Skillmon/tex_expkv}
              {^^A
                \rmfamily
                \bfseries
                {\color{expkvgrey}e\kern-.05em x\kern-.05em}^^A
                \lower.493ex
                  \hbox{{\color{expkvgrey}P}\kern-.1em{\color{expkvred}k}}^^A
                \kern-.18em{\color{expkvred}v}^^A
              }^^A
            \EndAccSupp{}^^A
          }^^A
      }
      {expkv}^^A
  }
\newcommand*\expkvd
  {^^A
    \texorpdfstring
      {^^A
        \mbox
          {^^A
            \BeginAccSupp{ActualText=expkv-def}^^A
            \href{https://github.com/Skillmon/tex_expkv-def}
              {^^A
                \rmfamily
                \bfseries
                {\color{expkvgrey}e\kern-.05em x\kern-.05em}^^A
                \lower.493ex
                  \hbox{{\color{expkvgrey}P}\kern-.1em{\color{expkvred}k}}^^A
                \kern-.18em{\color{expkvred}v}^^A
                {\color{expkvgrey}^^A
                  \kern.05em\rule[-.1ex]{.08em}{1.2ex}\kern.05em\textsc{def}^^A
                }^^A
              }^^A
            \EndAccSupp{}^^A
          }^^A
      }
      {expkv-def}^^A
  }
\newcommand\kv{\meta{key}=\meta{value}}
\newcommand\key{\meta{key}}
\newcommand\val{\meta{value}}
\newcommand\set{\meta{set}}
\newcommand\enfprefix{\textcolor{black}}
\newcommand\allprefix{\textcolor{gray}}
\newcommand\notprefix{\textcolor{red!80!black}}
\newcommand\printprefix[2]
  {^^A
    \ifcase\numexpr#1\relax\or
      \enfprefix{#2}\or
      \allprefix{#2}\or
      \notprefix{#2}\fi
  }
\newcommand\prefixes[4]
  {^^A
    \hfill
    \printprefix{#1}{new}
    \printprefix{#2}{also}
    \printprefix{#3}{protected}
    \printprefix{#4}{long}^^A
  }
\hypersetup{linkcolor=red!80!black,urlcolor=purple!80!black}
\DoNotIndex{\def,\edef,\,,\=,\begingroup,\catcode,\chardef,\csname,\endcsname}
\DoNotIndex{\endgroup,\endinput,\errmessage,\expandafter,\input,\let,\long}
\DoNotIndex{\protected,\ProvidesFile,\ProvidesPackage,\relax,\space}
\DoNotIndex{\@,\unexpanded,\string,\expanded,\dimexpr,\global,\glueexpr,\hbox}
\DoNotIndex{\numexpr,\RequirePackage,\setbox,\the,\unless,\xdef,\@firstofone}
\DoNotIndex{\@firstoftwo,\@gobble,\@secondoftwo,\meaning,\detokenize}
\DoNotIndex{\romannumeral,\^}
\DoNotIndex{\ifcsname}
\DoNotIndex{\ifx}
\DoNotIndex{\ifdefined}
\DoNotIndex{\iffalse}
\DoNotIndex{\iftrue}
\DoNotIndex{\else}
\DoNotIndex{\fi}
\@gobble\fi ^^A ignoring \ifx, \ifcsname, etc., but only one \fi
\@gobble\fi ^^A ignoring \ifx, \ifcsname, etc., but only one \fi
\@gobble\fi ^^A ignoring \ifx, \ifcsname, etc., but only one \fi
\@gobble\fi ^^A ignoring \ifx, \ifcsname, etc., but only one \fi
\@ifdefinable\gobbledocstriptag{\def\gobbledocstriptag#1>{}}
\makeatother
\begin{document}
  \title
    {^^A
      \texorpdfstring
        {^^A
          \huge\expkvd
          \\[\medskipamount]
          \Large a key-defining frontend for \expkv
        }{expkv-def - a key-defining frontend for expkv}^^A
    }
  \date{\ekvdDate\space v\ekvdVersion}
  \author{Jonathan P. Spratte\thanks{\protect\randomize{jspratte@yahoo.de}}}
  \DocInput{expkv-def.dtx}
\end{document}
%</driver>^^A=<<
% \fi
%
% \maketitle
% \renewcommand*\thefootnote{\arabic{footnote}}
%
% \begin{abstract}
% \noindent\parfillskip=0pt
% \expkvd\ provides a small \kv\ interface to define keys for \expkv. Key-types
% are declared using prefixes, similar to static typed languages. The stylised
% name is \expkvd\ but the files use \file{expkv-def}, this is due to CTAN-rules
% which don't allow \string| in package names since that is the pipe symbol in
% *nix shells.
% \end{abstract}
%
% \tableofcontents
%
% \begin{documentation}^^A>>=
%
% \section{Documentation}
%
% Since the trend for the last couple of years goes to defining keys for a \kv\
% interface using a \kv\ interface, I thought that maybe providing such an
% interface for \expkv\ will make it more attractive for actual use, besides its
% unique selling points of being fully expandable, and fast and reliable. But at
% the same time I don't want to widen \expkv's initial scope. So here it is
% \expkvd, go define \kv\ interfaces with \kv\ interfaces.
%
% Unlike many of the other established \kv\ interfaces to define keys, \expkvd\
% works using prefixes instead of suffixes (\emph{e.g.}, |.tl_set:N| of
% \pkg{l3keys}) or directory like handlers (\emph{e.g.}, |/.store in| of
% \pkg{pgfkeys}). This was decided as a personal preference, more over in \TeX\
% parsing for the first space is way easier than parsing for the last one.
% \expkvd's prefixes are sorted into two categories: |p|-type, which are
% equivalent to \TeX's prefixes like |\long|, and |t|-type defining the type of
% the key. For a description of the available |p|-prefixes take a look at
% \autoref{sec:prefix:p}, the |t|-prefixes are described in
% \autoref{sec:prefix:t}.
%
% \expkvd\ is usable as generic code, as a \LaTeX\ package, and as a \ConTeXt\
% module. It'll automatically load \expkv\ in the same mode as well. To use it,
% just use one of
% \begin{lstlisting}
% \input expkv-def       % plainTeX
% \usepackage{expkv-def} % LaTeX
% \usemodule[expkv-def]  % ConTeXt
% \end{lstlisting}
%
% \subsection{Macros}
% Apart from version and date containers there is only a single user-facing
% macro, and that should be used to define keys.
%
% \begin{function}{\ekvdefinekeys}
%   \begin{syntax}
%     \cs{ekvdefinekeys}\marg{set}\{\kv, \ldots\}
%   \end{syntax}
%   In \set, define \key\ to have definition \val. The general syntax for \key\
%   should be
%   \begin{quote}\ttfamily\small
%     \meta{prefix} \meta{name}
%   \end{quote}
%   Where \meta{prefix} is a space separated list of optional |p|-type prefixes
%   followed by one |t|-type prefix. The syntax of \val\ is dependent on the
%   used |t|-prefix.
% \end{function}
%
% \begin{function}{\ekvdDate,\ekvdVersion}
%   These two macros store the version and date of the package.
% \end{function}
%
% \subsection{Prefixes}
% As already said there are |p|-prefixes and |t|-prefixes. Not every |p|-prefix
% is allowed for all |t|-prefixes.
%
% \subsubsection{\texttt{p}-Prefixes}\label{sec:prefix:p}
% The two |p|-type prefixes |long| and |protected| are pretty simple by nature,
% so their description is pretty simple. They affect the \key\ at use-time, so
% omitting |long| doesn't mean that a \meta{definition} can't contain a |\par|
% token, only that the \key\ will not accept a |\par| in \val.  On the other
% hand |new| and |also| might be simple on first sight as well, but their rules
% are a bit more complicated.
%
% \begin{function}{also}
%   The following key type will be \emph{added} to an existing \key's
%   definition. You can't add a type taking an argument at use time to an
%   existing key which doesn't take an argument and vice versa. Also you'll get
%   an error if you try to add an action which isn't allowed to be either |long|
%   or |protected| to a key which already is |long| or |protected| (the opposite
%   order would be suboptimal as well, but can't be really captured with the
%   current code).
%
%   A key already defined as |long| or |protected| will stay |long| or
%   |protected|, but you can as well add |long| or |protected| with the |also|
%   definition.
%
%   As a small example, suppose you want to create a boolean key, but
%   additionally to setting a boolean value you want to execute some more code
%   as well, you can use the following
% \begin{lstlisting}
% \ekvdefinekeys{also-example}
%   {
%      bool key      = \ifmybool
%     ,also code key = \domystuff{#1}
%   }
% \end{lstlisting}
%
%   If you use |also| on a |choice|, |bool|, |invbool|, or |boolpair| key it is
%   tried to determine if the key already is of one of those types. If this test
%   is true the declared choices will be added to the possible choices but the
%   key's definition will not be changed other than that. If that wouldn't have
%   been done, the callbacks of the different choices could get called multiple
%   times.
% \end{function}
%
% \begin{function}{protected,protect}
%   The following key will be defined |\protected|. Note that key-types which
%   can't be defined expandable will always use |\protected|.
% \end{function}
%
% \begin{function}{long}
%   The following key will be defined |\long|.
% \end{function}
%
% \begin{function}{new}
%   The following key must be new (so previously undefined). An error is thrown
%   if it is already defined and the new definition is ignored. |new| only
%   asserts that there are no conflicts between |NoVal| keys and other |NoVal|
%   keys or value taking keys and other value taking keys. For example you can
%   use the following without an error:
% \begin{lstlisting}
% \ekvdefinekeys{new-example}
%   {
%      code key       = \domystuffwitharg{#1}
%     ,new  noval key = \domystuffwithoutarg
%   }
% \end{lstlisting}
% \end{function}
%
% \subsubsection{\texttt{t}-Prefixes}\label{sec:prefix:t}
% Since the |p|-type prefixes apply to some of the |t|-prefixes automatically
% but sometimes one might be disallowed we need some way to highlight this
% behaviour. In the following an enforced prefix will be printed black
% (\texttt{\enfprefix{protected}}), allowed prefixes will be grey
% (\texttt{\allprefix{protected}}), and disallowed prefixes will be red
% (\texttt{\notprefix{protected}}). This will be put flush-right in the syntax
% showing line.
%
% \begin{function}{code,ecode}
%   \begin{syntax}
%     code \key\ = \marg{definition} \prefixes2222
%   \end{syntax}
%   Define \key\ to expand to \meta{definition}. The \key\ will require a \val\
%   for which you can use |#1| inside \meta{definition}. The |ecode| variant
%   will fully expand \meta{definition} inside an |\edef|.
% \end{function}
%
% \begin{function}{noval,enoval}
%   \begin{syntax}
%     noval \key\ = \marg{definition} \prefixes2223
%   \end{syntax}
%   The |noval| type defines \key\ to expand to \meta{definition}. The \key\
%   will not take a \val. |enoval| fully expands \meta{definition} inside an
%   |\edef|.
% \end{function}
%
% \begin{function}{default,qdefault,odefault,fdefault,edefault}
%   \begin{syntax}
%     default \key\ = \marg{definition} \prefixes3223
%   \end{syntax}
%   This serves to place a default \val\ for a \key\ that takes an argument, the
%   \key\ can be of any argument-grabbing kind, and when used without a \val\
%   it will be passed \meta{definition} instead. The |qdefault| variant will
%   expand the \key's code once, so will be slightly quicker, but not change if
%   you redefine \key. |odefault| is just another name for |qdefault|. The
%   |fdefault| version will expand the key code until a non-expandable token or
%   a space is found, a space would be gobbled.\footnotemark{} The |edefault| on
%   the other hand fully expands the \key-code with \meta{definition} as its
%   argument inside of an |\edef|.
% \end{function}
% \footnotetext
%   {^^A
%     For those familiar with \TeX-coding: This uses a
%     \cs[no-index]{romannumeral}-expansion.
%   }
%
% \begin{function}{initial,oinitial,finitial,einitial}
%   \begin{syntax}
%     initial \key\ = \{\val\} \prefixes3333
%     initial \key
%   \end{syntax}
%   With |initial| you can set an initial \val\ for an already defined argument
%   taking \key. It'll just call the key-macro of \key\ and pass it \val. The
%   |einitial| variant will expand \val\ using an |\edef| expansion prior to
%   passing it to the key-macro and the |oinitial| variant will expand the first
%   token in \val\ once. |finitial| will expand \val\ until a non-expandable
%   token or a space is found, a space would be
%   gobbled.\footnotemark
%
%   If you don't provide a value (and no equals sign) a |noval| \key\ of the
%   same name is called once (or, if you specified a |default| for a value
%   taking key that would be used).
% \end{function}
% \footnotetext{Again using \cs[no-index]{romannumeral}.}
%
% \begin{function}{bool,gbool,boolTF,gboolTF}
%   \begin{syntax}
%     bool \key\ = \meta{cs} \prefixes2223
%   \end{syntax}
%   The \meta{cs} should be a single control sequence, such as |\iffoo|.
%   This will define \key\ to be a boolean key, which only takes the values
%   |true| or |false| and will throw an error for other values. If the key is
%   used without a \val\ it'll have the same effect as if you use
%   \texttt{\key=true}. |bool| and |gbool| will behave like \TeX-ifs so either
%   be |\iftrue| or |\iffalse|. The |boolTF| and |gboolTF| variants will both
%   take two arguments and if true the first will be used else the second, so
%   they are always either |\@firstoftwo| or |\@secondoftwo|. The variants with
%   a leading |g| will set the control sequence globally, the others locally. If
%   \meta{cs} is not yet defined it'll be initialised as the |false| version.
%   Note that the initialisation is \emph{not} done with |\newif|, so you will
%   not be able to do |\footrue| outside of the \kv\ interface, but you could
%   use |\newif| yourself. Even if the \key\ will not be |\protected| the
%   commands which execute the |true| or |false| choice will be, so the usage
%   should be safe in an expansion context (\emph{e.g.}, you can use
%   \texttt{edefault \key\ = false} without an issue to change the default
%   behaviour to execute the |false| choice). Internally a |bool| \key\ is the
%   same as a |choice| key which is set up to handle |true| and |false| as
%   choices.
% \end{function}
%
% \begin{function}{invbool,ginvbool,invboolTF,ginvboolTF}
%   \begin{syntax}
%     bool \key\ = \meta{cs} \prefixes2223
%   \end{syntax}
%   These are inverse boolean keys, they behave like |bool| and friends but set
%   the opposite meaning to the macro \meta{cs} in each case. So if
%   \texttt{key=true} is used |invbool| will set \meta{cs} to |\iffalse| and
%   vice versa.
% \end{function}
%
% \begin{function}{boolpair,gboolpair,boolpairTF,gboolpairTF}
%   \begin{syntax}
%     boolpair \key\ = \meta{cs_1}\meta{cs_2} \prefixes2223
%   \end{syntax}
%   The |boolpair| key type behaves like both |bool| and |invbool|, the
%   \meta{cs_1} will be set to the meaning according to the rules of |bool|, and
%   \meta{cs_2} will be set to the opposite.
% \end{function}
%
% \begin{function}{store,estore,gstore,xstore}
%   \begin{syntax}
%     store \key\ = \meta{cs} \prefixes2212
%   \end{syntax}
%   The \meta{cs} should be a single control sequence, such as |\foo|.
%   This will define \key\ to store \val\ inside of the control sequence. If
%   \meta{cs} isn't yet defined it will be initialised as empty. The variants
%   behave similarly to their |\def|, |\edef|, |\gdef|, and |\xdef|
%   counterparts, but |store| and |gstore| will allow you to store macro
%   parameters inside of them by using |\unexpanded|.
% \end{function}
%
% \begin{function}{data,edata,gdata,xdata}
%   \begin{syntax}
%     data \key\ = \meta{cs} \prefixes2212
%   \end{syntax}
%   The \meta{cs} should be a single control sequence, such as |\foo|. This will
%   define \key\ to store \val\ inside of the control sequence. But unlike the
%   |store| type, the macro \meta{cs} will be a switch at the same time, it'll
%   take two arguments and if \key\ was used expands to the first argument
%   followed by \val\ in braces, if \key\ was not used \meta{cs} will expand to
%   the second argument (so behave like |\@secondoftwo|). The idea is that with
%   this type you can define a key which should be typeset formatted. The
%   |edata| and |xdata| variants will fully expand \val, the |gdata| and |xdata|
%   variants will store \val\ inside \meta{cs} globally. The |p|-prefixes will
%   only affect the key-macro, \meta{cs} will always be expandable and |\long|.
% \end{function}
%
% \begin{function}{dataT,edataT,gdataT,xdataT}
%   \begin{syntax}
%     dataT \key\ = \meta{cs} \prefixes2212
%   \end{syntax}
%   Just like |data|, but instead of \meta{cs} grabbing two arguments it'll only
%   grab one, so by default it'll behave like |\@gobble|, and if a \val\ was
%   given to \key\ the \meta{cs} will behave like |\@firstofone| appended by
%   \marg{value}.
% \end{function}
%
% \begin{function}{int,eint,gint,xint}
%   \begin{syntax}
%     int \key\ = \meta{cs} \prefixes2212
%   \end{syntax}
%   The \meta{cs} should be a single control sequence, such as |\foo|.
%   An |int| key will be a \TeX-count register. If \meta{cs} isn't defined yet,
%   |\newcount| will be used to initialise it. The |eint| and |xint| versions
%   will use |\numexpr| to allow basic computations in their \val. The |gint|
%   and |xint| variants set the register globally.
% \end{function}
%
% \begin{function}{dimen,edimen,gdimen,xdimen}
%   \begin{syntax}
%     dimen \key\ = \meta{cs} \prefixes2212
%   \end{syntax}
%   The \meta{cs} should be a single control sequence, such as |\foo|.
%   This is just like |int| but uses a dimen register, |\newdimen| and
%   |\dimexpr| instead.
% \end{function}
%
% \begin{function}{skip,eskip,gskip,xskip}
%   \begin{syntax}
%     skip \key\ = \meta{cs} \prefixes2212
%   \end{syntax}
%   The \meta{cs} should be a single control sequence, such as |\foo|.
%   This is just like |int| but uses a skip register, |\newskip| and |\glueexpr|
%   instead.
% \end{function}
%
% \begin{function}{toks,gtoks,apptoks,gapptoks,pretoks,gpretoks}
%   \begin{syntax}
%     toks \key\ = \meta{cs} \prefixes2212
%   \end{syntax}
%   The \meta{cs} should be a single control sequence, such as |\foo|.
%   Store \val\ inside of a toks-register. The |g| variants use |\global|, the
%   |app| variants append \val\ to the contents of that register, the |pre|
%   variants will prepend \val. If \meta{cs} is not yet defined it will be
%   initialised with |\newtoks|.
% \end{function}
%
% \begin{function}{box,gbox}
%   \begin{syntax}
%     box \key\ = \meta{cs} \prefixes2212
%   \end{syntax}
%   The \meta{cs} should be a single control sequence, such as |\foo|.
%   Typesets \val\ into a |\hbox| and stores the result in a box register. The
%   boxes are colour safe. \expkvd\ doesn't provide a |vbox| type.
% \end{function}
%
% \begin{function}{meta}
%   \begin{syntax}
%     meta \key\ = \{\kv, \ldots\} \prefixes2222
%   \end{syntax}
%   This key type can set other keys, you can access the \val\ which was passed
%   to \key\ inside the \kv\ list with |#1|. It works by calling a sub-|\ekvset|
%   on the \kv\ list, so a |set| key will only affect that \kv\ list and not the
%   current |\ekvset|. Since it runs in a separate |\ekvset| you can't use
%   |\ekvsneak| using keys or similar macros in the way you normally could.
% \end{function}
%
% \begin{function}{nmeta}
%   \begin{syntax}
%     nmeta \key\ = \{\kv, \ldots\} \prefixes2223
%   \end{syntax}
%   This key type can set other keys, the difference to |meta| is, that this key
%   doesn't take a value, so the \kv\ list is static.
% \end{function}
%
% \begin{function}{smeta}
%   \begin{syntax}
%     smeta \key\ = \marg{set}\{\kv, \ldots\} \prefixes2222
%   \end{syntax}
%   Yet another |meta| variant. An |smeta| key will take a \val\ which you can
%   access using |#1|, but it sets the \kv\ list inside of \set, so is equal to
%   \texttt{\cs[no-index]{ekvset}\marg{set}\{\kv, \ldots\}}.
% \end{function}
%
% \begin{function}{snmeta}
%   \begin{syntax}
%     snmeta \key\ = \marg{set}\{\kv, \ldots\} \prefixes2223
%   \end{syntax}
%   And the last |meta| variant. |snmeta| is a combination of |smeta| and
%   |nmeta|. It doesn't take an argument and sets the \kv\ list inside of \set.
% \end{function}
%
% \begin{function}{set}
%   \begin{syntax}
%     set \key\ = \marg{set} \prefixes2233
%   \end{syntax}
%   This will define \key\ to change the set of the current |\ekvset|
%   invocation to \set. You can omit \set\ (including the equals sign),
%   which is the same as using \texttt{set \key\ = \{\key\}}. The created |set|
%   key will not take a \val. Note that just like in \expkv\ it'll not be
%   checked whether \set\ is defined and you'll get a low-level \TeX\ error if
%   you use an undefined \set.
% \end{function}
%
% \begin{function}{choice}
%   \begin{syntax}
%     choice \key\ = \{\val=\meta{definition}, \ldots\} \prefixes2223
%   \end{syntax}
%   Defines \key\ to be a |choice| key, meaning it will only accept a limited
%   set of values. You should define each possible \val\ inside of the
%   \val=\meta{definition} list. If a defined \val\ is passed to \key\ the
%   \meta{definition} will be left in the input stream. You can make
%   individual values |protected| inside the \val=\meta{definition} list. By
%   default a |choice| key is expandable, an undefined \val\ will throw an error
%   in an expandable way (but see the |unknown-choice| prefix). You can add
%   additional choices after the \key\ was created by using |choice| again for
%   the same \key, redefining choices is possible the same way, but there is no
%   interface to remove certain choices.
% \end{function}
%
% \begin{function}{choice-store}
%   \begin{syntax}
%     choice-store \key\ = \meta{cs}\{\val, \ldots\} \prefixes2223
%   \end{syntax}
%   This is defines a special type of |choice| key that'll store the given
%   choice inside the macro \meta{cs} (so \meta{cs} should be a single control
%   sequence name such as |\foo|). Since storing inside a macro can't be done
%   expandably every choice-code is |\protected|, you might define the
%   |choice-store| key itself as |protected| as well if you want. Since the
%   definition of each choice is predefined with this key type the choice list
%   should just be a comma separated list of valid choices.
% \end{function}
% This means that the following |choice| and |choice-store| keys are equivalent
% at use time:
% \begin{lstlisting}
% \newcommand*\mya{}
% \ekvdefinekeys{example}
%   {
%      choice key1 = {a=\def\mya{a}, b=\def\mya{b}, c=\def\mya{c}}
%     ,choice-store key2 = \mya{a,b,c}
%   }
% \end{lstlisting}
%
% \begin{function}{choice-enum}
%   \begin{syntax}
%     choice-enum \key\ = \meta{cs}\{\val, \ldots\} \prefixes2223
%   \end{syntax}
%   This is similar to |choice-store|, the differences are: \meta{cs} should be
%   a count-register or is initialised as such if the \meta{cs} is undefined
%   (via |\newcount|); instead of the value the position of the value in the
%   given list is stored in this register (zero-based).
% \end{function}
% This means that the following |choice| and |choice-enum| keys are equivalent
% at use time:
% \begin{lstlisting}
% \newcount\myb
% \ekvdefinekeys{example}
%   {
%      choice key1 = {a={\myb=0 }, b={\myb=1 }, c={\myb=2 }}
%     ,choice-enum key2 = \myb{a,b,c}
%   }
% \end{lstlisting}
%
% \begin{function}{unknown-choice}
%   \begin{syntax}
%     unknown-choice \key\ = \marg{definition} \prefixes2323
%   \end{syntax}
%   By default an unknown \val\ passed to a |choice| or |bool| key will throw an
%   error.  However, with this prefix you can define an alternative action which
%   should be executed if \key\ received an unknown choice. In \meta{definition}
%   you can refer to the choice which was passed in with |#1|.
% \end{function}
%
% \begin{function}{unknown code}
%   \begin{syntax}
%     unknown code = \marg{definition} \prefixes2322
%   \end{syntax}
%   By default \expkv\ throws errors when it encounters unknown keys in a set.
%   With the |unknown| prefix you can define handlers that deal with undefined
%   keys, instead of a \key\ name you have to specify a subtype for this prefix,
%   here the subtype is |code|.\par
%   With |unknown code| the \meta{definition} is used for unknown keys which
%   were provided a value (so corresponds to |\ekvdefunknown|), you can access
%   the key name with |#1| and the value with |#2|.\footnotemark
% \end{function}
% \footnotetext
%   {^^A
%     There is some trickery involved to get this more intuitive argument order
%     without any performance hit if you compare this to
%     \cs[no-index]{ekvdefunknown} directly.^^A
%   }
%
% \begin{function}{unknown noval}
%   \begin{syntax}
%     unknown noval = \marg{definition} \prefixes2323
%   \end{syntax}
%   This is like |unknown code| but uses \meta{definition} for unknown keys to
%   which no value was passed (so corresponds to |\ekvdefunknownNoVal|). You can
%   access the key name with |#1|.
% \end{function}
%
% \begin{function}{unknown redirect-code}
%   \begin{syntax}
%     unknown redirect-code = \marg{set-list} \prefixes2331
%   \end{syntax}
%   This uses a predefined action for |unknown code|. Instead of throwing an
%   error, it is tried to find the \key\ in each \set\ in the comma separated
%   \meta{set-list}. The first found match will be used and the remaining
%   options from the list discarded. If the \key\ isn't found in any \set\ an
%   expandable error will be thrown eventually. Internally \expkv's
%   |\ekvredirectunknown| will be used.
% \end{function}
%
% \begin{function}{unknown redirect-noval}
%   \begin{syntax}
%     unknown redirect-noval = \marg{set-list} \prefixes2333
%   \end{syntax}
%   This behaves just like |unknown redirect-code| but will set up means to
%   forward keys for |unknown noval|. Internally \expkv's
%   |\ekvredirectunknownNoVal| will be used.
% \end{function}
%
% \begin{function}{unknown redirect}
%   \begin{syntax}
%     unknown redirect = \marg{set-list} \prefixes2333
%   \end{syntax}
%   This is a short cut to apply both, |unknown redirect-code| and
%   |unknown redirect-noval|, as a result you might get doubled error messages,
%   one from each.
% \end{function}
%
% \subsection{Bugs}
% I don't think there are any (but every developer says that), if you find some
% please let me know, either via the email address on the first page or on
% GitHub:
% \url{https://github.com/Skillmon/tex_expkv-def}
%
% \subsection{Example}
% The following is an example code defining each base key-type once. Please
% admire the very creative key-name examples.
%
% \begin{lstlisting}
% \ekvdefinekeys{example}
%   {
%      long code keyA = #1
%     ,noval     keyA = NoVal given
%     ,bool      keyB = \keyB
%     ,boolTF    keyC = \keyC
%     ,store     keyD = \keyD
%     ,data      keyE = \keyE
%     ,dataT     keyF = \keyF
%     ,int       keyG = \keyG
%     ,dimen     keyH = \keyH
%     ,skip      keyI = \keyI
%     ,toks      keyJ = \keyJ
%     ,default   keyJ = \empty test
%     ,new box   keyK = \keyK
%     ,qdefault  keyK = K
%     ,choice    keyL =
%       {
%          protected 1 = \texttt{a}
%         ,2 = b
%         ,3 = c
%         ,4 = d
%         ,5 = e
%       }
%     ,edefault  keyL = 2
%     ,meta      keyM = {keyA={#1},keyB=false}
%     ,invbool   keyN = \keyN
%     ,boolpair  keyO = \keyOa\keyOb
%   }
% \end{lstlisting}
%
% \bigskip
% Since the |data| type might be a bit strange, here is another usage example
% for it.
% \begin{lstlisting}
% \ekvdefinekeys{ex}
%   {
%      data  name  = \Pname
%     ,data  age   = \Page
%     ,dataT hobby = \Phobby
%   }
% \newcommand\Person[1]
%   {%
%     \begingroup
%     \ekvset{ex}{#1}%
%     \begin{description}
%       \item[\Pname{}{\errmessage{A person requires a name}}]
%       \item[Age] \Page{\textit}{\errmessage{A person requires an age}}
%       \Phobby{\item[Hobbies]}
%     \end{description}
%     \endgroup
%   }
% \Person{name=Jonathan P. Spratte, age=young, hobby=\TeX\ coding}
% \Person{name=Some User, age=unknown, hobby=Reading Documentation}
% \Person{name=Anybody, age=any}
% \end{lstlisting}
% In this example a person should have a name and an age, but doesn't have to
% have hobbies. The name will be displayed as the description item and the age
% in Italics. If a person has no hobbies the description item will be silently
% left out.  The result of the above code looks like this:
%
% \begin{exresult}{\linewidth}
% \ekvdefinekeys{ex}
%   {
%      data  name  = \Pname
%     ,data  age   = \Page
%     ,dataT hobby = \Phobby
%   }
% \newcommand\Person[1]
%   {%
%     \begingroup
%     \ekvset{ex}{#1}%
%     \begin{description}
%       \item[\Pname{}{\errmessage{A person requires a name}}]
%       \item[Age] \Page{\textit}{\errmessage{A person requires an age}}
%       \Phobby{\item[Hobbies]}
%     \end{description}
%     \endgroup
%   }
% \Person{name=Jonathan P. Spratte, age=young, hobby=\TeX\ coding}
% \Person{name=Some User, age=unknown, hobby=Reading Documentation}
% \Person{name=Anybody, age=any}
% \end{exresult}
%
% \subsection{License}
%
% Copyright \textcopyright\ 2020\unless\ifnum\year=2020--\the\year\fi\
% Jonathan P. Spratte
%
% \medskip\noindent
% This work may be distributed and/or modified under the conditions of the
% \LaTeX\ Project Public License (LPPL), either version 1.3c of this license or
% (at your option) any later version. The latest version of this license is in
% the file:
%
% \url{http://www.latex-project.org/lppl.txt}
%
% \noindent
% This work is ``maintained'' (as per LPPL maintenance status) by
%
% \mbox{Jonathan P. Spratte}.
%
% \end{documentation}^^A=<<
%
% \begin{implementation}^^A>>=
%
% \clearpage
%
% \section{Implementation}^^A>>=
%^^A the LaTeX package >>=
% \subsection{The \LaTeX\ Package}
% Just like for \expkv\ we provide a small \LaTeX\ package that sets up things
% such that we behave nicely on \LaTeX\ packages and files system. It'll
% |\input| the generic code which implements the functionality.
% \gobbledocstriptag
%<*pkg>
%    \begin{macrocode}
\RequirePackage{expkv}
\def\ekvd@tmp
  {%
    \ProvidesFile{expkv-def.tex}%
      [\ekvdDate\space v\ekvdVersion\space a key-defining frontend for expkv]%
  }
\input{expkv-def.tex}
\ProvidesPackage{expkv-def}%
  [\ekvdDate\space v\ekvdVersion\space a key-defining frontend for expkv]
%    \end{macrocode}
% \gobbledocstriptag
%</pkg>
%^^A=<<
%^^A the ConTeXt module >>=
% \subsection{The \ConTeXt\ module}
% \gobbledocstriptag
%<*ctx>
%    \begin{macrocode}
\writestatus{loading}{ConTeXt User Module / expkv-def}
\usemodule[expkv]
\unprotect
\input expkv-def.tex
\writestatus{loading}
  {ConTeXt User Module / expkv-def / Version \ekvdVersion\space loaded}
\protect\endinput
%    \end{macrocode}
% \gobbledocstriptag
%</ctx>
%^^A=<<
%^^A main file >>=
% \subsection{The Generic Code}
% The rest of this implementation will be the generic code.
% \gobbledocstriptag
%<*tex>
%
% Load \expkv\ if the package didn't already do so -- since \expkv\ has
% safeguards against being loaded twice this does no harm and the overhead
% isn't that big. Also we reuse some of the internals of \expkv\ to save us from
% retyping them.
%    \begin{macrocode}
\input expkv
%    \end{macrocode}
%
% We make sure that \file{expkv-def.tex} is only input once:
%    \begin{macrocode}
\expandafter\ifx\csname ekvdVersion\endcsname\relax
\else
  \expandafter\endinput
\fi
%    \end{macrocode}
%
% \begin{macro}{\ekvdVersion,\ekvdDate}
% We're on our first input, so lets store the version and date in a macro.
%    \begin{macrocode}
\def\ekvdVersion{0.8c}
\def\ekvdDate{2021-09-20}
%    \end{macrocode}
% \end{macro}
%
% If the \LaTeX\ format is loaded we want to be a good file and report back who
% we are, for this the package will have defined |\ekvd@tmp| to use
% |\ProvidesFile|, else this will expand to a |\relax| and do no harm.
%    \begin{macrocode}
\csname ekvd@tmp\endcsname
%    \end{macrocode}
%
% Store the category code of |@| to later be able to reset it and change it to
% 11 for now.
%    \begin{macrocode}
\expandafter\chardef\csname ekvd@tmp\endcsname=\catcode`\@
\catcode`\@=11
%    \end{macrocode}
% |\ekvd@tmp| will be reused later to handle expansion during the key
% defining. But we don't need it to ever store information long-term after
% \expkvd\ was initialized.
%
% \begin{macro}[internal]{\ekvd@ifprimitive}
%    \begin{macrocode}
\protected\long\def\ekvd@ifprimitive#1%
  {%
    \begingroup
      \edef\ekvd@tmpa{\string #1}%
      \edef\ekvd@tmpb{\meaning#1}%
      \expandafter
    \endgroup
    \ifx\ekvd@tmpa\ekvd@tmpb
      \ekv@fi@firstoftwo
    \fi
    \@secondoftwo
  }
%    \end{macrocode}
% \end{macro}
%
% \begin{macro}[internal]
%   {\ekvd@long,\ekvd@prot,\ekvd@clear@prefixes,\ekvd@ifalso}
% \expkvd\ will use |\ekvd@long|, |\ekvd@prot|, and |\ekvd@ifalso| to store
% whether a key should be defined as |\long| or |\protected| or adds an action
% to an existing key, and we have to clear them for every new key. By default
% |long| and |protected| will just be empty, |ifalso| will be
% |\@secondoftwo|, and |ifnew| will just use its third argument.
%    \begin{macrocode}
\protected\def\ekvd@clear@prefixes
  {%
    \let\ekvd@long\ekv@empty
    \let\ekvd@prot\ekv@empty
    \let\ekvd@ifalso\@secondoftwo
    \long\def\ekvd@ifnew##1##2##3{##3}%
  }
\ekvd@clear@prefixes
%    \end{macrocode}
% \end{macro}
%
% \begin{macro}{\ekvdefinekeys}
% This is the one front-facing macro which provides the interface to define
% keys. It's using |\ekvparse| to handle the \kv\ list, the interpretation will
% be done by |\ekvd@noarg| and |\ekvd@|. The \set\ for which the keys should be
% defined is stored in |\ekvd@set|.
%    \begin{macrocode}
\protected\def\ekvdefinekeys#1%
  {%
    \def\ekvd@set{#1}%
    \ekvparse\ekvd@noarg\ekvd@arg
  }
%    \end{macrocode}
% \end{macro}
%
% \begin{macro}[internal]{\ekvd@noarg,\ekvd@arg,\ekvd@handle}
% |\ekvd@noarg| and |\ekvd@arg| store whether there was a value in the \kv\
% pair. |\ekvd@handle| has to test whether there is a space inside the key and
% if so calls the prefix grabbing routine, else we throw an error and ignore the
% key.
%    \begin{macrocode}
\protected\long\def\ekvd@noarg#1%
  {%
    \let\ekvd@ifnoarg\@firstoftwo
    \expandafter\ekvd@handle\detokenize{#1}\ekvd@stop{}%
  }
\protected\long\def\ekvd@arg#1%
  {%
    \let\ekvd@ifnoarg\@secondoftwo
    \expandafter\ekvd@handle\detokenize{#1}\ekvd@stop
  }
\protected\long\def\ekvd@handle#1\ekvd@stop#2%
  {%
    \ekvd@clear@prefixes
    \def\ekvd@cur{#1}%
    \ekvd@ifspace{#1}%
      {\ekvd@prefix\ekv@mark#1\ekv@stop{#2}}%
      \ekvd@err@missing@type
  }
%    \end{macrocode}
% \end{macro}
%
% \begin{macro}[internal]{\ekvd@prefix,\ekvd@prefix@}
% \expkvd\ separates prefixes into two groups, the first being prefixes in the
% \TeX\ sense (|long| and |protected|) which use |@p@| in their name, the other
% being key-types (|code|, |int|, \emph{etc.}) which use |@t@| instead.
% |\ekvd@prefix| splits at the first space and checks whether its a |@p@|
% or |@t@| type prefix. If it is neither throw an error and gobble the
% definition (the value).
%    \begin{macrocode}
\protected\def\ekvd@prefix#1 {\ekv@strip{#1}\ekvd@prefix@\ekv@mark}
\protected\def\ekvd@prefix@#1#2\ekv@stop
  {%
    \ekv@ifdefined{ekvd@t@#1}%
      {\ekv@strip{#2}{\csname ekvd@t@#1\endcsname}}%
      {%
        \ekv@ifdefined{ekvd@p@#1}%
          {\csname ekvd@p@#1\endcsname\ekvd@prefix@after@p{#2}}%
          {\ekvd@err@undefined@prefix{#1}\@gobble}%
      }%
  }
%    \end{macrocode}
% \end{macro}
%
% \begin{macro}[internal]{\ekvd@prefix@after@p}
% The |@p@| type prefixes are all just modifying a following |@t@| type, so they
% will need to search for another prefix. This is true for all of them, so we
% use a macro to handle this. It'll throw an error if there is no other prefix.
%    \begin{macrocode}
\protected\def\ekvd@prefix@after@p#1%
  {%
    \ekvd@ifspace{#1}%
      {\ekvd@prefix#1\ekv@stop}%
      {\ekvd@err@missing@type\@gobble}%
  }
%    \end{macrocode}
% \end{macro}
%
% \begin{macro}[internal]
%   {\ekvd@p@long,\ekvd@p@protected,\ekvd@p@protect,\ekvd@p@also,\ekvd@p@new}
% Define the |@p@| type prefixes, they all just store some information in a
% temporary macro.
%    \begin{macrocode}
\protected\def\ekvd@p@long{\let\ekvd@long\long}
\protected\def\ekvd@p@protected{\let\ekvd@prot\protected}
\let\ekvd@p@protect\ekvd@p@protected
\protected\def\ekvd@p@also{\let\ekvd@ifalso\@firstoftwo}
\protected\def\ekvd@p@new{\let\ekvd@ifnew\ekvd@assert@new}
%    \end{macrocode}
% \end{macro}
%
% \subsubsection{Key Types}
%
% \begin{macro}[internal]{\ekvd@type@set,\ekvd@t@set}
% The |set| type is quite straight forward, just define a |NoVal| key to call
% |\ekvchangeset|.
%    \begin{macrocode}
\protected\def\ekvd@type@set#1#2%
  {%
    \ekvd@assert@not@long
    \ekvd@assert@not@protected
    \ekvd@ifnew{NoVal}{#1}%
      {%
        \ekv@ifempty{#2}%
          {\ekvd@err@missing@definition}%
          {%
            \ekvd@ifalso
              {%
                \ekv@expargtwice{\ekvd@add@noval{#1}}%
                  {\ekvchangeset{#2}}%
                  \ekvd@assert@not@protected@also
              }%
              {\ekv@expargtwice{\ekvdefNoVal\ekvd@set{#1}}{\ekvchangeset{#2}}}%
          }%
      }%
  }
\protected\def\ekvd@t@set#1#2%
  {%
    \ekvd@ifnoarg
      {\ekvd@type@set{#1}{#1}}%
      {\ekvd@type@set{#1}{#2}}%
  }
%    \end{macrocode}
% \end{macro}
%
% \begin{macro}[internal]{\ekvd@type@noval,\ekvd@t@noval,\ekvd@t@enoval}
% Another pretty simple type, |noval| just needs to assert that there is a
% definition and that |long| wasn't specified.
% There are types where the difference in the variants is so small, that we
% define a common handler for them, those common handlers are named with
% |@type@|. |noval| and |enoval| are so similar that we can use such a |@type@|
% macro, even if we could've done |noval| in a slightly faster way without it.
%    \begin{macrocode}
\protected\long\def\ekvd@type@noval#1#2#3%
  {%
    \ekvd@ifnew{NoVal}{#2}%
      {%
        \ekvd@assert@arg
          {%
            \ekvd@assert@not@long
            \ekvd@prot#1\ekvd@tmp{#3}%
            \ekvd@ifalso
              {\ekv@exparg{\ekvd@add@noval{#2}}\ekvd@tmp{}}%
              {\ekvletNoVal\ekvd@set{#2}\ekvd@tmp}%
          }%
      }%
  }
\protected\def\ekvd@t@noval{\ekvd@type@noval\def}
\protected\def\ekvd@t@enoval{\ekvd@type@noval\edef}
%    \end{macrocode}
% \end{macro}
%
% \begin{macro}[internal]{\ekvd@type@code,\ekvd@t@code,\ekvd@t@ecode}
% |code| is simple as well, |ecode| has to use |\edef| on a temporary macro,
% since \expkv\ doesn't provide an |\ekvedef|.
%    \begin{macrocode}
\protected\long\def\ekvd@type@code#1#2#3%
  {%
    \ekvd@ifnew{}{#2}%
      {%
        \ekvd@assert@arg
          {%
            \ekvd@prot\ekvd@long#1\ekvd@tmp##1{#3}%
            \ekvd@ifalso
              {\ekv@exparg{\ekvd@add@val{#2}}{\ekvd@tmp{##1}}{}}%
              {\ekvlet\ekvd@set{#2}\ekvd@tmp}%
          }%
      }%
  }
\protected\def\ekvd@t@code{\ekvd@type@code\def}
\protected\def\ekvd@t@ecode{\ekvd@type@code\edef}
%    \end{macrocode}
% \end{macro}
%
% \begin{macro}[internal]
%   {
%     \ekvd@type@default,\ekvd@t@default,\ekvd@t@qdefault,\ekvd@t@odefault,
%     \ekvd@t@fdefault
%   }
% |\ekvd@type@default| asserts there was an argument, also the key for which one
% wants to set a default has to be already defined (this is not so important for
% |default|, but |qdefault| requires is). If everything is good, |\edef| a
% temporary macro that expands |\ekvd@set| and the |\csname| for the key, and in
% the case of |qdefault| does the first expansion step of the key-macro.
%    \begin{macrocode}
\protected\long\def\ekvd@type@default#1#2#3#4%
  {%
    \ekvd@assert@arg
      {%
        \ekvifdefined\ekvd@set{#3}%
          {%
            \ekvd@assert@not@new
            \ekvd@assert@not@long
            \ekvd@prot\edef\ekvd@tmp
              {%
                \ekv@unexpanded\expandafter#1%
                  {#2\csname\ekv@name\ekvd@set{#3}\endcsname{#4}}%
              }%
            \ekvd@ifalso
              {\ekv@exparg{\ekvd@add@noval{#3}}\ekvd@tmp{}}%
              {\ekvletNoVal\ekvd@set{#3}\ekvd@tmp}%
          }%
          {\ekvd@err@undefined@key{#3}}%
      }%
  }
\protected\def\ekvd@t@default{\ekvd@type@default{}{}}
\protected\def\ekvd@t@qdefault{\ekvd@type@default{\expandafter\expandafter}{}}
\let\ekvd@t@odefault\ekvd@t@qdefault
\protected\def\ekvd@t@fdefault{\ekvd@type@default{}{\romannumeral`\^^@}}
%    \end{macrocode}
% \end{macro}
%
% \begin{macro}[internal]{\ekvd@t@edefault}
% |edefault| is too different from |default| and |qdefault| to reuse the
% |@type@| macro, as it doesn't need |\unexpanded| inside of |\edef|.
%    \begin{macrocode}
\protected\long\def\ekvd@t@edefault#1#2%
  {%
    \ekvd@assert@arg
      {%
        \ekvifdefined\ekvd@set{#1}%
          {%
            \ekvd@assert@not@new
            \ekvd@assert@not@long
            \ekvd@prot\edef\ekvd@tmp
              {\csname\ekv@name\ekvd@set{#1}\endcsname{#2}}%
            \ekvd@ifalso
              {\ekv@exparg{\ekvd@add@noval{#1}}\ekvd@tmp{}}%
              {\ekvletNoVal\ekvd@set{#1}\ekvd@tmp}%
          }%
          {\ekvd@err@undefined@key{#1}}%
      }%
  }
%    \end{macrocode}
% \end{macro}
%
% \begin{macro}[internal]
%   {\ekvd@t@initial,\ekvd@t@oinitial,\ekvd@t@finitial,\ekvd@t@einitial}
%    \begin{macrocode}
\long\def\ekvd@type@initial#1#2#3#4%
  {%
    \ekvd@assert@not@new
    \ekvd@assert@not@also
    \ekvd@assert@not@long
    \ekvd@assert@not@protected
    \ekvd@ifnoarg
      {%
        \ekvifdefinedNoVal\ekvd@set{#3}%
          {\csname\ekv@name\ekvd@set{#3}N\endcsname}%
          {\ekvd@err@undefined@noval{#3}}%
      }%
      {%
        \ekvifdefined\ekvd@set{#3}%
          {%
            #1{#2#4}%
            \csname\ekv@name\ekvd@set{#3}\expandafter\endcsname\expandafter
              {\ekvd@tmp}%
          }%
          {\ekvd@err@undefined@key{#3}}%
      }%
  }
\def\ekvd@t@initial{\ekvd@type@initial{\def\ekvd@tmp}{}}
\def\ekvd@t@oinitial{\ekvd@type@initial{\ekv@exparg{\def\ekvd@tmp}}{}}
\def\ekvd@t@einitial{\ekvd@type@initial{\edef\ekvd@tmp}{}}
\def\ekvd@t@finitial
  {\ekvd@type@initial{\ekv@exparg{\def\ekvd@tmp}}{\romannumeral`\^^@}}
%    \end{macrocode}
% \end{macro}
%
% \begin{macro}[internal]
%   {
%     \ekvd@type@bool,\ekvd@t@bool,\ekvd@t@gbool,\ekvd@t@boolTF,\ekvd@t@gboolTF,
%     \ekvd@t@invbool,\ekvd@t@ginvbool,\ekvd@t@invboolTF,\ekvd@t@ginvboolTF,
%   }
% The boolean types are a quicker version of a |choice| that accept |true| and
% |false|, and set up the |NoVal| action to be identical to \texttt{\key=true}.
% The |true| and |false| actions are always just |\let|ting the macro in |#7| to
% some other macro (\emph{e.g.}, \cs[no-index]{iftrue}).
%    \begin{macrocode}
\protected\def\ekvd@type@bool#1#2#3#4#5%
  {%
    \ekvd@ifnew{}{#4}%
      {%
        \ekvd@ifnew{NoVal}{#4}%
          {%
            \ekvd@assert@filledarg{#5}%
              {%
                \ekvd@newlet#5#3%
                \ekvd@type@choice{#4}%
                \protected\ekvdefNoVal\ekvd@set{#4}{#1\let#5#2}%
                \protected\expandafter\def
                  \csname\ekvd@choice@name\ekvd@set{#4}{true}\endcsname
                  {#1\let#5#2}%
                \protected\expandafter\def
                  \csname\ekvd@choice@name\ekvd@set{#4}{false}\endcsname
                  {#1\let#5#3}%
              }%
          }%
      }%
  }
\protected\def\ekvd@t@bool{\ekvd@type@bool{}\iftrue\iffalse}
\protected\def\ekvd@t@gbool{\ekvd@type@bool\global\iftrue\iffalse}
\protected\def\ekvd@t@boolTF{\ekvd@type@bool{}\@firstoftwo\@secondoftwo}
\protected\def\ekvd@t@gboolTF{\ekvd@type@bool\global\@firstoftwo\@secondoftwo}
\protected\def\ekvd@t@invbool{\ekvd@type@bool{}\iffalse\iftrue}
\protected\def\ekvd@t@ginvbool{\ekvd@type@bool\global\iffalse\iftrue}
\protected\def\ekvd@t@invboolTF{\ekvd@type@bool{}\@secondoftwo\@firstoftwo}
\protected\def\ekvd@t@ginvboolTF
  {\ekvd@type@bool\global\@secondoftwo\@firstoftwo}
%    \end{macrocode}
% \end{macro}
%
% \begin{macro}[internal]
%   {
%     \ekvd@type@boolpair,
%     \ekvd@t@boolpair,\ekvd@t@gboolpair,\ekvd@t@boolpairTF,\ekvd@t@gboolpairTF
%   }
% The boolean pair types are essentially the same as the boolean types, but set
% two macros instead of one.
%    \begin{macrocode}
\protected\def\ekvd@type@boolpair#1#2#3#4#5#6%
  {%
    \ekvd@ifnew{}{#4}%
      {%
        \ekvd@ifnew{NoVal}{#4}%
          {%
            \ekvd@newlet#5#3%
            \ekvd@newlet#6#2%
            \ekvd@type@choice{#4}%
            \protected\ekvdefNoVal\ekvd@set{#4}{#1\let#5#2#1\let#6#3}%
            \protected\expandafter\def
              \csname\ekvd@choice@name\ekvd@set{#4}{true}\endcsname
              {#1\let#5#2#1\let#6#3}%
            \protected\expandafter\def
              \csname\ekvd@choice@name\ekvd@set{#4}{false}\endcsname
              {#1\let#5#3#1\let#6#2}%
          }%
      }%
  }
\protected\def\ekvd@t@boolpair#1#2%
  {\ekvd@assert@twoargs{#2}{\ekvd@type@boolpair{}\iftrue\iffalse{#1}#2}}
\protected\def\ekvd@t@gboolpair#1#2%
  {\ekvd@assert@twoargs{#2}{\ekvd@type@boolpair\global\iftrue\iffalse{#1}#2}}
\protected\def\ekvd@t@boolpairTF#1#2%
  {%
    \ekvd@assert@twoargs{#2}%
      {\ekvd@type@boolpair{}\@firstoftwo\@secondoftwo{#1}#2}%
  }
\protected\def\ekvd@t@gboolpairTF#1#2%
  {%
    \ekvd@assert@twoargs{#2}%
      {\ekvd@type@boolpair\global\@firstoftwo\@secondoftwo{#1}#2}%
  }
%    \end{macrocode}
% \end{macro}
%
% \begin{macro}[internal]
%   {
%     \ekvd@type@data,\ekvd@t@data,\ekvd@t@gdata,\ekvd@t@dataT,\ekvd@t@gdataT
%   }
%    \begin{macrocode}
\protected\def\ekvd@type@data#1#2#3#4#5#6%
  {%
    \ekvd@ifnew{}{#5}%
      {%
        \ekvd@assert@filledarg{#6}%
          {%
            \ekvd@newlet#6#1%
            \ekvd@ifalso
              {%
                \let\ekvd@prot\protected
                \ekvd@add@val{#5}{\long#2#6####1#3{####1{#4}}}{}%
              }%
              {%
                \protected\ekvd@long\ekvdef\ekvd@set{#5}%
                  {\long#2#6####1#3{####1{#4}}}%
              }%
          }%
      }%
  }
\protected\def\ekvd@t@data
  {\ekvd@type@data\@secondoftwo\edef{####2}{\ekv@unexpanded{##1}}}
\protected\def\ekvd@t@edata{\ekvd@type@data\@secondoftwo\edef{####2}{##1}}
\protected\def\ekvd@t@gdata
  {\ekvd@type@data\@secondoftwo\xdef{####2}{\ekv@unexpanded{##1}}}
\protected\def\ekvd@t@xdata{\ekvd@type@data\@secondoftwo\xdef{####2}{##1}}
\protected\def\ekvd@t@dataT
  {\ekvd@type@data\@gobble\edef{}{\ekv@unexpanded{##1}}}
\protected\def\ekvd@t@edataT{\ekvd@type@data\@gobble\edef{}{##1}}
\protected\def\ekvd@t@gdataT
  {\ekvd@type@data\@gobble\xdef{}{\ekv@unexpanded{##1}}}
\protected\def\ekvd@t@xdataT{\ekvd@type@data\@gobble\xdef{}{##1}}
%    \end{macrocode}
% \end{macro}
%
% \begin{macro}[internal]{\ekvd@type@box,\ekvd@t@box,\ekvd@t@gbox}
% Set up our boxes. Though we're a generic package we want to be colour safe, so
% we put an additional grouping level inside the box contents, for the case that
% someone uses \pkg{color}. |\ekvd@newreg| is a small wrapper which tests
% whether the first argument is defined and if not does
% |\csname new#2\endcsname#1|.
%    \begin{macrocode}
\protected\def\ekvd@type@box#1#2#3%
  {%
    \ekvd@ifnew{}{#2}%
      {%
        \ekvd@assert@filledarg{#3}%
          {%
            \ekvd@newreg#3{box}%
            \ekvd@ifalso
              {%
                \let\ekvd@prot\protected
                \ekvd@add@val{#2}{#1\setbox#3=\hbox{\begingroup##1\endgroup}}{}%
              }%
              {%
                \protected\ekvd@long\ekvdef\ekvd@set{#2}%
                  {#1\setbox#3=\hbox{\begingroup##1\endgroup}}%
              }%
          }%
      }%
  }
\protected\def\ekvd@t@box{\ekvd@type@box{}}
\protected\def\ekvd@t@gbox{\ekvd@type@box\global}
%    \end{macrocode}
% \end{macro}
%
% \begin{macro}[internal]{\ekvd@type@toks,\ekvd@t@toks,\ekvd@t@gtoks}
% Similar to |box|, but set the |toks|.
%    \begin{macrocode}
\protected\def\ekvd@type@toks#1#2#3%
  {%
    \ekvd@ifnew{}{#2}%
      {%
        \ekvd@assert@filledarg{#3}%
          {%
            \ekvd@newreg#3{toks}%
            \ekvd@ifalso
              {%
                \let\ekvd@prot\protected
                \ekvd@add@val{#2}{#1#3={##1}}{}%
              }%
              {\protected\ekvd@long\ekvdef\ekvd@set{#2}{#1#3={##1}}}%
          }%
      }%
  }
\protected\def\ekvd@t@toks{\ekvd@type@toks{}}
\protected\def\ekvd@t@gtoks{\ekvd@type@toks\global}
%    \end{macrocode}
% \end{macro}
%
% \begin{macro}[internal]{\ekvd@type@preapptoks,\ekvd@t@apptoks,\ekvd@t@gapptoks}
% Just like |toks|, but expand the current contents of the |toks| register to
% append the new contents.
%    \begin{macrocode}
\ekvd@ifprimitive\toksapp
  {%
    \protected\def\ekvd@type@preapptoks#1#2#3%
      {%
        \ekvd@ifnew{}{#2}%
          {%
            \ekvd@assert@filledarg{#3}%
              {%
                \ekvd@newreg#3{toks}%
                \ekvd@ifalso
                  {%
                    \let\ekvd@prot\protected
                    \ekvd@add@val{#2}{#1#3{##1}}{}%
                  }%
                  {\protected\ekvd@long\ekvdef\ekvd@set{#2}{#1#3{##1}}}%
              }%
          }%
      }
    \protected\def\ekvd@t@apptoks{\ekvd@type@preapptoks\toksapp}
    \protected\def\ekvd@t@gapptoks{\ekvd@type@preapptoks\gtoksapp}
    \protected\def\ekvd@t@pretoks{\ekvd@type@preapptoks\tokspre}
    \protected\def\ekvd@t@gpretoks{\ekvd@type@preapptoks\gtokspre}
  }
  {%
    \protected\def\ekvd@type@apptoks#1#2#3%
      {%
        \ekvd@ifnew{}{#2}%
          {%
            \ekvd@assert@filledarg{#3}%
              {%
                \ekvd@newreg#3{toks}%
                \ekvd@ifalso
                  {%
                    \let\ekvd@prot\protected
                    \ekvd@add@val{#2}{#1#3=\expandafter{\the#3##1}}{}%
                  }%
                  {%
                    \protected\ekvd@long\ekvdef\ekvd@set{#2}%
                      {#1#3=\expandafter{\the#3##1}}%
                  }%
              }%
          }%
      }
    \protected\def\ekvd@t@apptoks{\ekvd@type@apptoks{}}
    \protected\def\ekvd@t@gapptoks{\ekvd@type@apptoks\global}
    \newtoks\ekvd@toks
    \protected\def\ekvd@type@pretoks#1#2#3%
      {%
        \ekvd@ifnew{}{#2}%
          {%
            \ekvd@assert@filledarg{#3}%
              {%
                \ekvd@newreg#3{toks}%
                \ekvd@ifalso
                  {%
                    \let\ekvd@prot\protected
                    \ekvd@add@val{#2}%
                      {%
                        \ekvd@toks={##1}%
                        #1#3=\expandafter{\the\expandafter\ekvd@toks\the#3}%
                      }%
                      {}%
                  }%
                  {%
                    \protected\ekvd@long\ekvdef\ekvd@set{#2}%
                      {%
                        \ekvd@toks={##1}%
                        #1#3=\expandafter{\the\expandafter\ekvd@toks\the#3}%
                      }%
                  }%
              }%
          }%
      }
    \protected\def\ekvd@t@pretoks{\ekvd@type@pretoks{}}
    \protected\def\ekvd@t@gpretoks{\ekvd@type@pretoks\global}
  }
%    \end{macrocode}
% \end{macro}
%
% \begin{macro}[internal]
%   {
%     \ekvd@type@reg,\ekvd@t@int,\ekvd@t@eint,\ekvd@t@gint,\ekvd@t@xint,
%     \ekvd@t@dimen,\ekvd@t@edimen,\ekvd@t@gdimen,\ekvd@t@xdimen,
%     \ekvd@t@skip,\ekvd@t@eskip,\ekvd@t@gskip,\ekvd@t@xskip
%   }
% The |\ekvd@type@reg| can handle all the types for which the assignment will
% just be \texttt{\meta{register}=\meta{value}}.
%    \begin{macrocode}
\protected\def\ekvd@type@reg#1#2#3#4#5#6%
  {%
    \ekvd@ifnew{}{#5}%
      {%
        \ekvd@assert@filledarg{#6}%
          {%
            \ekvd@newreg#6{#1}%
            \ekvd@ifalso
              {%
                \let\evkd@prot\protected
                \ekvd@add@val{#5}{#2#6=#3##1#4\relax}{}%
              }%
              {\protected\ekvd@long\ekvdef\ekvd@set{#5}{#2#6=#3##1#4\relax}}%
          }%
      }%
  }
\protected\def\ekvd@t@int{\ekvd@type@reg{count}{}{}{}}
\protected\def\ekvd@t@eint{\ekvd@type@reg{count}{}\numexpr\relax}
\protected\def\ekvd@t@gint{\ekvd@type@reg{count}\global{}{}}
\protected\def\ekvd@t@xint{\ekvd@type@reg{count}\global\numexpr\relax}
\protected\def\ekvd@t@dimen{\ekvd@type@reg{dimen}{}{}{}}
\protected\def\ekvd@t@edimen{\ekvd@type@reg{dimen}{}\dimexpr\relax}
\protected\def\ekvd@t@gdimen{\ekvd@type@reg{dimen}\global{}{}}
\protected\def\ekvd@t@xdimen{\ekvd@type@reg{dimen}\global\dimexpr\relax}
\protected\def\ekvd@t@skip{\ekvd@type@reg{skip}{}{}{}}
\protected\def\ekvd@t@eskip{\ekvd@type@reg{skip}{}\glueexpr\relax}
\protected\def\ekvd@t@gskip{\ekvd@type@reg{skip}\global{}{}}
\protected\def\ekvd@t@xskip{\ekvd@type@reg{skip}\global\glueexpr\relax}
%    \end{macrocode}
% \end{macro}
%
% \begin{macro}[internal]{\ekvd@type@store,\ekvd@t@store,\ekvd@t@gstore}
% The none-expanding |store| types use an |\edef| or |\xdef| and |\unexpanded|
% to be able to also store |#| easily.
%    \begin{macrocode}
\protected\def\ekvd@type@store#1#2#3#4%
  {%
    \ekvd@ifnew{}{#3}%
      {%
        \ekvd@assert@filledarg{#4}%
          {%
            \ekvd@newlet#4\ekv@empty
            \ekvd@ifalso
              {%
                \let\ekvd@prot\protected
                \ekvd@add@val{#3}{#1#4{#2}}{}%
              }%
              {\protected\ekvd@long\ekvdef\ekvd@set{#3}{#1#4{#2}}}%
          }%
      }%
  }
\protected\def\ekvd@t@store{\ekvd@type@store\edef{\ekv@unexpanded{##1}}}
\protected\def\ekvd@t@gstore{\ekvd@type@store\xdef{\ekv@unexpanded{##1}}}
\protected\def\ekvd@t@estore{\ekvd@type@store\edef{##1}}
\protected\def\ekvd@t@xstore{\ekvd@type@store\xdef{##1}}
%    \end{macrocode}
% \end{macro}
%
% \begin{macro}[internal]
%   {
%     \ekvd@type@meta,\ekvd@type@meta@a,\ekvd@type@meta@b,\ekvd@type@meta@c,
%     \ekvd@t@meta,\ekvd@t@nmeta
%   }
% |meta| sets up things such that another instance of |\ekvset| will be run on
% the argument, with the same \set.
%    \begin{macrocode}
\protected\long\def\ekvd@type@meta#1#2#3#4#5#6#7%
  {%
    \ekvd@ifnew{#1}{#6}%
      {%
        \ekvd@assert@filledarg{#7}%
          {%
            \edef\ekvd@tmp{\ekvd@set}%
            \expandafter\ekvd@type@meta@a\expandafter{\ekvd@tmp}{#7}{#2}%
            \ekvd@ifalso
              {\ekv@exparg{#3{#6}}{\ekvd@tmp#4}{#5}}%
              {\csname ekvlet#1\endcsname\ekvd@set{#6}\ekvd@tmp}%
          }%
      }%
  }
\protected\long\def\ekvd@type@meta@a#1#2%
  {%
    \expandafter\ekvd@type@meta@b\expandafter{\ekvset{#1}{#2}}%
  }
\protected\def\ekvd@type@meta@b
  {%
    \expandafter\ekvd@type@meta@c\expandafter
  }
\protected\long\def\ekvd@type@meta@c#1#2%
  {%
    \ekvd@prot\ekvd@long\def\ekvd@tmp#2{#1}%
  }
\protected\def\ekvd@t@meta{\ekvd@type@meta{}{##1}\ekvd@add@val{{##1}}{}}
\protected\def\ekvd@t@nmeta
  {%
    \ekvd@assert@not@long
    \ekvd@type@meta{NoVal}{}\ekvd@add@noval{}\ekvd@assert@not@long@also
  }
%    \end{macrocode}
% \end{macro}
%
% \begin{macro}[internal]
%   {\ekvd@type@smeta,\ekvd@type@smeta@,\ekvd@t@smeta,\ekvd@t@snmeta}
% |smeta| is pretty similar to |meta|, but needs two arguments inside of \val,
% such that the first is the \set\ for which the sub-|\ekvset| and the second is
% the \kv\ list.
%    \begin{macrocode}
\protected\long\def\ekvd@type@smeta#1#2#3#4#5#6#7%
  {%
    \ekvd@ifnew{#1}{#6}%
      {%
        \ekvd@assert@twoargs{#7}%
          {%
            \ekvd@type@meta@a#7{#2}%
            \ekvd@ifalso
              {\ekv@exparg{#3{#6}}{\ekvd@tmp#4}{#5}}%
              {\csname ekvlet#1\endcsname\ekvd@set{#6}\ekvd@tmp}%
          }%
      }%
  }
\protected\def\ekvd@t@smeta{\ekvd@type@smeta{}{##1}\ekvd@add@val{{##1}}{}}
\protected\def\ekvd@t@snmeta
  {%
    \ekvd@assert@not@long
    \ekvd@type@smeta{NoVal}{}\ekvd@add@noval{}\ekvd@assert@not@long@also
  }
%    \end{macrocode}
% \end{macro}
%
% \begin{macro}[internal]
%   {
%     \ekvd@type@choice,
%     \ekvd@populate@choice,\ekvd@populate@choice@,\ekvd@populate@choice@noarg,
%     \ekvd@choice@prefix,\ekvd@choice@prefix@,\ekvd@choice@prefix@done,
%     \ekvd@choice@p@protected,\ekvd@choice@p@protect,
%     \ekvd@choice@p@long,\ekvd@choice@p@long@,
%     \ekvd@t@choice
%   }
% The |choice| type is by far the most complex type, as we have to run a
% sub-parser on the choice-definition list, which should support the |@p@| type
% prefixes as well (but |long| will always throw an error, as they are not
% allowed to be long). |\ekvd@type@choice| will just define the choice-key, the
% handling of the choices definition will be done by |\ekvd@populate@choice|.
%    \begin{macrocode}
\protected\def\ekvd@type@choice#1%
  {%
    \ekvd@assert@not@long
    \ekv@exparg{\ekvd@prot\def\ekvd@tmp##1}%
      {%
        \expandafter\ekvd@h@choice\expandafter
          {\ekvd@choice@name\ekvd@set{#1}{##1}}%
      }%
    \ekvd@ifalso
      {%
        \ekvd@assert@val{#1}%
          {%
            \ekvd@if@not@already@choice{#1}%
              {%
                \ekv@exparg
                  {%
                    \expandafter\ekvd@add@aux
                      \csname\ekv@name\ekvd@set{#1}\endcsname{{##1}}{#1}%
                  }%
                  {\ekvd@tmp{##1}}%
                  {\ekvd@long\ekvdef}\ekvd@assert@not@long@also
              }%
          }%
      }%
      {\ekvlet\ekvd@set{#1}\ekvd@tmp}%
  }
%    \end{macrocode}
% |\ekvd@populate@choice| just uses |\ekvparse| and then gives control to
% |\ekvd@populate@choice@noarg|, which throws an error, and
% |\ekvd@populate@choice@|.
%    \begin{macrocode}
\protected\def\ekvd@populate@choice
  {%
    \ekvparse\ekvd@populate@choice@noarg\ekvd@populate@choice@
  }
\protected\long\def\ekvd@populate@choice@noarg#1%
  {%
    \expandafter\ekvd@err@missing@definition@msg\expandafter{\ekvd@cur : #1}%
  }
%    \end{macrocode}
% |\ekvd@populate@choice@| runs the prefix-test, if there is none we can
% directly define the choice, for that |\ekvd@set@choice| will expand to the
% current choice-key's name, which will have been defined by |\ekvd@t@choice|.
% If there is a prefix run the prefix grabbing routine, which was altered for
% |@type@choice|.
%    \begin{macrocode}
\protected\long\def\ekvd@populate@choice@#1#2%
  {%
    \ekvd@clear@prefixes
    \ekvd@ifspace{#1}%
      {\ekvd@choice@prefix{\ekv@mark#1}\ekv@mark#1\ekv@stop}%
      {%
        \expandafter\edef
          \csname\ekvd@choice@name\ekvd@set\ekvd@set@choice{#1}\endcsname
      }%
      {\unexpanded{#2}}%
  }
\protected\def\ekvd@choice@prefix#1#2
  {%
    \ekv@strip{#2}{\ekvd@choice@prefix@{#1}}\ekv@mark
  }
\protected\def\ekvd@choice@prefix@#1#2#3\ekv@stop
  {%
    \ekv@ifdefined{ekvd@choice@p@#2}%
      {%
        \csname ekvd@choice@p@#2\endcsname
        \ekvd@ifspace{#3}%
          {\ekvd@choice@prefix{#3}#3\ekv@stop}%
          {\ekvd@choice@prefix@done{#3}}%
      }%
      {\ekvd@choice@prefix@done{#1}}%
  }
\protected\def\ekvd@choice@prefix@done#1%
  {%
    \ekvd@prot\expandafter\edef
      \csname
        \ekv@strip{#1}{\ekvd@choice@name\ekvd@set\ekvd@set@choice}%
      \endcsname
  }
\protected\def\ekvd@choice@p@protected{\let\ekvd@prot\protected}
\let\ekvd@choice@p@protect\ekvd@choice@p@protected
\protected\def\ekvd@choice@invalid@p#1\ekvd@ifspace#2%
  {%
    \expandafter\ekvd@choice@invalid@p@\expandafter{\ekv@gobble@mark#2}{#1}%
    \ekvd@ifspace{#2}%
  }
\protected\def\ekvd@choice@invalid@p@#1#2%
  {%
    \expandafter\ekvd@err@no@prefix@msg\expandafter{\ekvd@cur : #2 #1}{#2}%
  }
\protected\def\ekvd@choice@p@long{\ekvd@choice@invalid@p{long}}%
\protected\def\ekvd@choice@p@also{\ekvd@choice@invalid@p{also}}%
\protected\def\ekvd@choice@p@new{\ekvd@choice@invalid@p{new}}%
%    \end{macrocode}
% Finally we're able to set up the |@t@choice| macro, which has to store the
% current choice-key's name, define the key, and parse the available choices.
%    \begin{macrocode}
\protected\long\def\ekvd@t@choice#1#2%
  {%
    \ekvd@ifnew{}{#1}%
      {%
        \ekvd@assert@arg
          {%
            \ekvd@type@choice{#1}%
            \def\ekvd@set@choice{#1}%
            \ekvd@populate@choice{#2}%
          }%
      }%
  }
%    \end{macrocode}
% \end{macro}
%
% \begin{macro}[internal]{\ekvd@t@choice-store,\ekvd@t@choice-enum}
%   These two types define a special kind of |choice| key and are quite similar,
%   the only difference is what the different choices do (hence they use a
%   shared initialisation which differs in the chosen |populate| step).
%    \begin{macrocode}
\protected\long\expandafter\def\csname ekvd@t@choice-store\endcsname
  {\ekvd@type@choicespecial\ekvd@populate@choicestore}
\protected\long\expandafter\def\csname ekvd@t@choice-enum\endcsname
  {\ekvd@type@choicespecial\ekvd@populate@choiceenum}
%    \end{macrocode}
% \begin{macro}[internal]{\ekvd@type@choicespecial}
%   Initialise similar to a |choice| key. The difference is that we require two
%   arguments (which we assert), a macro to store things in, and a |csv|-list
%   containing the allowed values. |#1| is the |populate| macro according to the
%   type used.
%    \begin{macrocode}
\protected\long\def\ekvd@type@choicespecial#1#2#3%
  {%
    \ekvd@ifnew{}{#2}%
      {%
        \ekvd@assert@twoargs{#3}%
          {%
            \ekvd@type@choice{#2}%
            \def\ekvd@set@choice{#2}%
            #1#3%
          }%
      }%
  }
%    \end{macrocode}
% \end{macro}
% \begin{macro}[internal]
%   {\ekvd@populate@choicestore,\ekvd@populate@choicestore@}
%   We initialise the storing macro if it doesn't yet exist, and then we loop
%   over the value list. The |\edef|s with |\unexpanded| are both necessary to
%   be able to store macro parameter tokens (the outer protects at define time,
%   the inner at use time).
%    \begin{macrocode}
\protected\long\def\ekvd@populate@choicestore#1%
  {%
    \ekvd@newlet#1\ekv@empty
    \ekvcsvloop{\ekvd@populate@choicestore@#1}%
  }
\protected\long\def\ekvd@populate@choicestore@#1#2%
  {%
    \protected\expandafter\edef
      \csname\ekvd@choice@name\ekvd@set\ekvd@set@choice{#2}\endcsname
      {\unexpanded{\edef#1{\unexpanded{#2}}}}%
  }
%    \end{macrocode}
% \end{macro}
% \begin{macro}[internal]
%   {\ekvd@populate@choiceenum,\ekvd@populate@choiceenum@}
%   This is similar to the population of a |choice-store| type, but instead of
%   storing the values in a macro this initialises a count and stores the
%   position of the value in the list inside that count (zero-indexed). The
%   space is necessary to terminate the number scanning, which is the reason we
%   use |\@firstofone| (so that the space after the macro name isn't gobbled by
%   \TeX).
%    \begin{macrocode}
\protected\long\def\ekvd@populate@choiceenum#1%
  {%
    \ekvd@newreg#1{count}%
    \def\ekvd@tmp{0}%
    \ekvcsvloop{\ekvd@populate@choiceenum@#1}%
  }
\protected\long\def\ekvd@populate@choiceenum@#1#2%
  {%
    \protected\expandafter\edef
      \csname\ekvd@choice@name\ekvd@set\ekvd@set@choice{#2}\endcsname
      {#1=\@firstofone{\ekvd@tmp} }%
    \edef\ekvd@tmp{\the\numexpr\ekvd@tmp+1\relax}%
  }
%    \end{macrocode}
% \end{macro}
% \end{macro}
%
% \begin{macro}[internal]{\ekvd@t@unknown-choice}
%    \begin{macrocode}
\protected\long\expandafter\def\csname ekvd@t@unknown-choice\endcsname#1#2%
  {%
    \ekvd@assert@new@for@name{\ekvd@unknown@choice@name\ekvd@set{#1}}%
      {%
        \ekvd@assert@arg
          {%
            \ekvd@assert@not@long
            \ekvd@assert@not@also
            \ekvd@prot\expandafter
            \def\csname\ekvd@unknown@choice@name\ekvd@set{#1}\endcsname##1{#2}%
          }%
      }%
  }
%    \end{macrocode}
% \end{macro}
%
% \begin{macro}[internal]
%   {\ekvd@t@unknown,\ekvd@type@unknown@code,\ekvd@type@unknown@noval}
%   The |unknown| type has different subtypes which would be the key names for
%   other types. It is first checked whether that subtype is defined, if it
%   isn't throw an error, else use that subtype.
%    \begin{macrocode}
\protected\long\def\ekvd@t@unknown#1#2%
  {%
    \ekv@ifdefined{ekvd@type@unknown@\detokenize{#1}}%
      {\csname ekvd@type@unknown@\detokenize{#1}\endcsname{#2}}%
      \ekvd@err@misused@unknown
  }
%    \end{macrocode}
%   The |unknown noval| type can use |\ekvdefunknownNoVal| directly (after
%   asserting some prefixes).
%    \begin{macrocode}
\protected\long\def\ekvd@type@unknown@noval#1%
  {%
    \ekvd@assert@new@for@name{\ekv@name\ekvd@set{}uN}%
      {%
        \ekvd@assert@arg
          {%
            \ekvd@assert@not@also
            \ekvd@assert@not@long
            \ekvd@prot\ekvdefunknownNoVal\ekvd@set{#1}%
          }%
      }%
  }
%    \end{macrocode}
%   The |unknown code| type uses some trickery during the definition in order to
%   swap out |#1| and |#2| in the user supplied definition. This is done via a
%   temporary macro that stores the definition but gets the parameter numbers
%   reversed while the real definition is done.
%    \begin{macrocode}
\protected\long\def\ekvd@type@unknown@code#1%
  {%
    \ekvd@assert@new@for@name{\ekv@name\ekvd@set{}u}%
      {%
        \ekvd@assert@arg
          {%
            \ekvd@assert@not@also
            \begingroup
              \def\ekvd@tmp##1##2{#1}%
              \ekv@exparg
              {%
                \endgroup
                \ekvd@prot\ekvd@long\ekvdefunknown\ekvd@set
              }%
              {\ekvd@tmp{##2}{##1}}%
          }%
      }%
  }
%    \end{macrocode}
% \end{macro}
%
% \begin{macro}[internal]
%   {
%     \ekvd@type@unknown@redirect,
%     \ekvd@type@unknown@redirect-code,
%     \ekvd@type@unknown@redirect-noval
%   }
%   The |unknown redirect| types also just forward to |\ekvredirectunknown|
%   after asserting some prefixes.
%    \begin{macrocode}
\protected\edef\ekvd@type@unknown@redirect#1%
  {%
    \expandafter\noexpand\csname ekvd@type@unknown@redirect-code\endcsname{#1}%
    \expandafter\noexpand\csname ekvd@type@unknown@redirect-noval\endcsname{#1}%
  }
\protected\expandafter\def\csname ekvd@type@unknown@redirect-code\endcsname#1%
  {%
    \ekvd@assert@new@for@name{\ekv@name\ekvd@set{}u}%
      {%
        \ekvd@assert@arg
          {%
            \ekvd@assert@not@also
            \ekvd@assert@not@protected
            \expandafter\ekvredirectunknown\expandafter{\ekvd@set}{#1}%
          }%
      }%
  }
\protected\expandafter\def\csname ekvd@type@unknown@redirect-noval\endcsname#1%
  {%
    \ekvd@assert@new@for@name{\ekv@name\ekvd@set{}uN}%
      {%
        \ekvd@assert@arg
          {%
            \ekvd@assert@not@also
            \ekvd@assert@not@protected
            \ekvd@assert@not@long
            \expandafter\ekvredirectunknownNoVal\expandafter{\ekvd@set}{#1}%
          }%
      }%
  }
%    \end{macrocode}
% \end{macro}
%
% \subsubsection{Key Type Helpers}
% There are some keys that might need helpers during their execution (not during
% their definition, which are gathered as |@type@| macros). These helpers are
% named |@h@|.
%
% \begin{macro}[internal]{\ekvd@h@choice,\ekvd@h@choice@}
% The |choice| helper will just test whether the given choice was defined, if
% not throw an error expandably, else call the macro which stores the code for
% this choice.
%    \begin{macrocode}
\def\ekvd@h@choice#1%
  {%
    \expandafter\ekvd@h@choice@
      \csname\ifcsname#1\endcsname#1\else relax\fi\endcsname
      {#1}%
  }
\def\ekvd@h@choice@#1#2%
  {%
    \ifx#1\relax
      \ekvd@err@choice@invalid{#2}%
      \expandafter\@gobble
    \fi
    #1%
  }
%    \end{macrocode}
% \end{macro}
%
%
% \subsubsection{Handling \texttt{also}}
%
% \begin{macro}[internal]
%   {\ekvd@add@val,\ekvd@add@noval,\ekvd@add@aux,\ekvd@add@aux@}
%    \begin{macrocode}
\protected\long\def\ekvd@add@val#1#2#3%
  {%
    \ekvd@assert@val{#1}%
      {%
        \expandafter\ekvd@add@aux\csname\ekv@name\ekvd@set{#1}\endcsname{{##1}}%
          {#1}{#2}{\ekvd@long\ekvdef}{#3}%
      }%
  }
\protected\long\def\ekvd@add@noval#1#2#3%
  {%
    \ekvd@assert@noval{#1}%
      {%
        \expandafter\ekvd@add@aux\csname\ekv@name\ekvd@set{#1}N\endcsname{}%
          {#1}{#2}\ekvdefNoVal{#3}%
      }%
  }
\protected\long\def\ekvd@add@aux#1#2%
  {%
    \ekvd@extract@prefixes#1%
    \expandafter\ekvd@add@aux@\expandafter{#1#2}%
  }
\protected\long\def\ekvd@add@aux@#1#2#3#4#5%
  {%
    #5%
    \ekvd@prot#4\ekvd@set{#2}{#1#3}%
  }
%    \end{macrocode}
% \end{macro}
%
% \begin{macro}[internal]
%   {
%     \ekvd@extract@prefixes,\ekvd@extract@prefixes@,
%     \ekvd@extract@prefixes@long,\ekvd@extract@prefixes@prot
%   }
% This macro checks which prefixes were used for the definition of a macro and
% sets |\ekvd@long| and |\ekvd@prot| accordingly.
%    \begin{macrocode}
\protected\def\ekvd@extract@prefixes#1%
  {%
    \expandafter\ekvd@extract@prefixes@\meaning#1\ekvd@stop
  }
%    \end{macrocode}
% In the following definition |#1| will get replaced by |macro:|, |#2| by
% |\long| and |#3| by |\protected| (in each, all tokens will have category
% other). This allows us to parse the |\meaning| of a macro for those strings.
%    \begin{macrocode}
\protected\def\ekvd@extract@prefixes@#1#2#3%
  {%
    \protected\def\ekvd@extract@prefixes@##1#1##2\ekvd@stop
      {%
        \ekvd@extract@prefixes@long
          ##1\ekvd@mark\@firstofone#2\ekvd@mark\@gobble\ekvd@stop
          {\let\ekvd@long\long}%
        \ekvd@extract@prefixes@prot
          ##1\ekvd@mark\@firstofone#3\ekvd@mark\@gobble\ekvd@stop
          {\let\ekvd@prot\protected}%
      }%
    \protected\def\ekvd@extract@prefixes@long##1#2##2\ekvd@mark##3##4\ekvd@stop
      {##3}%
    \protected\def\ekvd@extract@prefixes@prot##1#3##2\ekvd@mark##3##4\ekvd@stop
      {##3}%
  }
%    \end{macrocode}
% We use a temporary macro to expand the three arguments of
% |\ekvd@extract@prefixes@|, which will set up the real meaning of itself and
% the parsing for |\long| and |\protected|.
%    \begin{macrocode}
\begingroup
\edef\ekvd@tmp
  {%
    \endgroup
    \ekvd@extract@prefixes@
      {\detokenize{macro:}}%
      {\string\long}%
      {\string\protected}%
  }
\ekvd@tmp
%    \end{macrocode}
% \end{macro}
%
%
% \subsubsection{Tests}
%
% \begin{macro}[internal]{\ekvd@newlet,\ekvd@newreg}
% These macros test whether a control sequence is defined, if it isn't they
% define it, either via |\let| or via the correct \cs[no-index]{new\meta{reg}}.
%    \begin{macrocode}
\protected\def\ekvd@newlet#1#2%
  {%
    \ifdefined#1\ekv@fi@gobble\fi\@firstofone{\let#1#2}%
  }
\protected\def\ekvd@newreg#1#2%
  {%
    \ifdefined#1\ekv@fi@gobble\fi\@firstofone{\csname new#2\endcsname#1}%
  }
%    \end{macrocode}
% \end{macro}
%
% \begin{macro}[internal]
%   {\ekvd@assert@twoargs,\ekvd@ifnottwoargs,\ekvd@ifempty@gtwo}
% A test for exactly two tokens can be reduced for an empty-test after gobbling
% two tokens, in the case that there are fewer tokens than two in the argument,
% only macros will be gobbled that are needed for the true branch, which doesn't
% hurt, and if there are more this will not be empty.
%    \begin{macrocode}
\long\def\ekvd@assert@twoargs#1%
  {%
    \ekvd@ifnottwoargs{#1}{\ekvd@err@missing@definition}%
  }
\long\def\ekvd@ifnottwoargs#1%
  {%
    \ekvd@ifempty@gtwo#1\ekv@ifempty@B
      \ekv@ifempty@false\ekv@ifempty@A\ekv@ifempty@B\@firstoftwo
  }
\long\def\ekvd@ifempty@gtwo#1#2{\ekv@ifempty@\ekv@ifempty@A}
%    \end{macrocode}
% \end{macro}
%
% \begin{macro}[internal]
%   {
%     \ekvd@assert@val,\ekvd@assert@val@,\ekvd@assert@noval,\ekvd@assert@noval@,
%     \ekvd@extract@args,\ekvd@extracted@args,\ekvd@one@arg@string
%   }
% Assert that a given key is defined as a value taking key or a |NoVal| key with
% the correct argument structure, respectively.
%    \begin{macrocode}
\protected\def\ekvd@assert@val#1%
  {%
    \ekvifdefined\ekvd@set{#1}%
      {\expandafter\ekvd@assert@val@\csname\ekv@name\ekvd@set{#1}\endcsname}%
      {%
        \ekvifdefinedNoVal\ekvd@set{#1}%
          \ekvd@err@add@val@on@noval
          {\ekvd@err@undefined@key{#1}}%
        \@gobble
      }%
  }
\protected\def\ekvd@assert@val@#1%
  {%
    \expandafter\ekvd@extract@args\meaning#1\ekvd@stop
    \unless\ifx\ekvd@extracted@args\ekvd@one@arg@string
      \ekvd@err@unsupported@arg
    \fi
    \@firstofone
  }%
\protected\def\ekvd@assert@noval#1%
  {%
    \ekvifdefinedNoVal\ekvd@set{#1}%
      {\expandafter\ekvd@assert@noval@\csname\ekv@name\ekvd@set{#1}N\endcsname}%
      {%
        \ekvifdefined\ekvd@set{#1}%
          \ekvd@err@add@noval@on@val
          {\ekvd@err@undefined@key{#1}}%
        \@gobble
      }%
  }
\protected\def\ekvd@assert@noval@#1%
  {%
    \expandafter\ekvd@extract@args\meaning#1\ekvd@stop
    \unless\ifx\ekvd@extracted@args\ekv@empty
      \ekvd@err@unsupported@arg
    \fi
    \@firstofone
  }
\protected\def\ekvd@extract@args#1%
  {%
    \protected\def\ekvd@extract@args##1#1##2->##3\ekvd@stop
      {\def\ekvd@extracted@args{##2}}%
  }
\expandafter\ekvd@extract@args\expandafter{\detokenize{macro:}}
\edef\ekvd@one@arg@string{\string#1}
%    \end{macrocode}
% \end{macro}
%
% \begin{macro}[internal]{\ekvd@assert@arg,\ekvd@assert@arg@msg,\ekvd@ifnoarg}
% There is no need to actually define |\ekvd@ifnoarg| here, as it will be set by
% either |\ekvd@arg| or |\ekvd@noarg|.
%    \begin{macrocode}
\def\ekvd@assert@arg{\ekvd@ifnoarg\ekvd@err@missing@definition}
\long\def\ekvd@assert@arg@msg#1%
  {%
    \ekvd@ifnoarg{\ekvd@err@missing@definition@msg{#1}}%
  }
%    \end{macrocode}
% \end{macro}
%
% \begin{macro}[internal]{\ekvd@assert@filledarg,\ekvd@ifnoarg@or@empty}
%    \begin{macrocode}
\long\def\ekvd@assert@filledarg#1%
  {%
    \ekvd@ifnoarg@or@empty{#1}\ekvd@err@missing@definition
  }
\long\def\ekvd@ifnoarg@or@empty#1%
  {%
    \ekvd@ifnoarg
      \@firstoftwo
      {\ekv@ifempty{#1}}%
  }
%    \end{macrocode}
% \end{macro}
%
% \begin{macro}[internal]
%   {
%     \ekvd@assert@not@long,\ekvd@assert@not@protected,\ekvd@assert@not@also
%     \ekvd@assert@not@long@also,\ekvd@assert@not@protected@also,
%     \ekvd@assert@new,\ekvd@assert@not@new
%   }
% Some key-types don't want to be |also|, |\long| or |\protected|, so we provide
% macros to test this and throw an error, this could be silently ignored but now
% users will learn to not use unnecessary stuff which slows the compilation
% down.
%    \begin{macrocode}
\def\ekvd@assert@not@long{\ifx\ekvd@long\long\ekvd@err@no@prefix{long}\fi}
\def\ekvd@assert@not@protected
  {\ifx\ekvd@prot\protected\ekvd@err@no@prefix{protected}\fi}
\def\ekvd@assert@not@also{\ekvd@ifalso{\ekvd@err@no@prefix{also}}{}}
\def\ekvd@assert@not@long@also
  {\ifx\ekvd@long\long\ekvd@err@no@prefix@also{long}\fi}
\def\ekvd@assert@not@protected@also
  {\ifx\ekvd@prot\protected\ekvd@err@no@prefix@also{protected}\fi}
\def\ekvd@assert@new#1#2%
  {\csname ekvifdefined#1\endcsname\ekvd@set{#2}{\ekvd@err@not@new}}
\def\ekvd@assert@not@new
  {\ifx\ekvd@ifnew\ekvd@assert@new\ekvd@err@no@prefix{new}\fi}
\def\ekvd@assert@new@for@name#1%
  {%
    \ifx\ekvd@ifnew\ekvd@assert@new
      \ekv@fi@firstoftwo
    \fi
    \@secondoftwo
      {\ekv@ifdefined{#1}\ekvd@err@not@new}%
      \@firstofone
  }
%    \end{macrocode}
% \end{macro}
%
% \begin{macro}[internal]
%   {
%     \ekvd@if@not@already@choice, \ekvd@if@not@already@choice@a,
%     \ekvd@if@not@already@choice@b
%   }
% It is bad to use |also| on a key that already contains a |choice|, as both
% choices would share the same valid values and thus lead to each callback being
% used twice. The following is a rudimentary test against this.
%    \begin{macrocode}
\protected\def\ekvd@if@not@already@choice#1%
  {%
    \expandafter\ekvd@if@not@already@choice@a
      \csname\ekv@name\ekvd@set{#1}\endcsname
      {}\ekvd@h@choice\ekvd@stop
  }
\protected\def\ekvd@if@not@already@choice@a
  {%
    \expandafter\ekvd@if@not@already@choice@b
  }
\long\protected\def\ekvd@if@not@already@choice@b#1\ekvd@h@choice#2\ekvd@stop
  {%
    \ekv@ifempty{#2}\@firstofone\@gobble
  }
%    \end{macrocode}
% \end{macro}
%
% \begin{macro}[internal]{\ekvd@ifspace,\ekvd@ifspace@}
% Yet another test which can be reduced to an if-empty, this time by gobbling
% everything up to the first space.
%    \begin{macrocode}
\long\def\ekvd@ifspace#1%
  {%
    \ekvd@ifspace@#1 \ekv@ifempty@B
      \ekv@ifempty@false\ekv@ifempty@A\ekv@ifempty@B\@firstoftwo
  }
\long\def\ekvd@ifspace@#1 % keep this space
  {%
    \ekv@ifempty@\ekv@ifempty@A
  }
%    \end{macrocode}
% \end{macro}
%
% \subsubsection{Messages}
%
% Most messages of \expkvd\ are not expandable, since they only appear during
% key-definition, which is not expandable anyway.
%
% \begin{macro}[internal]
%   {
%     \ekvd@errm,\ekvd@err@missing@definition,\ekvd@err@missing@definition@msg,
%     \ekvd@err@missing@type,\ekvd@err@undefined@prefix,\ekvd@err@undefined@key,
%     \ekvd@err@no@prefix,\ekvd@err@no@prefix@msg,\ekvd@err@no@prefix@also,
%     \ekvd@err@add@val@on@noval,\ekvd@err@add@noval@on@val,
%     \ekvd@err@unsupported@arg,\ekvd@err@not@new
%   }
% The non-expandable error messages are boring, so here they are:
%    \begin{macrocode}
\protected\def\ekvd@errm#1{\errmessage{expkv-def Error: #1}}
\protected\def\ekvd@err@missing@definition
  {\ekvd@errm{Missing definition for key `\ekvd@cur'}}
\protected\def\ekvd@err@missing@definition@msg#1%
  {\ekvd@errm{Missing definition for key `\ekv@unexpanded{#1}'}}
\protected\def\ekvd@err@missing@type
  {\ekvd@errm{Missing type prefix for key `\ekvd@cur'}}
\protected\def\ekvd@err@undefined@prefix#1%
  {%
    \ekvd@errm
      {%
        Undefined prefix `\ekv@unexpanded{#1}' found while processing
        `\ekvd@cur'%
      }%
  }
\protected\def\ekvd@err@undefined@key#1%
  {%
    \ekvd@errm
      {Undefined key `\ekv@unexpanded{#1}' found while processing `\ekvd@cur'}%
  }
\protected\def\ekvd@err@undefined@noval#1%
  {%
    \ekvd@errm
      {%
        Undefined noval key `\unexpanded{#1}' found while processing
        `\ekvd@cur'%
      }%
  }
\protected\def\ekvd@err@no@prefix#1%
  {\ekvd@errm{prefix `#1' not accepted in `\ekvd@cur'}}
\protected\def\ekvd@err@no@prefix@msg#1#2%
  {\ekvd@errm{prefix `#2' not accepted in `\ekv@unexpanded{#1}'}}
\protected\def\ekvd@err@no@prefix@also#1%
  {\ekvd@errm{`\ekvd@cur' not allowed with a `#1' key}}
\protected\def\ekvd@err@add@val@on@noval
  {\ekvd@errm{`\ekvd@cur' not allowed with a NoVal key}}
\protected\def\ekvd@err@add@noval@on@val
  {\ekvd@errm{`\ekvd@cur' not allowed with a value taking key}}
\protected\def\ekvd@err@unsupported@arg\fi\@firstofone#1%
  {%
    \fi
    \ekvd@errm
      {%
        Existing key-macro has the unsupported argument string
        `\ekvd@extracted@args' for key `\ekvd@cur'%
      }%
  }
\protected\def\ekvd@err@not@new
  {\ekvd@errm{The key for `\ekvd@cur' is already defined}}
\protected\long\def\ekvd@err@misused@unknown
  {\ekvd@errm{Misuse of the unknown type found while processing `\ekvd@cur'}}
%    \end{macrocode}
% \end{macro}
%
% \begin{macro}[internal]
%   {
%     \ekvd@err@choice@invalid,\ekvd@err@choice@invalid@,\ekvd@choice@name,
%     \ekvd@unknown@choice@name
%   }
% |\ekvd@err@choice@invalid| will have to use this mechanism to throw its
% message. Also we have to retrieve the name parts of the choice in an easy way,
% so we use parentheses of catcode 8 here, which should suffice in most cases to
% allow for a correct separation.
%    \begin{macrocode}
\def\ekvd@err@choice@invalid#1%
  {%
    \ekvd@err@choice@invalid@#1%
  }
\begingroup
\catcode40=8
\catcode41=8
\@firstofone{\endgroup
\def\ekvd@choice@name#1#2#3%
  {%
    ekvd#1(#2)\detokenize{#3}%
  }
\def\ekvd@unknown@choice@name#1#2%
  {%
    ekvd:u:#1(#2)%
  }
\def\ekvd@err@choice@invalid@ ekvd#1(#2)\detokenize#3%
  {%
    \ekv@ifdefined{\ekvd@unknown@choice@name{#1}{#2}}%
      {\csname\ekvd@unknown@choice@name{#1}{#2}\endcsname{#3}}%
      {\ekvd@err{invalid choice `#3' for `#2' in set `#1'}}%
  }
}
%    \end{macrocode}
% \end{macro}
%
% \begin{macro}[internal]{\ekvd@err}
% The expandable error messages use |\ekvd@err|, which is just like |\ekv@err|
% from \expkv. It uses a runaway argument to start the error message.
%    \begin{macrocode}
\ekv@exparg{\long\def\ekvd@err#1}{\ekverr{expkv-def}{#1}}
%    \end{macrocode}
% \end{macro}
%
% Now everything that's left is to reset the category code of |@|.
%    \begin{macrocode}
\catcode`\@=\ekvd@tmp
%    \end{macrocode}
%
% \gobbledocstriptag
%</tex>
%^^A=<<
%
%^^A=<<
%
% \end{implementation}^^A=<<
%
% \clearpage
% \PrintIndex
%
\endinput
%
^^A vim: ft=tex fdm=marker fmr=>>=,=<<

\ProvidesPackage{expkv-def}%
  [\ekvdDate\space v\ekvdVersion\space a key-defining frontend for expkv]
%    \end{macrocode}
% \gobbledocstriptag
%</pkg>
%^^A=<<
%^^A the ConTeXt module >>=
% \subsection{The \ConTeXt\ module}
% \gobbledocstriptag
%<*ctx>
%    \begin{macrocode}
\writestatus{loading}{ConTeXt User Module / expkv-def}
\usemodule[expkv]
\unprotect
\input expkv-def.tex
\writestatus{loading}
  {ConTeXt User Module / expkv-def / Version \ekvdVersion\space loaded}
\protect\endinput
%    \end{macrocode}
% \gobbledocstriptag
%</ctx>
%^^A=<<
%^^A main file >>=
% \subsection{The Generic Code}
% The rest of this implementation will be the generic code.
% \gobbledocstriptag
%<*tex>
%
% Load \expkv\ if the package didn't already do so -- since \expkv\ has
% safeguards against being loaded twice this does no harm and the overhead
% isn't that big. Also we reuse some of the internals of \expkv\ to save us from
% retyping them.
%    \begin{macrocode}
\input expkv
%    \end{macrocode}
%
% We make sure that \file{expkv-def.tex} is only input once:
%    \begin{macrocode}
\expandafter\ifx\csname ekvdVersion\endcsname\relax
\else
  \expandafter\endinput
\fi
%    \end{macrocode}
%
% \begin{macro}{\ekvdVersion,\ekvdDate}
% We're on our first input, so lets store the version and date in a macro.
%    \begin{macrocode}
\def\ekvdVersion{0.8c}
\def\ekvdDate{2021-09-20}
%    \end{macrocode}
% \end{macro}
%
% If the \LaTeX\ format is loaded we want to be a good file and report back who
% we are, for this the package will have defined |\ekvd@tmp| to use
% |\ProvidesFile|, else this will expand to a |\relax| and do no harm.
%    \begin{macrocode}
\csname ekvd@tmp\endcsname
%    \end{macrocode}
%
% Store the category code of |@| to later be able to reset it and change it to
% 11 for now.
%    \begin{macrocode}
\expandafter\chardef\csname ekvd@tmp\endcsname=\catcode`\@
\catcode`\@=11
%    \end{macrocode}
% |\ekvd@tmp| will be reused later to handle expansion during the key
% defining. But we don't need it to ever store information long-term after
% \expkvd\ was initialized.
%
% \begin{macro}[internal]{\ekvd@ifprimitive}
%    \begin{macrocode}
\protected\long\def\ekvd@ifprimitive#1%
  {%
    \begingroup
      \edef\ekvd@tmpa{\string #1}%
      \edef\ekvd@tmpb{\meaning#1}%
      \expandafter
    \endgroup
    \ifx\ekvd@tmpa\ekvd@tmpb
      \ekv@fi@firstoftwo
    \fi
    \@secondoftwo
  }
%    \end{macrocode}
% \end{macro}
%
% \begin{macro}[internal]
%   {\ekvd@long,\ekvd@prot,\ekvd@clear@prefixes,\ekvd@ifalso}
% \expkvd\ will use |\ekvd@long|, |\ekvd@prot|, and |\ekvd@ifalso| to store
% whether a key should be defined as |\long| or |\protected| or adds an action
% to an existing key, and we have to clear them for every new key. By default
% |long| and |protected| will just be empty, |ifalso| will be
% |\@secondoftwo|, and |ifnew| will just use its third argument.
%    \begin{macrocode}
\protected\def\ekvd@clear@prefixes
  {%
    \let\ekvd@long\ekv@empty
    \let\ekvd@prot\ekv@empty
    \let\ekvd@ifalso\@secondoftwo
    \long\def\ekvd@ifnew##1##2##3{##3}%
  }
\ekvd@clear@prefixes
%    \end{macrocode}
% \end{macro}
%
% \begin{macro}{\ekvdefinekeys}
% This is the one front-facing macro which provides the interface to define
% keys. It's using |\ekvparse| to handle the \kv\ list, the interpretation will
% be done by |\ekvd@noarg| and |\ekvd@|. The \set\ for which the keys should be
% defined is stored in |\ekvd@set|.
%    \begin{macrocode}
\protected\def\ekvdefinekeys#1%
  {%
    \def\ekvd@set{#1}%
    \ekvparse\ekvd@noarg\ekvd@arg
  }
%    \end{macrocode}
% \end{macro}
%
% \begin{macro}[internal]{\ekvd@noarg,\ekvd@arg,\ekvd@handle}
% |\ekvd@noarg| and |\ekvd@arg| store whether there was a value in the \kv\
% pair. |\ekvd@handle| has to test whether there is a space inside the key and
% if so calls the prefix grabbing routine, else we throw an error and ignore the
% key.
%    \begin{macrocode}
\protected\long\def\ekvd@noarg#1%
  {%
    \let\ekvd@ifnoarg\@firstoftwo
    \expandafter\ekvd@handle\detokenize{#1}\ekvd@stop{}%
  }
\protected\long\def\ekvd@arg#1%
  {%
    \let\ekvd@ifnoarg\@secondoftwo
    \expandafter\ekvd@handle\detokenize{#1}\ekvd@stop
  }
\protected\long\def\ekvd@handle#1\ekvd@stop#2%
  {%
    \ekvd@clear@prefixes
    \def\ekvd@cur{#1}%
    \ekvd@ifspace{#1}%
      {\ekvd@prefix\ekv@mark#1\ekv@stop{#2}}%
      \ekvd@err@missing@type
  }
%    \end{macrocode}
% \end{macro}
%
% \begin{macro}[internal]{\ekvd@prefix,\ekvd@prefix@}
% \expkvd\ separates prefixes into two groups, the first being prefixes in the
% \TeX\ sense (|long| and |protected|) which use |@p@| in their name, the other
% being key-types (|code|, |int|, \emph{etc.}) which use |@t@| instead.
% |\ekvd@prefix| splits at the first space and checks whether its a |@p@|
% or |@t@| type prefix. If it is neither throw an error and gobble the
% definition (the value).
%    \begin{macrocode}
\protected\def\ekvd@prefix#1 {\ekv@strip{#1}\ekvd@prefix@\ekv@mark}
\protected\def\ekvd@prefix@#1#2\ekv@stop
  {%
    \ekv@ifdefined{ekvd@t@#1}%
      {\ekv@strip{#2}{\csname ekvd@t@#1\endcsname}}%
      {%
        \ekv@ifdefined{ekvd@p@#1}%
          {\csname ekvd@p@#1\endcsname\ekvd@prefix@after@p{#2}}%
          {\ekvd@err@undefined@prefix{#1}\@gobble}%
      }%
  }
%    \end{macrocode}
% \end{macro}
%
% \begin{macro}[internal]{\ekvd@prefix@after@p}
% The |@p@| type prefixes are all just modifying a following |@t@| type, so they
% will need to search for another prefix. This is true for all of them, so we
% use a macro to handle this. It'll throw an error if there is no other prefix.
%    \begin{macrocode}
\protected\def\ekvd@prefix@after@p#1%
  {%
    \ekvd@ifspace{#1}%
      {\ekvd@prefix#1\ekv@stop}%
      {\ekvd@err@missing@type\@gobble}%
  }
%    \end{macrocode}
% \end{macro}
%
% \begin{macro}[internal]
%   {\ekvd@p@long,\ekvd@p@protected,\ekvd@p@protect,\ekvd@p@also,\ekvd@p@new}
% Define the |@p@| type prefixes, they all just store some information in a
% temporary macro.
%    \begin{macrocode}
\protected\def\ekvd@p@long{\let\ekvd@long\long}
\protected\def\ekvd@p@protected{\let\ekvd@prot\protected}
\let\ekvd@p@protect\ekvd@p@protected
\protected\def\ekvd@p@also{\let\ekvd@ifalso\@firstoftwo}
\protected\def\ekvd@p@new{\let\ekvd@ifnew\ekvd@assert@new}
%    \end{macrocode}
% \end{macro}
%
% \subsubsection{Key Types}
%
% \begin{macro}[internal]{\ekvd@type@set,\ekvd@t@set}
% The |set| type is quite straight forward, just define a |NoVal| key to call
% |\ekvchangeset|.
%    \begin{macrocode}
\protected\def\ekvd@type@set#1#2%
  {%
    \ekvd@assert@not@long
    \ekvd@assert@not@protected
    \ekvd@ifnew{NoVal}{#1}%
      {%
        \ekv@ifempty{#2}%
          {\ekvd@err@missing@definition}%
          {%
            \ekvd@ifalso
              {%
                \ekv@expargtwice{\ekvd@add@noval{#1}}%
                  {\ekvchangeset{#2}}%
                  \ekvd@assert@not@protected@also
              }%
              {\ekv@expargtwice{\ekvdefNoVal\ekvd@set{#1}}{\ekvchangeset{#2}}}%
          }%
      }%
  }
\protected\def\ekvd@t@set#1#2%
  {%
    \ekvd@ifnoarg
      {\ekvd@type@set{#1}{#1}}%
      {\ekvd@type@set{#1}{#2}}%
  }
%    \end{macrocode}
% \end{macro}
%
% \begin{macro}[internal]{\ekvd@type@noval,\ekvd@t@noval,\ekvd@t@enoval}
% Another pretty simple type, |noval| just needs to assert that there is a
% definition and that |long| wasn't specified.
% There are types where the difference in the variants is so small, that we
% define a common handler for them, those common handlers are named with
% |@type@|. |noval| and |enoval| are so similar that we can use such a |@type@|
% macro, even if we could've done |noval| in a slightly faster way without it.
%    \begin{macrocode}
\protected\long\def\ekvd@type@noval#1#2#3%
  {%
    \ekvd@ifnew{NoVal}{#2}%
      {%
        \ekvd@assert@arg
          {%
            \ekvd@assert@not@long
            \ekvd@prot#1\ekvd@tmp{#3}%
            \ekvd@ifalso
              {\ekv@exparg{\ekvd@add@noval{#2}}\ekvd@tmp{}}%
              {\ekvletNoVal\ekvd@set{#2}\ekvd@tmp}%
          }%
      }%
  }
\protected\def\ekvd@t@noval{\ekvd@type@noval\def}
\protected\def\ekvd@t@enoval{\ekvd@type@noval\edef}
%    \end{macrocode}
% \end{macro}
%
% \begin{macro}[internal]{\ekvd@type@code,\ekvd@t@code,\ekvd@t@ecode}
% |code| is simple as well, |ecode| has to use |\edef| on a temporary macro,
% since \expkv\ doesn't provide an |\ekvedef|.
%    \begin{macrocode}
\protected\long\def\ekvd@type@code#1#2#3%
  {%
    \ekvd@ifnew{}{#2}%
      {%
        \ekvd@assert@arg
          {%
            \ekvd@prot\ekvd@long#1\ekvd@tmp##1{#3}%
            \ekvd@ifalso
              {\ekv@exparg{\ekvd@add@val{#2}}{\ekvd@tmp{##1}}{}}%
              {\ekvlet\ekvd@set{#2}\ekvd@tmp}%
          }%
      }%
  }
\protected\def\ekvd@t@code{\ekvd@type@code\def}
\protected\def\ekvd@t@ecode{\ekvd@type@code\edef}
%    \end{macrocode}
% \end{macro}
%
% \begin{macro}[internal]
%   {
%     \ekvd@type@default,\ekvd@t@default,\ekvd@t@qdefault,\ekvd@t@odefault,
%     \ekvd@t@fdefault
%   }
% |\ekvd@type@default| asserts there was an argument, also the key for which one
% wants to set a default has to be already defined (this is not so important for
% |default|, but |qdefault| requires is). If everything is good, |\edef| a
% temporary macro that expands |\ekvd@set| and the |\csname| for the key, and in
% the case of |qdefault| does the first expansion step of the key-macro.
%    \begin{macrocode}
\protected\long\def\ekvd@type@default#1#2#3#4%
  {%
    \ekvd@assert@arg
      {%
        \ekvifdefined\ekvd@set{#3}%
          {%
            \ekvd@assert@not@new
            \ekvd@assert@not@long
            \ekvd@prot\edef\ekvd@tmp
              {%
                \ekv@unexpanded\expandafter#1%
                  {#2\csname\ekv@name\ekvd@set{#3}\endcsname{#4}}%
              }%
            \ekvd@ifalso
              {\ekv@exparg{\ekvd@add@noval{#3}}\ekvd@tmp{}}%
              {\ekvletNoVal\ekvd@set{#3}\ekvd@tmp}%
          }%
          {\ekvd@err@undefined@key{#3}}%
      }%
  }
\protected\def\ekvd@t@default{\ekvd@type@default{}{}}
\protected\def\ekvd@t@qdefault{\ekvd@type@default{\expandafter\expandafter}{}}
\let\ekvd@t@odefault\ekvd@t@qdefault
\protected\def\ekvd@t@fdefault{\ekvd@type@default{}{\romannumeral`\^^@}}
%    \end{macrocode}
% \end{macro}
%
% \begin{macro}[internal]{\ekvd@t@edefault}
% |edefault| is too different from |default| and |qdefault| to reuse the
% |@type@| macro, as it doesn't need |\unexpanded| inside of |\edef|.
%    \begin{macrocode}
\protected\long\def\ekvd@t@edefault#1#2%
  {%
    \ekvd@assert@arg
      {%
        \ekvifdefined\ekvd@set{#1}%
          {%
            \ekvd@assert@not@new
            \ekvd@assert@not@long
            \ekvd@prot\edef\ekvd@tmp
              {\csname\ekv@name\ekvd@set{#1}\endcsname{#2}}%
            \ekvd@ifalso
              {\ekv@exparg{\ekvd@add@noval{#1}}\ekvd@tmp{}}%
              {\ekvletNoVal\ekvd@set{#1}\ekvd@tmp}%
          }%
          {\ekvd@err@undefined@key{#1}}%
      }%
  }
%    \end{macrocode}
% \end{macro}
%
% \begin{macro}[internal]
%   {\ekvd@t@initial,\ekvd@t@oinitial,\ekvd@t@finitial,\ekvd@t@einitial}
%    \begin{macrocode}
\long\def\ekvd@type@initial#1#2#3#4%
  {%
    \ekvd@assert@not@new
    \ekvd@assert@not@also
    \ekvd@assert@not@long
    \ekvd@assert@not@protected
    \ekvd@ifnoarg
      {%
        \ekvifdefinedNoVal\ekvd@set{#3}%
          {\csname\ekv@name\ekvd@set{#3}N\endcsname}%
          {\ekvd@err@undefined@noval{#3}}%
      }%
      {%
        \ekvifdefined\ekvd@set{#3}%
          {%
            #1{#2#4}%
            \csname\ekv@name\ekvd@set{#3}\expandafter\endcsname\expandafter
              {\ekvd@tmp}%
          }%
          {\ekvd@err@undefined@key{#3}}%
      }%
  }
\def\ekvd@t@initial{\ekvd@type@initial{\def\ekvd@tmp}{}}
\def\ekvd@t@oinitial{\ekvd@type@initial{\ekv@exparg{\def\ekvd@tmp}}{}}
\def\ekvd@t@einitial{\ekvd@type@initial{\edef\ekvd@tmp}{}}
\def\ekvd@t@finitial
  {\ekvd@type@initial{\ekv@exparg{\def\ekvd@tmp}}{\romannumeral`\^^@}}
%    \end{macrocode}
% \end{macro}
%
% \begin{macro}[internal]
%   {
%     \ekvd@type@bool,\ekvd@t@bool,\ekvd@t@gbool,\ekvd@t@boolTF,\ekvd@t@gboolTF,
%     \ekvd@t@invbool,\ekvd@t@ginvbool,\ekvd@t@invboolTF,\ekvd@t@ginvboolTF,
%   }
% The boolean types are a quicker version of a |choice| that accept |true| and
% |false|, and set up the |NoVal| action to be identical to \texttt{\key=true}.
% The |true| and |false| actions are always just |\let|ting the macro in |#7| to
% some other macro (\emph{e.g.}, \cs[no-index]{iftrue}).
%    \begin{macrocode}
\protected\def\ekvd@type@bool#1#2#3#4#5%
  {%
    \ekvd@ifnew{}{#4}%
      {%
        \ekvd@ifnew{NoVal}{#4}%
          {%
            \ekvd@assert@filledarg{#5}%
              {%
                \ekvd@newlet#5#3%
                \ekvd@type@choice{#4}%
                \protected\ekvdefNoVal\ekvd@set{#4}{#1\let#5#2}%
                \protected\expandafter\def
                  \csname\ekvd@choice@name\ekvd@set{#4}{true}\endcsname
                  {#1\let#5#2}%
                \protected\expandafter\def
                  \csname\ekvd@choice@name\ekvd@set{#4}{false}\endcsname
                  {#1\let#5#3}%
              }%
          }%
      }%
  }
\protected\def\ekvd@t@bool{\ekvd@type@bool{}\iftrue\iffalse}
\protected\def\ekvd@t@gbool{\ekvd@type@bool\global\iftrue\iffalse}
\protected\def\ekvd@t@boolTF{\ekvd@type@bool{}\@firstoftwo\@secondoftwo}
\protected\def\ekvd@t@gboolTF{\ekvd@type@bool\global\@firstoftwo\@secondoftwo}
\protected\def\ekvd@t@invbool{\ekvd@type@bool{}\iffalse\iftrue}
\protected\def\ekvd@t@ginvbool{\ekvd@type@bool\global\iffalse\iftrue}
\protected\def\ekvd@t@invboolTF{\ekvd@type@bool{}\@secondoftwo\@firstoftwo}
\protected\def\ekvd@t@ginvboolTF
  {\ekvd@type@bool\global\@secondoftwo\@firstoftwo}
%    \end{macrocode}
% \end{macro}
%
% \begin{macro}[internal]
%   {
%     \ekvd@type@boolpair,
%     \ekvd@t@boolpair,\ekvd@t@gboolpair,\ekvd@t@boolpairTF,\ekvd@t@gboolpairTF
%   }
% The boolean pair types are essentially the same as the boolean types, but set
% two macros instead of one.
%    \begin{macrocode}
\protected\def\ekvd@type@boolpair#1#2#3#4#5#6%
  {%
    \ekvd@ifnew{}{#4}%
      {%
        \ekvd@ifnew{NoVal}{#4}%
          {%
            \ekvd@newlet#5#3%
            \ekvd@newlet#6#2%
            \ekvd@type@choice{#4}%
            \protected\ekvdefNoVal\ekvd@set{#4}{#1\let#5#2#1\let#6#3}%
            \protected\expandafter\def
              \csname\ekvd@choice@name\ekvd@set{#4}{true}\endcsname
              {#1\let#5#2#1\let#6#3}%
            \protected\expandafter\def
              \csname\ekvd@choice@name\ekvd@set{#4}{false}\endcsname
              {#1\let#5#3#1\let#6#2}%
          }%
      }%
  }
\protected\def\ekvd@t@boolpair#1#2%
  {\ekvd@assert@twoargs{#2}{\ekvd@type@boolpair{}\iftrue\iffalse{#1}#2}}
\protected\def\ekvd@t@gboolpair#1#2%
  {\ekvd@assert@twoargs{#2}{\ekvd@type@boolpair\global\iftrue\iffalse{#1}#2}}
\protected\def\ekvd@t@boolpairTF#1#2%
  {%
    \ekvd@assert@twoargs{#2}%
      {\ekvd@type@boolpair{}\@firstoftwo\@secondoftwo{#1}#2}%
  }
\protected\def\ekvd@t@gboolpairTF#1#2%
  {%
    \ekvd@assert@twoargs{#2}%
      {\ekvd@type@boolpair\global\@firstoftwo\@secondoftwo{#1}#2}%
  }
%    \end{macrocode}
% \end{macro}
%
% \begin{macro}[internal]
%   {
%     \ekvd@type@data,\ekvd@t@data,\ekvd@t@gdata,\ekvd@t@dataT,\ekvd@t@gdataT
%   }
%    \begin{macrocode}
\protected\def\ekvd@type@data#1#2#3#4#5#6%
  {%
    \ekvd@ifnew{}{#5}%
      {%
        \ekvd@assert@filledarg{#6}%
          {%
            \ekvd@newlet#6#1%
            \ekvd@ifalso
              {%
                \let\ekvd@prot\protected
                \ekvd@add@val{#5}{\long#2#6####1#3{####1{#4}}}{}%
              }%
              {%
                \protected\ekvd@long\ekvdef\ekvd@set{#5}%
                  {\long#2#6####1#3{####1{#4}}}%
              }%
          }%
      }%
  }
\protected\def\ekvd@t@data
  {\ekvd@type@data\@secondoftwo\edef{####2}{\ekv@unexpanded{##1}}}
\protected\def\ekvd@t@edata{\ekvd@type@data\@secondoftwo\edef{####2}{##1}}
\protected\def\ekvd@t@gdata
  {\ekvd@type@data\@secondoftwo\xdef{####2}{\ekv@unexpanded{##1}}}
\protected\def\ekvd@t@xdata{\ekvd@type@data\@secondoftwo\xdef{####2}{##1}}
\protected\def\ekvd@t@dataT
  {\ekvd@type@data\@gobble\edef{}{\ekv@unexpanded{##1}}}
\protected\def\ekvd@t@edataT{\ekvd@type@data\@gobble\edef{}{##1}}
\protected\def\ekvd@t@gdataT
  {\ekvd@type@data\@gobble\xdef{}{\ekv@unexpanded{##1}}}
\protected\def\ekvd@t@xdataT{\ekvd@type@data\@gobble\xdef{}{##1}}
%    \end{macrocode}
% \end{macro}
%
% \begin{macro}[internal]{\ekvd@type@box,\ekvd@t@box,\ekvd@t@gbox}
% Set up our boxes. Though we're a generic package we want to be colour safe, so
% we put an additional grouping level inside the box contents, for the case that
% someone uses \pkg{color}. |\ekvd@newreg| is a small wrapper which tests
% whether the first argument is defined and if not does
% |\csname new#2\endcsname#1|.
%    \begin{macrocode}
\protected\def\ekvd@type@box#1#2#3%
  {%
    \ekvd@ifnew{}{#2}%
      {%
        \ekvd@assert@filledarg{#3}%
          {%
            \ekvd@newreg#3{box}%
            \ekvd@ifalso
              {%
                \let\ekvd@prot\protected
                \ekvd@add@val{#2}{#1\setbox#3=\hbox{\begingroup##1\endgroup}}{}%
              }%
              {%
                \protected\ekvd@long\ekvdef\ekvd@set{#2}%
                  {#1\setbox#3=\hbox{\begingroup##1\endgroup}}%
              }%
          }%
      }%
  }
\protected\def\ekvd@t@box{\ekvd@type@box{}}
\protected\def\ekvd@t@gbox{\ekvd@type@box\global}
%    \end{macrocode}
% \end{macro}
%
% \begin{macro}[internal]{\ekvd@type@toks,\ekvd@t@toks,\ekvd@t@gtoks}
% Similar to |box|, but set the |toks|.
%    \begin{macrocode}
\protected\def\ekvd@type@toks#1#2#3%
  {%
    \ekvd@ifnew{}{#2}%
      {%
        \ekvd@assert@filledarg{#3}%
          {%
            \ekvd@newreg#3{toks}%
            \ekvd@ifalso
              {%
                \let\ekvd@prot\protected
                \ekvd@add@val{#2}{#1#3={##1}}{}%
              }%
              {\protected\ekvd@long\ekvdef\ekvd@set{#2}{#1#3={##1}}}%
          }%
      }%
  }
\protected\def\ekvd@t@toks{\ekvd@type@toks{}}
\protected\def\ekvd@t@gtoks{\ekvd@type@toks\global}
%    \end{macrocode}
% \end{macro}
%
% \begin{macro}[internal]{\ekvd@type@preapptoks,\ekvd@t@apptoks,\ekvd@t@gapptoks}
% Just like |toks|, but expand the current contents of the |toks| register to
% append the new contents.
%    \begin{macrocode}
\ekvd@ifprimitive\toksapp
  {%
    \protected\def\ekvd@type@preapptoks#1#2#3%
      {%
        \ekvd@ifnew{}{#2}%
          {%
            \ekvd@assert@filledarg{#3}%
              {%
                \ekvd@newreg#3{toks}%
                \ekvd@ifalso
                  {%
                    \let\ekvd@prot\protected
                    \ekvd@add@val{#2}{#1#3{##1}}{}%
                  }%
                  {\protected\ekvd@long\ekvdef\ekvd@set{#2}{#1#3{##1}}}%
              }%
          }%
      }
    \protected\def\ekvd@t@apptoks{\ekvd@type@preapptoks\toksapp}
    \protected\def\ekvd@t@gapptoks{\ekvd@type@preapptoks\gtoksapp}
    \protected\def\ekvd@t@pretoks{\ekvd@type@preapptoks\tokspre}
    \protected\def\ekvd@t@gpretoks{\ekvd@type@preapptoks\gtokspre}
  }
  {%
    \protected\def\ekvd@type@apptoks#1#2#3%
      {%
        \ekvd@ifnew{}{#2}%
          {%
            \ekvd@assert@filledarg{#3}%
              {%
                \ekvd@newreg#3{toks}%
                \ekvd@ifalso
                  {%
                    \let\ekvd@prot\protected
                    \ekvd@add@val{#2}{#1#3=\expandafter{\the#3##1}}{}%
                  }%
                  {%
                    \protected\ekvd@long\ekvdef\ekvd@set{#2}%
                      {#1#3=\expandafter{\the#3##1}}%
                  }%
              }%
          }%
      }
    \protected\def\ekvd@t@apptoks{\ekvd@type@apptoks{}}
    \protected\def\ekvd@t@gapptoks{\ekvd@type@apptoks\global}
    \newtoks\ekvd@toks
    \protected\def\ekvd@type@pretoks#1#2#3%
      {%
        \ekvd@ifnew{}{#2}%
          {%
            \ekvd@assert@filledarg{#3}%
              {%
                \ekvd@newreg#3{toks}%
                \ekvd@ifalso
                  {%
                    \let\ekvd@prot\protected
                    \ekvd@add@val{#2}%
                      {%
                        \ekvd@toks={##1}%
                        #1#3=\expandafter{\the\expandafter\ekvd@toks\the#3}%
                      }%
                      {}%
                  }%
                  {%
                    \protected\ekvd@long\ekvdef\ekvd@set{#2}%
                      {%
                        \ekvd@toks={##1}%
                        #1#3=\expandafter{\the\expandafter\ekvd@toks\the#3}%
                      }%
                  }%
              }%
          }%
      }
    \protected\def\ekvd@t@pretoks{\ekvd@type@pretoks{}}
    \protected\def\ekvd@t@gpretoks{\ekvd@type@pretoks\global}
  }
%    \end{macrocode}
% \end{macro}
%
% \begin{macro}[internal]
%   {
%     \ekvd@type@reg,\ekvd@t@int,\ekvd@t@eint,\ekvd@t@gint,\ekvd@t@xint,
%     \ekvd@t@dimen,\ekvd@t@edimen,\ekvd@t@gdimen,\ekvd@t@xdimen,
%     \ekvd@t@skip,\ekvd@t@eskip,\ekvd@t@gskip,\ekvd@t@xskip
%   }
% The |\ekvd@type@reg| can handle all the types for which the assignment will
% just be \texttt{\meta{register}=\meta{value}}.
%    \begin{macrocode}
\protected\def\ekvd@type@reg#1#2#3#4#5#6%
  {%
    \ekvd@ifnew{}{#5}%
      {%
        \ekvd@assert@filledarg{#6}%
          {%
            \ekvd@newreg#6{#1}%
            \ekvd@ifalso
              {%
                \let\evkd@prot\protected
                \ekvd@add@val{#5}{#2#6=#3##1#4\relax}{}%
              }%
              {\protected\ekvd@long\ekvdef\ekvd@set{#5}{#2#6=#3##1#4\relax}}%
          }%
      }%
  }
\protected\def\ekvd@t@int{\ekvd@type@reg{count}{}{}{}}
\protected\def\ekvd@t@eint{\ekvd@type@reg{count}{}\numexpr\relax}
\protected\def\ekvd@t@gint{\ekvd@type@reg{count}\global{}{}}
\protected\def\ekvd@t@xint{\ekvd@type@reg{count}\global\numexpr\relax}
\protected\def\ekvd@t@dimen{\ekvd@type@reg{dimen}{}{}{}}
\protected\def\ekvd@t@edimen{\ekvd@type@reg{dimen}{}\dimexpr\relax}
\protected\def\ekvd@t@gdimen{\ekvd@type@reg{dimen}\global{}{}}
\protected\def\ekvd@t@xdimen{\ekvd@type@reg{dimen}\global\dimexpr\relax}
\protected\def\ekvd@t@skip{\ekvd@type@reg{skip}{}{}{}}
\protected\def\ekvd@t@eskip{\ekvd@type@reg{skip}{}\glueexpr\relax}
\protected\def\ekvd@t@gskip{\ekvd@type@reg{skip}\global{}{}}
\protected\def\ekvd@t@xskip{\ekvd@type@reg{skip}\global\glueexpr\relax}
%    \end{macrocode}
% \end{macro}
%
% \begin{macro}[internal]{\ekvd@type@store,\ekvd@t@store,\ekvd@t@gstore}
% The none-expanding |store| types use an |\edef| or |\xdef| and |\unexpanded|
% to be able to also store |#| easily.
%    \begin{macrocode}
\protected\def\ekvd@type@store#1#2#3#4%
  {%
    \ekvd@ifnew{}{#3}%
      {%
        \ekvd@assert@filledarg{#4}%
          {%
            \ekvd@newlet#4\ekv@empty
            \ekvd@ifalso
              {%
                \let\ekvd@prot\protected
                \ekvd@add@val{#3}{#1#4{#2}}{}%
              }%
              {\protected\ekvd@long\ekvdef\ekvd@set{#3}{#1#4{#2}}}%
          }%
      }%
  }
\protected\def\ekvd@t@store{\ekvd@type@store\edef{\ekv@unexpanded{##1}}}
\protected\def\ekvd@t@gstore{\ekvd@type@store\xdef{\ekv@unexpanded{##1}}}
\protected\def\ekvd@t@estore{\ekvd@type@store\edef{##1}}
\protected\def\ekvd@t@xstore{\ekvd@type@store\xdef{##1}}
%    \end{macrocode}
% \end{macro}
%
% \begin{macro}[internal]
%   {
%     \ekvd@type@meta,\ekvd@type@meta@a,\ekvd@type@meta@b,\ekvd@type@meta@c,
%     \ekvd@t@meta,\ekvd@t@nmeta
%   }
% |meta| sets up things such that another instance of |\ekvset| will be run on
% the argument, with the same \set.
%    \begin{macrocode}
\protected\long\def\ekvd@type@meta#1#2#3#4#5#6#7%
  {%
    \ekvd@ifnew{#1}{#6}%
      {%
        \ekvd@assert@filledarg{#7}%
          {%
            \edef\ekvd@tmp{\ekvd@set}%
            \expandafter\ekvd@type@meta@a\expandafter{\ekvd@tmp}{#7}{#2}%
            \ekvd@ifalso
              {\ekv@exparg{#3{#6}}{\ekvd@tmp#4}{#5}}%
              {\csname ekvlet#1\endcsname\ekvd@set{#6}\ekvd@tmp}%
          }%
      }%
  }
\protected\long\def\ekvd@type@meta@a#1#2%
  {%
    \expandafter\ekvd@type@meta@b\expandafter{\ekvset{#1}{#2}}%
  }
\protected\def\ekvd@type@meta@b
  {%
    \expandafter\ekvd@type@meta@c\expandafter
  }
\protected\long\def\ekvd@type@meta@c#1#2%
  {%
    \ekvd@prot\ekvd@long\def\ekvd@tmp#2{#1}%
  }
\protected\def\ekvd@t@meta{\ekvd@type@meta{}{##1}\ekvd@add@val{{##1}}{}}
\protected\def\ekvd@t@nmeta
  {%
    \ekvd@assert@not@long
    \ekvd@type@meta{NoVal}{}\ekvd@add@noval{}\ekvd@assert@not@long@also
  }
%    \end{macrocode}
% \end{macro}
%
% \begin{macro}[internal]
%   {\ekvd@type@smeta,\ekvd@type@smeta@,\ekvd@t@smeta,\ekvd@t@snmeta}
% |smeta| is pretty similar to |meta|, but needs two arguments inside of \val,
% such that the first is the \set\ for which the sub-|\ekvset| and the second is
% the \kv\ list.
%    \begin{macrocode}
\protected\long\def\ekvd@type@smeta#1#2#3#4#5#6#7%
  {%
    \ekvd@ifnew{#1}{#6}%
      {%
        \ekvd@assert@twoargs{#7}%
          {%
            \ekvd@type@meta@a#7{#2}%
            \ekvd@ifalso
              {\ekv@exparg{#3{#6}}{\ekvd@tmp#4}{#5}}%
              {\csname ekvlet#1\endcsname\ekvd@set{#6}\ekvd@tmp}%
          }%
      }%
  }
\protected\def\ekvd@t@smeta{\ekvd@type@smeta{}{##1}\ekvd@add@val{{##1}}{}}
\protected\def\ekvd@t@snmeta
  {%
    \ekvd@assert@not@long
    \ekvd@type@smeta{NoVal}{}\ekvd@add@noval{}\ekvd@assert@not@long@also
  }
%    \end{macrocode}
% \end{macro}
%
% \begin{macro}[internal]
%   {
%     \ekvd@type@choice,
%     \ekvd@populate@choice,\ekvd@populate@choice@,\ekvd@populate@choice@noarg,
%     \ekvd@choice@prefix,\ekvd@choice@prefix@,\ekvd@choice@prefix@done,
%     \ekvd@choice@p@protected,\ekvd@choice@p@protect,
%     \ekvd@choice@p@long,\ekvd@choice@p@long@,
%     \ekvd@t@choice
%   }
% The |choice| type is by far the most complex type, as we have to run a
% sub-parser on the choice-definition list, which should support the |@p@| type
% prefixes as well (but |long| will always throw an error, as they are not
% allowed to be long). |\ekvd@type@choice| will just define the choice-key, the
% handling of the choices definition will be done by |\ekvd@populate@choice|.
%    \begin{macrocode}
\protected\def\ekvd@type@choice#1%
  {%
    \ekvd@assert@not@long
    \ekv@exparg{\ekvd@prot\def\ekvd@tmp##1}%
      {%
        \expandafter\ekvd@h@choice\expandafter
          {\ekvd@choice@name\ekvd@set{#1}{##1}}%
      }%
    \ekvd@ifalso
      {%
        \ekvd@assert@val{#1}%
          {%
            \ekvd@if@not@already@choice{#1}%
              {%
                \ekv@exparg
                  {%
                    \expandafter\ekvd@add@aux
                      \csname\ekv@name\ekvd@set{#1}\endcsname{{##1}}{#1}%
                  }%
                  {\ekvd@tmp{##1}}%
                  {\ekvd@long\ekvdef}\ekvd@assert@not@long@also
              }%
          }%
      }%
      {\ekvlet\ekvd@set{#1}\ekvd@tmp}%
  }
%    \end{macrocode}
% |\ekvd@populate@choice| just uses |\ekvparse| and then gives control to
% |\ekvd@populate@choice@noarg|, which throws an error, and
% |\ekvd@populate@choice@|.
%    \begin{macrocode}
\protected\def\ekvd@populate@choice
  {%
    \ekvparse\ekvd@populate@choice@noarg\ekvd@populate@choice@
  }
\protected\long\def\ekvd@populate@choice@noarg#1%
  {%
    \expandafter\ekvd@err@missing@definition@msg\expandafter{\ekvd@cur : #1}%
  }
%    \end{macrocode}
% |\ekvd@populate@choice@| runs the prefix-test, if there is none we can
% directly define the choice, for that |\ekvd@set@choice| will expand to the
% current choice-key's name, which will have been defined by |\ekvd@t@choice|.
% If there is a prefix run the prefix grabbing routine, which was altered for
% |@type@choice|.
%    \begin{macrocode}
\protected\long\def\ekvd@populate@choice@#1#2%
  {%
    \ekvd@clear@prefixes
    \ekvd@ifspace{#1}%
      {\ekvd@choice@prefix{\ekv@mark#1}\ekv@mark#1\ekv@stop}%
      {%
        \expandafter\edef
          \csname\ekvd@choice@name\ekvd@set\ekvd@set@choice{#1}\endcsname
      }%
      {\unexpanded{#2}}%
  }
\protected\def\ekvd@choice@prefix#1#2
  {%
    \ekv@strip{#2}{\ekvd@choice@prefix@{#1}}\ekv@mark
  }
\protected\def\ekvd@choice@prefix@#1#2#3\ekv@stop
  {%
    \ekv@ifdefined{ekvd@choice@p@#2}%
      {%
        \csname ekvd@choice@p@#2\endcsname
        \ekvd@ifspace{#3}%
          {\ekvd@choice@prefix{#3}#3\ekv@stop}%
          {\ekvd@choice@prefix@done{#3}}%
      }%
      {\ekvd@choice@prefix@done{#1}}%
  }
\protected\def\ekvd@choice@prefix@done#1%
  {%
    \ekvd@prot\expandafter\edef
      \csname
        \ekv@strip{#1}{\ekvd@choice@name\ekvd@set\ekvd@set@choice}%
      \endcsname
  }
\protected\def\ekvd@choice@p@protected{\let\ekvd@prot\protected}
\let\ekvd@choice@p@protect\ekvd@choice@p@protected
\protected\def\ekvd@choice@invalid@p#1\ekvd@ifspace#2%
  {%
    \expandafter\ekvd@choice@invalid@p@\expandafter{\ekv@gobble@mark#2}{#1}%
    \ekvd@ifspace{#2}%
  }
\protected\def\ekvd@choice@invalid@p@#1#2%
  {%
    \expandafter\ekvd@err@no@prefix@msg\expandafter{\ekvd@cur : #2 #1}{#2}%
  }
\protected\def\ekvd@choice@p@long{\ekvd@choice@invalid@p{long}}%
\protected\def\ekvd@choice@p@also{\ekvd@choice@invalid@p{also}}%
\protected\def\ekvd@choice@p@new{\ekvd@choice@invalid@p{new}}%
%    \end{macrocode}
% Finally we're able to set up the |@t@choice| macro, which has to store the
% current choice-key's name, define the key, and parse the available choices.
%    \begin{macrocode}
\protected\long\def\ekvd@t@choice#1#2%
  {%
    \ekvd@ifnew{}{#1}%
      {%
        \ekvd@assert@arg
          {%
            \ekvd@type@choice{#1}%
            \def\ekvd@set@choice{#1}%
            \ekvd@populate@choice{#2}%
          }%
      }%
  }
%    \end{macrocode}
% \end{macro}
%
% \begin{macro}[internal]{\ekvd@t@choice-store,\ekvd@t@choice-enum}
%   These two types define a special kind of |choice| key and are quite similar,
%   the only difference is what the different choices do (hence they use a
%   shared initialisation which differs in the chosen |populate| step).
%    \begin{macrocode}
\protected\long\expandafter\def\csname ekvd@t@choice-store\endcsname
  {\ekvd@type@choicespecial\ekvd@populate@choicestore}
\protected\long\expandafter\def\csname ekvd@t@choice-enum\endcsname
  {\ekvd@type@choicespecial\ekvd@populate@choiceenum}
%    \end{macrocode}
% \begin{macro}[internal]{\ekvd@type@choicespecial}
%   Initialise similar to a |choice| key. The difference is that we require two
%   arguments (which we assert), a macro to store things in, and a |csv|-list
%   containing the allowed values. |#1| is the |populate| macro according to the
%   type used.
%    \begin{macrocode}
\protected\long\def\ekvd@type@choicespecial#1#2#3%
  {%
    \ekvd@ifnew{}{#2}%
      {%
        \ekvd@assert@twoargs{#3}%
          {%
            \ekvd@type@choice{#2}%
            \def\ekvd@set@choice{#2}%
            #1#3%
          }%
      }%
  }
%    \end{macrocode}
% \end{macro}
% \begin{macro}[internal]
%   {\ekvd@populate@choicestore,\ekvd@populate@choicestore@}
%   We initialise the storing macro if it doesn't yet exist, and then we loop
%   over the value list. The |\edef|s with |\unexpanded| are both necessary to
%   be able to store macro parameter tokens (the outer protects at define time,
%   the inner at use time).
%    \begin{macrocode}
\protected\long\def\ekvd@populate@choicestore#1%
  {%
    \ekvd@newlet#1\ekv@empty
    \ekvcsvloop{\ekvd@populate@choicestore@#1}%
  }
\protected\long\def\ekvd@populate@choicestore@#1#2%
  {%
    \protected\expandafter\edef
      \csname\ekvd@choice@name\ekvd@set\ekvd@set@choice{#2}\endcsname
      {\unexpanded{\edef#1{\unexpanded{#2}}}}%
  }
%    \end{macrocode}
% \end{macro}
% \begin{macro}[internal]
%   {\ekvd@populate@choiceenum,\ekvd@populate@choiceenum@}
%   This is similar to the population of a |choice-store| type, but instead of
%   storing the values in a macro this initialises a count and stores the
%   position of the value in the list inside that count (zero-indexed). The
%   space is necessary to terminate the number scanning, which is the reason we
%   use |\@firstofone| (so that the space after the macro name isn't gobbled by
%   \TeX).
%    \begin{macrocode}
\protected\long\def\ekvd@populate@choiceenum#1%
  {%
    \ekvd@newreg#1{count}%
    \def\ekvd@tmp{0}%
    \ekvcsvloop{\ekvd@populate@choiceenum@#1}%
  }
\protected\long\def\ekvd@populate@choiceenum@#1#2%
  {%
    \protected\expandafter\edef
      \csname\ekvd@choice@name\ekvd@set\ekvd@set@choice{#2}\endcsname
      {#1=\@firstofone{\ekvd@tmp} }%
    \edef\ekvd@tmp{\the\numexpr\ekvd@tmp+1\relax}%
  }
%    \end{macrocode}
% \end{macro}
% \end{macro}
%
% \begin{macro}[internal]{\ekvd@t@unknown-choice}
%    \begin{macrocode}
\protected\long\expandafter\def\csname ekvd@t@unknown-choice\endcsname#1#2%
  {%
    \ekvd@assert@new@for@name{\ekvd@unknown@choice@name\ekvd@set{#1}}%
      {%
        \ekvd@assert@arg
          {%
            \ekvd@assert@not@long
            \ekvd@assert@not@also
            \ekvd@prot\expandafter
            \def\csname\ekvd@unknown@choice@name\ekvd@set{#1}\endcsname##1{#2}%
          }%
      }%
  }
%    \end{macrocode}
% \end{macro}
%
% \begin{macro}[internal]
%   {\ekvd@t@unknown,\ekvd@type@unknown@code,\ekvd@type@unknown@noval}
%   The |unknown| type has different subtypes which would be the key names for
%   other types. It is first checked whether that subtype is defined, if it
%   isn't throw an error, else use that subtype.
%    \begin{macrocode}
\protected\long\def\ekvd@t@unknown#1#2%
  {%
    \ekv@ifdefined{ekvd@type@unknown@\detokenize{#1}}%
      {\csname ekvd@type@unknown@\detokenize{#1}\endcsname{#2}}%
      \ekvd@err@misused@unknown
  }
%    \end{macrocode}
%   The |unknown noval| type can use |\ekvdefunknownNoVal| directly (after
%   asserting some prefixes).
%    \begin{macrocode}
\protected\long\def\ekvd@type@unknown@noval#1%
  {%
    \ekvd@assert@new@for@name{\ekv@name\ekvd@set{}uN}%
      {%
        \ekvd@assert@arg
          {%
            \ekvd@assert@not@also
            \ekvd@assert@not@long
            \ekvd@prot\ekvdefunknownNoVal\ekvd@set{#1}%
          }%
      }%
  }
%    \end{macrocode}
%   The |unknown code| type uses some trickery during the definition in order to
%   swap out |#1| and |#2| in the user supplied definition. This is done via a
%   temporary macro that stores the definition but gets the parameter numbers
%   reversed while the real definition is done.
%    \begin{macrocode}
\protected\long\def\ekvd@type@unknown@code#1%
  {%
    \ekvd@assert@new@for@name{\ekv@name\ekvd@set{}u}%
      {%
        \ekvd@assert@arg
          {%
            \ekvd@assert@not@also
            \begingroup
              \def\ekvd@tmp##1##2{#1}%
              \ekv@exparg
              {%
                \endgroup
                \ekvd@prot\ekvd@long\ekvdefunknown\ekvd@set
              }%
              {\ekvd@tmp{##2}{##1}}%
          }%
      }%
  }
%    \end{macrocode}
% \end{macro}
%
% \begin{macro}[internal]
%   {
%     \ekvd@type@unknown@redirect,
%     \ekvd@type@unknown@redirect-code,
%     \ekvd@type@unknown@redirect-noval
%   }
%   The |unknown redirect| types also just forward to |\ekvredirectunknown|
%   after asserting some prefixes.
%    \begin{macrocode}
\protected\edef\ekvd@type@unknown@redirect#1%
  {%
    \expandafter\noexpand\csname ekvd@type@unknown@redirect-code\endcsname{#1}%
    \expandafter\noexpand\csname ekvd@type@unknown@redirect-noval\endcsname{#1}%
  }
\protected\expandafter\def\csname ekvd@type@unknown@redirect-code\endcsname#1%
  {%
    \ekvd@assert@new@for@name{\ekv@name\ekvd@set{}u}%
      {%
        \ekvd@assert@arg
          {%
            \ekvd@assert@not@also
            \ekvd@assert@not@protected
            \expandafter\ekvredirectunknown\expandafter{\ekvd@set}{#1}%
          }%
      }%
  }
\protected\expandafter\def\csname ekvd@type@unknown@redirect-noval\endcsname#1%
  {%
    \ekvd@assert@new@for@name{\ekv@name\ekvd@set{}uN}%
      {%
        \ekvd@assert@arg
          {%
            \ekvd@assert@not@also
            \ekvd@assert@not@protected
            \ekvd@assert@not@long
            \expandafter\ekvredirectunknownNoVal\expandafter{\ekvd@set}{#1}%
          }%
      }%
  }
%    \end{macrocode}
% \end{macro}
%
% \subsubsection{Key Type Helpers}
% There are some keys that might need helpers during their execution (not during
% their definition, which are gathered as |@type@| macros). These helpers are
% named |@h@|.
%
% \begin{macro}[internal]{\ekvd@h@choice,\ekvd@h@choice@}
% The |choice| helper will just test whether the given choice was defined, if
% not throw an error expandably, else call the macro which stores the code for
% this choice.
%    \begin{macrocode}
\def\ekvd@h@choice#1%
  {%
    \expandafter\ekvd@h@choice@
      \csname\ifcsname#1\endcsname#1\else relax\fi\endcsname
      {#1}%
  }
\def\ekvd@h@choice@#1#2%
  {%
    \ifx#1\relax
      \ekvd@err@choice@invalid{#2}%
      \expandafter\@gobble
    \fi
    #1%
  }
%    \end{macrocode}
% \end{macro}
%
%
% \subsubsection{Handling \texttt{also}}
%
% \begin{macro}[internal]
%   {\ekvd@add@val,\ekvd@add@noval,\ekvd@add@aux,\ekvd@add@aux@}
%    \begin{macrocode}
\protected\long\def\ekvd@add@val#1#2#3%
  {%
    \ekvd@assert@val{#1}%
      {%
        \expandafter\ekvd@add@aux\csname\ekv@name\ekvd@set{#1}\endcsname{{##1}}%
          {#1}{#2}{\ekvd@long\ekvdef}{#3}%
      }%
  }
\protected\long\def\ekvd@add@noval#1#2#3%
  {%
    \ekvd@assert@noval{#1}%
      {%
        \expandafter\ekvd@add@aux\csname\ekv@name\ekvd@set{#1}N\endcsname{}%
          {#1}{#2}\ekvdefNoVal{#3}%
      }%
  }
\protected\long\def\ekvd@add@aux#1#2%
  {%
    \ekvd@extract@prefixes#1%
    \expandafter\ekvd@add@aux@\expandafter{#1#2}%
  }
\protected\long\def\ekvd@add@aux@#1#2#3#4#5%
  {%
    #5%
    \ekvd@prot#4\ekvd@set{#2}{#1#3}%
  }
%    \end{macrocode}
% \end{macro}
%
% \begin{macro}[internal]
%   {
%     \ekvd@extract@prefixes,\ekvd@extract@prefixes@,
%     \ekvd@extract@prefixes@long,\ekvd@extract@prefixes@prot
%   }
% This macro checks which prefixes were used for the definition of a macro and
% sets |\ekvd@long| and |\ekvd@prot| accordingly.
%    \begin{macrocode}
\protected\def\ekvd@extract@prefixes#1%
  {%
    \expandafter\ekvd@extract@prefixes@\meaning#1\ekvd@stop
  }
%    \end{macrocode}
% In the following definition |#1| will get replaced by |macro:|, |#2| by
% |\long| and |#3| by |\protected| (in each, all tokens will have category
% other). This allows us to parse the |\meaning| of a macro for those strings.
%    \begin{macrocode}
\protected\def\ekvd@extract@prefixes@#1#2#3%
  {%
    \protected\def\ekvd@extract@prefixes@##1#1##2\ekvd@stop
      {%
        \ekvd@extract@prefixes@long
          ##1\ekvd@mark\@firstofone#2\ekvd@mark\@gobble\ekvd@stop
          {\let\ekvd@long\long}%
        \ekvd@extract@prefixes@prot
          ##1\ekvd@mark\@firstofone#3\ekvd@mark\@gobble\ekvd@stop
          {\let\ekvd@prot\protected}%
      }%
    \protected\def\ekvd@extract@prefixes@long##1#2##2\ekvd@mark##3##4\ekvd@stop
      {##3}%
    \protected\def\ekvd@extract@prefixes@prot##1#3##2\ekvd@mark##3##4\ekvd@stop
      {##3}%
  }
%    \end{macrocode}
% We use a temporary macro to expand the three arguments of
% |\ekvd@extract@prefixes@|, which will set up the real meaning of itself and
% the parsing for |\long| and |\protected|.
%    \begin{macrocode}
\begingroup
\edef\ekvd@tmp
  {%
    \endgroup
    \ekvd@extract@prefixes@
      {\detokenize{macro:}}%
      {\string\long}%
      {\string\protected}%
  }
\ekvd@tmp
%    \end{macrocode}
% \end{macro}
%
%
% \subsubsection{Tests}
%
% \begin{macro}[internal]{\ekvd@newlet,\ekvd@newreg}
% These macros test whether a control sequence is defined, if it isn't they
% define it, either via |\let| or via the correct \cs[no-index]{new\meta{reg}}.
%    \begin{macrocode}
\protected\def\ekvd@newlet#1#2%
  {%
    \ifdefined#1\ekv@fi@gobble\fi\@firstofone{\let#1#2}%
  }
\protected\def\ekvd@newreg#1#2%
  {%
    \ifdefined#1\ekv@fi@gobble\fi\@firstofone{\csname new#2\endcsname#1}%
  }
%    \end{macrocode}
% \end{macro}
%
% \begin{macro}[internal]
%   {\ekvd@assert@twoargs,\ekvd@ifnottwoargs,\ekvd@ifempty@gtwo}
% A test for exactly two tokens can be reduced for an empty-test after gobbling
% two tokens, in the case that there are fewer tokens than two in the argument,
% only macros will be gobbled that are needed for the true branch, which doesn't
% hurt, and if there are more this will not be empty.
%    \begin{macrocode}
\long\def\ekvd@assert@twoargs#1%
  {%
    \ekvd@ifnottwoargs{#1}{\ekvd@err@missing@definition}%
  }
\long\def\ekvd@ifnottwoargs#1%
  {%
    \ekvd@ifempty@gtwo#1\ekv@ifempty@B
      \ekv@ifempty@false\ekv@ifempty@A\ekv@ifempty@B\@firstoftwo
  }
\long\def\ekvd@ifempty@gtwo#1#2{\ekv@ifempty@\ekv@ifempty@A}
%    \end{macrocode}
% \end{macro}
%
% \begin{macro}[internal]
%   {
%     \ekvd@assert@val,\ekvd@assert@val@,\ekvd@assert@noval,\ekvd@assert@noval@,
%     \ekvd@extract@args,\ekvd@extracted@args,\ekvd@one@arg@string
%   }
% Assert that a given key is defined as a value taking key or a |NoVal| key with
% the correct argument structure, respectively.
%    \begin{macrocode}
\protected\def\ekvd@assert@val#1%
  {%
    \ekvifdefined\ekvd@set{#1}%
      {\expandafter\ekvd@assert@val@\csname\ekv@name\ekvd@set{#1}\endcsname}%
      {%
        \ekvifdefinedNoVal\ekvd@set{#1}%
          \ekvd@err@add@val@on@noval
          {\ekvd@err@undefined@key{#1}}%
        \@gobble
      }%
  }
\protected\def\ekvd@assert@val@#1%
  {%
    \expandafter\ekvd@extract@args\meaning#1\ekvd@stop
    \unless\ifx\ekvd@extracted@args\ekvd@one@arg@string
      \ekvd@err@unsupported@arg
    \fi
    \@firstofone
  }%
\protected\def\ekvd@assert@noval#1%
  {%
    \ekvifdefinedNoVal\ekvd@set{#1}%
      {\expandafter\ekvd@assert@noval@\csname\ekv@name\ekvd@set{#1}N\endcsname}%
      {%
        \ekvifdefined\ekvd@set{#1}%
          \ekvd@err@add@noval@on@val
          {\ekvd@err@undefined@key{#1}}%
        \@gobble
      }%
  }
\protected\def\ekvd@assert@noval@#1%
  {%
    \expandafter\ekvd@extract@args\meaning#1\ekvd@stop
    \unless\ifx\ekvd@extracted@args\ekv@empty
      \ekvd@err@unsupported@arg
    \fi
    \@firstofone
  }
\protected\def\ekvd@extract@args#1%
  {%
    \protected\def\ekvd@extract@args##1#1##2->##3\ekvd@stop
      {\def\ekvd@extracted@args{##2}}%
  }
\expandafter\ekvd@extract@args\expandafter{\detokenize{macro:}}
\edef\ekvd@one@arg@string{\string#1}
%    \end{macrocode}
% \end{macro}
%
% \begin{macro}[internal]{\ekvd@assert@arg,\ekvd@assert@arg@msg,\ekvd@ifnoarg}
% There is no need to actually define |\ekvd@ifnoarg| here, as it will be set by
% either |\ekvd@arg| or |\ekvd@noarg|.
%    \begin{macrocode}
\def\ekvd@assert@arg{\ekvd@ifnoarg\ekvd@err@missing@definition}
\long\def\ekvd@assert@arg@msg#1%
  {%
    \ekvd@ifnoarg{\ekvd@err@missing@definition@msg{#1}}%
  }
%    \end{macrocode}
% \end{macro}
%
% \begin{macro}[internal]{\ekvd@assert@filledarg,\ekvd@ifnoarg@or@empty}
%    \begin{macrocode}
\long\def\ekvd@assert@filledarg#1%
  {%
    \ekvd@ifnoarg@or@empty{#1}\ekvd@err@missing@definition
  }
\long\def\ekvd@ifnoarg@or@empty#1%
  {%
    \ekvd@ifnoarg
      \@firstoftwo
      {\ekv@ifempty{#1}}%
  }
%    \end{macrocode}
% \end{macro}
%
% \begin{macro}[internal]
%   {
%     \ekvd@assert@not@long,\ekvd@assert@not@protected,\ekvd@assert@not@also
%     \ekvd@assert@not@long@also,\ekvd@assert@not@protected@also,
%     \ekvd@assert@new,\ekvd@assert@not@new
%   }
% Some key-types don't want to be |also|, |\long| or |\protected|, so we provide
% macros to test this and throw an error, this could be silently ignored but now
% users will learn to not use unnecessary stuff which slows the compilation
% down.
%    \begin{macrocode}
\def\ekvd@assert@not@long{\ifx\ekvd@long\long\ekvd@err@no@prefix{long}\fi}
\def\ekvd@assert@not@protected
  {\ifx\ekvd@prot\protected\ekvd@err@no@prefix{protected}\fi}
\def\ekvd@assert@not@also{\ekvd@ifalso{\ekvd@err@no@prefix{also}}{}}
\def\ekvd@assert@not@long@also
  {\ifx\ekvd@long\long\ekvd@err@no@prefix@also{long}\fi}
\def\ekvd@assert@not@protected@also
  {\ifx\ekvd@prot\protected\ekvd@err@no@prefix@also{protected}\fi}
\def\ekvd@assert@new#1#2%
  {\csname ekvifdefined#1\endcsname\ekvd@set{#2}{\ekvd@err@not@new}}
\def\ekvd@assert@not@new
  {\ifx\ekvd@ifnew\ekvd@assert@new\ekvd@err@no@prefix{new}\fi}
\def\ekvd@assert@new@for@name#1%
  {%
    \ifx\ekvd@ifnew\ekvd@assert@new
      \ekv@fi@firstoftwo
    \fi
    \@secondoftwo
      {\ekv@ifdefined{#1}\ekvd@err@not@new}%
      \@firstofone
  }
%    \end{macrocode}
% \end{macro}
%
% \begin{macro}[internal]
%   {
%     \ekvd@if@not@already@choice, \ekvd@if@not@already@choice@a,
%     \ekvd@if@not@already@choice@b
%   }
% It is bad to use |also| on a key that already contains a |choice|, as both
% choices would share the same valid values and thus lead to each callback being
% used twice. The following is a rudimentary test against this.
%    \begin{macrocode}
\protected\def\ekvd@if@not@already@choice#1%
  {%
    \expandafter\ekvd@if@not@already@choice@a
      \csname\ekv@name\ekvd@set{#1}\endcsname
      {}\ekvd@h@choice\ekvd@stop
  }
\protected\def\ekvd@if@not@already@choice@a
  {%
    \expandafter\ekvd@if@not@already@choice@b
  }
\long\protected\def\ekvd@if@not@already@choice@b#1\ekvd@h@choice#2\ekvd@stop
  {%
    \ekv@ifempty{#2}\@firstofone\@gobble
  }
%    \end{macrocode}
% \end{macro}
%
% \begin{macro}[internal]{\ekvd@ifspace,\ekvd@ifspace@}
% Yet another test which can be reduced to an if-empty, this time by gobbling
% everything up to the first space.
%    \begin{macrocode}
\long\def\ekvd@ifspace#1%
  {%
    \ekvd@ifspace@#1 \ekv@ifempty@B
      \ekv@ifempty@false\ekv@ifempty@A\ekv@ifempty@B\@firstoftwo
  }
\long\def\ekvd@ifspace@#1 % keep this space
  {%
    \ekv@ifempty@\ekv@ifempty@A
  }
%    \end{macrocode}
% \end{macro}
%
% \subsubsection{Messages}
%
% Most messages of \expkvd\ are not expandable, since they only appear during
% key-definition, which is not expandable anyway.
%
% \begin{macro}[internal]
%   {
%     \ekvd@errm,\ekvd@err@missing@definition,\ekvd@err@missing@definition@msg,
%     \ekvd@err@missing@type,\ekvd@err@undefined@prefix,\ekvd@err@undefined@key,
%     \ekvd@err@no@prefix,\ekvd@err@no@prefix@msg,\ekvd@err@no@prefix@also,
%     \ekvd@err@add@val@on@noval,\ekvd@err@add@noval@on@val,
%     \ekvd@err@unsupported@arg,\ekvd@err@not@new
%   }
% The non-expandable error messages are boring, so here they are:
%    \begin{macrocode}
\protected\def\ekvd@errm#1{\errmessage{expkv-def Error: #1}}
\protected\def\ekvd@err@missing@definition
  {\ekvd@errm{Missing definition for key `\ekvd@cur'}}
\protected\def\ekvd@err@missing@definition@msg#1%
  {\ekvd@errm{Missing definition for key `\ekv@unexpanded{#1}'}}
\protected\def\ekvd@err@missing@type
  {\ekvd@errm{Missing type prefix for key `\ekvd@cur'}}
\protected\def\ekvd@err@undefined@prefix#1%
  {%
    \ekvd@errm
      {%
        Undefined prefix `\ekv@unexpanded{#1}' found while processing
        `\ekvd@cur'%
      }%
  }
\protected\def\ekvd@err@undefined@key#1%
  {%
    \ekvd@errm
      {Undefined key `\ekv@unexpanded{#1}' found while processing `\ekvd@cur'}%
  }
\protected\def\ekvd@err@undefined@noval#1%
  {%
    \ekvd@errm
      {%
        Undefined noval key `\unexpanded{#1}' found while processing
        `\ekvd@cur'%
      }%
  }
\protected\def\ekvd@err@no@prefix#1%
  {\ekvd@errm{prefix `#1' not accepted in `\ekvd@cur'}}
\protected\def\ekvd@err@no@prefix@msg#1#2%
  {\ekvd@errm{prefix `#2' not accepted in `\ekv@unexpanded{#1}'}}
\protected\def\ekvd@err@no@prefix@also#1%
  {\ekvd@errm{`\ekvd@cur' not allowed with a `#1' key}}
\protected\def\ekvd@err@add@val@on@noval
  {\ekvd@errm{`\ekvd@cur' not allowed with a NoVal key}}
\protected\def\ekvd@err@add@noval@on@val
  {\ekvd@errm{`\ekvd@cur' not allowed with a value taking key}}
\protected\def\ekvd@err@unsupported@arg\fi\@firstofone#1%
  {%
    \fi
    \ekvd@errm
      {%
        Existing key-macro has the unsupported argument string
        `\ekvd@extracted@args' for key `\ekvd@cur'%
      }%
  }
\protected\def\ekvd@err@not@new
  {\ekvd@errm{The key for `\ekvd@cur' is already defined}}
\protected\long\def\ekvd@err@misused@unknown
  {\ekvd@errm{Misuse of the unknown type found while processing `\ekvd@cur'}}
%    \end{macrocode}
% \end{macro}
%
% \begin{macro}[internal]
%   {
%     \ekvd@err@choice@invalid,\ekvd@err@choice@invalid@,\ekvd@choice@name,
%     \ekvd@unknown@choice@name
%   }
% |\ekvd@err@choice@invalid| will have to use this mechanism to throw its
% message. Also we have to retrieve the name parts of the choice in an easy way,
% so we use parentheses of catcode 8 here, which should suffice in most cases to
% allow for a correct separation.
%    \begin{macrocode}
\def\ekvd@err@choice@invalid#1%
  {%
    \ekvd@err@choice@invalid@#1%
  }
\begingroup
\catcode40=8
\catcode41=8
\@firstofone{\endgroup
\def\ekvd@choice@name#1#2#3%
  {%
    ekvd#1(#2)\detokenize{#3}%
  }
\def\ekvd@unknown@choice@name#1#2%
  {%
    ekvd:u:#1(#2)%
  }
\def\ekvd@err@choice@invalid@ ekvd#1(#2)\detokenize#3%
  {%
    \ekv@ifdefined{\ekvd@unknown@choice@name{#1}{#2}}%
      {\csname\ekvd@unknown@choice@name{#1}{#2}\endcsname{#3}}%
      {\ekvd@err{invalid choice `#3' for `#2' in set `#1'}}%
  }
}
%    \end{macrocode}
% \end{macro}
%
% \begin{macro}[internal]{\ekvd@err}
% The expandable error messages use |\ekvd@err|, which is just like |\ekv@err|
% from \expkv. It uses a runaway argument to start the error message.
%    \begin{macrocode}
\ekv@exparg{\long\def\ekvd@err#1}{\ekverr{expkv-def}{#1}}
%    \end{macrocode}
% \end{macro}
%
% Now everything that's left is to reset the category code of |@|.
%    \begin{macrocode}
\catcode`\@=\ekvd@tmp
%    \end{macrocode}
%
% \gobbledocstriptag
%</tex>
%^^A=<<
%
%^^A=<<
%
% \end{implementation}^^A=<<
%
% \clearpage
% \PrintIndex
%
\endinput
%
^^A vim: ft=tex fdm=marker fmr=>>=,=<<
}{}
\ProvidesFile{expkv-def.dtx}
  [\csname ekvdDate\endcsname\ a key-defining frontend for expkv]
\PassOptionsToPackage{full}{textcomp}
\documentclass{l3doc}
\RequirePackage[oldstylenums,nott]{kpfonts}
\input{glyphtounicode}
\pdfgentounicode=1
\RequirePackage{tcolorbox}
\newtcolorbox{exresult}[2][]
  {colback=expkvgrey!10!white,colframe=expkvgrey,fontupper=\small,width={#2},#1}
\newtcbox\exres[1][]
  {
     colback=expkvgrey!10!white
    ,colframe=expkvgrey
    ,size=small
    ,nobeforeafter
    ,tcbox raise base
    ,fontupper=\small
    ,#1
  }
\RequirePackage{listings}
\RequirePackage{booktabs}
\RequirePackage{array}
\RequirePackage{collcell}
\RequirePackage{siunitx}
\RequirePackage{xcolor}
\RequirePackage{caption}
\RequirePackage{microtype}
\RequirePackage{accsupp}
\lstdefinelanguage{expkvd}[primitive]{TeX}
  {
    ,moretexcs=[4]^^A e-TeX
      {
        expanded,
        numexpr,
        protected,
      }
    ,moretexcs=[5]^^A plain/LaTeX
      {
        approx,
        begin,
        empty,
        item,
        LaTeX,
        makeatletter,makeatother,
        newcommand,newdimen,
        RequirePackage,
        rule,
        TeX,
        textit,texttt,
        usepackage,
      }
    ,moretexcs=[6]^^A used packages
      {
        ^^A xfp
        fpeval,
        ^^A keyval
        setkeys,
        ^^A yax
        defactiveparameter,storevalue,setparameterlist
      }
    ,moretexcs=[2]^^A expkv-def macros
      {
        ekvdefinekeys
      }
    ,moretexcs=[3]^^A expkv-pkg macros
      {
        ^^A expkv
        ekvdef,ekvdefNoVal,^^A
        ekvlet,ekvletNoVal,ekvletkv,ekvletkvNoVal,^^A
        ekvdefunknown,ekvdefunknownNoVal,^^A
        ekvredirectunknown,ekvredirectunknownNoVal,^^A
        ekvset,ekvsetSneaked,^^A
        ekvsetdef,ekvsetSneakeddef,ekvsetdefSneaked,^^A
        ekvparse,^^A
        ekvVersion,ekvDate,^^A
        ekvifdefined,ekvifdefinedNoVal,ekvifdefinedset,^^A
        ekvbreak,ekvbreakPreSneak,ekvbreakPostSneak,^^A
        ekvsneak,ekvsneakPre,^^A
        ekvchangeset,^^A
        ekvoptarg,ekvoptargTF,^^A
        ekverr,
        ^^A expkv-cs
        ekvcSecondaryKeys,ekvcSplitAndForward,
      }
  }
\colorlet{codeparam}{cyan!65!black}
\def\mylstwd{.55em}
\lstset
  {
    ,language=expkvd
    ,flexiblecolumns=false
    ,basewidth=\mylstwd
    ,gobble=2
    ,basicstyle=\fontfamily{jkp}\itshape
    ,texcsstyle=*[1]{\bfseries\color{expkvgrey}} ^^A primitives
    ,texcsstyle=*[2]{\bfseries\color{expkvred}}  ^^A expkv
    ,texcsstyle=*[3]{\color{expkvred}}           ^^A expkv-pkg
    ,texcsstyle=*[4]{\bfseries\color{expkvgrey}} ^^A e-TeX
    ,texcsstyle=*[5]{\bfseries\color{expkvgrey}} ^^A plain/LaTeX macros
    ,texcsstyle=*[6]{}                           ^^A macros of other packages
    ,commentstyle=\color[gray]{0.4}
    ,literate=
              {\{} {{\CodeUpBf\{}}{1}
              {\}} {{\CodeUpBf\}}}{1}
              {$}  {{\CodeUpBf\$}}{1}
              {[}  {{\CodeUp[}}{1}
              {]}  {{\CodeUp]}}{1}
              {(}  {{\CodeUp(}}{1}
              {)}  {{\CodeUp)}}{1}
              {*}  {{$*$}}{1}
              {1}  {{\CodeUp{1}}}{1}
              {2}  {{\CodeUp{2}}}{1}
              {3}  {{\CodeUp{3}}}{1}
              {4}  {{\CodeUp{4}}}{1}
              {5}  {{\CodeUp{5}}}{1}
              {6}  {{\CodeUp{6}}}{1}
              {7}  {{\CodeUp{7}}}{1}
              {8}  {{\CodeUp{8}}}{1}
              {9}  {{\CodeUp{9}}}{1}
              {0}  {{\CodeUp{0}}}{1}
              {##} {{\CodeColored{codeparam}{1}{\#}}}{1}
              {##1}{{\CodeColored{codeparam}{2}{\#1}}}{2}
              {##2}{{\CodeColored{codeparam}{2}{\#2}}}{2}
              {##3}{{\CodeColored{codeparam}{2}{\#3}}}{2}
              {##4}{{\CodeColored{codeparam}{2}{\#4}}}{2}
              {##5}{{\CodeColored{codeparam}{2}{\#5}}}{2}
              {##6}{{\CodeColored{codeparam}{2}{\#6}}}{2}
              {##7}{{\CodeColored{codeparam}{2}{\#7}}}{2}
              {##8}{{\CodeColored{codeparam}{2}{\#8}}}{2}
              {##9}{{\CodeColored{codeparam}{2}{\#9}}}{2}
              {<key>}{{$\langle$}key{$\rangle$}}{5}
              {<set>}{{$\langle$}set{$\rangle$}}{5}
              {<set1>}{{$\langle$}set1{$\rangle$}}{6}
              {<set2>}{{$\langle$}set2{$\rangle$}}{6}
    ^^A,literate=*{<key>}{\key}{4}{<set>}{\set}{4}
  }
\newcommand*\CodeColored[3]{\textcolor{#1}{\makebox[\dimexpr\mylstwd*#2]{#3}}}
\newcommand*\CodeUpBf[1]{\makebox[\mylstwd]{\textup{\textbf{#1}}}}
\newcommand*\CodeUp[1]{\makebox[\mylstwd]{\textup{#1}}}
\RequirePackage{randtext}
\let\metaORIG\meta
\protected\def\meta #1{\texttt{\metaORIG{#1}}}
\renewcommand*\thefootnote{\fnsymbol{footnote}}
\definecolor{expkvred}{HTML}{9F393D}
\colorlet{expkvgrey}{black!75}
\makeatletter
\newcommand*\expkv
  {^^A
    \texorpdfstring
      {^^A
        \mbox
          {^^A
            \BeginAccSupp{ActualText=expkv}^^A
            \href{https://github.com/Skillmon/tex_expkv}
              {^^A
                \rmfamily
                \bfseries
                {\color{expkvgrey}e\kern-.05em x\kern-.05em}^^A
                \lower.493ex
                  \hbox{{\color{expkvgrey}P}\kern-.1em{\color{expkvred}k}}^^A
                \kern-.18em{\color{expkvred}v}^^A
              }^^A
            \EndAccSupp{}^^A
          }^^A
      }
      {expkv}^^A
  }
\newcommand*\expkvd
  {^^A
    \texorpdfstring
      {^^A
        \mbox
          {^^A
            \BeginAccSupp{ActualText=expkv-def}^^A
            \href{https://github.com/Skillmon/tex_expkv-def}
              {^^A
                \rmfamily
                \bfseries
                {\color{expkvgrey}e\kern-.05em x\kern-.05em}^^A
                \lower.493ex
                  \hbox{{\color{expkvgrey}P}\kern-.1em{\color{expkvred}k}}^^A
                \kern-.18em{\color{expkvred}v}^^A
                {\color{expkvgrey}^^A
                  \kern.05em\rule[-.1ex]{.08em}{1.2ex}\kern.05em\textsc{def}^^A
                }^^A
              }^^A
            \EndAccSupp{}^^A
          }^^A
      }
      {expkv-def}^^A
  }
\newcommand\kv{\meta{key}=\meta{value}}
\newcommand\key{\meta{key}}
\newcommand\val{\meta{value}}
\newcommand\set{\meta{set}}
\newcommand\enfprefix{\textcolor{black}}
\newcommand\allprefix{\textcolor{gray}}
\newcommand\notprefix{\textcolor{red!80!black}}
\newcommand\printprefix[2]
  {^^A
    \ifcase\numexpr#1\relax\or
      \enfprefix{#2}\or
      \allprefix{#2}\or
      \notprefix{#2}\fi
  }
\newcommand\prefixes[4]
  {^^A
    \hfill
    \printprefix{#1}{new}
    \printprefix{#2}{also}
    \printprefix{#3}{protected}
    \printprefix{#4}{long}^^A
  }
\hypersetup{linkcolor=red!80!black,urlcolor=purple!80!black}
\DoNotIndex{\def,\edef,\,,\=,\begingroup,\catcode,\chardef,\csname,\endcsname}
\DoNotIndex{\endgroup,\endinput,\errmessage,\expandafter,\input,\let,\long}
\DoNotIndex{\protected,\ProvidesFile,\ProvidesPackage,\relax,\space}
\DoNotIndex{\@,\unexpanded,\string,\expanded,\dimexpr,\global,\glueexpr,\hbox}
\DoNotIndex{\numexpr,\RequirePackage,\setbox,\the,\unless,\xdef,\@firstofone}
\DoNotIndex{\@firstoftwo,\@gobble,\@secondoftwo,\meaning,\detokenize}
\DoNotIndex{\romannumeral,\^}
\DoNotIndex{\ifcsname}
\DoNotIndex{\ifx}
\DoNotIndex{\ifdefined}
\DoNotIndex{\iffalse}
\DoNotIndex{\iftrue}
\DoNotIndex{\else}
\DoNotIndex{\fi}
\@gobble\fi ^^A ignoring \ifx, \ifcsname, etc., but only one \fi
\@gobble\fi ^^A ignoring \ifx, \ifcsname, etc., but only one \fi
\@gobble\fi ^^A ignoring \ifx, \ifcsname, etc., but only one \fi
\@gobble\fi ^^A ignoring \ifx, \ifcsname, etc., but only one \fi
\@ifdefinable\gobbledocstriptag{\def\gobbledocstriptag#1>{}}
\makeatother
\begin{document}
  \title
    {^^A
      \texorpdfstring
        {^^A
          \huge\expkvd
          \\[\medskipamount]
          \Large a key-defining frontend for \expkv
        }{expkv-def - a key-defining frontend for expkv}^^A
    }
  \date{\ekvdDate\space v\ekvdVersion}
  \author{Jonathan P. Spratte\thanks{\protect\randomize{jspratte@yahoo.de}}}
  \DocInput{expkv-def.dtx}
\end{document}
%</driver>^^A=<<
% \fi
%
% \maketitle
% \renewcommand*\thefootnote{\arabic{footnote}}
%
% \begin{abstract}
% \noindent\parfillskip=0pt
% \expkvd\ provides a small \kv\ interface to define keys for \expkv. Key-types
% are declared using prefixes, similar to static typed languages. The stylised
% name is \expkvd\ but the files use \file{expkv-def}, this is due to CTAN-rules
% which don't allow \string| in package names since that is the pipe symbol in
% *nix shells.
% \end{abstract}
%
% \tableofcontents
%
% \begin{documentation}^^A>>=
%
% \section{Documentation}
%
% Since the trend for the last couple of years goes to defining keys for a \kv\
% interface using a \kv\ interface, I thought that maybe providing such an
% interface for \expkv\ will make it more attractive for actual use, besides its
% unique selling points of being fully expandable, and fast and reliable. But at
% the same time I don't want to widen \expkv's initial scope. So here it is
% \expkvd, go define \kv\ interfaces with \kv\ interfaces.
%
% Unlike many of the other established \kv\ interfaces to define keys, \expkvd\
% works using prefixes instead of suffixes (\emph{e.g.}, |.tl_set:N| of
% \pkg{l3keys}) or directory like handlers (\emph{e.g.}, |/.store in| of
% \pkg{pgfkeys}). This was decided as a personal preference, more over in \TeX\
% parsing for the first space is way easier than parsing for the last one.
% \expkvd's prefixes are sorted into two categories: |p|-type, which are
% equivalent to \TeX's prefixes like |\long|, and |t|-type defining the type of
% the key. For a description of the available |p|-prefixes take a look at
% \autoref{sec:prefix:p}, the |t|-prefixes are described in
% \autoref{sec:prefix:t}.
%
% \expkvd\ is usable as generic code, as a \LaTeX\ package, and as a \ConTeXt\
% module. It'll automatically load \expkv\ in the same mode as well. To use it,
% just use one of
% \begin{lstlisting}
% \input expkv-def       % plainTeX
% \usepackage{expkv-def} % LaTeX
% \usemodule[expkv-def]  % ConTeXt
% \end{lstlisting}
%
% \subsection{Macros}
% Apart from version and date containers there is only a single user-facing
% macro, and that should be used to define keys.
%
% \begin{function}{\ekvdefinekeys}
%   \begin{syntax}
%     \cs{ekvdefinekeys}\marg{set}\{\kv, \ldots\}
%   \end{syntax}
%   In \set, define \key\ to have definition \val. The general syntax for \key\
%   should be
%   \begin{quote}\ttfamily\small
%     \meta{prefix} \meta{name}
%   \end{quote}
%   Where \meta{prefix} is a space separated list of optional |p|-type prefixes
%   followed by one |t|-type prefix. The syntax of \val\ is dependent on the
%   used |t|-prefix.
% \end{function}
%
% \begin{function}{\ekvdDate,\ekvdVersion}
%   These two macros store the version and date of the package.
% \end{function}
%
% \subsection{Prefixes}
% As already said there are |p|-prefixes and |t|-prefixes. Not every |p|-prefix
% is allowed for all |t|-prefixes.
%
% \subsubsection{\texttt{p}-Prefixes}\label{sec:prefix:p}
% The two |p|-type prefixes |long| and |protected| are pretty simple by nature,
% so their description is pretty simple. They affect the \key\ at use-time, so
% omitting |long| doesn't mean that a \meta{definition} can't contain a |\par|
% token, only that the \key\ will not accept a |\par| in \val.  On the other
% hand |new| and |also| might be simple on first sight as well, but their rules
% are a bit more complicated.
%
% \begin{function}{also}
%   The following key type will be \emph{added} to an existing \key's
%   definition. You can't add a type taking an argument at use time to an
%   existing key which doesn't take an argument and vice versa. Also you'll get
%   an error if you try to add an action which isn't allowed to be either |long|
%   or |protected| to a key which already is |long| or |protected| (the opposite
%   order would be suboptimal as well, but can't be really captured with the
%   current code).
%
%   A key already defined as |long| or |protected| will stay |long| or
%   |protected|, but you can as well add |long| or |protected| with the |also|
%   definition.
%
%   As a small example, suppose you want to create a boolean key, but
%   additionally to setting a boolean value you want to execute some more code
%   as well, you can use the following
% \begin{lstlisting}
% \ekvdefinekeys{also-example}
%   {
%      bool key      = \ifmybool
%     ,also code key = \domystuff{#1}
%   }
% \end{lstlisting}
%
%   If you use |also| on a |choice|, |bool|, |invbool|, or |boolpair| key it is
%   tried to determine if the key already is of one of those types. If this test
%   is true the declared choices will be added to the possible choices but the
%   key's definition will not be changed other than that. If that wouldn't have
%   been done, the callbacks of the different choices could get called multiple
%   times.
% \end{function}
%
% \begin{function}{protected,protect}
%   The following key will be defined |\protected|. Note that key-types which
%   can't be defined expandable will always use |\protected|.
% \end{function}
%
% \begin{function}{long}
%   The following key will be defined |\long|.
% \end{function}
%
% \begin{function}{new}
%   The following key must be new (so previously undefined). An error is thrown
%   if it is already defined and the new definition is ignored. |new| only
%   asserts that there are no conflicts between |NoVal| keys and other |NoVal|
%   keys or value taking keys and other value taking keys. For example you can
%   use the following without an error:
% \begin{lstlisting}
% \ekvdefinekeys{new-example}
%   {
%      code key       = \domystuffwitharg{#1}
%     ,new  noval key = \domystuffwithoutarg
%   }
% \end{lstlisting}
% \end{function}
%
% \subsubsection{\texttt{t}-Prefixes}\label{sec:prefix:t}
% Since the |p|-type prefixes apply to some of the |t|-prefixes automatically
% but sometimes one might be disallowed we need some way to highlight this
% behaviour. In the following an enforced prefix will be printed black
% (\texttt{\enfprefix{protected}}), allowed prefixes will be grey
% (\texttt{\allprefix{protected}}), and disallowed prefixes will be red
% (\texttt{\notprefix{protected}}). This will be put flush-right in the syntax
% showing line.
%
% \begin{function}{code,ecode}
%   \begin{syntax}
%     code \key\ = \marg{definition} \prefixes2222
%   \end{syntax}
%   Define \key\ to expand to \meta{definition}. The \key\ will require a \val\
%   for which you can use |#1| inside \meta{definition}. The |ecode| variant
%   will fully expand \meta{definition} inside an |\edef|.
% \end{function}
%
% \begin{function}{noval,enoval}
%   \begin{syntax}
%     noval \key\ = \marg{definition} \prefixes2223
%   \end{syntax}
%   The |noval| type defines \key\ to expand to \meta{definition}. The \key\
%   will not take a \val. |enoval| fully expands \meta{definition} inside an
%   |\edef|.
% \end{function}
%
% \begin{function}{default,qdefault,odefault,fdefault,edefault}
%   \begin{syntax}
%     default \key\ = \marg{definition} \prefixes3223
%   \end{syntax}
%   This serves to place a default \val\ for a \key\ that takes an argument, the
%   \key\ can be of any argument-grabbing kind, and when used without a \val\
%   it will be passed \meta{definition} instead. The |qdefault| variant will
%   expand the \key's code once, so will be slightly quicker, but not change if
%   you redefine \key. |odefault| is just another name for |qdefault|. The
%   |fdefault| version will expand the key code until a non-expandable token or
%   a space is found, a space would be gobbled.\footnotemark{} The |edefault| on
%   the other hand fully expands the \key-code with \meta{definition} as its
%   argument inside of an |\edef|.
% \end{function}
% \footnotetext
%   {^^A
%     For those familiar with \TeX-coding: This uses a
%     \cs[no-index]{romannumeral}-expansion.
%   }
%
% \begin{function}{initial,oinitial,finitial,einitial}
%   \begin{syntax}
%     initial \key\ = \{\val\} \prefixes3333
%     initial \key
%   \end{syntax}
%   With |initial| you can set an initial \val\ for an already defined argument
%   taking \key. It'll just call the key-macro of \key\ and pass it \val. The
%   |einitial| variant will expand \val\ using an |\edef| expansion prior to
%   passing it to the key-macro and the |oinitial| variant will expand the first
%   token in \val\ once. |finitial| will expand \val\ until a non-expandable
%   token or a space is found, a space would be
%   gobbled.\footnotemark
%
%   If you don't provide a value (and no equals sign) a |noval| \key\ of the
%   same name is called once (or, if you specified a |default| for a value
%   taking key that would be used).
% \end{function}
% \footnotetext{Again using \cs[no-index]{romannumeral}.}
%
% \begin{function}{bool,gbool,boolTF,gboolTF}
%   \begin{syntax}
%     bool \key\ = \meta{cs} \prefixes2223
%   \end{syntax}
%   The \meta{cs} should be a single control sequence, such as |\iffoo|.
%   This will define \key\ to be a boolean key, which only takes the values
%   |true| or |false| and will throw an error for other values. If the key is
%   used without a \val\ it'll have the same effect as if you use
%   \texttt{\key=true}. |bool| and |gbool| will behave like \TeX-ifs so either
%   be |\iftrue| or |\iffalse|. The |boolTF| and |gboolTF| variants will both
%   take two arguments and if true the first will be used else the second, so
%   they are always either |\@firstoftwo| or |\@secondoftwo|. The variants with
%   a leading |g| will set the control sequence globally, the others locally. If
%   \meta{cs} is not yet defined it'll be initialised as the |false| version.
%   Note that the initialisation is \emph{not} done with |\newif|, so you will
%   not be able to do |\footrue| outside of the \kv\ interface, but you could
%   use |\newif| yourself. Even if the \key\ will not be |\protected| the
%   commands which execute the |true| or |false| choice will be, so the usage
%   should be safe in an expansion context (\emph{e.g.}, you can use
%   \texttt{edefault \key\ = false} without an issue to change the default
%   behaviour to execute the |false| choice). Internally a |bool| \key\ is the
%   same as a |choice| key which is set up to handle |true| and |false| as
%   choices.
% \end{function}
%
% \begin{function}{invbool,ginvbool,invboolTF,ginvboolTF}
%   \begin{syntax}
%     bool \key\ = \meta{cs} \prefixes2223
%   \end{syntax}
%   These are inverse boolean keys, they behave like |bool| and friends but set
%   the opposite meaning to the macro \meta{cs} in each case. So if
%   \texttt{key=true} is used |invbool| will set \meta{cs} to |\iffalse| and
%   vice versa.
% \end{function}
%
% \begin{function}{boolpair,gboolpair,boolpairTF,gboolpairTF}
%   \begin{syntax}
%     boolpair \key\ = \meta{cs_1}\meta{cs_2} \prefixes2223
%   \end{syntax}
%   The |boolpair| key type behaves like both |bool| and |invbool|, the
%   \meta{cs_1} will be set to the meaning according to the rules of |bool|, and
%   \meta{cs_2} will be set to the opposite.
% \end{function}
%
% \begin{function}{store,estore,gstore,xstore}
%   \begin{syntax}
%     store \key\ = \meta{cs} \prefixes2212
%   \end{syntax}
%   The \meta{cs} should be a single control sequence, such as |\foo|.
%   This will define \key\ to store \val\ inside of the control sequence. If
%   \meta{cs} isn't yet defined it will be initialised as empty. The variants
%   behave similarly to their |\def|, |\edef|, |\gdef|, and |\xdef|
%   counterparts, but |store| and |gstore| will allow you to store macro
%   parameters inside of them by using |\unexpanded|.
% \end{function}
%
% \begin{function}{data,edata,gdata,xdata}
%   \begin{syntax}
%     data \key\ = \meta{cs} \prefixes2212
%   \end{syntax}
%   The \meta{cs} should be a single control sequence, such as |\foo|. This will
%   define \key\ to store \val\ inside of the control sequence. But unlike the
%   |store| type, the macro \meta{cs} will be a switch at the same time, it'll
%   take two arguments and if \key\ was used expands to the first argument
%   followed by \val\ in braces, if \key\ was not used \meta{cs} will expand to
%   the second argument (so behave like |\@secondoftwo|). The idea is that with
%   this type you can define a key which should be typeset formatted. The
%   |edata| and |xdata| variants will fully expand \val, the |gdata| and |xdata|
%   variants will store \val\ inside \meta{cs} globally. The |p|-prefixes will
%   only affect the key-macro, \meta{cs} will always be expandable and |\long|.
% \end{function}
%
% \begin{function}{dataT,edataT,gdataT,xdataT}
%   \begin{syntax}
%     dataT \key\ = \meta{cs} \prefixes2212
%   \end{syntax}
%   Just like |data|, but instead of \meta{cs} grabbing two arguments it'll only
%   grab one, so by default it'll behave like |\@gobble|, and if a \val\ was
%   given to \key\ the \meta{cs} will behave like |\@firstofone| appended by
%   \marg{value}.
% \end{function}
%
% \begin{function}{int,eint,gint,xint}
%   \begin{syntax}
%     int \key\ = \meta{cs} \prefixes2212
%   \end{syntax}
%   The \meta{cs} should be a single control sequence, such as |\foo|.
%   An |int| key will be a \TeX-count register. If \meta{cs} isn't defined yet,
%   |\newcount| will be used to initialise it. The |eint| and |xint| versions
%   will use |\numexpr| to allow basic computations in their \val. The |gint|
%   and |xint| variants set the register globally.
% \end{function}
%
% \begin{function}{dimen,edimen,gdimen,xdimen}
%   \begin{syntax}
%     dimen \key\ = \meta{cs} \prefixes2212
%   \end{syntax}
%   The \meta{cs} should be a single control sequence, such as |\foo|.
%   This is just like |int| but uses a dimen register, |\newdimen| and
%   |\dimexpr| instead.
% \end{function}
%
% \begin{function}{skip,eskip,gskip,xskip}
%   \begin{syntax}
%     skip \key\ = \meta{cs} \prefixes2212
%   \end{syntax}
%   The \meta{cs} should be a single control sequence, such as |\foo|.
%   This is just like |int| but uses a skip register, |\newskip| and |\glueexpr|
%   instead.
% \end{function}
%
% \begin{function}{toks,gtoks,apptoks,gapptoks,pretoks,gpretoks}
%   \begin{syntax}
%     toks \key\ = \meta{cs} \prefixes2212
%   \end{syntax}
%   The \meta{cs} should be a single control sequence, such as |\foo|.
%   Store \val\ inside of a toks-register. The |g| variants use |\global|, the
%   |app| variants append \val\ to the contents of that register, the |pre|
%   variants will prepend \val. If \meta{cs} is not yet defined it will be
%   initialised with |\newtoks|.
% \end{function}
%
% \begin{function}{box,gbox}
%   \begin{syntax}
%     box \key\ = \meta{cs} \prefixes2212
%   \end{syntax}
%   The \meta{cs} should be a single control sequence, such as |\foo|.
%   Typesets \val\ into a |\hbox| and stores the result in a box register. The
%   boxes are colour safe. \expkvd\ doesn't provide a |vbox| type.
% \end{function}
%
% \begin{function}{meta}
%   \begin{syntax}
%     meta \key\ = \{\kv, \ldots\} \prefixes2222
%   \end{syntax}
%   This key type can set other keys, you can access the \val\ which was passed
%   to \key\ inside the \kv\ list with |#1|. It works by calling a sub-|\ekvset|
%   on the \kv\ list, so a |set| key will only affect that \kv\ list and not the
%   current |\ekvset|. Since it runs in a separate |\ekvset| you can't use
%   |\ekvsneak| using keys or similar macros in the way you normally could.
% \end{function}
%
% \begin{function}{nmeta}
%   \begin{syntax}
%     nmeta \key\ = \{\kv, \ldots\} \prefixes2223
%   \end{syntax}
%   This key type can set other keys, the difference to |meta| is, that this key
%   doesn't take a value, so the \kv\ list is static.
% \end{function}
%
% \begin{function}{smeta}
%   \begin{syntax}
%     smeta \key\ = \marg{set}\{\kv, \ldots\} \prefixes2222
%   \end{syntax}
%   Yet another |meta| variant. An |smeta| key will take a \val\ which you can
%   access using |#1|, but it sets the \kv\ list inside of \set, so is equal to
%   \texttt{\cs[no-index]{ekvset}\marg{set}\{\kv, \ldots\}}.
% \end{function}
%
% \begin{function}{snmeta}
%   \begin{syntax}
%     snmeta \key\ = \marg{set}\{\kv, \ldots\} \prefixes2223
%   \end{syntax}
%   And the last |meta| variant. |snmeta| is a combination of |smeta| and
%   |nmeta|. It doesn't take an argument and sets the \kv\ list inside of \set.
% \end{function}
%
% \begin{function}{set}
%   \begin{syntax}
%     set \key\ = \marg{set} \prefixes2233
%   \end{syntax}
%   This will define \key\ to change the set of the current |\ekvset|
%   invocation to \set. You can omit \set\ (including the equals sign),
%   which is the same as using \texttt{set \key\ = \{\key\}}. The created |set|
%   key will not take a \val. Note that just like in \expkv\ it'll not be
%   checked whether \set\ is defined and you'll get a low-level \TeX\ error if
%   you use an undefined \set.
% \end{function}
%
% \begin{function}{choice}
%   \begin{syntax}
%     choice \key\ = \{\val=\meta{definition}, \ldots\} \prefixes2223
%   \end{syntax}
%   Defines \key\ to be a |choice| key, meaning it will only accept a limited
%   set of values. You should define each possible \val\ inside of the
%   \val=\meta{definition} list. If a defined \val\ is passed to \key\ the
%   \meta{definition} will be left in the input stream. You can make
%   individual values |protected| inside the \val=\meta{definition} list. By
%   default a |choice| key is expandable, an undefined \val\ will throw an error
%   in an expandable way (but see the |unknown-choice| prefix). You can add
%   additional choices after the \key\ was created by using |choice| again for
%   the same \key, redefining choices is possible the same way, but there is no
%   interface to remove certain choices.
% \end{function}
%
% \begin{function}{choice-store}
%   \begin{syntax}
%     choice-store \key\ = \meta{cs}\{\val, \ldots\} \prefixes2223
%   \end{syntax}
%   This is defines a special type of |choice| key that'll store the given
%   choice inside the macro \meta{cs} (so \meta{cs} should be a single control
%   sequence name such as |\foo|). Since storing inside a macro can't be done
%   expandably every choice-code is |\protected|, you might define the
%   |choice-store| key itself as |protected| as well if you want. Since the
%   definition of each choice is predefined with this key type the choice list
%   should just be a comma separated list of valid choices.
% \end{function}
% This means that the following |choice| and |choice-store| keys are equivalent
% at use time:
% \begin{lstlisting}
% \newcommand*\mya{}
% \ekvdefinekeys{example}
%   {
%      choice key1 = {a=\def\mya{a}, b=\def\mya{b}, c=\def\mya{c}}
%     ,choice-store key2 = \mya{a,b,c}
%   }
% \end{lstlisting}
%
% \begin{function}{choice-enum}
%   \begin{syntax}
%     choice-enum \key\ = \meta{cs}\{\val, \ldots\} \prefixes2223
%   \end{syntax}
%   This is similar to |choice-store|, the differences are: \meta{cs} should be
%   a count-register or is initialised as such if the \meta{cs} is undefined
%   (via |\newcount|); instead of the value the position of the value in the
%   given list is stored in this register (zero-based).
% \end{function}
% This means that the following |choice| and |choice-enum| keys are equivalent
% at use time:
% \begin{lstlisting}
% \newcount\myb
% \ekvdefinekeys{example}
%   {
%      choice key1 = {a={\myb=0 }, b={\myb=1 }, c={\myb=2 }}
%     ,choice-enum key2 = \myb{a,b,c}
%   }
% \end{lstlisting}
%
% \begin{function}{unknown-choice}
%   \begin{syntax}
%     unknown-choice \key\ = \marg{definition} \prefixes2323
%   \end{syntax}
%   By default an unknown \val\ passed to a |choice| or |bool| key will throw an
%   error.  However, with this prefix you can define an alternative action which
%   should be executed if \key\ received an unknown choice. In \meta{definition}
%   you can refer to the choice which was passed in with |#1|.
% \end{function}
%
% \begin{function}{unknown code}
%   \begin{syntax}
%     unknown code = \marg{definition} \prefixes2322
%   \end{syntax}
%   By default \expkv\ throws errors when it encounters unknown keys in a set.
%   With the |unknown| prefix you can define handlers that deal with undefined
%   keys, instead of a \key\ name you have to specify a subtype for this prefix,
%   here the subtype is |code|.\par
%   With |unknown code| the \meta{definition} is used for unknown keys which
%   were provided a value (so corresponds to |\ekvdefunknown|), you can access
%   the key name with |#1| and the value with |#2|.\footnotemark
% \end{function}
% \footnotetext
%   {^^A
%     There is some trickery involved to get this more intuitive argument order
%     without any performance hit if you compare this to
%     \cs[no-index]{ekvdefunknown} directly.^^A
%   }
%
% \begin{function}{unknown noval}
%   \begin{syntax}
%     unknown noval = \marg{definition} \prefixes2323
%   \end{syntax}
%   This is like |unknown code| but uses \meta{definition} for unknown keys to
%   which no value was passed (so corresponds to |\ekvdefunknownNoVal|). You can
%   access the key name with |#1|.
% \end{function}
%
% \begin{function}{unknown redirect-code}
%   \begin{syntax}
%     unknown redirect-code = \marg{set-list} \prefixes2331
%   \end{syntax}
%   This uses a predefined action for |unknown code|. Instead of throwing an
%   error, it is tried to find the \key\ in each \set\ in the comma separated
%   \meta{set-list}. The first found match will be used and the remaining
%   options from the list discarded. If the \key\ isn't found in any \set\ an
%   expandable error will be thrown eventually. Internally \expkv's
%   |\ekvredirectunknown| will be used.
% \end{function}
%
% \begin{function}{unknown redirect-noval}
%   \begin{syntax}
%     unknown redirect-noval = \marg{set-list} \prefixes2333
%   \end{syntax}
%   This behaves just like |unknown redirect-code| but will set up means to
%   forward keys for |unknown noval|. Internally \expkv's
%   |\ekvredirectunknownNoVal| will be used.
% \end{function}
%
% \begin{function}{unknown redirect}
%   \begin{syntax}
%     unknown redirect = \marg{set-list} \prefixes2333
%   \end{syntax}
%   This is a short cut to apply both, |unknown redirect-code| and
%   |unknown redirect-noval|, as a result you might get doubled error messages,
%   one from each.
% \end{function}
%
% \subsection{Bugs}
% I don't think there are any (but every developer says that), if you find some
% please let me know, either via the email address on the first page or on
% GitHub:
% \url{https://github.com/Skillmon/tex_expkv-def}
%
% \subsection{Example}
% The following is an example code defining each base key-type once. Please
% admire the very creative key-name examples.
%
% \begin{lstlisting}
% \ekvdefinekeys{example}
%   {
%      long code keyA = #1
%     ,noval     keyA = NoVal given
%     ,bool      keyB = \keyB
%     ,boolTF    keyC = \keyC
%     ,store     keyD = \keyD
%     ,data      keyE = \keyE
%     ,dataT     keyF = \keyF
%     ,int       keyG = \keyG
%     ,dimen     keyH = \keyH
%     ,skip      keyI = \keyI
%     ,toks      keyJ = \keyJ
%     ,default   keyJ = \empty test
%     ,new box   keyK = \keyK
%     ,qdefault  keyK = K
%     ,choice    keyL =
%       {
%          protected 1 = \texttt{a}
%         ,2 = b
%         ,3 = c
%         ,4 = d
%         ,5 = e
%       }
%     ,edefault  keyL = 2
%     ,meta      keyM = {keyA={#1},keyB=false}
%     ,invbool   keyN = \keyN
%     ,boolpair  keyO = \keyOa\keyOb
%   }
% \end{lstlisting}
%
% \bigskip
% Since the |data| type might be a bit strange, here is another usage example
% for it.
% \begin{lstlisting}
% \ekvdefinekeys{ex}
%   {
%      data  name  = \Pname
%     ,data  age   = \Page
%     ,dataT hobby = \Phobby
%   }
% \newcommand\Person[1]
%   {%
%     \begingroup
%     \ekvset{ex}{#1}%
%     \begin{description}
%       \item[\Pname{}{\errmessage{A person requires a name}}]
%       \item[Age] \Page{\textit}{\errmessage{A person requires an age}}
%       \Phobby{\item[Hobbies]}
%     \end{description}
%     \endgroup
%   }
% \Person{name=Jonathan P. Spratte, age=young, hobby=\TeX\ coding}
% \Person{name=Some User, age=unknown, hobby=Reading Documentation}
% \Person{name=Anybody, age=any}
% \end{lstlisting}
% In this example a person should have a name and an age, but doesn't have to
% have hobbies. The name will be displayed as the description item and the age
% in Italics. If a person has no hobbies the description item will be silently
% left out.  The result of the above code looks like this:
%
% \begin{exresult}{\linewidth}
% \ekvdefinekeys{ex}
%   {
%      data  name  = \Pname
%     ,data  age   = \Page
%     ,dataT hobby = \Phobby
%   }
% \newcommand\Person[1]
%   {%
%     \begingroup
%     \ekvset{ex}{#1}%
%     \begin{description}
%       \item[\Pname{}{\errmessage{A person requires a name}}]
%       \item[Age] \Page{\textit}{\errmessage{A person requires an age}}
%       \Phobby{\item[Hobbies]}
%     \end{description}
%     \endgroup
%   }
% \Person{name=Jonathan P. Spratte, age=young, hobby=\TeX\ coding}
% \Person{name=Some User, age=unknown, hobby=Reading Documentation}
% \Person{name=Anybody, age=any}
% \end{exresult}
%
% \subsection{License}
%
% Copyright \textcopyright\ 2020\unless\ifnum\year=2020--\the\year\fi\
% Jonathan P. Spratte
%
% \medskip\noindent
% This work may be distributed and/or modified under the conditions of the
% \LaTeX\ Project Public License (LPPL), either version 1.3c of this license or
% (at your option) any later version. The latest version of this license is in
% the file:
%
% \url{http://www.latex-project.org/lppl.txt}
%
% \noindent
% This work is ``maintained'' (as per LPPL maintenance status) by
%
% \mbox{Jonathan P. Spratte}.
%
% \end{documentation}^^A=<<
%
% \begin{implementation}^^A>>=
%
% \clearpage
%
% \section{Implementation}^^A>>=
%^^A the LaTeX package >>=
% \subsection{The \LaTeX\ Package}
% Just like for \expkv\ we provide a small \LaTeX\ package that sets up things
% such that we behave nicely on \LaTeX\ packages and files system. It'll
% |\input| the generic code which implements the functionality.
% \gobbledocstriptag
%<*pkg>
%    \begin{macrocode}
\RequirePackage{expkv}
\def\ekvd@tmp
  {%
    \ProvidesFile{expkv-def.tex}%
      [\ekvdDate\space v\ekvdVersion\space a key-defining frontend for expkv]%
  }
% \iffalse meta-comment
%
% File: expkv-def.dtx Copyright (C) 2020-2021 Jonathan P. Spratte
%
% This work  may be  distributed and/or  modified under  the conditions  of the
% LaTeX Project Public License (LPPL),  either version 1.3c  of this license or
% (at your option) any later version.  The latest version of this license is in
% the file:
%
%   http://www.latex-project.org/lppl.txt
%
% ------------------------------------------------------------------------------
%
%<*driver>^^A>>=
\def\nameofplainTeX{plain}
\ifx\fmtname\nameofplainTeX\else
  \expandafter\begingroup
\fi
\input l3docstrip.tex
\askforoverwritefalse
\preamble

--------------------------------------------------------------
expkv-def -- a key-defining frontend for expkv
E-mail: jspratte@yahoo.de
Released under the LaTeX Project Public License v1.3c or later
See http://www.latex-project.org/lppl.txt
--------------------------------------------------------------

Copyright (C) 2020-2021 Jonathan P. Spratte

This  work may be  distributed and/or  modified under  the conditions  of the
LaTeX Project Public License (LPPL),  either version 1.3c  of this license or
(at your option) any later version.  The latest version of this license is in
the file:

  http://www.latex-project.org/lppl.txt

This work is "maintained" (as per LPPL maintenance status) by
  Jonathan P. Spratte.

This work consists of the file  expkv-def.dtx
and the derived files           expkv-def.pdf
                                expkv-def.sty
                                expkv-def.tex
                                t-expkv-def.tex

\endpreamble
% stop docstrip adding \endinput
\postamble
\endpostamble
\generate{\file{expkv-def.sty}{\from{expkv-def.dtx}{pkg}}}
\generate{\file{expkv-def.tex}{\from{expkv-def.dtx}{tex}}}
\generate{\file{t-expkv-def.tex}{\from{expkv-def.dtx}{ctx}}}
\ifx\fmtname\nameofplainTeX
  \expandafter\endbatchfile
\else
  \expandafter\endgroup
\fi
%
\IfFileExists{expkv-def.tex}{% \iffalse meta-comment
%
% File: expkv-def.dtx Copyright (C) 2020-2021 Jonathan P. Spratte
%
% This work  may be  distributed and/or  modified under  the conditions  of the
% LaTeX Project Public License (LPPL),  either version 1.3c  of this license or
% (at your option) any later version.  The latest version of this license is in
% the file:
%
%   http://www.latex-project.org/lppl.txt
%
% ------------------------------------------------------------------------------
%
%<*driver>^^A>>=
\def\nameofplainTeX{plain}
\ifx\fmtname\nameofplainTeX\else
  \expandafter\begingroup
\fi
\input l3docstrip.tex
\askforoverwritefalse
\preamble

--------------------------------------------------------------
expkv-def -- a key-defining frontend for expkv
E-mail: jspratte@yahoo.de
Released under the LaTeX Project Public License v1.3c or later
See http://www.latex-project.org/lppl.txt
--------------------------------------------------------------

Copyright (C) 2020-2021 Jonathan P. Spratte

This  work may be  distributed and/or  modified under  the conditions  of the
LaTeX Project Public License (LPPL),  either version 1.3c  of this license or
(at your option) any later version.  The latest version of this license is in
the file:

  http://www.latex-project.org/lppl.txt

This work is "maintained" (as per LPPL maintenance status) by
  Jonathan P. Spratte.

This work consists of the file  expkv-def.dtx
and the derived files           expkv-def.pdf
                                expkv-def.sty
                                expkv-def.tex
                                t-expkv-def.tex

\endpreamble
% stop docstrip adding \endinput
\postamble
\endpostamble
\generate{\file{expkv-def.sty}{\from{expkv-def.dtx}{pkg}}}
\generate{\file{expkv-def.tex}{\from{expkv-def.dtx}{tex}}}
\generate{\file{t-expkv-def.tex}{\from{expkv-def.dtx}{ctx}}}
\ifx\fmtname\nameofplainTeX
  \expandafter\endbatchfile
\else
  \expandafter\endgroup
\fi
%
\IfFileExists{expkv-def.tex}{\input{expkv-def.tex}}{}
\ProvidesFile{expkv-def.dtx}
  [\csname ekvdDate\endcsname\ a key-defining frontend for expkv]
\PassOptionsToPackage{full}{textcomp}
\documentclass{l3doc}
\RequirePackage[oldstylenums,nott]{kpfonts}
\input{glyphtounicode}
\pdfgentounicode=1
\RequirePackage{tcolorbox}
\newtcolorbox{exresult}[2][]
  {colback=expkvgrey!10!white,colframe=expkvgrey,fontupper=\small,width={#2},#1}
\newtcbox\exres[1][]
  {
     colback=expkvgrey!10!white
    ,colframe=expkvgrey
    ,size=small
    ,nobeforeafter
    ,tcbox raise base
    ,fontupper=\small
    ,#1
  }
\RequirePackage{listings}
\RequirePackage{booktabs}
\RequirePackage{array}
\RequirePackage{collcell}
\RequirePackage{siunitx}
\RequirePackage{xcolor}
\RequirePackage{caption}
\RequirePackage{microtype}
\RequirePackage{accsupp}
\lstdefinelanguage{expkvd}[primitive]{TeX}
  {
    ,moretexcs=[4]^^A e-TeX
      {
        expanded,
        numexpr,
        protected,
      }
    ,moretexcs=[5]^^A plain/LaTeX
      {
        approx,
        begin,
        empty,
        item,
        LaTeX,
        makeatletter,makeatother,
        newcommand,newdimen,
        RequirePackage,
        rule,
        TeX,
        textit,texttt,
        usepackage,
      }
    ,moretexcs=[6]^^A used packages
      {
        ^^A xfp
        fpeval,
        ^^A keyval
        setkeys,
        ^^A yax
        defactiveparameter,storevalue,setparameterlist
      }
    ,moretexcs=[2]^^A expkv-def macros
      {
        ekvdefinekeys
      }
    ,moretexcs=[3]^^A expkv-pkg macros
      {
        ^^A expkv
        ekvdef,ekvdefNoVal,^^A
        ekvlet,ekvletNoVal,ekvletkv,ekvletkvNoVal,^^A
        ekvdefunknown,ekvdefunknownNoVal,^^A
        ekvredirectunknown,ekvredirectunknownNoVal,^^A
        ekvset,ekvsetSneaked,^^A
        ekvsetdef,ekvsetSneakeddef,ekvsetdefSneaked,^^A
        ekvparse,^^A
        ekvVersion,ekvDate,^^A
        ekvifdefined,ekvifdefinedNoVal,ekvifdefinedset,^^A
        ekvbreak,ekvbreakPreSneak,ekvbreakPostSneak,^^A
        ekvsneak,ekvsneakPre,^^A
        ekvchangeset,^^A
        ekvoptarg,ekvoptargTF,^^A
        ekverr,
        ^^A expkv-cs
        ekvcSecondaryKeys,ekvcSplitAndForward,
      }
  }
\colorlet{codeparam}{cyan!65!black}
\def\mylstwd{.55em}
\lstset
  {
    ,language=expkvd
    ,flexiblecolumns=false
    ,basewidth=\mylstwd
    ,gobble=2
    ,basicstyle=\fontfamily{jkp}\itshape
    ,texcsstyle=*[1]{\bfseries\color{expkvgrey}} ^^A primitives
    ,texcsstyle=*[2]{\bfseries\color{expkvred}}  ^^A expkv
    ,texcsstyle=*[3]{\color{expkvred}}           ^^A expkv-pkg
    ,texcsstyle=*[4]{\bfseries\color{expkvgrey}} ^^A e-TeX
    ,texcsstyle=*[5]{\bfseries\color{expkvgrey}} ^^A plain/LaTeX macros
    ,texcsstyle=*[6]{}                           ^^A macros of other packages
    ,commentstyle=\color[gray]{0.4}
    ,literate=
              {\{} {{\CodeUpBf\{}}{1}
              {\}} {{\CodeUpBf\}}}{1}
              {$}  {{\CodeUpBf\$}}{1}
              {[}  {{\CodeUp[}}{1}
              {]}  {{\CodeUp]}}{1}
              {(}  {{\CodeUp(}}{1}
              {)}  {{\CodeUp)}}{1}
              {*}  {{$*$}}{1}
              {1}  {{\CodeUp{1}}}{1}
              {2}  {{\CodeUp{2}}}{1}
              {3}  {{\CodeUp{3}}}{1}
              {4}  {{\CodeUp{4}}}{1}
              {5}  {{\CodeUp{5}}}{1}
              {6}  {{\CodeUp{6}}}{1}
              {7}  {{\CodeUp{7}}}{1}
              {8}  {{\CodeUp{8}}}{1}
              {9}  {{\CodeUp{9}}}{1}
              {0}  {{\CodeUp{0}}}{1}
              {##} {{\CodeColored{codeparam}{1}{\#}}}{1}
              {##1}{{\CodeColored{codeparam}{2}{\#1}}}{2}
              {##2}{{\CodeColored{codeparam}{2}{\#2}}}{2}
              {##3}{{\CodeColored{codeparam}{2}{\#3}}}{2}
              {##4}{{\CodeColored{codeparam}{2}{\#4}}}{2}
              {##5}{{\CodeColored{codeparam}{2}{\#5}}}{2}
              {##6}{{\CodeColored{codeparam}{2}{\#6}}}{2}
              {##7}{{\CodeColored{codeparam}{2}{\#7}}}{2}
              {##8}{{\CodeColored{codeparam}{2}{\#8}}}{2}
              {##9}{{\CodeColored{codeparam}{2}{\#9}}}{2}
              {<key>}{{$\langle$}key{$\rangle$}}{5}
              {<set>}{{$\langle$}set{$\rangle$}}{5}
              {<set1>}{{$\langle$}set1{$\rangle$}}{6}
              {<set2>}{{$\langle$}set2{$\rangle$}}{6}
    ^^A,literate=*{<key>}{\key}{4}{<set>}{\set}{4}
  }
\newcommand*\CodeColored[3]{\textcolor{#1}{\makebox[\dimexpr\mylstwd*#2]{#3}}}
\newcommand*\CodeUpBf[1]{\makebox[\mylstwd]{\textup{\textbf{#1}}}}
\newcommand*\CodeUp[1]{\makebox[\mylstwd]{\textup{#1}}}
\RequirePackage{randtext}
\let\metaORIG\meta
\protected\def\meta #1{\texttt{\metaORIG{#1}}}
\renewcommand*\thefootnote{\fnsymbol{footnote}}
\definecolor{expkvred}{HTML}{9F393D}
\colorlet{expkvgrey}{black!75}
\makeatletter
\newcommand*\expkv
  {^^A
    \texorpdfstring
      {^^A
        \mbox
          {^^A
            \BeginAccSupp{ActualText=expkv}^^A
            \href{https://github.com/Skillmon/tex_expkv}
              {^^A
                \rmfamily
                \bfseries
                {\color{expkvgrey}e\kern-.05em x\kern-.05em}^^A
                \lower.493ex
                  \hbox{{\color{expkvgrey}P}\kern-.1em{\color{expkvred}k}}^^A
                \kern-.18em{\color{expkvred}v}^^A
              }^^A
            \EndAccSupp{}^^A
          }^^A
      }
      {expkv}^^A
  }
\newcommand*\expkvd
  {^^A
    \texorpdfstring
      {^^A
        \mbox
          {^^A
            \BeginAccSupp{ActualText=expkv-def}^^A
            \href{https://github.com/Skillmon/tex_expkv-def}
              {^^A
                \rmfamily
                \bfseries
                {\color{expkvgrey}e\kern-.05em x\kern-.05em}^^A
                \lower.493ex
                  \hbox{{\color{expkvgrey}P}\kern-.1em{\color{expkvred}k}}^^A
                \kern-.18em{\color{expkvred}v}^^A
                {\color{expkvgrey}^^A
                  \kern.05em\rule[-.1ex]{.08em}{1.2ex}\kern.05em\textsc{def}^^A
                }^^A
              }^^A
            \EndAccSupp{}^^A
          }^^A
      }
      {expkv-def}^^A
  }
\newcommand\kv{\meta{key}=\meta{value}}
\newcommand\key{\meta{key}}
\newcommand\val{\meta{value}}
\newcommand\set{\meta{set}}
\newcommand\enfprefix{\textcolor{black}}
\newcommand\allprefix{\textcolor{gray}}
\newcommand\notprefix{\textcolor{red!80!black}}
\newcommand\printprefix[2]
  {^^A
    \ifcase\numexpr#1\relax\or
      \enfprefix{#2}\or
      \allprefix{#2}\or
      \notprefix{#2}\fi
  }
\newcommand\prefixes[4]
  {^^A
    \hfill
    \printprefix{#1}{new}
    \printprefix{#2}{also}
    \printprefix{#3}{protected}
    \printprefix{#4}{long}^^A
  }
\hypersetup{linkcolor=red!80!black,urlcolor=purple!80!black}
\DoNotIndex{\def,\edef,\,,\=,\begingroup,\catcode,\chardef,\csname,\endcsname}
\DoNotIndex{\endgroup,\endinput,\errmessage,\expandafter,\input,\let,\long}
\DoNotIndex{\protected,\ProvidesFile,\ProvidesPackage,\relax,\space}
\DoNotIndex{\@,\unexpanded,\string,\expanded,\dimexpr,\global,\glueexpr,\hbox}
\DoNotIndex{\numexpr,\RequirePackage,\setbox,\the,\unless,\xdef,\@firstofone}
\DoNotIndex{\@firstoftwo,\@gobble,\@secondoftwo,\meaning,\detokenize}
\DoNotIndex{\romannumeral,\^}
\DoNotIndex{\ifcsname}
\DoNotIndex{\ifx}
\DoNotIndex{\ifdefined}
\DoNotIndex{\iffalse}
\DoNotIndex{\iftrue}
\DoNotIndex{\else}
\DoNotIndex{\fi}
\@gobble\fi ^^A ignoring \ifx, \ifcsname, etc., but only one \fi
\@gobble\fi ^^A ignoring \ifx, \ifcsname, etc., but only one \fi
\@gobble\fi ^^A ignoring \ifx, \ifcsname, etc., but only one \fi
\@gobble\fi ^^A ignoring \ifx, \ifcsname, etc., but only one \fi
\@ifdefinable\gobbledocstriptag{\def\gobbledocstriptag#1>{}}
\makeatother
\begin{document}
  \title
    {^^A
      \texorpdfstring
        {^^A
          \huge\expkvd
          \\[\medskipamount]
          \Large a key-defining frontend for \expkv
        }{expkv-def - a key-defining frontend for expkv}^^A
    }
  \date{\ekvdDate\space v\ekvdVersion}
  \author{Jonathan P. Spratte\thanks{\protect\randomize{jspratte@yahoo.de}}}
  \DocInput{expkv-def.dtx}
\end{document}
%</driver>^^A=<<
% \fi
%
% \maketitle
% \renewcommand*\thefootnote{\arabic{footnote}}
%
% \begin{abstract}
% \noindent\parfillskip=0pt
% \expkvd\ provides a small \kv\ interface to define keys for \expkv. Key-types
% are declared using prefixes, similar to static typed languages. The stylised
% name is \expkvd\ but the files use \file{expkv-def}, this is due to CTAN-rules
% which don't allow \string| in package names since that is the pipe symbol in
% *nix shells.
% \end{abstract}
%
% \tableofcontents
%
% \begin{documentation}^^A>>=
%
% \section{Documentation}
%
% Since the trend for the last couple of years goes to defining keys for a \kv\
% interface using a \kv\ interface, I thought that maybe providing such an
% interface for \expkv\ will make it more attractive for actual use, besides its
% unique selling points of being fully expandable, and fast and reliable. But at
% the same time I don't want to widen \expkv's initial scope. So here it is
% \expkvd, go define \kv\ interfaces with \kv\ interfaces.
%
% Unlike many of the other established \kv\ interfaces to define keys, \expkvd\
% works using prefixes instead of suffixes (\emph{e.g.}, |.tl_set:N| of
% \pkg{l3keys}) or directory like handlers (\emph{e.g.}, |/.store in| of
% \pkg{pgfkeys}). This was decided as a personal preference, more over in \TeX\
% parsing for the first space is way easier than parsing for the last one.
% \expkvd's prefixes are sorted into two categories: |p|-type, which are
% equivalent to \TeX's prefixes like |\long|, and |t|-type defining the type of
% the key. For a description of the available |p|-prefixes take a look at
% \autoref{sec:prefix:p}, the |t|-prefixes are described in
% \autoref{sec:prefix:t}.
%
% \expkvd\ is usable as generic code, as a \LaTeX\ package, and as a \ConTeXt\
% module. It'll automatically load \expkv\ in the same mode as well. To use it,
% just use one of
% \begin{lstlisting}
% \input expkv-def       % plainTeX
% \usepackage{expkv-def} % LaTeX
% \usemodule[expkv-def]  % ConTeXt
% \end{lstlisting}
%
% \subsection{Macros}
% Apart from version and date containers there is only a single user-facing
% macro, and that should be used to define keys.
%
% \begin{function}{\ekvdefinekeys}
%   \begin{syntax}
%     \cs{ekvdefinekeys}\marg{set}\{\kv, \ldots\}
%   \end{syntax}
%   In \set, define \key\ to have definition \val. The general syntax for \key\
%   should be
%   \begin{quote}\ttfamily\small
%     \meta{prefix} \meta{name}
%   \end{quote}
%   Where \meta{prefix} is a space separated list of optional |p|-type prefixes
%   followed by one |t|-type prefix. The syntax of \val\ is dependent on the
%   used |t|-prefix.
% \end{function}
%
% \begin{function}{\ekvdDate,\ekvdVersion}
%   These two macros store the version and date of the package.
% \end{function}
%
% \subsection{Prefixes}
% As already said there are |p|-prefixes and |t|-prefixes. Not every |p|-prefix
% is allowed for all |t|-prefixes.
%
% \subsubsection{\texttt{p}-Prefixes}\label{sec:prefix:p}
% The two |p|-type prefixes |long| and |protected| are pretty simple by nature,
% so their description is pretty simple. They affect the \key\ at use-time, so
% omitting |long| doesn't mean that a \meta{definition} can't contain a |\par|
% token, only that the \key\ will not accept a |\par| in \val.  On the other
% hand |new| and |also| might be simple on first sight as well, but their rules
% are a bit more complicated.
%
% \begin{function}{also}
%   The following key type will be \emph{added} to an existing \key's
%   definition. You can't add a type taking an argument at use time to an
%   existing key which doesn't take an argument and vice versa. Also you'll get
%   an error if you try to add an action which isn't allowed to be either |long|
%   or |protected| to a key which already is |long| or |protected| (the opposite
%   order would be suboptimal as well, but can't be really captured with the
%   current code).
%
%   A key already defined as |long| or |protected| will stay |long| or
%   |protected|, but you can as well add |long| or |protected| with the |also|
%   definition.
%
%   As a small example, suppose you want to create a boolean key, but
%   additionally to setting a boolean value you want to execute some more code
%   as well, you can use the following
% \begin{lstlisting}
% \ekvdefinekeys{also-example}
%   {
%      bool key      = \ifmybool
%     ,also code key = \domystuff{#1}
%   }
% \end{lstlisting}
%
%   If you use |also| on a |choice|, |bool|, |invbool|, or |boolpair| key it is
%   tried to determine if the key already is of one of those types. If this test
%   is true the declared choices will be added to the possible choices but the
%   key's definition will not be changed other than that. If that wouldn't have
%   been done, the callbacks of the different choices could get called multiple
%   times.
% \end{function}
%
% \begin{function}{protected,protect}
%   The following key will be defined |\protected|. Note that key-types which
%   can't be defined expandable will always use |\protected|.
% \end{function}
%
% \begin{function}{long}
%   The following key will be defined |\long|.
% \end{function}
%
% \begin{function}{new}
%   The following key must be new (so previously undefined). An error is thrown
%   if it is already defined and the new definition is ignored. |new| only
%   asserts that there are no conflicts between |NoVal| keys and other |NoVal|
%   keys or value taking keys and other value taking keys. For example you can
%   use the following without an error:
% \begin{lstlisting}
% \ekvdefinekeys{new-example}
%   {
%      code key       = \domystuffwitharg{#1}
%     ,new  noval key = \domystuffwithoutarg
%   }
% \end{lstlisting}
% \end{function}
%
% \subsubsection{\texttt{t}-Prefixes}\label{sec:prefix:t}
% Since the |p|-type prefixes apply to some of the |t|-prefixes automatically
% but sometimes one might be disallowed we need some way to highlight this
% behaviour. In the following an enforced prefix will be printed black
% (\texttt{\enfprefix{protected}}), allowed prefixes will be grey
% (\texttt{\allprefix{protected}}), and disallowed prefixes will be red
% (\texttt{\notprefix{protected}}). This will be put flush-right in the syntax
% showing line.
%
% \begin{function}{code,ecode}
%   \begin{syntax}
%     code \key\ = \marg{definition} \prefixes2222
%   \end{syntax}
%   Define \key\ to expand to \meta{definition}. The \key\ will require a \val\
%   for which you can use |#1| inside \meta{definition}. The |ecode| variant
%   will fully expand \meta{definition} inside an |\edef|.
% \end{function}
%
% \begin{function}{noval,enoval}
%   \begin{syntax}
%     noval \key\ = \marg{definition} \prefixes2223
%   \end{syntax}
%   The |noval| type defines \key\ to expand to \meta{definition}. The \key\
%   will not take a \val. |enoval| fully expands \meta{definition} inside an
%   |\edef|.
% \end{function}
%
% \begin{function}{default,qdefault,odefault,fdefault,edefault}
%   \begin{syntax}
%     default \key\ = \marg{definition} \prefixes3223
%   \end{syntax}
%   This serves to place a default \val\ for a \key\ that takes an argument, the
%   \key\ can be of any argument-grabbing kind, and when used without a \val\
%   it will be passed \meta{definition} instead. The |qdefault| variant will
%   expand the \key's code once, so will be slightly quicker, but not change if
%   you redefine \key. |odefault| is just another name for |qdefault|. The
%   |fdefault| version will expand the key code until a non-expandable token or
%   a space is found, a space would be gobbled.\footnotemark{} The |edefault| on
%   the other hand fully expands the \key-code with \meta{definition} as its
%   argument inside of an |\edef|.
% \end{function}
% \footnotetext
%   {^^A
%     For those familiar with \TeX-coding: This uses a
%     \cs[no-index]{romannumeral}-expansion.
%   }
%
% \begin{function}{initial,oinitial,finitial,einitial}
%   \begin{syntax}
%     initial \key\ = \{\val\} \prefixes3333
%     initial \key
%   \end{syntax}
%   With |initial| you can set an initial \val\ for an already defined argument
%   taking \key. It'll just call the key-macro of \key\ and pass it \val. The
%   |einitial| variant will expand \val\ using an |\edef| expansion prior to
%   passing it to the key-macro and the |oinitial| variant will expand the first
%   token in \val\ once. |finitial| will expand \val\ until a non-expandable
%   token or a space is found, a space would be
%   gobbled.\footnotemark
%
%   If you don't provide a value (and no equals sign) a |noval| \key\ of the
%   same name is called once (or, if you specified a |default| for a value
%   taking key that would be used).
% \end{function}
% \footnotetext{Again using \cs[no-index]{romannumeral}.}
%
% \begin{function}{bool,gbool,boolTF,gboolTF}
%   \begin{syntax}
%     bool \key\ = \meta{cs} \prefixes2223
%   \end{syntax}
%   The \meta{cs} should be a single control sequence, such as |\iffoo|.
%   This will define \key\ to be a boolean key, which only takes the values
%   |true| or |false| and will throw an error for other values. If the key is
%   used without a \val\ it'll have the same effect as if you use
%   \texttt{\key=true}. |bool| and |gbool| will behave like \TeX-ifs so either
%   be |\iftrue| or |\iffalse|. The |boolTF| and |gboolTF| variants will both
%   take two arguments and if true the first will be used else the second, so
%   they are always either |\@firstoftwo| or |\@secondoftwo|. The variants with
%   a leading |g| will set the control sequence globally, the others locally. If
%   \meta{cs} is not yet defined it'll be initialised as the |false| version.
%   Note that the initialisation is \emph{not} done with |\newif|, so you will
%   not be able to do |\footrue| outside of the \kv\ interface, but you could
%   use |\newif| yourself. Even if the \key\ will not be |\protected| the
%   commands which execute the |true| or |false| choice will be, so the usage
%   should be safe in an expansion context (\emph{e.g.}, you can use
%   \texttt{edefault \key\ = false} without an issue to change the default
%   behaviour to execute the |false| choice). Internally a |bool| \key\ is the
%   same as a |choice| key which is set up to handle |true| and |false| as
%   choices.
% \end{function}
%
% \begin{function}{invbool,ginvbool,invboolTF,ginvboolTF}
%   \begin{syntax}
%     bool \key\ = \meta{cs} \prefixes2223
%   \end{syntax}
%   These are inverse boolean keys, they behave like |bool| and friends but set
%   the opposite meaning to the macro \meta{cs} in each case. So if
%   \texttt{key=true} is used |invbool| will set \meta{cs} to |\iffalse| and
%   vice versa.
% \end{function}
%
% \begin{function}{boolpair,gboolpair,boolpairTF,gboolpairTF}
%   \begin{syntax}
%     boolpair \key\ = \meta{cs_1}\meta{cs_2} \prefixes2223
%   \end{syntax}
%   The |boolpair| key type behaves like both |bool| and |invbool|, the
%   \meta{cs_1} will be set to the meaning according to the rules of |bool|, and
%   \meta{cs_2} will be set to the opposite.
% \end{function}
%
% \begin{function}{store,estore,gstore,xstore}
%   \begin{syntax}
%     store \key\ = \meta{cs} \prefixes2212
%   \end{syntax}
%   The \meta{cs} should be a single control sequence, such as |\foo|.
%   This will define \key\ to store \val\ inside of the control sequence. If
%   \meta{cs} isn't yet defined it will be initialised as empty. The variants
%   behave similarly to their |\def|, |\edef|, |\gdef|, and |\xdef|
%   counterparts, but |store| and |gstore| will allow you to store macro
%   parameters inside of them by using |\unexpanded|.
% \end{function}
%
% \begin{function}{data,edata,gdata,xdata}
%   \begin{syntax}
%     data \key\ = \meta{cs} \prefixes2212
%   \end{syntax}
%   The \meta{cs} should be a single control sequence, such as |\foo|. This will
%   define \key\ to store \val\ inside of the control sequence. But unlike the
%   |store| type, the macro \meta{cs} will be a switch at the same time, it'll
%   take two arguments and if \key\ was used expands to the first argument
%   followed by \val\ in braces, if \key\ was not used \meta{cs} will expand to
%   the second argument (so behave like |\@secondoftwo|). The idea is that with
%   this type you can define a key which should be typeset formatted. The
%   |edata| and |xdata| variants will fully expand \val, the |gdata| and |xdata|
%   variants will store \val\ inside \meta{cs} globally. The |p|-prefixes will
%   only affect the key-macro, \meta{cs} will always be expandable and |\long|.
% \end{function}
%
% \begin{function}{dataT,edataT,gdataT,xdataT}
%   \begin{syntax}
%     dataT \key\ = \meta{cs} \prefixes2212
%   \end{syntax}
%   Just like |data|, but instead of \meta{cs} grabbing two arguments it'll only
%   grab one, so by default it'll behave like |\@gobble|, and if a \val\ was
%   given to \key\ the \meta{cs} will behave like |\@firstofone| appended by
%   \marg{value}.
% \end{function}
%
% \begin{function}{int,eint,gint,xint}
%   \begin{syntax}
%     int \key\ = \meta{cs} \prefixes2212
%   \end{syntax}
%   The \meta{cs} should be a single control sequence, such as |\foo|.
%   An |int| key will be a \TeX-count register. If \meta{cs} isn't defined yet,
%   |\newcount| will be used to initialise it. The |eint| and |xint| versions
%   will use |\numexpr| to allow basic computations in their \val. The |gint|
%   and |xint| variants set the register globally.
% \end{function}
%
% \begin{function}{dimen,edimen,gdimen,xdimen}
%   \begin{syntax}
%     dimen \key\ = \meta{cs} \prefixes2212
%   \end{syntax}
%   The \meta{cs} should be a single control sequence, such as |\foo|.
%   This is just like |int| but uses a dimen register, |\newdimen| and
%   |\dimexpr| instead.
% \end{function}
%
% \begin{function}{skip,eskip,gskip,xskip}
%   \begin{syntax}
%     skip \key\ = \meta{cs} \prefixes2212
%   \end{syntax}
%   The \meta{cs} should be a single control sequence, such as |\foo|.
%   This is just like |int| but uses a skip register, |\newskip| and |\glueexpr|
%   instead.
% \end{function}
%
% \begin{function}{toks,gtoks,apptoks,gapptoks,pretoks,gpretoks}
%   \begin{syntax}
%     toks \key\ = \meta{cs} \prefixes2212
%   \end{syntax}
%   The \meta{cs} should be a single control sequence, such as |\foo|.
%   Store \val\ inside of a toks-register. The |g| variants use |\global|, the
%   |app| variants append \val\ to the contents of that register, the |pre|
%   variants will prepend \val. If \meta{cs} is not yet defined it will be
%   initialised with |\newtoks|.
% \end{function}
%
% \begin{function}{box,gbox}
%   \begin{syntax}
%     box \key\ = \meta{cs} \prefixes2212
%   \end{syntax}
%   The \meta{cs} should be a single control sequence, such as |\foo|.
%   Typesets \val\ into a |\hbox| and stores the result in a box register. The
%   boxes are colour safe. \expkvd\ doesn't provide a |vbox| type.
% \end{function}
%
% \begin{function}{meta}
%   \begin{syntax}
%     meta \key\ = \{\kv, \ldots\} \prefixes2222
%   \end{syntax}
%   This key type can set other keys, you can access the \val\ which was passed
%   to \key\ inside the \kv\ list with |#1|. It works by calling a sub-|\ekvset|
%   on the \kv\ list, so a |set| key will only affect that \kv\ list and not the
%   current |\ekvset|. Since it runs in a separate |\ekvset| you can't use
%   |\ekvsneak| using keys or similar macros in the way you normally could.
% \end{function}
%
% \begin{function}{nmeta}
%   \begin{syntax}
%     nmeta \key\ = \{\kv, \ldots\} \prefixes2223
%   \end{syntax}
%   This key type can set other keys, the difference to |meta| is, that this key
%   doesn't take a value, so the \kv\ list is static.
% \end{function}
%
% \begin{function}{smeta}
%   \begin{syntax}
%     smeta \key\ = \marg{set}\{\kv, \ldots\} \prefixes2222
%   \end{syntax}
%   Yet another |meta| variant. An |smeta| key will take a \val\ which you can
%   access using |#1|, but it sets the \kv\ list inside of \set, so is equal to
%   \texttt{\cs[no-index]{ekvset}\marg{set}\{\kv, \ldots\}}.
% \end{function}
%
% \begin{function}{snmeta}
%   \begin{syntax}
%     snmeta \key\ = \marg{set}\{\kv, \ldots\} \prefixes2223
%   \end{syntax}
%   And the last |meta| variant. |snmeta| is a combination of |smeta| and
%   |nmeta|. It doesn't take an argument and sets the \kv\ list inside of \set.
% \end{function}
%
% \begin{function}{set}
%   \begin{syntax}
%     set \key\ = \marg{set} \prefixes2233
%   \end{syntax}
%   This will define \key\ to change the set of the current |\ekvset|
%   invocation to \set. You can omit \set\ (including the equals sign),
%   which is the same as using \texttt{set \key\ = \{\key\}}. The created |set|
%   key will not take a \val. Note that just like in \expkv\ it'll not be
%   checked whether \set\ is defined and you'll get a low-level \TeX\ error if
%   you use an undefined \set.
% \end{function}
%
% \begin{function}{choice}
%   \begin{syntax}
%     choice \key\ = \{\val=\meta{definition}, \ldots\} \prefixes2223
%   \end{syntax}
%   Defines \key\ to be a |choice| key, meaning it will only accept a limited
%   set of values. You should define each possible \val\ inside of the
%   \val=\meta{definition} list. If a defined \val\ is passed to \key\ the
%   \meta{definition} will be left in the input stream. You can make
%   individual values |protected| inside the \val=\meta{definition} list. By
%   default a |choice| key is expandable, an undefined \val\ will throw an error
%   in an expandable way (but see the |unknown-choice| prefix). You can add
%   additional choices after the \key\ was created by using |choice| again for
%   the same \key, redefining choices is possible the same way, but there is no
%   interface to remove certain choices.
% \end{function}
%
% \begin{function}{choice-store}
%   \begin{syntax}
%     choice-store \key\ = \meta{cs}\{\val, \ldots\} \prefixes2223
%   \end{syntax}
%   This is defines a special type of |choice| key that'll store the given
%   choice inside the macro \meta{cs} (so \meta{cs} should be a single control
%   sequence name such as |\foo|). Since storing inside a macro can't be done
%   expandably every choice-code is |\protected|, you might define the
%   |choice-store| key itself as |protected| as well if you want. Since the
%   definition of each choice is predefined with this key type the choice list
%   should just be a comma separated list of valid choices.
% \end{function}
% This means that the following |choice| and |choice-store| keys are equivalent
% at use time:
% \begin{lstlisting}
% \newcommand*\mya{}
% \ekvdefinekeys{example}
%   {
%      choice key1 = {a=\def\mya{a}, b=\def\mya{b}, c=\def\mya{c}}
%     ,choice-store key2 = \mya{a,b,c}
%   }
% \end{lstlisting}
%
% \begin{function}{choice-enum}
%   \begin{syntax}
%     choice-enum \key\ = \meta{cs}\{\val, \ldots\} \prefixes2223
%   \end{syntax}
%   This is similar to |choice-store|, the differences are: \meta{cs} should be
%   a count-register or is initialised as such if the \meta{cs} is undefined
%   (via |\newcount|); instead of the value the position of the value in the
%   given list is stored in this register (zero-based).
% \end{function}
% This means that the following |choice| and |choice-enum| keys are equivalent
% at use time:
% \begin{lstlisting}
% \newcount\myb
% \ekvdefinekeys{example}
%   {
%      choice key1 = {a={\myb=0 }, b={\myb=1 }, c={\myb=2 }}
%     ,choice-enum key2 = \myb{a,b,c}
%   }
% \end{lstlisting}
%
% \begin{function}{unknown-choice}
%   \begin{syntax}
%     unknown-choice \key\ = \marg{definition} \prefixes2323
%   \end{syntax}
%   By default an unknown \val\ passed to a |choice| or |bool| key will throw an
%   error.  However, with this prefix you can define an alternative action which
%   should be executed if \key\ received an unknown choice. In \meta{definition}
%   you can refer to the choice which was passed in with |#1|.
% \end{function}
%
% \begin{function}{unknown code}
%   \begin{syntax}
%     unknown code = \marg{definition} \prefixes2322
%   \end{syntax}
%   By default \expkv\ throws errors when it encounters unknown keys in a set.
%   With the |unknown| prefix you can define handlers that deal with undefined
%   keys, instead of a \key\ name you have to specify a subtype for this prefix,
%   here the subtype is |code|.\par
%   With |unknown code| the \meta{definition} is used for unknown keys which
%   were provided a value (so corresponds to |\ekvdefunknown|), you can access
%   the key name with |#1| and the value with |#2|.\footnotemark
% \end{function}
% \footnotetext
%   {^^A
%     There is some trickery involved to get this more intuitive argument order
%     without any performance hit if you compare this to
%     \cs[no-index]{ekvdefunknown} directly.^^A
%   }
%
% \begin{function}{unknown noval}
%   \begin{syntax}
%     unknown noval = \marg{definition} \prefixes2323
%   \end{syntax}
%   This is like |unknown code| but uses \meta{definition} for unknown keys to
%   which no value was passed (so corresponds to |\ekvdefunknownNoVal|). You can
%   access the key name with |#1|.
% \end{function}
%
% \begin{function}{unknown redirect-code}
%   \begin{syntax}
%     unknown redirect-code = \marg{set-list} \prefixes2331
%   \end{syntax}
%   This uses a predefined action for |unknown code|. Instead of throwing an
%   error, it is tried to find the \key\ in each \set\ in the comma separated
%   \meta{set-list}. The first found match will be used and the remaining
%   options from the list discarded. If the \key\ isn't found in any \set\ an
%   expandable error will be thrown eventually. Internally \expkv's
%   |\ekvredirectunknown| will be used.
% \end{function}
%
% \begin{function}{unknown redirect-noval}
%   \begin{syntax}
%     unknown redirect-noval = \marg{set-list} \prefixes2333
%   \end{syntax}
%   This behaves just like |unknown redirect-code| but will set up means to
%   forward keys for |unknown noval|. Internally \expkv's
%   |\ekvredirectunknownNoVal| will be used.
% \end{function}
%
% \begin{function}{unknown redirect}
%   \begin{syntax}
%     unknown redirect = \marg{set-list} \prefixes2333
%   \end{syntax}
%   This is a short cut to apply both, |unknown redirect-code| and
%   |unknown redirect-noval|, as a result you might get doubled error messages,
%   one from each.
% \end{function}
%
% \subsection{Bugs}
% I don't think there are any (but every developer says that), if you find some
% please let me know, either via the email address on the first page or on
% GitHub:
% \url{https://github.com/Skillmon/tex_expkv-def}
%
% \subsection{Example}
% The following is an example code defining each base key-type once. Please
% admire the very creative key-name examples.
%
% \begin{lstlisting}
% \ekvdefinekeys{example}
%   {
%      long code keyA = #1
%     ,noval     keyA = NoVal given
%     ,bool      keyB = \keyB
%     ,boolTF    keyC = \keyC
%     ,store     keyD = \keyD
%     ,data      keyE = \keyE
%     ,dataT     keyF = \keyF
%     ,int       keyG = \keyG
%     ,dimen     keyH = \keyH
%     ,skip      keyI = \keyI
%     ,toks      keyJ = \keyJ
%     ,default   keyJ = \empty test
%     ,new box   keyK = \keyK
%     ,qdefault  keyK = K
%     ,choice    keyL =
%       {
%          protected 1 = \texttt{a}
%         ,2 = b
%         ,3 = c
%         ,4 = d
%         ,5 = e
%       }
%     ,edefault  keyL = 2
%     ,meta      keyM = {keyA={#1},keyB=false}
%     ,invbool   keyN = \keyN
%     ,boolpair  keyO = \keyOa\keyOb
%   }
% \end{lstlisting}
%
% \bigskip
% Since the |data| type might be a bit strange, here is another usage example
% for it.
% \begin{lstlisting}
% \ekvdefinekeys{ex}
%   {
%      data  name  = \Pname
%     ,data  age   = \Page
%     ,dataT hobby = \Phobby
%   }
% \newcommand\Person[1]
%   {%
%     \begingroup
%     \ekvset{ex}{#1}%
%     \begin{description}
%       \item[\Pname{}{\errmessage{A person requires a name}}]
%       \item[Age] \Page{\textit}{\errmessage{A person requires an age}}
%       \Phobby{\item[Hobbies]}
%     \end{description}
%     \endgroup
%   }
% \Person{name=Jonathan P. Spratte, age=young, hobby=\TeX\ coding}
% \Person{name=Some User, age=unknown, hobby=Reading Documentation}
% \Person{name=Anybody, age=any}
% \end{lstlisting}
% In this example a person should have a name and an age, but doesn't have to
% have hobbies. The name will be displayed as the description item and the age
% in Italics. If a person has no hobbies the description item will be silently
% left out.  The result of the above code looks like this:
%
% \begin{exresult}{\linewidth}
% \ekvdefinekeys{ex}
%   {
%      data  name  = \Pname
%     ,data  age   = \Page
%     ,dataT hobby = \Phobby
%   }
% \newcommand\Person[1]
%   {%
%     \begingroup
%     \ekvset{ex}{#1}%
%     \begin{description}
%       \item[\Pname{}{\errmessage{A person requires a name}}]
%       \item[Age] \Page{\textit}{\errmessage{A person requires an age}}
%       \Phobby{\item[Hobbies]}
%     \end{description}
%     \endgroup
%   }
% \Person{name=Jonathan P. Spratte, age=young, hobby=\TeX\ coding}
% \Person{name=Some User, age=unknown, hobby=Reading Documentation}
% \Person{name=Anybody, age=any}
% \end{exresult}
%
% \subsection{License}
%
% Copyright \textcopyright\ 2020\unless\ifnum\year=2020--\the\year\fi\
% Jonathan P. Spratte
%
% \medskip\noindent
% This work may be distributed and/or modified under the conditions of the
% \LaTeX\ Project Public License (LPPL), either version 1.3c of this license or
% (at your option) any later version. The latest version of this license is in
% the file:
%
% \url{http://www.latex-project.org/lppl.txt}
%
% \noindent
% This work is ``maintained'' (as per LPPL maintenance status) by
%
% \mbox{Jonathan P. Spratte}.
%
% \end{documentation}^^A=<<
%
% \begin{implementation}^^A>>=
%
% \clearpage
%
% \section{Implementation}^^A>>=
%^^A the LaTeX package >>=
% \subsection{The \LaTeX\ Package}
% Just like for \expkv\ we provide a small \LaTeX\ package that sets up things
% such that we behave nicely on \LaTeX\ packages and files system. It'll
% |\input| the generic code which implements the functionality.
% \gobbledocstriptag
%<*pkg>
%    \begin{macrocode}
\RequirePackage{expkv}
\def\ekvd@tmp
  {%
    \ProvidesFile{expkv-def.tex}%
      [\ekvdDate\space v\ekvdVersion\space a key-defining frontend for expkv]%
  }
\input{expkv-def.tex}
\ProvidesPackage{expkv-def}%
  [\ekvdDate\space v\ekvdVersion\space a key-defining frontend for expkv]
%    \end{macrocode}
% \gobbledocstriptag
%</pkg>
%^^A=<<
%^^A the ConTeXt module >>=
% \subsection{The \ConTeXt\ module}
% \gobbledocstriptag
%<*ctx>
%    \begin{macrocode}
\writestatus{loading}{ConTeXt User Module / expkv-def}
\usemodule[expkv]
\unprotect
\input expkv-def.tex
\writestatus{loading}
  {ConTeXt User Module / expkv-def / Version \ekvdVersion\space loaded}
\protect\endinput
%    \end{macrocode}
% \gobbledocstriptag
%</ctx>
%^^A=<<
%^^A main file >>=
% \subsection{The Generic Code}
% The rest of this implementation will be the generic code.
% \gobbledocstriptag
%<*tex>
%
% Load \expkv\ if the package didn't already do so -- since \expkv\ has
% safeguards against being loaded twice this does no harm and the overhead
% isn't that big. Also we reuse some of the internals of \expkv\ to save us from
% retyping them.
%    \begin{macrocode}
\input expkv
%    \end{macrocode}
%
% We make sure that \file{expkv-def.tex} is only input once:
%    \begin{macrocode}
\expandafter\ifx\csname ekvdVersion\endcsname\relax
\else
  \expandafter\endinput
\fi
%    \end{macrocode}
%
% \begin{macro}{\ekvdVersion,\ekvdDate}
% We're on our first input, so lets store the version and date in a macro.
%    \begin{macrocode}
\def\ekvdVersion{0.8c}
\def\ekvdDate{2021-09-20}
%    \end{macrocode}
% \end{macro}
%
% If the \LaTeX\ format is loaded we want to be a good file and report back who
% we are, for this the package will have defined |\ekvd@tmp| to use
% |\ProvidesFile|, else this will expand to a |\relax| and do no harm.
%    \begin{macrocode}
\csname ekvd@tmp\endcsname
%    \end{macrocode}
%
% Store the category code of |@| to later be able to reset it and change it to
% 11 for now.
%    \begin{macrocode}
\expandafter\chardef\csname ekvd@tmp\endcsname=\catcode`\@
\catcode`\@=11
%    \end{macrocode}
% |\ekvd@tmp| will be reused later to handle expansion during the key
% defining. But we don't need it to ever store information long-term after
% \expkvd\ was initialized.
%
% \begin{macro}[internal]{\ekvd@ifprimitive}
%    \begin{macrocode}
\protected\long\def\ekvd@ifprimitive#1%
  {%
    \begingroup
      \edef\ekvd@tmpa{\string #1}%
      \edef\ekvd@tmpb{\meaning#1}%
      \expandafter
    \endgroup
    \ifx\ekvd@tmpa\ekvd@tmpb
      \ekv@fi@firstoftwo
    \fi
    \@secondoftwo
  }
%    \end{macrocode}
% \end{macro}
%
% \begin{macro}[internal]
%   {\ekvd@long,\ekvd@prot,\ekvd@clear@prefixes,\ekvd@ifalso}
% \expkvd\ will use |\ekvd@long|, |\ekvd@prot|, and |\ekvd@ifalso| to store
% whether a key should be defined as |\long| or |\protected| or adds an action
% to an existing key, and we have to clear them for every new key. By default
% |long| and |protected| will just be empty, |ifalso| will be
% |\@secondoftwo|, and |ifnew| will just use its third argument.
%    \begin{macrocode}
\protected\def\ekvd@clear@prefixes
  {%
    \let\ekvd@long\ekv@empty
    \let\ekvd@prot\ekv@empty
    \let\ekvd@ifalso\@secondoftwo
    \long\def\ekvd@ifnew##1##2##3{##3}%
  }
\ekvd@clear@prefixes
%    \end{macrocode}
% \end{macro}
%
% \begin{macro}{\ekvdefinekeys}
% This is the one front-facing macro which provides the interface to define
% keys. It's using |\ekvparse| to handle the \kv\ list, the interpretation will
% be done by |\ekvd@noarg| and |\ekvd@|. The \set\ for which the keys should be
% defined is stored in |\ekvd@set|.
%    \begin{macrocode}
\protected\def\ekvdefinekeys#1%
  {%
    \def\ekvd@set{#1}%
    \ekvparse\ekvd@noarg\ekvd@arg
  }
%    \end{macrocode}
% \end{macro}
%
% \begin{macro}[internal]{\ekvd@noarg,\ekvd@arg,\ekvd@handle}
% |\ekvd@noarg| and |\ekvd@arg| store whether there was a value in the \kv\
% pair. |\ekvd@handle| has to test whether there is a space inside the key and
% if so calls the prefix grabbing routine, else we throw an error and ignore the
% key.
%    \begin{macrocode}
\protected\long\def\ekvd@noarg#1%
  {%
    \let\ekvd@ifnoarg\@firstoftwo
    \expandafter\ekvd@handle\detokenize{#1}\ekvd@stop{}%
  }
\protected\long\def\ekvd@arg#1%
  {%
    \let\ekvd@ifnoarg\@secondoftwo
    \expandafter\ekvd@handle\detokenize{#1}\ekvd@stop
  }
\protected\long\def\ekvd@handle#1\ekvd@stop#2%
  {%
    \ekvd@clear@prefixes
    \def\ekvd@cur{#1}%
    \ekvd@ifspace{#1}%
      {\ekvd@prefix\ekv@mark#1\ekv@stop{#2}}%
      \ekvd@err@missing@type
  }
%    \end{macrocode}
% \end{macro}
%
% \begin{macro}[internal]{\ekvd@prefix,\ekvd@prefix@}
% \expkvd\ separates prefixes into two groups, the first being prefixes in the
% \TeX\ sense (|long| and |protected|) which use |@p@| in their name, the other
% being key-types (|code|, |int|, \emph{etc.}) which use |@t@| instead.
% |\ekvd@prefix| splits at the first space and checks whether its a |@p@|
% or |@t@| type prefix. If it is neither throw an error and gobble the
% definition (the value).
%    \begin{macrocode}
\protected\def\ekvd@prefix#1 {\ekv@strip{#1}\ekvd@prefix@\ekv@mark}
\protected\def\ekvd@prefix@#1#2\ekv@stop
  {%
    \ekv@ifdefined{ekvd@t@#1}%
      {\ekv@strip{#2}{\csname ekvd@t@#1\endcsname}}%
      {%
        \ekv@ifdefined{ekvd@p@#1}%
          {\csname ekvd@p@#1\endcsname\ekvd@prefix@after@p{#2}}%
          {\ekvd@err@undefined@prefix{#1}\@gobble}%
      }%
  }
%    \end{macrocode}
% \end{macro}
%
% \begin{macro}[internal]{\ekvd@prefix@after@p}
% The |@p@| type prefixes are all just modifying a following |@t@| type, so they
% will need to search for another prefix. This is true for all of them, so we
% use a macro to handle this. It'll throw an error if there is no other prefix.
%    \begin{macrocode}
\protected\def\ekvd@prefix@after@p#1%
  {%
    \ekvd@ifspace{#1}%
      {\ekvd@prefix#1\ekv@stop}%
      {\ekvd@err@missing@type\@gobble}%
  }
%    \end{macrocode}
% \end{macro}
%
% \begin{macro}[internal]
%   {\ekvd@p@long,\ekvd@p@protected,\ekvd@p@protect,\ekvd@p@also,\ekvd@p@new}
% Define the |@p@| type prefixes, they all just store some information in a
% temporary macro.
%    \begin{macrocode}
\protected\def\ekvd@p@long{\let\ekvd@long\long}
\protected\def\ekvd@p@protected{\let\ekvd@prot\protected}
\let\ekvd@p@protect\ekvd@p@protected
\protected\def\ekvd@p@also{\let\ekvd@ifalso\@firstoftwo}
\protected\def\ekvd@p@new{\let\ekvd@ifnew\ekvd@assert@new}
%    \end{macrocode}
% \end{macro}
%
% \subsubsection{Key Types}
%
% \begin{macro}[internal]{\ekvd@type@set,\ekvd@t@set}
% The |set| type is quite straight forward, just define a |NoVal| key to call
% |\ekvchangeset|.
%    \begin{macrocode}
\protected\def\ekvd@type@set#1#2%
  {%
    \ekvd@assert@not@long
    \ekvd@assert@not@protected
    \ekvd@ifnew{NoVal}{#1}%
      {%
        \ekv@ifempty{#2}%
          {\ekvd@err@missing@definition}%
          {%
            \ekvd@ifalso
              {%
                \ekv@expargtwice{\ekvd@add@noval{#1}}%
                  {\ekvchangeset{#2}}%
                  \ekvd@assert@not@protected@also
              }%
              {\ekv@expargtwice{\ekvdefNoVal\ekvd@set{#1}}{\ekvchangeset{#2}}}%
          }%
      }%
  }
\protected\def\ekvd@t@set#1#2%
  {%
    \ekvd@ifnoarg
      {\ekvd@type@set{#1}{#1}}%
      {\ekvd@type@set{#1}{#2}}%
  }
%    \end{macrocode}
% \end{macro}
%
% \begin{macro}[internal]{\ekvd@type@noval,\ekvd@t@noval,\ekvd@t@enoval}
% Another pretty simple type, |noval| just needs to assert that there is a
% definition and that |long| wasn't specified.
% There are types where the difference in the variants is so small, that we
% define a common handler for them, those common handlers are named with
% |@type@|. |noval| and |enoval| are so similar that we can use such a |@type@|
% macro, even if we could've done |noval| in a slightly faster way without it.
%    \begin{macrocode}
\protected\long\def\ekvd@type@noval#1#2#3%
  {%
    \ekvd@ifnew{NoVal}{#2}%
      {%
        \ekvd@assert@arg
          {%
            \ekvd@assert@not@long
            \ekvd@prot#1\ekvd@tmp{#3}%
            \ekvd@ifalso
              {\ekv@exparg{\ekvd@add@noval{#2}}\ekvd@tmp{}}%
              {\ekvletNoVal\ekvd@set{#2}\ekvd@tmp}%
          }%
      }%
  }
\protected\def\ekvd@t@noval{\ekvd@type@noval\def}
\protected\def\ekvd@t@enoval{\ekvd@type@noval\edef}
%    \end{macrocode}
% \end{macro}
%
% \begin{macro}[internal]{\ekvd@type@code,\ekvd@t@code,\ekvd@t@ecode}
% |code| is simple as well, |ecode| has to use |\edef| on a temporary macro,
% since \expkv\ doesn't provide an |\ekvedef|.
%    \begin{macrocode}
\protected\long\def\ekvd@type@code#1#2#3%
  {%
    \ekvd@ifnew{}{#2}%
      {%
        \ekvd@assert@arg
          {%
            \ekvd@prot\ekvd@long#1\ekvd@tmp##1{#3}%
            \ekvd@ifalso
              {\ekv@exparg{\ekvd@add@val{#2}}{\ekvd@tmp{##1}}{}}%
              {\ekvlet\ekvd@set{#2}\ekvd@tmp}%
          }%
      }%
  }
\protected\def\ekvd@t@code{\ekvd@type@code\def}
\protected\def\ekvd@t@ecode{\ekvd@type@code\edef}
%    \end{macrocode}
% \end{macro}
%
% \begin{macro}[internal]
%   {
%     \ekvd@type@default,\ekvd@t@default,\ekvd@t@qdefault,\ekvd@t@odefault,
%     \ekvd@t@fdefault
%   }
% |\ekvd@type@default| asserts there was an argument, also the key for which one
% wants to set a default has to be already defined (this is not so important for
% |default|, but |qdefault| requires is). If everything is good, |\edef| a
% temporary macro that expands |\ekvd@set| and the |\csname| for the key, and in
% the case of |qdefault| does the first expansion step of the key-macro.
%    \begin{macrocode}
\protected\long\def\ekvd@type@default#1#2#3#4%
  {%
    \ekvd@assert@arg
      {%
        \ekvifdefined\ekvd@set{#3}%
          {%
            \ekvd@assert@not@new
            \ekvd@assert@not@long
            \ekvd@prot\edef\ekvd@tmp
              {%
                \ekv@unexpanded\expandafter#1%
                  {#2\csname\ekv@name\ekvd@set{#3}\endcsname{#4}}%
              }%
            \ekvd@ifalso
              {\ekv@exparg{\ekvd@add@noval{#3}}\ekvd@tmp{}}%
              {\ekvletNoVal\ekvd@set{#3}\ekvd@tmp}%
          }%
          {\ekvd@err@undefined@key{#3}}%
      }%
  }
\protected\def\ekvd@t@default{\ekvd@type@default{}{}}
\protected\def\ekvd@t@qdefault{\ekvd@type@default{\expandafter\expandafter}{}}
\let\ekvd@t@odefault\ekvd@t@qdefault
\protected\def\ekvd@t@fdefault{\ekvd@type@default{}{\romannumeral`\^^@}}
%    \end{macrocode}
% \end{macro}
%
% \begin{macro}[internal]{\ekvd@t@edefault}
% |edefault| is too different from |default| and |qdefault| to reuse the
% |@type@| macro, as it doesn't need |\unexpanded| inside of |\edef|.
%    \begin{macrocode}
\protected\long\def\ekvd@t@edefault#1#2%
  {%
    \ekvd@assert@arg
      {%
        \ekvifdefined\ekvd@set{#1}%
          {%
            \ekvd@assert@not@new
            \ekvd@assert@not@long
            \ekvd@prot\edef\ekvd@tmp
              {\csname\ekv@name\ekvd@set{#1}\endcsname{#2}}%
            \ekvd@ifalso
              {\ekv@exparg{\ekvd@add@noval{#1}}\ekvd@tmp{}}%
              {\ekvletNoVal\ekvd@set{#1}\ekvd@tmp}%
          }%
          {\ekvd@err@undefined@key{#1}}%
      }%
  }
%    \end{macrocode}
% \end{macro}
%
% \begin{macro}[internal]
%   {\ekvd@t@initial,\ekvd@t@oinitial,\ekvd@t@finitial,\ekvd@t@einitial}
%    \begin{macrocode}
\long\def\ekvd@type@initial#1#2#3#4%
  {%
    \ekvd@assert@not@new
    \ekvd@assert@not@also
    \ekvd@assert@not@long
    \ekvd@assert@not@protected
    \ekvd@ifnoarg
      {%
        \ekvifdefinedNoVal\ekvd@set{#3}%
          {\csname\ekv@name\ekvd@set{#3}N\endcsname}%
          {\ekvd@err@undefined@noval{#3}}%
      }%
      {%
        \ekvifdefined\ekvd@set{#3}%
          {%
            #1{#2#4}%
            \csname\ekv@name\ekvd@set{#3}\expandafter\endcsname\expandafter
              {\ekvd@tmp}%
          }%
          {\ekvd@err@undefined@key{#3}}%
      }%
  }
\def\ekvd@t@initial{\ekvd@type@initial{\def\ekvd@tmp}{}}
\def\ekvd@t@oinitial{\ekvd@type@initial{\ekv@exparg{\def\ekvd@tmp}}{}}
\def\ekvd@t@einitial{\ekvd@type@initial{\edef\ekvd@tmp}{}}
\def\ekvd@t@finitial
  {\ekvd@type@initial{\ekv@exparg{\def\ekvd@tmp}}{\romannumeral`\^^@}}
%    \end{macrocode}
% \end{macro}
%
% \begin{macro}[internal]
%   {
%     \ekvd@type@bool,\ekvd@t@bool,\ekvd@t@gbool,\ekvd@t@boolTF,\ekvd@t@gboolTF,
%     \ekvd@t@invbool,\ekvd@t@ginvbool,\ekvd@t@invboolTF,\ekvd@t@ginvboolTF,
%   }
% The boolean types are a quicker version of a |choice| that accept |true| and
% |false|, and set up the |NoVal| action to be identical to \texttt{\key=true}.
% The |true| and |false| actions are always just |\let|ting the macro in |#7| to
% some other macro (\emph{e.g.}, \cs[no-index]{iftrue}).
%    \begin{macrocode}
\protected\def\ekvd@type@bool#1#2#3#4#5%
  {%
    \ekvd@ifnew{}{#4}%
      {%
        \ekvd@ifnew{NoVal}{#4}%
          {%
            \ekvd@assert@filledarg{#5}%
              {%
                \ekvd@newlet#5#3%
                \ekvd@type@choice{#4}%
                \protected\ekvdefNoVal\ekvd@set{#4}{#1\let#5#2}%
                \protected\expandafter\def
                  \csname\ekvd@choice@name\ekvd@set{#4}{true}\endcsname
                  {#1\let#5#2}%
                \protected\expandafter\def
                  \csname\ekvd@choice@name\ekvd@set{#4}{false}\endcsname
                  {#1\let#5#3}%
              }%
          }%
      }%
  }
\protected\def\ekvd@t@bool{\ekvd@type@bool{}\iftrue\iffalse}
\protected\def\ekvd@t@gbool{\ekvd@type@bool\global\iftrue\iffalse}
\protected\def\ekvd@t@boolTF{\ekvd@type@bool{}\@firstoftwo\@secondoftwo}
\protected\def\ekvd@t@gboolTF{\ekvd@type@bool\global\@firstoftwo\@secondoftwo}
\protected\def\ekvd@t@invbool{\ekvd@type@bool{}\iffalse\iftrue}
\protected\def\ekvd@t@ginvbool{\ekvd@type@bool\global\iffalse\iftrue}
\protected\def\ekvd@t@invboolTF{\ekvd@type@bool{}\@secondoftwo\@firstoftwo}
\protected\def\ekvd@t@ginvboolTF
  {\ekvd@type@bool\global\@secondoftwo\@firstoftwo}
%    \end{macrocode}
% \end{macro}
%
% \begin{macro}[internal]
%   {
%     \ekvd@type@boolpair,
%     \ekvd@t@boolpair,\ekvd@t@gboolpair,\ekvd@t@boolpairTF,\ekvd@t@gboolpairTF
%   }
% The boolean pair types are essentially the same as the boolean types, but set
% two macros instead of one.
%    \begin{macrocode}
\protected\def\ekvd@type@boolpair#1#2#3#4#5#6%
  {%
    \ekvd@ifnew{}{#4}%
      {%
        \ekvd@ifnew{NoVal}{#4}%
          {%
            \ekvd@newlet#5#3%
            \ekvd@newlet#6#2%
            \ekvd@type@choice{#4}%
            \protected\ekvdefNoVal\ekvd@set{#4}{#1\let#5#2#1\let#6#3}%
            \protected\expandafter\def
              \csname\ekvd@choice@name\ekvd@set{#4}{true}\endcsname
              {#1\let#5#2#1\let#6#3}%
            \protected\expandafter\def
              \csname\ekvd@choice@name\ekvd@set{#4}{false}\endcsname
              {#1\let#5#3#1\let#6#2}%
          }%
      }%
  }
\protected\def\ekvd@t@boolpair#1#2%
  {\ekvd@assert@twoargs{#2}{\ekvd@type@boolpair{}\iftrue\iffalse{#1}#2}}
\protected\def\ekvd@t@gboolpair#1#2%
  {\ekvd@assert@twoargs{#2}{\ekvd@type@boolpair\global\iftrue\iffalse{#1}#2}}
\protected\def\ekvd@t@boolpairTF#1#2%
  {%
    \ekvd@assert@twoargs{#2}%
      {\ekvd@type@boolpair{}\@firstoftwo\@secondoftwo{#1}#2}%
  }
\protected\def\ekvd@t@gboolpairTF#1#2%
  {%
    \ekvd@assert@twoargs{#2}%
      {\ekvd@type@boolpair\global\@firstoftwo\@secondoftwo{#1}#2}%
  }
%    \end{macrocode}
% \end{macro}
%
% \begin{macro}[internal]
%   {
%     \ekvd@type@data,\ekvd@t@data,\ekvd@t@gdata,\ekvd@t@dataT,\ekvd@t@gdataT
%   }
%    \begin{macrocode}
\protected\def\ekvd@type@data#1#2#3#4#5#6%
  {%
    \ekvd@ifnew{}{#5}%
      {%
        \ekvd@assert@filledarg{#6}%
          {%
            \ekvd@newlet#6#1%
            \ekvd@ifalso
              {%
                \let\ekvd@prot\protected
                \ekvd@add@val{#5}{\long#2#6####1#3{####1{#4}}}{}%
              }%
              {%
                \protected\ekvd@long\ekvdef\ekvd@set{#5}%
                  {\long#2#6####1#3{####1{#4}}}%
              }%
          }%
      }%
  }
\protected\def\ekvd@t@data
  {\ekvd@type@data\@secondoftwo\edef{####2}{\ekv@unexpanded{##1}}}
\protected\def\ekvd@t@edata{\ekvd@type@data\@secondoftwo\edef{####2}{##1}}
\protected\def\ekvd@t@gdata
  {\ekvd@type@data\@secondoftwo\xdef{####2}{\ekv@unexpanded{##1}}}
\protected\def\ekvd@t@xdata{\ekvd@type@data\@secondoftwo\xdef{####2}{##1}}
\protected\def\ekvd@t@dataT
  {\ekvd@type@data\@gobble\edef{}{\ekv@unexpanded{##1}}}
\protected\def\ekvd@t@edataT{\ekvd@type@data\@gobble\edef{}{##1}}
\protected\def\ekvd@t@gdataT
  {\ekvd@type@data\@gobble\xdef{}{\ekv@unexpanded{##1}}}
\protected\def\ekvd@t@xdataT{\ekvd@type@data\@gobble\xdef{}{##1}}
%    \end{macrocode}
% \end{macro}
%
% \begin{macro}[internal]{\ekvd@type@box,\ekvd@t@box,\ekvd@t@gbox}
% Set up our boxes. Though we're a generic package we want to be colour safe, so
% we put an additional grouping level inside the box contents, for the case that
% someone uses \pkg{color}. |\ekvd@newreg| is a small wrapper which tests
% whether the first argument is defined and if not does
% |\csname new#2\endcsname#1|.
%    \begin{macrocode}
\protected\def\ekvd@type@box#1#2#3%
  {%
    \ekvd@ifnew{}{#2}%
      {%
        \ekvd@assert@filledarg{#3}%
          {%
            \ekvd@newreg#3{box}%
            \ekvd@ifalso
              {%
                \let\ekvd@prot\protected
                \ekvd@add@val{#2}{#1\setbox#3=\hbox{\begingroup##1\endgroup}}{}%
              }%
              {%
                \protected\ekvd@long\ekvdef\ekvd@set{#2}%
                  {#1\setbox#3=\hbox{\begingroup##1\endgroup}}%
              }%
          }%
      }%
  }
\protected\def\ekvd@t@box{\ekvd@type@box{}}
\protected\def\ekvd@t@gbox{\ekvd@type@box\global}
%    \end{macrocode}
% \end{macro}
%
% \begin{macro}[internal]{\ekvd@type@toks,\ekvd@t@toks,\ekvd@t@gtoks}
% Similar to |box|, but set the |toks|.
%    \begin{macrocode}
\protected\def\ekvd@type@toks#1#2#3%
  {%
    \ekvd@ifnew{}{#2}%
      {%
        \ekvd@assert@filledarg{#3}%
          {%
            \ekvd@newreg#3{toks}%
            \ekvd@ifalso
              {%
                \let\ekvd@prot\protected
                \ekvd@add@val{#2}{#1#3={##1}}{}%
              }%
              {\protected\ekvd@long\ekvdef\ekvd@set{#2}{#1#3={##1}}}%
          }%
      }%
  }
\protected\def\ekvd@t@toks{\ekvd@type@toks{}}
\protected\def\ekvd@t@gtoks{\ekvd@type@toks\global}
%    \end{macrocode}
% \end{macro}
%
% \begin{macro}[internal]{\ekvd@type@preapptoks,\ekvd@t@apptoks,\ekvd@t@gapptoks}
% Just like |toks|, but expand the current contents of the |toks| register to
% append the new contents.
%    \begin{macrocode}
\ekvd@ifprimitive\toksapp
  {%
    \protected\def\ekvd@type@preapptoks#1#2#3%
      {%
        \ekvd@ifnew{}{#2}%
          {%
            \ekvd@assert@filledarg{#3}%
              {%
                \ekvd@newreg#3{toks}%
                \ekvd@ifalso
                  {%
                    \let\ekvd@prot\protected
                    \ekvd@add@val{#2}{#1#3{##1}}{}%
                  }%
                  {\protected\ekvd@long\ekvdef\ekvd@set{#2}{#1#3{##1}}}%
              }%
          }%
      }
    \protected\def\ekvd@t@apptoks{\ekvd@type@preapptoks\toksapp}
    \protected\def\ekvd@t@gapptoks{\ekvd@type@preapptoks\gtoksapp}
    \protected\def\ekvd@t@pretoks{\ekvd@type@preapptoks\tokspre}
    \protected\def\ekvd@t@gpretoks{\ekvd@type@preapptoks\gtokspre}
  }
  {%
    \protected\def\ekvd@type@apptoks#1#2#3%
      {%
        \ekvd@ifnew{}{#2}%
          {%
            \ekvd@assert@filledarg{#3}%
              {%
                \ekvd@newreg#3{toks}%
                \ekvd@ifalso
                  {%
                    \let\ekvd@prot\protected
                    \ekvd@add@val{#2}{#1#3=\expandafter{\the#3##1}}{}%
                  }%
                  {%
                    \protected\ekvd@long\ekvdef\ekvd@set{#2}%
                      {#1#3=\expandafter{\the#3##1}}%
                  }%
              }%
          }%
      }
    \protected\def\ekvd@t@apptoks{\ekvd@type@apptoks{}}
    \protected\def\ekvd@t@gapptoks{\ekvd@type@apptoks\global}
    \newtoks\ekvd@toks
    \protected\def\ekvd@type@pretoks#1#2#3%
      {%
        \ekvd@ifnew{}{#2}%
          {%
            \ekvd@assert@filledarg{#3}%
              {%
                \ekvd@newreg#3{toks}%
                \ekvd@ifalso
                  {%
                    \let\ekvd@prot\protected
                    \ekvd@add@val{#2}%
                      {%
                        \ekvd@toks={##1}%
                        #1#3=\expandafter{\the\expandafter\ekvd@toks\the#3}%
                      }%
                      {}%
                  }%
                  {%
                    \protected\ekvd@long\ekvdef\ekvd@set{#2}%
                      {%
                        \ekvd@toks={##1}%
                        #1#3=\expandafter{\the\expandafter\ekvd@toks\the#3}%
                      }%
                  }%
              }%
          }%
      }
    \protected\def\ekvd@t@pretoks{\ekvd@type@pretoks{}}
    \protected\def\ekvd@t@gpretoks{\ekvd@type@pretoks\global}
  }
%    \end{macrocode}
% \end{macro}
%
% \begin{macro}[internal]
%   {
%     \ekvd@type@reg,\ekvd@t@int,\ekvd@t@eint,\ekvd@t@gint,\ekvd@t@xint,
%     \ekvd@t@dimen,\ekvd@t@edimen,\ekvd@t@gdimen,\ekvd@t@xdimen,
%     \ekvd@t@skip,\ekvd@t@eskip,\ekvd@t@gskip,\ekvd@t@xskip
%   }
% The |\ekvd@type@reg| can handle all the types for which the assignment will
% just be \texttt{\meta{register}=\meta{value}}.
%    \begin{macrocode}
\protected\def\ekvd@type@reg#1#2#3#4#5#6%
  {%
    \ekvd@ifnew{}{#5}%
      {%
        \ekvd@assert@filledarg{#6}%
          {%
            \ekvd@newreg#6{#1}%
            \ekvd@ifalso
              {%
                \let\evkd@prot\protected
                \ekvd@add@val{#5}{#2#6=#3##1#4\relax}{}%
              }%
              {\protected\ekvd@long\ekvdef\ekvd@set{#5}{#2#6=#3##1#4\relax}}%
          }%
      }%
  }
\protected\def\ekvd@t@int{\ekvd@type@reg{count}{}{}{}}
\protected\def\ekvd@t@eint{\ekvd@type@reg{count}{}\numexpr\relax}
\protected\def\ekvd@t@gint{\ekvd@type@reg{count}\global{}{}}
\protected\def\ekvd@t@xint{\ekvd@type@reg{count}\global\numexpr\relax}
\protected\def\ekvd@t@dimen{\ekvd@type@reg{dimen}{}{}{}}
\protected\def\ekvd@t@edimen{\ekvd@type@reg{dimen}{}\dimexpr\relax}
\protected\def\ekvd@t@gdimen{\ekvd@type@reg{dimen}\global{}{}}
\protected\def\ekvd@t@xdimen{\ekvd@type@reg{dimen}\global\dimexpr\relax}
\protected\def\ekvd@t@skip{\ekvd@type@reg{skip}{}{}{}}
\protected\def\ekvd@t@eskip{\ekvd@type@reg{skip}{}\glueexpr\relax}
\protected\def\ekvd@t@gskip{\ekvd@type@reg{skip}\global{}{}}
\protected\def\ekvd@t@xskip{\ekvd@type@reg{skip}\global\glueexpr\relax}
%    \end{macrocode}
% \end{macro}
%
% \begin{macro}[internal]{\ekvd@type@store,\ekvd@t@store,\ekvd@t@gstore}
% The none-expanding |store| types use an |\edef| or |\xdef| and |\unexpanded|
% to be able to also store |#| easily.
%    \begin{macrocode}
\protected\def\ekvd@type@store#1#2#3#4%
  {%
    \ekvd@ifnew{}{#3}%
      {%
        \ekvd@assert@filledarg{#4}%
          {%
            \ekvd@newlet#4\ekv@empty
            \ekvd@ifalso
              {%
                \let\ekvd@prot\protected
                \ekvd@add@val{#3}{#1#4{#2}}{}%
              }%
              {\protected\ekvd@long\ekvdef\ekvd@set{#3}{#1#4{#2}}}%
          }%
      }%
  }
\protected\def\ekvd@t@store{\ekvd@type@store\edef{\ekv@unexpanded{##1}}}
\protected\def\ekvd@t@gstore{\ekvd@type@store\xdef{\ekv@unexpanded{##1}}}
\protected\def\ekvd@t@estore{\ekvd@type@store\edef{##1}}
\protected\def\ekvd@t@xstore{\ekvd@type@store\xdef{##1}}
%    \end{macrocode}
% \end{macro}
%
% \begin{macro}[internal]
%   {
%     \ekvd@type@meta,\ekvd@type@meta@a,\ekvd@type@meta@b,\ekvd@type@meta@c,
%     \ekvd@t@meta,\ekvd@t@nmeta
%   }
% |meta| sets up things such that another instance of |\ekvset| will be run on
% the argument, with the same \set.
%    \begin{macrocode}
\protected\long\def\ekvd@type@meta#1#2#3#4#5#6#7%
  {%
    \ekvd@ifnew{#1}{#6}%
      {%
        \ekvd@assert@filledarg{#7}%
          {%
            \edef\ekvd@tmp{\ekvd@set}%
            \expandafter\ekvd@type@meta@a\expandafter{\ekvd@tmp}{#7}{#2}%
            \ekvd@ifalso
              {\ekv@exparg{#3{#6}}{\ekvd@tmp#4}{#5}}%
              {\csname ekvlet#1\endcsname\ekvd@set{#6}\ekvd@tmp}%
          }%
      }%
  }
\protected\long\def\ekvd@type@meta@a#1#2%
  {%
    \expandafter\ekvd@type@meta@b\expandafter{\ekvset{#1}{#2}}%
  }
\protected\def\ekvd@type@meta@b
  {%
    \expandafter\ekvd@type@meta@c\expandafter
  }
\protected\long\def\ekvd@type@meta@c#1#2%
  {%
    \ekvd@prot\ekvd@long\def\ekvd@tmp#2{#1}%
  }
\protected\def\ekvd@t@meta{\ekvd@type@meta{}{##1}\ekvd@add@val{{##1}}{}}
\protected\def\ekvd@t@nmeta
  {%
    \ekvd@assert@not@long
    \ekvd@type@meta{NoVal}{}\ekvd@add@noval{}\ekvd@assert@not@long@also
  }
%    \end{macrocode}
% \end{macro}
%
% \begin{macro}[internal]
%   {\ekvd@type@smeta,\ekvd@type@smeta@,\ekvd@t@smeta,\ekvd@t@snmeta}
% |smeta| is pretty similar to |meta|, but needs two arguments inside of \val,
% such that the first is the \set\ for which the sub-|\ekvset| and the second is
% the \kv\ list.
%    \begin{macrocode}
\protected\long\def\ekvd@type@smeta#1#2#3#4#5#6#7%
  {%
    \ekvd@ifnew{#1}{#6}%
      {%
        \ekvd@assert@twoargs{#7}%
          {%
            \ekvd@type@meta@a#7{#2}%
            \ekvd@ifalso
              {\ekv@exparg{#3{#6}}{\ekvd@tmp#4}{#5}}%
              {\csname ekvlet#1\endcsname\ekvd@set{#6}\ekvd@tmp}%
          }%
      }%
  }
\protected\def\ekvd@t@smeta{\ekvd@type@smeta{}{##1}\ekvd@add@val{{##1}}{}}
\protected\def\ekvd@t@snmeta
  {%
    \ekvd@assert@not@long
    \ekvd@type@smeta{NoVal}{}\ekvd@add@noval{}\ekvd@assert@not@long@also
  }
%    \end{macrocode}
% \end{macro}
%
% \begin{macro}[internal]
%   {
%     \ekvd@type@choice,
%     \ekvd@populate@choice,\ekvd@populate@choice@,\ekvd@populate@choice@noarg,
%     \ekvd@choice@prefix,\ekvd@choice@prefix@,\ekvd@choice@prefix@done,
%     \ekvd@choice@p@protected,\ekvd@choice@p@protect,
%     \ekvd@choice@p@long,\ekvd@choice@p@long@,
%     \ekvd@t@choice
%   }
% The |choice| type is by far the most complex type, as we have to run a
% sub-parser on the choice-definition list, which should support the |@p@| type
% prefixes as well (but |long| will always throw an error, as they are not
% allowed to be long). |\ekvd@type@choice| will just define the choice-key, the
% handling of the choices definition will be done by |\ekvd@populate@choice|.
%    \begin{macrocode}
\protected\def\ekvd@type@choice#1%
  {%
    \ekvd@assert@not@long
    \ekv@exparg{\ekvd@prot\def\ekvd@tmp##1}%
      {%
        \expandafter\ekvd@h@choice\expandafter
          {\ekvd@choice@name\ekvd@set{#1}{##1}}%
      }%
    \ekvd@ifalso
      {%
        \ekvd@assert@val{#1}%
          {%
            \ekvd@if@not@already@choice{#1}%
              {%
                \ekv@exparg
                  {%
                    \expandafter\ekvd@add@aux
                      \csname\ekv@name\ekvd@set{#1}\endcsname{{##1}}{#1}%
                  }%
                  {\ekvd@tmp{##1}}%
                  {\ekvd@long\ekvdef}\ekvd@assert@not@long@also
              }%
          }%
      }%
      {\ekvlet\ekvd@set{#1}\ekvd@tmp}%
  }
%    \end{macrocode}
% |\ekvd@populate@choice| just uses |\ekvparse| and then gives control to
% |\ekvd@populate@choice@noarg|, which throws an error, and
% |\ekvd@populate@choice@|.
%    \begin{macrocode}
\protected\def\ekvd@populate@choice
  {%
    \ekvparse\ekvd@populate@choice@noarg\ekvd@populate@choice@
  }
\protected\long\def\ekvd@populate@choice@noarg#1%
  {%
    \expandafter\ekvd@err@missing@definition@msg\expandafter{\ekvd@cur : #1}%
  }
%    \end{macrocode}
% |\ekvd@populate@choice@| runs the prefix-test, if there is none we can
% directly define the choice, for that |\ekvd@set@choice| will expand to the
% current choice-key's name, which will have been defined by |\ekvd@t@choice|.
% If there is a prefix run the prefix grabbing routine, which was altered for
% |@type@choice|.
%    \begin{macrocode}
\protected\long\def\ekvd@populate@choice@#1#2%
  {%
    \ekvd@clear@prefixes
    \ekvd@ifspace{#1}%
      {\ekvd@choice@prefix{\ekv@mark#1}\ekv@mark#1\ekv@stop}%
      {%
        \expandafter\edef
          \csname\ekvd@choice@name\ekvd@set\ekvd@set@choice{#1}\endcsname
      }%
      {\unexpanded{#2}}%
  }
\protected\def\ekvd@choice@prefix#1#2
  {%
    \ekv@strip{#2}{\ekvd@choice@prefix@{#1}}\ekv@mark
  }
\protected\def\ekvd@choice@prefix@#1#2#3\ekv@stop
  {%
    \ekv@ifdefined{ekvd@choice@p@#2}%
      {%
        \csname ekvd@choice@p@#2\endcsname
        \ekvd@ifspace{#3}%
          {\ekvd@choice@prefix{#3}#3\ekv@stop}%
          {\ekvd@choice@prefix@done{#3}}%
      }%
      {\ekvd@choice@prefix@done{#1}}%
  }
\protected\def\ekvd@choice@prefix@done#1%
  {%
    \ekvd@prot\expandafter\edef
      \csname
        \ekv@strip{#1}{\ekvd@choice@name\ekvd@set\ekvd@set@choice}%
      \endcsname
  }
\protected\def\ekvd@choice@p@protected{\let\ekvd@prot\protected}
\let\ekvd@choice@p@protect\ekvd@choice@p@protected
\protected\def\ekvd@choice@invalid@p#1\ekvd@ifspace#2%
  {%
    \expandafter\ekvd@choice@invalid@p@\expandafter{\ekv@gobble@mark#2}{#1}%
    \ekvd@ifspace{#2}%
  }
\protected\def\ekvd@choice@invalid@p@#1#2%
  {%
    \expandafter\ekvd@err@no@prefix@msg\expandafter{\ekvd@cur : #2 #1}{#2}%
  }
\protected\def\ekvd@choice@p@long{\ekvd@choice@invalid@p{long}}%
\protected\def\ekvd@choice@p@also{\ekvd@choice@invalid@p{also}}%
\protected\def\ekvd@choice@p@new{\ekvd@choice@invalid@p{new}}%
%    \end{macrocode}
% Finally we're able to set up the |@t@choice| macro, which has to store the
% current choice-key's name, define the key, and parse the available choices.
%    \begin{macrocode}
\protected\long\def\ekvd@t@choice#1#2%
  {%
    \ekvd@ifnew{}{#1}%
      {%
        \ekvd@assert@arg
          {%
            \ekvd@type@choice{#1}%
            \def\ekvd@set@choice{#1}%
            \ekvd@populate@choice{#2}%
          }%
      }%
  }
%    \end{macrocode}
% \end{macro}
%
% \begin{macro}[internal]{\ekvd@t@choice-store,\ekvd@t@choice-enum}
%   These two types define a special kind of |choice| key and are quite similar,
%   the only difference is what the different choices do (hence they use a
%   shared initialisation which differs in the chosen |populate| step).
%    \begin{macrocode}
\protected\long\expandafter\def\csname ekvd@t@choice-store\endcsname
  {\ekvd@type@choicespecial\ekvd@populate@choicestore}
\protected\long\expandafter\def\csname ekvd@t@choice-enum\endcsname
  {\ekvd@type@choicespecial\ekvd@populate@choiceenum}
%    \end{macrocode}
% \begin{macro}[internal]{\ekvd@type@choicespecial}
%   Initialise similar to a |choice| key. The difference is that we require two
%   arguments (which we assert), a macro to store things in, and a |csv|-list
%   containing the allowed values. |#1| is the |populate| macro according to the
%   type used.
%    \begin{macrocode}
\protected\long\def\ekvd@type@choicespecial#1#2#3%
  {%
    \ekvd@ifnew{}{#2}%
      {%
        \ekvd@assert@twoargs{#3}%
          {%
            \ekvd@type@choice{#2}%
            \def\ekvd@set@choice{#2}%
            #1#3%
          }%
      }%
  }
%    \end{macrocode}
% \end{macro}
% \begin{macro}[internal]
%   {\ekvd@populate@choicestore,\ekvd@populate@choicestore@}
%   We initialise the storing macro if it doesn't yet exist, and then we loop
%   over the value list. The |\edef|s with |\unexpanded| are both necessary to
%   be able to store macro parameter tokens (the outer protects at define time,
%   the inner at use time).
%    \begin{macrocode}
\protected\long\def\ekvd@populate@choicestore#1%
  {%
    \ekvd@newlet#1\ekv@empty
    \ekvcsvloop{\ekvd@populate@choicestore@#1}%
  }
\protected\long\def\ekvd@populate@choicestore@#1#2%
  {%
    \protected\expandafter\edef
      \csname\ekvd@choice@name\ekvd@set\ekvd@set@choice{#2}\endcsname
      {\unexpanded{\edef#1{\unexpanded{#2}}}}%
  }
%    \end{macrocode}
% \end{macro}
% \begin{macro}[internal]
%   {\ekvd@populate@choiceenum,\ekvd@populate@choiceenum@}
%   This is similar to the population of a |choice-store| type, but instead of
%   storing the values in a macro this initialises a count and stores the
%   position of the value in the list inside that count (zero-indexed). The
%   space is necessary to terminate the number scanning, which is the reason we
%   use |\@firstofone| (so that the space after the macro name isn't gobbled by
%   \TeX).
%    \begin{macrocode}
\protected\long\def\ekvd@populate@choiceenum#1%
  {%
    \ekvd@newreg#1{count}%
    \def\ekvd@tmp{0}%
    \ekvcsvloop{\ekvd@populate@choiceenum@#1}%
  }
\protected\long\def\ekvd@populate@choiceenum@#1#2%
  {%
    \protected\expandafter\edef
      \csname\ekvd@choice@name\ekvd@set\ekvd@set@choice{#2}\endcsname
      {#1=\@firstofone{\ekvd@tmp} }%
    \edef\ekvd@tmp{\the\numexpr\ekvd@tmp+1\relax}%
  }
%    \end{macrocode}
% \end{macro}
% \end{macro}
%
% \begin{macro}[internal]{\ekvd@t@unknown-choice}
%    \begin{macrocode}
\protected\long\expandafter\def\csname ekvd@t@unknown-choice\endcsname#1#2%
  {%
    \ekvd@assert@new@for@name{\ekvd@unknown@choice@name\ekvd@set{#1}}%
      {%
        \ekvd@assert@arg
          {%
            \ekvd@assert@not@long
            \ekvd@assert@not@also
            \ekvd@prot\expandafter
            \def\csname\ekvd@unknown@choice@name\ekvd@set{#1}\endcsname##1{#2}%
          }%
      }%
  }
%    \end{macrocode}
% \end{macro}
%
% \begin{macro}[internal]
%   {\ekvd@t@unknown,\ekvd@type@unknown@code,\ekvd@type@unknown@noval}
%   The |unknown| type has different subtypes which would be the key names for
%   other types. It is first checked whether that subtype is defined, if it
%   isn't throw an error, else use that subtype.
%    \begin{macrocode}
\protected\long\def\ekvd@t@unknown#1#2%
  {%
    \ekv@ifdefined{ekvd@type@unknown@\detokenize{#1}}%
      {\csname ekvd@type@unknown@\detokenize{#1}\endcsname{#2}}%
      \ekvd@err@misused@unknown
  }
%    \end{macrocode}
%   The |unknown noval| type can use |\ekvdefunknownNoVal| directly (after
%   asserting some prefixes).
%    \begin{macrocode}
\protected\long\def\ekvd@type@unknown@noval#1%
  {%
    \ekvd@assert@new@for@name{\ekv@name\ekvd@set{}uN}%
      {%
        \ekvd@assert@arg
          {%
            \ekvd@assert@not@also
            \ekvd@assert@not@long
            \ekvd@prot\ekvdefunknownNoVal\ekvd@set{#1}%
          }%
      }%
  }
%    \end{macrocode}
%   The |unknown code| type uses some trickery during the definition in order to
%   swap out |#1| and |#2| in the user supplied definition. This is done via a
%   temporary macro that stores the definition but gets the parameter numbers
%   reversed while the real definition is done.
%    \begin{macrocode}
\protected\long\def\ekvd@type@unknown@code#1%
  {%
    \ekvd@assert@new@for@name{\ekv@name\ekvd@set{}u}%
      {%
        \ekvd@assert@arg
          {%
            \ekvd@assert@not@also
            \begingroup
              \def\ekvd@tmp##1##2{#1}%
              \ekv@exparg
              {%
                \endgroup
                \ekvd@prot\ekvd@long\ekvdefunknown\ekvd@set
              }%
              {\ekvd@tmp{##2}{##1}}%
          }%
      }%
  }
%    \end{macrocode}
% \end{macro}
%
% \begin{macro}[internal]
%   {
%     \ekvd@type@unknown@redirect,
%     \ekvd@type@unknown@redirect-code,
%     \ekvd@type@unknown@redirect-noval
%   }
%   The |unknown redirect| types also just forward to |\ekvredirectunknown|
%   after asserting some prefixes.
%    \begin{macrocode}
\protected\edef\ekvd@type@unknown@redirect#1%
  {%
    \expandafter\noexpand\csname ekvd@type@unknown@redirect-code\endcsname{#1}%
    \expandafter\noexpand\csname ekvd@type@unknown@redirect-noval\endcsname{#1}%
  }
\protected\expandafter\def\csname ekvd@type@unknown@redirect-code\endcsname#1%
  {%
    \ekvd@assert@new@for@name{\ekv@name\ekvd@set{}u}%
      {%
        \ekvd@assert@arg
          {%
            \ekvd@assert@not@also
            \ekvd@assert@not@protected
            \expandafter\ekvredirectunknown\expandafter{\ekvd@set}{#1}%
          }%
      }%
  }
\protected\expandafter\def\csname ekvd@type@unknown@redirect-noval\endcsname#1%
  {%
    \ekvd@assert@new@for@name{\ekv@name\ekvd@set{}uN}%
      {%
        \ekvd@assert@arg
          {%
            \ekvd@assert@not@also
            \ekvd@assert@not@protected
            \ekvd@assert@not@long
            \expandafter\ekvredirectunknownNoVal\expandafter{\ekvd@set}{#1}%
          }%
      }%
  }
%    \end{macrocode}
% \end{macro}
%
% \subsubsection{Key Type Helpers}
% There are some keys that might need helpers during their execution (not during
% their definition, which are gathered as |@type@| macros). These helpers are
% named |@h@|.
%
% \begin{macro}[internal]{\ekvd@h@choice,\ekvd@h@choice@}
% The |choice| helper will just test whether the given choice was defined, if
% not throw an error expandably, else call the macro which stores the code for
% this choice.
%    \begin{macrocode}
\def\ekvd@h@choice#1%
  {%
    \expandafter\ekvd@h@choice@
      \csname\ifcsname#1\endcsname#1\else relax\fi\endcsname
      {#1}%
  }
\def\ekvd@h@choice@#1#2%
  {%
    \ifx#1\relax
      \ekvd@err@choice@invalid{#2}%
      \expandafter\@gobble
    \fi
    #1%
  }
%    \end{macrocode}
% \end{macro}
%
%
% \subsubsection{Handling \texttt{also}}
%
% \begin{macro}[internal]
%   {\ekvd@add@val,\ekvd@add@noval,\ekvd@add@aux,\ekvd@add@aux@}
%    \begin{macrocode}
\protected\long\def\ekvd@add@val#1#2#3%
  {%
    \ekvd@assert@val{#1}%
      {%
        \expandafter\ekvd@add@aux\csname\ekv@name\ekvd@set{#1}\endcsname{{##1}}%
          {#1}{#2}{\ekvd@long\ekvdef}{#3}%
      }%
  }
\protected\long\def\ekvd@add@noval#1#2#3%
  {%
    \ekvd@assert@noval{#1}%
      {%
        \expandafter\ekvd@add@aux\csname\ekv@name\ekvd@set{#1}N\endcsname{}%
          {#1}{#2}\ekvdefNoVal{#3}%
      }%
  }
\protected\long\def\ekvd@add@aux#1#2%
  {%
    \ekvd@extract@prefixes#1%
    \expandafter\ekvd@add@aux@\expandafter{#1#2}%
  }
\protected\long\def\ekvd@add@aux@#1#2#3#4#5%
  {%
    #5%
    \ekvd@prot#4\ekvd@set{#2}{#1#3}%
  }
%    \end{macrocode}
% \end{macro}
%
% \begin{macro}[internal]
%   {
%     \ekvd@extract@prefixes,\ekvd@extract@prefixes@,
%     \ekvd@extract@prefixes@long,\ekvd@extract@prefixes@prot
%   }
% This macro checks which prefixes were used for the definition of a macro and
% sets |\ekvd@long| and |\ekvd@prot| accordingly.
%    \begin{macrocode}
\protected\def\ekvd@extract@prefixes#1%
  {%
    \expandafter\ekvd@extract@prefixes@\meaning#1\ekvd@stop
  }
%    \end{macrocode}
% In the following definition |#1| will get replaced by |macro:|, |#2| by
% |\long| and |#3| by |\protected| (in each, all tokens will have category
% other). This allows us to parse the |\meaning| of a macro for those strings.
%    \begin{macrocode}
\protected\def\ekvd@extract@prefixes@#1#2#3%
  {%
    \protected\def\ekvd@extract@prefixes@##1#1##2\ekvd@stop
      {%
        \ekvd@extract@prefixes@long
          ##1\ekvd@mark\@firstofone#2\ekvd@mark\@gobble\ekvd@stop
          {\let\ekvd@long\long}%
        \ekvd@extract@prefixes@prot
          ##1\ekvd@mark\@firstofone#3\ekvd@mark\@gobble\ekvd@stop
          {\let\ekvd@prot\protected}%
      }%
    \protected\def\ekvd@extract@prefixes@long##1#2##2\ekvd@mark##3##4\ekvd@stop
      {##3}%
    \protected\def\ekvd@extract@prefixes@prot##1#3##2\ekvd@mark##3##4\ekvd@stop
      {##3}%
  }
%    \end{macrocode}
% We use a temporary macro to expand the three arguments of
% |\ekvd@extract@prefixes@|, which will set up the real meaning of itself and
% the parsing for |\long| and |\protected|.
%    \begin{macrocode}
\begingroup
\edef\ekvd@tmp
  {%
    \endgroup
    \ekvd@extract@prefixes@
      {\detokenize{macro:}}%
      {\string\long}%
      {\string\protected}%
  }
\ekvd@tmp
%    \end{macrocode}
% \end{macro}
%
%
% \subsubsection{Tests}
%
% \begin{macro}[internal]{\ekvd@newlet,\ekvd@newreg}
% These macros test whether a control sequence is defined, if it isn't they
% define it, either via |\let| or via the correct \cs[no-index]{new\meta{reg}}.
%    \begin{macrocode}
\protected\def\ekvd@newlet#1#2%
  {%
    \ifdefined#1\ekv@fi@gobble\fi\@firstofone{\let#1#2}%
  }
\protected\def\ekvd@newreg#1#2%
  {%
    \ifdefined#1\ekv@fi@gobble\fi\@firstofone{\csname new#2\endcsname#1}%
  }
%    \end{macrocode}
% \end{macro}
%
% \begin{macro}[internal]
%   {\ekvd@assert@twoargs,\ekvd@ifnottwoargs,\ekvd@ifempty@gtwo}
% A test for exactly two tokens can be reduced for an empty-test after gobbling
% two tokens, in the case that there are fewer tokens than two in the argument,
% only macros will be gobbled that are needed for the true branch, which doesn't
% hurt, and if there are more this will not be empty.
%    \begin{macrocode}
\long\def\ekvd@assert@twoargs#1%
  {%
    \ekvd@ifnottwoargs{#1}{\ekvd@err@missing@definition}%
  }
\long\def\ekvd@ifnottwoargs#1%
  {%
    \ekvd@ifempty@gtwo#1\ekv@ifempty@B
      \ekv@ifempty@false\ekv@ifempty@A\ekv@ifempty@B\@firstoftwo
  }
\long\def\ekvd@ifempty@gtwo#1#2{\ekv@ifempty@\ekv@ifempty@A}
%    \end{macrocode}
% \end{macro}
%
% \begin{macro}[internal]
%   {
%     \ekvd@assert@val,\ekvd@assert@val@,\ekvd@assert@noval,\ekvd@assert@noval@,
%     \ekvd@extract@args,\ekvd@extracted@args,\ekvd@one@arg@string
%   }
% Assert that a given key is defined as a value taking key or a |NoVal| key with
% the correct argument structure, respectively.
%    \begin{macrocode}
\protected\def\ekvd@assert@val#1%
  {%
    \ekvifdefined\ekvd@set{#1}%
      {\expandafter\ekvd@assert@val@\csname\ekv@name\ekvd@set{#1}\endcsname}%
      {%
        \ekvifdefinedNoVal\ekvd@set{#1}%
          \ekvd@err@add@val@on@noval
          {\ekvd@err@undefined@key{#1}}%
        \@gobble
      }%
  }
\protected\def\ekvd@assert@val@#1%
  {%
    \expandafter\ekvd@extract@args\meaning#1\ekvd@stop
    \unless\ifx\ekvd@extracted@args\ekvd@one@arg@string
      \ekvd@err@unsupported@arg
    \fi
    \@firstofone
  }%
\protected\def\ekvd@assert@noval#1%
  {%
    \ekvifdefinedNoVal\ekvd@set{#1}%
      {\expandafter\ekvd@assert@noval@\csname\ekv@name\ekvd@set{#1}N\endcsname}%
      {%
        \ekvifdefined\ekvd@set{#1}%
          \ekvd@err@add@noval@on@val
          {\ekvd@err@undefined@key{#1}}%
        \@gobble
      }%
  }
\protected\def\ekvd@assert@noval@#1%
  {%
    \expandafter\ekvd@extract@args\meaning#1\ekvd@stop
    \unless\ifx\ekvd@extracted@args\ekv@empty
      \ekvd@err@unsupported@arg
    \fi
    \@firstofone
  }
\protected\def\ekvd@extract@args#1%
  {%
    \protected\def\ekvd@extract@args##1#1##2->##3\ekvd@stop
      {\def\ekvd@extracted@args{##2}}%
  }
\expandafter\ekvd@extract@args\expandafter{\detokenize{macro:}}
\edef\ekvd@one@arg@string{\string#1}
%    \end{macrocode}
% \end{macro}
%
% \begin{macro}[internal]{\ekvd@assert@arg,\ekvd@assert@arg@msg,\ekvd@ifnoarg}
% There is no need to actually define |\ekvd@ifnoarg| here, as it will be set by
% either |\ekvd@arg| or |\ekvd@noarg|.
%    \begin{macrocode}
\def\ekvd@assert@arg{\ekvd@ifnoarg\ekvd@err@missing@definition}
\long\def\ekvd@assert@arg@msg#1%
  {%
    \ekvd@ifnoarg{\ekvd@err@missing@definition@msg{#1}}%
  }
%    \end{macrocode}
% \end{macro}
%
% \begin{macro}[internal]{\ekvd@assert@filledarg,\ekvd@ifnoarg@or@empty}
%    \begin{macrocode}
\long\def\ekvd@assert@filledarg#1%
  {%
    \ekvd@ifnoarg@or@empty{#1}\ekvd@err@missing@definition
  }
\long\def\ekvd@ifnoarg@or@empty#1%
  {%
    \ekvd@ifnoarg
      \@firstoftwo
      {\ekv@ifempty{#1}}%
  }
%    \end{macrocode}
% \end{macro}
%
% \begin{macro}[internal]
%   {
%     \ekvd@assert@not@long,\ekvd@assert@not@protected,\ekvd@assert@not@also
%     \ekvd@assert@not@long@also,\ekvd@assert@not@protected@also,
%     \ekvd@assert@new,\ekvd@assert@not@new
%   }
% Some key-types don't want to be |also|, |\long| or |\protected|, so we provide
% macros to test this and throw an error, this could be silently ignored but now
% users will learn to not use unnecessary stuff which slows the compilation
% down.
%    \begin{macrocode}
\def\ekvd@assert@not@long{\ifx\ekvd@long\long\ekvd@err@no@prefix{long}\fi}
\def\ekvd@assert@not@protected
  {\ifx\ekvd@prot\protected\ekvd@err@no@prefix{protected}\fi}
\def\ekvd@assert@not@also{\ekvd@ifalso{\ekvd@err@no@prefix{also}}{}}
\def\ekvd@assert@not@long@also
  {\ifx\ekvd@long\long\ekvd@err@no@prefix@also{long}\fi}
\def\ekvd@assert@not@protected@also
  {\ifx\ekvd@prot\protected\ekvd@err@no@prefix@also{protected}\fi}
\def\ekvd@assert@new#1#2%
  {\csname ekvifdefined#1\endcsname\ekvd@set{#2}{\ekvd@err@not@new}}
\def\ekvd@assert@not@new
  {\ifx\ekvd@ifnew\ekvd@assert@new\ekvd@err@no@prefix{new}\fi}
\def\ekvd@assert@new@for@name#1%
  {%
    \ifx\ekvd@ifnew\ekvd@assert@new
      \ekv@fi@firstoftwo
    \fi
    \@secondoftwo
      {\ekv@ifdefined{#1}\ekvd@err@not@new}%
      \@firstofone
  }
%    \end{macrocode}
% \end{macro}
%
% \begin{macro}[internal]
%   {
%     \ekvd@if@not@already@choice, \ekvd@if@not@already@choice@a,
%     \ekvd@if@not@already@choice@b
%   }
% It is bad to use |also| on a key that already contains a |choice|, as both
% choices would share the same valid values and thus lead to each callback being
% used twice. The following is a rudimentary test against this.
%    \begin{macrocode}
\protected\def\ekvd@if@not@already@choice#1%
  {%
    \expandafter\ekvd@if@not@already@choice@a
      \csname\ekv@name\ekvd@set{#1}\endcsname
      {}\ekvd@h@choice\ekvd@stop
  }
\protected\def\ekvd@if@not@already@choice@a
  {%
    \expandafter\ekvd@if@not@already@choice@b
  }
\long\protected\def\ekvd@if@not@already@choice@b#1\ekvd@h@choice#2\ekvd@stop
  {%
    \ekv@ifempty{#2}\@firstofone\@gobble
  }
%    \end{macrocode}
% \end{macro}
%
% \begin{macro}[internal]{\ekvd@ifspace,\ekvd@ifspace@}
% Yet another test which can be reduced to an if-empty, this time by gobbling
% everything up to the first space.
%    \begin{macrocode}
\long\def\ekvd@ifspace#1%
  {%
    \ekvd@ifspace@#1 \ekv@ifempty@B
      \ekv@ifempty@false\ekv@ifempty@A\ekv@ifempty@B\@firstoftwo
  }
\long\def\ekvd@ifspace@#1 % keep this space
  {%
    \ekv@ifempty@\ekv@ifempty@A
  }
%    \end{macrocode}
% \end{macro}
%
% \subsubsection{Messages}
%
% Most messages of \expkvd\ are not expandable, since they only appear during
% key-definition, which is not expandable anyway.
%
% \begin{macro}[internal]
%   {
%     \ekvd@errm,\ekvd@err@missing@definition,\ekvd@err@missing@definition@msg,
%     \ekvd@err@missing@type,\ekvd@err@undefined@prefix,\ekvd@err@undefined@key,
%     \ekvd@err@no@prefix,\ekvd@err@no@prefix@msg,\ekvd@err@no@prefix@also,
%     \ekvd@err@add@val@on@noval,\ekvd@err@add@noval@on@val,
%     \ekvd@err@unsupported@arg,\ekvd@err@not@new
%   }
% The non-expandable error messages are boring, so here they are:
%    \begin{macrocode}
\protected\def\ekvd@errm#1{\errmessage{expkv-def Error: #1}}
\protected\def\ekvd@err@missing@definition
  {\ekvd@errm{Missing definition for key `\ekvd@cur'}}
\protected\def\ekvd@err@missing@definition@msg#1%
  {\ekvd@errm{Missing definition for key `\ekv@unexpanded{#1}'}}
\protected\def\ekvd@err@missing@type
  {\ekvd@errm{Missing type prefix for key `\ekvd@cur'}}
\protected\def\ekvd@err@undefined@prefix#1%
  {%
    \ekvd@errm
      {%
        Undefined prefix `\ekv@unexpanded{#1}' found while processing
        `\ekvd@cur'%
      }%
  }
\protected\def\ekvd@err@undefined@key#1%
  {%
    \ekvd@errm
      {Undefined key `\ekv@unexpanded{#1}' found while processing `\ekvd@cur'}%
  }
\protected\def\ekvd@err@undefined@noval#1%
  {%
    \ekvd@errm
      {%
        Undefined noval key `\unexpanded{#1}' found while processing
        `\ekvd@cur'%
      }%
  }
\protected\def\ekvd@err@no@prefix#1%
  {\ekvd@errm{prefix `#1' not accepted in `\ekvd@cur'}}
\protected\def\ekvd@err@no@prefix@msg#1#2%
  {\ekvd@errm{prefix `#2' not accepted in `\ekv@unexpanded{#1}'}}
\protected\def\ekvd@err@no@prefix@also#1%
  {\ekvd@errm{`\ekvd@cur' not allowed with a `#1' key}}
\protected\def\ekvd@err@add@val@on@noval
  {\ekvd@errm{`\ekvd@cur' not allowed with a NoVal key}}
\protected\def\ekvd@err@add@noval@on@val
  {\ekvd@errm{`\ekvd@cur' not allowed with a value taking key}}
\protected\def\ekvd@err@unsupported@arg\fi\@firstofone#1%
  {%
    \fi
    \ekvd@errm
      {%
        Existing key-macro has the unsupported argument string
        `\ekvd@extracted@args' for key `\ekvd@cur'%
      }%
  }
\protected\def\ekvd@err@not@new
  {\ekvd@errm{The key for `\ekvd@cur' is already defined}}
\protected\long\def\ekvd@err@misused@unknown
  {\ekvd@errm{Misuse of the unknown type found while processing `\ekvd@cur'}}
%    \end{macrocode}
% \end{macro}
%
% \begin{macro}[internal]
%   {
%     \ekvd@err@choice@invalid,\ekvd@err@choice@invalid@,\ekvd@choice@name,
%     \ekvd@unknown@choice@name
%   }
% |\ekvd@err@choice@invalid| will have to use this mechanism to throw its
% message. Also we have to retrieve the name parts of the choice in an easy way,
% so we use parentheses of catcode 8 here, which should suffice in most cases to
% allow for a correct separation.
%    \begin{macrocode}
\def\ekvd@err@choice@invalid#1%
  {%
    \ekvd@err@choice@invalid@#1%
  }
\begingroup
\catcode40=8
\catcode41=8
\@firstofone{\endgroup
\def\ekvd@choice@name#1#2#3%
  {%
    ekvd#1(#2)\detokenize{#3}%
  }
\def\ekvd@unknown@choice@name#1#2%
  {%
    ekvd:u:#1(#2)%
  }
\def\ekvd@err@choice@invalid@ ekvd#1(#2)\detokenize#3%
  {%
    \ekv@ifdefined{\ekvd@unknown@choice@name{#1}{#2}}%
      {\csname\ekvd@unknown@choice@name{#1}{#2}\endcsname{#3}}%
      {\ekvd@err{invalid choice `#3' for `#2' in set `#1'}}%
  }
}
%    \end{macrocode}
% \end{macro}
%
% \begin{macro}[internal]{\ekvd@err}
% The expandable error messages use |\ekvd@err|, which is just like |\ekv@err|
% from \expkv. It uses a runaway argument to start the error message.
%    \begin{macrocode}
\ekv@exparg{\long\def\ekvd@err#1}{\ekverr{expkv-def}{#1}}
%    \end{macrocode}
% \end{macro}
%
% Now everything that's left is to reset the category code of |@|.
%    \begin{macrocode}
\catcode`\@=\ekvd@tmp
%    \end{macrocode}
%
% \gobbledocstriptag
%</tex>
%^^A=<<
%
%^^A=<<
%
% \end{implementation}^^A=<<
%
% \clearpage
% \PrintIndex
%
\endinput
%
^^A vim: ft=tex fdm=marker fmr=>>=,=<<
}{}
\ProvidesFile{expkv-def.dtx}
  [\csname ekvdDate\endcsname\ a key-defining frontend for expkv]
\PassOptionsToPackage{full}{textcomp}
\documentclass{l3doc}
\RequirePackage[oldstylenums,nott]{kpfonts}
\input{glyphtounicode}
\pdfgentounicode=1
\RequirePackage{tcolorbox}
\newtcolorbox{exresult}[2][]
  {colback=expkvgrey!10!white,colframe=expkvgrey,fontupper=\small,width={#2},#1}
\newtcbox\exres[1][]
  {
     colback=expkvgrey!10!white
    ,colframe=expkvgrey
    ,size=small
    ,nobeforeafter
    ,tcbox raise base
    ,fontupper=\small
    ,#1
  }
\RequirePackage{listings}
\RequirePackage{booktabs}
\RequirePackage{array}
\RequirePackage{collcell}
\RequirePackage{siunitx}
\RequirePackage{xcolor}
\RequirePackage{caption}
\RequirePackage{microtype}
\RequirePackage{accsupp}
\lstdefinelanguage{expkvd}[primitive]{TeX}
  {
    ,moretexcs=[4]^^A e-TeX
      {
        expanded,
        numexpr,
        protected,
      }
    ,moretexcs=[5]^^A plain/LaTeX
      {
        approx,
        begin,
        empty,
        item,
        LaTeX,
        makeatletter,makeatother,
        newcommand,newdimen,
        RequirePackage,
        rule,
        TeX,
        textit,texttt,
        usepackage,
      }
    ,moretexcs=[6]^^A used packages
      {
        ^^A xfp
        fpeval,
        ^^A keyval
        setkeys,
        ^^A yax
        defactiveparameter,storevalue,setparameterlist
      }
    ,moretexcs=[2]^^A expkv-def macros
      {
        ekvdefinekeys
      }
    ,moretexcs=[3]^^A expkv-pkg macros
      {
        ^^A expkv
        ekvdef,ekvdefNoVal,^^A
        ekvlet,ekvletNoVal,ekvletkv,ekvletkvNoVal,^^A
        ekvdefunknown,ekvdefunknownNoVal,^^A
        ekvredirectunknown,ekvredirectunknownNoVal,^^A
        ekvset,ekvsetSneaked,^^A
        ekvsetdef,ekvsetSneakeddef,ekvsetdefSneaked,^^A
        ekvparse,^^A
        ekvVersion,ekvDate,^^A
        ekvifdefined,ekvifdefinedNoVal,ekvifdefinedset,^^A
        ekvbreak,ekvbreakPreSneak,ekvbreakPostSneak,^^A
        ekvsneak,ekvsneakPre,^^A
        ekvchangeset,^^A
        ekvoptarg,ekvoptargTF,^^A
        ekverr,
        ^^A expkv-cs
        ekvcSecondaryKeys,ekvcSplitAndForward,
      }
  }
\colorlet{codeparam}{cyan!65!black}
\def\mylstwd{.55em}
\lstset
  {
    ,language=expkvd
    ,flexiblecolumns=false
    ,basewidth=\mylstwd
    ,gobble=2
    ,basicstyle=\fontfamily{jkp}\itshape
    ,texcsstyle=*[1]{\bfseries\color{expkvgrey}} ^^A primitives
    ,texcsstyle=*[2]{\bfseries\color{expkvred}}  ^^A expkv
    ,texcsstyle=*[3]{\color{expkvred}}           ^^A expkv-pkg
    ,texcsstyle=*[4]{\bfseries\color{expkvgrey}} ^^A e-TeX
    ,texcsstyle=*[5]{\bfseries\color{expkvgrey}} ^^A plain/LaTeX macros
    ,texcsstyle=*[6]{}                           ^^A macros of other packages
    ,commentstyle=\color[gray]{0.4}
    ,literate=
              {\{} {{\CodeUpBf\{}}{1}
              {\}} {{\CodeUpBf\}}}{1}
              {$}  {{\CodeUpBf\$}}{1}
              {[}  {{\CodeUp[}}{1}
              {]}  {{\CodeUp]}}{1}
              {(}  {{\CodeUp(}}{1}
              {)}  {{\CodeUp)}}{1}
              {*}  {{$*$}}{1}
              {1}  {{\CodeUp{1}}}{1}
              {2}  {{\CodeUp{2}}}{1}
              {3}  {{\CodeUp{3}}}{1}
              {4}  {{\CodeUp{4}}}{1}
              {5}  {{\CodeUp{5}}}{1}
              {6}  {{\CodeUp{6}}}{1}
              {7}  {{\CodeUp{7}}}{1}
              {8}  {{\CodeUp{8}}}{1}
              {9}  {{\CodeUp{9}}}{1}
              {0}  {{\CodeUp{0}}}{1}
              {##} {{\CodeColored{codeparam}{1}{\#}}}{1}
              {##1}{{\CodeColored{codeparam}{2}{\#1}}}{2}
              {##2}{{\CodeColored{codeparam}{2}{\#2}}}{2}
              {##3}{{\CodeColored{codeparam}{2}{\#3}}}{2}
              {##4}{{\CodeColored{codeparam}{2}{\#4}}}{2}
              {##5}{{\CodeColored{codeparam}{2}{\#5}}}{2}
              {##6}{{\CodeColored{codeparam}{2}{\#6}}}{2}
              {##7}{{\CodeColored{codeparam}{2}{\#7}}}{2}
              {##8}{{\CodeColored{codeparam}{2}{\#8}}}{2}
              {##9}{{\CodeColored{codeparam}{2}{\#9}}}{2}
              {<key>}{{$\langle$}key{$\rangle$}}{5}
              {<set>}{{$\langle$}set{$\rangle$}}{5}
              {<set1>}{{$\langle$}set1{$\rangle$}}{6}
              {<set2>}{{$\langle$}set2{$\rangle$}}{6}
    ^^A,literate=*{<key>}{\key}{4}{<set>}{\set}{4}
  }
\newcommand*\CodeColored[3]{\textcolor{#1}{\makebox[\dimexpr\mylstwd*#2]{#3}}}
\newcommand*\CodeUpBf[1]{\makebox[\mylstwd]{\textup{\textbf{#1}}}}
\newcommand*\CodeUp[1]{\makebox[\mylstwd]{\textup{#1}}}
\RequirePackage{randtext}
\let\metaORIG\meta
\protected\def\meta #1{\texttt{\metaORIG{#1}}}
\renewcommand*\thefootnote{\fnsymbol{footnote}}
\definecolor{expkvred}{HTML}{9F393D}
\colorlet{expkvgrey}{black!75}
\makeatletter
\newcommand*\expkv
  {^^A
    \texorpdfstring
      {^^A
        \mbox
          {^^A
            \BeginAccSupp{ActualText=expkv}^^A
            \href{https://github.com/Skillmon/tex_expkv}
              {^^A
                \rmfamily
                \bfseries
                {\color{expkvgrey}e\kern-.05em x\kern-.05em}^^A
                \lower.493ex
                  \hbox{{\color{expkvgrey}P}\kern-.1em{\color{expkvred}k}}^^A
                \kern-.18em{\color{expkvred}v}^^A
              }^^A
            \EndAccSupp{}^^A
          }^^A
      }
      {expkv}^^A
  }
\newcommand*\expkvd
  {^^A
    \texorpdfstring
      {^^A
        \mbox
          {^^A
            \BeginAccSupp{ActualText=expkv-def}^^A
            \href{https://github.com/Skillmon/tex_expkv-def}
              {^^A
                \rmfamily
                \bfseries
                {\color{expkvgrey}e\kern-.05em x\kern-.05em}^^A
                \lower.493ex
                  \hbox{{\color{expkvgrey}P}\kern-.1em{\color{expkvred}k}}^^A
                \kern-.18em{\color{expkvred}v}^^A
                {\color{expkvgrey}^^A
                  \kern.05em\rule[-.1ex]{.08em}{1.2ex}\kern.05em\textsc{def}^^A
                }^^A
              }^^A
            \EndAccSupp{}^^A
          }^^A
      }
      {expkv-def}^^A
  }
\newcommand\kv{\meta{key}=\meta{value}}
\newcommand\key{\meta{key}}
\newcommand\val{\meta{value}}
\newcommand\set{\meta{set}}
\newcommand\enfprefix{\textcolor{black}}
\newcommand\allprefix{\textcolor{gray}}
\newcommand\notprefix{\textcolor{red!80!black}}
\newcommand\printprefix[2]
  {^^A
    \ifcase\numexpr#1\relax\or
      \enfprefix{#2}\or
      \allprefix{#2}\or
      \notprefix{#2}\fi
  }
\newcommand\prefixes[4]
  {^^A
    \hfill
    \printprefix{#1}{new}
    \printprefix{#2}{also}
    \printprefix{#3}{protected}
    \printprefix{#4}{long}^^A
  }
\hypersetup{linkcolor=red!80!black,urlcolor=purple!80!black}
\DoNotIndex{\def,\edef,\,,\=,\begingroup,\catcode,\chardef,\csname,\endcsname}
\DoNotIndex{\endgroup,\endinput,\errmessage,\expandafter,\input,\let,\long}
\DoNotIndex{\protected,\ProvidesFile,\ProvidesPackage,\relax,\space}
\DoNotIndex{\@,\unexpanded,\string,\expanded,\dimexpr,\global,\glueexpr,\hbox}
\DoNotIndex{\numexpr,\RequirePackage,\setbox,\the,\unless,\xdef,\@firstofone}
\DoNotIndex{\@firstoftwo,\@gobble,\@secondoftwo,\meaning,\detokenize}
\DoNotIndex{\romannumeral,\^}
\DoNotIndex{\ifcsname}
\DoNotIndex{\ifx}
\DoNotIndex{\ifdefined}
\DoNotIndex{\iffalse}
\DoNotIndex{\iftrue}
\DoNotIndex{\else}
\DoNotIndex{\fi}
\@gobble\fi ^^A ignoring \ifx, \ifcsname, etc., but only one \fi
\@gobble\fi ^^A ignoring \ifx, \ifcsname, etc., but only one \fi
\@gobble\fi ^^A ignoring \ifx, \ifcsname, etc., but only one \fi
\@gobble\fi ^^A ignoring \ifx, \ifcsname, etc., but only one \fi
\@ifdefinable\gobbledocstriptag{\def\gobbledocstriptag#1>{}}
\makeatother
\begin{document}
  \title
    {^^A
      \texorpdfstring
        {^^A
          \huge\expkvd
          \\[\medskipamount]
          \Large a key-defining frontend for \expkv
        }{expkv-def - a key-defining frontend for expkv}^^A
    }
  \date{\ekvdDate\space v\ekvdVersion}
  \author{Jonathan P. Spratte\thanks{\protect\randomize{jspratte@yahoo.de}}}
  \DocInput{expkv-def.dtx}
\end{document}
%</driver>^^A=<<
% \fi
%
% \maketitle
% \renewcommand*\thefootnote{\arabic{footnote}}
%
% \begin{abstract}
% \noindent\parfillskip=0pt
% \expkvd\ provides a small \kv\ interface to define keys for \expkv. Key-types
% are declared using prefixes, similar to static typed languages. The stylised
% name is \expkvd\ but the files use \file{expkv-def}, this is due to CTAN-rules
% which don't allow \string| in package names since that is the pipe symbol in
% *nix shells.
% \end{abstract}
%
% \tableofcontents
%
% \begin{documentation}^^A>>=
%
% \section{Documentation}
%
% Since the trend for the last couple of years goes to defining keys for a \kv\
% interface using a \kv\ interface, I thought that maybe providing such an
% interface for \expkv\ will make it more attractive for actual use, besides its
% unique selling points of being fully expandable, and fast and reliable. But at
% the same time I don't want to widen \expkv's initial scope. So here it is
% \expkvd, go define \kv\ interfaces with \kv\ interfaces.
%
% Unlike many of the other established \kv\ interfaces to define keys, \expkvd\
% works using prefixes instead of suffixes (\emph{e.g.}, |.tl_set:N| of
% \pkg{l3keys}) or directory like handlers (\emph{e.g.}, |/.store in| of
% \pkg{pgfkeys}). This was decided as a personal preference, more over in \TeX\
% parsing for the first space is way easier than parsing for the last one.
% \expkvd's prefixes are sorted into two categories: |p|-type, which are
% equivalent to \TeX's prefixes like |\long|, and |t|-type defining the type of
% the key. For a description of the available |p|-prefixes take a look at
% \autoref{sec:prefix:p}, the |t|-prefixes are described in
% \autoref{sec:prefix:t}.
%
% \expkvd\ is usable as generic code, as a \LaTeX\ package, and as a \ConTeXt\
% module. It'll automatically load \expkv\ in the same mode as well. To use it,
% just use one of
% \begin{lstlisting}
% \input expkv-def       % plainTeX
% \usepackage{expkv-def} % LaTeX
% \usemodule[expkv-def]  % ConTeXt
% \end{lstlisting}
%
% \subsection{Macros}
% Apart from version and date containers there is only a single user-facing
% macro, and that should be used to define keys.
%
% \begin{function}{\ekvdefinekeys}
%   \begin{syntax}
%     \cs{ekvdefinekeys}\marg{set}\{\kv, \ldots\}
%   \end{syntax}
%   In \set, define \key\ to have definition \val. The general syntax for \key\
%   should be
%   \begin{quote}\ttfamily\small
%     \meta{prefix} \meta{name}
%   \end{quote}
%   Where \meta{prefix} is a space separated list of optional |p|-type prefixes
%   followed by one |t|-type prefix. The syntax of \val\ is dependent on the
%   used |t|-prefix.
% \end{function}
%
% \begin{function}{\ekvdDate,\ekvdVersion}
%   These two macros store the version and date of the package.
% \end{function}
%
% \subsection{Prefixes}
% As already said there are |p|-prefixes and |t|-prefixes. Not every |p|-prefix
% is allowed for all |t|-prefixes.
%
% \subsubsection{\texttt{p}-Prefixes}\label{sec:prefix:p}
% The two |p|-type prefixes |long| and |protected| are pretty simple by nature,
% so their description is pretty simple. They affect the \key\ at use-time, so
% omitting |long| doesn't mean that a \meta{definition} can't contain a |\par|
% token, only that the \key\ will not accept a |\par| in \val.  On the other
% hand |new| and |also| might be simple on first sight as well, but their rules
% are a bit more complicated.
%
% \begin{function}{also}
%   The following key type will be \emph{added} to an existing \key's
%   definition. You can't add a type taking an argument at use time to an
%   existing key which doesn't take an argument and vice versa. Also you'll get
%   an error if you try to add an action which isn't allowed to be either |long|
%   or |protected| to a key which already is |long| or |protected| (the opposite
%   order would be suboptimal as well, but can't be really captured with the
%   current code).
%
%   A key already defined as |long| or |protected| will stay |long| or
%   |protected|, but you can as well add |long| or |protected| with the |also|
%   definition.
%
%   As a small example, suppose you want to create a boolean key, but
%   additionally to setting a boolean value you want to execute some more code
%   as well, you can use the following
% \begin{lstlisting}
% \ekvdefinekeys{also-example}
%   {
%      bool key      = \ifmybool
%     ,also code key = \domystuff{#1}
%   }
% \end{lstlisting}
%
%   If you use |also| on a |choice|, |bool|, |invbool|, or |boolpair| key it is
%   tried to determine if the key already is of one of those types. If this test
%   is true the declared choices will be added to the possible choices but the
%   key's definition will not be changed other than that. If that wouldn't have
%   been done, the callbacks of the different choices could get called multiple
%   times.
% \end{function}
%
% \begin{function}{protected,protect}
%   The following key will be defined |\protected|. Note that key-types which
%   can't be defined expandable will always use |\protected|.
% \end{function}
%
% \begin{function}{long}
%   The following key will be defined |\long|.
% \end{function}
%
% \begin{function}{new}
%   The following key must be new (so previously undefined). An error is thrown
%   if it is already defined and the new definition is ignored. |new| only
%   asserts that there are no conflicts between |NoVal| keys and other |NoVal|
%   keys or value taking keys and other value taking keys. For example you can
%   use the following without an error:
% \begin{lstlisting}
% \ekvdefinekeys{new-example}
%   {
%      code key       = \domystuffwitharg{#1}
%     ,new  noval key = \domystuffwithoutarg
%   }
% \end{lstlisting}
% \end{function}
%
% \subsubsection{\texttt{t}-Prefixes}\label{sec:prefix:t}
% Since the |p|-type prefixes apply to some of the |t|-prefixes automatically
% but sometimes one might be disallowed we need some way to highlight this
% behaviour. In the following an enforced prefix will be printed black
% (\texttt{\enfprefix{protected}}), allowed prefixes will be grey
% (\texttt{\allprefix{protected}}), and disallowed prefixes will be red
% (\texttt{\notprefix{protected}}). This will be put flush-right in the syntax
% showing line.
%
% \begin{function}{code,ecode}
%   \begin{syntax}
%     code \key\ = \marg{definition} \prefixes2222
%   \end{syntax}
%   Define \key\ to expand to \meta{definition}. The \key\ will require a \val\
%   for which you can use |#1| inside \meta{definition}. The |ecode| variant
%   will fully expand \meta{definition} inside an |\edef|.
% \end{function}
%
% \begin{function}{noval,enoval}
%   \begin{syntax}
%     noval \key\ = \marg{definition} \prefixes2223
%   \end{syntax}
%   The |noval| type defines \key\ to expand to \meta{definition}. The \key\
%   will not take a \val. |enoval| fully expands \meta{definition} inside an
%   |\edef|.
% \end{function}
%
% \begin{function}{default,qdefault,odefault,fdefault,edefault}
%   \begin{syntax}
%     default \key\ = \marg{definition} \prefixes3223
%   \end{syntax}
%   This serves to place a default \val\ for a \key\ that takes an argument, the
%   \key\ can be of any argument-grabbing kind, and when used without a \val\
%   it will be passed \meta{definition} instead. The |qdefault| variant will
%   expand the \key's code once, so will be slightly quicker, but not change if
%   you redefine \key. |odefault| is just another name for |qdefault|. The
%   |fdefault| version will expand the key code until a non-expandable token or
%   a space is found, a space would be gobbled.\footnotemark{} The |edefault| on
%   the other hand fully expands the \key-code with \meta{definition} as its
%   argument inside of an |\edef|.
% \end{function}
% \footnotetext
%   {^^A
%     For those familiar with \TeX-coding: This uses a
%     \cs[no-index]{romannumeral}-expansion.
%   }
%
% \begin{function}{initial,oinitial,finitial,einitial}
%   \begin{syntax}
%     initial \key\ = \{\val\} \prefixes3333
%     initial \key
%   \end{syntax}
%   With |initial| you can set an initial \val\ for an already defined argument
%   taking \key. It'll just call the key-macro of \key\ and pass it \val. The
%   |einitial| variant will expand \val\ using an |\edef| expansion prior to
%   passing it to the key-macro and the |oinitial| variant will expand the first
%   token in \val\ once. |finitial| will expand \val\ until a non-expandable
%   token or a space is found, a space would be
%   gobbled.\footnotemark
%
%   If you don't provide a value (and no equals sign) a |noval| \key\ of the
%   same name is called once (or, if you specified a |default| for a value
%   taking key that would be used).
% \end{function}
% \footnotetext{Again using \cs[no-index]{romannumeral}.}
%
% \begin{function}{bool,gbool,boolTF,gboolTF}
%   \begin{syntax}
%     bool \key\ = \meta{cs} \prefixes2223
%   \end{syntax}
%   The \meta{cs} should be a single control sequence, such as |\iffoo|.
%   This will define \key\ to be a boolean key, which only takes the values
%   |true| or |false| and will throw an error for other values. If the key is
%   used without a \val\ it'll have the same effect as if you use
%   \texttt{\key=true}. |bool| and |gbool| will behave like \TeX-ifs so either
%   be |\iftrue| or |\iffalse|. The |boolTF| and |gboolTF| variants will both
%   take two arguments and if true the first will be used else the second, so
%   they are always either |\@firstoftwo| or |\@secondoftwo|. The variants with
%   a leading |g| will set the control sequence globally, the others locally. If
%   \meta{cs} is not yet defined it'll be initialised as the |false| version.
%   Note that the initialisation is \emph{not} done with |\newif|, so you will
%   not be able to do |\footrue| outside of the \kv\ interface, but you could
%   use |\newif| yourself. Even if the \key\ will not be |\protected| the
%   commands which execute the |true| or |false| choice will be, so the usage
%   should be safe in an expansion context (\emph{e.g.}, you can use
%   \texttt{edefault \key\ = false} without an issue to change the default
%   behaviour to execute the |false| choice). Internally a |bool| \key\ is the
%   same as a |choice| key which is set up to handle |true| and |false| as
%   choices.
% \end{function}
%
% \begin{function}{invbool,ginvbool,invboolTF,ginvboolTF}
%   \begin{syntax}
%     bool \key\ = \meta{cs} \prefixes2223
%   \end{syntax}
%   These are inverse boolean keys, they behave like |bool| and friends but set
%   the opposite meaning to the macro \meta{cs} in each case. So if
%   \texttt{key=true} is used |invbool| will set \meta{cs} to |\iffalse| and
%   vice versa.
% \end{function}
%
% \begin{function}{boolpair,gboolpair,boolpairTF,gboolpairTF}
%   \begin{syntax}
%     boolpair \key\ = \meta{cs_1}\meta{cs_2} \prefixes2223
%   \end{syntax}
%   The |boolpair| key type behaves like both |bool| and |invbool|, the
%   \meta{cs_1} will be set to the meaning according to the rules of |bool|, and
%   \meta{cs_2} will be set to the opposite.
% \end{function}
%
% \begin{function}{store,estore,gstore,xstore}
%   \begin{syntax}
%     store \key\ = \meta{cs} \prefixes2212
%   \end{syntax}
%   The \meta{cs} should be a single control sequence, such as |\foo|.
%   This will define \key\ to store \val\ inside of the control sequence. If
%   \meta{cs} isn't yet defined it will be initialised as empty. The variants
%   behave similarly to their |\def|, |\edef|, |\gdef|, and |\xdef|
%   counterparts, but |store| and |gstore| will allow you to store macro
%   parameters inside of them by using |\unexpanded|.
% \end{function}
%
% \begin{function}{data,edata,gdata,xdata}
%   \begin{syntax}
%     data \key\ = \meta{cs} \prefixes2212
%   \end{syntax}
%   The \meta{cs} should be a single control sequence, such as |\foo|. This will
%   define \key\ to store \val\ inside of the control sequence. But unlike the
%   |store| type, the macro \meta{cs} will be a switch at the same time, it'll
%   take two arguments and if \key\ was used expands to the first argument
%   followed by \val\ in braces, if \key\ was not used \meta{cs} will expand to
%   the second argument (so behave like |\@secondoftwo|). The idea is that with
%   this type you can define a key which should be typeset formatted. The
%   |edata| and |xdata| variants will fully expand \val, the |gdata| and |xdata|
%   variants will store \val\ inside \meta{cs} globally. The |p|-prefixes will
%   only affect the key-macro, \meta{cs} will always be expandable and |\long|.
% \end{function}
%
% \begin{function}{dataT,edataT,gdataT,xdataT}
%   \begin{syntax}
%     dataT \key\ = \meta{cs} \prefixes2212
%   \end{syntax}
%   Just like |data|, but instead of \meta{cs} grabbing two arguments it'll only
%   grab one, so by default it'll behave like |\@gobble|, and if a \val\ was
%   given to \key\ the \meta{cs} will behave like |\@firstofone| appended by
%   \marg{value}.
% \end{function}
%
% \begin{function}{int,eint,gint,xint}
%   \begin{syntax}
%     int \key\ = \meta{cs} \prefixes2212
%   \end{syntax}
%   The \meta{cs} should be a single control sequence, such as |\foo|.
%   An |int| key will be a \TeX-count register. If \meta{cs} isn't defined yet,
%   |\newcount| will be used to initialise it. The |eint| and |xint| versions
%   will use |\numexpr| to allow basic computations in their \val. The |gint|
%   and |xint| variants set the register globally.
% \end{function}
%
% \begin{function}{dimen,edimen,gdimen,xdimen}
%   \begin{syntax}
%     dimen \key\ = \meta{cs} \prefixes2212
%   \end{syntax}
%   The \meta{cs} should be a single control sequence, such as |\foo|.
%   This is just like |int| but uses a dimen register, |\newdimen| and
%   |\dimexpr| instead.
% \end{function}
%
% \begin{function}{skip,eskip,gskip,xskip}
%   \begin{syntax}
%     skip \key\ = \meta{cs} \prefixes2212
%   \end{syntax}
%   The \meta{cs} should be a single control sequence, such as |\foo|.
%   This is just like |int| but uses a skip register, |\newskip| and |\glueexpr|
%   instead.
% \end{function}
%
% \begin{function}{toks,gtoks,apptoks,gapptoks,pretoks,gpretoks}
%   \begin{syntax}
%     toks \key\ = \meta{cs} \prefixes2212
%   \end{syntax}
%   The \meta{cs} should be a single control sequence, such as |\foo|.
%   Store \val\ inside of a toks-register. The |g| variants use |\global|, the
%   |app| variants append \val\ to the contents of that register, the |pre|
%   variants will prepend \val. If \meta{cs} is not yet defined it will be
%   initialised with |\newtoks|.
% \end{function}
%
% \begin{function}{box,gbox}
%   \begin{syntax}
%     box \key\ = \meta{cs} \prefixes2212
%   \end{syntax}
%   The \meta{cs} should be a single control sequence, such as |\foo|.
%   Typesets \val\ into a |\hbox| and stores the result in a box register. The
%   boxes are colour safe. \expkvd\ doesn't provide a |vbox| type.
% \end{function}
%
% \begin{function}{meta}
%   \begin{syntax}
%     meta \key\ = \{\kv, \ldots\} \prefixes2222
%   \end{syntax}
%   This key type can set other keys, you can access the \val\ which was passed
%   to \key\ inside the \kv\ list with |#1|. It works by calling a sub-|\ekvset|
%   on the \kv\ list, so a |set| key will only affect that \kv\ list and not the
%   current |\ekvset|. Since it runs in a separate |\ekvset| you can't use
%   |\ekvsneak| using keys or similar macros in the way you normally could.
% \end{function}
%
% \begin{function}{nmeta}
%   \begin{syntax}
%     nmeta \key\ = \{\kv, \ldots\} \prefixes2223
%   \end{syntax}
%   This key type can set other keys, the difference to |meta| is, that this key
%   doesn't take a value, so the \kv\ list is static.
% \end{function}
%
% \begin{function}{smeta}
%   \begin{syntax}
%     smeta \key\ = \marg{set}\{\kv, \ldots\} \prefixes2222
%   \end{syntax}
%   Yet another |meta| variant. An |smeta| key will take a \val\ which you can
%   access using |#1|, but it sets the \kv\ list inside of \set, so is equal to
%   \texttt{\cs[no-index]{ekvset}\marg{set}\{\kv, \ldots\}}.
% \end{function}
%
% \begin{function}{snmeta}
%   \begin{syntax}
%     snmeta \key\ = \marg{set}\{\kv, \ldots\} \prefixes2223
%   \end{syntax}
%   And the last |meta| variant. |snmeta| is a combination of |smeta| and
%   |nmeta|. It doesn't take an argument and sets the \kv\ list inside of \set.
% \end{function}
%
% \begin{function}{set}
%   \begin{syntax}
%     set \key\ = \marg{set} \prefixes2233
%   \end{syntax}
%   This will define \key\ to change the set of the current |\ekvset|
%   invocation to \set. You can omit \set\ (including the equals sign),
%   which is the same as using \texttt{set \key\ = \{\key\}}. The created |set|
%   key will not take a \val. Note that just like in \expkv\ it'll not be
%   checked whether \set\ is defined and you'll get a low-level \TeX\ error if
%   you use an undefined \set.
% \end{function}
%
% \begin{function}{choice}
%   \begin{syntax}
%     choice \key\ = \{\val=\meta{definition}, \ldots\} \prefixes2223
%   \end{syntax}
%   Defines \key\ to be a |choice| key, meaning it will only accept a limited
%   set of values. You should define each possible \val\ inside of the
%   \val=\meta{definition} list. If a defined \val\ is passed to \key\ the
%   \meta{definition} will be left in the input stream. You can make
%   individual values |protected| inside the \val=\meta{definition} list. By
%   default a |choice| key is expandable, an undefined \val\ will throw an error
%   in an expandable way (but see the |unknown-choice| prefix). You can add
%   additional choices after the \key\ was created by using |choice| again for
%   the same \key, redefining choices is possible the same way, but there is no
%   interface to remove certain choices.
% \end{function}
%
% \begin{function}{choice-store}
%   \begin{syntax}
%     choice-store \key\ = \meta{cs}\{\val, \ldots\} \prefixes2223
%   \end{syntax}
%   This is defines a special type of |choice| key that'll store the given
%   choice inside the macro \meta{cs} (so \meta{cs} should be a single control
%   sequence name such as |\foo|). Since storing inside a macro can't be done
%   expandably every choice-code is |\protected|, you might define the
%   |choice-store| key itself as |protected| as well if you want. Since the
%   definition of each choice is predefined with this key type the choice list
%   should just be a comma separated list of valid choices.
% \end{function}
% This means that the following |choice| and |choice-store| keys are equivalent
% at use time:
% \begin{lstlisting}
% \newcommand*\mya{}
% \ekvdefinekeys{example}
%   {
%      choice key1 = {a=\def\mya{a}, b=\def\mya{b}, c=\def\mya{c}}
%     ,choice-store key2 = \mya{a,b,c}
%   }
% \end{lstlisting}
%
% \begin{function}{choice-enum}
%   \begin{syntax}
%     choice-enum \key\ = \meta{cs}\{\val, \ldots\} \prefixes2223
%   \end{syntax}
%   This is similar to |choice-store|, the differences are: \meta{cs} should be
%   a count-register or is initialised as such if the \meta{cs} is undefined
%   (via |\newcount|); instead of the value the position of the value in the
%   given list is stored in this register (zero-based).
% \end{function}
% This means that the following |choice| and |choice-enum| keys are equivalent
% at use time:
% \begin{lstlisting}
% \newcount\myb
% \ekvdefinekeys{example}
%   {
%      choice key1 = {a={\myb=0 }, b={\myb=1 }, c={\myb=2 }}
%     ,choice-enum key2 = \myb{a,b,c}
%   }
% \end{lstlisting}
%
% \begin{function}{unknown-choice}
%   \begin{syntax}
%     unknown-choice \key\ = \marg{definition} \prefixes2323
%   \end{syntax}
%   By default an unknown \val\ passed to a |choice| or |bool| key will throw an
%   error.  However, with this prefix you can define an alternative action which
%   should be executed if \key\ received an unknown choice. In \meta{definition}
%   you can refer to the choice which was passed in with |#1|.
% \end{function}
%
% \begin{function}{unknown code}
%   \begin{syntax}
%     unknown code = \marg{definition} \prefixes2322
%   \end{syntax}
%   By default \expkv\ throws errors when it encounters unknown keys in a set.
%   With the |unknown| prefix you can define handlers that deal with undefined
%   keys, instead of a \key\ name you have to specify a subtype for this prefix,
%   here the subtype is |code|.\par
%   With |unknown code| the \meta{definition} is used for unknown keys which
%   were provided a value (so corresponds to |\ekvdefunknown|), you can access
%   the key name with |#1| and the value with |#2|.\footnotemark
% \end{function}
% \footnotetext
%   {^^A
%     There is some trickery involved to get this more intuitive argument order
%     without any performance hit if you compare this to
%     \cs[no-index]{ekvdefunknown} directly.^^A
%   }
%
% \begin{function}{unknown noval}
%   \begin{syntax}
%     unknown noval = \marg{definition} \prefixes2323
%   \end{syntax}
%   This is like |unknown code| but uses \meta{definition} for unknown keys to
%   which no value was passed (so corresponds to |\ekvdefunknownNoVal|). You can
%   access the key name with |#1|.
% \end{function}
%
% \begin{function}{unknown redirect-code}
%   \begin{syntax}
%     unknown redirect-code = \marg{set-list} \prefixes2331
%   \end{syntax}
%   This uses a predefined action for |unknown code|. Instead of throwing an
%   error, it is tried to find the \key\ in each \set\ in the comma separated
%   \meta{set-list}. The first found match will be used and the remaining
%   options from the list discarded. If the \key\ isn't found in any \set\ an
%   expandable error will be thrown eventually. Internally \expkv's
%   |\ekvredirectunknown| will be used.
% \end{function}
%
% \begin{function}{unknown redirect-noval}
%   \begin{syntax}
%     unknown redirect-noval = \marg{set-list} \prefixes2333
%   \end{syntax}
%   This behaves just like |unknown redirect-code| but will set up means to
%   forward keys for |unknown noval|. Internally \expkv's
%   |\ekvredirectunknownNoVal| will be used.
% \end{function}
%
% \begin{function}{unknown redirect}
%   \begin{syntax}
%     unknown redirect = \marg{set-list} \prefixes2333
%   \end{syntax}
%   This is a short cut to apply both, |unknown redirect-code| and
%   |unknown redirect-noval|, as a result you might get doubled error messages,
%   one from each.
% \end{function}
%
% \subsection{Bugs}
% I don't think there are any (but every developer says that), if you find some
% please let me know, either via the email address on the first page or on
% GitHub:
% \url{https://github.com/Skillmon/tex_expkv-def}
%
% \subsection{Example}
% The following is an example code defining each base key-type once. Please
% admire the very creative key-name examples.
%
% \begin{lstlisting}
% \ekvdefinekeys{example}
%   {
%      long code keyA = #1
%     ,noval     keyA = NoVal given
%     ,bool      keyB = \keyB
%     ,boolTF    keyC = \keyC
%     ,store     keyD = \keyD
%     ,data      keyE = \keyE
%     ,dataT     keyF = \keyF
%     ,int       keyG = \keyG
%     ,dimen     keyH = \keyH
%     ,skip      keyI = \keyI
%     ,toks      keyJ = \keyJ
%     ,default   keyJ = \empty test
%     ,new box   keyK = \keyK
%     ,qdefault  keyK = K
%     ,choice    keyL =
%       {
%          protected 1 = \texttt{a}
%         ,2 = b
%         ,3 = c
%         ,4 = d
%         ,5 = e
%       }
%     ,edefault  keyL = 2
%     ,meta      keyM = {keyA={#1},keyB=false}
%     ,invbool   keyN = \keyN
%     ,boolpair  keyO = \keyOa\keyOb
%   }
% \end{lstlisting}
%
% \bigskip
% Since the |data| type might be a bit strange, here is another usage example
% for it.
% \begin{lstlisting}
% \ekvdefinekeys{ex}
%   {
%      data  name  = \Pname
%     ,data  age   = \Page
%     ,dataT hobby = \Phobby
%   }
% \newcommand\Person[1]
%   {%
%     \begingroup
%     \ekvset{ex}{#1}%
%     \begin{description}
%       \item[\Pname{}{\errmessage{A person requires a name}}]
%       \item[Age] \Page{\textit}{\errmessage{A person requires an age}}
%       \Phobby{\item[Hobbies]}
%     \end{description}
%     \endgroup
%   }
% \Person{name=Jonathan P. Spratte, age=young, hobby=\TeX\ coding}
% \Person{name=Some User, age=unknown, hobby=Reading Documentation}
% \Person{name=Anybody, age=any}
% \end{lstlisting}
% In this example a person should have a name and an age, but doesn't have to
% have hobbies. The name will be displayed as the description item and the age
% in Italics. If a person has no hobbies the description item will be silently
% left out.  The result of the above code looks like this:
%
% \begin{exresult}{\linewidth}
% \ekvdefinekeys{ex}
%   {
%      data  name  = \Pname
%     ,data  age   = \Page
%     ,dataT hobby = \Phobby
%   }
% \newcommand\Person[1]
%   {%
%     \begingroup
%     \ekvset{ex}{#1}%
%     \begin{description}
%       \item[\Pname{}{\errmessage{A person requires a name}}]
%       \item[Age] \Page{\textit}{\errmessage{A person requires an age}}
%       \Phobby{\item[Hobbies]}
%     \end{description}
%     \endgroup
%   }
% \Person{name=Jonathan P. Spratte, age=young, hobby=\TeX\ coding}
% \Person{name=Some User, age=unknown, hobby=Reading Documentation}
% \Person{name=Anybody, age=any}
% \end{exresult}
%
% \subsection{License}
%
% Copyright \textcopyright\ 2020\unless\ifnum\year=2020--\the\year\fi\
% Jonathan P. Spratte
%
% \medskip\noindent
% This work may be distributed and/or modified under the conditions of the
% \LaTeX\ Project Public License (LPPL), either version 1.3c of this license or
% (at your option) any later version. The latest version of this license is in
% the file:
%
% \url{http://www.latex-project.org/lppl.txt}
%
% \noindent
% This work is ``maintained'' (as per LPPL maintenance status) by
%
% \mbox{Jonathan P. Spratte}.
%
% \end{documentation}^^A=<<
%
% \begin{implementation}^^A>>=
%
% \clearpage
%
% \section{Implementation}^^A>>=
%^^A the LaTeX package >>=
% \subsection{The \LaTeX\ Package}
% Just like for \expkv\ we provide a small \LaTeX\ package that sets up things
% such that we behave nicely on \LaTeX\ packages and files system. It'll
% |\input| the generic code which implements the functionality.
% \gobbledocstriptag
%<*pkg>
%    \begin{macrocode}
\RequirePackage{expkv}
\def\ekvd@tmp
  {%
    \ProvidesFile{expkv-def.tex}%
      [\ekvdDate\space v\ekvdVersion\space a key-defining frontend for expkv]%
  }
% \iffalse meta-comment
%
% File: expkv-def.dtx Copyright (C) 2020-2021 Jonathan P. Spratte
%
% This work  may be  distributed and/or  modified under  the conditions  of the
% LaTeX Project Public License (LPPL),  either version 1.3c  of this license or
% (at your option) any later version.  The latest version of this license is in
% the file:
%
%   http://www.latex-project.org/lppl.txt
%
% ------------------------------------------------------------------------------
%
%<*driver>^^A>>=
\def\nameofplainTeX{plain}
\ifx\fmtname\nameofplainTeX\else
  \expandafter\begingroup
\fi
\input l3docstrip.tex
\askforoverwritefalse
\preamble

--------------------------------------------------------------
expkv-def -- a key-defining frontend for expkv
E-mail: jspratte@yahoo.de
Released under the LaTeX Project Public License v1.3c or later
See http://www.latex-project.org/lppl.txt
--------------------------------------------------------------

Copyright (C) 2020-2021 Jonathan P. Spratte

This  work may be  distributed and/or  modified under  the conditions  of the
LaTeX Project Public License (LPPL),  either version 1.3c  of this license or
(at your option) any later version.  The latest version of this license is in
the file:

  http://www.latex-project.org/lppl.txt

This work is "maintained" (as per LPPL maintenance status) by
  Jonathan P. Spratte.

This work consists of the file  expkv-def.dtx
and the derived files           expkv-def.pdf
                                expkv-def.sty
                                expkv-def.tex
                                t-expkv-def.tex

\endpreamble
% stop docstrip adding \endinput
\postamble
\endpostamble
\generate{\file{expkv-def.sty}{\from{expkv-def.dtx}{pkg}}}
\generate{\file{expkv-def.tex}{\from{expkv-def.dtx}{tex}}}
\generate{\file{t-expkv-def.tex}{\from{expkv-def.dtx}{ctx}}}
\ifx\fmtname\nameofplainTeX
  \expandafter\endbatchfile
\else
  \expandafter\endgroup
\fi
%
\IfFileExists{expkv-def.tex}{\input{expkv-def.tex}}{}
\ProvidesFile{expkv-def.dtx}
  [\csname ekvdDate\endcsname\ a key-defining frontend for expkv]
\PassOptionsToPackage{full}{textcomp}
\documentclass{l3doc}
\RequirePackage[oldstylenums,nott]{kpfonts}
\input{glyphtounicode}
\pdfgentounicode=1
\RequirePackage{tcolorbox}
\newtcolorbox{exresult}[2][]
  {colback=expkvgrey!10!white,colframe=expkvgrey,fontupper=\small,width={#2},#1}
\newtcbox\exres[1][]
  {
     colback=expkvgrey!10!white
    ,colframe=expkvgrey
    ,size=small
    ,nobeforeafter
    ,tcbox raise base
    ,fontupper=\small
    ,#1
  }
\RequirePackage{listings}
\RequirePackage{booktabs}
\RequirePackage{array}
\RequirePackage{collcell}
\RequirePackage{siunitx}
\RequirePackage{xcolor}
\RequirePackage{caption}
\RequirePackage{microtype}
\RequirePackage{accsupp}
\lstdefinelanguage{expkvd}[primitive]{TeX}
  {
    ,moretexcs=[4]^^A e-TeX
      {
        expanded,
        numexpr,
        protected,
      }
    ,moretexcs=[5]^^A plain/LaTeX
      {
        approx,
        begin,
        empty,
        item,
        LaTeX,
        makeatletter,makeatother,
        newcommand,newdimen,
        RequirePackage,
        rule,
        TeX,
        textit,texttt,
        usepackage,
      }
    ,moretexcs=[6]^^A used packages
      {
        ^^A xfp
        fpeval,
        ^^A keyval
        setkeys,
        ^^A yax
        defactiveparameter,storevalue,setparameterlist
      }
    ,moretexcs=[2]^^A expkv-def macros
      {
        ekvdefinekeys
      }
    ,moretexcs=[3]^^A expkv-pkg macros
      {
        ^^A expkv
        ekvdef,ekvdefNoVal,^^A
        ekvlet,ekvletNoVal,ekvletkv,ekvletkvNoVal,^^A
        ekvdefunknown,ekvdefunknownNoVal,^^A
        ekvredirectunknown,ekvredirectunknownNoVal,^^A
        ekvset,ekvsetSneaked,^^A
        ekvsetdef,ekvsetSneakeddef,ekvsetdefSneaked,^^A
        ekvparse,^^A
        ekvVersion,ekvDate,^^A
        ekvifdefined,ekvifdefinedNoVal,ekvifdefinedset,^^A
        ekvbreak,ekvbreakPreSneak,ekvbreakPostSneak,^^A
        ekvsneak,ekvsneakPre,^^A
        ekvchangeset,^^A
        ekvoptarg,ekvoptargTF,^^A
        ekverr,
        ^^A expkv-cs
        ekvcSecondaryKeys,ekvcSplitAndForward,
      }
  }
\colorlet{codeparam}{cyan!65!black}
\def\mylstwd{.55em}
\lstset
  {
    ,language=expkvd
    ,flexiblecolumns=false
    ,basewidth=\mylstwd
    ,gobble=2
    ,basicstyle=\fontfamily{jkp}\itshape
    ,texcsstyle=*[1]{\bfseries\color{expkvgrey}} ^^A primitives
    ,texcsstyle=*[2]{\bfseries\color{expkvred}}  ^^A expkv
    ,texcsstyle=*[3]{\color{expkvred}}           ^^A expkv-pkg
    ,texcsstyle=*[4]{\bfseries\color{expkvgrey}} ^^A e-TeX
    ,texcsstyle=*[5]{\bfseries\color{expkvgrey}} ^^A plain/LaTeX macros
    ,texcsstyle=*[6]{}                           ^^A macros of other packages
    ,commentstyle=\color[gray]{0.4}
    ,literate=
              {\{} {{\CodeUpBf\{}}{1}
              {\}} {{\CodeUpBf\}}}{1}
              {$}  {{\CodeUpBf\$}}{1}
              {[}  {{\CodeUp[}}{1}
              {]}  {{\CodeUp]}}{1}
              {(}  {{\CodeUp(}}{1}
              {)}  {{\CodeUp)}}{1}
              {*}  {{$*$}}{1}
              {1}  {{\CodeUp{1}}}{1}
              {2}  {{\CodeUp{2}}}{1}
              {3}  {{\CodeUp{3}}}{1}
              {4}  {{\CodeUp{4}}}{1}
              {5}  {{\CodeUp{5}}}{1}
              {6}  {{\CodeUp{6}}}{1}
              {7}  {{\CodeUp{7}}}{1}
              {8}  {{\CodeUp{8}}}{1}
              {9}  {{\CodeUp{9}}}{1}
              {0}  {{\CodeUp{0}}}{1}
              {##} {{\CodeColored{codeparam}{1}{\#}}}{1}
              {##1}{{\CodeColored{codeparam}{2}{\#1}}}{2}
              {##2}{{\CodeColored{codeparam}{2}{\#2}}}{2}
              {##3}{{\CodeColored{codeparam}{2}{\#3}}}{2}
              {##4}{{\CodeColored{codeparam}{2}{\#4}}}{2}
              {##5}{{\CodeColored{codeparam}{2}{\#5}}}{2}
              {##6}{{\CodeColored{codeparam}{2}{\#6}}}{2}
              {##7}{{\CodeColored{codeparam}{2}{\#7}}}{2}
              {##8}{{\CodeColored{codeparam}{2}{\#8}}}{2}
              {##9}{{\CodeColored{codeparam}{2}{\#9}}}{2}
              {<key>}{{$\langle$}key{$\rangle$}}{5}
              {<set>}{{$\langle$}set{$\rangle$}}{5}
              {<set1>}{{$\langle$}set1{$\rangle$}}{6}
              {<set2>}{{$\langle$}set2{$\rangle$}}{6}
    ^^A,literate=*{<key>}{\key}{4}{<set>}{\set}{4}
  }
\newcommand*\CodeColored[3]{\textcolor{#1}{\makebox[\dimexpr\mylstwd*#2]{#3}}}
\newcommand*\CodeUpBf[1]{\makebox[\mylstwd]{\textup{\textbf{#1}}}}
\newcommand*\CodeUp[1]{\makebox[\mylstwd]{\textup{#1}}}
\RequirePackage{randtext}
\let\metaORIG\meta
\protected\def\meta #1{\texttt{\metaORIG{#1}}}
\renewcommand*\thefootnote{\fnsymbol{footnote}}
\definecolor{expkvred}{HTML}{9F393D}
\colorlet{expkvgrey}{black!75}
\makeatletter
\newcommand*\expkv
  {^^A
    \texorpdfstring
      {^^A
        \mbox
          {^^A
            \BeginAccSupp{ActualText=expkv}^^A
            \href{https://github.com/Skillmon/tex_expkv}
              {^^A
                \rmfamily
                \bfseries
                {\color{expkvgrey}e\kern-.05em x\kern-.05em}^^A
                \lower.493ex
                  \hbox{{\color{expkvgrey}P}\kern-.1em{\color{expkvred}k}}^^A
                \kern-.18em{\color{expkvred}v}^^A
              }^^A
            \EndAccSupp{}^^A
          }^^A
      }
      {expkv}^^A
  }
\newcommand*\expkvd
  {^^A
    \texorpdfstring
      {^^A
        \mbox
          {^^A
            \BeginAccSupp{ActualText=expkv-def}^^A
            \href{https://github.com/Skillmon/tex_expkv-def}
              {^^A
                \rmfamily
                \bfseries
                {\color{expkvgrey}e\kern-.05em x\kern-.05em}^^A
                \lower.493ex
                  \hbox{{\color{expkvgrey}P}\kern-.1em{\color{expkvred}k}}^^A
                \kern-.18em{\color{expkvred}v}^^A
                {\color{expkvgrey}^^A
                  \kern.05em\rule[-.1ex]{.08em}{1.2ex}\kern.05em\textsc{def}^^A
                }^^A
              }^^A
            \EndAccSupp{}^^A
          }^^A
      }
      {expkv-def}^^A
  }
\newcommand\kv{\meta{key}=\meta{value}}
\newcommand\key{\meta{key}}
\newcommand\val{\meta{value}}
\newcommand\set{\meta{set}}
\newcommand\enfprefix{\textcolor{black}}
\newcommand\allprefix{\textcolor{gray}}
\newcommand\notprefix{\textcolor{red!80!black}}
\newcommand\printprefix[2]
  {^^A
    \ifcase\numexpr#1\relax\or
      \enfprefix{#2}\or
      \allprefix{#2}\or
      \notprefix{#2}\fi
  }
\newcommand\prefixes[4]
  {^^A
    \hfill
    \printprefix{#1}{new}
    \printprefix{#2}{also}
    \printprefix{#3}{protected}
    \printprefix{#4}{long}^^A
  }
\hypersetup{linkcolor=red!80!black,urlcolor=purple!80!black}
\DoNotIndex{\def,\edef,\,,\=,\begingroup,\catcode,\chardef,\csname,\endcsname}
\DoNotIndex{\endgroup,\endinput,\errmessage,\expandafter,\input,\let,\long}
\DoNotIndex{\protected,\ProvidesFile,\ProvidesPackage,\relax,\space}
\DoNotIndex{\@,\unexpanded,\string,\expanded,\dimexpr,\global,\glueexpr,\hbox}
\DoNotIndex{\numexpr,\RequirePackage,\setbox,\the,\unless,\xdef,\@firstofone}
\DoNotIndex{\@firstoftwo,\@gobble,\@secondoftwo,\meaning,\detokenize}
\DoNotIndex{\romannumeral,\^}
\DoNotIndex{\ifcsname}
\DoNotIndex{\ifx}
\DoNotIndex{\ifdefined}
\DoNotIndex{\iffalse}
\DoNotIndex{\iftrue}
\DoNotIndex{\else}
\DoNotIndex{\fi}
\@gobble\fi ^^A ignoring \ifx, \ifcsname, etc., but only one \fi
\@gobble\fi ^^A ignoring \ifx, \ifcsname, etc., but only one \fi
\@gobble\fi ^^A ignoring \ifx, \ifcsname, etc., but only one \fi
\@gobble\fi ^^A ignoring \ifx, \ifcsname, etc., but only one \fi
\@ifdefinable\gobbledocstriptag{\def\gobbledocstriptag#1>{}}
\makeatother
\begin{document}
  \title
    {^^A
      \texorpdfstring
        {^^A
          \huge\expkvd
          \\[\medskipamount]
          \Large a key-defining frontend for \expkv
        }{expkv-def - a key-defining frontend for expkv}^^A
    }
  \date{\ekvdDate\space v\ekvdVersion}
  \author{Jonathan P. Spratte\thanks{\protect\randomize{jspratte@yahoo.de}}}
  \DocInput{expkv-def.dtx}
\end{document}
%</driver>^^A=<<
% \fi
%
% \maketitle
% \renewcommand*\thefootnote{\arabic{footnote}}
%
% \begin{abstract}
% \noindent\parfillskip=0pt
% \expkvd\ provides a small \kv\ interface to define keys for \expkv. Key-types
% are declared using prefixes, similar to static typed languages. The stylised
% name is \expkvd\ but the files use \file{expkv-def}, this is due to CTAN-rules
% which don't allow \string| in package names since that is the pipe symbol in
% *nix shells.
% \end{abstract}
%
% \tableofcontents
%
% \begin{documentation}^^A>>=
%
% \section{Documentation}
%
% Since the trend for the last couple of years goes to defining keys for a \kv\
% interface using a \kv\ interface, I thought that maybe providing such an
% interface for \expkv\ will make it more attractive for actual use, besides its
% unique selling points of being fully expandable, and fast and reliable. But at
% the same time I don't want to widen \expkv's initial scope. So here it is
% \expkvd, go define \kv\ interfaces with \kv\ interfaces.
%
% Unlike many of the other established \kv\ interfaces to define keys, \expkvd\
% works using prefixes instead of suffixes (\emph{e.g.}, |.tl_set:N| of
% \pkg{l3keys}) or directory like handlers (\emph{e.g.}, |/.store in| of
% \pkg{pgfkeys}). This was decided as a personal preference, more over in \TeX\
% parsing for the first space is way easier than parsing for the last one.
% \expkvd's prefixes are sorted into two categories: |p|-type, which are
% equivalent to \TeX's prefixes like |\long|, and |t|-type defining the type of
% the key. For a description of the available |p|-prefixes take a look at
% \autoref{sec:prefix:p}, the |t|-prefixes are described in
% \autoref{sec:prefix:t}.
%
% \expkvd\ is usable as generic code, as a \LaTeX\ package, and as a \ConTeXt\
% module. It'll automatically load \expkv\ in the same mode as well. To use it,
% just use one of
% \begin{lstlisting}
% \input expkv-def       % plainTeX
% \usepackage{expkv-def} % LaTeX
% \usemodule[expkv-def]  % ConTeXt
% \end{lstlisting}
%
% \subsection{Macros}
% Apart from version and date containers there is only a single user-facing
% macro, and that should be used to define keys.
%
% \begin{function}{\ekvdefinekeys}
%   \begin{syntax}
%     \cs{ekvdefinekeys}\marg{set}\{\kv, \ldots\}
%   \end{syntax}
%   In \set, define \key\ to have definition \val. The general syntax for \key\
%   should be
%   \begin{quote}\ttfamily\small
%     \meta{prefix} \meta{name}
%   \end{quote}
%   Where \meta{prefix} is a space separated list of optional |p|-type prefixes
%   followed by one |t|-type prefix. The syntax of \val\ is dependent on the
%   used |t|-prefix.
% \end{function}
%
% \begin{function}{\ekvdDate,\ekvdVersion}
%   These two macros store the version and date of the package.
% \end{function}
%
% \subsection{Prefixes}
% As already said there are |p|-prefixes and |t|-prefixes. Not every |p|-prefix
% is allowed for all |t|-prefixes.
%
% \subsubsection{\texttt{p}-Prefixes}\label{sec:prefix:p}
% The two |p|-type prefixes |long| and |protected| are pretty simple by nature,
% so their description is pretty simple. They affect the \key\ at use-time, so
% omitting |long| doesn't mean that a \meta{definition} can't contain a |\par|
% token, only that the \key\ will not accept a |\par| in \val.  On the other
% hand |new| and |also| might be simple on first sight as well, but their rules
% are a bit more complicated.
%
% \begin{function}{also}
%   The following key type will be \emph{added} to an existing \key's
%   definition. You can't add a type taking an argument at use time to an
%   existing key which doesn't take an argument and vice versa. Also you'll get
%   an error if you try to add an action which isn't allowed to be either |long|
%   or |protected| to a key which already is |long| or |protected| (the opposite
%   order would be suboptimal as well, but can't be really captured with the
%   current code).
%
%   A key already defined as |long| or |protected| will stay |long| or
%   |protected|, but you can as well add |long| or |protected| with the |also|
%   definition.
%
%   As a small example, suppose you want to create a boolean key, but
%   additionally to setting a boolean value you want to execute some more code
%   as well, you can use the following
% \begin{lstlisting}
% \ekvdefinekeys{also-example}
%   {
%      bool key      = \ifmybool
%     ,also code key = \domystuff{#1}
%   }
% \end{lstlisting}
%
%   If you use |also| on a |choice|, |bool|, |invbool|, or |boolpair| key it is
%   tried to determine if the key already is of one of those types. If this test
%   is true the declared choices will be added to the possible choices but the
%   key's definition will not be changed other than that. If that wouldn't have
%   been done, the callbacks of the different choices could get called multiple
%   times.
% \end{function}
%
% \begin{function}{protected,protect}
%   The following key will be defined |\protected|. Note that key-types which
%   can't be defined expandable will always use |\protected|.
% \end{function}
%
% \begin{function}{long}
%   The following key will be defined |\long|.
% \end{function}
%
% \begin{function}{new}
%   The following key must be new (so previously undefined). An error is thrown
%   if it is already defined and the new definition is ignored. |new| only
%   asserts that there are no conflicts between |NoVal| keys and other |NoVal|
%   keys or value taking keys and other value taking keys. For example you can
%   use the following without an error:
% \begin{lstlisting}
% \ekvdefinekeys{new-example}
%   {
%      code key       = \domystuffwitharg{#1}
%     ,new  noval key = \domystuffwithoutarg
%   }
% \end{lstlisting}
% \end{function}
%
% \subsubsection{\texttt{t}-Prefixes}\label{sec:prefix:t}
% Since the |p|-type prefixes apply to some of the |t|-prefixes automatically
% but sometimes one might be disallowed we need some way to highlight this
% behaviour. In the following an enforced prefix will be printed black
% (\texttt{\enfprefix{protected}}), allowed prefixes will be grey
% (\texttt{\allprefix{protected}}), and disallowed prefixes will be red
% (\texttt{\notprefix{protected}}). This will be put flush-right in the syntax
% showing line.
%
% \begin{function}{code,ecode}
%   \begin{syntax}
%     code \key\ = \marg{definition} \prefixes2222
%   \end{syntax}
%   Define \key\ to expand to \meta{definition}. The \key\ will require a \val\
%   for which you can use |#1| inside \meta{definition}. The |ecode| variant
%   will fully expand \meta{definition} inside an |\edef|.
% \end{function}
%
% \begin{function}{noval,enoval}
%   \begin{syntax}
%     noval \key\ = \marg{definition} \prefixes2223
%   \end{syntax}
%   The |noval| type defines \key\ to expand to \meta{definition}. The \key\
%   will not take a \val. |enoval| fully expands \meta{definition} inside an
%   |\edef|.
% \end{function}
%
% \begin{function}{default,qdefault,odefault,fdefault,edefault}
%   \begin{syntax}
%     default \key\ = \marg{definition} \prefixes3223
%   \end{syntax}
%   This serves to place a default \val\ for a \key\ that takes an argument, the
%   \key\ can be of any argument-grabbing kind, and when used without a \val\
%   it will be passed \meta{definition} instead. The |qdefault| variant will
%   expand the \key's code once, so will be slightly quicker, but not change if
%   you redefine \key. |odefault| is just another name for |qdefault|. The
%   |fdefault| version will expand the key code until a non-expandable token or
%   a space is found, a space would be gobbled.\footnotemark{} The |edefault| on
%   the other hand fully expands the \key-code with \meta{definition} as its
%   argument inside of an |\edef|.
% \end{function}
% \footnotetext
%   {^^A
%     For those familiar with \TeX-coding: This uses a
%     \cs[no-index]{romannumeral}-expansion.
%   }
%
% \begin{function}{initial,oinitial,finitial,einitial}
%   \begin{syntax}
%     initial \key\ = \{\val\} \prefixes3333
%     initial \key
%   \end{syntax}
%   With |initial| you can set an initial \val\ for an already defined argument
%   taking \key. It'll just call the key-macro of \key\ and pass it \val. The
%   |einitial| variant will expand \val\ using an |\edef| expansion prior to
%   passing it to the key-macro and the |oinitial| variant will expand the first
%   token in \val\ once. |finitial| will expand \val\ until a non-expandable
%   token or a space is found, a space would be
%   gobbled.\footnotemark
%
%   If you don't provide a value (and no equals sign) a |noval| \key\ of the
%   same name is called once (or, if you specified a |default| for a value
%   taking key that would be used).
% \end{function}
% \footnotetext{Again using \cs[no-index]{romannumeral}.}
%
% \begin{function}{bool,gbool,boolTF,gboolTF}
%   \begin{syntax}
%     bool \key\ = \meta{cs} \prefixes2223
%   \end{syntax}
%   The \meta{cs} should be a single control sequence, such as |\iffoo|.
%   This will define \key\ to be a boolean key, which only takes the values
%   |true| or |false| and will throw an error for other values. If the key is
%   used without a \val\ it'll have the same effect as if you use
%   \texttt{\key=true}. |bool| and |gbool| will behave like \TeX-ifs so either
%   be |\iftrue| or |\iffalse|. The |boolTF| and |gboolTF| variants will both
%   take two arguments and if true the first will be used else the second, so
%   they are always either |\@firstoftwo| or |\@secondoftwo|. The variants with
%   a leading |g| will set the control sequence globally, the others locally. If
%   \meta{cs} is not yet defined it'll be initialised as the |false| version.
%   Note that the initialisation is \emph{not} done with |\newif|, so you will
%   not be able to do |\footrue| outside of the \kv\ interface, but you could
%   use |\newif| yourself. Even if the \key\ will not be |\protected| the
%   commands which execute the |true| or |false| choice will be, so the usage
%   should be safe in an expansion context (\emph{e.g.}, you can use
%   \texttt{edefault \key\ = false} without an issue to change the default
%   behaviour to execute the |false| choice). Internally a |bool| \key\ is the
%   same as a |choice| key which is set up to handle |true| and |false| as
%   choices.
% \end{function}
%
% \begin{function}{invbool,ginvbool,invboolTF,ginvboolTF}
%   \begin{syntax}
%     bool \key\ = \meta{cs} \prefixes2223
%   \end{syntax}
%   These are inverse boolean keys, they behave like |bool| and friends but set
%   the opposite meaning to the macro \meta{cs} in each case. So if
%   \texttt{key=true} is used |invbool| will set \meta{cs} to |\iffalse| and
%   vice versa.
% \end{function}
%
% \begin{function}{boolpair,gboolpair,boolpairTF,gboolpairTF}
%   \begin{syntax}
%     boolpair \key\ = \meta{cs_1}\meta{cs_2} \prefixes2223
%   \end{syntax}
%   The |boolpair| key type behaves like both |bool| and |invbool|, the
%   \meta{cs_1} will be set to the meaning according to the rules of |bool|, and
%   \meta{cs_2} will be set to the opposite.
% \end{function}
%
% \begin{function}{store,estore,gstore,xstore}
%   \begin{syntax}
%     store \key\ = \meta{cs} \prefixes2212
%   \end{syntax}
%   The \meta{cs} should be a single control sequence, such as |\foo|.
%   This will define \key\ to store \val\ inside of the control sequence. If
%   \meta{cs} isn't yet defined it will be initialised as empty. The variants
%   behave similarly to their |\def|, |\edef|, |\gdef|, and |\xdef|
%   counterparts, but |store| and |gstore| will allow you to store macro
%   parameters inside of them by using |\unexpanded|.
% \end{function}
%
% \begin{function}{data,edata,gdata,xdata}
%   \begin{syntax}
%     data \key\ = \meta{cs} \prefixes2212
%   \end{syntax}
%   The \meta{cs} should be a single control sequence, such as |\foo|. This will
%   define \key\ to store \val\ inside of the control sequence. But unlike the
%   |store| type, the macro \meta{cs} will be a switch at the same time, it'll
%   take two arguments and if \key\ was used expands to the first argument
%   followed by \val\ in braces, if \key\ was not used \meta{cs} will expand to
%   the second argument (so behave like |\@secondoftwo|). The idea is that with
%   this type you can define a key which should be typeset formatted. The
%   |edata| and |xdata| variants will fully expand \val, the |gdata| and |xdata|
%   variants will store \val\ inside \meta{cs} globally. The |p|-prefixes will
%   only affect the key-macro, \meta{cs} will always be expandable and |\long|.
% \end{function}
%
% \begin{function}{dataT,edataT,gdataT,xdataT}
%   \begin{syntax}
%     dataT \key\ = \meta{cs} \prefixes2212
%   \end{syntax}
%   Just like |data|, but instead of \meta{cs} grabbing two arguments it'll only
%   grab one, so by default it'll behave like |\@gobble|, and if a \val\ was
%   given to \key\ the \meta{cs} will behave like |\@firstofone| appended by
%   \marg{value}.
% \end{function}
%
% \begin{function}{int,eint,gint,xint}
%   \begin{syntax}
%     int \key\ = \meta{cs} \prefixes2212
%   \end{syntax}
%   The \meta{cs} should be a single control sequence, such as |\foo|.
%   An |int| key will be a \TeX-count register. If \meta{cs} isn't defined yet,
%   |\newcount| will be used to initialise it. The |eint| and |xint| versions
%   will use |\numexpr| to allow basic computations in their \val. The |gint|
%   and |xint| variants set the register globally.
% \end{function}
%
% \begin{function}{dimen,edimen,gdimen,xdimen}
%   \begin{syntax}
%     dimen \key\ = \meta{cs} \prefixes2212
%   \end{syntax}
%   The \meta{cs} should be a single control sequence, such as |\foo|.
%   This is just like |int| but uses a dimen register, |\newdimen| and
%   |\dimexpr| instead.
% \end{function}
%
% \begin{function}{skip,eskip,gskip,xskip}
%   \begin{syntax}
%     skip \key\ = \meta{cs} \prefixes2212
%   \end{syntax}
%   The \meta{cs} should be a single control sequence, such as |\foo|.
%   This is just like |int| but uses a skip register, |\newskip| and |\glueexpr|
%   instead.
% \end{function}
%
% \begin{function}{toks,gtoks,apptoks,gapptoks,pretoks,gpretoks}
%   \begin{syntax}
%     toks \key\ = \meta{cs} \prefixes2212
%   \end{syntax}
%   The \meta{cs} should be a single control sequence, such as |\foo|.
%   Store \val\ inside of a toks-register. The |g| variants use |\global|, the
%   |app| variants append \val\ to the contents of that register, the |pre|
%   variants will prepend \val. If \meta{cs} is not yet defined it will be
%   initialised with |\newtoks|.
% \end{function}
%
% \begin{function}{box,gbox}
%   \begin{syntax}
%     box \key\ = \meta{cs} \prefixes2212
%   \end{syntax}
%   The \meta{cs} should be a single control sequence, such as |\foo|.
%   Typesets \val\ into a |\hbox| and stores the result in a box register. The
%   boxes are colour safe. \expkvd\ doesn't provide a |vbox| type.
% \end{function}
%
% \begin{function}{meta}
%   \begin{syntax}
%     meta \key\ = \{\kv, \ldots\} \prefixes2222
%   \end{syntax}
%   This key type can set other keys, you can access the \val\ which was passed
%   to \key\ inside the \kv\ list with |#1|. It works by calling a sub-|\ekvset|
%   on the \kv\ list, so a |set| key will only affect that \kv\ list and not the
%   current |\ekvset|. Since it runs in a separate |\ekvset| you can't use
%   |\ekvsneak| using keys or similar macros in the way you normally could.
% \end{function}
%
% \begin{function}{nmeta}
%   \begin{syntax}
%     nmeta \key\ = \{\kv, \ldots\} \prefixes2223
%   \end{syntax}
%   This key type can set other keys, the difference to |meta| is, that this key
%   doesn't take a value, so the \kv\ list is static.
% \end{function}
%
% \begin{function}{smeta}
%   \begin{syntax}
%     smeta \key\ = \marg{set}\{\kv, \ldots\} \prefixes2222
%   \end{syntax}
%   Yet another |meta| variant. An |smeta| key will take a \val\ which you can
%   access using |#1|, but it sets the \kv\ list inside of \set, so is equal to
%   \texttt{\cs[no-index]{ekvset}\marg{set}\{\kv, \ldots\}}.
% \end{function}
%
% \begin{function}{snmeta}
%   \begin{syntax}
%     snmeta \key\ = \marg{set}\{\kv, \ldots\} \prefixes2223
%   \end{syntax}
%   And the last |meta| variant. |snmeta| is a combination of |smeta| and
%   |nmeta|. It doesn't take an argument and sets the \kv\ list inside of \set.
% \end{function}
%
% \begin{function}{set}
%   \begin{syntax}
%     set \key\ = \marg{set} \prefixes2233
%   \end{syntax}
%   This will define \key\ to change the set of the current |\ekvset|
%   invocation to \set. You can omit \set\ (including the equals sign),
%   which is the same as using \texttt{set \key\ = \{\key\}}. The created |set|
%   key will not take a \val. Note that just like in \expkv\ it'll not be
%   checked whether \set\ is defined and you'll get a low-level \TeX\ error if
%   you use an undefined \set.
% \end{function}
%
% \begin{function}{choice}
%   \begin{syntax}
%     choice \key\ = \{\val=\meta{definition}, \ldots\} \prefixes2223
%   \end{syntax}
%   Defines \key\ to be a |choice| key, meaning it will only accept a limited
%   set of values. You should define each possible \val\ inside of the
%   \val=\meta{definition} list. If a defined \val\ is passed to \key\ the
%   \meta{definition} will be left in the input stream. You can make
%   individual values |protected| inside the \val=\meta{definition} list. By
%   default a |choice| key is expandable, an undefined \val\ will throw an error
%   in an expandable way (but see the |unknown-choice| prefix). You can add
%   additional choices after the \key\ was created by using |choice| again for
%   the same \key, redefining choices is possible the same way, but there is no
%   interface to remove certain choices.
% \end{function}
%
% \begin{function}{choice-store}
%   \begin{syntax}
%     choice-store \key\ = \meta{cs}\{\val, \ldots\} \prefixes2223
%   \end{syntax}
%   This is defines a special type of |choice| key that'll store the given
%   choice inside the macro \meta{cs} (so \meta{cs} should be a single control
%   sequence name such as |\foo|). Since storing inside a macro can't be done
%   expandably every choice-code is |\protected|, you might define the
%   |choice-store| key itself as |protected| as well if you want. Since the
%   definition of each choice is predefined with this key type the choice list
%   should just be a comma separated list of valid choices.
% \end{function}
% This means that the following |choice| and |choice-store| keys are equivalent
% at use time:
% \begin{lstlisting}
% \newcommand*\mya{}
% \ekvdefinekeys{example}
%   {
%      choice key1 = {a=\def\mya{a}, b=\def\mya{b}, c=\def\mya{c}}
%     ,choice-store key2 = \mya{a,b,c}
%   }
% \end{lstlisting}
%
% \begin{function}{choice-enum}
%   \begin{syntax}
%     choice-enum \key\ = \meta{cs}\{\val, \ldots\} \prefixes2223
%   \end{syntax}
%   This is similar to |choice-store|, the differences are: \meta{cs} should be
%   a count-register or is initialised as such if the \meta{cs} is undefined
%   (via |\newcount|); instead of the value the position of the value in the
%   given list is stored in this register (zero-based).
% \end{function}
% This means that the following |choice| and |choice-enum| keys are equivalent
% at use time:
% \begin{lstlisting}
% \newcount\myb
% \ekvdefinekeys{example}
%   {
%      choice key1 = {a={\myb=0 }, b={\myb=1 }, c={\myb=2 }}
%     ,choice-enum key2 = \myb{a,b,c}
%   }
% \end{lstlisting}
%
% \begin{function}{unknown-choice}
%   \begin{syntax}
%     unknown-choice \key\ = \marg{definition} \prefixes2323
%   \end{syntax}
%   By default an unknown \val\ passed to a |choice| or |bool| key will throw an
%   error.  However, with this prefix you can define an alternative action which
%   should be executed if \key\ received an unknown choice. In \meta{definition}
%   you can refer to the choice which was passed in with |#1|.
% \end{function}
%
% \begin{function}{unknown code}
%   \begin{syntax}
%     unknown code = \marg{definition} \prefixes2322
%   \end{syntax}
%   By default \expkv\ throws errors when it encounters unknown keys in a set.
%   With the |unknown| prefix you can define handlers that deal with undefined
%   keys, instead of a \key\ name you have to specify a subtype for this prefix,
%   here the subtype is |code|.\par
%   With |unknown code| the \meta{definition} is used for unknown keys which
%   were provided a value (so corresponds to |\ekvdefunknown|), you can access
%   the key name with |#1| and the value with |#2|.\footnotemark
% \end{function}
% \footnotetext
%   {^^A
%     There is some trickery involved to get this more intuitive argument order
%     without any performance hit if you compare this to
%     \cs[no-index]{ekvdefunknown} directly.^^A
%   }
%
% \begin{function}{unknown noval}
%   \begin{syntax}
%     unknown noval = \marg{definition} \prefixes2323
%   \end{syntax}
%   This is like |unknown code| but uses \meta{definition} for unknown keys to
%   which no value was passed (so corresponds to |\ekvdefunknownNoVal|). You can
%   access the key name with |#1|.
% \end{function}
%
% \begin{function}{unknown redirect-code}
%   \begin{syntax}
%     unknown redirect-code = \marg{set-list} \prefixes2331
%   \end{syntax}
%   This uses a predefined action for |unknown code|. Instead of throwing an
%   error, it is tried to find the \key\ in each \set\ in the comma separated
%   \meta{set-list}. The first found match will be used and the remaining
%   options from the list discarded. If the \key\ isn't found in any \set\ an
%   expandable error will be thrown eventually. Internally \expkv's
%   |\ekvredirectunknown| will be used.
% \end{function}
%
% \begin{function}{unknown redirect-noval}
%   \begin{syntax}
%     unknown redirect-noval = \marg{set-list} \prefixes2333
%   \end{syntax}
%   This behaves just like |unknown redirect-code| but will set up means to
%   forward keys for |unknown noval|. Internally \expkv's
%   |\ekvredirectunknownNoVal| will be used.
% \end{function}
%
% \begin{function}{unknown redirect}
%   \begin{syntax}
%     unknown redirect = \marg{set-list} \prefixes2333
%   \end{syntax}
%   This is a short cut to apply both, |unknown redirect-code| and
%   |unknown redirect-noval|, as a result you might get doubled error messages,
%   one from each.
% \end{function}
%
% \subsection{Bugs}
% I don't think there are any (but every developer says that), if you find some
% please let me know, either via the email address on the first page or on
% GitHub:
% \url{https://github.com/Skillmon/tex_expkv-def}
%
% \subsection{Example}
% The following is an example code defining each base key-type once. Please
% admire the very creative key-name examples.
%
% \begin{lstlisting}
% \ekvdefinekeys{example}
%   {
%      long code keyA = #1
%     ,noval     keyA = NoVal given
%     ,bool      keyB = \keyB
%     ,boolTF    keyC = \keyC
%     ,store     keyD = \keyD
%     ,data      keyE = \keyE
%     ,dataT     keyF = \keyF
%     ,int       keyG = \keyG
%     ,dimen     keyH = \keyH
%     ,skip      keyI = \keyI
%     ,toks      keyJ = \keyJ
%     ,default   keyJ = \empty test
%     ,new box   keyK = \keyK
%     ,qdefault  keyK = K
%     ,choice    keyL =
%       {
%          protected 1 = \texttt{a}
%         ,2 = b
%         ,3 = c
%         ,4 = d
%         ,5 = e
%       }
%     ,edefault  keyL = 2
%     ,meta      keyM = {keyA={#1},keyB=false}
%     ,invbool   keyN = \keyN
%     ,boolpair  keyO = \keyOa\keyOb
%   }
% \end{lstlisting}
%
% \bigskip
% Since the |data| type might be a bit strange, here is another usage example
% for it.
% \begin{lstlisting}
% \ekvdefinekeys{ex}
%   {
%      data  name  = \Pname
%     ,data  age   = \Page
%     ,dataT hobby = \Phobby
%   }
% \newcommand\Person[1]
%   {%
%     \begingroup
%     \ekvset{ex}{#1}%
%     \begin{description}
%       \item[\Pname{}{\errmessage{A person requires a name}}]
%       \item[Age] \Page{\textit}{\errmessage{A person requires an age}}
%       \Phobby{\item[Hobbies]}
%     \end{description}
%     \endgroup
%   }
% \Person{name=Jonathan P. Spratte, age=young, hobby=\TeX\ coding}
% \Person{name=Some User, age=unknown, hobby=Reading Documentation}
% \Person{name=Anybody, age=any}
% \end{lstlisting}
% In this example a person should have a name and an age, but doesn't have to
% have hobbies. The name will be displayed as the description item and the age
% in Italics. If a person has no hobbies the description item will be silently
% left out.  The result of the above code looks like this:
%
% \begin{exresult}{\linewidth}
% \ekvdefinekeys{ex}
%   {
%      data  name  = \Pname
%     ,data  age   = \Page
%     ,dataT hobby = \Phobby
%   }
% \newcommand\Person[1]
%   {%
%     \begingroup
%     \ekvset{ex}{#1}%
%     \begin{description}
%       \item[\Pname{}{\errmessage{A person requires a name}}]
%       \item[Age] \Page{\textit}{\errmessage{A person requires an age}}
%       \Phobby{\item[Hobbies]}
%     \end{description}
%     \endgroup
%   }
% \Person{name=Jonathan P. Spratte, age=young, hobby=\TeX\ coding}
% \Person{name=Some User, age=unknown, hobby=Reading Documentation}
% \Person{name=Anybody, age=any}
% \end{exresult}
%
% \subsection{License}
%
% Copyright \textcopyright\ 2020\unless\ifnum\year=2020--\the\year\fi\
% Jonathan P. Spratte
%
% \medskip\noindent
% This work may be distributed and/or modified under the conditions of the
% \LaTeX\ Project Public License (LPPL), either version 1.3c of this license or
% (at your option) any later version. The latest version of this license is in
% the file:
%
% \url{http://www.latex-project.org/lppl.txt}
%
% \noindent
% This work is ``maintained'' (as per LPPL maintenance status) by
%
% \mbox{Jonathan P. Spratte}.
%
% \end{documentation}^^A=<<
%
% \begin{implementation}^^A>>=
%
% \clearpage
%
% \section{Implementation}^^A>>=
%^^A the LaTeX package >>=
% \subsection{The \LaTeX\ Package}
% Just like for \expkv\ we provide a small \LaTeX\ package that sets up things
% such that we behave nicely on \LaTeX\ packages and files system. It'll
% |\input| the generic code which implements the functionality.
% \gobbledocstriptag
%<*pkg>
%    \begin{macrocode}
\RequirePackage{expkv}
\def\ekvd@tmp
  {%
    \ProvidesFile{expkv-def.tex}%
      [\ekvdDate\space v\ekvdVersion\space a key-defining frontend for expkv]%
  }
\input{expkv-def.tex}
\ProvidesPackage{expkv-def}%
  [\ekvdDate\space v\ekvdVersion\space a key-defining frontend for expkv]
%    \end{macrocode}
% \gobbledocstriptag
%</pkg>
%^^A=<<
%^^A the ConTeXt module >>=
% \subsection{The \ConTeXt\ module}
% \gobbledocstriptag
%<*ctx>
%    \begin{macrocode}
\writestatus{loading}{ConTeXt User Module / expkv-def}
\usemodule[expkv]
\unprotect
\input expkv-def.tex
\writestatus{loading}
  {ConTeXt User Module / expkv-def / Version \ekvdVersion\space loaded}
\protect\endinput
%    \end{macrocode}
% \gobbledocstriptag
%</ctx>
%^^A=<<
%^^A main file >>=
% \subsection{The Generic Code}
% The rest of this implementation will be the generic code.
% \gobbledocstriptag
%<*tex>
%
% Load \expkv\ if the package didn't already do so -- since \expkv\ has
% safeguards against being loaded twice this does no harm and the overhead
% isn't that big. Also we reuse some of the internals of \expkv\ to save us from
% retyping them.
%    \begin{macrocode}
\input expkv
%    \end{macrocode}
%
% We make sure that \file{expkv-def.tex} is only input once:
%    \begin{macrocode}
\expandafter\ifx\csname ekvdVersion\endcsname\relax
\else
  \expandafter\endinput
\fi
%    \end{macrocode}
%
% \begin{macro}{\ekvdVersion,\ekvdDate}
% We're on our first input, so lets store the version and date in a macro.
%    \begin{macrocode}
\def\ekvdVersion{0.8c}
\def\ekvdDate{2021-09-20}
%    \end{macrocode}
% \end{macro}
%
% If the \LaTeX\ format is loaded we want to be a good file and report back who
% we are, for this the package will have defined |\ekvd@tmp| to use
% |\ProvidesFile|, else this will expand to a |\relax| and do no harm.
%    \begin{macrocode}
\csname ekvd@tmp\endcsname
%    \end{macrocode}
%
% Store the category code of |@| to later be able to reset it and change it to
% 11 for now.
%    \begin{macrocode}
\expandafter\chardef\csname ekvd@tmp\endcsname=\catcode`\@
\catcode`\@=11
%    \end{macrocode}
% |\ekvd@tmp| will be reused later to handle expansion during the key
% defining. But we don't need it to ever store information long-term after
% \expkvd\ was initialized.
%
% \begin{macro}[internal]{\ekvd@ifprimitive}
%    \begin{macrocode}
\protected\long\def\ekvd@ifprimitive#1%
  {%
    \begingroup
      \edef\ekvd@tmpa{\string #1}%
      \edef\ekvd@tmpb{\meaning#1}%
      \expandafter
    \endgroup
    \ifx\ekvd@tmpa\ekvd@tmpb
      \ekv@fi@firstoftwo
    \fi
    \@secondoftwo
  }
%    \end{macrocode}
% \end{macro}
%
% \begin{macro}[internal]
%   {\ekvd@long,\ekvd@prot,\ekvd@clear@prefixes,\ekvd@ifalso}
% \expkvd\ will use |\ekvd@long|, |\ekvd@prot|, and |\ekvd@ifalso| to store
% whether a key should be defined as |\long| or |\protected| or adds an action
% to an existing key, and we have to clear them for every new key. By default
% |long| and |protected| will just be empty, |ifalso| will be
% |\@secondoftwo|, and |ifnew| will just use its third argument.
%    \begin{macrocode}
\protected\def\ekvd@clear@prefixes
  {%
    \let\ekvd@long\ekv@empty
    \let\ekvd@prot\ekv@empty
    \let\ekvd@ifalso\@secondoftwo
    \long\def\ekvd@ifnew##1##2##3{##3}%
  }
\ekvd@clear@prefixes
%    \end{macrocode}
% \end{macro}
%
% \begin{macro}{\ekvdefinekeys}
% This is the one front-facing macro which provides the interface to define
% keys. It's using |\ekvparse| to handle the \kv\ list, the interpretation will
% be done by |\ekvd@noarg| and |\ekvd@|. The \set\ for which the keys should be
% defined is stored in |\ekvd@set|.
%    \begin{macrocode}
\protected\def\ekvdefinekeys#1%
  {%
    \def\ekvd@set{#1}%
    \ekvparse\ekvd@noarg\ekvd@arg
  }
%    \end{macrocode}
% \end{macro}
%
% \begin{macro}[internal]{\ekvd@noarg,\ekvd@arg,\ekvd@handle}
% |\ekvd@noarg| and |\ekvd@arg| store whether there was a value in the \kv\
% pair. |\ekvd@handle| has to test whether there is a space inside the key and
% if so calls the prefix grabbing routine, else we throw an error and ignore the
% key.
%    \begin{macrocode}
\protected\long\def\ekvd@noarg#1%
  {%
    \let\ekvd@ifnoarg\@firstoftwo
    \expandafter\ekvd@handle\detokenize{#1}\ekvd@stop{}%
  }
\protected\long\def\ekvd@arg#1%
  {%
    \let\ekvd@ifnoarg\@secondoftwo
    \expandafter\ekvd@handle\detokenize{#1}\ekvd@stop
  }
\protected\long\def\ekvd@handle#1\ekvd@stop#2%
  {%
    \ekvd@clear@prefixes
    \def\ekvd@cur{#1}%
    \ekvd@ifspace{#1}%
      {\ekvd@prefix\ekv@mark#1\ekv@stop{#2}}%
      \ekvd@err@missing@type
  }
%    \end{macrocode}
% \end{macro}
%
% \begin{macro}[internal]{\ekvd@prefix,\ekvd@prefix@}
% \expkvd\ separates prefixes into two groups, the first being prefixes in the
% \TeX\ sense (|long| and |protected|) which use |@p@| in their name, the other
% being key-types (|code|, |int|, \emph{etc.}) which use |@t@| instead.
% |\ekvd@prefix| splits at the first space and checks whether its a |@p@|
% or |@t@| type prefix. If it is neither throw an error and gobble the
% definition (the value).
%    \begin{macrocode}
\protected\def\ekvd@prefix#1 {\ekv@strip{#1}\ekvd@prefix@\ekv@mark}
\protected\def\ekvd@prefix@#1#2\ekv@stop
  {%
    \ekv@ifdefined{ekvd@t@#1}%
      {\ekv@strip{#2}{\csname ekvd@t@#1\endcsname}}%
      {%
        \ekv@ifdefined{ekvd@p@#1}%
          {\csname ekvd@p@#1\endcsname\ekvd@prefix@after@p{#2}}%
          {\ekvd@err@undefined@prefix{#1}\@gobble}%
      }%
  }
%    \end{macrocode}
% \end{macro}
%
% \begin{macro}[internal]{\ekvd@prefix@after@p}
% The |@p@| type prefixes are all just modifying a following |@t@| type, so they
% will need to search for another prefix. This is true for all of them, so we
% use a macro to handle this. It'll throw an error if there is no other prefix.
%    \begin{macrocode}
\protected\def\ekvd@prefix@after@p#1%
  {%
    \ekvd@ifspace{#1}%
      {\ekvd@prefix#1\ekv@stop}%
      {\ekvd@err@missing@type\@gobble}%
  }
%    \end{macrocode}
% \end{macro}
%
% \begin{macro}[internal]
%   {\ekvd@p@long,\ekvd@p@protected,\ekvd@p@protect,\ekvd@p@also,\ekvd@p@new}
% Define the |@p@| type prefixes, they all just store some information in a
% temporary macro.
%    \begin{macrocode}
\protected\def\ekvd@p@long{\let\ekvd@long\long}
\protected\def\ekvd@p@protected{\let\ekvd@prot\protected}
\let\ekvd@p@protect\ekvd@p@protected
\protected\def\ekvd@p@also{\let\ekvd@ifalso\@firstoftwo}
\protected\def\ekvd@p@new{\let\ekvd@ifnew\ekvd@assert@new}
%    \end{macrocode}
% \end{macro}
%
% \subsubsection{Key Types}
%
% \begin{macro}[internal]{\ekvd@type@set,\ekvd@t@set}
% The |set| type is quite straight forward, just define a |NoVal| key to call
% |\ekvchangeset|.
%    \begin{macrocode}
\protected\def\ekvd@type@set#1#2%
  {%
    \ekvd@assert@not@long
    \ekvd@assert@not@protected
    \ekvd@ifnew{NoVal}{#1}%
      {%
        \ekv@ifempty{#2}%
          {\ekvd@err@missing@definition}%
          {%
            \ekvd@ifalso
              {%
                \ekv@expargtwice{\ekvd@add@noval{#1}}%
                  {\ekvchangeset{#2}}%
                  \ekvd@assert@not@protected@also
              }%
              {\ekv@expargtwice{\ekvdefNoVal\ekvd@set{#1}}{\ekvchangeset{#2}}}%
          }%
      }%
  }
\protected\def\ekvd@t@set#1#2%
  {%
    \ekvd@ifnoarg
      {\ekvd@type@set{#1}{#1}}%
      {\ekvd@type@set{#1}{#2}}%
  }
%    \end{macrocode}
% \end{macro}
%
% \begin{macro}[internal]{\ekvd@type@noval,\ekvd@t@noval,\ekvd@t@enoval}
% Another pretty simple type, |noval| just needs to assert that there is a
% definition and that |long| wasn't specified.
% There are types where the difference in the variants is so small, that we
% define a common handler for them, those common handlers are named with
% |@type@|. |noval| and |enoval| are so similar that we can use such a |@type@|
% macro, even if we could've done |noval| in a slightly faster way without it.
%    \begin{macrocode}
\protected\long\def\ekvd@type@noval#1#2#3%
  {%
    \ekvd@ifnew{NoVal}{#2}%
      {%
        \ekvd@assert@arg
          {%
            \ekvd@assert@not@long
            \ekvd@prot#1\ekvd@tmp{#3}%
            \ekvd@ifalso
              {\ekv@exparg{\ekvd@add@noval{#2}}\ekvd@tmp{}}%
              {\ekvletNoVal\ekvd@set{#2}\ekvd@tmp}%
          }%
      }%
  }
\protected\def\ekvd@t@noval{\ekvd@type@noval\def}
\protected\def\ekvd@t@enoval{\ekvd@type@noval\edef}
%    \end{macrocode}
% \end{macro}
%
% \begin{macro}[internal]{\ekvd@type@code,\ekvd@t@code,\ekvd@t@ecode}
% |code| is simple as well, |ecode| has to use |\edef| on a temporary macro,
% since \expkv\ doesn't provide an |\ekvedef|.
%    \begin{macrocode}
\protected\long\def\ekvd@type@code#1#2#3%
  {%
    \ekvd@ifnew{}{#2}%
      {%
        \ekvd@assert@arg
          {%
            \ekvd@prot\ekvd@long#1\ekvd@tmp##1{#3}%
            \ekvd@ifalso
              {\ekv@exparg{\ekvd@add@val{#2}}{\ekvd@tmp{##1}}{}}%
              {\ekvlet\ekvd@set{#2}\ekvd@tmp}%
          }%
      }%
  }
\protected\def\ekvd@t@code{\ekvd@type@code\def}
\protected\def\ekvd@t@ecode{\ekvd@type@code\edef}
%    \end{macrocode}
% \end{macro}
%
% \begin{macro}[internal]
%   {
%     \ekvd@type@default,\ekvd@t@default,\ekvd@t@qdefault,\ekvd@t@odefault,
%     \ekvd@t@fdefault
%   }
% |\ekvd@type@default| asserts there was an argument, also the key for which one
% wants to set a default has to be already defined (this is not so important for
% |default|, but |qdefault| requires is). If everything is good, |\edef| a
% temporary macro that expands |\ekvd@set| and the |\csname| for the key, and in
% the case of |qdefault| does the first expansion step of the key-macro.
%    \begin{macrocode}
\protected\long\def\ekvd@type@default#1#2#3#4%
  {%
    \ekvd@assert@arg
      {%
        \ekvifdefined\ekvd@set{#3}%
          {%
            \ekvd@assert@not@new
            \ekvd@assert@not@long
            \ekvd@prot\edef\ekvd@tmp
              {%
                \ekv@unexpanded\expandafter#1%
                  {#2\csname\ekv@name\ekvd@set{#3}\endcsname{#4}}%
              }%
            \ekvd@ifalso
              {\ekv@exparg{\ekvd@add@noval{#3}}\ekvd@tmp{}}%
              {\ekvletNoVal\ekvd@set{#3}\ekvd@tmp}%
          }%
          {\ekvd@err@undefined@key{#3}}%
      }%
  }
\protected\def\ekvd@t@default{\ekvd@type@default{}{}}
\protected\def\ekvd@t@qdefault{\ekvd@type@default{\expandafter\expandafter}{}}
\let\ekvd@t@odefault\ekvd@t@qdefault
\protected\def\ekvd@t@fdefault{\ekvd@type@default{}{\romannumeral`\^^@}}
%    \end{macrocode}
% \end{macro}
%
% \begin{macro}[internal]{\ekvd@t@edefault}
% |edefault| is too different from |default| and |qdefault| to reuse the
% |@type@| macro, as it doesn't need |\unexpanded| inside of |\edef|.
%    \begin{macrocode}
\protected\long\def\ekvd@t@edefault#1#2%
  {%
    \ekvd@assert@arg
      {%
        \ekvifdefined\ekvd@set{#1}%
          {%
            \ekvd@assert@not@new
            \ekvd@assert@not@long
            \ekvd@prot\edef\ekvd@tmp
              {\csname\ekv@name\ekvd@set{#1}\endcsname{#2}}%
            \ekvd@ifalso
              {\ekv@exparg{\ekvd@add@noval{#1}}\ekvd@tmp{}}%
              {\ekvletNoVal\ekvd@set{#1}\ekvd@tmp}%
          }%
          {\ekvd@err@undefined@key{#1}}%
      }%
  }
%    \end{macrocode}
% \end{macro}
%
% \begin{macro}[internal]
%   {\ekvd@t@initial,\ekvd@t@oinitial,\ekvd@t@finitial,\ekvd@t@einitial}
%    \begin{macrocode}
\long\def\ekvd@type@initial#1#2#3#4%
  {%
    \ekvd@assert@not@new
    \ekvd@assert@not@also
    \ekvd@assert@not@long
    \ekvd@assert@not@protected
    \ekvd@ifnoarg
      {%
        \ekvifdefinedNoVal\ekvd@set{#3}%
          {\csname\ekv@name\ekvd@set{#3}N\endcsname}%
          {\ekvd@err@undefined@noval{#3}}%
      }%
      {%
        \ekvifdefined\ekvd@set{#3}%
          {%
            #1{#2#4}%
            \csname\ekv@name\ekvd@set{#3}\expandafter\endcsname\expandafter
              {\ekvd@tmp}%
          }%
          {\ekvd@err@undefined@key{#3}}%
      }%
  }
\def\ekvd@t@initial{\ekvd@type@initial{\def\ekvd@tmp}{}}
\def\ekvd@t@oinitial{\ekvd@type@initial{\ekv@exparg{\def\ekvd@tmp}}{}}
\def\ekvd@t@einitial{\ekvd@type@initial{\edef\ekvd@tmp}{}}
\def\ekvd@t@finitial
  {\ekvd@type@initial{\ekv@exparg{\def\ekvd@tmp}}{\romannumeral`\^^@}}
%    \end{macrocode}
% \end{macro}
%
% \begin{macro}[internal]
%   {
%     \ekvd@type@bool,\ekvd@t@bool,\ekvd@t@gbool,\ekvd@t@boolTF,\ekvd@t@gboolTF,
%     \ekvd@t@invbool,\ekvd@t@ginvbool,\ekvd@t@invboolTF,\ekvd@t@ginvboolTF,
%   }
% The boolean types are a quicker version of a |choice| that accept |true| and
% |false|, and set up the |NoVal| action to be identical to \texttt{\key=true}.
% The |true| and |false| actions are always just |\let|ting the macro in |#7| to
% some other macro (\emph{e.g.}, \cs[no-index]{iftrue}).
%    \begin{macrocode}
\protected\def\ekvd@type@bool#1#2#3#4#5%
  {%
    \ekvd@ifnew{}{#4}%
      {%
        \ekvd@ifnew{NoVal}{#4}%
          {%
            \ekvd@assert@filledarg{#5}%
              {%
                \ekvd@newlet#5#3%
                \ekvd@type@choice{#4}%
                \protected\ekvdefNoVal\ekvd@set{#4}{#1\let#5#2}%
                \protected\expandafter\def
                  \csname\ekvd@choice@name\ekvd@set{#4}{true}\endcsname
                  {#1\let#5#2}%
                \protected\expandafter\def
                  \csname\ekvd@choice@name\ekvd@set{#4}{false}\endcsname
                  {#1\let#5#3}%
              }%
          }%
      }%
  }
\protected\def\ekvd@t@bool{\ekvd@type@bool{}\iftrue\iffalse}
\protected\def\ekvd@t@gbool{\ekvd@type@bool\global\iftrue\iffalse}
\protected\def\ekvd@t@boolTF{\ekvd@type@bool{}\@firstoftwo\@secondoftwo}
\protected\def\ekvd@t@gboolTF{\ekvd@type@bool\global\@firstoftwo\@secondoftwo}
\protected\def\ekvd@t@invbool{\ekvd@type@bool{}\iffalse\iftrue}
\protected\def\ekvd@t@ginvbool{\ekvd@type@bool\global\iffalse\iftrue}
\protected\def\ekvd@t@invboolTF{\ekvd@type@bool{}\@secondoftwo\@firstoftwo}
\protected\def\ekvd@t@ginvboolTF
  {\ekvd@type@bool\global\@secondoftwo\@firstoftwo}
%    \end{macrocode}
% \end{macro}
%
% \begin{macro}[internal]
%   {
%     \ekvd@type@boolpair,
%     \ekvd@t@boolpair,\ekvd@t@gboolpair,\ekvd@t@boolpairTF,\ekvd@t@gboolpairTF
%   }
% The boolean pair types are essentially the same as the boolean types, but set
% two macros instead of one.
%    \begin{macrocode}
\protected\def\ekvd@type@boolpair#1#2#3#4#5#6%
  {%
    \ekvd@ifnew{}{#4}%
      {%
        \ekvd@ifnew{NoVal}{#4}%
          {%
            \ekvd@newlet#5#3%
            \ekvd@newlet#6#2%
            \ekvd@type@choice{#4}%
            \protected\ekvdefNoVal\ekvd@set{#4}{#1\let#5#2#1\let#6#3}%
            \protected\expandafter\def
              \csname\ekvd@choice@name\ekvd@set{#4}{true}\endcsname
              {#1\let#5#2#1\let#6#3}%
            \protected\expandafter\def
              \csname\ekvd@choice@name\ekvd@set{#4}{false}\endcsname
              {#1\let#5#3#1\let#6#2}%
          }%
      }%
  }
\protected\def\ekvd@t@boolpair#1#2%
  {\ekvd@assert@twoargs{#2}{\ekvd@type@boolpair{}\iftrue\iffalse{#1}#2}}
\protected\def\ekvd@t@gboolpair#1#2%
  {\ekvd@assert@twoargs{#2}{\ekvd@type@boolpair\global\iftrue\iffalse{#1}#2}}
\protected\def\ekvd@t@boolpairTF#1#2%
  {%
    \ekvd@assert@twoargs{#2}%
      {\ekvd@type@boolpair{}\@firstoftwo\@secondoftwo{#1}#2}%
  }
\protected\def\ekvd@t@gboolpairTF#1#2%
  {%
    \ekvd@assert@twoargs{#2}%
      {\ekvd@type@boolpair\global\@firstoftwo\@secondoftwo{#1}#2}%
  }
%    \end{macrocode}
% \end{macro}
%
% \begin{macro}[internal]
%   {
%     \ekvd@type@data,\ekvd@t@data,\ekvd@t@gdata,\ekvd@t@dataT,\ekvd@t@gdataT
%   }
%    \begin{macrocode}
\protected\def\ekvd@type@data#1#2#3#4#5#6%
  {%
    \ekvd@ifnew{}{#5}%
      {%
        \ekvd@assert@filledarg{#6}%
          {%
            \ekvd@newlet#6#1%
            \ekvd@ifalso
              {%
                \let\ekvd@prot\protected
                \ekvd@add@val{#5}{\long#2#6####1#3{####1{#4}}}{}%
              }%
              {%
                \protected\ekvd@long\ekvdef\ekvd@set{#5}%
                  {\long#2#6####1#3{####1{#4}}}%
              }%
          }%
      }%
  }
\protected\def\ekvd@t@data
  {\ekvd@type@data\@secondoftwo\edef{####2}{\ekv@unexpanded{##1}}}
\protected\def\ekvd@t@edata{\ekvd@type@data\@secondoftwo\edef{####2}{##1}}
\protected\def\ekvd@t@gdata
  {\ekvd@type@data\@secondoftwo\xdef{####2}{\ekv@unexpanded{##1}}}
\protected\def\ekvd@t@xdata{\ekvd@type@data\@secondoftwo\xdef{####2}{##1}}
\protected\def\ekvd@t@dataT
  {\ekvd@type@data\@gobble\edef{}{\ekv@unexpanded{##1}}}
\protected\def\ekvd@t@edataT{\ekvd@type@data\@gobble\edef{}{##1}}
\protected\def\ekvd@t@gdataT
  {\ekvd@type@data\@gobble\xdef{}{\ekv@unexpanded{##1}}}
\protected\def\ekvd@t@xdataT{\ekvd@type@data\@gobble\xdef{}{##1}}
%    \end{macrocode}
% \end{macro}
%
% \begin{macro}[internal]{\ekvd@type@box,\ekvd@t@box,\ekvd@t@gbox}
% Set up our boxes. Though we're a generic package we want to be colour safe, so
% we put an additional grouping level inside the box contents, for the case that
% someone uses \pkg{color}. |\ekvd@newreg| is a small wrapper which tests
% whether the first argument is defined and if not does
% |\csname new#2\endcsname#1|.
%    \begin{macrocode}
\protected\def\ekvd@type@box#1#2#3%
  {%
    \ekvd@ifnew{}{#2}%
      {%
        \ekvd@assert@filledarg{#3}%
          {%
            \ekvd@newreg#3{box}%
            \ekvd@ifalso
              {%
                \let\ekvd@prot\protected
                \ekvd@add@val{#2}{#1\setbox#3=\hbox{\begingroup##1\endgroup}}{}%
              }%
              {%
                \protected\ekvd@long\ekvdef\ekvd@set{#2}%
                  {#1\setbox#3=\hbox{\begingroup##1\endgroup}}%
              }%
          }%
      }%
  }
\protected\def\ekvd@t@box{\ekvd@type@box{}}
\protected\def\ekvd@t@gbox{\ekvd@type@box\global}
%    \end{macrocode}
% \end{macro}
%
% \begin{macro}[internal]{\ekvd@type@toks,\ekvd@t@toks,\ekvd@t@gtoks}
% Similar to |box|, but set the |toks|.
%    \begin{macrocode}
\protected\def\ekvd@type@toks#1#2#3%
  {%
    \ekvd@ifnew{}{#2}%
      {%
        \ekvd@assert@filledarg{#3}%
          {%
            \ekvd@newreg#3{toks}%
            \ekvd@ifalso
              {%
                \let\ekvd@prot\protected
                \ekvd@add@val{#2}{#1#3={##1}}{}%
              }%
              {\protected\ekvd@long\ekvdef\ekvd@set{#2}{#1#3={##1}}}%
          }%
      }%
  }
\protected\def\ekvd@t@toks{\ekvd@type@toks{}}
\protected\def\ekvd@t@gtoks{\ekvd@type@toks\global}
%    \end{macrocode}
% \end{macro}
%
% \begin{macro}[internal]{\ekvd@type@preapptoks,\ekvd@t@apptoks,\ekvd@t@gapptoks}
% Just like |toks|, but expand the current contents of the |toks| register to
% append the new contents.
%    \begin{macrocode}
\ekvd@ifprimitive\toksapp
  {%
    \protected\def\ekvd@type@preapptoks#1#2#3%
      {%
        \ekvd@ifnew{}{#2}%
          {%
            \ekvd@assert@filledarg{#3}%
              {%
                \ekvd@newreg#3{toks}%
                \ekvd@ifalso
                  {%
                    \let\ekvd@prot\protected
                    \ekvd@add@val{#2}{#1#3{##1}}{}%
                  }%
                  {\protected\ekvd@long\ekvdef\ekvd@set{#2}{#1#3{##1}}}%
              }%
          }%
      }
    \protected\def\ekvd@t@apptoks{\ekvd@type@preapptoks\toksapp}
    \protected\def\ekvd@t@gapptoks{\ekvd@type@preapptoks\gtoksapp}
    \protected\def\ekvd@t@pretoks{\ekvd@type@preapptoks\tokspre}
    \protected\def\ekvd@t@gpretoks{\ekvd@type@preapptoks\gtokspre}
  }
  {%
    \protected\def\ekvd@type@apptoks#1#2#3%
      {%
        \ekvd@ifnew{}{#2}%
          {%
            \ekvd@assert@filledarg{#3}%
              {%
                \ekvd@newreg#3{toks}%
                \ekvd@ifalso
                  {%
                    \let\ekvd@prot\protected
                    \ekvd@add@val{#2}{#1#3=\expandafter{\the#3##1}}{}%
                  }%
                  {%
                    \protected\ekvd@long\ekvdef\ekvd@set{#2}%
                      {#1#3=\expandafter{\the#3##1}}%
                  }%
              }%
          }%
      }
    \protected\def\ekvd@t@apptoks{\ekvd@type@apptoks{}}
    \protected\def\ekvd@t@gapptoks{\ekvd@type@apptoks\global}
    \newtoks\ekvd@toks
    \protected\def\ekvd@type@pretoks#1#2#3%
      {%
        \ekvd@ifnew{}{#2}%
          {%
            \ekvd@assert@filledarg{#3}%
              {%
                \ekvd@newreg#3{toks}%
                \ekvd@ifalso
                  {%
                    \let\ekvd@prot\protected
                    \ekvd@add@val{#2}%
                      {%
                        \ekvd@toks={##1}%
                        #1#3=\expandafter{\the\expandafter\ekvd@toks\the#3}%
                      }%
                      {}%
                  }%
                  {%
                    \protected\ekvd@long\ekvdef\ekvd@set{#2}%
                      {%
                        \ekvd@toks={##1}%
                        #1#3=\expandafter{\the\expandafter\ekvd@toks\the#3}%
                      }%
                  }%
              }%
          }%
      }
    \protected\def\ekvd@t@pretoks{\ekvd@type@pretoks{}}
    \protected\def\ekvd@t@gpretoks{\ekvd@type@pretoks\global}
  }
%    \end{macrocode}
% \end{macro}
%
% \begin{macro}[internal]
%   {
%     \ekvd@type@reg,\ekvd@t@int,\ekvd@t@eint,\ekvd@t@gint,\ekvd@t@xint,
%     \ekvd@t@dimen,\ekvd@t@edimen,\ekvd@t@gdimen,\ekvd@t@xdimen,
%     \ekvd@t@skip,\ekvd@t@eskip,\ekvd@t@gskip,\ekvd@t@xskip
%   }
% The |\ekvd@type@reg| can handle all the types for which the assignment will
% just be \texttt{\meta{register}=\meta{value}}.
%    \begin{macrocode}
\protected\def\ekvd@type@reg#1#2#3#4#5#6%
  {%
    \ekvd@ifnew{}{#5}%
      {%
        \ekvd@assert@filledarg{#6}%
          {%
            \ekvd@newreg#6{#1}%
            \ekvd@ifalso
              {%
                \let\evkd@prot\protected
                \ekvd@add@val{#5}{#2#6=#3##1#4\relax}{}%
              }%
              {\protected\ekvd@long\ekvdef\ekvd@set{#5}{#2#6=#3##1#4\relax}}%
          }%
      }%
  }
\protected\def\ekvd@t@int{\ekvd@type@reg{count}{}{}{}}
\protected\def\ekvd@t@eint{\ekvd@type@reg{count}{}\numexpr\relax}
\protected\def\ekvd@t@gint{\ekvd@type@reg{count}\global{}{}}
\protected\def\ekvd@t@xint{\ekvd@type@reg{count}\global\numexpr\relax}
\protected\def\ekvd@t@dimen{\ekvd@type@reg{dimen}{}{}{}}
\protected\def\ekvd@t@edimen{\ekvd@type@reg{dimen}{}\dimexpr\relax}
\protected\def\ekvd@t@gdimen{\ekvd@type@reg{dimen}\global{}{}}
\protected\def\ekvd@t@xdimen{\ekvd@type@reg{dimen}\global\dimexpr\relax}
\protected\def\ekvd@t@skip{\ekvd@type@reg{skip}{}{}{}}
\protected\def\ekvd@t@eskip{\ekvd@type@reg{skip}{}\glueexpr\relax}
\protected\def\ekvd@t@gskip{\ekvd@type@reg{skip}\global{}{}}
\protected\def\ekvd@t@xskip{\ekvd@type@reg{skip}\global\glueexpr\relax}
%    \end{macrocode}
% \end{macro}
%
% \begin{macro}[internal]{\ekvd@type@store,\ekvd@t@store,\ekvd@t@gstore}
% The none-expanding |store| types use an |\edef| or |\xdef| and |\unexpanded|
% to be able to also store |#| easily.
%    \begin{macrocode}
\protected\def\ekvd@type@store#1#2#3#4%
  {%
    \ekvd@ifnew{}{#3}%
      {%
        \ekvd@assert@filledarg{#4}%
          {%
            \ekvd@newlet#4\ekv@empty
            \ekvd@ifalso
              {%
                \let\ekvd@prot\protected
                \ekvd@add@val{#3}{#1#4{#2}}{}%
              }%
              {\protected\ekvd@long\ekvdef\ekvd@set{#3}{#1#4{#2}}}%
          }%
      }%
  }
\protected\def\ekvd@t@store{\ekvd@type@store\edef{\ekv@unexpanded{##1}}}
\protected\def\ekvd@t@gstore{\ekvd@type@store\xdef{\ekv@unexpanded{##1}}}
\protected\def\ekvd@t@estore{\ekvd@type@store\edef{##1}}
\protected\def\ekvd@t@xstore{\ekvd@type@store\xdef{##1}}
%    \end{macrocode}
% \end{macro}
%
% \begin{macro}[internal]
%   {
%     \ekvd@type@meta,\ekvd@type@meta@a,\ekvd@type@meta@b,\ekvd@type@meta@c,
%     \ekvd@t@meta,\ekvd@t@nmeta
%   }
% |meta| sets up things such that another instance of |\ekvset| will be run on
% the argument, with the same \set.
%    \begin{macrocode}
\protected\long\def\ekvd@type@meta#1#2#3#4#5#6#7%
  {%
    \ekvd@ifnew{#1}{#6}%
      {%
        \ekvd@assert@filledarg{#7}%
          {%
            \edef\ekvd@tmp{\ekvd@set}%
            \expandafter\ekvd@type@meta@a\expandafter{\ekvd@tmp}{#7}{#2}%
            \ekvd@ifalso
              {\ekv@exparg{#3{#6}}{\ekvd@tmp#4}{#5}}%
              {\csname ekvlet#1\endcsname\ekvd@set{#6}\ekvd@tmp}%
          }%
      }%
  }
\protected\long\def\ekvd@type@meta@a#1#2%
  {%
    \expandafter\ekvd@type@meta@b\expandafter{\ekvset{#1}{#2}}%
  }
\protected\def\ekvd@type@meta@b
  {%
    \expandafter\ekvd@type@meta@c\expandafter
  }
\protected\long\def\ekvd@type@meta@c#1#2%
  {%
    \ekvd@prot\ekvd@long\def\ekvd@tmp#2{#1}%
  }
\protected\def\ekvd@t@meta{\ekvd@type@meta{}{##1}\ekvd@add@val{{##1}}{}}
\protected\def\ekvd@t@nmeta
  {%
    \ekvd@assert@not@long
    \ekvd@type@meta{NoVal}{}\ekvd@add@noval{}\ekvd@assert@not@long@also
  }
%    \end{macrocode}
% \end{macro}
%
% \begin{macro}[internal]
%   {\ekvd@type@smeta,\ekvd@type@smeta@,\ekvd@t@smeta,\ekvd@t@snmeta}
% |smeta| is pretty similar to |meta|, but needs two arguments inside of \val,
% such that the first is the \set\ for which the sub-|\ekvset| and the second is
% the \kv\ list.
%    \begin{macrocode}
\protected\long\def\ekvd@type@smeta#1#2#3#4#5#6#7%
  {%
    \ekvd@ifnew{#1}{#6}%
      {%
        \ekvd@assert@twoargs{#7}%
          {%
            \ekvd@type@meta@a#7{#2}%
            \ekvd@ifalso
              {\ekv@exparg{#3{#6}}{\ekvd@tmp#4}{#5}}%
              {\csname ekvlet#1\endcsname\ekvd@set{#6}\ekvd@tmp}%
          }%
      }%
  }
\protected\def\ekvd@t@smeta{\ekvd@type@smeta{}{##1}\ekvd@add@val{{##1}}{}}
\protected\def\ekvd@t@snmeta
  {%
    \ekvd@assert@not@long
    \ekvd@type@smeta{NoVal}{}\ekvd@add@noval{}\ekvd@assert@not@long@also
  }
%    \end{macrocode}
% \end{macro}
%
% \begin{macro}[internal]
%   {
%     \ekvd@type@choice,
%     \ekvd@populate@choice,\ekvd@populate@choice@,\ekvd@populate@choice@noarg,
%     \ekvd@choice@prefix,\ekvd@choice@prefix@,\ekvd@choice@prefix@done,
%     \ekvd@choice@p@protected,\ekvd@choice@p@protect,
%     \ekvd@choice@p@long,\ekvd@choice@p@long@,
%     \ekvd@t@choice
%   }
% The |choice| type is by far the most complex type, as we have to run a
% sub-parser on the choice-definition list, which should support the |@p@| type
% prefixes as well (but |long| will always throw an error, as they are not
% allowed to be long). |\ekvd@type@choice| will just define the choice-key, the
% handling of the choices definition will be done by |\ekvd@populate@choice|.
%    \begin{macrocode}
\protected\def\ekvd@type@choice#1%
  {%
    \ekvd@assert@not@long
    \ekv@exparg{\ekvd@prot\def\ekvd@tmp##1}%
      {%
        \expandafter\ekvd@h@choice\expandafter
          {\ekvd@choice@name\ekvd@set{#1}{##1}}%
      }%
    \ekvd@ifalso
      {%
        \ekvd@assert@val{#1}%
          {%
            \ekvd@if@not@already@choice{#1}%
              {%
                \ekv@exparg
                  {%
                    \expandafter\ekvd@add@aux
                      \csname\ekv@name\ekvd@set{#1}\endcsname{{##1}}{#1}%
                  }%
                  {\ekvd@tmp{##1}}%
                  {\ekvd@long\ekvdef}\ekvd@assert@not@long@also
              }%
          }%
      }%
      {\ekvlet\ekvd@set{#1}\ekvd@tmp}%
  }
%    \end{macrocode}
% |\ekvd@populate@choice| just uses |\ekvparse| and then gives control to
% |\ekvd@populate@choice@noarg|, which throws an error, and
% |\ekvd@populate@choice@|.
%    \begin{macrocode}
\protected\def\ekvd@populate@choice
  {%
    \ekvparse\ekvd@populate@choice@noarg\ekvd@populate@choice@
  }
\protected\long\def\ekvd@populate@choice@noarg#1%
  {%
    \expandafter\ekvd@err@missing@definition@msg\expandafter{\ekvd@cur : #1}%
  }
%    \end{macrocode}
% |\ekvd@populate@choice@| runs the prefix-test, if there is none we can
% directly define the choice, for that |\ekvd@set@choice| will expand to the
% current choice-key's name, which will have been defined by |\ekvd@t@choice|.
% If there is a prefix run the prefix grabbing routine, which was altered for
% |@type@choice|.
%    \begin{macrocode}
\protected\long\def\ekvd@populate@choice@#1#2%
  {%
    \ekvd@clear@prefixes
    \ekvd@ifspace{#1}%
      {\ekvd@choice@prefix{\ekv@mark#1}\ekv@mark#1\ekv@stop}%
      {%
        \expandafter\edef
          \csname\ekvd@choice@name\ekvd@set\ekvd@set@choice{#1}\endcsname
      }%
      {\unexpanded{#2}}%
  }
\protected\def\ekvd@choice@prefix#1#2
  {%
    \ekv@strip{#2}{\ekvd@choice@prefix@{#1}}\ekv@mark
  }
\protected\def\ekvd@choice@prefix@#1#2#3\ekv@stop
  {%
    \ekv@ifdefined{ekvd@choice@p@#2}%
      {%
        \csname ekvd@choice@p@#2\endcsname
        \ekvd@ifspace{#3}%
          {\ekvd@choice@prefix{#3}#3\ekv@stop}%
          {\ekvd@choice@prefix@done{#3}}%
      }%
      {\ekvd@choice@prefix@done{#1}}%
  }
\protected\def\ekvd@choice@prefix@done#1%
  {%
    \ekvd@prot\expandafter\edef
      \csname
        \ekv@strip{#1}{\ekvd@choice@name\ekvd@set\ekvd@set@choice}%
      \endcsname
  }
\protected\def\ekvd@choice@p@protected{\let\ekvd@prot\protected}
\let\ekvd@choice@p@protect\ekvd@choice@p@protected
\protected\def\ekvd@choice@invalid@p#1\ekvd@ifspace#2%
  {%
    \expandafter\ekvd@choice@invalid@p@\expandafter{\ekv@gobble@mark#2}{#1}%
    \ekvd@ifspace{#2}%
  }
\protected\def\ekvd@choice@invalid@p@#1#2%
  {%
    \expandafter\ekvd@err@no@prefix@msg\expandafter{\ekvd@cur : #2 #1}{#2}%
  }
\protected\def\ekvd@choice@p@long{\ekvd@choice@invalid@p{long}}%
\protected\def\ekvd@choice@p@also{\ekvd@choice@invalid@p{also}}%
\protected\def\ekvd@choice@p@new{\ekvd@choice@invalid@p{new}}%
%    \end{macrocode}
% Finally we're able to set up the |@t@choice| macro, which has to store the
% current choice-key's name, define the key, and parse the available choices.
%    \begin{macrocode}
\protected\long\def\ekvd@t@choice#1#2%
  {%
    \ekvd@ifnew{}{#1}%
      {%
        \ekvd@assert@arg
          {%
            \ekvd@type@choice{#1}%
            \def\ekvd@set@choice{#1}%
            \ekvd@populate@choice{#2}%
          }%
      }%
  }
%    \end{macrocode}
% \end{macro}
%
% \begin{macro}[internal]{\ekvd@t@choice-store,\ekvd@t@choice-enum}
%   These two types define a special kind of |choice| key and are quite similar,
%   the only difference is what the different choices do (hence they use a
%   shared initialisation which differs in the chosen |populate| step).
%    \begin{macrocode}
\protected\long\expandafter\def\csname ekvd@t@choice-store\endcsname
  {\ekvd@type@choicespecial\ekvd@populate@choicestore}
\protected\long\expandafter\def\csname ekvd@t@choice-enum\endcsname
  {\ekvd@type@choicespecial\ekvd@populate@choiceenum}
%    \end{macrocode}
% \begin{macro}[internal]{\ekvd@type@choicespecial}
%   Initialise similar to a |choice| key. The difference is that we require two
%   arguments (which we assert), a macro to store things in, and a |csv|-list
%   containing the allowed values. |#1| is the |populate| macro according to the
%   type used.
%    \begin{macrocode}
\protected\long\def\ekvd@type@choicespecial#1#2#3%
  {%
    \ekvd@ifnew{}{#2}%
      {%
        \ekvd@assert@twoargs{#3}%
          {%
            \ekvd@type@choice{#2}%
            \def\ekvd@set@choice{#2}%
            #1#3%
          }%
      }%
  }
%    \end{macrocode}
% \end{macro}
% \begin{macro}[internal]
%   {\ekvd@populate@choicestore,\ekvd@populate@choicestore@}
%   We initialise the storing macro if it doesn't yet exist, and then we loop
%   over the value list. The |\edef|s with |\unexpanded| are both necessary to
%   be able to store macro parameter tokens (the outer protects at define time,
%   the inner at use time).
%    \begin{macrocode}
\protected\long\def\ekvd@populate@choicestore#1%
  {%
    \ekvd@newlet#1\ekv@empty
    \ekvcsvloop{\ekvd@populate@choicestore@#1}%
  }
\protected\long\def\ekvd@populate@choicestore@#1#2%
  {%
    \protected\expandafter\edef
      \csname\ekvd@choice@name\ekvd@set\ekvd@set@choice{#2}\endcsname
      {\unexpanded{\edef#1{\unexpanded{#2}}}}%
  }
%    \end{macrocode}
% \end{macro}
% \begin{macro}[internal]
%   {\ekvd@populate@choiceenum,\ekvd@populate@choiceenum@}
%   This is similar to the population of a |choice-store| type, but instead of
%   storing the values in a macro this initialises a count and stores the
%   position of the value in the list inside that count (zero-indexed). The
%   space is necessary to terminate the number scanning, which is the reason we
%   use |\@firstofone| (so that the space after the macro name isn't gobbled by
%   \TeX).
%    \begin{macrocode}
\protected\long\def\ekvd@populate@choiceenum#1%
  {%
    \ekvd@newreg#1{count}%
    \def\ekvd@tmp{0}%
    \ekvcsvloop{\ekvd@populate@choiceenum@#1}%
  }
\protected\long\def\ekvd@populate@choiceenum@#1#2%
  {%
    \protected\expandafter\edef
      \csname\ekvd@choice@name\ekvd@set\ekvd@set@choice{#2}\endcsname
      {#1=\@firstofone{\ekvd@tmp} }%
    \edef\ekvd@tmp{\the\numexpr\ekvd@tmp+1\relax}%
  }
%    \end{macrocode}
% \end{macro}
% \end{macro}
%
% \begin{macro}[internal]{\ekvd@t@unknown-choice}
%    \begin{macrocode}
\protected\long\expandafter\def\csname ekvd@t@unknown-choice\endcsname#1#2%
  {%
    \ekvd@assert@new@for@name{\ekvd@unknown@choice@name\ekvd@set{#1}}%
      {%
        \ekvd@assert@arg
          {%
            \ekvd@assert@not@long
            \ekvd@assert@not@also
            \ekvd@prot\expandafter
            \def\csname\ekvd@unknown@choice@name\ekvd@set{#1}\endcsname##1{#2}%
          }%
      }%
  }
%    \end{macrocode}
% \end{macro}
%
% \begin{macro}[internal]
%   {\ekvd@t@unknown,\ekvd@type@unknown@code,\ekvd@type@unknown@noval}
%   The |unknown| type has different subtypes which would be the key names for
%   other types. It is first checked whether that subtype is defined, if it
%   isn't throw an error, else use that subtype.
%    \begin{macrocode}
\protected\long\def\ekvd@t@unknown#1#2%
  {%
    \ekv@ifdefined{ekvd@type@unknown@\detokenize{#1}}%
      {\csname ekvd@type@unknown@\detokenize{#1}\endcsname{#2}}%
      \ekvd@err@misused@unknown
  }
%    \end{macrocode}
%   The |unknown noval| type can use |\ekvdefunknownNoVal| directly (after
%   asserting some prefixes).
%    \begin{macrocode}
\protected\long\def\ekvd@type@unknown@noval#1%
  {%
    \ekvd@assert@new@for@name{\ekv@name\ekvd@set{}uN}%
      {%
        \ekvd@assert@arg
          {%
            \ekvd@assert@not@also
            \ekvd@assert@not@long
            \ekvd@prot\ekvdefunknownNoVal\ekvd@set{#1}%
          }%
      }%
  }
%    \end{macrocode}
%   The |unknown code| type uses some trickery during the definition in order to
%   swap out |#1| and |#2| in the user supplied definition. This is done via a
%   temporary macro that stores the definition but gets the parameter numbers
%   reversed while the real definition is done.
%    \begin{macrocode}
\protected\long\def\ekvd@type@unknown@code#1%
  {%
    \ekvd@assert@new@for@name{\ekv@name\ekvd@set{}u}%
      {%
        \ekvd@assert@arg
          {%
            \ekvd@assert@not@also
            \begingroup
              \def\ekvd@tmp##1##2{#1}%
              \ekv@exparg
              {%
                \endgroup
                \ekvd@prot\ekvd@long\ekvdefunknown\ekvd@set
              }%
              {\ekvd@tmp{##2}{##1}}%
          }%
      }%
  }
%    \end{macrocode}
% \end{macro}
%
% \begin{macro}[internal]
%   {
%     \ekvd@type@unknown@redirect,
%     \ekvd@type@unknown@redirect-code,
%     \ekvd@type@unknown@redirect-noval
%   }
%   The |unknown redirect| types also just forward to |\ekvredirectunknown|
%   after asserting some prefixes.
%    \begin{macrocode}
\protected\edef\ekvd@type@unknown@redirect#1%
  {%
    \expandafter\noexpand\csname ekvd@type@unknown@redirect-code\endcsname{#1}%
    \expandafter\noexpand\csname ekvd@type@unknown@redirect-noval\endcsname{#1}%
  }
\protected\expandafter\def\csname ekvd@type@unknown@redirect-code\endcsname#1%
  {%
    \ekvd@assert@new@for@name{\ekv@name\ekvd@set{}u}%
      {%
        \ekvd@assert@arg
          {%
            \ekvd@assert@not@also
            \ekvd@assert@not@protected
            \expandafter\ekvredirectunknown\expandafter{\ekvd@set}{#1}%
          }%
      }%
  }
\protected\expandafter\def\csname ekvd@type@unknown@redirect-noval\endcsname#1%
  {%
    \ekvd@assert@new@for@name{\ekv@name\ekvd@set{}uN}%
      {%
        \ekvd@assert@arg
          {%
            \ekvd@assert@not@also
            \ekvd@assert@not@protected
            \ekvd@assert@not@long
            \expandafter\ekvredirectunknownNoVal\expandafter{\ekvd@set}{#1}%
          }%
      }%
  }
%    \end{macrocode}
% \end{macro}
%
% \subsubsection{Key Type Helpers}
% There are some keys that might need helpers during their execution (not during
% their definition, which are gathered as |@type@| macros). These helpers are
% named |@h@|.
%
% \begin{macro}[internal]{\ekvd@h@choice,\ekvd@h@choice@}
% The |choice| helper will just test whether the given choice was defined, if
% not throw an error expandably, else call the macro which stores the code for
% this choice.
%    \begin{macrocode}
\def\ekvd@h@choice#1%
  {%
    \expandafter\ekvd@h@choice@
      \csname\ifcsname#1\endcsname#1\else relax\fi\endcsname
      {#1}%
  }
\def\ekvd@h@choice@#1#2%
  {%
    \ifx#1\relax
      \ekvd@err@choice@invalid{#2}%
      \expandafter\@gobble
    \fi
    #1%
  }
%    \end{macrocode}
% \end{macro}
%
%
% \subsubsection{Handling \texttt{also}}
%
% \begin{macro}[internal]
%   {\ekvd@add@val,\ekvd@add@noval,\ekvd@add@aux,\ekvd@add@aux@}
%    \begin{macrocode}
\protected\long\def\ekvd@add@val#1#2#3%
  {%
    \ekvd@assert@val{#1}%
      {%
        \expandafter\ekvd@add@aux\csname\ekv@name\ekvd@set{#1}\endcsname{{##1}}%
          {#1}{#2}{\ekvd@long\ekvdef}{#3}%
      }%
  }
\protected\long\def\ekvd@add@noval#1#2#3%
  {%
    \ekvd@assert@noval{#1}%
      {%
        \expandafter\ekvd@add@aux\csname\ekv@name\ekvd@set{#1}N\endcsname{}%
          {#1}{#2}\ekvdefNoVal{#3}%
      }%
  }
\protected\long\def\ekvd@add@aux#1#2%
  {%
    \ekvd@extract@prefixes#1%
    \expandafter\ekvd@add@aux@\expandafter{#1#2}%
  }
\protected\long\def\ekvd@add@aux@#1#2#3#4#5%
  {%
    #5%
    \ekvd@prot#4\ekvd@set{#2}{#1#3}%
  }
%    \end{macrocode}
% \end{macro}
%
% \begin{macro}[internal]
%   {
%     \ekvd@extract@prefixes,\ekvd@extract@prefixes@,
%     \ekvd@extract@prefixes@long,\ekvd@extract@prefixes@prot
%   }
% This macro checks which prefixes were used for the definition of a macro and
% sets |\ekvd@long| and |\ekvd@prot| accordingly.
%    \begin{macrocode}
\protected\def\ekvd@extract@prefixes#1%
  {%
    \expandafter\ekvd@extract@prefixes@\meaning#1\ekvd@stop
  }
%    \end{macrocode}
% In the following definition |#1| will get replaced by |macro:|, |#2| by
% |\long| and |#3| by |\protected| (in each, all tokens will have category
% other). This allows us to parse the |\meaning| of a macro for those strings.
%    \begin{macrocode}
\protected\def\ekvd@extract@prefixes@#1#2#3%
  {%
    \protected\def\ekvd@extract@prefixes@##1#1##2\ekvd@stop
      {%
        \ekvd@extract@prefixes@long
          ##1\ekvd@mark\@firstofone#2\ekvd@mark\@gobble\ekvd@stop
          {\let\ekvd@long\long}%
        \ekvd@extract@prefixes@prot
          ##1\ekvd@mark\@firstofone#3\ekvd@mark\@gobble\ekvd@stop
          {\let\ekvd@prot\protected}%
      }%
    \protected\def\ekvd@extract@prefixes@long##1#2##2\ekvd@mark##3##4\ekvd@stop
      {##3}%
    \protected\def\ekvd@extract@prefixes@prot##1#3##2\ekvd@mark##3##4\ekvd@stop
      {##3}%
  }
%    \end{macrocode}
% We use a temporary macro to expand the three arguments of
% |\ekvd@extract@prefixes@|, which will set up the real meaning of itself and
% the parsing for |\long| and |\protected|.
%    \begin{macrocode}
\begingroup
\edef\ekvd@tmp
  {%
    \endgroup
    \ekvd@extract@prefixes@
      {\detokenize{macro:}}%
      {\string\long}%
      {\string\protected}%
  }
\ekvd@tmp
%    \end{macrocode}
% \end{macro}
%
%
% \subsubsection{Tests}
%
% \begin{macro}[internal]{\ekvd@newlet,\ekvd@newreg}
% These macros test whether a control sequence is defined, if it isn't they
% define it, either via |\let| or via the correct \cs[no-index]{new\meta{reg}}.
%    \begin{macrocode}
\protected\def\ekvd@newlet#1#2%
  {%
    \ifdefined#1\ekv@fi@gobble\fi\@firstofone{\let#1#2}%
  }
\protected\def\ekvd@newreg#1#2%
  {%
    \ifdefined#1\ekv@fi@gobble\fi\@firstofone{\csname new#2\endcsname#1}%
  }
%    \end{macrocode}
% \end{macro}
%
% \begin{macro}[internal]
%   {\ekvd@assert@twoargs,\ekvd@ifnottwoargs,\ekvd@ifempty@gtwo}
% A test for exactly two tokens can be reduced for an empty-test after gobbling
% two tokens, in the case that there are fewer tokens than two in the argument,
% only macros will be gobbled that are needed for the true branch, which doesn't
% hurt, and if there are more this will not be empty.
%    \begin{macrocode}
\long\def\ekvd@assert@twoargs#1%
  {%
    \ekvd@ifnottwoargs{#1}{\ekvd@err@missing@definition}%
  }
\long\def\ekvd@ifnottwoargs#1%
  {%
    \ekvd@ifempty@gtwo#1\ekv@ifempty@B
      \ekv@ifempty@false\ekv@ifempty@A\ekv@ifempty@B\@firstoftwo
  }
\long\def\ekvd@ifempty@gtwo#1#2{\ekv@ifempty@\ekv@ifempty@A}
%    \end{macrocode}
% \end{macro}
%
% \begin{macro}[internal]
%   {
%     \ekvd@assert@val,\ekvd@assert@val@,\ekvd@assert@noval,\ekvd@assert@noval@,
%     \ekvd@extract@args,\ekvd@extracted@args,\ekvd@one@arg@string
%   }
% Assert that a given key is defined as a value taking key or a |NoVal| key with
% the correct argument structure, respectively.
%    \begin{macrocode}
\protected\def\ekvd@assert@val#1%
  {%
    \ekvifdefined\ekvd@set{#1}%
      {\expandafter\ekvd@assert@val@\csname\ekv@name\ekvd@set{#1}\endcsname}%
      {%
        \ekvifdefinedNoVal\ekvd@set{#1}%
          \ekvd@err@add@val@on@noval
          {\ekvd@err@undefined@key{#1}}%
        \@gobble
      }%
  }
\protected\def\ekvd@assert@val@#1%
  {%
    \expandafter\ekvd@extract@args\meaning#1\ekvd@stop
    \unless\ifx\ekvd@extracted@args\ekvd@one@arg@string
      \ekvd@err@unsupported@arg
    \fi
    \@firstofone
  }%
\protected\def\ekvd@assert@noval#1%
  {%
    \ekvifdefinedNoVal\ekvd@set{#1}%
      {\expandafter\ekvd@assert@noval@\csname\ekv@name\ekvd@set{#1}N\endcsname}%
      {%
        \ekvifdefined\ekvd@set{#1}%
          \ekvd@err@add@noval@on@val
          {\ekvd@err@undefined@key{#1}}%
        \@gobble
      }%
  }
\protected\def\ekvd@assert@noval@#1%
  {%
    \expandafter\ekvd@extract@args\meaning#1\ekvd@stop
    \unless\ifx\ekvd@extracted@args\ekv@empty
      \ekvd@err@unsupported@arg
    \fi
    \@firstofone
  }
\protected\def\ekvd@extract@args#1%
  {%
    \protected\def\ekvd@extract@args##1#1##2->##3\ekvd@stop
      {\def\ekvd@extracted@args{##2}}%
  }
\expandafter\ekvd@extract@args\expandafter{\detokenize{macro:}}
\edef\ekvd@one@arg@string{\string#1}
%    \end{macrocode}
% \end{macro}
%
% \begin{macro}[internal]{\ekvd@assert@arg,\ekvd@assert@arg@msg,\ekvd@ifnoarg}
% There is no need to actually define |\ekvd@ifnoarg| here, as it will be set by
% either |\ekvd@arg| or |\ekvd@noarg|.
%    \begin{macrocode}
\def\ekvd@assert@arg{\ekvd@ifnoarg\ekvd@err@missing@definition}
\long\def\ekvd@assert@arg@msg#1%
  {%
    \ekvd@ifnoarg{\ekvd@err@missing@definition@msg{#1}}%
  }
%    \end{macrocode}
% \end{macro}
%
% \begin{macro}[internal]{\ekvd@assert@filledarg,\ekvd@ifnoarg@or@empty}
%    \begin{macrocode}
\long\def\ekvd@assert@filledarg#1%
  {%
    \ekvd@ifnoarg@or@empty{#1}\ekvd@err@missing@definition
  }
\long\def\ekvd@ifnoarg@or@empty#1%
  {%
    \ekvd@ifnoarg
      \@firstoftwo
      {\ekv@ifempty{#1}}%
  }
%    \end{macrocode}
% \end{macro}
%
% \begin{macro}[internal]
%   {
%     \ekvd@assert@not@long,\ekvd@assert@not@protected,\ekvd@assert@not@also
%     \ekvd@assert@not@long@also,\ekvd@assert@not@protected@also,
%     \ekvd@assert@new,\ekvd@assert@not@new
%   }
% Some key-types don't want to be |also|, |\long| or |\protected|, so we provide
% macros to test this and throw an error, this could be silently ignored but now
% users will learn to not use unnecessary stuff which slows the compilation
% down.
%    \begin{macrocode}
\def\ekvd@assert@not@long{\ifx\ekvd@long\long\ekvd@err@no@prefix{long}\fi}
\def\ekvd@assert@not@protected
  {\ifx\ekvd@prot\protected\ekvd@err@no@prefix{protected}\fi}
\def\ekvd@assert@not@also{\ekvd@ifalso{\ekvd@err@no@prefix{also}}{}}
\def\ekvd@assert@not@long@also
  {\ifx\ekvd@long\long\ekvd@err@no@prefix@also{long}\fi}
\def\ekvd@assert@not@protected@also
  {\ifx\ekvd@prot\protected\ekvd@err@no@prefix@also{protected}\fi}
\def\ekvd@assert@new#1#2%
  {\csname ekvifdefined#1\endcsname\ekvd@set{#2}{\ekvd@err@not@new}}
\def\ekvd@assert@not@new
  {\ifx\ekvd@ifnew\ekvd@assert@new\ekvd@err@no@prefix{new}\fi}
\def\ekvd@assert@new@for@name#1%
  {%
    \ifx\ekvd@ifnew\ekvd@assert@new
      \ekv@fi@firstoftwo
    \fi
    \@secondoftwo
      {\ekv@ifdefined{#1}\ekvd@err@not@new}%
      \@firstofone
  }
%    \end{macrocode}
% \end{macro}
%
% \begin{macro}[internal]
%   {
%     \ekvd@if@not@already@choice, \ekvd@if@not@already@choice@a,
%     \ekvd@if@not@already@choice@b
%   }
% It is bad to use |also| on a key that already contains a |choice|, as both
% choices would share the same valid values and thus lead to each callback being
% used twice. The following is a rudimentary test against this.
%    \begin{macrocode}
\protected\def\ekvd@if@not@already@choice#1%
  {%
    \expandafter\ekvd@if@not@already@choice@a
      \csname\ekv@name\ekvd@set{#1}\endcsname
      {}\ekvd@h@choice\ekvd@stop
  }
\protected\def\ekvd@if@not@already@choice@a
  {%
    \expandafter\ekvd@if@not@already@choice@b
  }
\long\protected\def\ekvd@if@not@already@choice@b#1\ekvd@h@choice#2\ekvd@stop
  {%
    \ekv@ifempty{#2}\@firstofone\@gobble
  }
%    \end{macrocode}
% \end{macro}
%
% \begin{macro}[internal]{\ekvd@ifspace,\ekvd@ifspace@}
% Yet another test which can be reduced to an if-empty, this time by gobbling
% everything up to the first space.
%    \begin{macrocode}
\long\def\ekvd@ifspace#1%
  {%
    \ekvd@ifspace@#1 \ekv@ifempty@B
      \ekv@ifempty@false\ekv@ifempty@A\ekv@ifempty@B\@firstoftwo
  }
\long\def\ekvd@ifspace@#1 % keep this space
  {%
    \ekv@ifempty@\ekv@ifempty@A
  }
%    \end{macrocode}
% \end{macro}
%
% \subsubsection{Messages}
%
% Most messages of \expkvd\ are not expandable, since they only appear during
% key-definition, which is not expandable anyway.
%
% \begin{macro}[internal]
%   {
%     \ekvd@errm,\ekvd@err@missing@definition,\ekvd@err@missing@definition@msg,
%     \ekvd@err@missing@type,\ekvd@err@undefined@prefix,\ekvd@err@undefined@key,
%     \ekvd@err@no@prefix,\ekvd@err@no@prefix@msg,\ekvd@err@no@prefix@also,
%     \ekvd@err@add@val@on@noval,\ekvd@err@add@noval@on@val,
%     \ekvd@err@unsupported@arg,\ekvd@err@not@new
%   }
% The non-expandable error messages are boring, so here they are:
%    \begin{macrocode}
\protected\def\ekvd@errm#1{\errmessage{expkv-def Error: #1}}
\protected\def\ekvd@err@missing@definition
  {\ekvd@errm{Missing definition for key `\ekvd@cur'}}
\protected\def\ekvd@err@missing@definition@msg#1%
  {\ekvd@errm{Missing definition for key `\ekv@unexpanded{#1}'}}
\protected\def\ekvd@err@missing@type
  {\ekvd@errm{Missing type prefix for key `\ekvd@cur'}}
\protected\def\ekvd@err@undefined@prefix#1%
  {%
    \ekvd@errm
      {%
        Undefined prefix `\ekv@unexpanded{#1}' found while processing
        `\ekvd@cur'%
      }%
  }
\protected\def\ekvd@err@undefined@key#1%
  {%
    \ekvd@errm
      {Undefined key `\ekv@unexpanded{#1}' found while processing `\ekvd@cur'}%
  }
\protected\def\ekvd@err@undefined@noval#1%
  {%
    \ekvd@errm
      {%
        Undefined noval key `\unexpanded{#1}' found while processing
        `\ekvd@cur'%
      }%
  }
\protected\def\ekvd@err@no@prefix#1%
  {\ekvd@errm{prefix `#1' not accepted in `\ekvd@cur'}}
\protected\def\ekvd@err@no@prefix@msg#1#2%
  {\ekvd@errm{prefix `#2' not accepted in `\ekv@unexpanded{#1}'}}
\protected\def\ekvd@err@no@prefix@also#1%
  {\ekvd@errm{`\ekvd@cur' not allowed with a `#1' key}}
\protected\def\ekvd@err@add@val@on@noval
  {\ekvd@errm{`\ekvd@cur' not allowed with a NoVal key}}
\protected\def\ekvd@err@add@noval@on@val
  {\ekvd@errm{`\ekvd@cur' not allowed with a value taking key}}
\protected\def\ekvd@err@unsupported@arg\fi\@firstofone#1%
  {%
    \fi
    \ekvd@errm
      {%
        Existing key-macro has the unsupported argument string
        `\ekvd@extracted@args' for key `\ekvd@cur'%
      }%
  }
\protected\def\ekvd@err@not@new
  {\ekvd@errm{The key for `\ekvd@cur' is already defined}}
\protected\long\def\ekvd@err@misused@unknown
  {\ekvd@errm{Misuse of the unknown type found while processing `\ekvd@cur'}}
%    \end{macrocode}
% \end{macro}
%
% \begin{macro}[internal]
%   {
%     \ekvd@err@choice@invalid,\ekvd@err@choice@invalid@,\ekvd@choice@name,
%     \ekvd@unknown@choice@name
%   }
% |\ekvd@err@choice@invalid| will have to use this mechanism to throw its
% message. Also we have to retrieve the name parts of the choice in an easy way,
% so we use parentheses of catcode 8 here, which should suffice in most cases to
% allow for a correct separation.
%    \begin{macrocode}
\def\ekvd@err@choice@invalid#1%
  {%
    \ekvd@err@choice@invalid@#1%
  }
\begingroup
\catcode40=8
\catcode41=8
\@firstofone{\endgroup
\def\ekvd@choice@name#1#2#3%
  {%
    ekvd#1(#2)\detokenize{#3}%
  }
\def\ekvd@unknown@choice@name#1#2%
  {%
    ekvd:u:#1(#2)%
  }
\def\ekvd@err@choice@invalid@ ekvd#1(#2)\detokenize#3%
  {%
    \ekv@ifdefined{\ekvd@unknown@choice@name{#1}{#2}}%
      {\csname\ekvd@unknown@choice@name{#1}{#2}\endcsname{#3}}%
      {\ekvd@err{invalid choice `#3' for `#2' in set `#1'}}%
  }
}
%    \end{macrocode}
% \end{macro}
%
% \begin{macro}[internal]{\ekvd@err}
% The expandable error messages use |\ekvd@err|, which is just like |\ekv@err|
% from \expkv. It uses a runaway argument to start the error message.
%    \begin{macrocode}
\ekv@exparg{\long\def\ekvd@err#1}{\ekverr{expkv-def}{#1}}
%    \end{macrocode}
% \end{macro}
%
% Now everything that's left is to reset the category code of |@|.
%    \begin{macrocode}
\catcode`\@=\ekvd@tmp
%    \end{macrocode}
%
% \gobbledocstriptag
%</tex>
%^^A=<<
%
%^^A=<<
%
% \end{implementation}^^A=<<
%
% \clearpage
% \PrintIndex
%
\endinput
%
^^A vim: ft=tex fdm=marker fmr=>>=,=<<

\ProvidesPackage{expkv-def}%
  [\ekvdDate\space v\ekvdVersion\space a key-defining frontend for expkv]
%    \end{macrocode}
% \gobbledocstriptag
%</pkg>
%^^A=<<
%^^A the ConTeXt module >>=
% \subsection{The \ConTeXt\ module}
% \gobbledocstriptag
%<*ctx>
%    \begin{macrocode}
\writestatus{loading}{ConTeXt User Module / expkv-def}
\usemodule[expkv]
\unprotect
\input expkv-def.tex
\writestatus{loading}
  {ConTeXt User Module / expkv-def / Version \ekvdVersion\space loaded}
\protect\endinput
%    \end{macrocode}
% \gobbledocstriptag
%</ctx>
%^^A=<<
%^^A main file >>=
% \subsection{The Generic Code}
% The rest of this implementation will be the generic code.
% \gobbledocstriptag
%<*tex>
%
% Load \expkv\ if the package didn't already do so -- since \expkv\ has
% safeguards against being loaded twice this does no harm and the overhead
% isn't that big. Also we reuse some of the internals of \expkv\ to save us from
% retyping them.
%    \begin{macrocode}
\input expkv
%    \end{macrocode}
%
% We make sure that \file{expkv-def.tex} is only input once:
%    \begin{macrocode}
\expandafter\ifx\csname ekvdVersion\endcsname\relax
\else
  \expandafter\endinput
\fi
%    \end{macrocode}
%
% \begin{macro}{\ekvdVersion,\ekvdDate}
% We're on our first input, so lets store the version and date in a macro.
%    \begin{macrocode}
\def\ekvdVersion{0.8c}
\def\ekvdDate{2021-09-20}
%    \end{macrocode}
% \end{macro}
%
% If the \LaTeX\ format is loaded we want to be a good file and report back who
% we are, for this the package will have defined |\ekvd@tmp| to use
% |\ProvidesFile|, else this will expand to a |\relax| and do no harm.
%    \begin{macrocode}
\csname ekvd@tmp\endcsname
%    \end{macrocode}
%
% Store the category code of |@| to later be able to reset it and change it to
% 11 for now.
%    \begin{macrocode}
\expandafter\chardef\csname ekvd@tmp\endcsname=\catcode`\@
\catcode`\@=11
%    \end{macrocode}
% |\ekvd@tmp| will be reused later to handle expansion during the key
% defining. But we don't need it to ever store information long-term after
% \expkvd\ was initialized.
%
% \begin{macro}[internal]{\ekvd@ifprimitive}
%    \begin{macrocode}
\protected\long\def\ekvd@ifprimitive#1%
  {%
    \begingroup
      \edef\ekvd@tmpa{\string #1}%
      \edef\ekvd@tmpb{\meaning#1}%
      \expandafter
    \endgroup
    \ifx\ekvd@tmpa\ekvd@tmpb
      \ekv@fi@firstoftwo
    \fi
    \@secondoftwo
  }
%    \end{macrocode}
% \end{macro}
%
% \begin{macro}[internal]
%   {\ekvd@long,\ekvd@prot,\ekvd@clear@prefixes,\ekvd@ifalso}
% \expkvd\ will use |\ekvd@long|, |\ekvd@prot|, and |\ekvd@ifalso| to store
% whether a key should be defined as |\long| or |\protected| or adds an action
% to an existing key, and we have to clear them for every new key. By default
% |long| and |protected| will just be empty, |ifalso| will be
% |\@secondoftwo|, and |ifnew| will just use its third argument.
%    \begin{macrocode}
\protected\def\ekvd@clear@prefixes
  {%
    \let\ekvd@long\ekv@empty
    \let\ekvd@prot\ekv@empty
    \let\ekvd@ifalso\@secondoftwo
    \long\def\ekvd@ifnew##1##2##3{##3}%
  }
\ekvd@clear@prefixes
%    \end{macrocode}
% \end{macro}
%
% \begin{macro}{\ekvdefinekeys}
% This is the one front-facing macro which provides the interface to define
% keys. It's using |\ekvparse| to handle the \kv\ list, the interpretation will
% be done by |\ekvd@noarg| and |\ekvd@|. The \set\ for which the keys should be
% defined is stored in |\ekvd@set|.
%    \begin{macrocode}
\protected\def\ekvdefinekeys#1%
  {%
    \def\ekvd@set{#1}%
    \ekvparse\ekvd@noarg\ekvd@arg
  }
%    \end{macrocode}
% \end{macro}
%
% \begin{macro}[internal]{\ekvd@noarg,\ekvd@arg,\ekvd@handle}
% |\ekvd@noarg| and |\ekvd@arg| store whether there was a value in the \kv\
% pair. |\ekvd@handle| has to test whether there is a space inside the key and
% if so calls the prefix grabbing routine, else we throw an error and ignore the
% key.
%    \begin{macrocode}
\protected\long\def\ekvd@noarg#1%
  {%
    \let\ekvd@ifnoarg\@firstoftwo
    \expandafter\ekvd@handle\detokenize{#1}\ekvd@stop{}%
  }
\protected\long\def\ekvd@arg#1%
  {%
    \let\ekvd@ifnoarg\@secondoftwo
    \expandafter\ekvd@handle\detokenize{#1}\ekvd@stop
  }
\protected\long\def\ekvd@handle#1\ekvd@stop#2%
  {%
    \ekvd@clear@prefixes
    \def\ekvd@cur{#1}%
    \ekvd@ifspace{#1}%
      {\ekvd@prefix\ekv@mark#1\ekv@stop{#2}}%
      \ekvd@err@missing@type
  }
%    \end{macrocode}
% \end{macro}
%
% \begin{macro}[internal]{\ekvd@prefix,\ekvd@prefix@}
% \expkvd\ separates prefixes into two groups, the first being prefixes in the
% \TeX\ sense (|long| and |protected|) which use |@p@| in their name, the other
% being key-types (|code|, |int|, \emph{etc.}) which use |@t@| instead.
% |\ekvd@prefix| splits at the first space and checks whether its a |@p@|
% or |@t@| type prefix. If it is neither throw an error and gobble the
% definition (the value).
%    \begin{macrocode}
\protected\def\ekvd@prefix#1 {\ekv@strip{#1}\ekvd@prefix@\ekv@mark}
\protected\def\ekvd@prefix@#1#2\ekv@stop
  {%
    \ekv@ifdefined{ekvd@t@#1}%
      {\ekv@strip{#2}{\csname ekvd@t@#1\endcsname}}%
      {%
        \ekv@ifdefined{ekvd@p@#1}%
          {\csname ekvd@p@#1\endcsname\ekvd@prefix@after@p{#2}}%
          {\ekvd@err@undefined@prefix{#1}\@gobble}%
      }%
  }
%    \end{macrocode}
% \end{macro}
%
% \begin{macro}[internal]{\ekvd@prefix@after@p}
% The |@p@| type prefixes are all just modifying a following |@t@| type, so they
% will need to search for another prefix. This is true for all of them, so we
% use a macro to handle this. It'll throw an error if there is no other prefix.
%    \begin{macrocode}
\protected\def\ekvd@prefix@after@p#1%
  {%
    \ekvd@ifspace{#1}%
      {\ekvd@prefix#1\ekv@stop}%
      {\ekvd@err@missing@type\@gobble}%
  }
%    \end{macrocode}
% \end{macro}
%
% \begin{macro}[internal]
%   {\ekvd@p@long,\ekvd@p@protected,\ekvd@p@protect,\ekvd@p@also,\ekvd@p@new}
% Define the |@p@| type prefixes, they all just store some information in a
% temporary macro.
%    \begin{macrocode}
\protected\def\ekvd@p@long{\let\ekvd@long\long}
\protected\def\ekvd@p@protected{\let\ekvd@prot\protected}
\let\ekvd@p@protect\ekvd@p@protected
\protected\def\ekvd@p@also{\let\ekvd@ifalso\@firstoftwo}
\protected\def\ekvd@p@new{\let\ekvd@ifnew\ekvd@assert@new}
%    \end{macrocode}
% \end{macro}
%
% \subsubsection{Key Types}
%
% \begin{macro}[internal]{\ekvd@type@set,\ekvd@t@set}
% The |set| type is quite straight forward, just define a |NoVal| key to call
% |\ekvchangeset|.
%    \begin{macrocode}
\protected\def\ekvd@type@set#1#2%
  {%
    \ekvd@assert@not@long
    \ekvd@assert@not@protected
    \ekvd@ifnew{NoVal}{#1}%
      {%
        \ekv@ifempty{#2}%
          {\ekvd@err@missing@definition}%
          {%
            \ekvd@ifalso
              {%
                \ekv@expargtwice{\ekvd@add@noval{#1}}%
                  {\ekvchangeset{#2}}%
                  \ekvd@assert@not@protected@also
              }%
              {\ekv@expargtwice{\ekvdefNoVal\ekvd@set{#1}}{\ekvchangeset{#2}}}%
          }%
      }%
  }
\protected\def\ekvd@t@set#1#2%
  {%
    \ekvd@ifnoarg
      {\ekvd@type@set{#1}{#1}}%
      {\ekvd@type@set{#1}{#2}}%
  }
%    \end{macrocode}
% \end{macro}
%
% \begin{macro}[internal]{\ekvd@type@noval,\ekvd@t@noval,\ekvd@t@enoval}
% Another pretty simple type, |noval| just needs to assert that there is a
% definition and that |long| wasn't specified.
% There are types where the difference in the variants is so small, that we
% define a common handler for them, those common handlers are named with
% |@type@|. |noval| and |enoval| are so similar that we can use such a |@type@|
% macro, even if we could've done |noval| in a slightly faster way without it.
%    \begin{macrocode}
\protected\long\def\ekvd@type@noval#1#2#3%
  {%
    \ekvd@ifnew{NoVal}{#2}%
      {%
        \ekvd@assert@arg
          {%
            \ekvd@assert@not@long
            \ekvd@prot#1\ekvd@tmp{#3}%
            \ekvd@ifalso
              {\ekv@exparg{\ekvd@add@noval{#2}}\ekvd@tmp{}}%
              {\ekvletNoVal\ekvd@set{#2}\ekvd@tmp}%
          }%
      }%
  }
\protected\def\ekvd@t@noval{\ekvd@type@noval\def}
\protected\def\ekvd@t@enoval{\ekvd@type@noval\edef}
%    \end{macrocode}
% \end{macro}
%
% \begin{macro}[internal]{\ekvd@type@code,\ekvd@t@code,\ekvd@t@ecode}
% |code| is simple as well, |ecode| has to use |\edef| on a temporary macro,
% since \expkv\ doesn't provide an |\ekvedef|.
%    \begin{macrocode}
\protected\long\def\ekvd@type@code#1#2#3%
  {%
    \ekvd@ifnew{}{#2}%
      {%
        \ekvd@assert@arg
          {%
            \ekvd@prot\ekvd@long#1\ekvd@tmp##1{#3}%
            \ekvd@ifalso
              {\ekv@exparg{\ekvd@add@val{#2}}{\ekvd@tmp{##1}}{}}%
              {\ekvlet\ekvd@set{#2}\ekvd@tmp}%
          }%
      }%
  }
\protected\def\ekvd@t@code{\ekvd@type@code\def}
\protected\def\ekvd@t@ecode{\ekvd@type@code\edef}
%    \end{macrocode}
% \end{macro}
%
% \begin{macro}[internal]
%   {
%     \ekvd@type@default,\ekvd@t@default,\ekvd@t@qdefault,\ekvd@t@odefault,
%     \ekvd@t@fdefault
%   }
% |\ekvd@type@default| asserts there was an argument, also the key for which one
% wants to set a default has to be already defined (this is not so important for
% |default|, but |qdefault| requires is). If everything is good, |\edef| a
% temporary macro that expands |\ekvd@set| and the |\csname| for the key, and in
% the case of |qdefault| does the first expansion step of the key-macro.
%    \begin{macrocode}
\protected\long\def\ekvd@type@default#1#2#3#4%
  {%
    \ekvd@assert@arg
      {%
        \ekvifdefined\ekvd@set{#3}%
          {%
            \ekvd@assert@not@new
            \ekvd@assert@not@long
            \ekvd@prot\edef\ekvd@tmp
              {%
                \ekv@unexpanded\expandafter#1%
                  {#2\csname\ekv@name\ekvd@set{#3}\endcsname{#4}}%
              }%
            \ekvd@ifalso
              {\ekv@exparg{\ekvd@add@noval{#3}}\ekvd@tmp{}}%
              {\ekvletNoVal\ekvd@set{#3}\ekvd@tmp}%
          }%
          {\ekvd@err@undefined@key{#3}}%
      }%
  }
\protected\def\ekvd@t@default{\ekvd@type@default{}{}}
\protected\def\ekvd@t@qdefault{\ekvd@type@default{\expandafter\expandafter}{}}
\let\ekvd@t@odefault\ekvd@t@qdefault
\protected\def\ekvd@t@fdefault{\ekvd@type@default{}{\romannumeral`\^^@}}
%    \end{macrocode}
% \end{macro}
%
% \begin{macro}[internal]{\ekvd@t@edefault}
% |edefault| is too different from |default| and |qdefault| to reuse the
% |@type@| macro, as it doesn't need |\unexpanded| inside of |\edef|.
%    \begin{macrocode}
\protected\long\def\ekvd@t@edefault#1#2%
  {%
    \ekvd@assert@arg
      {%
        \ekvifdefined\ekvd@set{#1}%
          {%
            \ekvd@assert@not@new
            \ekvd@assert@not@long
            \ekvd@prot\edef\ekvd@tmp
              {\csname\ekv@name\ekvd@set{#1}\endcsname{#2}}%
            \ekvd@ifalso
              {\ekv@exparg{\ekvd@add@noval{#1}}\ekvd@tmp{}}%
              {\ekvletNoVal\ekvd@set{#1}\ekvd@tmp}%
          }%
          {\ekvd@err@undefined@key{#1}}%
      }%
  }
%    \end{macrocode}
% \end{macro}
%
% \begin{macro}[internal]
%   {\ekvd@t@initial,\ekvd@t@oinitial,\ekvd@t@finitial,\ekvd@t@einitial}
%    \begin{macrocode}
\long\def\ekvd@type@initial#1#2#3#4%
  {%
    \ekvd@assert@not@new
    \ekvd@assert@not@also
    \ekvd@assert@not@long
    \ekvd@assert@not@protected
    \ekvd@ifnoarg
      {%
        \ekvifdefinedNoVal\ekvd@set{#3}%
          {\csname\ekv@name\ekvd@set{#3}N\endcsname}%
          {\ekvd@err@undefined@noval{#3}}%
      }%
      {%
        \ekvifdefined\ekvd@set{#3}%
          {%
            #1{#2#4}%
            \csname\ekv@name\ekvd@set{#3}\expandafter\endcsname\expandafter
              {\ekvd@tmp}%
          }%
          {\ekvd@err@undefined@key{#3}}%
      }%
  }
\def\ekvd@t@initial{\ekvd@type@initial{\def\ekvd@tmp}{}}
\def\ekvd@t@oinitial{\ekvd@type@initial{\ekv@exparg{\def\ekvd@tmp}}{}}
\def\ekvd@t@einitial{\ekvd@type@initial{\edef\ekvd@tmp}{}}
\def\ekvd@t@finitial
  {\ekvd@type@initial{\ekv@exparg{\def\ekvd@tmp}}{\romannumeral`\^^@}}
%    \end{macrocode}
% \end{macro}
%
% \begin{macro}[internal]
%   {
%     \ekvd@type@bool,\ekvd@t@bool,\ekvd@t@gbool,\ekvd@t@boolTF,\ekvd@t@gboolTF,
%     \ekvd@t@invbool,\ekvd@t@ginvbool,\ekvd@t@invboolTF,\ekvd@t@ginvboolTF,
%   }
% The boolean types are a quicker version of a |choice| that accept |true| and
% |false|, and set up the |NoVal| action to be identical to \texttt{\key=true}.
% The |true| and |false| actions are always just |\let|ting the macro in |#7| to
% some other macro (\emph{e.g.}, \cs[no-index]{iftrue}).
%    \begin{macrocode}
\protected\def\ekvd@type@bool#1#2#3#4#5%
  {%
    \ekvd@ifnew{}{#4}%
      {%
        \ekvd@ifnew{NoVal}{#4}%
          {%
            \ekvd@assert@filledarg{#5}%
              {%
                \ekvd@newlet#5#3%
                \ekvd@type@choice{#4}%
                \protected\ekvdefNoVal\ekvd@set{#4}{#1\let#5#2}%
                \protected\expandafter\def
                  \csname\ekvd@choice@name\ekvd@set{#4}{true}\endcsname
                  {#1\let#5#2}%
                \protected\expandafter\def
                  \csname\ekvd@choice@name\ekvd@set{#4}{false}\endcsname
                  {#1\let#5#3}%
              }%
          }%
      }%
  }
\protected\def\ekvd@t@bool{\ekvd@type@bool{}\iftrue\iffalse}
\protected\def\ekvd@t@gbool{\ekvd@type@bool\global\iftrue\iffalse}
\protected\def\ekvd@t@boolTF{\ekvd@type@bool{}\@firstoftwo\@secondoftwo}
\protected\def\ekvd@t@gboolTF{\ekvd@type@bool\global\@firstoftwo\@secondoftwo}
\protected\def\ekvd@t@invbool{\ekvd@type@bool{}\iffalse\iftrue}
\protected\def\ekvd@t@ginvbool{\ekvd@type@bool\global\iffalse\iftrue}
\protected\def\ekvd@t@invboolTF{\ekvd@type@bool{}\@secondoftwo\@firstoftwo}
\protected\def\ekvd@t@ginvboolTF
  {\ekvd@type@bool\global\@secondoftwo\@firstoftwo}
%    \end{macrocode}
% \end{macro}
%
% \begin{macro}[internal]
%   {
%     \ekvd@type@boolpair,
%     \ekvd@t@boolpair,\ekvd@t@gboolpair,\ekvd@t@boolpairTF,\ekvd@t@gboolpairTF
%   }
% The boolean pair types are essentially the same as the boolean types, but set
% two macros instead of one.
%    \begin{macrocode}
\protected\def\ekvd@type@boolpair#1#2#3#4#5#6%
  {%
    \ekvd@ifnew{}{#4}%
      {%
        \ekvd@ifnew{NoVal}{#4}%
          {%
            \ekvd@newlet#5#3%
            \ekvd@newlet#6#2%
            \ekvd@type@choice{#4}%
            \protected\ekvdefNoVal\ekvd@set{#4}{#1\let#5#2#1\let#6#3}%
            \protected\expandafter\def
              \csname\ekvd@choice@name\ekvd@set{#4}{true}\endcsname
              {#1\let#5#2#1\let#6#3}%
            \protected\expandafter\def
              \csname\ekvd@choice@name\ekvd@set{#4}{false}\endcsname
              {#1\let#5#3#1\let#6#2}%
          }%
      }%
  }
\protected\def\ekvd@t@boolpair#1#2%
  {\ekvd@assert@twoargs{#2}{\ekvd@type@boolpair{}\iftrue\iffalse{#1}#2}}
\protected\def\ekvd@t@gboolpair#1#2%
  {\ekvd@assert@twoargs{#2}{\ekvd@type@boolpair\global\iftrue\iffalse{#1}#2}}
\protected\def\ekvd@t@boolpairTF#1#2%
  {%
    \ekvd@assert@twoargs{#2}%
      {\ekvd@type@boolpair{}\@firstoftwo\@secondoftwo{#1}#2}%
  }
\protected\def\ekvd@t@gboolpairTF#1#2%
  {%
    \ekvd@assert@twoargs{#2}%
      {\ekvd@type@boolpair\global\@firstoftwo\@secondoftwo{#1}#2}%
  }
%    \end{macrocode}
% \end{macro}
%
% \begin{macro}[internal]
%   {
%     \ekvd@type@data,\ekvd@t@data,\ekvd@t@gdata,\ekvd@t@dataT,\ekvd@t@gdataT
%   }
%    \begin{macrocode}
\protected\def\ekvd@type@data#1#2#3#4#5#6%
  {%
    \ekvd@ifnew{}{#5}%
      {%
        \ekvd@assert@filledarg{#6}%
          {%
            \ekvd@newlet#6#1%
            \ekvd@ifalso
              {%
                \let\ekvd@prot\protected
                \ekvd@add@val{#5}{\long#2#6####1#3{####1{#4}}}{}%
              }%
              {%
                \protected\ekvd@long\ekvdef\ekvd@set{#5}%
                  {\long#2#6####1#3{####1{#4}}}%
              }%
          }%
      }%
  }
\protected\def\ekvd@t@data
  {\ekvd@type@data\@secondoftwo\edef{####2}{\ekv@unexpanded{##1}}}
\protected\def\ekvd@t@edata{\ekvd@type@data\@secondoftwo\edef{####2}{##1}}
\protected\def\ekvd@t@gdata
  {\ekvd@type@data\@secondoftwo\xdef{####2}{\ekv@unexpanded{##1}}}
\protected\def\ekvd@t@xdata{\ekvd@type@data\@secondoftwo\xdef{####2}{##1}}
\protected\def\ekvd@t@dataT
  {\ekvd@type@data\@gobble\edef{}{\ekv@unexpanded{##1}}}
\protected\def\ekvd@t@edataT{\ekvd@type@data\@gobble\edef{}{##1}}
\protected\def\ekvd@t@gdataT
  {\ekvd@type@data\@gobble\xdef{}{\ekv@unexpanded{##1}}}
\protected\def\ekvd@t@xdataT{\ekvd@type@data\@gobble\xdef{}{##1}}
%    \end{macrocode}
% \end{macro}
%
% \begin{macro}[internal]{\ekvd@type@box,\ekvd@t@box,\ekvd@t@gbox}
% Set up our boxes. Though we're a generic package we want to be colour safe, so
% we put an additional grouping level inside the box contents, for the case that
% someone uses \pkg{color}. |\ekvd@newreg| is a small wrapper which tests
% whether the first argument is defined and if not does
% |\csname new#2\endcsname#1|.
%    \begin{macrocode}
\protected\def\ekvd@type@box#1#2#3%
  {%
    \ekvd@ifnew{}{#2}%
      {%
        \ekvd@assert@filledarg{#3}%
          {%
            \ekvd@newreg#3{box}%
            \ekvd@ifalso
              {%
                \let\ekvd@prot\protected
                \ekvd@add@val{#2}{#1\setbox#3=\hbox{\begingroup##1\endgroup}}{}%
              }%
              {%
                \protected\ekvd@long\ekvdef\ekvd@set{#2}%
                  {#1\setbox#3=\hbox{\begingroup##1\endgroup}}%
              }%
          }%
      }%
  }
\protected\def\ekvd@t@box{\ekvd@type@box{}}
\protected\def\ekvd@t@gbox{\ekvd@type@box\global}
%    \end{macrocode}
% \end{macro}
%
% \begin{macro}[internal]{\ekvd@type@toks,\ekvd@t@toks,\ekvd@t@gtoks}
% Similar to |box|, but set the |toks|.
%    \begin{macrocode}
\protected\def\ekvd@type@toks#1#2#3%
  {%
    \ekvd@ifnew{}{#2}%
      {%
        \ekvd@assert@filledarg{#3}%
          {%
            \ekvd@newreg#3{toks}%
            \ekvd@ifalso
              {%
                \let\ekvd@prot\protected
                \ekvd@add@val{#2}{#1#3={##1}}{}%
              }%
              {\protected\ekvd@long\ekvdef\ekvd@set{#2}{#1#3={##1}}}%
          }%
      }%
  }
\protected\def\ekvd@t@toks{\ekvd@type@toks{}}
\protected\def\ekvd@t@gtoks{\ekvd@type@toks\global}
%    \end{macrocode}
% \end{macro}
%
% \begin{macro}[internal]{\ekvd@type@preapptoks,\ekvd@t@apptoks,\ekvd@t@gapptoks}
% Just like |toks|, but expand the current contents of the |toks| register to
% append the new contents.
%    \begin{macrocode}
\ekvd@ifprimitive\toksapp
  {%
    \protected\def\ekvd@type@preapptoks#1#2#3%
      {%
        \ekvd@ifnew{}{#2}%
          {%
            \ekvd@assert@filledarg{#3}%
              {%
                \ekvd@newreg#3{toks}%
                \ekvd@ifalso
                  {%
                    \let\ekvd@prot\protected
                    \ekvd@add@val{#2}{#1#3{##1}}{}%
                  }%
                  {\protected\ekvd@long\ekvdef\ekvd@set{#2}{#1#3{##1}}}%
              }%
          }%
      }
    \protected\def\ekvd@t@apptoks{\ekvd@type@preapptoks\toksapp}
    \protected\def\ekvd@t@gapptoks{\ekvd@type@preapptoks\gtoksapp}
    \protected\def\ekvd@t@pretoks{\ekvd@type@preapptoks\tokspre}
    \protected\def\ekvd@t@gpretoks{\ekvd@type@preapptoks\gtokspre}
  }
  {%
    \protected\def\ekvd@type@apptoks#1#2#3%
      {%
        \ekvd@ifnew{}{#2}%
          {%
            \ekvd@assert@filledarg{#3}%
              {%
                \ekvd@newreg#3{toks}%
                \ekvd@ifalso
                  {%
                    \let\ekvd@prot\protected
                    \ekvd@add@val{#2}{#1#3=\expandafter{\the#3##1}}{}%
                  }%
                  {%
                    \protected\ekvd@long\ekvdef\ekvd@set{#2}%
                      {#1#3=\expandafter{\the#3##1}}%
                  }%
              }%
          }%
      }
    \protected\def\ekvd@t@apptoks{\ekvd@type@apptoks{}}
    \protected\def\ekvd@t@gapptoks{\ekvd@type@apptoks\global}
    \newtoks\ekvd@toks
    \protected\def\ekvd@type@pretoks#1#2#3%
      {%
        \ekvd@ifnew{}{#2}%
          {%
            \ekvd@assert@filledarg{#3}%
              {%
                \ekvd@newreg#3{toks}%
                \ekvd@ifalso
                  {%
                    \let\ekvd@prot\protected
                    \ekvd@add@val{#2}%
                      {%
                        \ekvd@toks={##1}%
                        #1#3=\expandafter{\the\expandafter\ekvd@toks\the#3}%
                      }%
                      {}%
                  }%
                  {%
                    \protected\ekvd@long\ekvdef\ekvd@set{#2}%
                      {%
                        \ekvd@toks={##1}%
                        #1#3=\expandafter{\the\expandafter\ekvd@toks\the#3}%
                      }%
                  }%
              }%
          }%
      }
    \protected\def\ekvd@t@pretoks{\ekvd@type@pretoks{}}
    \protected\def\ekvd@t@gpretoks{\ekvd@type@pretoks\global}
  }
%    \end{macrocode}
% \end{macro}
%
% \begin{macro}[internal]
%   {
%     \ekvd@type@reg,\ekvd@t@int,\ekvd@t@eint,\ekvd@t@gint,\ekvd@t@xint,
%     \ekvd@t@dimen,\ekvd@t@edimen,\ekvd@t@gdimen,\ekvd@t@xdimen,
%     \ekvd@t@skip,\ekvd@t@eskip,\ekvd@t@gskip,\ekvd@t@xskip
%   }
% The |\ekvd@type@reg| can handle all the types for which the assignment will
% just be \texttt{\meta{register}=\meta{value}}.
%    \begin{macrocode}
\protected\def\ekvd@type@reg#1#2#3#4#5#6%
  {%
    \ekvd@ifnew{}{#5}%
      {%
        \ekvd@assert@filledarg{#6}%
          {%
            \ekvd@newreg#6{#1}%
            \ekvd@ifalso
              {%
                \let\evkd@prot\protected
                \ekvd@add@val{#5}{#2#6=#3##1#4\relax}{}%
              }%
              {\protected\ekvd@long\ekvdef\ekvd@set{#5}{#2#6=#3##1#4\relax}}%
          }%
      }%
  }
\protected\def\ekvd@t@int{\ekvd@type@reg{count}{}{}{}}
\protected\def\ekvd@t@eint{\ekvd@type@reg{count}{}\numexpr\relax}
\protected\def\ekvd@t@gint{\ekvd@type@reg{count}\global{}{}}
\protected\def\ekvd@t@xint{\ekvd@type@reg{count}\global\numexpr\relax}
\protected\def\ekvd@t@dimen{\ekvd@type@reg{dimen}{}{}{}}
\protected\def\ekvd@t@edimen{\ekvd@type@reg{dimen}{}\dimexpr\relax}
\protected\def\ekvd@t@gdimen{\ekvd@type@reg{dimen}\global{}{}}
\protected\def\ekvd@t@xdimen{\ekvd@type@reg{dimen}\global\dimexpr\relax}
\protected\def\ekvd@t@skip{\ekvd@type@reg{skip}{}{}{}}
\protected\def\ekvd@t@eskip{\ekvd@type@reg{skip}{}\glueexpr\relax}
\protected\def\ekvd@t@gskip{\ekvd@type@reg{skip}\global{}{}}
\protected\def\ekvd@t@xskip{\ekvd@type@reg{skip}\global\glueexpr\relax}
%    \end{macrocode}
% \end{macro}
%
% \begin{macro}[internal]{\ekvd@type@store,\ekvd@t@store,\ekvd@t@gstore}
% The none-expanding |store| types use an |\edef| or |\xdef| and |\unexpanded|
% to be able to also store |#| easily.
%    \begin{macrocode}
\protected\def\ekvd@type@store#1#2#3#4%
  {%
    \ekvd@ifnew{}{#3}%
      {%
        \ekvd@assert@filledarg{#4}%
          {%
            \ekvd@newlet#4\ekv@empty
            \ekvd@ifalso
              {%
                \let\ekvd@prot\protected
                \ekvd@add@val{#3}{#1#4{#2}}{}%
              }%
              {\protected\ekvd@long\ekvdef\ekvd@set{#3}{#1#4{#2}}}%
          }%
      }%
  }
\protected\def\ekvd@t@store{\ekvd@type@store\edef{\ekv@unexpanded{##1}}}
\protected\def\ekvd@t@gstore{\ekvd@type@store\xdef{\ekv@unexpanded{##1}}}
\protected\def\ekvd@t@estore{\ekvd@type@store\edef{##1}}
\protected\def\ekvd@t@xstore{\ekvd@type@store\xdef{##1}}
%    \end{macrocode}
% \end{macro}
%
% \begin{macro}[internal]
%   {
%     \ekvd@type@meta,\ekvd@type@meta@a,\ekvd@type@meta@b,\ekvd@type@meta@c,
%     \ekvd@t@meta,\ekvd@t@nmeta
%   }
% |meta| sets up things such that another instance of |\ekvset| will be run on
% the argument, with the same \set.
%    \begin{macrocode}
\protected\long\def\ekvd@type@meta#1#2#3#4#5#6#7%
  {%
    \ekvd@ifnew{#1}{#6}%
      {%
        \ekvd@assert@filledarg{#7}%
          {%
            \edef\ekvd@tmp{\ekvd@set}%
            \expandafter\ekvd@type@meta@a\expandafter{\ekvd@tmp}{#7}{#2}%
            \ekvd@ifalso
              {\ekv@exparg{#3{#6}}{\ekvd@tmp#4}{#5}}%
              {\csname ekvlet#1\endcsname\ekvd@set{#6}\ekvd@tmp}%
          }%
      }%
  }
\protected\long\def\ekvd@type@meta@a#1#2%
  {%
    \expandafter\ekvd@type@meta@b\expandafter{\ekvset{#1}{#2}}%
  }
\protected\def\ekvd@type@meta@b
  {%
    \expandafter\ekvd@type@meta@c\expandafter
  }
\protected\long\def\ekvd@type@meta@c#1#2%
  {%
    \ekvd@prot\ekvd@long\def\ekvd@tmp#2{#1}%
  }
\protected\def\ekvd@t@meta{\ekvd@type@meta{}{##1}\ekvd@add@val{{##1}}{}}
\protected\def\ekvd@t@nmeta
  {%
    \ekvd@assert@not@long
    \ekvd@type@meta{NoVal}{}\ekvd@add@noval{}\ekvd@assert@not@long@also
  }
%    \end{macrocode}
% \end{macro}
%
% \begin{macro}[internal]
%   {\ekvd@type@smeta,\ekvd@type@smeta@,\ekvd@t@smeta,\ekvd@t@snmeta}
% |smeta| is pretty similar to |meta|, but needs two arguments inside of \val,
% such that the first is the \set\ for which the sub-|\ekvset| and the second is
% the \kv\ list.
%    \begin{macrocode}
\protected\long\def\ekvd@type@smeta#1#2#3#4#5#6#7%
  {%
    \ekvd@ifnew{#1}{#6}%
      {%
        \ekvd@assert@twoargs{#7}%
          {%
            \ekvd@type@meta@a#7{#2}%
            \ekvd@ifalso
              {\ekv@exparg{#3{#6}}{\ekvd@tmp#4}{#5}}%
              {\csname ekvlet#1\endcsname\ekvd@set{#6}\ekvd@tmp}%
          }%
      }%
  }
\protected\def\ekvd@t@smeta{\ekvd@type@smeta{}{##1}\ekvd@add@val{{##1}}{}}
\protected\def\ekvd@t@snmeta
  {%
    \ekvd@assert@not@long
    \ekvd@type@smeta{NoVal}{}\ekvd@add@noval{}\ekvd@assert@not@long@also
  }
%    \end{macrocode}
% \end{macro}
%
% \begin{macro}[internal]
%   {
%     \ekvd@type@choice,
%     \ekvd@populate@choice,\ekvd@populate@choice@,\ekvd@populate@choice@noarg,
%     \ekvd@choice@prefix,\ekvd@choice@prefix@,\ekvd@choice@prefix@done,
%     \ekvd@choice@p@protected,\ekvd@choice@p@protect,
%     \ekvd@choice@p@long,\ekvd@choice@p@long@,
%     \ekvd@t@choice
%   }
% The |choice| type is by far the most complex type, as we have to run a
% sub-parser on the choice-definition list, which should support the |@p@| type
% prefixes as well (but |long| will always throw an error, as they are not
% allowed to be long). |\ekvd@type@choice| will just define the choice-key, the
% handling of the choices definition will be done by |\ekvd@populate@choice|.
%    \begin{macrocode}
\protected\def\ekvd@type@choice#1%
  {%
    \ekvd@assert@not@long
    \ekv@exparg{\ekvd@prot\def\ekvd@tmp##1}%
      {%
        \expandafter\ekvd@h@choice\expandafter
          {\ekvd@choice@name\ekvd@set{#1}{##1}}%
      }%
    \ekvd@ifalso
      {%
        \ekvd@assert@val{#1}%
          {%
            \ekvd@if@not@already@choice{#1}%
              {%
                \ekv@exparg
                  {%
                    \expandafter\ekvd@add@aux
                      \csname\ekv@name\ekvd@set{#1}\endcsname{{##1}}{#1}%
                  }%
                  {\ekvd@tmp{##1}}%
                  {\ekvd@long\ekvdef}\ekvd@assert@not@long@also
              }%
          }%
      }%
      {\ekvlet\ekvd@set{#1}\ekvd@tmp}%
  }
%    \end{macrocode}
% |\ekvd@populate@choice| just uses |\ekvparse| and then gives control to
% |\ekvd@populate@choice@noarg|, which throws an error, and
% |\ekvd@populate@choice@|.
%    \begin{macrocode}
\protected\def\ekvd@populate@choice
  {%
    \ekvparse\ekvd@populate@choice@noarg\ekvd@populate@choice@
  }
\protected\long\def\ekvd@populate@choice@noarg#1%
  {%
    \expandafter\ekvd@err@missing@definition@msg\expandafter{\ekvd@cur : #1}%
  }
%    \end{macrocode}
% |\ekvd@populate@choice@| runs the prefix-test, if there is none we can
% directly define the choice, for that |\ekvd@set@choice| will expand to the
% current choice-key's name, which will have been defined by |\ekvd@t@choice|.
% If there is a prefix run the prefix grabbing routine, which was altered for
% |@type@choice|.
%    \begin{macrocode}
\protected\long\def\ekvd@populate@choice@#1#2%
  {%
    \ekvd@clear@prefixes
    \ekvd@ifspace{#1}%
      {\ekvd@choice@prefix{\ekv@mark#1}\ekv@mark#1\ekv@stop}%
      {%
        \expandafter\edef
          \csname\ekvd@choice@name\ekvd@set\ekvd@set@choice{#1}\endcsname
      }%
      {\unexpanded{#2}}%
  }
\protected\def\ekvd@choice@prefix#1#2
  {%
    \ekv@strip{#2}{\ekvd@choice@prefix@{#1}}\ekv@mark
  }
\protected\def\ekvd@choice@prefix@#1#2#3\ekv@stop
  {%
    \ekv@ifdefined{ekvd@choice@p@#2}%
      {%
        \csname ekvd@choice@p@#2\endcsname
        \ekvd@ifspace{#3}%
          {\ekvd@choice@prefix{#3}#3\ekv@stop}%
          {\ekvd@choice@prefix@done{#3}}%
      }%
      {\ekvd@choice@prefix@done{#1}}%
  }
\protected\def\ekvd@choice@prefix@done#1%
  {%
    \ekvd@prot\expandafter\edef
      \csname
        \ekv@strip{#1}{\ekvd@choice@name\ekvd@set\ekvd@set@choice}%
      \endcsname
  }
\protected\def\ekvd@choice@p@protected{\let\ekvd@prot\protected}
\let\ekvd@choice@p@protect\ekvd@choice@p@protected
\protected\def\ekvd@choice@invalid@p#1\ekvd@ifspace#2%
  {%
    \expandafter\ekvd@choice@invalid@p@\expandafter{\ekv@gobble@mark#2}{#1}%
    \ekvd@ifspace{#2}%
  }
\protected\def\ekvd@choice@invalid@p@#1#2%
  {%
    \expandafter\ekvd@err@no@prefix@msg\expandafter{\ekvd@cur : #2 #1}{#2}%
  }
\protected\def\ekvd@choice@p@long{\ekvd@choice@invalid@p{long}}%
\protected\def\ekvd@choice@p@also{\ekvd@choice@invalid@p{also}}%
\protected\def\ekvd@choice@p@new{\ekvd@choice@invalid@p{new}}%
%    \end{macrocode}
% Finally we're able to set up the |@t@choice| macro, which has to store the
% current choice-key's name, define the key, and parse the available choices.
%    \begin{macrocode}
\protected\long\def\ekvd@t@choice#1#2%
  {%
    \ekvd@ifnew{}{#1}%
      {%
        \ekvd@assert@arg
          {%
            \ekvd@type@choice{#1}%
            \def\ekvd@set@choice{#1}%
            \ekvd@populate@choice{#2}%
          }%
      }%
  }
%    \end{macrocode}
% \end{macro}
%
% \begin{macro}[internal]{\ekvd@t@choice-store,\ekvd@t@choice-enum}
%   These two types define a special kind of |choice| key and are quite similar,
%   the only difference is what the different choices do (hence they use a
%   shared initialisation which differs in the chosen |populate| step).
%    \begin{macrocode}
\protected\long\expandafter\def\csname ekvd@t@choice-store\endcsname
  {\ekvd@type@choicespecial\ekvd@populate@choicestore}
\protected\long\expandafter\def\csname ekvd@t@choice-enum\endcsname
  {\ekvd@type@choicespecial\ekvd@populate@choiceenum}
%    \end{macrocode}
% \begin{macro}[internal]{\ekvd@type@choicespecial}
%   Initialise similar to a |choice| key. The difference is that we require two
%   arguments (which we assert), a macro to store things in, and a |csv|-list
%   containing the allowed values. |#1| is the |populate| macro according to the
%   type used.
%    \begin{macrocode}
\protected\long\def\ekvd@type@choicespecial#1#2#3%
  {%
    \ekvd@ifnew{}{#2}%
      {%
        \ekvd@assert@twoargs{#3}%
          {%
            \ekvd@type@choice{#2}%
            \def\ekvd@set@choice{#2}%
            #1#3%
          }%
      }%
  }
%    \end{macrocode}
% \end{macro}
% \begin{macro}[internal]
%   {\ekvd@populate@choicestore,\ekvd@populate@choicestore@}
%   We initialise the storing macro if it doesn't yet exist, and then we loop
%   over the value list. The |\edef|s with |\unexpanded| are both necessary to
%   be able to store macro parameter tokens (the outer protects at define time,
%   the inner at use time).
%    \begin{macrocode}
\protected\long\def\ekvd@populate@choicestore#1%
  {%
    \ekvd@newlet#1\ekv@empty
    \ekvcsvloop{\ekvd@populate@choicestore@#1}%
  }
\protected\long\def\ekvd@populate@choicestore@#1#2%
  {%
    \protected\expandafter\edef
      \csname\ekvd@choice@name\ekvd@set\ekvd@set@choice{#2}\endcsname
      {\unexpanded{\edef#1{\unexpanded{#2}}}}%
  }
%    \end{macrocode}
% \end{macro}
% \begin{macro}[internal]
%   {\ekvd@populate@choiceenum,\ekvd@populate@choiceenum@}
%   This is similar to the population of a |choice-store| type, but instead of
%   storing the values in a macro this initialises a count and stores the
%   position of the value in the list inside that count (zero-indexed). The
%   space is necessary to terminate the number scanning, which is the reason we
%   use |\@firstofone| (so that the space after the macro name isn't gobbled by
%   \TeX).
%    \begin{macrocode}
\protected\long\def\ekvd@populate@choiceenum#1%
  {%
    \ekvd@newreg#1{count}%
    \def\ekvd@tmp{0}%
    \ekvcsvloop{\ekvd@populate@choiceenum@#1}%
  }
\protected\long\def\ekvd@populate@choiceenum@#1#2%
  {%
    \protected\expandafter\edef
      \csname\ekvd@choice@name\ekvd@set\ekvd@set@choice{#2}\endcsname
      {#1=\@firstofone{\ekvd@tmp} }%
    \edef\ekvd@tmp{\the\numexpr\ekvd@tmp+1\relax}%
  }
%    \end{macrocode}
% \end{macro}
% \end{macro}
%
% \begin{macro}[internal]{\ekvd@t@unknown-choice}
%    \begin{macrocode}
\protected\long\expandafter\def\csname ekvd@t@unknown-choice\endcsname#1#2%
  {%
    \ekvd@assert@new@for@name{\ekvd@unknown@choice@name\ekvd@set{#1}}%
      {%
        \ekvd@assert@arg
          {%
            \ekvd@assert@not@long
            \ekvd@assert@not@also
            \ekvd@prot\expandafter
            \def\csname\ekvd@unknown@choice@name\ekvd@set{#1}\endcsname##1{#2}%
          }%
      }%
  }
%    \end{macrocode}
% \end{macro}
%
% \begin{macro}[internal]
%   {\ekvd@t@unknown,\ekvd@type@unknown@code,\ekvd@type@unknown@noval}
%   The |unknown| type has different subtypes which would be the key names for
%   other types. It is first checked whether that subtype is defined, if it
%   isn't throw an error, else use that subtype.
%    \begin{macrocode}
\protected\long\def\ekvd@t@unknown#1#2%
  {%
    \ekv@ifdefined{ekvd@type@unknown@\detokenize{#1}}%
      {\csname ekvd@type@unknown@\detokenize{#1}\endcsname{#2}}%
      \ekvd@err@misused@unknown
  }
%    \end{macrocode}
%   The |unknown noval| type can use |\ekvdefunknownNoVal| directly (after
%   asserting some prefixes).
%    \begin{macrocode}
\protected\long\def\ekvd@type@unknown@noval#1%
  {%
    \ekvd@assert@new@for@name{\ekv@name\ekvd@set{}uN}%
      {%
        \ekvd@assert@arg
          {%
            \ekvd@assert@not@also
            \ekvd@assert@not@long
            \ekvd@prot\ekvdefunknownNoVal\ekvd@set{#1}%
          }%
      }%
  }
%    \end{macrocode}
%   The |unknown code| type uses some trickery during the definition in order to
%   swap out |#1| and |#2| in the user supplied definition. This is done via a
%   temporary macro that stores the definition but gets the parameter numbers
%   reversed while the real definition is done.
%    \begin{macrocode}
\protected\long\def\ekvd@type@unknown@code#1%
  {%
    \ekvd@assert@new@for@name{\ekv@name\ekvd@set{}u}%
      {%
        \ekvd@assert@arg
          {%
            \ekvd@assert@not@also
            \begingroup
              \def\ekvd@tmp##1##2{#1}%
              \ekv@exparg
              {%
                \endgroup
                \ekvd@prot\ekvd@long\ekvdefunknown\ekvd@set
              }%
              {\ekvd@tmp{##2}{##1}}%
          }%
      }%
  }
%    \end{macrocode}
% \end{macro}
%
% \begin{macro}[internal]
%   {
%     \ekvd@type@unknown@redirect,
%     \ekvd@type@unknown@redirect-code,
%     \ekvd@type@unknown@redirect-noval
%   }
%   The |unknown redirect| types also just forward to |\ekvredirectunknown|
%   after asserting some prefixes.
%    \begin{macrocode}
\protected\edef\ekvd@type@unknown@redirect#1%
  {%
    \expandafter\noexpand\csname ekvd@type@unknown@redirect-code\endcsname{#1}%
    \expandafter\noexpand\csname ekvd@type@unknown@redirect-noval\endcsname{#1}%
  }
\protected\expandafter\def\csname ekvd@type@unknown@redirect-code\endcsname#1%
  {%
    \ekvd@assert@new@for@name{\ekv@name\ekvd@set{}u}%
      {%
        \ekvd@assert@arg
          {%
            \ekvd@assert@not@also
            \ekvd@assert@not@protected
            \expandafter\ekvredirectunknown\expandafter{\ekvd@set}{#1}%
          }%
      }%
  }
\protected\expandafter\def\csname ekvd@type@unknown@redirect-noval\endcsname#1%
  {%
    \ekvd@assert@new@for@name{\ekv@name\ekvd@set{}uN}%
      {%
        \ekvd@assert@arg
          {%
            \ekvd@assert@not@also
            \ekvd@assert@not@protected
            \ekvd@assert@not@long
            \expandafter\ekvredirectunknownNoVal\expandafter{\ekvd@set}{#1}%
          }%
      }%
  }
%    \end{macrocode}
% \end{macro}
%
% \subsubsection{Key Type Helpers}
% There are some keys that might need helpers during their execution (not during
% their definition, which are gathered as |@type@| macros). These helpers are
% named |@h@|.
%
% \begin{macro}[internal]{\ekvd@h@choice,\ekvd@h@choice@}
% The |choice| helper will just test whether the given choice was defined, if
% not throw an error expandably, else call the macro which stores the code for
% this choice.
%    \begin{macrocode}
\def\ekvd@h@choice#1%
  {%
    \expandafter\ekvd@h@choice@
      \csname\ifcsname#1\endcsname#1\else relax\fi\endcsname
      {#1}%
  }
\def\ekvd@h@choice@#1#2%
  {%
    \ifx#1\relax
      \ekvd@err@choice@invalid{#2}%
      \expandafter\@gobble
    \fi
    #1%
  }
%    \end{macrocode}
% \end{macro}
%
%
% \subsubsection{Handling \texttt{also}}
%
% \begin{macro}[internal]
%   {\ekvd@add@val,\ekvd@add@noval,\ekvd@add@aux,\ekvd@add@aux@}
%    \begin{macrocode}
\protected\long\def\ekvd@add@val#1#2#3%
  {%
    \ekvd@assert@val{#1}%
      {%
        \expandafter\ekvd@add@aux\csname\ekv@name\ekvd@set{#1}\endcsname{{##1}}%
          {#1}{#2}{\ekvd@long\ekvdef}{#3}%
      }%
  }
\protected\long\def\ekvd@add@noval#1#2#3%
  {%
    \ekvd@assert@noval{#1}%
      {%
        \expandafter\ekvd@add@aux\csname\ekv@name\ekvd@set{#1}N\endcsname{}%
          {#1}{#2}\ekvdefNoVal{#3}%
      }%
  }
\protected\long\def\ekvd@add@aux#1#2%
  {%
    \ekvd@extract@prefixes#1%
    \expandafter\ekvd@add@aux@\expandafter{#1#2}%
  }
\protected\long\def\ekvd@add@aux@#1#2#3#4#5%
  {%
    #5%
    \ekvd@prot#4\ekvd@set{#2}{#1#3}%
  }
%    \end{macrocode}
% \end{macro}
%
% \begin{macro}[internal]
%   {
%     \ekvd@extract@prefixes,\ekvd@extract@prefixes@,
%     \ekvd@extract@prefixes@long,\ekvd@extract@prefixes@prot
%   }
% This macro checks which prefixes were used for the definition of a macro and
% sets |\ekvd@long| and |\ekvd@prot| accordingly.
%    \begin{macrocode}
\protected\def\ekvd@extract@prefixes#1%
  {%
    \expandafter\ekvd@extract@prefixes@\meaning#1\ekvd@stop
  }
%    \end{macrocode}
% In the following definition |#1| will get replaced by |macro:|, |#2| by
% |\long| and |#3| by |\protected| (in each, all tokens will have category
% other). This allows us to parse the |\meaning| of a macro for those strings.
%    \begin{macrocode}
\protected\def\ekvd@extract@prefixes@#1#2#3%
  {%
    \protected\def\ekvd@extract@prefixes@##1#1##2\ekvd@stop
      {%
        \ekvd@extract@prefixes@long
          ##1\ekvd@mark\@firstofone#2\ekvd@mark\@gobble\ekvd@stop
          {\let\ekvd@long\long}%
        \ekvd@extract@prefixes@prot
          ##1\ekvd@mark\@firstofone#3\ekvd@mark\@gobble\ekvd@stop
          {\let\ekvd@prot\protected}%
      }%
    \protected\def\ekvd@extract@prefixes@long##1#2##2\ekvd@mark##3##4\ekvd@stop
      {##3}%
    \protected\def\ekvd@extract@prefixes@prot##1#3##2\ekvd@mark##3##4\ekvd@stop
      {##3}%
  }
%    \end{macrocode}
% We use a temporary macro to expand the three arguments of
% |\ekvd@extract@prefixes@|, which will set up the real meaning of itself and
% the parsing for |\long| and |\protected|.
%    \begin{macrocode}
\begingroup
\edef\ekvd@tmp
  {%
    \endgroup
    \ekvd@extract@prefixes@
      {\detokenize{macro:}}%
      {\string\long}%
      {\string\protected}%
  }
\ekvd@tmp
%    \end{macrocode}
% \end{macro}
%
%
% \subsubsection{Tests}
%
% \begin{macro}[internal]{\ekvd@newlet,\ekvd@newreg}
% These macros test whether a control sequence is defined, if it isn't they
% define it, either via |\let| or via the correct \cs[no-index]{new\meta{reg}}.
%    \begin{macrocode}
\protected\def\ekvd@newlet#1#2%
  {%
    \ifdefined#1\ekv@fi@gobble\fi\@firstofone{\let#1#2}%
  }
\protected\def\ekvd@newreg#1#2%
  {%
    \ifdefined#1\ekv@fi@gobble\fi\@firstofone{\csname new#2\endcsname#1}%
  }
%    \end{macrocode}
% \end{macro}
%
% \begin{macro}[internal]
%   {\ekvd@assert@twoargs,\ekvd@ifnottwoargs,\ekvd@ifempty@gtwo}
% A test for exactly two tokens can be reduced for an empty-test after gobbling
% two tokens, in the case that there are fewer tokens than two in the argument,
% only macros will be gobbled that are needed for the true branch, which doesn't
% hurt, and if there are more this will not be empty.
%    \begin{macrocode}
\long\def\ekvd@assert@twoargs#1%
  {%
    \ekvd@ifnottwoargs{#1}{\ekvd@err@missing@definition}%
  }
\long\def\ekvd@ifnottwoargs#1%
  {%
    \ekvd@ifempty@gtwo#1\ekv@ifempty@B
      \ekv@ifempty@false\ekv@ifempty@A\ekv@ifempty@B\@firstoftwo
  }
\long\def\ekvd@ifempty@gtwo#1#2{\ekv@ifempty@\ekv@ifempty@A}
%    \end{macrocode}
% \end{macro}
%
% \begin{macro}[internal]
%   {
%     \ekvd@assert@val,\ekvd@assert@val@,\ekvd@assert@noval,\ekvd@assert@noval@,
%     \ekvd@extract@args,\ekvd@extracted@args,\ekvd@one@arg@string
%   }
% Assert that a given key is defined as a value taking key or a |NoVal| key with
% the correct argument structure, respectively.
%    \begin{macrocode}
\protected\def\ekvd@assert@val#1%
  {%
    \ekvifdefined\ekvd@set{#1}%
      {\expandafter\ekvd@assert@val@\csname\ekv@name\ekvd@set{#1}\endcsname}%
      {%
        \ekvifdefinedNoVal\ekvd@set{#1}%
          \ekvd@err@add@val@on@noval
          {\ekvd@err@undefined@key{#1}}%
        \@gobble
      }%
  }
\protected\def\ekvd@assert@val@#1%
  {%
    \expandafter\ekvd@extract@args\meaning#1\ekvd@stop
    \unless\ifx\ekvd@extracted@args\ekvd@one@arg@string
      \ekvd@err@unsupported@arg
    \fi
    \@firstofone
  }%
\protected\def\ekvd@assert@noval#1%
  {%
    \ekvifdefinedNoVal\ekvd@set{#1}%
      {\expandafter\ekvd@assert@noval@\csname\ekv@name\ekvd@set{#1}N\endcsname}%
      {%
        \ekvifdefined\ekvd@set{#1}%
          \ekvd@err@add@noval@on@val
          {\ekvd@err@undefined@key{#1}}%
        \@gobble
      }%
  }
\protected\def\ekvd@assert@noval@#1%
  {%
    \expandafter\ekvd@extract@args\meaning#1\ekvd@stop
    \unless\ifx\ekvd@extracted@args\ekv@empty
      \ekvd@err@unsupported@arg
    \fi
    \@firstofone
  }
\protected\def\ekvd@extract@args#1%
  {%
    \protected\def\ekvd@extract@args##1#1##2->##3\ekvd@stop
      {\def\ekvd@extracted@args{##2}}%
  }
\expandafter\ekvd@extract@args\expandafter{\detokenize{macro:}}
\edef\ekvd@one@arg@string{\string#1}
%    \end{macrocode}
% \end{macro}
%
% \begin{macro}[internal]{\ekvd@assert@arg,\ekvd@assert@arg@msg,\ekvd@ifnoarg}
% There is no need to actually define |\ekvd@ifnoarg| here, as it will be set by
% either |\ekvd@arg| or |\ekvd@noarg|.
%    \begin{macrocode}
\def\ekvd@assert@arg{\ekvd@ifnoarg\ekvd@err@missing@definition}
\long\def\ekvd@assert@arg@msg#1%
  {%
    \ekvd@ifnoarg{\ekvd@err@missing@definition@msg{#1}}%
  }
%    \end{macrocode}
% \end{macro}
%
% \begin{macro}[internal]{\ekvd@assert@filledarg,\ekvd@ifnoarg@or@empty}
%    \begin{macrocode}
\long\def\ekvd@assert@filledarg#1%
  {%
    \ekvd@ifnoarg@or@empty{#1}\ekvd@err@missing@definition
  }
\long\def\ekvd@ifnoarg@or@empty#1%
  {%
    \ekvd@ifnoarg
      \@firstoftwo
      {\ekv@ifempty{#1}}%
  }
%    \end{macrocode}
% \end{macro}
%
% \begin{macro}[internal]
%   {
%     \ekvd@assert@not@long,\ekvd@assert@not@protected,\ekvd@assert@not@also
%     \ekvd@assert@not@long@also,\ekvd@assert@not@protected@also,
%     \ekvd@assert@new,\ekvd@assert@not@new
%   }
% Some key-types don't want to be |also|, |\long| or |\protected|, so we provide
% macros to test this and throw an error, this could be silently ignored but now
% users will learn to not use unnecessary stuff which slows the compilation
% down.
%    \begin{macrocode}
\def\ekvd@assert@not@long{\ifx\ekvd@long\long\ekvd@err@no@prefix{long}\fi}
\def\ekvd@assert@not@protected
  {\ifx\ekvd@prot\protected\ekvd@err@no@prefix{protected}\fi}
\def\ekvd@assert@not@also{\ekvd@ifalso{\ekvd@err@no@prefix{also}}{}}
\def\ekvd@assert@not@long@also
  {\ifx\ekvd@long\long\ekvd@err@no@prefix@also{long}\fi}
\def\ekvd@assert@not@protected@also
  {\ifx\ekvd@prot\protected\ekvd@err@no@prefix@also{protected}\fi}
\def\ekvd@assert@new#1#2%
  {\csname ekvifdefined#1\endcsname\ekvd@set{#2}{\ekvd@err@not@new}}
\def\ekvd@assert@not@new
  {\ifx\ekvd@ifnew\ekvd@assert@new\ekvd@err@no@prefix{new}\fi}
\def\ekvd@assert@new@for@name#1%
  {%
    \ifx\ekvd@ifnew\ekvd@assert@new
      \ekv@fi@firstoftwo
    \fi
    \@secondoftwo
      {\ekv@ifdefined{#1}\ekvd@err@not@new}%
      \@firstofone
  }
%    \end{macrocode}
% \end{macro}
%
% \begin{macro}[internal]
%   {
%     \ekvd@if@not@already@choice, \ekvd@if@not@already@choice@a,
%     \ekvd@if@not@already@choice@b
%   }
% It is bad to use |also| on a key that already contains a |choice|, as both
% choices would share the same valid values and thus lead to each callback being
% used twice. The following is a rudimentary test against this.
%    \begin{macrocode}
\protected\def\ekvd@if@not@already@choice#1%
  {%
    \expandafter\ekvd@if@not@already@choice@a
      \csname\ekv@name\ekvd@set{#1}\endcsname
      {}\ekvd@h@choice\ekvd@stop
  }
\protected\def\ekvd@if@not@already@choice@a
  {%
    \expandafter\ekvd@if@not@already@choice@b
  }
\long\protected\def\ekvd@if@not@already@choice@b#1\ekvd@h@choice#2\ekvd@stop
  {%
    \ekv@ifempty{#2}\@firstofone\@gobble
  }
%    \end{macrocode}
% \end{macro}
%
% \begin{macro}[internal]{\ekvd@ifspace,\ekvd@ifspace@}
% Yet another test which can be reduced to an if-empty, this time by gobbling
% everything up to the first space.
%    \begin{macrocode}
\long\def\ekvd@ifspace#1%
  {%
    \ekvd@ifspace@#1 \ekv@ifempty@B
      \ekv@ifempty@false\ekv@ifempty@A\ekv@ifempty@B\@firstoftwo
  }
\long\def\ekvd@ifspace@#1 % keep this space
  {%
    \ekv@ifempty@\ekv@ifempty@A
  }
%    \end{macrocode}
% \end{macro}
%
% \subsubsection{Messages}
%
% Most messages of \expkvd\ are not expandable, since they only appear during
% key-definition, which is not expandable anyway.
%
% \begin{macro}[internal]
%   {
%     \ekvd@errm,\ekvd@err@missing@definition,\ekvd@err@missing@definition@msg,
%     \ekvd@err@missing@type,\ekvd@err@undefined@prefix,\ekvd@err@undefined@key,
%     \ekvd@err@no@prefix,\ekvd@err@no@prefix@msg,\ekvd@err@no@prefix@also,
%     \ekvd@err@add@val@on@noval,\ekvd@err@add@noval@on@val,
%     \ekvd@err@unsupported@arg,\ekvd@err@not@new
%   }
% The non-expandable error messages are boring, so here they are:
%    \begin{macrocode}
\protected\def\ekvd@errm#1{\errmessage{expkv-def Error: #1}}
\protected\def\ekvd@err@missing@definition
  {\ekvd@errm{Missing definition for key `\ekvd@cur'}}
\protected\def\ekvd@err@missing@definition@msg#1%
  {\ekvd@errm{Missing definition for key `\ekv@unexpanded{#1}'}}
\protected\def\ekvd@err@missing@type
  {\ekvd@errm{Missing type prefix for key `\ekvd@cur'}}
\protected\def\ekvd@err@undefined@prefix#1%
  {%
    \ekvd@errm
      {%
        Undefined prefix `\ekv@unexpanded{#1}' found while processing
        `\ekvd@cur'%
      }%
  }
\protected\def\ekvd@err@undefined@key#1%
  {%
    \ekvd@errm
      {Undefined key `\ekv@unexpanded{#1}' found while processing `\ekvd@cur'}%
  }
\protected\def\ekvd@err@undefined@noval#1%
  {%
    \ekvd@errm
      {%
        Undefined noval key `\unexpanded{#1}' found while processing
        `\ekvd@cur'%
      }%
  }
\protected\def\ekvd@err@no@prefix#1%
  {\ekvd@errm{prefix `#1' not accepted in `\ekvd@cur'}}
\protected\def\ekvd@err@no@prefix@msg#1#2%
  {\ekvd@errm{prefix `#2' not accepted in `\ekv@unexpanded{#1}'}}
\protected\def\ekvd@err@no@prefix@also#1%
  {\ekvd@errm{`\ekvd@cur' not allowed with a `#1' key}}
\protected\def\ekvd@err@add@val@on@noval
  {\ekvd@errm{`\ekvd@cur' not allowed with a NoVal key}}
\protected\def\ekvd@err@add@noval@on@val
  {\ekvd@errm{`\ekvd@cur' not allowed with a value taking key}}
\protected\def\ekvd@err@unsupported@arg\fi\@firstofone#1%
  {%
    \fi
    \ekvd@errm
      {%
        Existing key-macro has the unsupported argument string
        `\ekvd@extracted@args' for key `\ekvd@cur'%
      }%
  }
\protected\def\ekvd@err@not@new
  {\ekvd@errm{The key for `\ekvd@cur' is already defined}}
\protected\long\def\ekvd@err@misused@unknown
  {\ekvd@errm{Misuse of the unknown type found while processing `\ekvd@cur'}}
%    \end{macrocode}
% \end{macro}
%
% \begin{macro}[internal]
%   {
%     \ekvd@err@choice@invalid,\ekvd@err@choice@invalid@,\ekvd@choice@name,
%     \ekvd@unknown@choice@name
%   }
% |\ekvd@err@choice@invalid| will have to use this mechanism to throw its
% message. Also we have to retrieve the name parts of the choice in an easy way,
% so we use parentheses of catcode 8 here, which should suffice in most cases to
% allow for a correct separation.
%    \begin{macrocode}
\def\ekvd@err@choice@invalid#1%
  {%
    \ekvd@err@choice@invalid@#1%
  }
\begingroup
\catcode40=8
\catcode41=8
\@firstofone{\endgroup
\def\ekvd@choice@name#1#2#3%
  {%
    ekvd#1(#2)\detokenize{#3}%
  }
\def\ekvd@unknown@choice@name#1#2%
  {%
    ekvd:u:#1(#2)%
  }
\def\ekvd@err@choice@invalid@ ekvd#1(#2)\detokenize#3%
  {%
    \ekv@ifdefined{\ekvd@unknown@choice@name{#1}{#2}}%
      {\csname\ekvd@unknown@choice@name{#1}{#2}\endcsname{#3}}%
      {\ekvd@err{invalid choice `#3' for `#2' in set `#1'}}%
  }
}
%    \end{macrocode}
% \end{macro}
%
% \begin{macro}[internal]{\ekvd@err}
% The expandable error messages use |\ekvd@err|, which is just like |\ekv@err|
% from \expkv. It uses a runaway argument to start the error message.
%    \begin{macrocode}
\ekv@exparg{\long\def\ekvd@err#1}{\ekverr{expkv-def}{#1}}
%    \end{macrocode}
% \end{macro}
%
% Now everything that's left is to reset the category code of |@|.
%    \begin{macrocode}
\catcode`\@=\ekvd@tmp
%    \end{macrocode}
%
% \gobbledocstriptag
%</tex>
%^^A=<<
%
%^^A=<<
%
% \end{implementation}^^A=<<
%
% \clearpage
% \PrintIndex
%
\endinput
%
^^A vim: ft=tex fdm=marker fmr=>>=,=<<

\ProvidesPackage{expkv-def}%
  [\ekvdDate\space v\ekvdVersion\space a key-defining frontend for expkv]
%    \end{macrocode}
% \gobbledocstriptag
%</pkg>
%^^A=<<
%^^A the ConTeXt module >>=
% \subsection{The \ConTeXt\ module}
% \gobbledocstriptag
%<*ctx>
%    \begin{macrocode}
\writestatus{loading}{ConTeXt User Module / expkv-def}
\usemodule[expkv]
\unprotect
\input expkv-def.tex
\writestatus{loading}
  {ConTeXt User Module / expkv-def / Version \ekvdVersion\space loaded}
\protect\endinput
%    \end{macrocode}
% \gobbledocstriptag
%</ctx>
%^^A=<<
%^^A main file >>=
% \subsection{The Generic Code}
% The rest of this implementation will be the generic code.
% \gobbledocstriptag
%<*tex>
%
% Load \expkv\ if the package didn't already do so -- since \expkv\ has
% safeguards against being loaded twice this does no harm and the overhead
% isn't that big. Also we reuse some of the internals of \expkv\ to save us from
% retyping them.
%    \begin{macrocode}
\input expkv
%    \end{macrocode}
%
% We make sure that \file{expkv-def.tex} is only input once:
%    \begin{macrocode}
\expandafter\ifx\csname ekvdVersion\endcsname\relax
\else
  \expandafter\endinput
\fi
%    \end{macrocode}
%
% \begin{macro}{\ekvdVersion,\ekvdDate}
% We're on our first input, so lets store the version and date in a macro.
%    \begin{macrocode}
\def\ekvdVersion{0.8c}
\def\ekvdDate{2021-09-20}
%    \end{macrocode}
% \end{macro}
%
% If the \LaTeX\ format is loaded we want to be a good file and report back who
% we are, for this the package will have defined |\ekvd@tmp| to use
% |\ProvidesFile|, else this will expand to a |\relax| and do no harm.
%    \begin{macrocode}
\csname ekvd@tmp\endcsname
%    \end{macrocode}
%
% Store the category code of |@| to later be able to reset it and change it to
% 11 for now.
%    \begin{macrocode}
\expandafter\chardef\csname ekvd@tmp\endcsname=\catcode`\@
\catcode`\@=11
%    \end{macrocode}
% |\ekvd@tmp| will be reused later to handle expansion during the key
% defining. But we don't need it to ever store information long-term after
% \expkvd\ was initialized.
%
% \begin{macro}[internal]{\ekvd@ifprimitive}
%    \begin{macrocode}
\protected\long\def\ekvd@ifprimitive#1%
  {%
    \begingroup
      \edef\ekvd@tmpa{\string #1}%
      \edef\ekvd@tmpb{\meaning#1}%
      \expandafter
    \endgroup
    \ifx\ekvd@tmpa\ekvd@tmpb
      \ekv@fi@firstoftwo
    \fi
    \@secondoftwo
  }
%    \end{macrocode}
% \end{macro}
%
% \begin{macro}[internal]
%   {\ekvd@long,\ekvd@prot,\ekvd@clear@prefixes,\ekvd@ifalso}
% \expkvd\ will use |\ekvd@long|, |\ekvd@prot|, and |\ekvd@ifalso| to store
% whether a key should be defined as |\long| or |\protected| or adds an action
% to an existing key, and we have to clear them for every new key. By default
% |long| and |protected| will just be empty, |ifalso| will be
% |\@secondoftwo|, and |ifnew| will just use its third argument.
%    \begin{macrocode}
\protected\def\ekvd@clear@prefixes
  {%
    \let\ekvd@long\ekv@empty
    \let\ekvd@prot\ekv@empty
    \let\ekvd@ifalso\@secondoftwo
    \long\def\ekvd@ifnew##1##2##3{##3}%
  }
\ekvd@clear@prefixes
%    \end{macrocode}
% \end{macro}
%
% \begin{macro}{\ekvdefinekeys}
% This is the one front-facing macro which provides the interface to define
% keys. It's using |\ekvparse| to handle the \kv\ list, the interpretation will
% be done by |\ekvd@noarg| and |\ekvd@|. The \set\ for which the keys should be
% defined is stored in |\ekvd@set|.
%    \begin{macrocode}
\protected\def\ekvdefinekeys#1%
  {%
    \def\ekvd@set{#1}%
    \ekvparse\ekvd@noarg\ekvd@arg
  }
%    \end{macrocode}
% \end{macro}
%
% \begin{macro}[internal]{\ekvd@noarg,\ekvd@arg,\ekvd@handle}
% |\ekvd@noarg| and |\ekvd@arg| store whether there was a value in the \kv\
% pair. |\ekvd@handle| has to test whether there is a space inside the key and
% if so calls the prefix grabbing routine, else we throw an error and ignore the
% key.
%    \begin{macrocode}
\protected\long\def\ekvd@noarg#1%
  {%
    \let\ekvd@ifnoarg\@firstoftwo
    \expandafter\ekvd@handle\detokenize{#1}\ekvd@stop{}%
  }
\protected\long\def\ekvd@arg#1%
  {%
    \let\ekvd@ifnoarg\@secondoftwo
    \expandafter\ekvd@handle\detokenize{#1}\ekvd@stop
  }
\protected\long\def\ekvd@handle#1\ekvd@stop#2%
  {%
    \ekvd@clear@prefixes
    \def\ekvd@cur{#1}%
    \ekvd@ifspace{#1}%
      {\ekvd@prefix\ekv@mark#1\ekv@stop{#2}}%
      \ekvd@err@missing@type
  }
%    \end{macrocode}
% \end{macro}
%
% \begin{macro}[internal]{\ekvd@prefix,\ekvd@prefix@}
% \expkvd\ separates prefixes into two groups, the first being prefixes in the
% \TeX\ sense (|long| and |protected|) which use |@p@| in their name, the other
% being key-types (|code|, |int|, \emph{etc.}) which use |@t@| instead.
% |\ekvd@prefix| splits at the first space and checks whether its a |@p@|
% or |@t@| type prefix. If it is neither throw an error and gobble the
% definition (the value).
%    \begin{macrocode}
\protected\def\ekvd@prefix#1 {\ekv@strip{#1}\ekvd@prefix@\ekv@mark}
\protected\def\ekvd@prefix@#1#2\ekv@stop
  {%
    \ekv@ifdefined{ekvd@t@#1}%
      {\ekv@strip{#2}{\csname ekvd@t@#1\endcsname}}%
      {%
        \ekv@ifdefined{ekvd@p@#1}%
          {\csname ekvd@p@#1\endcsname\ekvd@prefix@after@p{#2}}%
          {\ekvd@err@undefined@prefix{#1}\@gobble}%
      }%
  }
%    \end{macrocode}
% \end{macro}
%
% \begin{macro}[internal]{\ekvd@prefix@after@p}
% The |@p@| type prefixes are all just modifying a following |@t@| type, so they
% will need to search for another prefix. This is true for all of them, so we
% use a macro to handle this. It'll throw an error if there is no other prefix.
%    \begin{macrocode}
\protected\def\ekvd@prefix@after@p#1%
  {%
    \ekvd@ifspace{#1}%
      {\ekvd@prefix#1\ekv@stop}%
      {\ekvd@err@missing@type\@gobble}%
  }
%    \end{macrocode}
% \end{macro}
%
% \begin{macro}[internal]
%   {\ekvd@p@long,\ekvd@p@protected,\ekvd@p@protect,\ekvd@p@also,\ekvd@p@new}
% Define the |@p@| type prefixes, they all just store some information in a
% temporary macro.
%    \begin{macrocode}
\protected\def\ekvd@p@long{\let\ekvd@long\long}
\protected\def\ekvd@p@protected{\let\ekvd@prot\protected}
\let\ekvd@p@protect\ekvd@p@protected
\protected\def\ekvd@p@also{\let\ekvd@ifalso\@firstoftwo}
\protected\def\ekvd@p@new{\let\ekvd@ifnew\ekvd@assert@new}
%    \end{macrocode}
% \end{macro}
%
% \subsubsection{Key Types}
%
% \begin{macro}[internal]{\ekvd@type@set,\ekvd@t@set}
% The |set| type is quite straight forward, just define a |NoVal| key to call
% |\ekvchangeset|.
%    \begin{macrocode}
\protected\def\ekvd@type@set#1#2%
  {%
    \ekvd@assert@not@long
    \ekvd@assert@not@protected
    \ekvd@ifnew{NoVal}{#1}%
      {%
        \ekv@ifempty{#2}%
          {\ekvd@err@missing@definition}%
          {%
            \ekvd@ifalso
              {%
                \ekv@expargtwice{\ekvd@add@noval{#1}}%
                  {\ekvchangeset{#2}}%
                  \ekvd@assert@not@protected@also
              }%
              {\ekv@expargtwice{\ekvdefNoVal\ekvd@set{#1}}{\ekvchangeset{#2}}}%
          }%
      }%
  }
\protected\def\ekvd@t@set#1#2%
  {%
    \ekvd@ifnoarg
      {\ekvd@type@set{#1}{#1}}%
      {\ekvd@type@set{#1}{#2}}%
  }
%    \end{macrocode}
% \end{macro}
%
% \begin{macro}[internal]{\ekvd@type@noval,\ekvd@t@noval,\ekvd@t@enoval}
% Another pretty simple type, |noval| just needs to assert that there is a
% definition and that |long| wasn't specified.
% There are types where the difference in the variants is so small, that we
% define a common handler for them, those common handlers are named with
% |@type@|. |noval| and |enoval| are so similar that we can use such a |@type@|
% macro, even if we could've done |noval| in a slightly faster way without it.
%    \begin{macrocode}
\protected\long\def\ekvd@type@noval#1#2#3%
  {%
    \ekvd@ifnew{NoVal}{#2}%
      {%
        \ekvd@assert@arg
          {%
            \ekvd@assert@not@long
            \ekvd@prot#1\ekvd@tmp{#3}%
            \ekvd@ifalso
              {\ekv@exparg{\ekvd@add@noval{#2}}\ekvd@tmp{}}%
              {\ekvletNoVal\ekvd@set{#2}\ekvd@tmp}%
          }%
      }%
  }
\protected\def\ekvd@t@noval{\ekvd@type@noval\def}
\protected\def\ekvd@t@enoval{\ekvd@type@noval\edef}
%    \end{macrocode}
% \end{macro}
%
% \begin{macro}[internal]{\ekvd@type@code,\ekvd@t@code,\ekvd@t@ecode}
% |code| is simple as well, |ecode| has to use |\edef| on a temporary macro,
% since \expkv\ doesn't provide an |\ekvedef|.
%    \begin{macrocode}
\protected\long\def\ekvd@type@code#1#2#3%
  {%
    \ekvd@ifnew{}{#2}%
      {%
        \ekvd@assert@arg
          {%
            \ekvd@prot\ekvd@long#1\ekvd@tmp##1{#3}%
            \ekvd@ifalso
              {\ekv@exparg{\ekvd@add@val{#2}}{\ekvd@tmp{##1}}{}}%
              {\ekvlet\ekvd@set{#2}\ekvd@tmp}%
          }%
      }%
  }
\protected\def\ekvd@t@code{\ekvd@type@code\def}
\protected\def\ekvd@t@ecode{\ekvd@type@code\edef}
%    \end{macrocode}
% \end{macro}
%
% \begin{macro}[internal]
%   {
%     \ekvd@type@default,\ekvd@t@default,\ekvd@t@qdefault,\ekvd@t@odefault,
%     \ekvd@t@fdefault
%   }
% |\ekvd@type@default| asserts there was an argument, also the key for which one
% wants to set a default has to be already defined (this is not so important for
% |default|, but |qdefault| requires is). If everything is good, |\edef| a
% temporary macro that expands |\ekvd@set| and the |\csname| for the key, and in
% the case of |qdefault| does the first expansion step of the key-macro.
%    \begin{macrocode}
\protected\long\def\ekvd@type@default#1#2#3#4%
  {%
    \ekvd@assert@arg
      {%
        \ekvifdefined\ekvd@set{#3}%
          {%
            \ekvd@assert@not@new
            \ekvd@assert@not@long
            \ekvd@prot\edef\ekvd@tmp
              {%
                \ekv@unexpanded\expandafter#1%
                  {#2\csname\ekv@name\ekvd@set{#3}\endcsname{#4}}%
              }%
            \ekvd@ifalso
              {\ekv@exparg{\ekvd@add@noval{#3}}\ekvd@tmp{}}%
              {\ekvletNoVal\ekvd@set{#3}\ekvd@tmp}%
          }%
          {\ekvd@err@undefined@key{#3}}%
      }%
  }
\protected\def\ekvd@t@default{\ekvd@type@default{}{}}
\protected\def\ekvd@t@qdefault{\ekvd@type@default{\expandafter\expandafter}{}}
\let\ekvd@t@odefault\ekvd@t@qdefault
\protected\def\ekvd@t@fdefault{\ekvd@type@default{}{\romannumeral`\^^@}}
%    \end{macrocode}
% \end{macro}
%
% \begin{macro}[internal]{\ekvd@t@edefault}
% |edefault| is too different from |default| and |qdefault| to reuse the
% |@type@| macro, as it doesn't need |\unexpanded| inside of |\edef|.
%    \begin{macrocode}
\protected\long\def\ekvd@t@edefault#1#2%
  {%
    \ekvd@assert@arg
      {%
        \ekvifdefined\ekvd@set{#1}%
          {%
            \ekvd@assert@not@new
            \ekvd@assert@not@long
            \ekvd@prot\edef\ekvd@tmp
              {\csname\ekv@name\ekvd@set{#1}\endcsname{#2}}%
            \ekvd@ifalso
              {\ekv@exparg{\ekvd@add@noval{#1}}\ekvd@tmp{}}%
              {\ekvletNoVal\ekvd@set{#1}\ekvd@tmp}%
          }%
          {\ekvd@err@undefined@key{#1}}%
      }%
  }
%    \end{macrocode}
% \end{macro}
%
% \begin{macro}[internal]
%   {\ekvd@t@initial,\ekvd@t@oinitial,\ekvd@t@finitial,\ekvd@t@einitial}
%    \begin{macrocode}
\long\def\ekvd@type@initial#1#2#3#4%
  {%
    \ekvd@assert@not@new
    \ekvd@assert@not@also
    \ekvd@assert@not@long
    \ekvd@assert@not@protected
    \ekvd@ifnoarg
      {%
        \ekvifdefinedNoVal\ekvd@set{#3}%
          {\csname\ekv@name\ekvd@set{#3}N\endcsname}%
          {\ekvd@err@undefined@noval{#3}}%
      }%
      {%
        \ekvifdefined\ekvd@set{#3}%
          {%
            #1{#2#4}%
            \csname\ekv@name\ekvd@set{#3}\expandafter\endcsname\expandafter
              {\ekvd@tmp}%
          }%
          {\ekvd@err@undefined@key{#3}}%
      }%
  }
\def\ekvd@t@initial{\ekvd@type@initial{\def\ekvd@tmp}{}}
\def\ekvd@t@oinitial{\ekvd@type@initial{\ekv@exparg{\def\ekvd@tmp}}{}}
\def\ekvd@t@einitial{\ekvd@type@initial{\edef\ekvd@tmp}{}}
\def\ekvd@t@finitial
  {\ekvd@type@initial{\ekv@exparg{\def\ekvd@tmp}}{\romannumeral`\^^@}}
%    \end{macrocode}
% \end{macro}
%
% \begin{macro}[internal]
%   {
%     \ekvd@type@bool,\ekvd@t@bool,\ekvd@t@gbool,\ekvd@t@boolTF,\ekvd@t@gboolTF,
%     \ekvd@t@invbool,\ekvd@t@ginvbool,\ekvd@t@invboolTF,\ekvd@t@ginvboolTF,
%   }
% The boolean types are a quicker version of a |choice| that accept |true| and
% |false|, and set up the |NoVal| action to be identical to \texttt{\key=true}.
% The |true| and |false| actions are always just |\let|ting the macro in |#7| to
% some other macro (\emph{e.g.}, \cs[no-index]{iftrue}).
%    \begin{macrocode}
\protected\def\ekvd@type@bool#1#2#3#4#5%
  {%
    \ekvd@ifnew{}{#4}%
      {%
        \ekvd@ifnew{NoVal}{#4}%
          {%
            \ekvd@assert@filledarg{#5}%
              {%
                \ekvd@newlet#5#3%
                \ekvd@type@choice{#4}%
                \protected\ekvdefNoVal\ekvd@set{#4}{#1\let#5#2}%
                \protected\expandafter\def
                  \csname\ekvd@choice@name\ekvd@set{#4}{true}\endcsname
                  {#1\let#5#2}%
                \protected\expandafter\def
                  \csname\ekvd@choice@name\ekvd@set{#4}{false}\endcsname
                  {#1\let#5#3}%
              }%
          }%
      }%
  }
\protected\def\ekvd@t@bool{\ekvd@type@bool{}\iftrue\iffalse}
\protected\def\ekvd@t@gbool{\ekvd@type@bool\global\iftrue\iffalse}
\protected\def\ekvd@t@boolTF{\ekvd@type@bool{}\@firstoftwo\@secondoftwo}
\protected\def\ekvd@t@gboolTF{\ekvd@type@bool\global\@firstoftwo\@secondoftwo}
\protected\def\ekvd@t@invbool{\ekvd@type@bool{}\iffalse\iftrue}
\protected\def\ekvd@t@ginvbool{\ekvd@type@bool\global\iffalse\iftrue}
\protected\def\ekvd@t@invboolTF{\ekvd@type@bool{}\@secondoftwo\@firstoftwo}
\protected\def\ekvd@t@ginvboolTF
  {\ekvd@type@bool\global\@secondoftwo\@firstoftwo}
%    \end{macrocode}
% \end{macro}
%
% \begin{macro}[internal]
%   {
%     \ekvd@type@boolpair,
%     \ekvd@t@boolpair,\ekvd@t@gboolpair,\ekvd@t@boolpairTF,\ekvd@t@gboolpairTF
%   }
% The boolean pair types are essentially the same as the boolean types, but set
% two macros instead of one.
%    \begin{macrocode}
\protected\def\ekvd@type@boolpair#1#2#3#4#5#6%
  {%
    \ekvd@ifnew{}{#4}%
      {%
        \ekvd@ifnew{NoVal}{#4}%
          {%
            \ekvd@newlet#5#3%
            \ekvd@newlet#6#2%
            \ekvd@type@choice{#4}%
            \protected\ekvdefNoVal\ekvd@set{#4}{#1\let#5#2#1\let#6#3}%
            \protected\expandafter\def
              \csname\ekvd@choice@name\ekvd@set{#4}{true}\endcsname
              {#1\let#5#2#1\let#6#3}%
            \protected\expandafter\def
              \csname\ekvd@choice@name\ekvd@set{#4}{false}\endcsname
              {#1\let#5#3#1\let#6#2}%
          }%
      }%
  }
\protected\def\ekvd@t@boolpair#1#2%
  {\ekvd@assert@twoargs{#2}{\ekvd@type@boolpair{}\iftrue\iffalse{#1}#2}}
\protected\def\ekvd@t@gboolpair#1#2%
  {\ekvd@assert@twoargs{#2}{\ekvd@type@boolpair\global\iftrue\iffalse{#1}#2}}
\protected\def\ekvd@t@boolpairTF#1#2%
  {%
    \ekvd@assert@twoargs{#2}%
      {\ekvd@type@boolpair{}\@firstoftwo\@secondoftwo{#1}#2}%
  }
\protected\def\ekvd@t@gboolpairTF#1#2%
  {%
    \ekvd@assert@twoargs{#2}%
      {\ekvd@type@boolpair\global\@firstoftwo\@secondoftwo{#1}#2}%
  }
%    \end{macrocode}
% \end{macro}
%
% \begin{macro}[internal]
%   {
%     \ekvd@type@data,\ekvd@t@data,\ekvd@t@gdata,\ekvd@t@dataT,\ekvd@t@gdataT
%   }
%    \begin{macrocode}
\protected\def\ekvd@type@data#1#2#3#4#5#6%
  {%
    \ekvd@ifnew{}{#5}%
      {%
        \ekvd@assert@filledarg{#6}%
          {%
            \ekvd@newlet#6#1%
            \ekvd@ifalso
              {%
                \let\ekvd@prot\protected
                \ekvd@add@val{#5}{\long#2#6####1#3{####1{#4}}}{}%
              }%
              {%
                \protected\ekvd@long\ekvdef\ekvd@set{#5}%
                  {\long#2#6####1#3{####1{#4}}}%
              }%
          }%
      }%
  }
\protected\def\ekvd@t@data
  {\ekvd@type@data\@secondoftwo\edef{####2}{\ekv@unexpanded{##1}}}
\protected\def\ekvd@t@edata{\ekvd@type@data\@secondoftwo\edef{####2}{##1}}
\protected\def\ekvd@t@gdata
  {\ekvd@type@data\@secondoftwo\xdef{####2}{\ekv@unexpanded{##1}}}
\protected\def\ekvd@t@xdata{\ekvd@type@data\@secondoftwo\xdef{####2}{##1}}
\protected\def\ekvd@t@dataT
  {\ekvd@type@data\@gobble\edef{}{\ekv@unexpanded{##1}}}
\protected\def\ekvd@t@edataT{\ekvd@type@data\@gobble\edef{}{##1}}
\protected\def\ekvd@t@gdataT
  {\ekvd@type@data\@gobble\xdef{}{\ekv@unexpanded{##1}}}
\protected\def\ekvd@t@xdataT{\ekvd@type@data\@gobble\xdef{}{##1}}
%    \end{macrocode}
% \end{macro}
%
% \begin{macro}[internal]{\ekvd@type@box,\ekvd@t@box,\ekvd@t@gbox}
% Set up our boxes. Though we're a generic package we want to be colour safe, so
% we put an additional grouping level inside the box contents, for the case that
% someone uses \pkg{color}. |\ekvd@newreg| is a small wrapper which tests
% whether the first argument is defined and if not does
% |\csname new#2\endcsname#1|.
%    \begin{macrocode}
\protected\def\ekvd@type@box#1#2#3%
  {%
    \ekvd@ifnew{}{#2}%
      {%
        \ekvd@assert@filledarg{#3}%
          {%
            \ekvd@newreg#3{box}%
            \ekvd@ifalso
              {%
                \let\ekvd@prot\protected
                \ekvd@add@val{#2}{#1\setbox#3=\hbox{\begingroup##1\endgroup}}{}%
              }%
              {%
                \protected\ekvd@long\ekvdef\ekvd@set{#2}%
                  {#1\setbox#3=\hbox{\begingroup##1\endgroup}}%
              }%
          }%
      }%
  }
\protected\def\ekvd@t@box{\ekvd@type@box{}}
\protected\def\ekvd@t@gbox{\ekvd@type@box\global}
%    \end{macrocode}
% \end{macro}
%
% \begin{macro}[internal]{\ekvd@type@toks,\ekvd@t@toks,\ekvd@t@gtoks}
% Similar to |box|, but set the |toks|.
%    \begin{macrocode}
\protected\def\ekvd@type@toks#1#2#3%
  {%
    \ekvd@ifnew{}{#2}%
      {%
        \ekvd@assert@filledarg{#3}%
          {%
            \ekvd@newreg#3{toks}%
            \ekvd@ifalso
              {%
                \let\ekvd@prot\protected
                \ekvd@add@val{#2}{#1#3={##1}}{}%
              }%
              {\protected\ekvd@long\ekvdef\ekvd@set{#2}{#1#3={##1}}}%
          }%
      }%
  }
\protected\def\ekvd@t@toks{\ekvd@type@toks{}}
\protected\def\ekvd@t@gtoks{\ekvd@type@toks\global}
%    \end{macrocode}
% \end{macro}
%
% \begin{macro}[internal]{\ekvd@type@preapptoks,\ekvd@t@apptoks,\ekvd@t@gapptoks}
% Just like |toks|, but expand the current contents of the |toks| register to
% append the new contents.
%    \begin{macrocode}
\ekvd@ifprimitive\toksapp
  {%
    \protected\def\ekvd@type@preapptoks#1#2#3%
      {%
        \ekvd@ifnew{}{#2}%
          {%
            \ekvd@assert@filledarg{#3}%
              {%
                \ekvd@newreg#3{toks}%
                \ekvd@ifalso
                  {%
                    \let\ekvd@prot\protected
                    \ekvd@add@val{#2}{#1#3{##1}}{}%
                  }%
                  {\protected\ekvd@long\ekvdef\ekvd@set{#2}{#1#3{##1}}}%
              }%
          }%
      }
    \protected\def\ekvd@t@apptoks{\ekvd@type@preapptoks\toksapp}
    \protected\def\ekvd@t@gapptoks{\ekvd@type@preapptoks\gtoksapp}
    \protected\def\ekvd@t@pretoks{\ekvd@type@preapptoks\tokspre}
    \protected\def\ekvd@t@gpretoks{\ekvd@type@preapptoks\gtokspre}
  }
  {%
    \protected\def\ekvd@type@apptoks#1#2#3%
      {%
        \ekvd@ifnew{}{#2}%
          {%
            \ekvd@assert@filledarg{#3}%
              {%
                \ekvd@newreg#3{toks}%
                \ekvd@ifalso
                  {%
                    \let\ekvd@prot\protected
                    \ekvd@add@val{#2}{#1#3=\expandafter{\the#3##1}}{}%
                  }%
                  {%
                    \protected\ekvd@long\ekvdef\ekvd@set{#2}%
                      {#1#3=\expandafter{\the#3##1}}%
                  }%
              }%
          }%
      }
    \protected\def\ekvd@t@apptoks{\ekvd@type@apptoks{}}
    \protected\def\ekvd@t@gapptoks{\ekvd@type@apptoks\global}
    \newtoks\ekvd@toks
    \protected\def\ekvd@type@pretoks#1#2#3%
      {%
        \ekvd@ifnew{}{#2}%
          {%
            \ekvd@assert@filledarg{#3}%
              {%
                \ekvd@newreg#3{toks}%
                \ekvd@ifalso
                  {%
                    \let\ekvd@prot\protected
                    \ekvd@add@val{#2}%
                      {%
                        \ekvd@toks={##1}%
                        #1#3=\expandafter{\the\expandafter\ekvd@toks\the#3}%
                      }%
                      {}%
                  }%
                  {%
                    \protected\ekvd@long\ekvdef\ekvd@set{#2}%
                      {%
                        \ekvd@toks={##1}%
                        #1#3=\expandafter{\the\expandafter\ekvd@toks\the#3}%
                      }%
                  }%
              }%
          }%
      }
    \protected\def\ekvd@t@pretoks{\ekvd@type@pretoks{}}
    \protected\def\ekvd@t@gpretoks{\ekvd@type@pretoks\global}
  }
%    \end{macrocode}
% \end{macro}
%
% \begin{macro}[internal]
%   {
%     \ekvd@type@reg,\ekvd@t@int,\ekvd@t@eint,\ekvd@t@gint,\ekvd@t@xint,
%     \ekvd@t@dimen,\ekvd@t@edimen,\ekvd@t@gdimen,\ekvd@t@xdimen,
%     \ekvd@t@skip,\ekvd@t@eskip,\ekvd@t@gskip,\ekvd@t@xskip
%   }
% The |\ekvd@type@reg| can handle all the types for which the assignment will
% just be \texttt{\meta{register}=\meta{value}}.
%    \begin{macrocode}
\protected\def\ekvd@type@reg#1#2#3#4#5#6%
  {%
    \ekvd@ifnew{}{#5}%
      {%
        \ekvd@assert@filledarg{#6}%
          {%
            \ekvd@newreg#6{#1}%
            \ekvd@ifalso
              {%
                \let\evkd@prot\protected
                \ekvd@add@val{#5}{#2#6=#3##1#4\relax}{}%
              }%
              {\protected\ekvd@long\ekvdef\ekvd@set{#5}{#2#6=#3##1#4\relax}}%
          }%
      }%
  }
\protected\def\ekvd@t@int{\ekvd@type@reg{count}{}{}{}}
\protected\def\ekvd@t@eint{\ekvd@type@reg{count}{}\numexpr\relax}
\protected\def\ekvd@t@gint{\ekvd@type@reg{count}\global{}{}}
\protected\def\ekvd@t@xint{\ekvd@type@reg{count}\global\numexpr\relax}
\protected\def\ekvd@t@dimen{\ekvd@type@reg{dimen}{}{}{}}
\protected\def\ekvd@t@edimen{\ekvd@type@reg{dimen}{}\dimexpr\relax}
\protected\def\ekvd@t@gdimen{\ekvd@type@reg{dimen}\global{}{}}
\protected\def\ekvd@t@xdimen{\ekvd@type@reg{dimen}\global\dimexpr\relax}
\protected\def\ekvd@t@skip{\ekvd@type@reg{skip}{}{}{}}
\protected\def\ekvd@t@eskip{\ekvd@type@reg{skip}{}\glueexpr\relax}
\protected\def\ekvd@t@gskip{\ekvd@type@reg{skip}\global{}{}}
\protected\def\ekvd@t@xskip{\ekvd@type@reg{skip}\global\glueexpr\relax}
%    \end{macrocode}
% \end{macro}
%
% \begin{macro}[internal]{\ekvd@type@store,\ekvd@t@store,\ekvd@t@gstore}
% The none-expanding |store| types use an |\edef| or |\xdef| and |\unexpanded|
% to be able to also store |#| easily.
%    \begin{macrocode}
\protected\def\ekvd@type@store#1#2#3#4%
  {%
    \ekvd@ifnew{}{#3}%
      {%
        \ekvd@assert@filledarg{#4}%
          {%
            \ekvd@newlet#4\ekv@empty
            \ekvd@ifalso
              {%
                \let\ekvd@prot\protected
                \ekvd@add@val{#3}{#1#4{#2}}{}%
              }%
              {\protected\ekvd@long\ekvdef\ekvd@set{#3}{#1#4{#2}}}%
          }%
      }%
  }
\protected\def\ekvd@t@store{\ekvd@type@store\edef{\ekv@unexpanded{##1}}}
\protected\def\ekvd@t@gstore{\ekvd@type@store\xdef{\ekv@unexpanded{##1}}}
\protected\def\ekvd@t@estore{\ekvd@type@store\edef{##1}}
\protected\def\ekvd@t@xstore{\ekvd@type@store\xdef{##1}}
%    \end{macrocode}
% \end{macro}
%
% \begin{macro}[internal]
%   {
%     \ekvd@type@meta,\ekvd@type@meta@a,\ekvd@type@meta@b,\ekvd@type@meta@c,
%     \ekvd@t@meta,\ekvd@t@nmeta
%   }
% |meta| sets up things such that another instance of |\ekvset| will be run on
% the argument, with the same \set.
%    \begin{macrocode}
\protected\long\def\ekvd@type@meta#1#2#3#4#5#6#7%
  {%
    \ekvd@ifnew{#1}{#6}%
      {%
        \ekvd@assert@filledarg{#7}%
          {%
            \edef\ekvd@tmp{\ekvd@set}%
            \expandafter\ekvd@type@meta@a\expandafter{\ekvd@tmp}{#7}{#2}%
            \ekvd@ifalso
              {\ekv@exparg{#3{#6}}{\ekvd@tmp#4}{#5}}%
              {\csname ekvlet#1\endcsname\ekvd@set{#6}\ekvd@tmp}%
          }%
      }%
  }
\protected\long\def\ekvd@type@meta@a#1#2%
  {%
    \expandafter\ekvd@type@meta@b\expandafter{\ekvset{#1}{#2}}%
  }
\protected\def\ekvd@type@meta@b
  {%
    \expandafter\ekvd@type@meta@c\expandafter
  }
\protected\long\def\ekvd@type@meta@c#1#2%
  {%
    \ekvd@prot\ekvd@long\def\ekvd@tmp#2{#1}%
  }
\protected\def\ekvd@t@meta{\ekvd@type@meta{}{##1}\ekvd@add@val{{##1}}{}}
\protected\def\ekvd@t@nmeta
  {%
    \ekvd@assert@not@long
    \ekvd@type@meta{NoVal}{}\ekvd@add@noval{}\ekvd@assert@not@long@also
  }
%    \end{macrocode}
% \end{macro}
%
% \begin{macro}[internal]
%   {\ekvd@type@smeta,\ekvd@type@smeta@,\ekvd@t@smeta,\ekvd@t@snmeta}
% |smeta| is pretty similar to |meta|, but needs two arguments inside of \val,
% such that the first is the \set\ for which the sub-|\ekvset| and the second is
% the \kv\ list.
%    \begin{macrocode}
\protected\long\def\ekvd@type@smeta#1#2#3#4#5#6#7%
  {%
    \ekvd@ifnew{#1}{#6}%
      {%
        \ekvd@assert@twoargs{#7}%
          {%
            \ekvd@type@meta@a#7{#2}%
            \ekvd@ifalso
              {\ekv@exparg{#3{#6}}{\ekvd@tmp#4}{#5}}%
              {\csname ekvlet#1\endcsname\ekvd@set{#6}\ekvd@tmp}%
          }%
      }%
  }
\protected\def\ekvd@t@smeta{\ekvd@type@smeta{}{##1}\ekvd@add@val{{##1}}{}}
\protected\def\ekvd@t@snmeta
  {%
    \ekvd@assert@not@long
    \ekvd@type@smeta{NoVal}{}\ekvd@add@noval{}\ekvd@assert@not@long@also
  }
%    \end{macrocode}
% \end{macro}
%
% \begin{macro}[internal]
%   {
%     \ekvd@type@choice,
%     \ekvd@populate@choice,\ekvd@populate@choice@,\ekvd@populate@choice@noarg,
%     \ekvd@choice@prefix,\ekvd@choice@prefix@,\ekvd@choice@prefix@done,
%     \ekvd@choice@p@protected,\ekvd@choice@p@protect,
%     \ekvd@choice@p@long,\ekvd@choice@p@long@,
%     \ekvd@t@choice
%   }
% The |choice| type is by far the most complex type, as we have to run a
% sub-parser on the choice-definition list, which should support the |@p@| type
% prefixes as well (but |long| will always throw an error, as they are not
% allowed to be long). |\ekvd@type@choice| will just define the choice-key, the
% handling of the choices definition will be done by |\ekvd@populate@choice|.
%    \begin{macrocode}
\protected\def\ekvd@type@choice#1%
  {%
    \ekvd@assert@not@long
    \ekv@exparg{\ekvd@prot\def\ekvd@tmp##1}%
      {%
        \expandafter\ekvd@h@choice\expandafter
          {\ekvd@choice@name\ekvd@set{#1}{##1}}%
      }%
    \ekvd@ifalso
      {%
        \ekvd@assert@val{#1}%
          {%
            \ekvd@if@not@already@choice{#1}%
              {%
                \ekv@exparg
                  {%
                    \expandafter\ekvd@add@aux
                      \csname\ekv@name\ekvd@set{#1}\endcsname{{##1}}{#1}%
                  }%
                  {\ekvd@tmp{##1}}%
                  {\ekvd@long\ekvdef}\ekvd@assert@not@long@also
              }%
          }%
      }%
      {\ekvlet\ekvd@set{#1}\ekvd@tmp}%
  }
%    \end{macrocode}
% |\ekvd@populate@choice| just uses |\ekvparse| and then gives control to
% |\ekvd@populate@choice@noarg|, which throws an error, and
% |\ekvd@populate@choice@|.
%    \begin{macrocode}
\protected\def\ekvd@populate@choice
  {%
    \ekvparse\ekvd@populate@choice@noarg\ekvd@populate@choice@
  }
\protected\long\def\ekvd@populate@choice@noarg#1%
  {%
    \expandafter\ekvd@err@missing@definition@msg\expandafter{\ekvd@cur : #1}%
  }
%    \end{macrocode}
% |\ekvd@populate@choice@| runs the prefix-test, if there is none we can
% directly define the choice, for that |\ekvd@set@choice| will expand to the
% current choice-key's name, which will have been defined by |\ekvd@t@choice|.
% If there is a prefix run the prefix grabbing routine, which was altered for
% |@type@choice|.
%    \begin{macrocode}
\protected\long\def\ekvd@populate@choice@#1#2%
  {%
    \ekvd@clear@prefixes
    \ekvd@ifspace{#1}%
      {\ekvd@choice@prefix{\ekv@mark#1}\ekv@mark#1\ekv@stop}%
      {%
        \expandafter\edef
          \csname\ekvd@choice@name\ekvd@set\ekvd@set@choice{#1}\endcsname
      }%
      {\unexpanded{#2}}%
  }
\protected\def\ekvd@choice@prefix#1#2
  {%
    \ekv@strip{#2}{\ekvd@choice@prefix@{#1}}\ekv@mark
  }
\protected\def\ekvd@choice@prefix@#1#2#3\ekv@stop
  {%
    \ekv@ifdefined{ekvd@choice@p@#2}%
      {%
        \csname ekvd@choice@p@#2\endcsname
        \ekvd@ifspace{#3}%
          {\ekvd@choice@prefix{#3}#3\ekv@stop}%
          {\ekvd@choice@prefix@done{#3}}%
      }%
      {\ekvd@choice@prefix@done{#1}}%
  }
\protected\def\ekvd@choice@prefix@done#1%
  {%
    \ekvd@prot\expandafter\edef
      \csname
        \ekv@strip{#1}{\ekvd@choice@name\ekvd@set\ekvd@set@choice}%
      \endcsname
  }
\protected\def\ekvd@choice@p@protected{\let\ekvd@prot\protected}
\let\ekvd@choice@p@protect\ekvd@choice@p@protected
\protected\def\ekvd@choice@invalid@p#1\ekvd@ifspace#2%
  {%
    \expandafter\ekvd@choice@invalid@p@\expandafter{\ekv@gobble@mark#2}{#1}%
    \ekvd@ifspace{#2}%
  }
\protected\def\ekvd@choice@invalid@p@#1#2%
  {%
    \expandafter\ekvd@err@no@prefix@msg\expandafter{\ekvd@cur : #2 #1}{#2}%
  }
\protected\def\ekvd@choice@p@long{\ekvd@choice@invalid@p{long}}%
\protected\def\ekvd@choice@p@also{\ekvd@choice@invalid@p{also}}%
\protected\def\ekvd@choice@p@new{\ekvd@choice@invalid@p{new}}%
%    \end{macrocode}
% Finally we're able to set up the |@t@choice| macro, which has to store the
% current choice-key's name, define the key, and parse the available choices.
%    \begin{macrocode}
\protected\long\def\ekvd@t@choice#1#2%
  {%
    \ekvd@ifnew{}{#1}%
      {%
        \ekvd@assert@arg
          {%
            \ekvd@type@choice{#1}%
            \def\ekvd@set@choice{#1}%
            \ekvd@populate@choice{#2}%
          }%
      }%
  }
%    \end{macrocode}
% \end{macro}
%
% \begin{macro}[internal]{\ekvd@t@choice-store,\ekvd@t@choice-enum}
%   These two types define a special kind of |choice| key and are quite similar,
%   the only difference is what the different choices do (hence they use a
%   shared initialisation which differs in the chosen |populate| step).
%    \begin{macrocode}
\protected\long\expandafter\def\csname ekvd@t@choice-store\endcsname
  {\ekvd@type@choicespecial\ekvd@populate@choicestore}
\protected\long\expandafter\def\csname ekvd@t@choice-enum\endcsname
  {\ekvd@type@choicespecial\ekvd@populate@choiceenum}
%    \end{macrocode}
% \begin{macro}[internal]{\ekvd@type@choicespecial}
%   Initialise similar to a |choice| key. The difference is that we require two
%   arguments (which we assert), a macro to store things in, and a |csv|-list
%   containing the allowed values. |#1| is the |populate| macro according to the
%   type used.
%    \begin{macrocode}
\protected\long\def\ekvd@type@choicespecial#1#2#3%
  {%
    \ekvd@ifnew{}{#2}%
      {%
        \ekvd@assert@twoargs{#3}%
          {%
            \ekvd@type@choice{#2}%
            \def\ekvd@set@choice{#2}%
            #1#3%
          }%
      }%
  }
%    \end{macrocode}
% \end{macro}
% \begin{macro}[internal]
%   {\ekvd@populate@choicestore,\ekvd@populate@choicestore@}
%   We initialise the storing macro if it doesn't yet exist, and then we loop
%   over the value list. The |\edef|s with |\unexpanded| are both necessary to
%   be able to store macro parameter tokens (the outer protects at define time,
%   the inner at use time).
%    \begin{macrocode}
\protected\long\def\ekvd@populate@choicestore#1%
  {%
    \ekvd@newlet#1\ekv@empty
    \ekvcsvloop{\ekvd@populate@choicestore@#1}%
  }
\protected\long\def\ekvd@populate@choicestore@#1#2%
  {%
    \protected\expandafter\edef
      \csname\ekvd@choice@name\ekvd@set\ekvd@set@choice{#2}\endcsname
      {\unexpanded{\edef#1{\unexpanded{#2}}}}%
  }
%    \end{macrocode}
% \end{macro}
% \begin{macro}[internal]
%   {\ekvd@populate@choiceenum,\ekvd@populate@choiceenum@}
%   This is similar to the population of a |choice-store| type, but instead of
%   storing the values in a macro this initialises a count and stores the
%   position of the value in the list inside that count (zero-indexed). The
%   space is necessary to terminate the number scanning, which is the reason we
%   use |\@firstofone| (so that the space after the macro name isn't gobbled by
%   \TeX).
%    \begin{macrocode}
\protected\long\def\ekvd@populate@choiceenum#1%
  {%
    \ekvd@newreg#1{count}%
    \def\ekvd@tmp{0}%
    \ekvcsvloop{\ekvd@populate@choiceenum@#1}%
  }
\protected\long\def\ekvd@populate@choiceenum@#1#2%
  {%
    \protected\expandafter\edef
      \csname\ekvd@choice@name\ekvd@set\ekvd@set@choice{#2}\endcsname
      {#1=\@firstofone{\ekvd@tmp} }%
    \edef\ekvd@tmp{\the\numexpr\ekvd@tmp+1\relax}%
  }
%    \end{macrocode}
% \end{macro}
% \end{macro}
%
% \begin{macro}[internal]{\ekvd@t@unknown-choice}
%    \begin{macrocode}
\protected\long\expandafter\def\csname ekvd@t@unknown-choice\endcsname#1#2%
  {%
    \ekvd@assert@new@for@name{\ekvd@unknown@choice@name\ekvd@set{#1}}%
      {%
        \ekvd@assert@arg
          {%
            \ekvd@assert@not@long
            \ekvd@assert@not@also
            \ekvd@prot\expandafter
            \def\csname\ekvd@unknown@choice@name\ekvd@set{#1}\endcsname##1{#2}%
          }%
      }%
  }
%    \end{macrocode}
% \end{macro}
%
% \begin{macro}[internal]
%   {\ekvd@t@unknown,\ekvd@type@unknown@code,\ekvd@type@unknown@noval}
%   The |unknown| type has different subtypes which would be the key names for
%   other types. It is first checked whether that subtype is defined, if it
%   isn't throw an error, else use that subtype.
%    \begin{macrocode}
\protected\long\def\ekvd@t@unknown#1#2%
  {%
    \ekv@ifdefined{ekvd@type@unknown@\detokenize{#1}}%
      {\csname ekvd@type@unknown@\detokenize{#1}\endcsname{#2}}%
      \ekvd@err@misused@unknown
  }
%    \end{macrocode}
%   The |unknown noval| type can use |\ekvdefunknownNoVal| directly (after
%   asserting some prefixes).
%    \begin{macrocode}
\protected\long\def\ekvd@type@unknown@noval#1%
  {%
    \ekvd@assert@new@for@name{\ekv@name\ekvd@set{}uN}%
      {%
        \ekvd@assert@arg
          {%
            \ekvd@assert@not@also
            \ekvd@assert@not@long
            \ekvd@prot\ekvdefunknownNoVal\ekvd@set{#1}%
          }%
      }%
  }
%    \end{macrocode}
%   The |unknown code| type uses some trickery during the definition in order to
%   swap out |#1| and |#2| in the user supplied definition. This is done via a
%   temporary macro that stores the definition but gets the parameter numbers
%   reversed while the real definition is done.
%    \begin{macrocode}
\protected\long\def\ekvd@type@unknown@code#1%
  {%
    \ekvd@assert@new@for@name{\ekv@name\ekvd@set{}u}%
      {%
        \ekvd@assert@arg
          {%
            \ekvd@assert@not@also
            \begingroup
              \def\ekvd@tmp##1##2{#1}%
              \ekv@exparg
              {%
                \endgroup
                \ekvd@prot\ekvd@long\ekvdefunknown\ekvd@set
              }%
              {\ekvd@tmp{##2}{##1}}%
          }%
      }%
  }
%    \end{macrocode}
% \end{macro}
%
% \begin{macro}[internal]
%   {
%     \ekvd@type@unknown@redirect,
%     \ekvd@type@unknown@redirect-code,
%     \ekvd@type@unknown@redirect-noval
%   }
%   The |unknown redirect| types also just forward to |\ekvredirectunknown|
%   after asserting some prefixes.
%    \begin{macrocode}
\protected\edef\ekvd@type@unknown@redirect#1%
  {%
    \expandafter\noexpand\csname ekvd@type@unknown@redirect-code\endcsname{#1}%
    \expandafter\noexpand\csname ekvd@type@unknown@redirect-noval\endcsname{#1}%
  }
\protected\expandafter\def\csname ekvd@type@unknown@redirect-code\endcsname#1%
  {%
    \ekvd@assert@new@for@name{\ekv@name\ekvd@set{}u}%
      {%
        \ekvd@assert@arg
          {%
            \ekvd@assert@not@also
            \ekvd@assert@not@protected
            \expandafter\ekvredirectunknown\expandafter{\ekvd@set}{#1}%
          }%
      }%
  }
\protected\expandafter\def\csname ekvd@type@unknown@redirect-noval\endcsname#1%
  {%
    \ekvd@assert@new@for@name{\ekv@name\ekvd@set{}uN}%
      {%
        \ekvd@assert@arg
          {%
            \ekvd@assert@not@also
            \ekvd@assert@not@protected
            \ekvd@assert@not@long
            \expandafter\ekvredirectunknownNoVal\expandafter{\ekvd@set}{#1}%
          }%
      }%
  }
%    \end{macrocode}
% \end{macro}
%
% \subsubsection{Key Type Helpers}
% There are some keys that might need helpers during their execution (not during
% their definition, which are gathered as |@type@| macros). These helpers are
% named |@h@|.
%
% \begin{macro}[internal]{\ekvd@h@choice,\ekvd@h@choice@}
% The |choice| helper will just test whether the given choice was defined, if
% not throw an error expandably, else call the macro which stores the code for
% this choice.
%    \begin{macrocode}
\def\ekvd@h@choice#1%
  {%
    \expandafter\ekvd@h@choice@
      \csname\ifcsname#1\endcsname#1\else relax\fi\endcsname
      {#1}%
  }
\def\ekvd@h@choice@#1#2%
  {%
    \ifx#1\relax
      \ekvd@err@choice@invalid{#2}%
      \expandafter\@gobble
    \fi
    #1%
  }
%    \end{macrocode}
% \end{macro}
%
%
% \subsubsection{Handling \texttt{also}}
%
% \begin{macro}[internal]
%   {\ekvd@add@val,\ekvd@add@noval,\ekvd@add@aux,\ekvd@add@aux@}
%    \begin{macrocode}
\protected\long\def\ekvd@add@val#1#2#3%
  {%
    \ekvd@assert@val{#1}%
      {%
        \expandafter\ekvd@add@aux\csname\ekv@name\ekvd@set{#1}\endcsname{{##1}}%
          {#1}{#2}{\ekvd@long\ekvdef}{#3}%
      }%
  }
\protected\long\def\ekvd@add@noval#1#2#3%
  {%
    \ekvd@assert@noval{#1}%
      {%
        \expandafter\ekvd@add@aux\csname\ekv@name\ekvd@set{#1}N\endcsname{}%
          {#1}{#2}\ekvdefNoVal{#3}%
      }%
  }
\protected\long\def\ekvd@add@aux#1#2%
  {%
    \ekvd@extract@prefixes#1%
    \expandafter\ekvd@add@aux@\expandafter{#1#2}%
  }
\protected\long\def\ekvd@add@aux@#1#2#3#4#5%
  {%
    #5%
    \ekvd@prot#4\ekvd@set{#2}{#1#3}%
  }
%    \end{macrocode}
% \end{macro}
%
% \begin{macro}[internal]
%   {
%     \ekvd@extract@prefixes,\ekvd@extract@prefixes@,
%     \ekvd@extract@prefixes@long,\ekvd@extract@prefixes@prot
%   }
% This macro checks which prefixes were used for the definition of a macro and
% sets |\ekvd@long| and |\ekvd@prot| accordingly.
%    \begin{macrocode}
\protected\def\ekvd@extract@prefixes#1%
  {%
    \expandafter\ekvd@extract@prefixes@\meaning#1\ekvd@stop
  }
%    \end{macrocode}
% In the following definition |#1| will get replaced by |macro:|, |#2| by
% |\long| and |#3| by |\protected| (in each, all tokens will have category
% other). This allows us to parse the |\meaning| of a macro for those strings.
%    \begin{macrocode}
\protected\def\ekvd@extract@prefixes@#1#2#3%
  {%
    \protected\def\ekvd@extract@prefixes@##1#1##2\ekvd@stop
      {%
        \ekvd@extract@prefixes@long
          ##1\ekvd@mark\@firstofone#2\ekvd@mark\@gobble\ekvd@stop
          {\let\ekvd@long\long}%
        \ekvd@extract@prefixes@prot
          ##1\ekvd@mark\@firstofone#3\ekvd@mark\@gobble\ekvd@stop
          {\let\ekvd@prot\protected}%
      }%
    \protected\def\ekvd@extract@prefixes@long##1#2##2\ekvd@mark##3##4\ekvd@stop
      {##3}%
    \protected\def\ekvd@extract@prefixes@prot##1#3##2\ekvd@mark##3##4\ekvd@stop
      {##3}%
  }
%    \end{macrocode}
% We use a temporary macro to expand the three arguments of
% |\ekvd@extract@prefixes@|, which will set up the real meaning of itself and
% the parsing for |\long| and |\protected|.
%    \begin{macrocode}
\begingroup
\edef\ekvd@tmp
  {%
    \endgroup
    \ekvd@extract@prefixes@
      {\detokenize{macro:}}%
      {\string\long}%
      {\string\protected}%
  }
\ekvd@tmp
%    \end{macrocode}
% \end{macro}
%
%
% \subsubsection{Tests}
%
% \begin{macro}[internal]{\ekvd@newlet,\ekvd@newreg}
% These macros test whether a control sequence is defined, if it isn't they
% define it, either via |\let| or via the correct \cs[no-index]{new\meta{reg}}.
%    \begin{macrocode}
\protected\def\ekvd@newlet#1#2%
  {%
    \ifdefined#1\ekv@fi@gobble\fi\@firstofone{\let#1#2}%
  }
\protected\def\ekvd@newreg#1#2%
  {%
    \ifdefined#1\ekv@fi@gobble\fi\@firstofone{\csname new#2\endcsname#1}%
  }
%    \end{macrocode}
% \end{macro}
%
% \begin{macro}[internal]
%   {\ekvd@assert@twoargs,\ekvd@ifnottwoargs,\ekvd@ifempty@gtwo}
% A test for exactly two tokens can be reduced for an empty-test after gobbling
% two tokens, in the case that there are fewer tokens than two in the argument,
% only macros will be gobbled that are needed for the true branch, which doesn't
% hurt, and if there are more this will not be empty.
%    \begin{macrocode}
\long\def\ekvd@assert@twoargs#1%
  {%
    \ekvd@ifnottwoargs{#1}{\ekvd@err@missing@definition}%
  }
\long\def\ekvd@ifnottwoargs#1%
  {%
    \ekvd@ifempty@gtwo#1\ekv@ifempty@B
      \ekv@ifempty@false\ekv@ifempty@A\ekv@ifempty@B\@firstoftwo
  }
\long\def\ekvd@ifempty@gtwo#1#2{\ekv@ifempty@\ekv@ifempty@A}
%    \end{macrocode}
% \end{macro}
%
% \begin{macro}[internal]
%   {
%     \ekvd@assert@val,\ekvd@assert@val@,\ekvd@assert@noval,\ekvd@assert@noval@,
%     \ekvd@extract@args,\ekvd@extracted@args,\ekvd@one@arg@string
%   }
% Assert that a given key is defined as a value taking key or a |NoVal| key with
% the correct argument structure, respectively.
%    \begin{macrocode}
\protected\def\ekvd@assert@val#1%
  {%
    \ekvifdefined\ekvd@set{#1}%
      {\expandafter\ekvd@assert@val@\csname\ekv@name\ekvd@set{#1}\endcsname}%
      {%
        \ekvifdefinedNoVal\ekvd@set{#1}%
          \ekvd@err@add@val@on@noval
          {\ekvd@err@undefined@key{#1}}%
        \@gobble
      }%
  }
\protected\def\ekvd@assert@val@#1%
  {%
    \expandafter\ekvd@extract@args\meaning#1\ekvd@stop
    \unless\ifx\ekvd@extracted@args\ekvd@one@arg@string
      \ekvd@err@unsupported@arg
    \fi
    \@firstofone
  }%
\protected\def\ekvd@assert@noval#1%
  {%
    \ekvifdefinedNoVal\ekvd@set{#1}%
      {\expandafter\ekvd@assert@noval@\csname\ekv@name\ekvd@set{#1}N\endcsname}%
      {%
        \ekvifdefined\ekvd@set{#1}%
          \ekvd@err@add@noval@on@val
          {\ekvd@err@undefined@key{#1}}%
        \@gobble
      }%
  }
\protected\def\ekvd@assert@noval@#1%
  {%
    \expandafter\ekvd@extract@args\meaning#1\ekvd@stop
    \unless\ifx\ekvd@extracted@args\ekv@empty
      \ekvd@err@unsupported@arg
    \fi
    \@firstofone
  }
\protected\def\ekvd@extract@args#1%
  {%
    \protected\def\ekvd@extract@args##1#1##2->##3\ekvd@stop
      {\def\ekvd@extracted@args{##2}}%
  }
\expandafter\ekvd@extract@args\expandafter{\detokenize{macro:}}
\edef\ekvd@one@arg@string{\string#1}
%    \end{macrocode}
% \end{macro}
%
% \begin{macro}[internal]{\ekvd@assert@arg,\ekvd@assert@arg@msg,\ekvd@ifnoarg}
% There is no need to actually define |\ekvd@ifnoarg| here, as it will be set by
% either |\ekvd@arg| or |\ekvd@noarg|.
%    \begin{macrocode}
\def\ekvd@assert@arg{\ekvd@ifnoarg\ekvd@err@missing@definition}
\long\def\ekvd@assert@arg@msg#1%
  {%
    \ekvd@ifnoarg{\ekvd@err@missing@definition@msg{#1}}%
  }
%    \end{macrocode}
% \end{macro}
%
% \begin{macro}[internal]{\ekvd@assert@filledarg,\ekvd@ifnoarg@or@empty}
%    \begin{macrocode}
\long\def\ekvd@assert@filledarg#1%
  {%
    \ekvd@ifnoarg@or@empty{#1}\ekvd@err@missing@definition
  }
\long\def\ekvd@ifnoarg@or@empty#1%
  {%
    \ekvd@ifnoarg
      \@firstoftwo
      {\ekv@ifempty{#1}}%
  }
%    \end{macrocode}
% \end{macro}
%
% \begin{macro}[internal]
%   {
%     \ekvd@assert@not@long,\ekvd@assert@not@protected,\ekvd@assert@not@also
%     \ekvd@assert@not@long@also,\ekvd@assert@not@protected@also,
%     \ekvd@assert@new,\ekvd@assert@not@new
%   }
% Some key-types don't want to be |also|, |\long| or |\protected|, so we provide
% macros to test this and throw an error, this could be silently ignored but now
% users will learn to not use unnecessary stuff which slows the compilation
% down.
%    \begin{macrocode}
\def\ekvd@assert@not@long{\ifx\ekvd@long\long\ekvd@err@no@prefix{long}\fi}
\def\ekvd@assert@not@protected
  {\ifx\ekvd@prot\protected\ekvd@err@no@prefix{protected}\fi}
\def\ekvd@assert@not@also{\ekvd@ifalso{\ekvd@err@no@prefix{also}}{}}
\def\ekvd@assert@not@long@also
  {\ifx\ekvd@long\long\ekvd@err@no@prefix@also{long}\fi}
\def\ekvd@assert@not@protected@also
  {\ifx\ekvd@prot\protected\ekvd@err@no@prefix@also{protected}\fi}
\def\ekvd@assert@new#1#2%
  {\csname ekvifdefined#1\endcsname\ekvd@set{#2}{\ekvd@err@not@new}}
\def\ekvd@assert@not@new
  {\ifx\ekvd@ifnew\ekvd@assert@new\ekvd@err@no@prefix{new}\fi}
\def\ekvd@assert@new@for@name#1%
  {%
    \ifx\ekvd@ifnew\ekvd@assert@new
      \ekv@fi@firstoftwo
    \fi
    \@secondoftwo
      {\ekv@ifdefined{#1}\ekvd@err@not@new}%
      \@firstofone
  }
%    \end{macrocode}
% \end{macro}
%
% \begin{macro}[internal]
%   {
%     \ekvd@if@not@already@choice, \ekvd@if@not@already@choice@a,
%     \ekvd@if@not@already@choice@b
%   }
% It is bad to use |also| on a key that already contains a |choice|, as both
% choices would share the same valid values and thus lead to each callback being
% used twice. The following is a rudimentary test against this.
%    \begin{macrocode}
\protected\def\ekvd@if@not@already@choice#1%
  {%
    \expandafter\ekvd@if@not@already@choice@a
      \csname\ekv@name\ekvd@set{#1}\endcsname
      {}\ekvd@h@choice\ekvd@stop
  }
\protected\def\ekvd@if@not@already@choice@a
  {%
    \expandafter\ekvd@if@not@already@choice@b
  }
\long\protected\def\ekvd@if@not@already@choice@b#1\ekvd@h@choice#2\ekvd@stop
  {%
    \ekv@ifempty{#2}\@firstofone\@gobble
  }
%    \end{macrocode}
% \end{macro}
%
% \begin{macro}[internal]{\ekvd@ifspace,\ekvd@ifspace@}
% Yet another test which can be reduced to an if-empty, this time by gobbling
% everything up to the first space.
%    \begin{macrocode}
\long\def\ekvd@ifspace#1%
  {%
    \ekvd@ifspace@#1 \ekv@ifempty@B
      \ekv@ifempty@false\ekv@ifempty@A\ekv@ifempty@B\@firstoftwo
  }
\long\def\ekvd@ifspace@#1 % keep this space
  {%
    \ekv@ifempty@\ekv@ifempty@A
  }
%    \end{macrocode}
% \end{macro}
%
% \subsubsection{Messages}
%
% Most messages of \expkvd\ are not expandable, since they only appear during
% key-definition, which is not expandable anyway.
%
% \begin{macro}[internal]
%   {
%     \ekvd@errm,\ekvd@err@missing@definition,\ekvd@err@missing@definition@msg,
%     \ekvd@err@missing@type,\ekvd@err@undefined@prefix,\ekvd@err@undefined@key,
%     \ekvd@err@no@prefix,\ekvd@err@no@prefix@msg,\ekvd@err@no@prefix@also,
%     \ekvd@err@add@val@on@noval,\ekvd@err@add@noval@on@val,
%     \ekvd@err@unsupported@arg,\ekvd@err@not@new
%   }
% The non-expandable error messages are boring, so here they are:
%    \begin{macrocode}
\protected\def\ekvd@errm#1{\errmessage{expkv-def Error: #1}}
\protected\def\ekvd@err@missing@definition
  {\ekvd@errm{Missing definition for key `\ekvd@cur'}}
\protected\def\ekvd@err@missing@definition@msg#1%
  {\ekvd@errm{Missing definition for key `\ekv@unexpanded{#1}'}}
\protected\def\ekvd@err@missing@type
  {\ekvd@errm{Missing type prefix for key `\ekvd@cur'}}
\protected\def\ekvd@err@undefined@prefix#1%
  {%
    \ekvd@errm
      {%
        Undefined prefix `\ekv@unexpanded{#1}' found while processing
        `\ekvd@cur'%
      }%
  }
\protected\def\ekvd@err@undefined@key#1%
  {%
    \ekvd@errm
      {Undefined key `\ekv@unexpanded{#1}' found while processing `\ekvd@cur'}%
  }
\protected\def\ekvd@err@undefined@noval#1%
  {%
    \ekvd@errm
      {%
        Undefined noval key `\unexpanded{#1}' found while processing
        `\ekvd@cur'%
      }%
  }
\protected\def\ekvd@err@no@prefix#1%
  {\ekvd@errm{prefix `#1' not accepted in `\ekvd@cur'}}
\protected\def\ekvd@err@no@prefix@msg#1#2%
  {\ekvd@errm{prefix `#2' not accepted in `\ekv@unexpanded{#1}'}}
\protected\def\ekvd@err@no@prefix@also#1%
  {\ekvd@errm{`\ekvd@cur' not allowed with a `#1' key}}
\protected\def\ekvd@err@add@val@on@noval
  {\ekvd@errm{`\ekvd@cur' not allowed with a NoVal key}}
\protected\def\ekvd@err@add@noval@on@val
  {\ekvd@errm{`\ekvd@cur' not allowed with a value taking key}}
\protected\def\ekvd@err@unsupported@arg\fi\@firstofone#1%
  {%
    \fi
    \ekvd@errm
      {%
        Existing key-macro has the unsupported argument string
        `\ekvd@extracted@args' for key `\ekvd@cur'%
      }%
  }
\protected\def\ekvd@err@not@new
  {\ekvd@errm{The key for `\ekvd@cur' is already defined}}
\protected\long\def\ekvd@err@misused@unknown
  {\ekvd@errm{Misuse of the unknown type found while processing `\ekvd@cur'}}
%    \end{macrocode}
% \end{macro}
%
% \begin{macro}[internal]
%   {
%     \ekvd@err@choice@invalid,\ekvd@err@choice@invalid@,\ekvd@choice@name,
%     \ekvd@unknown@choice@name
%   }
% |\ekvd@err@choice@invalid| will have to use this mechanism to throw its
% message. Also we have to retrieve the name parts of the choice in an easy way,
% so we use parentheses of catcode 8 here, which should suffice in most cases to
% allow for a correct separation.
%    \begin{macrocode}
\def\ekvd@err@choice@invalid#1%
  {%
    \ekvd@err@choice@invalid@#1%
  }
\begingroup
\catcode40=8
\catcode41=8
\@firstofone{\endgroup
\def\ekvd@choice@name#1#2#3%
  {%
    ekvd#1(#2)\detokenize{#3}%
  }
\def\ekvd@unknown@choice@name#1#2%
  {%
    ekvd:u:#1(#2)%
  }
\def\ekvd@err@choice@invalid@ ekvd#1(#2)\detokenize#3%
  {%
    \ekv@ifdefined{\ekvd@unknown@choice@name{#1}{#2}}%
      {\csname\ekvd@unknown@choice@name{#1}{#2}\endcsname{#3}}%
      {\ekvd@err{invalid choice `#3' for `#2' in set `#1'}}%
  }
}
%    \end{macrocode}
% \end{macro}
%
% \begin{macro}[internal]{\ekvd@err}
% The expandable error messages use |\ekvd@err|, which is just like |\ekv@err|
% from \expkv. It uses a runaway argument to start the error message.
%    \begin{macrocode}
\ekv@exparg{\long\def\ekvd@err#1}{\ekverr{expkv-def}{#1}}
%    \end{macrocode}
% \end{macro}
%
% Now everything that's left is to reset the category code of |@|.
%    \begin{macrocode}
\catcode`\@=\ekvd@tmp
%    \end{macrocode}
%
% \gobbledocstriptag
%</tex>
%^^A=<<
%
%^^A=<<
%
% \end{implementation}^^A=<<
%
% \clearpage
% \PrintIndex
%
\endinput
%
^^A vim: ft=tex fdm=marker fmr=>>=,=<<

\ProvidesPackage{expkv-def}%
  [\ekvdDate\space v\ekvdVersion\space a key-defining frontend for expkv]
%    \end{macrocode}
% \gobbledocstriptag
%</pkg>
%^^A=<<
%^^A main file >>=
% \subsection{The Generic Code}
% The rest of this implementation will be the generic code.
% \gobbledocstriptag
%<*tex>
%
% Load \expkv\ if the package didn't already do so -- since \expkv\ has
% safeguards against being loaded twice this does no harm and the overhead
% isn't that big. Also we reuse some of the internals of \expkv\ to save us from
% retyping them.
%    \begin{macrocode}
\input expkv
%    \end{macrocode}
%
% We make sure that \file{expkv-def.tex} is only input once:
%    \begin{macrocode}
\expandafter\ifx\csname ekvdVersion\endcsname\relax
\else
  \expandafter\endinput
\fi
%    \end{macrocode}
%
% \begin{macro}{\ekvdVersion,\ekvdDate}
% We're on our first input, so lets store the version and date in a macro.
%    \begin{macrocode}
\def\ekvdVersion{0.1}
\def\ekvdDate{2020-02-26}
%    \end{macrocode}
% \end{macro}
%
% If the \LaTeX\ format is loaded we want to be a good file and report back who
% we are, for this the package will have defined |\ekvd@tmp| to use
% |\ProvidesFile|, else this will expand to a |\relax| and do no harm.
%    \begin{macrocode}
\csname ekvd@tmp\endcsname
%    \end{macrocode}
%
% Store the category code of |@| to later be able to reset it and change it to
% 11 for now.
%    \begin{macrocode}
\expandafter\chardef\csname ekvd@tmp\endcsname=\catcode`\@
\catcode`\@=11
%    \end{macrocode}
% |\ekvd@tmp| will be reused later to handle expansion during the key
% defining. But we don't need it to ever store information long-term after
% \expkvd\ was initialized.
%
% \begin{macro}[internal]
%   {\ekvd@long,\ekvd@prot,\ekvd@clear@prefixes,\ekvd@empty}
% \expkvd\ will use |\ekvd@long| and |\ekvd@prot| to store whether a key should
% be defined an |\long| or |\protected|, and we have to clear them for every new
% key. By default they'll just be empty.
%    \begin{macrocode}
\def\ekvd@empty{}
\protected\def\ekvd@clear@prefixes
  {%
    \let\ekvd@long\ekvd@empty
    \let\ekvd@prot\ekvd@empty
  }
\ekvd@clear@prefixes
%    \end{macrocode}
% \end{macro}
%
% \begin{macro}{\ekvdefinekeys}
% This is the one front-facing macro which provides the interface to define
% keys. It's using |\ekvparse| to handle the \kv\ list, the interpretation will
% be done by |\ekvd@noarg| and |\ekvd@|. The \set\ for which the keys should be
% defined is stored in |\ekvd@set|.
%    \begin{macrocode}
\protected\def\ekvdefinekeys#1%
  {%
    \def\ekvd@set{#1}%
    \ekvparse\ekvd@noarg\ekvd@
  }
%    \end{macrocode}
% \end{macro}
%
% \begin{macro}[internal]{\ekvd@noarg,\ekvd@}
% |\ekvd@noarg| just places a special marker and gives control to |\ekvd@|.
% |\ekvd@| has to test whether there is a space inside the key and if so calls
% the prefix grabbing routine, else we throw an error and ignore the key.
%    \begin{macrocode}
\protected\def\ekvd@noarg#1{\ekvd@{#1}\ekvd@noarg@mark}
\protected\def\ekvd@#1#2%
  {%
    \ekvd@clear@prefixes
    \ekvd@ifspace{#1}%
      {\ekvd@prefix\ekv@mark#1\ekv@stop{#2}}%
      {\ekvd@missing@prefix{#1}}%
  }
%    \end{macrocode}
% \end{macro}
%
% \begin{macro}[internal]{\ekvd@prefix,\ekvd@prefix@}
% \expkvd\ separates prefixes into two groups, the first being prefixes in the
% \TeX\ sense (|long| and |protected|) which use |@p@| in their name, the other
% being key-types (|code|, |int|, \emph{etc.}) which use |@t@| instead.
% |\ekvd@prefix| splits at the first space and checks whether its a |@p@|
% or |@t@| type prefix. If it is neither throw an error and gobble the
% definition (the value).
%    \begin{macrocode}
\protected\def\ekvd@prefix#1 {\ekv@strip{#1}\ekvd@prefix@\ekv@mark}
\protected\def\ekvd@prefix@#1#2\ekv@stop
  {%
    \ekv@ifdefined{ekvd@t@#1}%
      {\ekv@strip{#2}{\csname ekvd@t@#1\endcsname}}%
      {%
        \ekv@ifdefined{ekvd@p@#1}%
          {\csname ekvd@p@#1\endcsname{#2}}%
          {\ekvd@err@undefined@prefix{#1}\@gobble}%
      }%
  }
%    \end{macrocode}
% \end{macro}
%
% \begin{macro}[internal]{\ekvd@prefix@after@p}
% The |@p@| type prefixes are all just modifying a following |@t@| type, so they
% will need to search for another prefix. This is true for all of them, so we
% use a macro to handle this. It'll throw an error if there is no other prefix.
%    \begin{macrocode}
\protected\def\ekvd@prefix@after@p#1%
  {%
    \ekvd@ifspace{#1}%
      {\ekvd@prefix#1\ekv@stop}%
      {%
        \expandafter\ekvd@err@missing@prefix\expandafter{\ekv@gobble@mark#1}%
        \@gobble
      }%
  }
%    \end{macrocode}
% \end{macro}
%
% \begin{macro}[internal]{\ekvd@p@long,\ekvd@p@protected,\ekvd@p@protect}
% Define the |@p@| type prefixes, they all just store some information in a
% temporary macro and call |\ekvd@prefix@after@p|.
%    \begin{macrocode}
\protected\def\ekvd@p@long{\let\ekvd@long\long\ekvd@prefix@after@p}
\protected\def\ekvd@p@protected{\let\ekvd@prot\protected\ekvd@prefix@after@p}
\let\ekvd@p@protect\ekvd@p@protected
%    \end{macrocode}
% \end{macro}
%
% \subsubsection{Key Types}
%
% \begin{macro}[internal]{\ekvd@t@set}
% The |set| type is quite straight forward, just define a |NoVal| key to call
% |\ekvchangeset|.
%    \begin{macrocode}
\protected\def\ekvd@t@set#1#2%
  {%
    \ekvd@assert@not@long{set #1}%
    \ekvd@assert@not@protected{set #1}%
    \ekvd@ifnoarg{#2}%
      {\ekvdefNoVal\ekvd@set{#1}{\ekvchangeset{#1}}}%
      {%
        \ekv@ifempty{#2}%
          {\ekvd@err@missing@definition{set #1}}%
          {\ekvdefNoVal\ekvd@set{#1}{\ekvchangeset{#2}}}%
      }%
  }
%    \end{macrocode}
% \end{macro}
%
% \begin{macro}[internal]{\ekvd@type@noval,\ekvd@t@noval,\ekvd@t@enoval}
% Another pretty simple type, |noval| just needs to assert that there is a
% definition and that |long| wasn't specified.
% There are types where the difference in the variants is so small, that we
% define a common handler for them, those common handlers are named with
% |@type@|. |noval| and |enoval| are so similar that we can use such a |@type@|
% macro, even if we could've done |noval| in a slightly faster way without it.
%    \begin{macrocode}
\protected\long\def\ekvd@type@noval#1#2#3#4%
  {%
    \ekvd@assert@arg{#1noval #3}{#4}%
      {%
        \ekvd@assert@not@long{#1noval #3}%
        \ekvd@prot#2\ekvd@tmp{#4}%
        \ekvletNoVal\ekvd@set{#3}\ekvd@tmp
      }%
  }
\protected\def\ekvd@t@noval{\ekvd@type@noval{}\def}
\protected\def\ekvd@t@enoval{\ekvd@type@noval e\edef}
%    \end{macrocode}
% \end{macro}
%
% \begin{macro}[internal]{\ekvd@type@code,\ekvd@t@code,\ekvd@t@ecode}
% |code| is simple as well, |ecode| has to use |\edef| on a temporary macro,
% since \expkv\ doesn't provide an |\ekvedef|.
%    \begin{macrocode}
\protected\long\def\ekvd@type@code#1#2#3#4%
  {%
    \ekvd@assert@arg{#1code #3}{#4}
      {%
        \ekvd@prot\ekvd@long#2\ekvd@tmp##1{#4}%
        \ekvlet\ekvd@set{#3}\ekvd@tmp
      }%
  }
\protected\def\ekvd@t@code{\ekvd@type@code{}\def}
\protected\def\ekvd@t@ecode{\ekvd@type@code e\edef}
%    \end{macrocode}
% \end{macro}
%
% \begin{macro}[internal]{\ekvd@type@default,\ekvd@t@default,\ekvd@t@qdefault}
% |\ekvd@type@default| asserts there was an argument, also the key for which one
% wants to set a default has to be already defined (this is not so important for
% |default|, but |qdefault| requires is). If everything is good, |\edef| a
% temporary macro that expands |\ekvd@set| and the |\csname| for the key, and in
% the case of |qdefault| does the first expansion step of the key-macro.
%    \begin{macrocode}
\protected\long\def\ekvd@type@default#1#2#3#4%
  {%
    \ekvd@assert@arg{#1default #3}{#4}%
      {%
        \ekvifdefined\ekvd@set{#3}%
          {%
            \ekvd@assert@not@long{#1default #3}%
            \ekvd@prot\edef\ekvd@tmp
              {%
                \unexpanded\expandafter#2%
                  {\csname\ekv@name\ekvd@set{#3}\endcsname{#4}}%
              }%
            \ekvletNoVal\ekvd@set{#3}\ekvd@tmp
          }%
          {\ekvd@err@undefined@key{#3}}%
      }%
  }
\protected\def\ekvd@t@default{\ekvd@type@default{}{}}
\protected\def\ekvd@t@qdefault{\ekvd@type@default q{\expandafter\expandafter}}
%    \end{macrocode}
% \end{macro}
%
% \begin{macro}[internal]{\ekvd@t@edefault}
% |edefault| is too different from |default| and |qdefault| to reuse the
% |@type@| macro, as it doesn't need |\unexpanded| inside of |\edef|.
%    \begin{macrocode}
\protected\long\def\ekvd@t@edefault#1#2%
  {%
    \ekvd@assert@arg{edefault #1}{#2}%
      {%
        \ekvifdefined\ekvd@set{#1}%
          {%
            \ekvd@assert@not@long{edefault #1}%
            \ekvd@prot\edef\ekvd@tmp
              {\csname\ekv@name\ekvd@set{#1}\endcsname{#2}}%
            \ekvletNoVal\ekvd@set{#1}\ekvd@tmp
          }%
          {\ekvd@err@undefined@key{#1}}%
      }%
  }
%    \end{macrocode}
% \end{macro}
%
% \begin{macro}[internal]{\ekvd@t@initial}
%    \begin{macrocode}
\long\def\ekvd@t@initial#1#2%
  {%
    \ekvd@assert@arg{initial #1}{#2}%
      {%
        \ekvifdefined\ekvd@set{#1}%
          {%
            \ekvd@assert@not@long{initial #1}%
            \ekvd@assert@not@protected{initial #1}%
            \csname\ekv@name\ekvd@set{#1}\endcsname{#2}%
          }%
          {\ekvd@err@undefined@key{#1}}%
      }%
  }
%    \end{macrocode}
% \end{macro}
%
% \begin{macro}[internal]
%   {\ekvd@type@bool,\ekvd@t@bool,\ekvd@t@gbool,\ekvd@t@boolTF,\ekvd@t@gboolTF}
% The boolean types are a quicker version of a |choice| that accept |true| and
% |false|, and set up the |NoVal| action to be identical to \texttt{\key=true}.
% The |true| and |false| actions are always just |\let|ting the macro in |#7| to
% some other macro (\emph{e.g.}, \cs[no-index]{iftrue}).
%    \begin{macrocode}
\protected\def\ekvd@type@bool#1#2#3#4#5#6#7%
  {%
    \ekvd@assert@filledarg{#1bool#2 #6}{#7}%
      {%
        \ekvd@newlet#7#5%
        \ekvd@type@choice{#1bool#2}{#6}%
        \protected\ekvdefNoVal\ekvd@set{#6}{#3\let#7#4}%
        \protected\expandafter\def
          \csname\ekvd@choice@name\ekvd@set{#6}{true}\endcsname
          {#3\let#7#4}%
        \protected\expandafter\def
          \csname\ekvd@choice@name\ekvd@set{#6}{false}\endcsname
          {#3\let#7#5}%
      }%
  }
\protected\def\ekvd@t@bool{\ekvd@type@bool{}{}{}\iftrue\iffalse}
\protected\def\ekvd@t@gbool{\ekvd@type@bool g{}\global\iftrue\iffalse}
\protected\def\ekvd@t@boolTF{\ekvd@type@bool{}{TF}{}\@firstoftwo\@secondoftwo}
\protected\def\ekvd@t@gboolTF
  {\ekvd@type@bool g{TF}\global\@firstoftwo\@secondoftwo}
%    \end{macrocode}
% \end{macro}
%
% \begin{macro}[internal]{\ekvd@type@box,\ekvd@t@box,\ekvd@t@gbox}
% Set up our boxes. Though we're a generic package we want to be colour safe, so
% we put an additional grouping level inside the box contents, for the case that
% someone uses \pkg{color}. |\ekvd@newreg| is a small wrapper which tests
% whether the first argument is defined and if not does
% |\csname new#2\endcsname#1|.
%    \begin{macrocode}
\protected\def\ekvd@type@box#1#2#3#4%
  {%
    \ekvd@assert@filledarg{#1box #3}{#4}%
      {%
        \ekvd@newreg#4{box}%
        \protected\ekvd@long\ekvdef\ekvd@set{#3}%
          {#2\setbox#4\hbox{\begingroup##1\endgroup}}%
      }%
  }
\protected\def\ekvd@t@box{\ekvd@type@box{}{}}
\protected\def\ekvd@t@gbox{\ekvd@type@box g\global}
%    \end{macrocode}
% \end{macro}
%
% \begin{macro}[internal]{\ekvd@type@toks,\ekvd@t@toks,\ekvd@t@gtoks}
% Similar to |box|, but set the |toks|.
%    \begin{macrocode}
\protected\def\ekvd@type@toks#1#2#3#4%
  {%
    \ekvd@assert@filledarg{#1toks #3}{#4}%
      {%
        \ekvd@newreg#4{toks}%
        \protected\ekvd@long\ekvdef\ekvd@set{#3}{#2#4{##1}}%
      }%
  }
\protected\def\ekvd@t@toks{\ekvd@type@toks{}{}}
\protected\def\ekvd@t@gtoks{\ekvd@type@toks{g}\global}
%    \end{macrocode}
% \end{macro}
%
% \begin{macro}[internal]{\ekvd@type@apptoks,\ekvd@t@apptoks,\ekvd@t@gapptoks}
% Just like |toks|, but expand the current contents of the |toks| register to
% append the new contents.
%    \begin{macrocode}
\protected\def\ekvd@type@apptoks#1#2#3#4%
  {%
    \ekvd@assert@filledarg{#1apptoks #3}{#4}%
      {%
        \ekvd@newreg#4{toks}%
        \protected\ekvd@long\ekvdef\ekvd@set{#3}{#2#4\expandafter{\the#4##1}}%
      }%
  }
\protected\def\ekvd@t@apptoks{\ekvd@type@apptoks{}{}}
\protected\def\ekvd@t@gapptoks{\ekvd@type@apptoks{g}\global}
%    \end{macrocode}
% \end{macro}
%
% \begin{macro}[internal]
%   {
%     \ekvd@type@reg,\ekvd@t@int,\ekvd@t@eint,\ekvd@t@gint,\ekvd@t@xint,
%     \ekvd@t@dimen,\ekvd@t@edimen,\ekvd@t@gdimen,\ekvd@t@xdimen,
%     \ekvd@t@skip,\ekvd@t@eskip,\ekvd@t@gskip,\ekvd@t@xskip
%   }
% The |\ekvd@type@reg| can handle all the types for which the assignment will
% just be \texttt{\meta{register}=\meta{value}}.
%    \begin{macrocode}
\protected\def\ekvd@type@reg#1#2#3#4#5#6#7%
  {%
    \ekvd@assert@filledarg{#1 #6}{#7}%
      {%
        \ekvd@newreg#7{#2}%
        \protected\ekvd@long\ekvdef\ekvd@set{#6}{#3#7=#4##1#5\relax}%
      }%
  }
\protected\def\ekvd@t@int{\ekvd@type@reg{int}{count}{}{}{}}
\protected\def\ekvd@t@eint{\ekvd@type@reg{eint}{count}{}\numexpr\relax}
\protected\def\ekvd@t@gint{\ekvd@type@reg{gint}{count}\global{}{}}
\protected\def\ekvd@t@xint{\ekvd@type@reg{xint}{count}\global\numexpr\relax}
\protected\def\ekvd@t@dimen{\ekvd@type@reg{dimen}{dimen}{}{}{}}
\protected\def\ekvd@t@edimen{\ekvd@type@reg{edimen}{dimen}{}\dimexpr\relax}
\protected\def\ekvd@t@gdimen{\ekvd@type@reg{gdimen}{dimen}\global{}{}}
\protected\def\ekvd@t@xdimen{\ekvd@type@reg{xdimen}{dimen}\global\dimexpr\relax}
\protected\def\ekvd@t@skip{\ekvd@type@reg{skip}{skip}{}{}{}}
\protected\def\ekvd@t@eskip{\ekvd@type@reg{eskip}{skip}{}\glueexpr\relax}
\protected\def\ekvd@t@gskip{\ekvd@type@reg{gskip}{skip}\global{}{}}
\protected\def\ekvd@t@xskip{\ekvd@type@reg{xskip}{skip}\global\glueexpr\relax}
%    \end{macrocode}
% \end{macro}
%
% \begin{macro}[internal]{\ekvd@type@store,\ekvd@t@store,\ekvd@t@gstore}
% The none-expanding |store| types use an |\edef| or |\xdef| and |\unexpanded|
% to be able to also store |#| easily.
%    \begin{macrocode}
\protected\def\ekvd@type@store#1#2#3#4%
  {%
    \ekvd@assert@filledarg{#1store #3}{#4}%
      {%
        \unless\ifdefined#4\let#4\ekvd@empty\fi
        \protected\ekvd@long\ekvdef\ekvd@set{#3}{#2#4{\unexpanded{##1}}}%
      }%
  }
\protected\def\ekvd@t@store{\ekvd@type@store{}\edef}
\protected\def\ekvd@t@gstore{\ekvd@type@store{g}\xdef}
%    \end{macrocode}
% \end{macro}
%
% \begin{macro}[internal]{\ekvd@type@estore,\ekvd@t@estore,\ekvd@t@xstore}
% And the straight forward |estore| types.
%    \begin{macrocode}
\protected\def\ekvd@type@estore#1#2#3#4%
  {%
    \ekvd@assert@filledarg{#1store #3}{#4}%
      {%
        \ekvd@newlet#4\ekvd@empty
        \protected\ekvd@long\ekvdef\ekvd@set{#3}{#2#4{##1}}%
      }%
  }
\protected\def\ekvd@t@estore{\ekvd@type@estore{e}\edef}
\protected\def\ekvd@t@xstore{\ekvd@type@estore{x}\xdef}
%    \end{macrocode}
% \end{macro}
%
% \begin{macro}[internal]
%   {\ekvd@type@meta,\ekvd@type@meta@,\ekvd@t@meta,\ekvd@t@nmeta}
% |meta| sets up things such that another instance of |\ekvset| will be run on
% the argument, with the same \set.
%    \begin{macrocode}
\protected\long\def\ekvd@type@meta#1#2#3#4#5%
  {%
    \ekvd@assert@filledarg{#1meta #4}{#5}%
      {%
        \edef\ekvd@tmp{\ekvd@set}%
        \expandafter\ekvd@type@meta@\expandafter{\ekvd@tmp}{#3}{#5}%
        #2\ekvd@set{#4}\ekvd@tmp
      }%
  }
\protected\long\def\ekvd@type@meta@#1#2#3%
  {%
    \ekvd@prot\ekvd@long\def\ekvd@tmp#2{\ekvset{#1}{#3}}%
  }
\protected\def\ekvd@t@meta{\ekvd@type@meta{}\ekvlet{##1}}
\protected\def\ekvd@t@nmeta{\ekvd@type@meta n\ekvletNoVal{}}
%    \end{macrocode}
% \end{macro}
%
% \begin{macro}[internal]
%   {\ekvd@type@smeta,\ekvd@type@smeta@,\ekvd@t@smeta,\ekvd@t@snmeta}
% |smeta| is pretty similar to |meta|, but needs two arguments inside of \val,
% such that the first is the \set\ for which the sub-|\ekvset| and the second is
% the \kv\ list.
%    \begin{macrocode}
\protected\long\def\ekvd@type@smeta#1#2#3#4#5%
  {%
    \ekvd@assert@twoargs{s#1meta #4}{#5}%
      {%
        \expandafter\ekvd@type@smeta@\expandafter{\@secondoftwo#5}{#5}{#3}
        #2\ekvd@set{#4}\ekvd@tmp
      }%
  }
\protected\long\def\ekvd@type@smeta@#1#2#3%
  {%
    \expandafter\ekvd@type@meta@\expandafter{\@firstoftwo#2}{#3}{#1}%
  }
\protected\def\ekvd@t@smeta{\ekvd@type@smeta{}\ekvlet{##1}}
\protected\def\ekvd@t@snmeta{\ekvd@type@smeta n\ekvletNoVal{}}
%    \end{macrocode}
% \end{macro}
%
% \begin{macro}[internal]
%   {
%     \ekvd@type@choice,\ekvd@populate@choice,\ekvd@populate@choice@,
%     \ekvd@populate@choice@noarg,\ekvd@choice@prefix,\ekvd@choice@prefix@,
%     \ekvd@choice@p@protected,\ekvd@choice@p@protect,
%     \ekvd@choice@p@long,\ekvd@choice@p@long@,\ekvd@t@choice
%   }
% The |choice| type is by far the most complex type, as we have to run a
% sub-parser on the choice-definition list, which should support the |@p@| type
% prefixes as well (but |long| will always throw an error, as they are not
% allowed to be long). |\ekvd@type@choice| will just define the choice-key, the
% handling of the choices definition will be done by |\ekvd@populate@choice|.
%    \begin{macrocode}
\protected\def\ekvd@type@choice#1#2%
  {%
    \ekvd@assert@not@long{#1 #2}%
    \ekvd@prot\edef\ekvd@tmp##1%
      {%
        \unexpanded{\ekvd@h@choice}{\ekvd@choice@name\ekvd@set{#2}{##1}}%
      }%
    \ekvlet\ekvd@set{#2}\ekvd@tmp
  }
%    \end{macrocode}
% |\ekvd@populate@choice| just uses |\ekvparse| and then gives control to
% |\ekvd@populate@choice@noarg|, which throws an error, and
% |\ekvd@populate@choice@|.
%    \begin{macrocode}
\protected\def\ekvd@populate@choice
  {%
    \ekvparse\ekvd@populate@choice@noarg\ekvd@populate@choice@
  }
\protected\def\ekvd@populate@choice@noarg#1%
  {%
    \expandafter\ekvd@err@missing@definition\expandafter{\ekvd@set@choice : #1}%
  }
%    \end{macrocode}
% |\ekvd@populate@choice@| runs the prefix-test, if there is none we can
% directly define the choice, for that |\ekvd@set@choice| will expand to the
% current choice-key's name, which will have been defined by |\ekvd@t@choice|.
% If there is a prefix run the prefix grabbing routine, which was altered for
% |@type@choice|.
%    \begin{macrocode}
\protected\long\def\ekvd@populate@choice@#1#2%
  {%
    \ekvd@clear@prefixes
    \expandafter\ekvd@assert@arg\expandafter{\ekvd@set@choice : #1}{#2}%
      {%
        \ekvd@ifspace{#1}%
          {\ekvd@choice@prefix\ekv@mark#1\ekv@stop}%
          {%
            \expandafter\def
              \csname\ekvd@choice@name\ekvd@set\ekvd@set@choice{#1}\endcsname
          }%
          {#2}%
      }%
  }
\protected\def\ekvd@choice@prefix#1
  {%
    \ekv@strip{#1}\ekvd@choice@prefix@\ekv@mark
  }
\protected\def\ekvd@choice@prefix@#1#2\ekv@stop
  {%
    \ekv@ifdefined{ekvd@choice@p@#1}%
      {%
        \csname ekvd@choice@p@#1\endcsname
        \ekvd@ifspace{#2}%
          {\ekvd@choice@prefix#2\ekv@stop}%
          {%
            \ekvd@prot\expandafter\def
              \csname
                \ekv@strip{#2}{\ekvd@choice@name\ekvd@set\ekvd@set@choice}%
              \endcsname
          }%
      }%
      {\ekvd@err@undefined@prefix{#1}\@gobble}%
  }
\protected\def\ekvd@choice@p@protected{\let\ekvd@prot\protected}
\let\ekvd@choice@p@protect\ekvd@choice@p@protected
\protected\def\ekvd@choice@p@long\ekvd@ifspace#1%
  {%
    \expandafter\ekvd@choice@p@long@\expandafter{\ekv@gobble@mark#1}%
    \ekvd@ifspace{#1}%
  }
\protected\def\ekvd@choice@p@long@#1%
  {%
    \expandafter\ekvd@err@no@long\expandafter
      {\ekvd@set@choice : long #1}%
  }
%    \end{macrocode}
% Finally we're able to set up the |@t@choice| macro, which has to store the
% current choice-key's name, define the key, and parse the available choices.
%    \begin{macrocode}
\protected\long\def\ekvd@t@choice#1#2%
  {%
    \ekvd@assert@arg{choice #1}{#2}%
      {%
        \ekvd@type@choice{choice}{#1}%
        \def\ekvd@set@choice{#1}%
        \ekvd@populate@choice{#2}%
      }%
  }
%    \end{macrocode}
% \end{macro}
%
% \subsubsection{Key Type Helpers}
% There are some keys that might need helpers during their execution (not during
% their definition, which are gathered as |@type@| macros). These helpers are
% named |@h@|.
%
% \begin{macro}[internal]{\ekvd@h@choice,\ekvd@h@choice@}
% The |choice| helper will just test whether the given choice was defined, if
% not throw an error expandably, else call the macro which stores the code for
% this choice.
%    \begin{macrocode}
\def\ekvd@h@choice#1%
  {%
    \expandafter\ekvd@h@choice@
      \csname\ifcsname#1\endcsname#1\else relax\fi\endcsname
      {#1}%
  }
\def\ekvd@h@choice@#1#2%
  {%
    \ifx#1\relax
      \ekvd@err@choice@invalid{#2}%
      \expandafter\@gobble
    \fi
    #1%
  }
%    \end{macrocode}
% \end{macro}
%
% \subsubsection{Tests}
%
% \begin{macro}[internal]{\ekvd@noarg@mark}
% This macro serves as a flag for the case that no \val\ was specified for a
% key. As such it is not a test, but exists only for some tests.
%    \begin{macrocode}
\def\ekvd@noarg@mark{\ekvd@noarg@mark}
%    \end{macrocode}
% \end{macro}
%
% \begin{macro}[internal]{\ekvd@fi@firstoftwo}
% While we can reuse many of the internals of \expkv\ the specific case for this
% branch wasn't needed by \expkv\ and hence isn't defined. We'll need it, so we
% define it.
%    \begin{macrocode}
\long\def\ekvd@fi@firstoftwo\fi\@secondoftwo#1#2{\fi#1}
%    \end{macrocode}
% \end{macro}
%
% \begin{macro}[internal]{\ekvd@newlet,\ekvd@newreg}
% These macros test whether a control sequence is defined, if it isn't they
% define it, either via |\let| or via the correct \cs[no-index]{new\meta{reg}}.
%    \begin{macrocode}
\protected\def\ekvd@newlet#1#2%
  {%
    \unless\ifdefined#1\let#1#2\fi
  }
\protected\def\ekvd@newreg#1#2%
  {%
    \unless\ifdefined#1\csname new#2\endcsname#1\fi
  }
%    \end{macrocode}
% \end{macro}
%
% \begin{macro}[internal]
%   {\ekvd@assert@twoargs,\ekvd@ifnottwoargs,\ekvd@ifempty@gtwo}
% A test for exactly two tokens can be reduced for an empty-test after gobbling
% two tokens, in the case that there are fewer tokens than two in the argument,
% only macros will be gobbled that are needed for the true branch, which doesn't
% hurt, and if there are more this will not be empty.
%    \begin{macrocode}
\long\def\ekvd@assert@twoargs#1#2%
  {%
    \ekvd@ifnottwoargs{#2}%
      {\ekvd@err@missing@definition{#1}}%
  }
\long\def\ekvd@ifnottwoargs#1%
  {%
    \ekvd@ifempty@gtwo#1\ekv@ifempty@B
      \ekv@ifempty@false\ekv@ifempty@A\ekv@ifempty@B\@firstoftwo
  }
\long\def\ekvd@ifempty@gtwo#1#2{\ekv@ifempty@\ekv@ifempty@A}
%    \end{macrocode}
% \end{macro}
%
% \begin{macro}[internal]{\ekvd@assert@arg,\ekvd@ifnoarg}
% The test for an argument is just an |\ifx| comparison with our |noarg@mark|.
%    \begin{macrocode}
\long\def\ekvd@assert@arg#1#2%
  {%
    \ekvd@ifnoarg{#2}%
      {\ekvd@err@missing@definition{#1}}%
  }
\long\def\ekvd@ifnoarg#1%
  {%
    \ifx\ekvd@noarg@mark#1%
      \ekvd@fi@firstoftwo
    \fi
    \@secondoftwo
  }
%    \end{macrocode}
% \end{macro}
%
% \begin{macro}[internal]{\ekvd@assert@filledarg,\ekvd@ifnoarg@or@empty}
%    \begin{macrocode}
\long\def\ekvd@assert@filledarg#1#2%
  {%
    \ekvd@ifnoarg@or@empty{#2}%
      {\ekvd@err@missing@definition{#1}}%
  }
\long\def\ekvd@ifnoarg@or@empty#1%
  {%
    \ekvd@ifnoarg{#1}%
      \@firstoftwo
      {\ekv@ifempty{#1}}%
  }
%    \end{macrocode}
% \end{macro}
%
% \begin{macro}[internal]{\ekvd@assert@not@long,\ekvd@assert@not@protected}
% Some key-types don't want to be |\long| or |\protected|, so we provide macros
% to test this and throw an error, this could be silently ignored but now users
% will learn to not use unnecessary stuff which slows the compilation down.
%    \begin{macrocode}
\long\def\ekvd@assert@not@long#1%
  {%
    \ifx\ekvd@long\long\ekvd@err@no@long{#1}\fi
  }
\long\def\ekvd@assert@not@protected#1%
  {%
    \ifx\ekvd@prot\protected\ekvd@err@no@protected{#1}\fi
  }
%    \end{macrocode}
% \end{macro}
%
% \begin{macro}[internal]{\ekvd@ifspace,\ekvd@ifspace@}
% Yet another test which can be reduced to an if-empty, this time by gobbling
% everything up to the first space.
%    \begin{macrocode}
\long\def\ekvd@ifspace#1%
  {%
    \ekvd@ifspace@#1 \ekv@ifempty@B
      \ekv@ifempty@false\ekv@ifempty@A\ekv@ifempty@B\@firstoftwo
  }
\long\def\ekvd@ifspace@#1 % keep this space
  {%
    \ekv@ifempty@\ekv@ifempty@A
  }
%    \end{macrocode}
% \end{macro}
%
% \subsubsection{Messages}
%
% Most messages of \expkvd\ are not expandable, since they only appear during
% key-definition, which is not expandable anyway.
%
% \begin{macro}[internal]
%   {
%     \ekvd@err@missing@definition,\ekvd@err@missing@prefix,
%     \ekvd@err@undefined@prefix,\ekvd@err@undefined@key,\ekvd@err@no@protected,
%     \ekvd@err@no@long
%   }
% The non-expandable error messages are boring, so here they are:
%    \begin{macrocode}
\protected\def\ekvd@err@missing@definition#1%
  {\errmessage{expkv-def Error: Missing definition for key `\unexpanded{#1}'}}
\protected\def\ekvd@err@missing@prefix#1%
  {\errmessage{expkv-def Error: Missing prefix for key `\unexpanded{#1}'}}
\protected\def\ekvd@err@undefined@prefix#1%
  {\errmessage{expkv-def Error: Undefined prefix `\unexpanded{#1}'}}
\protected\def\ekvd@err@undefined@key#1%
  {\errmessage{expkv-def Error: Undefined key `\unexpanded{#1}'}}
\protected\def\ekvd@err@no@protected#1%
  {%
    \errmessage
      {expkv-def Error: prefix `protected' not accepted for `\unexpanded{#1}'}%
  }
\protected\def\ekvd@err@no@long#1%
  {%
    \errmessage
      {expkv-def Error: prefix `long' not accepted for `\unexpanded{#1}'}%
  }
%    \end{macrocode}
% \end{macro}
%
% \begin{macro}[internal]
%   {
%     \ekvd@err@choice@invalid,\ekvd@err@choice@invalid@,\ekvd@choice@name,
%     \ekvd@err
%   }
% The expandable error messages use |\ekvd@err|, which is just like |\ekv@err|
% from \expkv\ or the way \pkg{expl3} throws expandable error messages. It uses
% an undefined control sequence to start the error message.
% |\ekvd@err@choice@invalid| will have to use this mechanism to throw its
% message. Also we have to retrieve the name parts of the choice in an easy way,
% so we use parentheses of catcode 8 here, which should suffice in most cases to
% allow for a correct separation.
%    \begin{macrocode}
\def\ekvd@err@choice@invalid#1%
  {%
    \ekvd@err@choice@invalid@#1\ekv@stop
  }
\begingroup
\catcode40=8
\catcode41=8
\@firstofone{\endgroup
\def\ekvd@choice@name#1#2#3%
  {%
    ekvd#1(#2)#3%
  }
\def\ekvd@err@choice@invalid@ ekvd#1(#2)#3\ekv@stop%
  {%
    \ekvd@err{invalid choice `#3' (`#2', set `#1')}%
  }
}
\begingroup
\edef\ekvd@err
  {%
    \endgroup
    \unexpanded{\long\def\ekvd@err}##1%
      {%
        \unexpanded{\expandafter\ekv@err@\@firstofone}%
        {\expandafter\noexpand\csname ! expkv-def Error:\endcsname ##1.}%
        \unexpanded{\ekv@stop}%
      }%
  }
\ekvd@err
%    \end{macrocode}
% \end{macro}
%
% Now everything that's left is to reset the category code of |@|.
%    \begin{macrocode}
\catcode`\@=\ekvd@tmp
%    \end{macrocode}
%
% \gobbledocstriptag
%</tex>
%^^A=<<
%
%^^A=<<
%
% \end{implementation}^^A=<<
%
% \clearpage
% \PrintIndex
%
\endinput
%
^^A vim: ft=tex fdm=marker fmr=>>=,=<<
